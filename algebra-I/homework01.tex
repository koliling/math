\documentclass{article}
\usepackage[left=3cm,right=3cm,top=3cm,bottom=3cm]{geometry}
\usepackage{amsmath,amssymb,amsthm}
\usepackage{color}
%\setlength{\parindent}{0mm}

\newcommand{\TODO}[1]{\textcolor{red}{TODO: #1}}

\begin{document}
\title{Math 60210: Homework 1}
\author{Li Ling Ko\\ lko@nd.edu}
\date{\today}
\maketitle

\begin{enumerate}
  \item
    \begin{enumerate}
      \item Prove that the relation given by $a \sim b \Longleftrightarrow
        a-b \in \mathbb{Z} $ is an equivalence relation on the additive group
        $\mathbb{Q}$.
        \begin{proof}
          A relation is an equivalence one iff it is reflexive, symmetric,
          and transitive. The given relation is reflexive since
          $a-a=0\in\mathbb{Z}$. The relation is also symmetric since
          $b-a=-(a-b)$, so if $a-b\in\mathbb{Z}$, then $b-a\in\mathbb{Z}$.
          Finally, the relation is transitive because if $a-b\in\mathbb{Z}$
          and $b-c\in\mathbb{Z}$, then $a-c=(a-b)+(b-c)\in\mathbb{Z}$.
        \end{proof}
      \item Prove that the set of equivalence classes under this equivalence
        relation, which we denote by $\mathbb{Q}/\mathbb{Z}$, is an infinite
        Abelian group.
        \begin{proof}
          To show that the set of equivalence classes form a group, we show
          that addition is well-defined, identity exists, and inverses exist.
          Addition given by $[a]+[b]=[a+b]$ is well-defined because if
          $a_1-a_2\in\mathbb{Z}$, then
          $(a_1+b)-(a_2+b)=a_1-a_2\in\mathbb{Z}$, and similarly if
          $b_1-b_2\in\mathbb{Z}$, then
          $(a+b_1)-(a+b_2)=b_1-b_2\in\mathbb{Z}$.
          $[0]$ is an identity because for any $a\in\mathbb{Z}$,
          $[0]+[a]=[0+a]=[a]=[a+0]=[a]+[0]$. Finally, given any
          $[a]\in\mathbb{Q}/\mathbb{Z}$, defining its inverse as $[a]$ as
          $[-a]$ works since $[a]+[-a]=[a-a]=[0]=[-a+a]=[-a]+[a]$. \\

          The group is Abelian because given $a,b\in\mathbb{Z}$,
          $[a]+[b]=[a+b]=[b+a]=[b]+[a]$. \\

          It remains to prove that the group is infinite. The set
          $\{[1/n]:n\in\mathbb{Z}^+\}$ are unique elements of
          $\mathbb{Q}/\mathbb{Z}$ since $1/m-1/n\notin\mathbb{Z}$ for any
          $m\neq n\in\mathbb{Z}^+$.
        \end{proof}
    \end{enumerate}

  \item Show that if $g^2=e$ for elements $g$ of a group $G$, then $G$ is
    Abelian.
    \begin{proof}
      First note that the inverse of any $g\in G$ is itself since $g^2=e$.
      Given $g,h\in G$,
      \begin{align*}
        gh  & = (gh)^{-1}     && \text{(the inverse of $gh$ is itself)} \\
            & = h^{-1}g^{-1}  && (ghh^{-1}g^{-1}=e=h^{-1}g^{-1}gh) \\
            & = hg            && \text{(the inverse of $g,h$ are themselves)}
      \end{align*}
    \end{proof}

  \item Let $G$ be the group (under ordinary matrix multiplication)
    generated by the complex matrices \label{part_3}
    \begin{align*}
      A =
        \begin{bmatrix}
          0   & 1 \\
          -1  & 0
        \end{bmatrix}
        && and &&
      B =
        \begin{bmatrix}
          0   & i \\
          i   & 0
        \end{bmatrix}
    \end{align*}
    where $i^2=-1$. Show that $G$ is a non-Abelian group of order 8, and
    give its group operation table.
    \begin{proof}
      After some arithmetic, we observe the following rules that generate
      the group:
      \begin{equation} \label{eqn:G}
        \begin{cases}
          A^2=B^2=-I \\
          BA=-AB \\
          \pm A, \pm B, \pm AB, \pm I \text{are distinct} \\
        \end{cases}
      \end{equation}
      Using these rules, we can reduce any finite sequence of $A$'s and
      $B$'s into exactly one of the following eight distinct elements:
      $\{\pm A, \pm B, \pm AB, \pm I\}$. Also, the group is not Abelian
      because $AB\neq BA$. From these rules we construct the group
      operation table as follows: \\
      \begin{center}
        \begin{tabular}{|r||r|r|r|r|r|r|r|r|}
          \hline
          $G$       & $A$   & $-A$  & $B$   & $-B$  & $AB$  & $-AB$ & $I$   & $-I$ \\
          \hline\hline
          $A$       & $-I$  & $I$   & $AB$  & $-AB$ & $-B$  & $B$   & $A$   & $-A$ \\
          \hline
          $-A$      & $I$   & $-I$  & $-AB$ & $AB$  & $B$   & $-B$  & $-A$  & $A$ \\
          \hline
          $B$       & $-AB$ & $AB$  & $-I$  & $I$   & $A$   & $-A$  & $B$   & $-B$ \\
          \hline
          $-B$      & $AB$  & $-AB$ & $I$   & $-I$  & $-A$  & $A$   & $-B$  & $B$ \\
          \hline
          $AB$      & $B$   & $-B$  & $-A$  & $A$   & $-I$  & $I$   & $AB$  & $-AB$ \\
          \hline
          $-AB$     & $-B$  & $B$   & $A$   & $-A$  & $I$   & $-I$  & $-AB$ & $AB$ \\
          \hline
          $I$       & $A$   & $-A$  & $B$   & $-B$  & $AB$  & $-AB$ & $I$   & $-I$ \\
          \hline
          $-I$      & $-A$  & $A$   & $-B$  & $B$   & $-AB$ & $AB$  & $-I$  & $I$ \\
          \hline
        \end{tabular}
      \end{center}
    \end{proof}

  \item \label{part_4}
    \begin{enumerate}
      \item Let $H$ be the group (under ordinary matrix multiplication)
        generated by the real matrices \label{part_a}
        \begin{align*}
          A =
            \begin{bmatrix}
              0   & 1 \\
              -1  & 0
            \end{bmatrix}
            && and &&
          B =
            \begin{bmatrix}
              0   & 1 \\
              1   & 0
            \end{bmatrix}.
        \end{align*}
        Show that $H$ is a non-Abelian group of order 8, and give its group
        operation table.
        \begin{proof}
          After some arithmetic, we observe the following rules that generate
          the group:
          \begin{equation} \label{eqn:H}
            \begin{cases}
              A^2=-I \\
              B^2=I \\
              BA=-AB \\
              \pm A, \pm B, \pm AB, \pm I \text{are distinct} \\
            \end{cases}
          \end{equation}
          Using these rules, we can reduce any finite sequence of $A$'s and
          $B$'s into exactly one of the following eight distinct elements:
          $\{\pm A, \pm B, \pm AB, \pm I\}$. Also, the group is not Abelian
          because $AB\neq BA$. From these rules we construct the group
          operation table as follows: \\
          \begin{center}
            \begin{tabular}{|r||r|r|r|r|r|r|r|r|}
              \hline
              $H$       & $A$   & $-A$  & $B$   & $-B$  & $AB$  & $-AB$ & $I$   & $-I$ \\
              \hline\hline
              $A$       & $-I$  & $I$   & $AB$  & $-AB$ & $-B$  & $B$   & $A$   & $-A$ \\
              \hline
              $-A$      & $I$   & $-I$  & $-AB$ & $AB$  & $B$   & $-B$  & $-A$  & $A$ \\
              \hline
              $B$       & $-AB$ & $AB$  & $I$   & $-I$  & $-A$  & $A$   & $B$   & $-B$ \\
              \hline
              $-B$      & $AB$  & $-AB$ & $-I$  & $I$   & $A$   & $-A$  & $-B$  & $B$ \\
              \hline
              $AB$      & $B$   & $-B$  & $A$   & $-A$  & $I$   & $-I$  & $AB$  & $-AB$ \\
              \hline
              $-AB$     & $-B$  & $B$   & $-A$  & $A$   & $-I$  & $I$   & $-AB$ & $AB$ \\
              \hline
              $I$       & $A$   & $-A$  & $B$   & $-B$  & $AB$  & $-AB$ & $I$   & $-I$ \\
              \hline
              $-I$      & $-A$  & $A$   & $-B$  & $B$   & $-AB$ & $AB$  & $-I$  & $I$ \\
              \hline
            \end{tabular}
          \end{center}
        \end{proof}
      \item Compare your answer to part (\ref{part_a}) with your answer to
        \#\ref{part_3} above.
        \begin{proof}
          Even though groups $G$ and $H$ are non-Abelian and of order eight,
          they are not isomorphic because $G$ has only two elements of
          order two while $H$ has six.
        \end{proof}
    \end{enumerate}

  \item
    \begin{enumerate}
      \item Section 1.5 question 2: Write out the group tables for $S_3$,
        $D_8$, and $Q_8$. \label{part_5a}
        \begin{proof}
          Each element in $S_3$ can be expressed as a commutative product
          of disjoint cycles. From tracking the path of each element, we
          obtain the group table as follows:
          \begin{center}
            \begin{tabular}{|c||c|c|c|c|c|c|}
              \hline
              $S_3$     & e       & (1,2)   & (2,3)   & (1,3)   & (1,2,3) & (1,3,2) \\
              \hline\hline
              e         & e       & (1,2)   & (2,3)   & (1,3)   & (1,2,3) & (1,3,2) \\
              \hline
              (1,2)     & (1,2)   & e       & (1,2,3) & (1,3,2) & (2,3)   & (1,3) \\
              \hline
              (2,3)     & (2,3)   & (1,3,2) & e       & (1,2,3) & (1,3)   & (1,2) \\
              \hline
              (1,3)     & (1,3)   & (1,2,3) & (1,3,2) & e       & (1,2)   & (2,3) \\
              \hline
              (1,2,3)   & (1,2,3) & (1,3)   & (1,2)   & (2,3)   & (1,3,2) & e \\
              \hline
              (1,3,2)   & (1,3,2) & (2,3)   & (1,3)   & (1,2)   & e       & (1,2,3) \\
              \hline
            \end{tabular}
          \end{center}

          $D_8$ is generated by an element $r$ of order 4 and an element
          $s$ of order 2, using the following generating rules:
          \begin{equation} \label{eqn:D8}
            \begin{cases}
              r^4 = e \\
              s^2 = e \\
              rs = sr^{-1}
            \end{cases}
          \end{equation}
          \begin{center}
            \begin{tabular}{|c||c|c|c|c|c|c|c|c|}
              \hline
              $D_8$     & $e$   & $r$   & $r^2$   & $r^3$
                        & $s$   & $sr$  & $sr^2$  & $sr^3$ \\
              \hline\hline
              $e$       & $e$   & $r$   & $r^2$   & $r^3$
                        & $s$   & $sr$  & $sr^2$  & $sr^3$ \\
              \hline
              $r$       & $r$   & $r^2$ & $r^3$   & $e$
                        & $sr^3$& $s$   & $sr$    & $sr^2$ \\
              \hline
              $r^2$     & $r^2$ & $r^3$ & $e$     & $r$
                        & $sr^2$& $sr^3$& $s$     & $sr$ \\
              \hline
              $r^3$     & $r^3$ & $e$   & $r$     & $r^2$
                        & $sr$  & $sr^2$& $sr^3$  & $s$ \\
              \hline
              $s$       & $s$   & $sr$  & $sr^2$  & $sr^3$
                        & $e$   & $r$   & $r^2$   & $r^3$ \\
              \hline
              $sr$      & $sr$  & $sr^2$& $sr^3$  & $s$
                        & $r^3$ & $e$   & $r$     & $r^2$ \\
              \hline
              $sr^2$    & $sr^2$& $sr^3$& $s$     & $sr$
                        & $r^2$ & $r^3$ & $e$     & $r$ \\
              \hline
              $sr^3$    & $sr^3$& $s$   & $sr$    & $sr^2$
                        & $r$   & $r^2$ & $r^3$   & $e$ \\
              \hline
            \end{tabular}
          \end{center}

          Finally, $Q_8$ is generated by elements $-1,i,j,k$, and is
          defined by the following generating rules: 
          \begin{equation} \label{eqn:Q8}
            \begin{cases}
              (-1)\; \text{commutes with all generators} \\
              (-1)^2 = 1 \\
              i^2 = j^2 = k^2 = -1 \\
              ijk = -1
            \end{cases}
          \end{equation}
          \begin{center}
            \begin{tabular}{|c||c|c|c|c|c|c|c|c|}
              \hline
              $Q_8$     & $1$   & $-1$  & $i$     & $-i$  & $j$   & $-j$  & $k$     & $-k$ \\
              \hline\hline
              $1$       & $1$   & $-1$  & $i$     & $-i$  & $j$   & $-j$  & $k$     & $-k$ \\
              \hline
              $-1$      & $-1$  & $1$   & $-i$    & $i$   & $-j$  & $j$   & $-k$    & $k$ \\
              \hline
              $i$       & $i$   & $-i$  & $-1$    & $1$   & $k$   & $-k$  & $-j$    & $j$ \\
              \hline
              $-i$      & $-i$  & $i$   & $1$     & $-1$  & $-k$  & $k$   & $j$     & $-j$ \\
              \hline
              $j$       & $j$   & $-j$  & $-k$    & $k$   & $-1$  & $1$   & $i$     & $-i$ \\
              \hline
              $-j$      & $-j$  & $j$   & $k$     & $-k$  & $1$   & $-1$  & $-i$    & $i$ \\
              \hline
              $k$       & $k$   & $-k$  & $j$     & $-j$  & $-i$  & $i$   & $-1$    & $1$ \\
              \hline
              $-k$      & $-k$  & $k$   & $-j$    & $j$   & $i$   & $-i$  & $1$     & $-1$ \\
              \hline
            \end{tabular}
          \end{center}
        \end{proof}
      \item Compare your answers to \#\ref{part_5a} with your answers to
        problems \#\ref{part_3} and \ref{part_4}.
        \begin{proof}
          Group $S_3$ is clearly not isomorphic with any of the other
          groups because it is the only one that is not of order 8. Groups
          $H$ and $D_8$ have six elements of order 2, while groups $G$ and
          $Q_8$ have only two elements of order 2. Hence we compare $H$
          with $D_8$ and $G$ with $Q_8$ to check for isomorphism. \\

          To check if two groups are isomorphic, we map generators of one
          group to generators of the other and verify if maps give the same
          generating rules for each group. \\

          We first try to find an isomorphism between $H$ and $D_8$.
          Comparing the group tables of $D_8$ with $H$, we notice that both
          groups have exactly two generators, and generator $A\in H$ and
          $r\in D_8$ have order 4, and generator $B\in H$ and $s\in D_8$
          have order 2. Hence, we map $A$ to $r$ and $B$ to $s$ and check
          if this map gives us the same generating rules for both groups.
          We first check that the rules for $H$, given by
          equations~(\ref{eqn:H}), generate the rules for $D_8$, given by
          equations~(\ref{eqn:D8}). This is true since
          \[\left \{
            \begin{tabular}{lll}
              $ord(A)$  & = $4$         & (says $ord(r)=4$) \\
              $ord(B)$  & = $2$         & (says $ord(s)=2$) \\
              $AB$      & = $-BA$       & (from generating
              rules~(\ref{eqn:H})) \\
                        & = $B(-A)$     & ($\because -I commutes$) \\
                        & = $BA^{-1}$   & ($\because -A$ is the inverse of $A$) \\
                        &               & (says $rs=sr^{-1}$) \\
            \end{tabular}
          \right .\ \]

          Finally, we check that equation rules~(\ref{eqn:H}) generate
          equation rules~(\ref{eqn:D8}). Note that $-I\in H$ will be mapped
          to the image of $A^2$, which is $r^2$. Equation
          rules~(\ref{eqn:H}) generate group $H$ with elements $A,B,-I\in
          H$, where $-I$ commutes across all elements, and has order 2.
          Hence besides the rules listed in rules~(\ref{eqn:H}), we must
          also verify that $r^2$ has order 2 and commutes across generators
          $r$ and $s$.
          \[\left \{
            \begin{tabular}{lll}
              $r^2$ & = $r^2$       & (says $A^2=-I$) \\
              $s^2$ & = $e$         & (says $B^2=I$) \\
              $ord(r^2)$ & = $2$    & (says $ord(-I)=2$) \\
              $r^2$ & commutes across all elements & (straightforward to verify \\
                    &               & $r^2$ commutes across generators $r$ and $s$) \\
                    &               & (says $-I$ commutes) \\
              $sr$  & = $r^{-1}s$   & ($\because rs=sr^{-1}$) \\
                    & = $r^3s$      & ($\because r^{-1}=r^3$) \\
                    & = $(r^2)rs$   & (says to $BA=-AB$) \\
            \end{tabular}
          \right .\ \]

          Hence, group $H$ is isomorphic to $D_8$. Now we use the same
          approach to show that $G$ is isomorphic to $Q_8$. We map the
          only order 2 elements $-I$ and $-1$ to each other, and map $A$ to
          $i$, and $B$ to $j$. To find a map for $k$, Quaternion
          rules~(\ref{eqn:Q8}) show that $k=j^{-1}i^{-1}(-1)$, so looking at
          group $G$'s table, $k$ must be mapped from $(-B)(-A)(-1)=-BA=AB$.
          \\

          We now check that the map helps us to get equation
          rules~(\ref{eqn:Q8}) from rules~(\ref{eqn:G}):
          \[\left \{
            \begin{tabular}{lll}
              $-I$      & commutes across all generators
                                        & (says $-1$ commutes) \\
              $(-I)^2$  & = $I$         & (says $ord(-1) = 2$) \\
              $A^2$     & = $-I$        & (says $i^2=-1$) \\
              $B^2$     & = $-I$        & (says $j^2=-1$) \\
              $(AB)^2$  & = $ABAB$      & \\
                        & = $A^2B^2$    & ($\because BA=-AB$) \\
                        & = $-I$        & ($\because A^2=B^2=-I$) \\
                        &               & (says $k^2=-1$) \\
              $AB(AB)$  & = $-I$        & (from argument above) \\
                        &               & (says $ijk=-1$) \\
            \end{tabular}
          \right .\ \]

          Finally, we check the map helps us to get equation
          rules~(\ref{eqn:G}) from rules~(\ref{eqn:Q8}):
          \[\left \{
            \begin{tabular}{lll}
              $-1$      & commutes across all generators
                                        & (says $-I$ commutes) \\
              $(-1)^2$  & = $1$         & (says $ord(-I) = 2$) \\
              $i^2$     & = $-1$        & (says $A^2=-I$) \\
              $j^2$     & = $-1$        & (says $B^2=-I$) \\
              $ji$      & = $-ji^{-1}$  & ($\because i=-i^{-1}$) \\
                        & = $-ij^{-1}$  & ($\because i^2=j^2$) \\
                        & = $-ij$       & ($\because j=-j^{-1}$) \\
                        &               & (says $BA=-AB$) \\
            \end{tabular}
          \right .\ \]
        \end{proof}
    \end{enumerate}

  \item Section 1.1
    \begin{enumerate}
      \item Question 30: Prove that the elements $(a,1)$ and $(1,b)$ of
        $A\times B$ commute and deduce that the order of $(a,b)$ is the
        least common multiple of $|a|$ and $|b|$.
        \begin{proof}
          The elements commute because
          $(a,1)\times(1,b)=(a,b)=(1,b)\times(a,1)$. \\

          Now
          \begin{align*}
            (a,b)^n & = ((a,1)(1,b))^n && \\
                    & = (a,1)^n(1,b)^n && \text{because $(a,1)$ and $(1,b)$
                    commute} \\
                    & = (a^n,1)(1,b^n) && \\
                    & = (a^n,b^n),     &&
          \end{align*}
          which equals $(1,1)$ iff $a^n=b^n=1$. Hence $ord((a,b))$ is the
          least common multiple of $ord(a)$ and $ord(b)$.
        \end{proof}
      \item Question 31: Prove that any finite group $G$ of even order
        contains an element of order 2.
        \begin{proof}
          Let $t(G)$ be the set $\{g\in G | g \neq g^{-1}\}$. We first show
          that $t(G)$ has an even number of elements. Each element $g$ is
          in $t(G)$ iff its inverse, which is distinct from itself, is also
          in $t(G)$. Hence, we can pair the elements in $t(G)$ by their
          unique inverses, which implies that $t(G)$ can only have an even
          number of elements. \\

          Since $G$ and $t(G)$ have an even number of elements, $G-t(G)$
          must also have an even number of elements. Since $G-t(G)$
          contains the identity, this set must also contain a non-identiy
          element, which would have order 2 as required.
        \end{proof}
      \item Question 32: If $x$ is an element of finite order $n$ in $G$,
        prove that the elements $1,x,x^2,\ldots,x^{n-1}$ are all distinct.
        Deduce that $|x|\leq |G|$.
        \begin{proof}
          Assume by contradiction that $x^i=x^j$ for some $0\leq i<j \leq
          n-1$. Then $x^{(j-i)}=e$, where $j-i<n$, which contradicts the
          order of $x$ in $G$. \\

          Since the $n$ elements are distinct, $G$ must have at least $n$
          elements.
        \end{proof}
      \item Question 35: If $x$ is an element of finite order $n$ in $G$,
        use the Division Algorithm to show that any integral power of $x$
        equals one of the elements in the set $\{1,x,x^2,\ldots,x^{n-1}\}$.
        \begin{proof}
          Given any $s\in\mathbb{Z}$, we can use the Division Algorithm to
          uniquely express $s$ as $s=kn+r$ for some integers $k$ and $r$,
          where $0\leq r\leq n-1$. Then
          \begin{align*}
            x^s & = x^{kn+r} \\
                & = x^kn \times x^r \\
                & = (x^n)^k \times x^r \\
                & = e^k \times x^r \\
                & = e \times x^r \\
                & = x^r,
          \end{align*}
          as required.
        \end{proof}
      \item Question 36: Assume $G=\{1,a,b,c\}$ is a group of order 4 with
        identity 1. Assume also that $G$ has no elements of order 4. Use
        cancellation laws to show that there is a unique group table for
        $G$. Deduce that $G$ is Abelian.
        \begin{proof}
          Given any $x\neq y\in G-\{1\}$, their product $xy$ cannot equal
          $x$ otherwise $y=1$ by cancellation law. Similarly, $xy\neq y$,
          so $xy$ can only equal the only element in $G-\{x,y,1\}$. This
          property fills all non-diagonal entries of the group table, and
          also shows that $G$ is Abelian. \\

          It remains to fill the group table's diagonal entries. By
          question 32, each non-identity element has order 2 or 3. We show
          that the order can only be 2. Assume by contradiction that one of
          these elements has order 3. We can assume without loss of
          generality that this element is $a$, and that $a^2=b$. Then
          $ab=a\times a^2 = a^3 = 1$, which contradicts our earlier
          argument that $ab=c$. \\

          Summarizing, we get the following group table: \\
          \begin{center}
            \begin{tabular}{|c||c|c|c|c|}
              \hline
              $G$       & 1 & a & b & c \\
              \hline\hline
              1         & 1 & a & b & c \\
              \hline
              a         & a & 1 & c & b \\
              \hline
              b         & b & c & 1 & a \\
              \hline
              c         & c & b & a & 1 \\
              \hline
            \end{tabular}
          \end{center}
        \end{proof}
    \end{enumerate}

  \item Section 1.2
    \begin{enumerate}
      \item Question 3: Use the generators and relations above to show that
        every element of $D_{2n}$, which is not a power of $r$ has order 2.
        Deduce that $D_{2n}$ is generated by the two elements $s$ and $sr$,
        both of which have order 2.
        \begin{proof}
          Using the relations given, we can reduce any finite product of
          sequences $g\in\{r,s\}^{<\omega}$ into exactly one of the
          following:
          $\{e,r,r^2,\ldots,r^{n-1},s,sr,sr^2,\ldots,sr^{n-1}\}$. Hence the
          only elements of $D_{2n}$ that are not powers of $r$ are of the
          form $sr^k$ for some $k\in\{0,\ldots,n-1\}$. \\
          
          We show by induction on $k$ that all such elements $sr^k$ have
          order 2. The base case is true since $s$ has order 2. For the
          inductive step,
          \begin{align*}
            (sr^{k+1})^2  &= sr^{k+1}sr^{k+1}     && \\
                          &= sr^k(rs)r^{k+1}      && \\
                          &= sr^k(sr^{-1})r^{k+1} && (rs=sr^{-1}) \\
                          &= sr^ksr^k             && \\
                          &= (sr^k)(sr^k)         && \\
                          &= (sr^k)^2             && \\
                          &= e                    && \text{(induction).} \\
          \end{align*}
        \end{proof}
      \item Question 7: Show that $G=\langle a,b\,|\,
        a^2=b^2=(ab)^n=1\rangle$
        gives a presentation for $D_{2n}$ in terms of the two generators
        $a=s$ and $b=sr$ of order 2 computed in Exercise 3 above.
        \begin{proof}
          In Exercise 3 above, we have already shown that $a=s$ and $b=sr$
          have order 2. Also, $(ab)^n=(s^2r)^n=(e\cdot r)^n=r^n=1$. \\

          To show that the given group $G$ is equivalent to $D_{2n}$, we map
          the generators $a,b$ to elements in $D_{2n}$, and show that the
          map preserves the generating rules of $D_{2n}$ from
          the rules of $G$, and vice-versa. Mapping $a$ to $s$ and $b$ to
          $sr$ suggests that $r$ can be mapped from the pre-image of
          $r=s(sr)$, which is $ab$, and that $s$ can be mapped from $a$. \\

          We first use these maps to get the generating rules of $D_{2n}$
          from the generating rules of $G$:
          \[\left \{
            \begin{tabular}{lll}
              $ord(ab)$ & = $n$             & (says $ord(r)=n$) \\
              $ord(a)$  & = $2$             & (says $ord(s)=2$) \\
              $(ab)a$   & = $ab^{-1}a^{-1}$ & ($\because a^2=b^2=1$) \\
                        & = $a(ab)^{-1}$    & (says $rs=sr^{-1}$) \\
            \end{tabular}
          \right .\ \]

          Finally, we use these maps to get the generating rules of $G$
          from the generating rules of $D_{2n}$:
          \[\left \{
            \begin{tabular}{lll}
              $ord(s)$  & = $2$             & (says $ord(a)=2$) \\
              $ord(sr)$ & = $2$             & ($\because
              srsr=s(rs)r=s(sr^{-1})r=s^2=e$) \\
                        &                   & (says $ord(b)=2$) \\
              $r^n$     & = $e$             & (says $(ab)^n=1$) \\
            \end{tabular}
          \right .\ \]

        \end{proof}
    \end{enumerate}

  \item Section 1.3
    \begin{enumerate}
      \item Question 11: Let $\sigma$ be the $m$-cycle $(1 2 \ldots m)$.
        Show that $\sigma^i$ is also an $m$-cycle if and only if $i$ is
        relatively prime to $m$.
        \begin{proof}
          Let $d$ be the greatest common divisor of $i\in\mathbb{Z}$ and $m$.
          Then
          \begin{align*}
            (\sigma^{i})^{m/d}  &= \sigma^{im/d}      && \\
                                &= (\sigma^{i/d})^m   && \text{since $d|i$} \\
                                &= (\sigma^m)^{i/d}   && \\
                                &= e^{i/d}            && \\
                                &= e                  && \\
          \end{align*}
          Hence the order of $\sigma^i$ is less than or equal $m/d$, which
          is less than $m$ if $i$ is not relatively prime to $m$. \\

          To prove the converse, assume $i$ and $m$ are relatively prime.
          Then there are integers $a,b\in\mathbb{Z}$ such that $ia+mb=1$.
          Note that the order of $\sigma^i$ must be smaller or equal $m$
          since $(\sigma^i)^m = (\sigma^m)^i = e^i = e$.
          Assume by contradiction that $\sigma^i$ has an order $k$ smaller
          than $m$. Then
          \begin{align*}
            \sigma^k  &= \sigma^{1\cdot k} \\
                      &= \sigma^{(ia+mb)\cdot k} \\
                      &= \sigma^{(ika+mbk)} \\
                      &= ((\sigma^i)^k)^a \cdot (\sigma^m)^{bk} \\
                      &= e^a \cdot e^{bk} \\
                      &= e, \\
          \end{align*}
          which contradicts the order of $\sigma$.
        \end{proof}
      \item Question 14: Let $p$ be a prime. Show that an element has order
        $p$ in $S_n$ if and only if its cycle decomposition is a product of
        commuting $p$-cycles. Show by an explicit example that this need
        not be the case if $p$ is not prime.
        \begin{proof}
          First, note that any element $\sigma$ in $S_n$ can be decomposed
          into a finite product of cycles $\prod_{i<n}{c_i}$, where the
          cycles $c_i$ are disjoint in the sense that each integer appears
          at most once across the cycles. Because of disjointness, the
          cycles commute, hence given any $k\in\mathbb{Z}^+$,
          \begin{align*}
            \sigma^k  & = \left(\prod_{i<n}{c_i}\right)^k \\
                      & = \prod_{i<n}{c_i^k}. \\
          \end{align*}
          Also because of disjointness of the $c_i$'s and hence of the
          $c_i^k$'s, the power $\sigma^k$ equals $e$ if and only if each
          $c_i^k$ equals $e$, which occurs for the first time when $k$ is
          the lowest common multiple of the orders of each cycle $c_i$.
          Hence, $ord(\sigma) = lcm(|c_0|,\ldots,|c_{n-1}|)$. This order is
          a prime $p$ if and only if each $|c_i|$ is $p$, as we are
          required to show. \\

          $\sigma = (1,2)(3,4,5)$ is an explicit example of an element
          in $S_n$ that does not have a prime order. This element has order
          $|(1,2)|\cdot |(3,4,5)| = 2\cdot 3 = 6$, which is not prime.
        \end{proof}
      \item Question 15: Prove that the order of an element in $S_n$ equals
        the least common multiple of the lengths of the cycles in its cycle
        decomposition.
        \begin{proof}
          We proved this in the answer to section 1.3 question 14. 
        \end{proof}
      \item Question 18: Find all numbers $n$ such that $S_5$ contains an
        element of order $n$.
        \begin{proof}
          From question 15, we proved that the order of an element equals
          the least common multiple of the lengths of the cycles in its
          cycle decomposition. Hence to find all possible $n$, we consider
          all positive integer sums to 5, and let $n$ be the lowest common
          multiple of these solutions. Thus the possible values of $n$ are:
          \begin{align*}
            n && =         && lcm(1,1,1,1,1)  & = 1 \\
              && \text{or} && lcm(1,1,1,2)    & = 2 \\
              && \text{or} && lcm(1,1,3)      & = 3 \\
              && \text{or} && lcm(1,2,2)      & = 2 \\
              && \text{or} && lcm(1,4)        & = 4 \\
              && \text{or} && lcm(2,3)        & = 6 \\
              && \text{or} && lcm(5)          & = 5 \\
          \end{align*}
          Summarizing, the possible values of $n$ are $1,2,\ldots,6$.
        \end{proof}
    \end{enumerate}

  \item Section 1.4
    \begin{enumerate}
      \item Question 7: Let $p$ be a prime. Prove that the order of
        $GL_2(\mathbb{F}_p)$ is $p^4-p^3-p^2+p$.
        \begin{proof}
          The total number of $2\times 2$ matrices over $\mathbb{F}_p$ is
          $p^4$, since $|\mathbb{F}_p|=p$. A $2\times 2$ matrix is not
          invertible if and only if its first row is a multiple of the
          second. Excluding the row with both entries zero, the number of
          unique first rows is $p^2-1$, and for each of these rows, the
          number of unique second rows that is a multiple of the first is
          $(p^2-1)\times p = p^3-p$. Also, the number of matrices with
          first row being zero is $p^2$. These are exactly the matrices
          that are not invertible. Hence, subtracting from the total number
          of matrices, the number of invertible matrices is
          $p^4-p^3-p^2+p$.
        \end{proof}
      \item Question 11: Let
        $H(F) = \left \{
          \begin{bmatrix}
            1 & a & b \\
            0 & 1 & c \\
            0 & 0 & 1 \\
          \end{bmatrix}
        \,\middle|\,a,b,c\in F\right\}$ -- called the \textit{Heisenberg
        group} over $F$. Let
        $X =
          \begin{bmatrix}
            1 & a & b \\
            0 & 1 & c \\
            0 & 0 & 1 \\
          \end{bmatrix} $
        and
        $Y =
          \begin{bmatrix}
            1 & d & e \\
            0 & 1 & f \\
            0 & 0 & 1 \\
          \end{bmatrix} $
        be elements of $H(F)$.
        \begin{enumerate}
          \item Compute the matrix product $XY$ and deduce that $H(F)$ is
            closed under matarix multiplication. Exihibit explicit matrices
            such that $XY\neq YX$.
            \begin{proof}
              $XY = \begin{bmatrix}
                  1 & a & b \\
                  0 & 1 & c \\
                  0 & 0 & 1 \\
                \end{bmatrix}
                \begin{bmatrix}
                  1 & d & e \\
                  0 & 1 & f \\
                  0 & 0 & 1 \\
                \end{bmatrix}
                =
                \begin{bmatrix}
                  1 & a+d & b+e+af \\
                  0 & 1   & c+f \\
                  0 & 0   & 1 \\
                \end{bmatrix} \in H(F)$, hence $H(F)$ is closed under
              matrix multiplication. \\

              Consider the case where $X =
                \begin{bmatrix}
                  1 & 1 & 0 \\
                  0 & 1 & 0 \\
                  0 & 0 & 1 \\
                \end{bmatrix}$
              and $Y =
                \begin{bmatrix}
                  1 & 0 & 0 \\
                  0 & 1 & 1 \\
                  0 & 0 & 1 \\
                \end{bmatrix}$. \\

              Then $XY =
                \begin{bmatrix}
                  1 & 1 & 1 \\
                  0 & 1 & 1 \\
                  0 & 0 & 1 \\
                \end{bmatrix} \neq
                \begin{bmatrix}
                  1 & 1 & 0 \\
                  0 & 1 & 1 \\
                  0 & 0 & 1 \\
                \end{bmatrix} = YX$.
            \end{proof}
          \item Find an explicit formula for the matrix inverse $X^{-1}$
            and deduce that $H(F)$ is closed under inverses.
            \begin{proof}
              From the explicit computation of $XY$ in the previous part,
              given $X$, it suffices to find $d,e,f\in F$ such that
              $a+d=c+f=b+e+af=0$. Solving for $d,e,f$, we get $d=-a$,
              $f=-c$, and $e=-b-af=ac-b$. Summarizing, $X^{-1} =
                \begin{bmatrix}
                  1 & -a  & ac-b \\
                  0 & 1   & -c \\
                  0 & 0   & 1 \\
                \end{bmatrix}$, which is contained in $H(F)$, hence $H(F)$
              is closed under inverses.
            \end{proof}
          \item Prove the associative law for $H(F)$ and deduce that $H(F)$
            is a group of order $|F|^3$.
            \begin{proof}
              Since $H(F)$ is closed under product, it suffices to show
              that the three elements in the top right corner of the matrix
              satisfy the associative law. \\

              Given $X,Y,Z\in H(F)$, we wish to show that $(XY)Z = X(YZ)$.
              Observing the computation of $XY$ in the first part of
              question 11, the element in row 1 column 2 of the product
              depends only on the sum of the elements in the original
              matrices, hence for the element in this particular position,
              matrix multiplication is associative. A similar argument can
              be made for the element in row 2 column 1. \\

              It remains to show that associativity holds for the element
              in row 1 column 3. Let $Z =
                \begin{bmatrix}
                  1 & g & h \\
                  0 & 1 & i \\
                  0 & 0 & 1 \\
                \end{bmatrix}$.
              Then for the element in the top right position, $(XY)Z$ gives
              $b+e+af+h+(a+d)i$, while $X(YZ)$ gives $b+(e+h+di)+a(f+i)$,
              which are equal. Hence multiplication is associative. \\

              Since $H(F)$ is closed, associative, contains the unit, and
              inverses, it is a group. The order of $H(F)$ is $|F|^3$, for
              each combination of the three variables $a,b,c\in F$. 
            \end{proof}
          \item Find the order of each element of the finite group
            $H(\mathbb{Z}/2\mathbb{Z})$.
            \begin{proof}
              From earlier argument, this group has exactly
              $|\mathbb{Z}/2\mathbb{Z}|^3=8$ elements, defined by
              combinations of $a,b,c\in\mathbb{Z}/2\mathbb{Z}$. We compute
              the order of each of these elements. \\
              
              We make a few observations to speed up the computations.
              First, from the computation of $XY$ in the first part of
              question 11, replacing $Y$ with $X$, we get $X^2 =
                \begin{bmatrix}
                  1 & 2a  & 2b+ac \\
                  0 & 1   & 2c \\
                  0 & 0   & 1 \\
                \end{bmatrix} $,
              which will always equal to identity unless $a=c=1$. Hence all
              non-identiy matrices with either $a=0$ or $c=0$ have order 2.
              \\

              It remains to compute the order of the two matrices with
              $a=c=1$. Observe from the explicit formula of the inverse
              matrix computed earlier that these two matrices are inverses
              of each other, hence they have the same order. So it suffices
              to compute the order of the matrix with $a=c=1$ and $b=0$. We
              check that this matrix has order 4. \\

              The following table summarizes the computed orders:
              \begin{center}
                \begin{tabular}{|c|c|c||c|}
                  \hline
                  a & b & c & Order \\
                  \hline\hline
                  0 & 0 & 0 & 1 \\
                  \hline
                  0 & 0 & 1 & 2 \\
                  \hline
                  0 & 1 & 0 & 2 \\
                  \hline
                  0 & 1 & 1 & 2 \\
                  \hline
                  1 & 0 & 0 & 2 \\
                  \hline
                  1 & 0 & 1 & 4 \\
                  \hline
                  1 & 1 & 0 & 2 \\
                  \hline
                  1 & 1 & 1 & 4 \\
                  \hline
                \end{tabular}
              \end{center}
            \end{proof}
          \item Prove that every non-identity element of the group
            $H(\mathbb{R})$ has infinite order.
            \begin{proof}
              Let $X = \begin{bmatrix}
                  1 & a & b \\
                  0 & 1 & c \\
                  0 & 0 & 1 \\
                \end{bmatrix}$ be an element with finite order. We show
              that $X$ can only be identity. \\

              From the computation of $XY$ in the first part of question
              11, replacing $Y$ with $X$ and applying induction, we
              observe that the element in row 1 column 2 of $X^n$ is
              $n\cdot a$, and the element in row 2 column 1 of $X^n$ is
              $n\cdot c$. Hence for $X$ to have finite order $n$, we need
              $n\cdot a=n\cdot c=0$, which is only possible if $a=c=0$
              since all non-zero elements in $\mathbb{R}$ have infinite
              order. When $a=c=0$, we apply induction again to show that
              the element in row 1 column 2 of $X^n$ is $n\cdot b$. By
              similar argument, for $X$ to have finite order we also need
              $b=0$. Hence $X$ can only be the identity. 
            \end{proof}
        \end{enumerate}
    \end{enumerate}
\end{enumerate}

\end{document}
