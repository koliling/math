\documentclass{article}
\usepackage[left=3cm,right=3cm,top=3cm,bottom=3cm]{geometry}
\usepackage{amsmath,amssymb,amsthm}
%\setlength{\parindent}{0mm}

\begin{document}
\title{Math 60210: Homework 1}
\author{Li Ling Ko\\ lko@nd.edu}
\date{\today}
\maketitle

\begin{enumerate}
  \item
    \begin{enumerate}
      \item Prove that the relation given by $a \sim b \Longleftrightarrow
        a-b \in \mathbb{Z} $ is an equivalence relation on the additive group
        $\mathbb{Q}$.
        \begin{proof}
          A relation is an equivalence one iff it is reflexive, symmetric,
          and transitive. The given relation is reflexive since
          $a-a=0\in\mathbb{Z}$. The relation is also symmetric since
          $b-a=-(a-b)$, so if $a-b\in\mathbb{Z}$, then $b-a\in\mathbb{Z}$.
          Finally, the relation is transitive because if $a-b\in\mathbb{Z}$
          and $b-c\in\mathbb{Z}$, then $a-c=(a-b)+(b-c)\in\mathbb{Z}$.
        \end{proof}
      \item Prove that the set of equivalence classes under this equivalence
        relation, which we denote by $\mathbb{Q}/\mathbb{Z}$, is an infinite
        Abelian group.
        \begin{proof}
          To show that the set of equivalence classes form a group, we show
          that addition is well-defined, identity exists, and inverses exist.
          Addition given by $[a]+[b]=[a+b]$ is well-defined because if
          $a_1-a_2\in\mathbb{Z}$, then
          $(a_1+b)-(a_2+b)=a_1-a_2\in\mathbb{Z}$, and similarly if
          $b_1-b_2\in\mathbb{Z}$, then
          $(a+b_1)-(a+b_2)=b_1-b_2\in\mathbb{Z}$.
          $[0]$ is an identity because for any $a\in\mathbb{Z}$,
          $[0]+[a]=[0+a]=[a]=[a+0]=[a]+[0]$. Finally, given any
          $[a]\in\mathbb{Q}/\mathbb{Z}$, defining its inverse as $[a]$ as
          $[-a]$ works since $[a]+[-a]=[a-a]=[0]=[-a+a]=[-a]+[a]$. \\

          The group is Abelian because given $a,b\in\mathbb{Z}$,
          $[a]+[b]=[a+b]=[b+a]=[b]+[a]$. \\

          It remains to prove that the group is infinite. The set
          $\{[1/n]:n\in\mathbb{Z}^+\}$ are unique elements of
          $\mathbb{Q}/\mathbb{Z}$ since $1/m-1/n\notin\mathbb{Z}$ for any
          $m\neq n\in\mathbb{Z}^+$.
        \end{proof}
    \end{enumerate}

  \item Show that if $g^2=e$ for elements $g$ of a group $G$, then $G$ is
    Abelian.
    \begin{proof}
      First note that the inverse of any $g\in G$ is itself since $g^2=e$.
      Given $g,h\in G$,
      \begin{align*}
        gh  & = (gh)^{-1}     && \text{(the inverse of $gh$ is itself)} \\
            & = h^{-1}g^{-1}  && (ghh^{-1}g^{-1}=e=h^{-1}g^{-1}gh) \\
            & = hg            && \text{(the inverse of $g,h$ are themselves)}
      \end{align*}
    \end{proof}

  \item Let $G$ be the group (under ordinary matrix multiplication)
    generated by the complex matrices \label{part_3}
    \begin{align*}
      A =
        \begin{bmatrix}
          0   & 1 \\
          -1  & 0
        \end{bmatrix}
        && and &&
      B =
        \begin{bmatrix}
          0   & i \\
          i   & 0
        \end{bmatrix}
    \end{align*}
    where $i^2=-1$. Show that $G$ is a non-Abelian group of order 8, and
    give its group operation table.
    \begin{proof}
      After some arithmetic, we observe the following:
      \begin{itemize}
        \item $A^2=B^2=-I$
        \item $BA=-AB$
        \item $\pm A, \pm B, \pm AB, \pm I$ are distinct 
      \end{itemize}
      Using these rules, we can reduce any finite sequence of $A$'s and
      $B$'s into exactly one of the following eight distinct elements:
      $\{\pm A, \pm B, \pm AB, \pm I\}$. Also, the group is not Abelian
      because $AB\neq BA$. From these rules we construct the group
      operation table as follows: \\
      \begin{center}
        \begin{tabular}{|r||r|r|r|r|r|r|r|r|}
          \hline
          $G$       & $A$   & $-A$  & $B$   & $-B$  & $AB$  & $-AB$ & $I$   & $-I$ \\
          \hline\hline
          $A$       & $-I$  & $I$   & $AB$  & $-AB$ & $-B$  & $B$   & $A$   & $-A$ \\
          \hline
          $-A$      & $I$   & $-I$  & $-AB$ & $AB$  & $B$   & $-B$  & $-A$  & $A$ \\
          \hline
          $B$       & $-AB$ & $AB$  & $-I$  & $I$   & $A$   & $-A$  & $B$   & $-B$ \\
          \hline
          $-B$      & $AB$  & $-AB$ & $I$   & $-I$  & $-A$  & $A$   & $-B$  & $B$ \\
          \hline
          $AB$      & $B$   & $-B$  & $-A$  & $A$   & $-I$  & $I$   & $AB$  & $-AB$ \\
          \hline
          $-AB$     & $-B$  & $B$   & $A$   & $-A$  & $I$   & $-I$  & $-AB$ & $AB$ \\
          \hline
          $I$       & $A$   & $-A$  & $B$   & $-B$  & $AB$  & $-AB$ & $I$   & $-I$ \\
          \hline
          $-I$      & $-A$  & $A$   & $-B$  & $B$   & $-AB$ & $AB$  & $-I$  & $I$ \\
          \hline
        \end{tabular}
      \end{center}
    \end{proof}

  \item \label{part_4}
    \begin{enumerate}
      \item Let $H$ be the group (under ordinary matrix multiplication)
        generated by the real matrices \label{part_a}
        \begin{align*}
          A =
            \begin{bmatrix}
              0   & 1 \\
              -1  & 0
            \end{bmatrix}
            && and &&
          B =
            \begin{bmatrix}
              0   & 1 \\
              1   & 0
            \end{bmatrix}.
        \end{align*}
        Show that $H$ is a non-Abelian group of order 8, and give its group
        operation table.
        \begin{proof}
          After some arithmetic, we observe the following:
          \begin{itemize}
            \item $A^2=-I$
            \item $B^2=I$
            \item $BA=-AB$
            \item $\pm A, \pm B, \pm AB, \pm I$ are distinct 
          \end{itemize}
          Using these rules, we can reduce any finite sequence of $A$'s and
          $B$'s into exactly one of the following eight distinct elements:
          $\{\pm A, \pm B, \pm AB, \pm I\}$. Also, the group is not Abelian
          because $AB\neq BA$. From these rules we construct the group
          operation table as follows: \\
          \begin{center}
            \begin{tabular}{|r||r|r|r|r|r|r|r|r|}
              \hline
              $H$       & $A$   & $-A$  & $B$   & $-B$  & $AB$  & $-AB$ & $I$   & $-I$ \\
              \hline\hline
              $A$       & $-I$  & $I$   & $AB$  & $-AB$ & $-B$  & $B$   & $A$   & $-A$ \\
              \hline
              $-A$      & $I$   & $-I$  & $-AB$ & $AB$  & $B$   & $-B$  & $-A$  & $A$ \\
              \hline
              $B$       & $-AB$ & $AB$  & $I$   & $-I$  & $-A$  & $A$   & $B$   & $-B$ \\
              \hline
              $-B$      & $AB$  & $-AB$ & $-I$  & $I$   & $A$   & $-A$  & $-B$  & $B$ \\
              \hline
              $AB$      & $B$   & $-B$  & $A$   & $-A$  & $I$   & $-I$  & $AB$  & $-AB$ \\
              \hline
              $-AB$     & $-B$  & $B$   & $-A$  & $A$   & $-I$  & $I$   & $-AB$ & $AB$ \\
              \hline
              $I$       & $A$   & $-A$  & $B$   & $-B$  & $AB$  & $-AB$ & $I$   & $-I$ \\
              \hline
              $-I$      & $-A$  & $A$   & $-B$  & $B$   & $-AB$ & $AB$  & $-I$  & $I$ \\
              \hline
            \end{tabular}
          \end{center}
        \end{proof}
      \item Compare your answer to part (\ref{part_a}) with your answer to
        \#\ref{part_3} above.
        \begin{proof}
          Even though groups $G$ and $H$ are non-Abelian and of order eight,
          they are not isomorphic because $G$ has only two elements of
          order two while $H$ has six.
        \end{proof}
    \end{enumerate}

  \item
    \begin{enumerate}
      \item Section 1.5 question 2: Write out the group tables for $S_3$,
        $D_8$, and $Q_8$. \label{part_5a}
        \begin{proof}
          The group tables are as follows:
          \begin{center}
            \begin{tabular}{|c||c|c|c|c|c|c|}
              \hline
              $S_3$     & (1,2)   & (1,3)   & (2,3)   & (1,2,3) & (1,3,2) & e \\
              \hline\hline
              (1,2)     & e       & (1,3)   & (2,3)   & (1,2,3) & (1,3,2) & e \\
              \hline
            \end{tabular}
          \end{center}
        \end{proof}
      \item Compare your anwers to \#\ref{part_5a} with your answers to
        problems \#\ref{part_3} and \ref{part_4}.
    \end{enumerate}

  \item Section 1.1
    \begin{enumerate}
      \item Question 30: Prove that the elements $(a,1)$ and $(1,b)$ of
        $A\times B$ commute and deduce that the order of $(a,b)$ is the
        least common multiple of $|a|$ and $|b|$.
        \begin{proof}
          The elements commute because
          $(a,1)\times(1,b)=(a,b)=(1,b)\times(a,1)$. \\

          Now
          \begin{align*}
            (a,b)^n & = ((a,1)(1,b))^n && \\
                    & = (a,1)^n(1,b)^n && \text{because $(a,1)$ and $(1,b)$
                    commute} \\
                    & = (a^n,1)(1,b^n) && \\
                    & = (a^n,b^n),     &&
          \end{align*}
          which equals $(1,1)$ iff $a^n=b^n=1$. Hence $ord((a,b))$ is the
          least common multiple of $ord(a)$ and $ord(b)$.
        \end{proof}
      \item Question 31: Prove that any finite group $G$ of even order
        contains an element of order 2.
        \begin{proof}
          Let $t(G)$ be the set $\{g\in G | g \neq g^{-1}\}$. We first show
          that $t(G)$ has an even number of elements. Each element $g$ is
          in $t(G)$ iff its inverse, which is distinct from itself, is also
          in $t(G)$. Hence, we can pair the elements in $t(G)$ by their
          unique inverses, which implies that $t(G)$ can only have an even
          number of elements. \\

          Since $G$ and $t(G)$ have an even number of elements, $G-t(G)$
          must also have an even number of elements. Since $G-t(G)$
          contains the identity, this set must also contain a non-identiy
          element, which would have order 2 as required.
        \end{proof}
      \item Question 32: If $x$ is an element of finite order $n$ in $G$,
        prove that the elements $1,x,x^2,\ldots,x^{n-1}$ are all distinct.
        Deduce that $|x|\leq |G|$.
        \begin{proof}
          Assume by contradiction that $x^i=x^j$ for some $0\leq i<j \leq
          n-1$. Then $x^{(j-i)}=e$, where $j-i<n$, which contradicts the
          order of $x$ in $G$. \\

          Since the $n$ elements are distinct, $G$ must have at least $n$
          elements.
        \end{proof}
      \item Question 35: If $x$ is an element of finite order $n$ in $G$,
        use the Division Algorithm to show that any integral power of $x$
        equals one of the elements in the set $\{1,x,x^2,\ldots,x^{n-1}\}$.
        \begin{proof}
          Given any $s\in\mathbb{Z}$, we can use the Division Algorithm to
          uniquely express $s$ as $s=kn+r$ for some integers $k$ and $r$,
          where $0\leq r\leq n-1$. Hence
          \begin{align*}
            x^s & = x^{kn+r} \\
                & = x^kn \times x^r \\
                & = (x^n)^k \times x^r \\
                & = e^k \times x^r \\
                & = e \times x^r \\
                & = x^r,
          \end{align*}
          as required.
        \end{proof}
      \item Question 36: Assume $G=\{1,a,b,c\}$ is a group of order 4 with
        identity 1. Assume also that $G$ has no elements of order 4. Use
        cancellation laws to show that there is a unique group table for
        $G$. Deduce that $G$ is Abelian.
        \begin{proof}
          Given any distinct $x,y\in G-\{1\}$, $xy\neq x$ otherwise by
          cancellation law $y=1$. Similarly, $xy\neq y$, so $xy$ can only
          equal the only element in $G-\{x,y,1\}$. This property fills all
          non-diagonal entries of the group table, and also shows that $G$
          is Abelian. \\

          It remains to fill the diagonal entries of the group table. By
          question 32, each non-identity element has order 2 or 3. We show
          that the order can only be 2. Assume by contradiction that one of
          these elements has order 3. We can assume without loss of
          generality that this element is $a$, and that $a^2=b$. Then
          $ab=a\times a^2 = a^3 = 1$, which contradicts our earlier
          argument that $ab=c$. \\

          Summarizing, we get the following group table: \\
          \begin{center}
            \begin{tabular}{|c||c|c|c|c|}
              \hline
              $G$       & 1 & a & b & c \\
              \hline\hline
              1         & 1 & a & b & c \\
              \hline
              a         & a & 1 & c & b \\
              \hline
              b         & b & c & 1 & a \\
              \hline
              c         & c & b & a & 1 \\
              \hline
            \end{tabular}
          \end{center}
        \end{proof}
    \end{enumerate}

  \item Section 1.2
    \begin{enumerate}
      \item Question 3: Use the generators and relations above to show that
        every element of $D_{2n}$, which is not a power of $r$ has order 2.
        Deduce that $D_{2n}$ is generated by the two elements $s$ and $sr$,
        both of which have order 2.
        \begin{proof}
          Using the relations given, we can reduce any finite product of
          sequences $g\in\{r,s\}^{<\omega}$ into exactly one of the
          following:
          $\{e,r,r^2,\ldots,r^{n-1},s,sr,sr^2,\ldots,sr^{n-1}\}$. Hence the
          only elements of $D_{2n}$ that are not powers of $r$ are of the
          form $sr^k$ for some $k\in\{0,\ldots,n-1\}$. \\
          
          We show by induction on $k$ that all such elements $sr^k$ has
          order 2. The base case is true since $s$ has order 2. For the
          inductive step,
          \begin{align*}
            (sr^{k+1})^2  &= sr^{k+1}sr^{k+1}     && \\
                          &= sr^k(rs)r^{k+1}      && \\
                          &= sr^k(sr^{-1})r^{k+1} && (rs=sr^{-1}) \\
                          &= sr^ksr^k             && \\
                          &= (sr^k)(sr^k)         && \\
                          &= (sr^k)^2             && \\
                          &= e                    && \text{(induction).} \\
          \end{align*}
        \end{proof}
      \item Question 7: Show that $\langle a,b\,|\, a^2=b^2=(ab)^n=1\rangle$
        gives a presentation for $D_{2n}$ in terms of the two generators
        $a=s$ and $b=sr$ of order 2 computed in Exercise 3 above.
        \begin{proof}
          In Exercise 3 above, we have already shown that $a=s$ and $b=sr$
          have order 2. Also, $(ab)^n=(s^2r)^n=(e\cdot r)^n=r^n=1$.
          It remains to show that $a$ and $b$ generates $D_{2n}$.
          Since $D_{2n}$ is generated by $s$ and $r$, it suffices to show
          that $\{a,b\}$ generates both $s$ and $r$. This is true since
          $s=a$ and $r=s\cdot s\cdot r=ab$.
        \end{proof}
    \end{enumerate}

  \item Section 1.3
    \begin{enumerate}
      \item Question 11: Let $\sigma$ be the $m$-cycle $(1 2 \ldots m)$.
        Show that $\sigma^i$ is also an $m$-cycle if and only if $i$ is
        relatively prime to $m$.
        \begin{proof}
          Let $d$ be the greatest common divisor of $i\in\mathbb{Z}$ and $m$.
          Then
          \begin{align*}
            (\sigma^{i})^{m/d}  &= \sigma^{im/d}      && \\
                                &= (\sigma^{i/d})^m   && \text{since $d|i$} \\
                                &= (\sigma^m)^{i/d}   && \\
                                &= e^{i/d}            && \\
                                &= e                  && \\
          \end{align*}
          Hence the order of $\sigma^i$ is less than or equal $m/d$, which
          is less than $m$ if $i$ is not relatively prime to $m$. \\

          To prove the converse, assume $i$ and $m$ are relatively prime.
          Then there are integers $a,b\in\mathbb{Z}$ such that $ia+mb=1$.
          Note that the order of $\sigma^i$ must be smaller or equal $m$
          since $(\sigma^i)^m = (\sigma^m)^i = e^i = e$.
          Assume by contradiction that $\sigma^i$ has an order $k$ smaller
          than $m$. Then
          \begin{align*}
            \sigma^k  &= \sigma^{1\cdot k} \\
                      &= \sigma^{(ia+mb)\cdot k} \\
                      &= \sigma^{(ika+mbk)} \\
                      &= ((\sigma^i)^k)^a \cdot (\sigma^m)^{bk} \\
                      &= e^a \cdot e^{bk} \\
                      &= e, \\
          \end{align*}
          which contradicts the order of $\sigma$.
        \end{proof}
      \item Question 14: Let $p$ be a prime. Show that an element has order
        $p$ in $S_n$ if and only if its cycle decomposition is a product of
        commuting $p$-cycles. Show by an explicit example that this need
        not be the case if $p$ is not prime.
        \begin{proof}
          First, note that any element $\sigma$ in $S_n$ can be decomposed
          into a finite product of cycles $\prod_{i<n}{c_i}$, where the
          cycles $c_i$ are disjoint in the sense that each integer appears
          at most once across the cycles. Because of disjointness, the
          cycles commute, hence given any $k\in\mathbb{Z}^+$,
          \begin{align*}
            \sigma^k  & = \left(\prod_{i<n}{c_i}\right)^k \\
                      & = \prod_{i<n}{c_i^k}. \\
          \end{align*}
          Also because of disjointness of the $c_i$'s and hence of the
          $c_i^k$'s, the power $\sigma^k$ equals $e$ if and only if each
          $c_i^k$ equals $e$, which occurs for the first time when $k$ is
          the lowest common multiple of the orders of each cycle $c_i$.
          Hence, $ord(\sigma) = lcm(|c_0|,\ldots,|c_{n-1}|)$. This order is
          a prime $p$ if and only if each $|c_i|$ is $p$, as we are
          required to show. \\

          If $\sigma = (1,2)(3,4,5)$ is an explicit example of an element
          in $S_n$ that does not have a prime order. This element has order
          $|(1,2)|\cdot |(3,4,5)| = 2\cdot 3 = 6$, which is not prime.
        \end{proof}
      \item Question 15: Prove that the order of an element in $S_n$ equals
        the least common multiple of the lengths of the cycles in its cycle
        decomposition.
        \begin{proof}
          We proved this in the answer to section 1.3 question 14. 
        \end{proof}
      \item Question 18: Find all numbers $n$ such that $S_5$ contains an
        element of order $n$.
        \begin{proof}
          From question 15, we proved that the order of an element equals
          the least common multiple of the lengths of the cycles in its
          cycle decomposition. Hence to find all possible $n$, we consider
          integer solutions to $k_1+\ldots+k_5=5$, restricted to $0\leq
          k_1\leq\ldots\leq k_5\leq 5$. Then $n$ can be the product
          $\prod_{i:k_i>0} k_i$ of any such solution. \\

          From trying all possibilities for $S_5$, we get
          \begin{align*}
            n && =         && 1\cdot 1 \cdot 1 \cdot 1 \cdot 1  & = 1 \\
              && \text{or} && 1\cdot 1 \cdot 1 \cdot 2          & = 2 \\
              && \text{or} && 1\cdot 1 \cdot 3                  & = 3 \\
              && \text{or} && 1\cdot 2 \cdot 2                  & = 4 \\
              && \text{or} && 1\cdot 4                          & = 4 \\
              && \text{or} && 5                                 & = 5 \\
              && \text{or} && 2\cdot 3                          & = 6 \\
          \end{align*}
          Summarizing, the possible values of $n$ are $1,2,\ldots,6$.
        \end{proof}
    \end{enumerate}

  \item Section 1.4
    \begin{enumerate}
      \item Question 7: Let $p$ be a prime. Prove that the order of
        $GL_2(\mathbb{F}_p)$ is $p^4-p^3-p^2+p$.
        \begin{proof}
          The total number of $2\times 2$ matrices over $\mathbb{F}_p$ is
          $p^4$, since $|\mathbb{F}_p|=p$. A $2\times 2$ matrix is not
          invertible if and only if its first row is a multiple of the
          second. Excluding the row with both entries zero, the number of
          unique first rows is $p^2-1$, and for each of these rows, the
          number of unique second rows that is a multiple of the first is
          $(p^2-1)\times p = p^3-p$. Also, the number of matrices with
          first row being zero is $p^2$. These are exactly the matrices
          that are not invertible. Hence, subtracting from the total number
          of matrices, the number of invertible matrices is
          $p^4-p^3-p^2+p$.
        \end{proof}
      \item Question 11: Let
        $H(F) = \left \{
          \begin{bmatrix}
            1 & a & b \\
            0 & 1 & c \\
            0 & 0 & 1 \\
          \end{bmatrix}
        \,\middle|\,a,b,c\in F\right\}$ -- called the \textit{Heisenberg
        group} over $F$. Let
        $X =
          \begin{bmatrix}
            1 & a & b \\
            0 & 1 & c \\
            0 & 0 & 1 \\
          \end{bmatrix} $
        and
        $Y =
          \begin{bmatrix}
            1 & d & e \\
            0 & 1 & f \\
            0 & 0 & 1 \\
          \end{bmatrix} $
        be elements of $H(F)$.
        \begin{enumerate}
          \item Compute the matrix product $XY$ and deduce that $H(F)$ is
            closed under matarix multiplication. Exihibit explicit matrices
            such that $XY\neq YX$.
            \begin{proof}
              $XY = \begin{bmatrix}
                      1 & a & b \\
                      0 & 1 & c \\
                      0 & 0 & 1 \\
                    \end{bmatrix}
                    \begin{bmatrix}
                      1 & d & e \\
                      0 & 1 & f \\
                      0 & 0 & 1 \\
                    \end{bmatrix}
                    =
                    \begin{bmatrix}
                      1 & a+d & b+e+af \\
                      0 & 1   & c+f \\
                      0 & 0   & 1 \\
                    \end{bmatrix} \in H(F)$, hence $H(F)$ is closed under
              matrix multiplication. \\

              Consider the case where $X =
                \begin{bmatrix}
                  1 & 1 & 0 \\
                  0 & 1 & 0 \\
                  0 & 0 & 1 \\
                \end{bmatrix}$
              and $Y =
                \begin{bmatrix}
                  1 & 0 & 0 \\
                  0 & 1 & 1 \\
                  0 & 0 & 1 \\
                \end{bmatrix}$. \\

              Then $XY =
                \begin{bmatrix}
                  1 & 1 & 1 \\
                  0 & 1 & 1 \\
                  0 & 0 & 1 \\
                \end{bmatrix} \neq
                \begin{bmatrix}
                  1 & 1 & 0 \\
                  0 & 1 & 1 \\
                  0 & 0 & 1 \\
                \end{bmatrix} = YX$.
            \end{proof}
          \item Find an explicit formula for the matrix inverse $X^{-1}$
            and deduce that $H(F)$ is closed under inverses.
            \begin{proof}
            \end{proof}
        \end{enumerate}
    \end{enumerate}
\end{enumerate}

\end{document}
