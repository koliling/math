\documentclass{article}
\usepackage[left=3cm,right=3cm,top=3cm,bottom=3cm]{geometry}
\usepackage{amsmath,amssymb,amsthm}
\usepackage{color}
%\setlength{\parindent}{0mm}

\newcommand{\TODO}[1]{\textcolor{red}{TODO: #1}}

\begin{document}
\title{Graduate Algebra I: Homework 2}
\author{Li Ling Ko\\ lko@nd.edu}
\date{\today}
\maketitle

\begin{enumerate}
  \item
    \begin{enumerate}
      \item Prove that the set of all multiples of some fixed integer $n$
        is a subgroup of $\mathbb{Z}$ which is isomorphic to $\mathbb{Z}$.
        \begin{proof}
          Let $n\mathbb{Z}$ denote the set of all multiples of $n$. We
          first show that $n\mathbb{Z}$ is a subgroup of $\mathbb{Z}$.
          Given $ni,nj\in n\mathbb{Z}$, $ni-nj=n(i-j)\in n\mathbb{Z}$,
          which is sufficient to show that $n\mathbb{Z}$ is a subgroup of
          $\mathbb{Z}$. \\
          
          Now we show that the groups are isomorphic. Consider the map
          $\varphi:\mathbb{Z}\rightarrow n\mathbb{Z}$ that sends $i$ to
          $ni$. This map is a homomorphism since
          $\varphi(i+j)=n(i+j)=ni+nj=\varphi(i)+\varphi(j)$. This map is
          also surjective since $\varphi^{-1}(ni)=i$. Finally, this map is
          injective because its kernel contains only identity:
          $\varphi(i)=0\leftrightarrow ni=0\leftrightarrow i=0$.
        \end{proof}
      \item Page 22, question 27: Prove that if $x$ is an element of the
        group $G$ then $H=\{x^n|n\in\mathbb{Z}\}$ is a subgroup of $G$.

        \begin{proof}
          Given $x^m,x^n\in H$, $x^m\cdot(x^n)^{-1}=x^m\cdot
          x^{-n}=x^{m-n}\in H$. This is a sufficient condition for showing
          that $H$ is a subgroup of $G$.
        \end{proof}
    \end{enumerate}

  \item Give an example of an infinite group which as a finite non-trivial
    subgroup.

    \begin{proof}
      Consider the group $(\mathbb{Z}_2[x],+)$. This group is infinite
      since $\{0,x,x^2,x^3,\ldots\}$ are distinct elements in the group.
      Then $\mathbb{Z}_2=\{0,1\}$ is a non-trivial finite subgroup.
    \end{proof}

  \item What is the largest possible order of an element in $S_6$? What
    about in $S_7$? Explain your answers. 

    \begin{proof}
      Each permutation $\varphi\in S_n$ can be decomposed into a
      composition of disjoint cycles
      $\varphi=\sigma_1\circ\ldots\circ\sigma_k$ such that each element
      $i\in\{1,\ldots,n\}$ appears exactly once. We have proven in homework
      1 that the order of $\varphi$ is then
      $\mathrm{lcm}(|\sigma_1|\cdot\ldots\cdot|\sigma_k|)$. Hence, to find
      an element of the largest possible order, we consider all finite
      positive integer sums to $n$, and the lowest common multiples of each
      solution. \\

      For $n=6$, we find that the permutation $\varphi$ with the largest
      order occurs when $\varphi$ is a 6-cycle, or a composition of a 3-cycle
      with a disjoint 2-cycle, giving an order of 6. \\

      For $n=7$, we find that the permutation $\varphi$ with the largest
      order occurs when $\varphi$ is a composition of a 3-cycle
      with a disjoint 4-cycle, giving an order of 12. \\
    \end{proof}

  \item Let $Q_8$ act on itself by left multiplication. Use this action to
    find an embedding of $Q_8$ in $S_8$ using the numbering:
    $1\mapsto1,i\mapsto2,-1\mapsto3,-i\mapsto4,j\mapsto5,k\mapsto6,-j\mapsto7,-k\mapsto8$.
    That is, give the left regular representation of $Q_8$ in $S_8$.

    \begin{proof}
      Following the Cayley table of the $Q_8$ which we have obtained in the
      previous homework, we get the left regular representation of $Q_8$ in
      $S_8$ as follows:

      \begin{center}
        \begin{tabular}{|l||l|}
          \hline
          $1$   & $()$ \\ \hline
          $i$   & $(1,i,-1,-1)\circ(j,k,-j,-k)$ \\ \hline
          $-1$  & $(1,-1)\circ(i,-i)\circ(j,-j)\circ(k,-k)$ \\ \hline
          $-i$  & $(1,-i,-1,i)\circ(j,-k,-j,k)$ \\ \hline
          $j$   & $(1,j,-1,-j)\circ(i,-k,-i,k)$ \\ \hline
          $k$   & $(1,k,-1,-k)\circ(i,j,-i,-j)$ \\ \hline
          $-j$  & $(1,-j,-1,j)\circ(i,k,-i,-k)$ \\ \hline
          $-k$  & $(1,-k,-1,k)\circ(i,-j,-i,j)$ \\ \hline
        \end{tabular}
      \end{center}

      Replacing each element in $Q_8$ by its numbering, we get the
      following table:

      \begin{center}
        \begin{tabular}{|l||l|}
          \hline
          $1$   & $()$ \\ \hline
          $2$   & $(1,2,3,3)\circ(5,6,7,8)$ \\ \hline
          $3$   & $(1,3)\circ(2,4)\circ(5,7)\circ(6,8)$ \\ \hline
          $4$   & $(1,4,3,2)\circ(5,8,7,6)$ \\ \hline
          $5$   & $(1,5,3,7)\circ(2,8,4,6)$ \\ \hline
          $6$   & $(1,6,3,8)\circ(2,5,4,7)$ \\ \hline
          $7$   & $(1,7,3,5)\circ(2,6,4,8)$ \\ \hline
          $8$   & $(1,8,3,6)\circ(2,7,4,5)$ \\ \hline
        \end{tabular}
      \end{center}
    \end{proof}

  \item Section 1.6
    \begin{enumerate}
      \item Question 4: Prove that the multiplicative groups
        $\mathbb{R}-\{0\}$ and $\mathbb{C}-\{0\}$ are not isomorphic.

        \begin{proof}
          Isomorphism preserves order of elements. $\mathbb{C}-\{0\}$ has
          an element of order 4, given by $i$, but $\mathbb{R}-\{0\}$ only
          has elements of order 0 or 1. Hence the groups cannot be
          isomorphic.
        \end{proof}

      \item Question 5: Prove that the additive groups $\mathbb{R}$ and
        $\mathbb{Q}$ are not isomorphic.

        \begin{proof}
          An isomorphism between groups must be bijective. By Cantor's
          diagonal argument, since $\mathbb{R}$ is uncountable while
          $\mathbb{Q}$ is countable, there can never be a bijection between
          them. Hence there no isomorphism between the groups exist. 
        \end{proof}

      \item Question 6: Prove that the additive groups $\mathbb{Z}$ and
        $\mathbb{Q}$ are not isomorphic.

        \begin{proof}
          $\mathbb{Z}$ has a generator $1$ but $\mathbb{Q}$ does not have a
          generator, so the two groups cannot be isomorphic. More
          specifically, assume that
          $\varphi:\mathbb{Z}\rightarrow\mathbb{Q}$ is an isomorphism, and
          let $q=\varphi(1)$, and $n=\varphi^{-1}(q/2)$. Since isomorphisms
          preserve identities, $q\neq0$, which means $n>1$. Then
          $1=\varphi^{-1}(q)=\varphi^{-1}(q/2)+\varphi^{-1}(q/2)=n+n=2n>1$,
          a contradiction.
        \end{proof}

      \item Question 9: Prove that $D_{24}$ and $S_4$ are not isomorphic.
        \begin{proof}
          From homework 1, we know that $D_{24}$ is generated by an element
          $s$ of order 2 and an element $r$ of order 12. Since isomorphisms
          preserve orders, it suffices to show that $S_4$ does not contain
          an element of order 12. \\

          From homework 1, we showed that each permutation in $S_4$ can be
          decomposed into disjoint cycles of orders less than or equal 4,
          and each element appears exactly once across the cycles. The
          order of the permutation would then be the lowest common multiple
          of the orders of the cycles. Hence, the largest possible order of
          a permutation in $S_4$ is 4, which occurs when the permutation is
          a single 4-cycle or a composition of two disjoint 2-cycles. So
          $S_4$ does not have an element of order 12, implying it cannot be
          isomorphic to $D_{24}$.
        \end{proof}

      \item Question 13: Let $G$ and $H$ be groups and let
        $\varphi:G\rightarrow H$ be a homomorphism. Prove that the image of
        $\varphi$, $\varphi(G)$, is a subgroup of $H$. Prove that if
        $\varphi$ is injective then $G\cong\varphi(G)$. 

        \begin{proof}
          Let $h_1,h_2\in\varphi(G)$. Then there exists $g_1,g_2\in G$ such
          that $h_1=\varphi(g_1)$ and $h_2=\varphi(g_2)$. We wish to show
          that $h_1\circ h_2^{-1}\in \varphi(G)$. Now 

          \begin{align*}
            \varphi(g_1g_2^{-1})      & = \varphi(g_1)\circ\varphi(g_2^{-1})  & (\because\varphi\,\text{is a homomorphism}) \\
              & = \varphi(g_1)\circ\varphi(g_2)^{-1} & (\because\text{homomorphisms preserve inverses}) \\
              & = h_1\circ h_2^{-1},  & (\because h_1=\varphi(g_1)\,\text{and}\,h_2=\varphi(g_2)) \\
          \end{align*}
          which implies $h_1\circ h_2^{-1}\in \varphi(G)$, as we wish to
          show. \\

          Assume that $\varphi$ is injective.
          $\varphi\upharpoonright\varphi(G)$ is clearly surjective, so
          $\varphi$ is a bijective homomorphism. It suffices to show that
          all bijective homomorphisms are isomorphisms, for which we just
          need to show that the inverse map $\varphi^{-1}$ is a
          homomorphism. Let $h_1,h_2\in\varphi(G)$. Then there exists
          $g_1,g_2\in G$ such that $h_1=\varphi(g_1)$ and
          $h_2=\varphi(g_2)$. By homomorphism of $\varphi$,
          $\varphi(g_1g_2)=h_1\circ h_2$. Then by bijectiveness of
          $\varphi$, $\varphi^{-1}(h_1\circ
          h_2)=g_1g_2=\varphi(h_1)\varphi(h_2)$, which proves the
          homomorphism of $\varphi^{-1}$. 
        \end{proof}

      \item Question 14: Let $G$ and $H$ be groups and let
        $\varphi:G\rightarrow H$ be a homomorphism. Define the
        \textit{kernel} of $\varphi$ to be $\{g\in G\,|\,\varphi(g)=1_H\}$.
        Prove that the kernel of $\varphi$ is a subgroup of $G$. Prove that
        $\varphi$ is injective if and only if the kernel of $\varphi$ is
        the identity subgroup of $G$.

        \begin{proof}
          We first prove that $\ker(\varphi)$ is a subgroup of $G$. Given
          $g_1,g_2\in\ker(\varphi)$, it suffices to show that $g_1g_2^{-1}$
          is in $\ker(\varphi)$. Now
          \begin{align*}
            \varphi(g_1g_2^{-1})  & = \varphi(g_1)\circ\varphi(g_2^{-1})  &
            (\because\varphi\,\text{is a homomorphism}) \\
              & = 1_H\circ\varphi(g_2)^{-1} & (\because
              g_1\in\ker(\varphi)\,\text{and homomorphisms preserve inverses}) \\
              & = 1_H\circ(1_H)^{-1}        & (\because g_2\in\ker(\varphi)) \\
              & = 1_H\circ1_H \\
              & = 1_H, \\
          \end{align*}
          which means that $g_1,g_2\in\ker(\varphi)$, as we wish to show.
          \\

          If $\ker(\varphi)$ contains one non-identity element $g$, then
          $\varphi(g)=\varphi(1_G)=1_H$, so $\varphi$ cannot be injective.
          To show the converse, assume that $\varphi$ is not injective.
          Then there exists $g_1\neq g_2\in G$ such that
          $\varphi(g_1)=\varphi(g_2)$. Then
          \begin{align*}
            \varphi(g_1g_2^{-1})  & = \varphi(g_1)\circ\varphi(g_2^{-1})  &
            (\because\varphi\,\text{is a homomorphism}) \\
              & = \varphi(g_1)\circ\varphi(g_2)^{-1} & (\because
              g_1\in\ker(\varphi)\,\text{and homomorphisms preserve inverses}) \\
              & = \varphi(g_1)\circ\varphi(g_1)^{-1} &
              (\because\varphi(g_1)=\varphi(g_2)) \\
              & = 1_H, \\
          \end{align*}
          which implies that $g_1g_2^{-1}\in\ker(\varphi)$. Now
          $g_1g_2^{-1}\neq1_G$ otherwise $g_1=g_2$, so we $\ker(\varphi)$
          contains a non-identity element $g_1g_2^{-1}$. 
        \end{proof}

      \item Question 16: Let $A$ and $B$ be groups and let $G$ be their
        direct product, $A\times B$. Prove that the maps
        $\pi_1:G\rightarrow A$ and $\pi_2:G\rightarrow B$ defined by
        $\pi_1((a,b))=a$ and $\pi_2((a,b))=b$ are homomorphisms and find
        their kernels.

        \begin{proof}
          We first show that $\pi_1$ is a homomorphism by chasing
          definitions. Given $c_1=(a_1,b_1),c_2=(a_2,b_2)\in A\times B$,
          \begin{align*}
            \pi_1(c_1c_2) & = \pi_1((a_1,b_1)(a_2,b_2))  & \\
                          & = \pi_1((a_1a_2,b_1b_2))     & \\
                          & = a_1a_2  & (\text{by definition of $\pi_1$}) \\
                          & = \pi_1((a_1,b_1))\pi_1((a_2,b_2)), & (\text{by definition of $\pi_1$}) \\
          \end{align*}
          which proves the homomorphism of $\pi_1$. The homomorphism of
          $\pi_2$ can be proven in a similar manner. \\

          To find $\ker(\pi_1)$, we wish to find $(a,b)\in A\times B$ such
          that $pi_1((a,b))=a=1_A$. Clearly, we get
          $\ker(\pi_1)=\{1_A\}\times B$. Similarly for $\pi_2$, we get
          $\ker(\pi_2)=A\times\{1_B\}$.
        \end{proof}
    \end{enumerate}

  \item Section 1.7
    \begin{enumerate}
      \item Question 4: Let $G$ be a group acting on a set $A$ and fix some
        $a\in A$. Show that the following sets are subgroups of $G$:
        \begin{enumerate}
          \item the kernel of the action
          \item $\{g\in G\,|\,ga=a\}$ --- this subgroup is called the
            \textit{stabilizer} of $a$ in $G$.
        \end{enumerate}

        \begin{proof}
          In section 1.7 of Dummit and Foote, the authors showed that for
          each $g\in G$, $\sigma_g$ is a permutation of $A$, and the map
          from $G$ to $S_A$ defined by $g\mapsto\sigma_g$ is a
          homomorphism. We will use these findings in our proof. \\

          We first show that the kernel of the action is a subgroup of
          $G$. Now the kernel of the action is exacty the kernel of the
          homomorphism from $G$ to $S_A$ that sends $g$ to $\sigma_g$. From
          section 1.6 question 14, we have already shown that the kernel of
          a homomorphism is a subgroup of the domain group. Hence the
          kernel of the action is a subgrou pof $G$. \\

          Now we show that the stabilizer of $a$ in $G$ is a subgroup of
          $G$. Let $g_1,g_2\in G$ be in the stabilizer. It suffices to show
          that $g_1g_2^{-1}$ is also in the stabilizer. Now
          \begin{align*}
            g_2^{-1}\cdot a & = \sigma_{g_2^{-1}}(a) & \\
                            & = \sigma_{g_2}^{-1}(a) & (\because\,\text{the
            map from $G$ to $S_ A$ is a homomorphism}) \\
                            & = a, & (\because\,\sigma_{g_2}(a)=a)
          \end{align*}
          implying that $g_2^{-1}$ is also in the stabilizer. Then
          \begin{align*}
            (g_1g_2^{-1})\cdot a  & = g_1\cdot(g_2^{-1}\cdot a)  &
              (\text{by group action rules}) \\
                                  & = g_1\cdot(a) & (\because
                                  g_2^{-1}\,\text{is in the stabilizer}) \\
                                  & = a           & (\because a\,\text{is
                                  in the stabilizer}) \\
          \end{align*}
          Hence $g_1g_2^{-1}$ is also in the stabilizer, as we are required
          to show. \\
        \end{proof}

      \item Question 6: Prove that a group $G$ acts faithfully on a set $A$
        if and only if the kernel of the action is the set consisting only
        of the identity.

        \begin{proof}
          Dummit and Foote showed that for each $g\in G$, $\sigma_g$ is a
          permutation of $A$, and the map $f$ from $G$ to $S_A$ defined by
          $g\mapsto\sigma_g$ is a homomorphism. Hence a group $G$ acts
          faithfully if and only if homomorphism $f$ is injective. We have
          proven earlier that a homomorphism is injective if and only
          if its kernel contains only the identiy. Hence $G$ acts
          faithfully if and only if $ker(f)$ contains only $1_G$, which
          occurs if and only if the action of kernel of the action is the
          set consisting only the identity.
        \end{proof}

      \item Question 16: Let $G$ be any group and let $A=G$. Show that the
        maps defined by $g\cdot a=gag^{-1}$ for all $g,a\in G$ \textit{do}
        satisfy the axioms of a (left) group action.

        \begin{proof}
          First, $1_G\cdot a=1_Ga(1_G)^{-1}=1_Ga1_G=a$. Second, given
          $g_1,g_2\in G$, we have
          \begin{align*}
            (g_1g_2)\cdot a & = (g_1g_2)a(g_1g_2)^{-1}      & \\
                            & = (g_1g_2)a(g_2^{-1}g_1^{-1}) & \\
                            & = g_1(g_2ag_2^{-1})g_1^{-1}   & (\text{by
                            associativity of $G$}) \\
                            & = g_1(g_2\cdot a)g_1^{-1}     & \\
                            & = g_1\cdot(g_2\cdot a)        & \\
          \end{align*}
          These are the two conditions sufficient to show that the map is a
          left group action. 
        \end{proof}

      \item Question 17:
        \begin{proof}
        \end{proof}

      \item Question 18:
        \begin{proof}
        \end{proof}

      \item Question 19:
        \begin{proof}
        \end{proof}
    \end{enumerate}

  \item Section 2.1
    \begin{enumerate}
      \item Question 4:
        \begin{proof}
        \end{proof}

      \item Question 5:
        \begin{proof}
        \end{proof}

      \item Question 6:
        \begin{proof}
        \end{proof}

      \item Question 7:
        \begin{proof}
        \end{proof}

      \item Question 14:
        \begin{proof}
        \end{proof}
    \end{enumerate}
\end{enumerate}

\end{document}
