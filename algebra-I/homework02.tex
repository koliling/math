\documentclass{article}
\usepackage[left=3cm,right=3cm,top=3cm,bottom=3cm]{geometry}
\usepackage{amsmath,amssymb,amsthm}
\usepackage{color}
%\setlength{\parindent}{0mm}

\newcommand{\TODO}[1]{\textcolor{red}{TODO: #1}}

\begin{document}
\title{Graduate Algebra I: Homework 2}
\author{Li Ling Ko\\ lko@nd.edu}
\date{\today}
\maketitle

\begin{enumerate}
  \item
    \begin{enumerate}
      \item Prove that the set of all multiples of some fixed integer $n$
        is a subgroup of $\mathbb{Z}$ which is isomorphic to $\mathbb{Z}$.
        \begin{proof}
          Let $n\mathbb{Z}$ denote the set of all multiples of $n$. We
          first show that $n\mathbb{Z}$ is a subgroup of $\mathbb{Z}$.
          Given $ni,nj\in n\mathbb{Z}$, $ni-nj=n(i-j)\in n\mathbb{Z}$,
          which is sufficient to show that $n\mathbb{Z}$ is a subgroup of
          $\mathbb{Z}$. \\
          
          Now we show that the groups are isomorphic. Consider the map
          $\varphi:\mathbb{Z}\rightarrow n\mathbb{Z}$ that sends $i$ to
          $ni$. This map is a homomorphism since
          $\varphi(i+j)=n(i+j)=ni+nj=\varphi(i)+\varphi(j)$. This map is
          also surjective since $\varphi^{-1}(ni)=i$. Finally, this map is
          injective because its kernel contains only identity:
          $\varphi(i)=0\leftrightarrow ni=0\leftrightarrow i=0$.
        \end{proof}
      \item Page 22, question 27: Prove that if $x$ is an element of the
        group $G$ then $H=\{x^n|n\in\mathbb{Z}\}$ is a subgroup of $G$.

        \begin{proof}
          Given $x^m,x^n\in H$, $x^m\cdot(x^n)^{-1}=x^m\cdot
          x^{-n}=x^{m-n}\in H$. This is a sufficient condition for showing
          that $H$ is a subgroup of $G$.
        \end{proof}
    \end{enumerate}

  \item Give an example of an infinite group which as a finite non-trivial
    subgroup.

    \begin{proof}
      Consider the group $(\mathbb{Z}_2[x],+)$. This group is infinite
      since $\{0,x,x^2,x^3,\ldots\}$ are distinct elements in the group.
      Then $\mathbb{Z}_2=\{0,1\}$ is a non-trivial finite subgroup.
    \end{proof}

  \item What is the largest possible order of an element in $S_6$? What
    about in $S_7$? Explain your answers. 

    \begin{proof}
      Each permutation $\varphi\in S_n$ can be decomposed into a
      composition of disjoint cycles
      $\varphi=\sigma_1\circ\ldots\circ\sigma_k$ such that each element
      $i\in\{1,\ldots,n\}$ appears exactly once. We have proven in homework
      1 that the order of $\varphi$ is then
      $\mathrm{lcm}(|\sigma_1|\cdot\ldots\cdot|\sigma_k|)$. Hence, to find
      an element of the largest possible order, we consider all finite
      positive integer sums to $n$, and the lowest common multiples of each
      solution. \\

      For $n=6$, we find that the permutation $\varphi$ with the largest
      order occurs when $\varphi$ is a 6-cycle, or a composition of a 3-cycle
      with a disjoint 2-cycle, giving an order of 6. \\

      For $n=7$, we find that the permutation $\varphi$ with the largest
      order occurs when $\varphi$ is a composition of a 3-cycle
      with a disjoint 4-cycle, giving an order of 12. \\
    \end{proof}

  \item Let $Q_8$ act on itself by left multiplication. Use this action to
    find an embedding of $Q_8$ in $S_8$ using the numbering:
    $1\mapsto1,i\mapsto2,-1\mapsto3,-i\mapsto4,j\mapsto5,k\mapsto6,-j\mapsto7,-k\mapsto8$.
    That is, give the left regular representation of $Q_8$ in $S_8$.

    \begin{proof}
      Following the Cayley table of the $Q_8$ which we have obtained in the
      previous homework, we get the left regular representation of $Q_8$ in
      $S_8$ as follows:

      \begin{center}
        \begin{tabular}{|l||l|}
          \hline
          $1$   & $()$ \\ \hline
          $i$   & $(1,i,-1,-1)\circ(j,k,-j,-k)$ \\ \hline
          $-1$  & $(1,-1)\circ(i,-i)\circ(j,-j)\circ(k,-k)$ \\ \hline
          $-i$  & $(1,-i,-1,i)\circ(j,-k,-j,k)$ \\ \hline
          $j$   & $(1,j,-1,-j)\circ(i,-k,-i,k)$ \\ \hline
          $k$   & $(1,k,-1,-k)\circ(i,j,-i,-j)$ \\ \hline
          $-j$  & $(1,-j,-1,j)\circ(i,k,-i,-k)$ \\ \hline
          $-k$  & $(1,-k,-1,k)\circ(i,-j,-i,j)$ \\ \hline
        \end{tabular}
      \end{center}

      Replacing each element in $Q_8$ by its numbering, we get the
      following table:

      \begin{center}
        \begin{tabular}{|l||l|}
          \hline
          $1$   & $()$ \\ \hline
          $2$   & $(1,2,3,3)\circ(5,6,7,8)$ \\ \hline
          $3$   & $(1,3)\circ(2,4)\circ(5,7)\circ(6,8)$ \\ \hline
          $4$   & $(1,4,3,2)\circ(5,8,7,6)$ \\ \hline
          $5$   & $(1,5,3,7)\circ(2,8,4,6)$ \\ \hline
          $6$   & $(1,6,3,8)\circ(2,5,4,7)$ \\ \hline
          $7$   & $(1,7,3,5)\circ(2,6,4,8)$ \\ \hline
          $8$   & $(1,8,3,6)\circ(2,7,4,5)$ \\ \hline
        \end{tabular}
      \end{center}
    \end{proof}

  \item Section 1.6
    \begin{enumerate}
      \item Question 4:
        \begin{proof}
        \end{proof}

      \item Question 5:
        \begin{proof}
        \end{proof}

      \item Question 6:
        \begin{proof}
        \end{proof}

      \item Question 9:
        \begin{proof}
        \end{proof}

      \item Question 13:
        \begin{proof}
        \end{proof}

      \item Question 14:
        \begin{proof}
        \end{proof}

      \item Question 16:
        \begin{proof}
        \end{proof}
    \end{enumerate}

  \item Section 1.7
    \begin{enumerate}
      \item Question 4:
        \begin{proof}
        \end{proof}

      \item Question 6:
        \begin{proof}
        \end{proof}

      \item Question 16:
        \begin{proof}
        \end{proof}

      \item Question 17:
        \begin{proof}
        \end{proof}

      \item Question 18:
        \begin{proof}
        \end{proof}

      \item Question 19:
        \begin{proof}
        \end{proof}
    \end{enumerate}

  \item Section 2.1
    \begin{enumerate}
      \item Question 4:
        \begin{proof}
        \end{proof}

      \item Question 5:
        \begin{proof}
        \end{proof}

      \item Question 6:
        \begin{proof}
        \end{proof}

      \item Question 7:
        \begin{proof}
        \end{proof}

      \item Question 14:
        \begin{proof}
        \end{proof}
    \end{enumerate}
\end{enumerate}

\end{document}
