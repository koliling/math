

\documentclass{article}
\usepackage[left=3cm,right=3cm,top=3cm,bottom=3cm]{geometry}
\usepackage{amsmath,amssymb,amsthm}
\usepackage{color}
%\setlength{\parindent}{0mm}

\newcommand{\TODO}[1]{\textcolor{red}{TODO: #1}}

\begin{document}
\title{Graduate Algebra I: Homework 3}
\author{Li Ling Ko\\ lko@nd.edu}
\date{\today}
\maketitle

\begin{enumerate}
  \item
    \begin{enumerate}
      \item Consider $A=\{(1,2,3,4)\}\subset S_4$. Compute the centralizer
        and the normalizer of $A$ in $S_4$.

        \begin{proof}
          Since $A$ has a single element, by chasing definitions, its
          normalizer is the same as its centralizer.
          The elements of $S_4$ are either the identity, a 2-cycle, a
          3-cycle, two disjoint 2-cycles, or a 4-cycle. We check which of
          these elements $(1,2,3,4)$ commutes with. We observe that if a
          given element $g$ commutes with another element $h$, then $g$
          would commute with the subgroup generated by $h$, since $gh=hg$
          implies from induction on $r$ that $gh^r=h^rg$. Hence we do not
          need to check for commutativity with an element in the cyclic
          subgroup of another element which was found to commute with
          $(1,2,3,4)$. We also use symmetrical arguments to decrease the
          number of times we need to check for commutativity. \\

          Clearly the identity commutes with $(1,2,3,4)$. As for the
          2-cycles, we check that $(1,2)$ does not commute with
          $(1,2,3,4)$, so by symmetry, neither should $(2,3)$, $(3,4)$, or
          $(4,1)$. Also, we check that $(1,3)$ does not commute with
          $(1,2,3,4)$, so by symmetry, neither should $(2,4)$. Next, we
          check for commutativity with the 3-cycles. $(1,2,3,4)$ does not
          commute with $(1,2,3)$, and since $(1,2,3)$ is contained in the
          group generated by $(1,3,2)$, $(1,2,3,4)$ would not commute with
          $(1,3,2)$ either. Also by symmetry, $(1,2,3,4)$ should not
          commute with $(2,3,4)$ if it did not commute with $(1,2,3)$,
          which means $(1,2,3,4)$ would not commute with $(2,4,3)$.
          Finally, $(1,2,3,4)$ does not commute with $(1,2,4)$, so it will
          not commute with $(1,4,2)$. So $(1,2,3,4)$ does not commute with
          any 2-cycles or 3-cycles. \\

          Next, we check commutativity with the 4-cycles. Clearly
          $(1,2,3,4)$ commutes with itself and hence with any 4-cycle in
          its cyclic subgroup. We check that $(1,2,3,4)$ does not commute
          with $(1,2,4,3)$, $(1,3,4,2)$, $(1,3,2,4)$, $(1,4,2,3)$, or
          $(1,4,3,2)$. Any other 4-cycle not in $\langle A\rangle$
          generates a cyclic subgroup that contains one of these
          non-commutating elements, so the only 4-cycles that commute with
          $(1,2,3,4)$ are those in its cyclic subgroup. Finally, we check
          for commutativity with the disjoint 2-cycles. We check that
          $(1,2,3,4)$ does not commute with $(1,2)(3,4)$, $(1,3)(2,4)$,
          or $(1,4)(2,3)$, which are all the possible disjoint 2-cycles. We
          conclude that the centralizer of $A$, which is also its
          normalizer, is the its cyclic subgroup, which is
          $\langle(1,2,3,4)\rangle$. \\
        \end{proof}

      \item Consider $H=\langle(1,2,3,4)\rangle\subset S_4$. Compute the
        centralizer and the normalizer of $H$ in $S_4$.

        \begin{proof}
          We first find the centralizer. By the definition of centralizer,
          we have $C_G(\langle g\rangle)\subseteq C_G(\{g\})$ for any
          element $g\in G$. Also, if $h$ commutes with $g$, then $h$ will
          also commute with $g^r$, since $gh^r=h^rg$ by induction on $r$.
          Hence $C_G(\langle g\rangle)\supseteq C_G(\{g\})$, which implies
          that $C_G(\langle g\rangle)=C_G(\{g\})$. Therefore, the
          centralizer of $H$ is the centralizer of $A$, which we have
          computed earlier to equal to $H$ itself. \\

          Now we find the normalizer of $H$. By the definition of a
          normalizer, we have $N_G(\langle g\rangle)\supseteq N_G(\{g\})$
          for any element $g\in G$, hence $N(H)$ must contain at least
          $N(A)$, which we have found to be the cyclic group generated by
          $(1,2,3,4)$. We check which other elements of $S_4$ should be
          contained in $N(H)$. Let $a$ denote $(1,2,3,4)$. Then
          $gH=\{g,ga,ga^2,ga^3\}$, and $Hg=\{g,ag,a^2g,a^3g\}$. For $gH$
          to equal $Hg$, we need to find a surjection between the two
          sets, so $ga$ should be mapped to either $g$, $ag$, $a^2g$, or
          $a^3g$. If $ga$ is mapped to $g$, we get $a=1$, a
          contradiction. If $ga$ is mapped to $ag$, then $g$ would be in
          $C(H)$, which we have computed earlier to equal $H$. If $ga$ is
          mapped to $a^2g$, then $ga^2$ will be mapped to $a^2ga=a^4g=g$,
          which is a contradiction since the inverse image of $g$ should
          only be $g$. Finally, if $ga$ is mapped to $a^3g$, then we will
          get a bijection between $gH$ and $Hg$ which maps $g$ to $g$,
          $ga$ to $a^3g$, $ga^2$ to $a^2g$, and $ga^3$ to $ag$. Hence it
          remains to find all $g\in G$ such that $gag^{-1}$ equals $a^3$.
          \\

          Once we find an element $g\in G$ satisfying $gag^{-1}=a^3$, then
          we know all elements in the group generated by $a$ and $g$ must
          also be in the normalizer, since normalizers are subgroups. This
          reduces the number of checks we need to perform. \\

          We first check which 2-cycles will satisfy the criteria
          $gag^{-1}=a^3=(4,1,2,3)$. We find that $(1,3)$ satisfies the
          criteria. Therefore, from the above observation, we know that
          $\langle(1,3),(1,2,3,4)\rangle$ is in $N(H)$. We check that
          $\langle(1,3),(1,2,3,4)\rangle$ has eight distinct elements, each
          with order either 1, 2, or 4. Since the order of subgroups must
          divide the order of the group, the order of $N(H)$ must be a
          multiple of 8 and also a divisor of $|S_4|=24$, which means the
          order of $N(H)$ can only be 8 or 24, further implying that $N(H)$
          can only be $\langle(1,3),(1,2,3,4)\rangle$ or the whole of
          $S_4$. We check that $(1,2)$ does not satisfy
          $gag^{-1}=a^3=(4,1,2,3)$ and hence cannot be in $N(H)$, and so
          $N(H)=\langle(1,3),(1,2,3,4)\rangle$.
        \end{proof}
    \end{enumerate}

  \item Again consider $H=\langle(1,2,3,4)\rangle\leq S_4$.
    \begin{enumerate}
      \item Calculate the left coset $(1,2)H$ in $S_4$.
        \begin{proof}
          The table below summarizes the calculations.
          \begin{center}
            \begin{tabular}{|l|l|}
              \hline
              $h\in H$      & $(1,2)h$    \\ \hline\hline
              $e$           & $(1,2)$     \\ \hline
              $(1,2,3,4)$   & $(2,3,4)$   \\ \hline
              $(1,2,3,4)^2$ & $(1,3,2,4)$ \\ \hline
              $(1,2,3,4)^3$ & $(4,3,1)$   \\ \hline
            \end{tabular}
          \end{center}
        \end{proof}

      \item Calculate the right coset $H(1,2)$ in $S_4$.
        \begin{proof}
          The table below summarizes the calculations.
          \begin{center}
            \begin{tabular}{|l|l|}
              \hline
              $h\in H$      & $h(1,2)$    \\ \hline\hline
              $e$           & $(1,2)$     \\ \hline
              $(1,2,3,4)$   & $(1,3,4)$   \\ \hline
              $(1,2,3,4)^2$ & $(1,4,2,3)$ \\ \hline
              $(1,2,3,4)^3$ & $(2,4,3)$   \\ \hline
            \end{tabular}
          \end{center}
        \end{proof}

      \item Is $H$ normal in $G$?
        \begin{proof}
          $H$ is not normal in $G$. A subgroup is normal in a group if and
          only if the normalizer of the subgroup with respect to the group
          is the whole group. We have shown in question 1b that the
          normalizer of $H$ is not the whole group $S_4$, so $H$ cannot be
          normal in $S_4$.
        \end{proof}
    \end{enumerate}

  \item Let $H\subset S_4$ consist of all those permutations $\sigma$ such
    that $\sigma(4)=4$.

    \begin{enumerate}
      \item Prove that $H$ is a subgroup of $S_4$, and show that it is
        isomorphic to $S_3$.
        \begin{proof}
          Every permutation that fixes the last element is a permutation of
          the first three elements, and every permutation of only three
          elements can be considered a permutation of four elements but
          with the last element fixed. Hence the identity map from $S_3$
          to $S_4$ is a natural embedding with image $H$. This map is a
          homomorphism, so the image $H$ must be a subgroup of $S_4$ since
          the image of homomorphisms are subgroups of the codomain. Hence,
          $S_3$ is isomorphic to $H$.
        \end{proof}

      \item Is $H$ normal in $S_4$?
        \begin{proof}
          No. Consider the element $g=(1,4)(1,2)(1,4)^{-1}$. Since
          $(1,2)$ is contained in $H$, $g$ should also be contained in $H$
          if $H$ is normal. However, $g=(2,4)\not\in H$.
        \end{proof}
    \end{enumerate}

  \item Prove that if $H$ is a subgroup of $G$ of index 2, then $H$ is
    normal in $G$.
    \begin{proof}
      Let $g\in G\setminus H$, and consider the left coset $gH$ in $G$.
      Since left actions partition $G$ into equivalence classes of equal
      sizes, and the size of $gH$ must be the size of $eH=H$, which is half
      the size of $G$. Also, by the equivalence relation, $gH$ and $eH=H$
      must be disjoint since $g$ is not contained in $H$, which implies
      that $gH=G\setminus H$. Similarly for the right coset $Hg$ in $G$, we
      get $Hg=G\setminus H$. So $gH=Hg$ for any $g\in G\setminus H$. Also,
      for $g\in H$, clearly we have $gH=Hg$, hence $gH=Hg$ for all $g\in
      G$, which means $H$ is a normal subgroup of $G$.
    \end{proof}

  \item Section 2.2
    \begin{enumerate}
      \item Question 9: For any subgroup $H$ of $G$ and any nonempty subset
        $A$ of $G$ define $N_H(A)$ to be the set $\{h\in H\,|\;
        hAh^{-1}=A\}$. Show that $N_H(A)=N_G(A)\cap H$ and deduce that
        $N_H(A)$ is a subgroup of $H$.

        \begin{proof}
          The equivalence relation is true by chasing definitions: If $h\in
          N_H(A)$, then it must be contained in $H$, and also contained in
          the normalizer of $A$ with respect to $G$; if $h$ is contained in
          $H$ and is also contained in the normalizer of $A$ with respect
          to $G$, then $h$ must be in $N_H(A)$. \\

          We first show that the intersection of subgroups $H_1$ and $H_2$
          is a subgroup of the both subgroups: $H_1\cap H_2$ is non-empty
          since it contains $e$.  Also, if $a,b\in H_1\cap H_2$, then
          $ab^{-1}\in H_1,H_2$ since $H_1$ and $H_2$ are subgroups,
          implying that $ab^{-1}\in H_1\cap H_2$, which concludes our proof
          that $H_1\cap H_2$ is a subgroup of $H_1$ and $H_2$. So since
          both $H$ and $N_G(A)$ are subgroups of $G$, their intersection
          must be a subgroup of $H$, as we are required to show.
        \end{proof}

      \item Question 10: Let $H$ be a subgroup of order 2 in $G$. Show that
        $N_G(H)=C_G(H)$. Deduce that if $N_G(H)=G$ then $H\leq Z(G)$.

        \begin{proof}
          By definition of normalizers and centralizers, normalizers
          should contain the centralizers. We prove the opposite inclusion
          by chasing definitions. A subgroup $H$ of order 2 must be
          composed of an element $h$ of order 2 and the identity $e$. Hence
          $n\in N_G(H)$ is equivalent to
          $\{n,nh\}=n\{e,h\}=\{e,h\}n=\{n,hn\}$, which is equivalent to
          $nh=hn$, which is equivalent to $n$ being contained in $N_C(H)$.
          Hence $N_G(H)$ is also a subset of $N_C(H)$, so both groups are
          equivalent. \\

          From the above argument, $n\in N_G(H)$ for a subgroup $H=\{e,h\}$
          of order 2 is equivalent to $nh=hn$. So $N_G(H)=G$ implies that
          $h$ commutes with all elements of $G$, which means that $h$ is
          contained in $Z(G)$. Also, $e$ is contained in $Z(G)$, so $H$ is
          a subset of $Z(G)$. Since $H$ is also a group, it is a subgroup
          of $Z(G)$.
        \end{proof}

      \item Question 14:
        \begin{proof}
        \end{proof}
    \end{enumerate}

  \item Section 2.5
    \begin{enumerate}
      \item Question 6c: Use the given lattices to help find the
        centralizers of every element in $S_3$.
        \begin{proof}
          Since centralizers are subgroups, we use the given lattices to
          identify the subgroup that is the centralizer of a given element.
          From the lattice of $S_3$, we notice that there are only four
          non-trivial subgroups, each generated by a single element. Also,
          if a given element $g$ commutes with another element $h$, then
          $g$ should commute with the subgroup generated by $h$, since
          $gh=hg$ implies from induction on $r$ that $gh^r=h^rg$. Hence, to
          find out which subgroup is the centralizer of a given element
          $g$, it suffices to check which of the four generators $g$
          commutes with and classify $g$ according to the following rules:
          \begin{enumerate}
            \item If $g$ commutes with more than one generator $h_1$ and
              $h_2$ then its centralizer must be an ancestor of the
              subgroups $\langle h_1\rangle$ and $\langle h_2\rangle$,
              which can only be $S_3$, based on the lattice of $S_3$.
            \item If $g$ commutes with exactly one generator $h$ then its
              centralizer must be $\langle h\rangle$.
            \item Otherwise, $g$ commutes with none of the generators, so
              its centralizer must be $1$.
          \end{enumerate}

          Using the above rules, we obtain the centralizers as follows:
          \begin{center}
            \begin{tabular}{|l|l|}
              \hline
              $g\in S_3$  & $C_{S_3}(g)$              \\ \hline\hline
              $e$         & $S_3$                     \\ \hline
              $(1,2)$     & $\langle(1,2)\rangle$     \\ \hline
              $(1,3)$     & $\langle(1,3)\rangle$     \\ \hline
              $(2,3)$     & $\langle(2,3)\rangle$     \\ \hline
              $(1,2,3)$   & $\langle(1,2,3)\rangle$   \\ \hline
              $(1,3,2)$   & $\langle(1,2,3)\rangle$   \\ \hline
            \end{tabular}
          \end{center}
        \end{proof}
    \end{enumerate}
\end{enumerate}

\end{document}


