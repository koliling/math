

\documentclass{article}
\usepackage[left=3cm,right=3cm,top=3cm,bottom=3cm]{geometry}
\usepackage{amsmath,amssymb,amsthm}
\usepackage{color}
%\setlength{\parindent}{0mm}

\newcommand{\TODO}[1]{\textcolor{red}{TODO: #1}}

\begin{document}
\title{Graduate Algebra I: Homework 3}
\author{Li Ling Ko\\ lko@nd.edu}
\date{\today}
\maketitle

\begin{enumerate}
  \item
    \begin{enumerate}
      \item Consider $A=\{(1,2,3,4)\}\subset S_4$. Compute the centralizer
        and the normalizer of $A$ in $S_4$.

        \begin{proof}
          Since $A$ has a single element, by chasing definitions, its
          normalizer is the same as its centralizer. The centralizer and
          The elements of $S_4$ are either the identity, a 2-cycle, a
          3-cycle, two disjoint 2-cycles, or a 4-cycle. We check which of
          these elements $(1,2,3,4)$ commutes with. We observe that if a
          given element $g$ commutes with another element $h$, then $g$
          would commute with the subgroup generated by $h$, since $gh=hg$
          implies from induction on $r$ that $gh^r=h^rg$. Hence we do not
          need to check for commutativity with an element in the cyclic
          subgroup of another element which was found to commute with
          $(1,2,3,4)$. We also use symmetrical arguments to decrease the
          number of times we need to check for commutativity. \\

          Clearly the identity commutes with $(1,2,3,4)$. As for the
          2-cycles, we check that $(1,2)$ does not commute with
          $(1,2,3,4)$, so by symmetry, neither should $(2,3)$, $(3,4)$, or
          $(4,1)$. Also, we check that $(1,3)$ does not commute with
          $(1,2,3,4)$, so by symmetry, neither should $(2,4)$. Next, we
          check for commutativity with the 3-cycles. $(1,2,3,4)$ does not
          commute with $(1,2,3)$, and since $(1,2,3)$ is contained in the
          group generated by $(1,3,2)$, $(1,2,3,4)$ would not commute with
          $(1,3,2)$ either. Also by symmetry, $(1,2,3,4)$ should not
          commute with $(2,3,4)$ if it did not commute with $(1,2,3)$,
          which means $(1,2,3,4)$ would not commute with $(2,4,3)$.
          Finally, $(1,2,3,4)$ does not commute with $(1,2,4)$, so it will
          not commute with $(1,4,2)$. So $(1,2,3,4)$ does not commute with
          any 2-cycles or 3-cycles. \\

          Next, we check commutativity with the 4-cycles. Clearly
          $(1,2,3,4)$ commutes with itself and hence with any 4-cycle in
          its cyclic subgroup. We check that $(1,2,3,4)$ does not commute
          with $(1,2,4,3)$, $(1,3,4,2)$, $(1,3,2,4)$, $(1,4,2,3)$, or
          $(1,4,3,2)$. Any other 4-cycle not in $\langle A\rangle$
          generates a cyclic subgroup that contains one of these
          non-commutating elements, so the only 4-cycles that commute with
          $(1,2,3,4)$ are those in its cyclic subgroup. Finally, we check
          for commutativity with the disjoint 2-cycles. We check that
          $(1,2,3,4)$ does not commute with $(1,2)(3,4)$, $(1,3)(2,4)$,
          or $(1,4)(2,3)$, which are all the possible disjoint 2-cycles. We
          conclude that the centralizer of $A$, which is also its
          normalizer, is the its cyclic subgroup, which is
          $\langle(1,2,3,4)\rangle$. \\
        \end{proof}

      \item Consider $H=\langle(1,2,3,4)\rangle\subset S_4$. Compute the
        centralizer and the normalizer of $H$ in $S_4$.

        \begin{proof}
        \end{proof}
    \end{enumerate}

  \item Section 2.5
    \begin{enumerate}
      \item Question 6c: Use the given lattices to help find the
        centralizers of every element in $S_3$.
        \begin{proof}
          Since centralizers are subgroups, we use the given lattices to
          identify the subgroup that is the centralizer of a given element.
          From the lattice of $S_3$, we notice that there are only four
          non-trivial subgroups, each generated by a single element. Also,
          if a given element $g$ commutes with another element $h$, then
          $g$ should commute with the subgroup generated by $h$, since
          $gh=hg$ implies from induction on $r$ that $gh^r=h^rg$. Hence, to
          find out which subgroup is the centralizer of a given element
          $g$, it suffices to check which of the four generators $g$
          commutes with and classify $g$ according to the following rules:
          \begin{enumerate}
            \item If $g$ commutes with more than one generator $h_1$ and
              $h_2$ then its centralizer must be an ancestor of the
              subgroups $\langle h_1\rangle$ and $\langle h_2\rangle$,
              which can only be $S_3$, based on the lattice of $S_3$.
            \item If $g$ commutes with exactly one generator $h$ then its
              centralizer must be $\langle h\rangle$.
            \item Otherwise, $g$ commutes with none of the generators, so
              its centralizer must be $1$.
          \end{enumerate}

          Using the above rules, we obtain the centralizers as follows:
          \begin{center}
            \begin{tabular}{|l|l|}
              \hline
              $g\in S_3$  & $C_{S_3}(g)$              \\ \hline\hline
              $e$         & $S_3$                     \\ \hline
              $(1,2)$     & $\langle(1,2)\rangle$     \\ \hline
              $(1,3)$     & $\langle(1,3)\rangle$     \\ \hline
              $(2,3)$     & $\langle(2,3)\rangle$     \\ \hline
              $(1,2,3)$   & $\langle(1,2,3)\rangle$   \\ \hline
              $(1,3,2)$   & $\langle(1,2,3)\rangle$   \\ \hline
            \end{tabular}
          \end{center}
        \end{proof}
    \end{enumerate}
\end{enumerate}

\end{document}


