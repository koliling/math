\documentclass{article}
\usepackage[left=3cm,right=3cm,top=3cm,bottom=3cm]{geometry}
\usepackage{amsmath,amssymb,amsthm,pgfplots,tikz}
\usepackage{color}
%\setlength{\parindent}{0mm}

\newcommand{\TODO}[1]{\textcolor{red}{TODO: #1}}

\begin{document}
\title{Graduate Algebra I: Homework 4}
\author{Li Ling Ko\\ lko@nd.edu}
\date{\today}
\maketitle

\begin{enumerate}
  \item Assume $G$ is an abelian group of order $2n$ where $n$ is odd, and
    contains two elements $a,b$ of order 2. Then the subgroup $H=\langle
    a,b\rangle=\{1,a,b,ab\}$ has four distinct elements and has order 4,
    which contradicts Lagrange's theorem that the order of a subgroup should
    divide the order of the group.

  \item
    \begin{enumerate}
      \item \TODO{Complete.}
      \item \TODO{Complete.}
    \end{enumerate}

  \item Section 3.2 Question 13: Label the vertices of a square in
    clockwise direction by 1, 2, 3, and 4. Since $D_8$ is generated by two
    distinct elements $r$ and $s$ where $r$ is a rotation by 90 degrees and
    $s$ is a flip, we can associate $r$ with the permutation
    $\sigma_r=(1,2,3,4)$ and $s$ with the permutation
    $\sigma_s=(1,2)(3,4)$. This would identify $D_8$ as a subgroup of
    $S_4$. \\

    From Proposition 13, $|D_8\langle 1,2,3\rangle|=|D_8||\langle
    1,2,3\rangle|/|D_8\cap\langle 1,2,3\rangle|$. Now $|D_8|=8$ and
    $|\langle 1,2,3\rangle|=3$, and since intersections of subgroups are
    also subgroups whose order divides the original two subgroups by
    Lagrange's theorem, $|D_8\cap\langle 1,2,3\rangle|$ must divide 3 and
    8, implying that $D_8\cap\langle 1,2,3\rangle$ can only be the trivial
    subgroup $\{1\}$. Then $|D_8\langle 1,2,3\rangle|=8\times 3/1=24$,
    implying that $D_8\langle 1,2,3\rangle$ is the whole of $S_4$. \\

    Assume by contradiction that $D_8$ contains a non-identity element $x$
    and $D_8\langle 1,2,3\rangle$ contains a non-identity element $y$ such
    that $x$ commutes with $y$. Now since the order of an element must
    divide its subgroup, $ord(x)$ can only be 2 or 4 since $D_8$ is not
    cyclic, and $ord(y)$ can only be 3. Then from commutativity of $x$ and
    $y$, the order of $xy$ is either 6 or 12. However, no element of $S_4$
    has order 12: each element of $S_4$ can be decomposed into disjoint
    cycles of order less than or equal 4, such that the order of the
    element is the product of the orders of each cycle, hence the elements
    of $S_4$ can only have orders 1, 2, 3, or 4. Thus, non-identity
    elements of $D_8$ and $\langle 1,2,3\rangle$ do not commute.

  \item Section 3.2 Question 14: We first show that $S_4$ has no normal
    subgroup of order 8. Assume that $N\trianglelefteq S_4$ has order 8.
    Then the quotient group $S_4/N$ has order 3, and is therefore cyclic.
    So given any $g\in S_4$ outside $N$, $S_4/N$ will be generated by $gN$
    of order 3, and we have $(gN)^3=N$. In particular, $g^3=1$, so the
    order of $g$ must be a multiple of 3. Now from argument in the previous
    question, elements of $S_4$ only have orders 1, 2, 3, or 4. Hence $g$
    must have order 3. This implies there should be $|S_4|-|N|=24-8=16$
    elements of order 3. However, there are only 8 elements in $S_4$ of
    order 3: Each element in $S_4$ is a composition of disjoint cycles such
    that elements 1 to 4 appear less than once in all the cycles, and the
    order of an element is the product of the orders of the disjoint
    cycles, hence the only elements of order 3 are the 3 cycles, and there
    exactly 8 of them. Hence $S_4$ cannot have a normal subgroup of order
    8. \\

    Assume we have a normal subgroup $N$ of order 3. Since groups of prime
    order are cyclic, $N$ must be generated by some $n\in S_4$ of order 3.
    From argument in the earlier paragraph, there are exactly 8 elements in
    $S_4$ of order 3, so let $m\in S_4$ be one of them that is not in $N$.
    Now since the order of elements in a group should divide the order of
    the group, we have $ord(mN)\mid|S_4/N|=|S_4|/|N|=24/3=8$, which
    implies $(mN)^8=m^8N=N$, and in particular that $m^8=1$, which
    contradicts that the order of $m$ is 3.

  \item Section 3.2 Question 16: The multiplicative group
    $(\mathbb{Z}/p\mathbb{Z})^\times$ has $p-1$ elements
    $\{\bar{1},\ldots,\bar{p-1}\}$. By Lagrange's theorem, the order of any
    element $\bar{a}$ should divide the order of the group $p-1$, which
    implies that $\bar{a}^{p-1}=\bar{1}$. In other words, given any
    $a\in\mathbb{Z}$, we have $a^{p-1} \equiv 1\pmod{p}$, then multiplying
    by $a$ we have $a^p \equiv a\pmod{p}$ as required.
\end{enumerate}
\end{document}
