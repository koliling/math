\documentclass{article}
\usepackage[left=3cm,right=3cm,top=3cm,bottom=3cm]{geometry}
\usepackage{amsmath,amssymb,amsthm,pgfplots,tikz}
\usepackage{color}
%\setlength{\parindent}{0mm}

\newcommand{\TODO}[1]{\textcolor{red}{TODO: #1}}

\begin{document}
\title{Graduate Algebra I: Homework 5}
\author{Li Ling Ko\\ lko@nd.edu}
\date{\today}
\maketitle

\begin{enumerate}
  \item Section 4.2 Question 5
    \begin{enumerate}
      \item
        \begin{proof}
          Under the given representation, $\pi_H$ can be summarized in the
          following table:
          \begin{center}
            \begin{tabular}{|l|l|}
              \hline
              $g\in D_8$ & $\pi_H(g)\in S_4$ \\
              \hline\hline
              $1$     & $(1)$ \\
              $r$     & $(1234)$ \\
              $r^2$   & $(13)(24)$ \\
              $r^3$   & $(1432)$ \\
              $s$     & $(24)$ \\
              $sr$    & $(14)(23)$ \\
              $sr^2$  & $(13)$ \\
              $sr^3$  & $(12)(34)$ \\
              \hline
            \end{tabular}
          \end{center}
          The map $\pi_H:D_8\rightarrow S_4$ is a homomorphism, so its
          image is a subgroup of $S_4$. From the table, the kernel of this
          map contains only the identity, which implies that $\pi_H$ is
          injective, the representation is faithful, and $\pi_H(D_8)$ is
          isomorphic to a subgroup of $S_4$.
        \end{proof}
      \item
        \begin{proof}
          Let $\pi_H'$ be the representation after relabelling. Since
          relabelling swaps 2 and 3, we have the following relation:
          \begin{align*}
            \pi_H'(g) &= (23)\pi_H(g)(23)^{-1} \\
                      &= (23)\pi_H(g)(23).
          \end{align*}
          Hence from the above table for $\pi_H$ we obtain the following
          table for $\pi_H'$:
          \begin{center}
            \begin{tabular}{|l|l|}
              \hline
              $g\in D_8$ & $\pi_H'(g)\in S_4$ \\
              \hline\hline
              $1$     & $(1)$ \\
              $r$     & $(1324)$ \\
              $r^2$   & $(12)(34)$ \\
              $r^3$   & $(1423)$ \\
              $s$     & $(34)$ \\
              $sr$    & $(14)(23)$ \\
              $sr^2$  & $(12)$ \\
              $sr^3$  & $(13)(24)$ \\
              \hline
            \end{tabular}
          \end{center}
        \end{proof}
    \end{enumerate}
\end{enumerate}
\e
nd{document}
