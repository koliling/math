Section 4.3 Question 2:
\begin{enumerate}
  \item Find all conjugacy classes and their sizes in $D_8$.
    \begin{proof}
      Recall that $D_8$ is generated by distinct elements $r$ and $s$ via
      the following generating rules: $r^4=s^2=1$ and $rs=sr^{-1}$, and the
      distinct elements of $D_8$ are $D_8=\{1,r,r^2,r^3,s,sr,sr^2,sr^3\}$.
      We go through each element $g\in D_8$ to find their conjugacy class
      $\kappa(g)\subset D_8$. \\

      \begin{enumerate}
        \item $\kappa(1)$ = \{1\}.
        \item $\kappa(r^n)$: $r^n$ conjugated with elements of the form $r^k$
          still give $r^n$. When conjugated with elements of the form
          $sr^k$, we get $sr^kr^nr^{-k}s=sr^ns=r^{-n}$. Hence
          \begin{equation*}
            \kappa(r^n) = \{r^n,r^{-n}\}.
          \end{equation*}
        \item $\kappa(sr^n)$: $sr^n$ conjugated with elements of the form $r^k$
          give $r^ksr^nr^{-k}=sr^{n-2k}$.
          When conjugated with elements of the form $sr^k$, we get
          $sr^ksr^nr^{-k}s=r^{-k+n-k}s=sr^{2k-n}$. Hence
          \begin{equation*}
            \kappa(sr^n) = \{sr^n,sr^{n-2k},sr^{2k-n}\}.
          \end{equation*}
      \end{enumerate}

      Summarizing, we have the following conjugancy classes: $\{1\}$,
      $\{r,r^3\}$, $\{r^2\}$, $\{s,sr^2\}$, and $\{sr,sr^3\}$.
    \end{proof}

  \item Find all conjugacy classes and their sizes in $A_4$.
    \begin{proof}
      From Proposition 11 of Section 4.3, two elements of $S_n$ are
      conjugate if and only if they have the same cycle type. From
      Proposition 25 of Section 3.5, permutations are even if and only if
      they have an even number of even cycles in their cycle decomposition.
      From these two theorems, we get the conjugacy classes of $S_4$ as
      follows:
      \begin{align*}
        \kappa_1' &:= \{(1)\} \\
        \kappa_2' &:= \{(12)(34),(13)(24),(14)(23)\} \\
        \kappa_3' &:= \{(123),(132),(124),(142),(134),(143),(234),(243)\}. \\
      \end{align*}

      Now since $A_4\subset S_4$, for two elements in $A_4$ to belong to
      the same conjugacy class, they must first belong to the same
      conjugacy class in $S_4$. We first consider the conjugacy class of
      the disjoint 2x2 cycles in $\kappa_2'$. We check that $(12)(34)$ when
      conjugated with $(123)$ gives $(13)(24)$, hence $(12)(34)$ and
      $(13)(24)$ belong to the same conjugacy class. By renaming, the
      last 2x2 cycle must also be placed in the same conjugacy
      class. \\

      Next we consider the conjugacy class of the 3-cycles in $\kappa_3'$.
      We check that $(123)$ conjugated with $(12)(34)$ gives $(142)$, hence
      $(123)$ and $(142)$ belong to the same conjugacy class. By
      renaming 2 with 4 and 3 with 2, we get $(134)$ also belongs
      to the same conjugacy class. From applying renaming repeatedly, we
      get that the elements $\{(234),(123),(142),(134)\}$ belong to the
      same conjugacy class, and also $\{(243),(132),(143),(124)\}$ belong
      to the same conjugacy class. It remains to find out if these two sets
      of elements should belong to the same class. Now if these two sets
      should be combined, then we would have a conjugacy class of size 8,
      which does not divide $|A_4|=12$, contradicting Lagrange's theorem.
      \\

      Summarizing, $A_4$ has four conjugacy classes as follows:
      \begin{align*}
        \kappa_1 &:= \{(1)\} \\
        \kappa_2 &:= \{(12)(34),(13)(24),(14)(23)\} \\
        \kappa_3 &:= \{(234),(123),(142),(134)\} \\
        \kappa_4 &:= \{(243),(132),(143),(124)\}. \\
      \end{align*}
    \end{proof}
\end{enumerate}
