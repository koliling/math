Section 4.4 Question 13: Let $G$ be a group of order 203. Prove that if $H$
is a normal subgrou of order 7 in $G$ then $H\leq Z(G)$. Deduce that $G$ is
abelian in this case. 

\begin{proof}
  The argument for $H\leq Z(G)$ is the same as for Question 12 Section 4.4:
  From Proposition 13 of Section 4.4, $G/C_G(H)$ is isomorphic to a
  subgroup of $\text{Aut}(H)$. Now $\text{Aut}(H)$ has order 6 from
  Proposition 17.1 of Section 4.4.  Hence from Lagrange's theorem,
  $G/C_G(H)$ must have order that divides 6. Also, $|G/C_G(H)|$ must divide
  $|G|=203=7\cdot29$, and so $|G/C_G(H)|$ can only be 1, which implies that
  $C_G(H)=G$. Thus every element of $G$ commutes with every element of $H$,
  so $H\leq Z(G)$. \\

  Since $|G|=203=7\cdot29$, by Lagrange's theorem, $|Z(G)|$ can only be 1, 7,
  29, or 203. Since $H\leq Z(G)$ and $|H|=7$, by Lagrange's theorem again,
  $|Z(G)|$ can only be 7 or 203. If $|Z(G)|=203$, then $Z(G)=G$, which
  would imply that G is abelian. So assume $|Z(G)|=7$. Then
  $Z(G)=H\triangleleft G$, and $G/Z(G)\cong\mathbb{Z}_{29}$. So
  $G/Z(G)=\langle gZ(G)\rangle$ for some $g\in G$, which means every
  element in $G$ can be expressed as $g^kz$ for some $k\in\mathbb{Z}_{29}$
  and some $z\in Z(G)$. So given elements $g^{k_1}z_1$ and $g^{k_2}z_2$ in
  $G$, we have
  \begin{align*}
    (g^{k_1}z_1)(g^{k_2}z_2)  &= g^{k_1+k_2}z_2z_1  & (\because z_1,z_2\in
      Z(G)) \\
                              &= (g^{k_2}z_2)(g^{k_1}z_1),  & \\
  \end{align*}
  which implies that $G$ is abelian.
\end{proof}
