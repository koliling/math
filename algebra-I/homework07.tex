\documentclass{article}
\usepackage[left=3cm,right=3cm,top=3cm,bottom=3cm]{geometry}
\usepackage{amsmath,amssymb,amsthm,pgfplots,tikz}
\usepackage[inline]{enumitem}
\usepackage{color}
%\setlength{\parindent}{0mm}

\newcommand{\TODO}[1]{\textcolor{red}{TODO: #1}}

\begin{document}
\title{Graduate Algebra I: Homework 7}
\author{Li Ling Ko\\ lko@nd.edu}
\date{\today}
\maketitle

\begin{enumerate}[label={\bf Q\arabic*:}]
  \item Does $SL_n(\mathbb{F})$ have a complement in $GL_n(\mathbb{F})$? If
    so provide it and give the semi-direct product structure of
    $GL_n(\mathbb{F})$ in terms of $SL_n(\mathbb{F})$ and this complement. If
    not, prove that it does not.

    \begin{proof}
      Yes. Consider the subgroup $K\subset GL_n(\mathbb{F})$ defined by all
      diagonal matrices such that all diagonal entries that are not at the
      top left corner are 1. It is routine to show that $K$ is a subgroup
      of $GL_n(\mathbb{F})$, and $SL_n(\mathbb{F})\cap K=1$. It remains to
      show that $GL_n(\mathbb{F})=SL_n(\mathbb{F})\rtimes K$. Given $A\in
      GL_n(\mathbb{F})$, let $A'$ be the matrix where we divide the first
      column of $A$ by $|A|$, and let $B$ be the matrix in $K$ whose entry
      in the top left corner is $|A|$. Then $A=A'B$, where $A'\in
      SL_n(\mathbb{F})$ and $B\in K$. Hence
      $GL_n(\mathbb{F})=SL_n(\mathbb{F})\rtimes K$.
    \end{proof}

  \item Can $Q_8$ be expressed as a semi-direct product? That is, are there
    subgroups $H$, $K$ of $Q_8$ such that $Q_8\cong H\rtimes K$?

    \begin{proof}
      No. For the $Q_8\cong H\rtimes K$ we need to find two non-trivial
      subgroups of $Q_8$ that intersect trivially. The assertion follows
      because every non-trivial subgroup of $Q_8$ must contain both 1 and
      -1, since $i^2=(-i)^2=j^2=(-j)^2=k^2=(-k)^2=-1$.
    \end{proof}

  \item Section 5.4 Question 8: Assume that $x,y\in G$ and both $x$ and $y$
    commute with $[x,y]$. Prove that for all $n\in\mathbb{Z}^+$,
    $(xy)^n=x^ny^n[y,x]^{\frac{n(n-1)}{2}}$.


    \begin{proof}
      We prove by induction on $n$. The base case $n=1$ is trivial.
      Consider the inductive step $n+1$. We have
      \begin{align*}
        (xy)^{n+1}  = xy(xy)^n &
                    = xyx^ny^n[y,x]^{\frac{n(n-1)}{2}}, \\
      \end{align*}
      by the inductive hypothesis. For the inductive step to work, we want
      the last formula to equal
      \begin{align*}
        xyx^ny^n[y,x]^{\frac{n(n-1)}{2}} =
        x^{n+1}y^{n+1}[y,x]^{\frac{n(n+1)}{2}}. \\
      \end{align*}
      We work backwards to get the desired equality:
      \begin{align*}
                          & xyx^ny^n[y,x]^{\frac{n(n-1)}{2}} &=&
          x^{n+1}y^{n+1}[y,x]^{\frac{n(n+1)}{2}} & \\
        \Leftrightarrow\; & xyx^ny^n &=& x^{n+1}y^{n+1}[y,x]^n & \\
        \Leftrightarrow\; & xyx^ny^n[x,y]^n &=& x^{n+1}y^{n+1} & (\because
          [y,x]=[x,y]^{-1}) \\
        \Leftrightarrow\; & yx^ny^n[x,y]^n &=& x^{n}y^{n+1} & \\
        \Leftrightarrow\; & yx^n[x,y]^ny^n &=& x^{n}y^{n+1} & (\because
          y\; \text{commutes with}\; [x,y]) \\
        \Leftrightarrow\; & yx^n[x,y]^n &=& x^{n}y & \\
        \Leftrightarrow\; & y(x[x,y])^n &=& x^{n}y & (\because x\;
          \text{commutes with}\; [x,y]) \\
        \Leftrightarrow\; & (x[x,y])^n &=& y^{-1}x^{n}y & \\
        \Leftrightarrow\; & (x[x,y])^n &=& (y^{-1}xy)^{n} & \\
        \Leftarrow\;      & x[x,y] &=& y^{-1}xy & \\
      \end{align*}
      The last equality holds because $x[x,y]=xx^{-1}y^{-1}xy=y^{-1}xy$,
      which completes the inductive step.
    \end{proof}

  \item Section 5.4 Question 9: Prove that if $p$ is an odd prime and $P$
    is a group of order $p^3$ then the $p^{\text{th}}$ power map $x\mapsto
    x^p$ is a homomorphism of $P$ into $Z(P)$. If $P$ is not cyclic, show
    that the kernel of the $p^{\text{th}}$ power map has order $p^2$ or
    $p^3$. Is the squaring map a homomorphism in non-abelian groups of
    order 8? Where is the oddness of $p$ needed in the above proof?

    \begin{proof}
      From Theorem 8 of Section 4.3 and Lagrange's theorem, $Z(P)$ is
      non-trivial, so it has either order $p^3$, $p^2$, or $p$. If
      $|Z(P)|=p^3$ then the first statement holds trivially. If $|Z(P)|=p^2$,
      then $P/Z(P)$ is cyclic and then $P$ is abelian from a previous
      exercise. So $Z(P)=P$, then the first statement holds trivially.
      Finally if $|Z(P)|=p$, the quotient group $P/Z(P)$ will have order
      $p^2$, so from earlier exercise $P/Z(P)$ will be isomorphic to
      $\mathbb{Z}_{p^2}$ or $\mathbb{Z}_{p}\times\mathbb{Z}_{p}$, which are
      both abelian groups. Then from Proposition 7.4 of Section 5.4, $Z(P)$
      must contain $P'$. Then by Lagrange's theorem, $Z(P)$ must be $P'$.
      Then given $x,y\in P$,
      \begin{align*}
        (xy)^p &= x^py^p[y,x]^{\frac{p(p-1)}{2}}  & (\text{Exercise 8 of
               Section 5.4, proven above}) \\
               &= x^py^p([y,x]^p)^{\frac{(p-1)}{2}}  & (\because 2|(p-1)) \\
               &= x^py^p(1)^{\frac{(p-1)}{2}}     & (\because [y,x]\in P'\;
               \text{and}\; |P'|=p) \\
               &= x^py^p,                         & \\
      \end{align*}
      so the map sending $x$ to $x^p$ is a homomorphism. Finally, we need
      to show that $x^p\in Z(P)$ for every $x\in P$. Equivalently,
      we need $x^py=yx^p$ for every $x,y\in P$. We work backwards to get
      the required equation:
      \begin{align*}
                            & x^py                &=& yx^p   & \\
        \Leftrightarrow\;   & y^{-1}x^py          &=& x^p    & \\
        \Leftrightarrow\;   & (y^{-1}xy)^p        &=& x^p    & \\
        \Leftrightarrow\;   & (y^{-1}xy)^p(x^{-1})^p &=& 1    & \\
        \Leftrightarrow\;   & (y^{-1}xyx^{-1})^p  &=& 1    & (\because
          (ab)^p=a^pb^p\; \text{as shown earlier}) \\
        \Leftrightarrow\;   & [y,x^{-1}]^p        &=& 1    & (\because
          y^{-1}xyx^{-1}=[y,x^{-1}]), \\
      \end{align*}
      and the last statement holds because $[y,x^{-1}]\in P'$ and $P'$ has
      order $p$. Thus $x^p\in Z(P)$ for every $x\in P$, which completes the
      proof for the first statement. \\

      Now we prove the second assertion. If $|Z(P)|=p^3$, the group is
      cyclic. If $|Z(P)|=p^2$, then we have shown before that because
      $P/Z(P)$ is cyclic, $P$ will be abelian, which contradicts $Z(P)\neq
      P$. Hence assume $|Z(P)|=p$. Then from the earlier assertion, since
      the $p^{\text{th}}$ power map sends elements to $Z(P)$, the map is
      either surjective or trivial. If the map is trivial, then the kernel
      has order $p^3$. If the map is surjective, then from first
      isomorphism theorem, the kernel has order order $|P|/|Z(P)|=p^2$, as
      required. \\

      Squaring $D_8$ is not a homomorphism because $(sr)^2=srsr=1\neq r^2=
      s^2r^2$. \\

      Oddness of $p$ is used when showing that $(xy)^p=x^py^p$. In the
      prove above, there was a step that required 2 to divide $p-1$, which
      will not hold if $p=2$.
    \end{proof}

  \item Section 5.4 Question 10: Prove that a finite abelian group is the
    direct product of its Sylow subgroups.
    \begin{proof}
      For any abelian group $G$, each Sylow $p$-subgroup is normal in $G$,
      hence by Corollary 20 of Chapter 4.5 the subgroup is the only Sylow
      $p$-subgroup of $G$. Let $P_1,\ldots,P_n$ be the Sylow subgroups of
      $G$. First, we prove that $G=P_1\ldots P_n$. By $n$ applications of
      the fact that $|HK|=|H||K|/|H\cap K|$ for subgroups $H$ and $K$ and
      also the fact that $|(P_1\ldots P_i)\cap P_{i+1}|=1$ from Lagrange's
      theorem, we get $|G|=|P_1\ldots P_n|$. Hence from finiteness of
      $|G|$, we have $G=P_1\ldots P_n$. \\

      Next, we show by induction on $n$ that $G\cong P_1\times\ldots\times
      P_n$. The case $n=1$ is trivial. Consider the case of $n+1$. Since
      both $P_1\ldots P_n$ and $P_{n+1}$ are normal subgroups of $G$, by
      Theorem 9 of Section 5.4, $(P_1\ldots P_{n+1})\cong(P_1\ldots
      P_n)\times P_{n+1}$. In the preceding paragraph, we have $G\cong
      (P_1\ldots P_{n+1})$, and from
      induction hypothesis, $P_1\ldots P_n\cong P_1\times\ldots\times P_n$,
      so we get $G\cong P_1\times\ldots\times P_{n+1}$.
    \end{proof}

  \item Section 5.4 Question 11: Prove that if $G=HK$ where $H$ and $K$ are
    characteristic subgroups of $G$ with $H\cap K=1$ then
    $\text{Aut}(G)\cong\text{Aut}(H)\times\text{Aut}(K)$. Deduce that if
    $G$ is abelian group of finite order then $\text{Aut}(G)$ is isomorphic
    to the direct product of the automorphism groups of its Sylow
    subgroups.

    \begin{proof}
      Note that since characteristic subgroups are normal, from Theorem 9
      of Section 5.4, we have $G=HK\cong H\times K$. \\

      Consider the map
      $\phi:\text{Aut}(G)\rightarrow\text{Aut}(H)\times\text{Aut}(K)$ which
      sends $\varphi\in\text{Aut}(G)$ to $(\varphi|_H,\varphi|_K)$. This
      map is well-defined because $H$ and $K$ are characteristic subgroups
      of $G$. $\phi$ is a homomorphism because if $\varphi$ and $\theta$
      are automorphisms of $G$, then $(\varphi\circ\theta)|_H$ and
      $(\varphi\circ\theta)|_K$ are automorphisms of $H$ and $K$
      respectively, with
      $(\varphi\circ\theta)|_H=\varphi|_H\circ\theta|_H$, and similarly for
      $K$. \\

      The map $\phi$ is surjective because $G\cong H\times K$: Given
      $\varphi_H\in\text{Aut}(H)$ and $\varphi_K\in\text{Aut}(K)$, the map
      $\varphi_G$ that sends $(h,k)\in H\times K$ to
      $(\varphi_H(h),\varphi_K(k))$ is an automorphism of $G$ because
      \begin{align*}
        \varphi_G(h_1h_2,k_1k_2)  &= (\varphi_H(h_1,h_2),\varphi_K(k_1,k_2)) \\
          &= (\varphi_H(h_1)\varphi_H(h_2),\varphi_K(k_1)\varphi_K(k_2)) \\
          &= (\varphi_H(h_1),\varphi_K(k_1))(\varphi_H(h_2),\varphi_K(k_2)) \\
          &= \varphi_G(h_1,k_1)\varphi_G(h_2,k_2). \\
      \end{align*}
      Also, $\varphi_G$ gets sent to $(\varphi_H,\varphi_K)$ under $\phi$.
      \\

      The map $\phi$ is also injective: If $\varphi\in\text{Aut}(G)$
      preserves all $h\in H$ and $k\in K$, then given any $(h,k)\in G$, we
      have
      \begin{align*}
        \varphi((h,k))  &= \varphi((h,1)(1,k)) \\
                        &= \varphi((h,1))\varphi((1,k)) \\
                        &= (h,1)(1,k) \\
                        &= (h,k), \\
      \end{align*}
      which implies that $\varphi$ is the trivial automorphism. \\

      From the solution to the previous question 5.4.10, we showed that
      $G=P_1\ldots P_n$, where $P_1,\ldots,P_n$ are the Sylow subgroups of
      $G$. We prove the claim by induction on $n$. The case $n=1$ is
      trivially true. For $n+1$, we have 
      \begin{align*}
        \text{Aut}(G) &\cong \text{Aut}(P_1\ldots
          P_n)\times\text{Aut}(P_{n+1}) & (\text{by earlier assertion}) \\
                      &\cong
                      \text{Aut}(P_1)\times\ldots\times\text{Aut}(P_n)
          \times\text{Aut}(P_{n+1}). & (\text{by induction hypothesis}) \\
      \end{align*}
    \end{proof}

  \item Section 5.5 Question 6: Assume that $K$ is a cyclic group, $H$ is
    an arbitrary group and $\varphi_1$ and $\varphi_2$ are homomorphisms
    from $K$ into $\text{Aut}(H)$ such that  $\varphi_1(K)$ and
    $\varphi_2(K)$ are conjugage subgroups of $\text{Aut}(H)$. If $K$ is
    infinite assume $\varphi_1$ and $\varphi_2$ are injective. Prove by
    constructing an explicit isomorphism that $H\rtimes_{\varphi_1}K\cong
    H\rtimes_{\varphi_2}K$ (in particular, if the subgroups $\varphi_1(K)$
    and $\varphi_2(K)$ are equal in $\text{Aut}(H)$, then the resulting
    semidirect products are isomorphic).

    \begin{proof}
      Since $K$ is cyclic, it contains a generator $g\in K$. Then
      $\varphi_i(K)$ is generated by $\varphi_i(g)$ for $i=1,2$.  Also,
      since $\varphi_1(K)$ and $\varphi_2(K)$ are conjugates, there exists
      an automorphism $\sigma\in\text{Aut}(H)$ such that
      $\varphi_2(K)=\sigma\varphi_1(K)\sigma^{-1}$. Now since action by
      conjugation is an isomorphism and isomorphisms maps generators to
      generators, $\sigma^{-1}\varphi_2(g)\sigma$ should be a generator of
      $\varphi_1(K)$, and hence there must exist some $a\in\mathbb{Z}$ such
      that
      \begin{align*}
        \varphi_1(g)  &= (\sigma^{-1}\varphi_2(g)\sigma)^a \\
                      &= \sigma^{-1}\varphi_2(g)^a\sigma.
      \end{align*}
      Rearranging, we get $\sigma\varphi_1(g)\sigma^{-1}=\varphi_2(g)^a$.
      Then given any $k\in K$, since $g$ generates $K$, we have $k=g^r$ for
      some $r\in\mathbb{N}$. Then
      $\varphi_i(k)=\varphi_i(g^r)=\varphi_i(g)^r$, so
      \begin{align*}
        \varphi_1(k)  &= \varphi_1(g)^r & \\
                      &= (\sigma^{-1}\varphi_2(g)^a\sigma)^r & (\text{from
                        above}) \\
                      &= \sigma^{-1}\varphi_2(g)^{ar}\sigma & \\
                      &= \sigma^{-1}(\varphi_2(g)^r)^{a}\sigma & \\
                      &= \sigma^{-1}\varphi_2(k)^{a}\sigma. & \\
      \end{align*}

      Consider the map $\Psi:H\rtimes_{\varphi_1}K\rightarrow
      H\rtimes_{\varphi_2}K$ defined by $\Psi((h,k))=(\sigma(h),k^a)$. We
      show that $\Psi$ is a homomorphism. Given $(h_1,k_1),(h_2,k_2)\in
      H\rtimes_{\varphi_1}K$, we have
      \begin{align*}
        \Psi((h_1,k_1)\cdot_{\varphi_1}(h_2,k_2)) &=
          \Psi((h_1(\varphi_1(k_1)(h_2)),\;
          k_1k_2)) & \\
          &= (\sigma(h_1(\varphi_1(k_1)(h_2))),\; (k_1k_2)^a)) & \\
          &= (\sigma(h_1(\varphi_1(k_1)(h_2))),\; k_1^ak_2^a)) & (\because K\;
          \text{is cyclic and hence abelian}) \\
          &= (\sigma(h_1)\sigma(\varphi_1(k_1)(h_2)),\; k_1^ak_2^a)) & \\
          &= (\sigma(h_1)\sigma(\sigma^{-1}\varphi_2(k_1)^a\sigma(h_2)),\;
          k_1^ak_2^a)) & (\text{from above}) \\
          &= (\sigma(h_1)((\varphi_2(k_1)^a\sigma)(h_2)),\; k_1^ak_2^a)) & \\
          &= (\sigma(h_1)((\varphi_2(k_1^a)\sigma)(h_2)),\; k_1^ak_2^a)) & \\
          &= (\sigma(h_1)((\varphi_2(k_1^a)(\sigma(h_2))),\; k_1^ak_2^a)) & \\
          &= ((\sigma(h_1),k_1^a)) \cdot_{\varphi_2} ((\sigma(h_2),k_2^a)) & \\
          &= \Psi((h_1,k_1)) \cdot_{\varphi_2} \Psi((h_2,k_2)), & \\
      \end{align*}
      so $\Psi$ is a homomorphism. \\

      To show that $\Psi$ is bijective, we use a similar argument to
      construct a homomorphism \[\Theta:H\rtimes_{\varphi_2}K\rightarrow
      H\rtimes_{\varphi_1}K\] such that
      $\Psi\circ\Theta=\Theta\circ\Psi=\text{id}$: Like before, there
      exists $b\in\mathbb{N}$ such that
      \begin{align*}
        \varphi_2(g)  &= (\sigma\varphi_1(g)\sigma^{-1})^b. \\
      \end{align*}
      Then given any $k\in K$, we will have
      \begin{align*}
        \varphi_2(k)  &= \sigma\varphi_1(k)^b\sigma^{-1}. \\
      \end{align*}
      Then given any $k\in K$, we have
      \begin{align*}
        \varphi_1(k^{ab}) &= (\varphi_1(k)^b)^a & \\
                          &= (\sigma^{-1}\varphi_2(k)\sigma)^a & \\
                          &= \sigma^{-1}\varphi_2(k)^a\sigma & \\
                          &= \varphi_1(k), & \\
      \end{align*}
      so from injectivity of $\varphi_1$, we get $k^{ab}=k$ for all $k\in
      K$. Hence defining $\Theta$ to send $(h,k)$ to $(\sigma^{-1}(h),k^b)$
      will satisfy $\Psi\circ\Theta=\Theta\circ\Psi=\text{id}$, as
      required.
    \end{proof}

  \item Section 5.5 Question 8: Construct a non-abelian group of order 75.
    Classify all groups of order 75.

    \begin{proof}
      We first classify all groups of order 75. Note that $75=3\cdot5^2$.
      Let $G$ be a group of order 75. If $G$ is abelian, then by the
      elementary divisor decomposition of finite abelian groups, $G$ must
      be isomorphic to $P_5\times\mathbb{Z}_3$, where $P_5$ is a group of
      order $25$. By classification of groups of order $p^2$ for primes
      $p$, we know that $P_5$ is isomorphic to either $Z_5\times Z_5$ or
      $Z_{25}$. Hence if $G$ is abelian it is isomorphic to either
      $\mathbb{Z}_5\times\mathbb{Z}_5\times\mathbb{Z}_3$ or
      $\mathbb{Z}_{25}\times\mathbb{Z}_3$. \\

      Now consider the case when $G$ is non-abelian. By Sylow theorems, $G$
      must contain exactly one Sylow 5-subgroup $P_5$, which must therefore
      be normal in $G$. Let $P_3$ be any one of $G$'s Sylow 3-subgroups. By
      Lagrange's theorems, $P_5\cap P_3=1$, and hence $G=P_5P_3$ and then
      from Theorem 12 of Section 5.5, and $G\cong
      P_5\rtimes_{\varphi}\mathbb{Z}_3$ for some homomorphism
      $\varphi:\mathbb{Z}_3\rightarrow\text{Aut}(P_5)$. By the
      classification of groups of order the square of primes, $P_5$ is
      isomorphic to either $\mathbb{Z}_5\times\mathbb{Z}_5$ or
      $\mathbb{Z}_{25}$. If $P_5\cong\mathbb{Z}_{25}$, then
      $|\text{Aut}(\mathbb{Z}_{25})|=|\mathbb{Z}_{25}^\times|=20$ which is
      not a multiple of 3, hence $\varphi$ can only be trivial and group
      $G$ will be abelian. Hence assume that
      $P_5=\mathbb{Z}_5\times\mathbb{Z}_5$. Now
      $\text{Aut}(\mathbb{Z}_5\times\mathbb{Z}_5)=\text{GL}_2(\mathbb{Z}_5)$,
      which has order $5^4-5^3-5^2+5=32\cdot5\cdot3$, and so all subgroups
      of $\text{Aut}(\mathbb{Z}_5\times\mathbb{Z}_5)$ of order 3 must be
      Sylow 3-subgroups, which are conjugate with each other. Therefore
      from Section 5.5 Question 6, all non-trivial $\varphi$ will induce
      isomorphic groups
      $(\mathbb{Z}_5\times\mathbb{Z}_5)\rtimes_{\varphi}\mathbb{Z}_3$.
      Also from Sylow's theorems, a non-trivial $\varphi$ exists. Therefore
      there cannot be more than one non-abelian group of order 75 up to
      isomorphism. \\

      Now we construct a non-abelian group of order 75. From the preceding
      paragraph, we need to find a non-trivial subgroup of 
      $\text{Aut}(\mathbb{Z}_5\times\mathbb{Z}_5)$ of order 3. Since
      $\text{Aut}(\mathbb{Z}_5\times\mathbb{Z}_5)\cong\text{GL}_2(\mathbb{Z}_5)$,
      it suffices to find a matrix in $\text{GL}_2(\mathbb{Z}_5)$ of order
      3. We check that $M_\varphi=\begin{bmatrix}4&4\\1&0\\\end{bmatrix}$ works.
      This matrix will send $(1,0)\in\mathbb{Z}_5\times\mathbb{Z}_5$ to
      $(4,1)$ and $(0,1)\in\mathbb{Z}_5\times\mathbb{Z}_5$ to $(4,0)$. Then
      $(\mathbb{Z}_5\times\mathbb{Z}_5)\rtimes_{\varphi}\mathbb{Z}_3$ will
      be non-abelian, as witnessed by $((1,1),0)$ and $((1,0),1)$. \\

      Summarizing, there is exactly one non-abelian group of order 75 and
      two abelian groups of order 75, up to isomorphism. 
    \end{proof}

  \item Section 5.5 Question 15: Let $p$ be an odd prime. Prove that every
    element of order 2 in $\text{GL}_2(\mathbb{F}_p)$ is conjugate to a
    diagonal matrix with $\pm1$'s on the diagonal. Classify the groups of
    order $2p^2$.

    \begin{proof}
      We first classify the groups of order $2p^2$ where $p$ is an odd
      prime. Let $G$ be a group of order $2p^2$. If $G$ is abelian, then
      by the elementary divisor decomposition of finite abelian groups, $G$
      must be isomorphic to $P\times\mathbb{Z}_2$, where $P$ is a group
      of order $p^2$. By classification of groups of order $p^2$ for primes
      $p$, we know that $P$ is isomorphic to either $Z_p\times Z_p$ or
      $Z_{p^2}$. Hence if $G$ is abelian it is isomorphic to either
      $\mathbb{Z}_p\times\mathbb{Z}_p\times\mathbb{Z}_2$ or
      $\mathbb{Z}_{p^2}\times\mathbb{Z}_2$. \\

      Now consider the case when $G$ is non-abelian. By Sylow theorems, $G$
      must contain exactly one Sylow $p$-subgroup $P$, which must therefore
      be normal in $G$. Let $\mathbb{Z}_2$ be any one of $G$'s Sylow
      2-subgroups. By Lagrange's theorems, $P\cap \mathbb{Z}_2=1$, and
      hence $G=P\mathbb{Z}_2$ and then from Theorem 12 of Section 5.5, and
      $G\cong P\rtimes_{\varphi}\mathbb{Z}_2$ for some homomorphism
      $\varphi:\mathbb{Z}_2\rightarrow\text{Aut}(P)$. By the classification
      of groups of order the square of primes, $P$ is isomorphic to either
      $\mathbb{Z}_p\times\mathbb{Z}_p$ or $\mathbb{Z}_{p^2}$. If
      $P\cong\mathbb{Z}_{p^2}$, then
      $|\text{Aut}(\mathbb{Z}_{p^2})|=|\mathbb{Z}_{p^2}^\times|=p(p-1)$,
      which is a multiple of 2, hence $\text{Aut}(\mathbb{Z}_{p^2})$ must
      contain elements of order 2. Now any automorphism $\varphi_1$ of cyclic
      group $\mathbb{Z}_{p^2}$ is completely defined by its image on the
      generator $\overline{1}\in\mathbb{Z}_{p^2}$. We want to find
      $\varphi$ such that $\varphi(\varphi(\overline{1}))=\overline{1}$.
      Now
      $\varphi(\varphi(\overline{1}))=\overline{1}=\varphi(\overline{1})^2$,
      and setting the above to $\overline{1}$, we get
      $p^2|(\varphi(\overline{1})^2-1)$, so solving for
      $\varphi(\overline{1})\in\mathbb{Z}_{p^2}$ we get
      $\varphi(\overline{1})=p^2-1$ as the only non-trivial solution.
      Hence there is only one non-trivial non-abelian group isomorphic to
      $\mathbb{Z}_{p^2}\rtimes\mathbb{Z}_2$. \\

      Next consider the case when $P=\mathbb{Z}_p\times\mathbb{Z}_p$. Now
      $\text{Aut}(\mathbb{Z}_p\times\mathbb{Z}_p)=\text{GL}_2(\mathbb{Z}_p)$,
      and every non-trivial element of order 2 in
      $\text{GL}_2(\mathbb{Z}_p)$ is conjugate to a diagonal matrix with
      $\pm1$'s on the diagonal, hence from Section 5.5 Question 6, there
      can be at most 4 groups of the form
      $(\mathbb{Z}_p\times\mathbb{Z}_p)\rtimes_{\varphi}\mathbb{Z}_2$,
      induced by the four distinct diagonal matrices with $\pm1$'s on the
      diagonal. We can ignore the case where $\varphi$ is obtained from the
      identity matrix $I$ because that will give an abelian group. Hence
      there are at most three distinct non-abelian groups of the form
      $(\mathbb{Z}_p\times\mathbb{Z}_p)\rtimes_{\varphi}\mathbb{Z}_2$.
    \end{proof}

  \item Section 5.5 Question 16: Show that there are exactly 4 distinct
    homomorphisms from $\mathbb{Z}_2$ into $\text{Aut}(\mathbb{Z}_8)$.
    Prove that the resulting semidirect products are the groups are the
    groups: $\mathbb{Z}_8\times\mathbb{Z}_2$, $D_{16}$, the quasidihedral
    group $QD_{16}$ and the modular group $M$.

    \begin{proof}
      $\text{Aut}(\mathbb{Z}_8)$ is isomorphic to
      $\mathbb{Z}_8^\times\cong\mathbb{Z}_2\times\mathbb{Z}_2$, by
      Proposition 16 of Section 4.4. Each homomorphism of $\mathbb{Z}_2$ to
      $\mathbb{Z}_2\times\mathbb{Z}_2$ is determined uniquely by the image of
      $\overline{1}$, which will be well-defined if and only if the image
      has order dividing 2. Since all four elements of
      $\mathbb{Z}_2\times\mathbb{Z}_2$ has order dividing 2, there are
      exactly distinct homomorphisms from $\mathbb{Z}_2$ into
      $\text{Aut}(\mathbb{Z}_8)$. Let $\varphi_i$ for $i\in\{1,2,3,4\}$
      denote the homomorphisms that send $\overline{1}$ to
      $(0,0),(0,1),(1,0),(1,1)\in\mathbb{Z}_2\times\mathbb{Z}_2$
      respectively. \\

      The trivial homomorphism $\varphi_1$ gives the trivial homomorphism
      $\mathbb{Z}_8\rtimes_{\varphi_1}\mathbb{Z}_2\cong\mathbb{Z}_8\times\mathbb{Z}_2$.
    \end{proof}

  \item Section 7.1 Question 7: The center of a ring $R$ is $\{z\in R|
    zr=rz\; \text{for all}\; r\in R\}$. Prove that the center of a ring is
    a subring that contains the identity. Prove that the center of a
    division ring is a field.

    \begin{proof}
      We chase definitions. The center contains the identity since $1r=r1$
      for all $r\in R$ by definition of a ring. The center is also a
      subgroup of $R$ because it is non-empty and also given $z_1,z_2$ in
      the center and any $r\in R$, we have
      \begin{align*}
        (z_1-z_2)r  &= z_1r-z_2r  & (\text{distributive law}) \\
                    &= rz_1-rz_2  & (z_1,z_2\; \text{are in the center}) \\
                    &= r(z_1-z_2),  & (\text{distributive law}) \\
      \end{align*}
      which implies that $z_1-z_2$ is also in the center. Finally, the
      center is closed under multiplication because given $z_1,z_2$ in the
      center and any $r\in R$, we have $z_1z_2r=rz_1z_2$ because $r$
      commutes with both $z_1$ and $z_2$. Hence the center is a subring of
      $R$. \\

      If $R$ is a division ring, then from above, its center contains the
      identity. Also, given $z$ in the center, and given any $r\in R$, we
      have
      \begin{align*}
        z^{-1}r &= (r^{-1}z)^{-1} & \\
                &= (zr^{-1})^{-1} & (\because z\; \text{commutes with}\;
                r^{-1}) \\
                &= rz^{-1}, & \\
      \end{align*}
      which implies that $z^{-1}$ is also contained in the center. Hence
      each element $z$ in the center has a multiplicative inverse $z^{-1}$,
      which completes the proof that the center is a field.
    \end{proof}

  \item Section 7.1 Question 12: Prove that any subring of a field which
    contains the identity is an integral domain.

    \begin{proof}
      Let $R$ be a subring of a field $F$ such that $R$ contains the
      identity of $F$. Then this identity will also be an identity of $R$.
      Also, any zero divisor of $R$ will be a zero divisor of $F$. Since
      $F$ contains no zero divisors, neither can $R$. Hence $R$ is an
      integral domain.
    \end{proof}

  \item Section 7.1 Question 13: An element $x$ in $R$ is called nilpotent
    if $x^m=0$ for some $m\in\mathbb{Z}^+$.
    \begin{enumerate}
      \item Show that if $n=a^kb$ for some integers $a$ and $b$ then
        $\overline{ab}$ is a nilpotent element of $\mathbb{Z}_n$.
        \begin{proof}
          We have
          \begin{align*}
            \overline{ab}^k &= \overline{(ab)^k} \\
                            &= \overline{a^kb^k} \\
                            &= \overline{a^kbb^{k-1}} \\
                            &= \overline{nb^{k-1}} \\
                            &= \overline{n}\overline{b^{k-1}} \\
                            &= 0\overline{b^{k-1}} \\
                            &= 0. \\
          \end{align*}
        \end{proof}
        Hence $\overline{ab}$ is nilpotent.

      \item If $a\in\mathbb{Z}$ is an integer, show that element
        $\overline{a}\in\mathbb{Z}_n$ is nilpotent if and only if every
        prime divisor of $n$ is also a divisor of $a$. In particular,
        determine the nilpotent elements of $\mathbb{Z}_{72}$ explicitely.

        \begin{proof}
          Let $n=p_1^{r_1}\ldots p_k^{r_k}$ be the prime factor
          decompostion of $n$, let $r=\max(r_1,\ldots,r_k)$, and let
          $a=p_1\ldots p_k b$ for some integer $b\in\mathbb{Z}^+$. Then
          \begin{align*}
            \overline{a}^r  &= \overline{p_1\ldots p_k b}^r \\
                            &= \overline{p_1^r\ldots p_k^r b^r} \\
                            &= \overline{p_1^{r_1}\ldots
                            p_k^{r_k}(p_1^{r-r_1}\ldots p_k^{r-r_k}b^r)}
                            \\
                            &= \overline{n(p_1^{r-r_1}\ldots
                            p_k^{r-r_k}b^r)} \\
                            &= 0, \\
          \end{align*}
          so $\overline{a}$ is nilpotent. For the converse, let
          $\overline{a}$ be a nilpotent element of $\mathbb{Z}_n$, then
          there exists some $k\in\mathbb{Z}^+$ such that $p_i|a^k$ for all
          $i\in\{1,\ldots,k\}$. This implies that $p_i|a$ for all
          $i\in\{1,\ldots,k\}$, or in other words, every prime divisor of
          $n$ must divide $a$. \\

          The distinct prime factors of 72 are only 2 and 3. Hence from the
          previous assertion, the nilpotent elements of $\mathbb{Z}_{72}$
          are exactly the multiples of $2\cdot3=6$, which are
          \[\{\overline{a}:a\in\{0,6,12,18,24,30,36,42,48,54,60,66\}\}\].
        \end{proof}

      \item Let $R$ be the ring of functiosn from a non-empty set $X$ to a
        field $F$. Prove that $R$ contains no non-zero nilpotent elements.

        \begin{proof}
          Let $f:X\rightarrow F\in R=F^X$ be a nilpotent element. Then
          $f^m=0$ for some $m\in\mathbb{Z}^+$. This implies that
          $f(x)^m=0\in F$ for all $x\in X$. Since $F$ contains no
          zero-divisors, we have $f(x)=0$ for all $x\in X$, which means $f$
          can only be zero. Hence the only nilpotent element of $R$ is 0.
        \end{proof}
    \end{enumerate}

  \item Section 7.1 Question 14: Let $x$ be a nilpotent element of the
    commutative ring $R$.
    \begin{enumerate}
      \item Prove that $x$ is either zero or a zero divisor.
        \begin{proof}
          Let $m$ be the smallest positive integer such that $x^m=0$. If
          $m=1$ then $x$ is zero, hence assume $m>1$. Then $x$ is non-zero
          and $x^{m-1}$ is also non-zero by choice of $m$, yet
          $xx^{m-1}=x^m=0$, which implies $x$ is a zero divisor.
        \end{proof}

      \item Prove that $rx$ is nilpotent for all $r\in R$.
        \begin{proof}
          Let $m\in\mathbb{Z}^+$ such that $r^m=0$. Then $(rx)^m=r^mx^m$ by
          the commutativity of $R$, and since $r^m=0$, $(rx)^m$ must also
          equal 0, implying that $rx$ is also nilpotent.
        \end{proof}

      \item Prove that $1+x$ is a unit in $R$.
        \begin{proof}
          Let $m\in\mathbb{Z}^+$ such that $r^m=0$. Consider $y\in R$
          defined by \[y=\sum_{k=0}^{m-1} (-x)^k.\] Then
          \begin{align*}
            (1+x)y  &= (1+x)\sum_{k=0}^{m-1} (-x)^k & \\
                    &= \sum_{k=0}^{m-1} (-x)^k + \sum_{k=1}^m (-x)^k & \\
                    &= 1+(-x)^m & \\
                    &= 1+(-1)^mx^m & (\text{by commutativity}) \\
                    &= 1. & (\because x^m=0) \\
          \end{align*}
          Similarly, we also get $y(1-x)=0$, hence $(1+x)$ has
          multplicative inverse $y$, which means $(1+x)$ is a unit in $R$.
        \end{proof}

      \item Deduce that the sum of a nilpotent element and a unit is a
        unit.
        \begin{proof}
          Let $u$ be any unit in $R$ and $v$ be its multiplicative inverse.
          By part (b) of this question, $vx$ is nilpotent, then by part (c)
          of this question, $1+vx$ is nilpotent, and thus by part (b) of
          this question, $u(1+vx)=u+x$ must also be nilpotent.
        \end{proof}
    \end{enumerate}

  \item Section 7.1 Question 20: Let $R$ be the collection of sequences
    $(a_1,a_2,a_3,\ldots)$ of integers $a_1,a_2,a_3,\ldots$ where all but
    finitely many of the $a_i$ are 0. Prove that $R$ is a ring under
    componentwise addition and multiplication which does not have an
    identity.

    \begin{proof}
      The zero function is contained in $R$, and acts as the zero of $R$.
      Also, $R$ is closed under addition and multiplication because given
      $f_1,f_2\in R$, their sum $f_1+f_2$ or product $f_1\times f_2$ will
      still be a sequence with only finitely many non-zero $a_i$'s. $R$ is
      also closed under inverse because given $f\in R$, its inverse which
      is the function with replace all the non-zero $a_i$'s in $f$ by
      $-a_i$ will also be contained in $R$.  Finally, $R$ is abelian as a
      group because addition of integers is abelian. \\

      Multiplication is associative because multiplication by integers is
      associative. The distribution laws also hold from the distributive
      laws of the integers. \\

      It remains to show that $R$ does not have an identity. Assume
      $f=(a_1,\ldots)\in R$ acts as the identity. Let $f_i\in R$ denote the
      function in $R$ that is 1 at the $i$th integer and zero everywhere
      else. Then $ff_i$ will be the function that is $a_i$ at the $i$th
      integer and zero everywhere else. Hence for $ff_i=f_i$, we need
      $a_i=1$. Thus $a_i=1$ for all $i\in\mathbb{Z}^+$, which implies that
      $f$ is not a function in $R$. Hence no such identity exists in $R$.
    \end{proof}

  \item Section 7.1 Question 29: Let $A$ be any commutative ring with
    identity $1\neq0$. Let $R$ be the set of all group homomorphisms of the
    additive group $A$ to itself with addition defined as pointwise
    addition of functions and multiplication defined as function
    composition. Prove that these operations make $R$ into a ring with
    identity. Prove that the units of $R$ are the group automorphisms of
    $A$.

    \begin{proof}
      We first show that the defined addition forms an abelian group. $R$
      is closed under addition because given $f,g\in R$, $f+g$ will also be
      in $R$ because given any $a_1,a_2\in A$, we have 
      \begin{align*}
        (f+g)(a_1+a_2)  &:= f(a_1+a_2)+g(a_1+a_2) \\
                        &:= f(a_1)+f(a_2)+g(a_1)+g(a_2) \\
                        &:= (f+g)(a_1+a_2)+(f+g)(a_1+a_2), \\
      \end{align*}
      which implies that $f+g$ is also a group homomorphism of $A$ and
      should therefore be contained in $R$. The homomorphism that sends
      each element in $A$ to 0 is clearly a zero in $R$. Also, $R$ is
      closed under inverse because given $f\in R$, let $-f$ be the function
      that sends each $a\in A$ to $-f(a)$; $-f$ can be easily verified to
      be a group homomorphism because $f$ is one. Finally, $R$ inherits the
      abelian property from the abelian property of $A$ as an additive
      group. \\

      $R$ is also closed under multiplication because the composition of
      homomorphisms of a group remains a homomorphism. Associativity of
      multiplication follows from the associativity of the composition of
      functions. To show the distributive laws, let $a_1,a_2\in A$ and
      $f,g,h\in R$. Then
      \begin{align*}
        ((f+g)h)(a) &= (f+g)h(a) \\
                    &= f(h(a)) + g(h(a)) \\
                    &= fh(a) + gh(a) \\
                    &= (fh+gh)(a). \\
      \end{align*}
      Similarly, we can show that $(h(f+g))(a)=(hf+hg)(a)$. Hence the
      distributive laws hold. \\

      Finally, $R$ contains the identity map, which will function as the
      identity in $R$. \\

      Let $f$ be a group automorphism of $A$, and let $f^{-1}$ be the
      function that sends $a\in A$ to $-f(a)$. Then $f^{-1}$ is also a
      group automorphism of $A$, and $f\circ f^{-1}=f^{-1}\circ
      f=\text{id}$, hence $f^{-1}$ is the multiplicative inverse of $f$ in
      the ring $R$. Conversely, if $f\in R$ has a multiplicative inverse
      $g$, then $f\circ g=g\circ f=\text{id}$ as an additive group
      homomorphism of $A$, which implies that $f$ is a group automorphism
      of $A$.
    \end{proof}
\end{enumerate}
\end{document}
