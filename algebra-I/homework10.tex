\documentclass{article}
\usepackage[left=3cm,right=3cm,top=3cm,bottom=3cm]{geometry}
\usepackage{amsmath,amssymb,amsthm,pgfplots,tikz}
\usepackage[inline]{enumitem}
\usepackage{color}
\setlength{\parindent}{0mm} %So that we do not indent on new paragraphs
\newcommand{\TODO}[1]{\textcolor{red}{TODO: #1}}

\begin{document}
\title{Graduate Algebra I: Homework 10}
\author{Li Ling Ko\\ lko@nd.edu}
\date{\today}
\maketitle

Let $R$ be a ring with identity $1\neq0$. \\

\textbf{Section 7.6 Question 4:} Prove that if $R$ and $S$ are nonzero
  rings then $R\times S$ is never a field.
  \begin{proof}
    $R\times S$ has non-zero divisors $(0,1_S)$ and $(1_R,0)$ whose
    product equals 0. Thus $R\times S$ cannot be an integral domain and
    therefore cannot be a field.
  \end{proof}

\textbf{Section 7.6 Question 5:} Let $n_1,\ldots,n_k$ be integers which are
  relative prime in pairs: $(n_i,n_j)=1$ for all $i\neq j$.
  \begin{enumerate}[label={\bf(\alph*)}]
    \item Show that the Chinese Remainder Theorem implies that for any
      $a_1,\ldots,a_k\in\mathbb{Z}$ there is a solution $x\in\mathbb{Z}$ to
      the simultaneous congruences
      \[\begin{array}{cccc}
        x\equiv a_1\mod{n_1}, &
        x\equiv a_2\mod{n_2}, &
        \ldots, &
        x\equiv a_k\mod{n_k} \\
      \end{array}\]
      and that the solution $x$ is unique $\mod{n}=n_1n_2,\ldots,n_k$.

      \begin{proof}
        Consider the ring homomorphism $\varphi:\mathbb{Z}\rightarrow
        \mathbb{Z}_{n_1}^\times \times\ldots \mathbb{Z}_{n_k}^\times$,
        $m\mapsto (\bar{m})$. Since each $n_i\neq n_j\in\mathbb{Z}$ are
        relatively prime, the ideals $\langle n_i\rangle$, $\langle
        n_j\rangle$ are co-maximal. Thus from the Chinese Remainder
        Theorem, $\varphi$ is surjective with kernel $\langle n_1\rangle
        \cap\ldots \cap\langle n_k\rangle =\langle n\rangle$. By
        subjectivity, $a=\varphi^{-1}((a_1,\ldots,a_k))$ exists, and will
        satisfy the given congruences. Then from the first isomorphism
        theorem, the solution $a$ of the congruences is unique modulo
        $n$.
      \end{proof}

    \item Let $n_i'=n/n_i$ be the quotient of $n$ by $n_i$, which is
      relatively prime to $n_i$ by assumption. Let $t_i$ be the inverse of
      $n_i'\mod{n_i}$. Prove that the solution $x$ in (a) is given by
      \[x=a_1t_1n_1'+\ldots+a_kt_kn_k'\mod{n}.\]
      Note that the elements $t_i$ can be quickly found by the Euclidean
      Algorithm as described in Section 2 of the Preliminaries chapter
      (writing $an_i+bn_i'=(n_i,n_i')=1$ gives $t_i=b$) and that these then
      quickly give the solutions to the system of congruences above for any
      choice of $a_1,\ldots,a_k$.

      \begin{proof}
        We show that the given $x$ is a solution to the set of congruences.
        Given any $i\in\{1,\ldots,k\}$, we want to show that $x\equiv
        a_i\mod{n_i}$. Now for each $j\in\{1,\ldots,k\}$ with $j\neq i$, we
        have $n_j|n_i'$, and so $a_jt_jn_j'\equiv0\mod{n_i}$. Therefore
        $x\equiv a_it_in_i'\mod{n_i}$. Then since
        $t_in_i'\equiv1\mod{n_i}$, we have $x\equiv a_it_in_i'\equiv
        a_i\mod{n_i}$, as required.
      \end{proof}

    \item Solve the simultaneous system of congruences
      \[x\equiv1\mod{8},\;\; x\equiv2\mod{25},\;\; x\equiv3\mod{81}\]
      and the simultaneous system
      \[y\equiv5\mod{8},\;\; y\equiv12\mod{25},\;\; y\equiv47\mod{81}.\]

      \begin{proof}
        Note that $n_1=8$, $n_2=25$, and $n_3=81$ are pairwise relatively
        prime, hence we can use the formula given in part (b) above and
        also the Euclidean algorithm to get the $t_i$'s and solve the
        system of congruences. We have
        \[\begin{array}{lrrl}
          n_1'= &25\cdot81\equiv &1  &\mod{8}, \\
          n_2'= &8\cdot81\equiv  &23 &\mod{25}, \\
          n_3'= &8\cdot25\equiv  &38 &\mod{81}. \\
        \end{array}\]
        Then from the Euclidean algorithm, we get
        \[\begin{array}{lrll}
          t_1\equiv &1  &\mod{8},  &(\because n_1'\equiv1\mod{8}) \\
          t_2\equiv &12 &\mod{25}, &(\because n_2'\equiv-2\mod{25}\;
            \text{and}\; 1\cdot25-12\cdot(2)=1) \\
          t_3\equiv &32 &\mod{81}. &(\because -15\cdot81+32\cdot(81)=1). \\
        \end{array}\]
        Thus we have
        \[\begin{array}{llcrcrcrr}
          x\equiv &1\cdot1\cdot25\cdot81 &+ &2\cdot23\cdot8\cdot81
            &+ &3\cdot38\cdot8\cdot25 &\equiv &4377 &\mod{8\cdot25\cdot81},
            \\
          y\equiv &5\cdot1\cdot25\cdot81 &+ &12\cdot23\cdot8\cdot81
            &+ &47\cdot38\cdot8\cdot25 &\equiv &15437
            &\mod{8\cdot25\cdot81}. \\
        \end{array}\]
      \end{proof}
  \end{enumerate}
\end{document}
