\documentclass{article}
\usepackage[left=3cm,right=3cm,top=3cm,bottom=3cm]{geometry}
\usepackage{amsmath,amssymb,amsthm,pgfplots,tikz}
\usepackage[inline]{enumitem}
\usepackage{color}
\setlength{\parindent}{0mm} %So that we do not indent on new paragraphs
\newcommand{\TODO}[1]{\textcolor{red}{TODO: #1}}

\begin{document}
\title{Graduate Algebra I: Homework 10}
\author{Li Ling Ko\\ lko@nd.edu}
\date{\today}
\maketitle

Let $R$ be a ring with identity $1\neq0$. \\

\textbf{Section 7.6 Question 4:} Prove that if $R$ and $S$ are nonzero
  rings then $R\times S$ is never a field.
  \begin{proof}
    $R\times S$ has non-zero divisors $(0,1_S)$ and $(1_R,0)$ whose
    product equals 0. Thus $R\times S$ cannot be an integral domain and
    therefore cannot be a field.
  \end{proof}

\textbf{Section 7.6 Question 5:} Let $n_1,\ldots,n_k$ be integers which are
  relative prime in pairs: $(n_i,n_j)=1$ for all $i\neq j$.
  \begin{enumerate}[label={\bf(\alph*)}]
    \item Show that the Chinese Remainder Theorem implies that for any
      $a_1,\ldots,a_k\in\mathbb{Z}$ there is a solution $x\in\mathbb{Z}$ to
      the simultaneous congruences
      \[\begin{array}{cccc}
        x\equiv a_1\mod{n_1}, &
        x\equiv a_2\mod{n_2}, &
        \ldots, &
        x\equiv a_k\mod{n_k} \\
      \end{array}\]
      and that the solution $x$ is unique $\mod{n}=n_1n_2,\ldots,n_k$.

      \begin{proof}
        Consider the ring homomorphism $\varphi:\mathbb{Z}\rightarrow
        \mathbb{Z}_{n_1}^\times \times\ldots \mathbb{Z}_{n_k}^\times$,
        $m\mapsto (\bar{m})$. Since each $n_i\neq n_j\in\mathbb{Z}$ are
        relatively prime, the ideals $\langle n_i\rangle$, $\langle
        n_j\rangle$ are co-maximal. Thus from the Chinese Remainder
        Theorem, $\varphi$ is surjective with kernel $\langle n_1\rangle
        \cap\ldots \cap\langle n_k\rangle =\langle n\rangle$. By
        subjectivity, $a=\varphi^{-1}((a_1,\ldots,a_k))$ exists, and will
        satisfy the given congruences. Then from the first isomorphism
        theorem, the solution $a$ of the congruences is unique modulo
        $n$.
      \end{proof}

    \item Let $n_i'=n/n_i$ be the quotient of $n$ by $n_i$, which is
      relatively prime to $n_i$ by assumption. Let $t_i$ be the inverse of
      $n_i'\mod{n_i}$. Prove that the solution $x$ in (a) is given by
      \[x=a_1t_1n_1'+\ldots+a_kt_kn_k'\mod{n}.\]
      Note that the elements $t_i$ can be quickly found by the Euclidean
      Algorithm as described in Section 2 of the Preliminaries chapter
      (writing $an_i+bn_i'=(n_i,n_i')=1$ gives $t_i=b$) and that these then
      quickly give the solutions to the system of congruences above for any
      choice of $a_1,\ldots,a_k$.

      \begin{proof}
        We show that the given $x$ is a solution to the set of congruences.
        Given any $i\in\{1,\ldots,k\}$, we want to show that $x\equiv
        a_i\mod{n_i}$. Now for each $j\in\{1,\ldots,k\}$ with $j\neq i$, we
        have $n_j|n_i'$, and so $a_jt_jn_j'\equiv0\mod{n_i}$. Therefore
        $x\equiv a_it_in_i'\mod{n_i}$. Then since
        $t_in_i'\equiv1\mod{n_i}$, we have $x\equiv a_it_in_i'\equiv
        a_i\mod{n_i}$, as required.
      \end{proof}

    \item Solve the simultaneous system of congruences
      \[x\equiv1\mod{8},\;\; x\equiv2\mod{25},\;\; x\equiv3\mod{81}\]
      and the simultaneous system
      \[y\equiv5\mod{8},\;\; y\equiv12\mod{25},\;\; y\equiv47\mod{81}.\]

      \begin{proof}
        Note that $n_1=8$, $n_2=25$, and $n_3=81$ are pairwise relatively
        prime, hence we can use the formula given in part (b) above and
        also the Euclidean algorithm to get the $t_i$'s and solve the
        system of congruences. We have
        \[\begin{array}{lrrl}
          n_1'= &25\cdot81\equiv &1  &\mod{8}, \\
          n_2'= &8\cdot81\equiv  &23 &\mod{25}, \\
          n_3'= &8\cdot25\equiv  &38 &\mod{81}. \\
        \end{array}\]
        Then from the Euclidean algorithm, we get
        \[\begin{array}{lrll}
          t_1\equiv &1  &\mod{8},  &(\because n_1'\equiv1\mod{8}) \\
          t_2\equiv &12 &\mod{25}, &(\because n_2'\equiv-2\mod{25}\;
            \text{and}\; 1\cdot25-12\cdot(2)=1) \\
          t_3\equiv &32 &\mod{81}. &(\because -15\cdot81+32\cdot(81)=1). \\
        \end{array}\]
        Thus we have
        \[\begin{array}{llcrcrcrr}
          x\equiv &1\cdot1\cdot25\cdot81 &+ &2\cdot23\cdot8\cdot81
            &+ &3\cdot38\cdot8\cdot25 &\equiv &4377 &\mod{8\cdot25\cdot81},
            \\
          y\equiv &5\cdot1\cdot25\cdot81 &+ &12\cdot23\cdot8\cdot81
            &+ &47\cdot38\cdot8\cdot25 &\equiv &15437
            &\mod{8\cdot25\cdot81}. \\
        \end{array}\]
      \end{proof}
  \end{enumerate}

\textbf{Section 8.1 Question 1c:} Determine the greatest common divisor $d$
  of $a=11391$ and $b=5673$ and write $d$ as a linear combination $ax+by$
  of $a$ and $b$.
  \begin{proof}
    We apply Euclidean's algorithm to find $x$, $y$, and $d$:
    \[\begin{array}{rcrcrcr}
      11391 &= &2   &\cdot &5673 &+ &45 \\
      5673  &= &126 &\cdot &45   &+ &3 \\
      45    &= &15  &\cdot &3    &+ &0 \\
    \end{array}\]
    Thus $d=\text{gcd}(11391,5673)=3$. To determine $x$ and $y$, we
    ``reverse'' the Euclidean algorithm as follows:
    \begin{align*}
      3 &=5673-126\cdot45 \\
        &=5673-126\cdot(11391-2\cdot5673) \\
        &=-126\cdot11391+(1+126\cdot2)\cdot5673 \\
        &=-126\cdot11391+253\cdot5673 \\
    \end{align*}
    Thus $x=-126$ and $y=253$.
  \end{proof}

\textbf{Section 8.1 Question 2c:} Show that $a=1891$ is relatively prime to
  $n=3797$ and determine the inverse of $a\mod{n}$.
  \begin{proof}
    To check for relative primeness, we apply Euclidean's algorithm to
    verify that $d=\text{gcd}(a,n)=1$:
    \[\begin{array}{rcrcrcr}
      3797 &= &2   &\cdot &1891 &+ &15 \\
      1891 &= &126 &\cdot &15   &+ &1 \\
      15   &= &15  &\cdot &1    &+ &0 \\
    \end{array}\]
    Thus $d=\text{gcd}(3797,1891)=1$, which implies that $a$ and $n$ are
    relatively prime. To determine the inverse of $a\mod{n}$, we first
    ``reverse'' the Euclidean algorithm to solve for $x$ and $y$ in the
    equation $d=ax+ny$:
    \begin{align*}
      1 &=1891-126\cdot15 \\
        &=1891-126\cdot(3797-2\cdot1891) \\
        &=-126\cdot3797+(1+126\cdot2)\cdot1891 \\
        &=-126\cdot3797+253\cdot1891 \\
    \end{align*}
    Rearranging, we get $253\cdot1891=126\cdot3797+1$, and therefore the
    inverse of $1891\mod{3797}$ is $253$.
  \end{proof}

\textbf{Section 8.1 Question 5b:} Determine all integer solutions of
  $17x+29y=31$.
  \begin{proof}
    Let $d=\text{gcd}(17,29)$. Note that if $x_0,y_0$ and $x,y$ are
    both integer solutions of the equation, then $17(x_0-x)=29(y-y_0)$,
    which gives $x=x_0-29a/d$ for some $a\in\mathbb{Z}$, and
    $y=y_0+17a/d$. Conversely, if $x_0,y_0$ are integer solutions and
    $x=x_0-29a/d$ and $y=y_0+17a/d$ for some $a\in\mathbb{Z}$, then $x,y$
    will also be integer solutions. Thus, fixing an integer solution
    $x_0,y_0$ of the given equation, the integer solutions are exactly
    $(x_0-29a/d,y_0+17a/d)$ for some $a\in\mathbb{Z}$. \\

    Now since $d=(17,29)=1$, so there are integers $x_0$ and $y_0$
    such that $17x_0'+29y_0'=\text{gcd}(17,29)=1$. Multiplying throughout by
    31 would give $17(31x_0')+29(31y_0')=31$. Thus integer solutions
    $x_0=31x_0'$, $y_0=31y_0'$ exist. We apply Euclidean's algorithm to
    find $x_0',y_0'$ such that $17x_0'+29y_0'=1$:
    \[\begin{array}{rcrcrcr}
      29 &= &1 &\cdot &17 &+ &12 \\
      17 &= &1 &\cdot &12 &+ &5 \\
      12 &= &2 &\cdot &5  &+ &2 \\
       5 &= &2 &\cdot &2  &+ &1 \\
       2 &= &2 &\cdot &1  &+ &0 \\
    \end{array}\]

    We ``reverse'' the Euclidean algorithm to solve for $x_0'$ and $y_0'$:
    \begin{align*}
      1 &=5-2\cdot2 \\
        &=5-2\cdot(12-2\cdot5) \\
        &=-2\cdot12+5\cdot5 \\
        &=-2\cdot12+5\cdot(17-12) \\
        &=-7\cdot12+5\cdot17 \\
        &=-7\cdot(29-17)+5\cdot17 \\
        &=-7\cdot29+12\cdot17 \\
    \end{align*}
    Thus $x_0'=12$ and $y_0'=-7$. Then $x_0=31\cdot x_0'=31\cdot12=372$, and
    $y_0=31\cdot y_0'=-31\cdot7=-217$. So from the above argument, the
    integer solutions are exactly
    \[\{(372-29a,-217+17a):a\in\mathbb{Z}\}.\]
  \end{proof}

\textbf{Section 8.1 Question 7:} Find a generator for the ideal
  $(85,1+13i)$ in $\mathbb{Z}[i]$, i.e., a greatest common divisor for 85
  and $1+13i$ by the Euclidean Algorithm. Do the same for the ideal
  $(47-13i,53+56i)$.

  \begin{proof}
    Note that the textbook has shown that the Gaussian integers
    $\mathbb{Z}[i]$ is an Euclidean Domain with respect to the norm
    $N(a+bi)=a^2+b^2$. Thus $\mathbb{Z}[i]$ is a PID, and so the notion of
    greatest common divisor is well-defined. \\

    Now $N(\text{gcd}(85,1+13i)) =N(\text{gcd}(N(85),N(1+13i)))
    =\text{gcd}(85,170)=85$. Thus we are looking for $a,b\in\mathbb{Z}$
    such that $a^2+b^2=85$. Since $\sqrt{85}<10$, there are at most 5
    integer solutions we need to test to satisfy $a^2+b^2=85$. We check
    that the possible solutions are $a=\pm2,b=\pm9$ and $a=\pm6,b=\pm7$. We
    check that of these possibilities, only $\pm6\mp7i$ divides $1+13i$ in
    $\mathbb{Z}[i]$. Thus a GCD of 85 and $1+13i$ is $6-7i$. \\

    For the second part of the question, we apply the Euclidean algorithm
    with the given norm to find $\text{gcd}(47-13i,53+56i)$. At each step
    of the algorithm, to solve for $q,r\in\mathbb{Z}[i]$ in the equation
    $b=qa+r$ given $a,b\mathbb{Z}[i]$, we apply the norm to get
    $N(b)=N(q)N(a)+N(r)$ and find $q,r\in\mathbb{Z}[i]$ such that the norm
    of $q$ is $N(q)$ such that $N(r)<N(a)$:
    \[\begin{array}{rcrcrcr}
      53+56i &= &(1+i)   &\cdot &(47-13i) &+ &(-7+22i) \\
      47+13i &= &(-1-2i) &\cdot &(-7-22i) &+ &(-4-5i) \\
      -7-22i &= &(-2-3i) &\cdot &(-4-5i)  &+ &0 \\
    \end{array}\]
    Thus a GCD of $53+56i$ and $47+13i$ is $4+5i$. \\
  \end{proof}

\textbf{Section 8.1 Question 9:} Prove that the ring of integers
  $\mathcal{O}$ in the quadratic integer ring $\mathbb{Q}(\sqrt{2})$ is a
  Euclidean Domain with respect to the norm given by the absolute value of
  the field norm $N$ in Section 7.1.

  \begin{proof}
    The norm was given by $N(a+b\sqrt{2})=|a^2-2b^2|$. Given
    $a=a_1+a_2\sqrt{2}$ and $b=b_1+b_2\sqrt{2}$ in $\mathbb{Z}(\sqrt{2})$,
    we wish to show that there exists $q,r\in\mathbb{Z}(\sqrt{2})$ such
    that $a=bq+r$ and $N(r)<N(b)$. Now $a/b=c=c_1+c_2\sqrt{2}$, where
    \[c_1=\frac{a_1b_1-2a_2b_2}{b_1^2-2b_2^2}\; \text{and}\;
    c_2=\frac{a_2b_1-a_1b_2}{b_1^2-2b_2^2}.\] Let $q_1$ be the integer
    closest to $c_1$ and $q_2$ the integer closest to $c_2$, and let
    $q=q_1+q_2\sqrt{2}\in\mathbb{Z}(\sqrt{2})$. Then
    $r=r_1+r_2\sqrt{2}=a-bq$, and $|c_1-q_1|\leq0.5$, and
    $|c_2-q_2|\leq0.5$. We need to show that $N(r)<N(b)$. We have
    \begin{align*}
      N(r) &= N(a-bq) \\
        &=N(b)N(a/b-q) &(\text{for}\; x,y\in\mathbb{Q}[\sqrt{2}],\;
          \text{we have}\; N(xy)=N(x)N(y)) \\
        &=N(b)N(c-q) \\
        &=N(b)N\left((c_1-q_1)+(c_2-q_2)\sqrt{2}\right) \\
        &=N(b)\; \left|(c_1-q_1)^2-2(c_2-q_2)^2\right| \\
        &\leq N(b)\; \left|0-2(0.5)^2\right| &(\because
          |c_1-q_1|,|c_2-q_2|\leq0.5) \\
        &=0.5N(b) \\
        &<N(b). \\
    \end{align*}
  \end{proof}

\textbf{Section 8.2 Question 3:} Prove that a quotient of a PID by a prime
  ideal is again a PID.
  \begin{proof}
    Let $R$ be a PID and $P$ be a prime ideal of $R$.
    By the lattice isomorphism for rings, every ideal of $R/P$ has the form
    $I/P$ for some ideal $I$ of $R$. We wish to show that $I/P$ is a
    principle in $R/P$. Now since $R$ is a PID, $I=\langle a\rangle$ for
    some $a\in R$. Then $I/P$ will be generated by the single element
    $a+P$, and is therefore principle. Since $I/P$ was an arbitrary ideal of
    $R/P$, we have $R/P$ is a PID.
  \end{proof}

\textbf{Section 8.2 Question 5:} Let $R$ be the quadratic integer ring
  $\mathbb{Z}[\sqrt{-5}]$. Define the ideals $I_2=(2,1+\sqrt{-5})$,
  $I_3=(3,2+\sqrt{-5})$, $I_3'=(3,2-\sqrt{-5})$,
  \begin{enumerate}[label={\bf(\alph*)}]
    \item Prove that $I_2$, $I_3$, and $I_3'$ are non-principal ideals in
      $R$.
      \begin{proof}
        To show that a finitely generated ideal is non-principal is
        equivalent to showing that the generators do not generate one of
        their greatest common divisors. To check for divisors of elements
        in $R$, we define function $N:R\rightarrow\mathbb{N}$,
        $a+b\sqrt{-5}\mapsto a^2+5b^2$. The book as shown that given
        $x,y\in R$, we have $N(xy)=N(x)N(y)$. Thus an element $d\in R$
        divides $x\in R$ implies that $N(d)|N(x)$. We use this rule to try
        to find common divisors of the generators of the given ideals. \\

        For $I_2$, let $a+b\sqrt{-5}$ be a common divisor of $2$ and
        $1+\sqrt{-5}$. Then $N(a+b\sqrt{-5})=a^2+5b^2$ divides that
        $N(2)=4$ and $N(1+\sqrt{-5})=6$, which means $a^2+5b^2$ equals 1 or
        2. Now since $a^2+5b^2\leq5$, we must have $b=0$. Then the only
        integer solution $a^2+5b^2$ would be $a=\pm1$ and $b=0$. It remains
        to show that there are no solutions in $R$ for
        $2x+(1+\sqrt{-5})y=1$. Multiplying by $(1-\sqrt{-5})$ gives
        $2(1-\sqrt{-5}+3y)=1-\sqrt{-5}$, which has no solutions in $R$
        because $2\not|\; (1-\sqrt{-5})$ in $R$. \\

        For $I_3$, let $a+b\sqrt{-5}$ be a common divisor of $3$ and
        $2+\sqrt{-5}$. Then $N(a+b\sqrt{-5})=a^2+5b^2$ divides that
        $N(3)=9$ and $N(2+\sqrt{-5})=9$, which means $a^2+5b^2$ equals 1,
        3, or 9. On one hand, if $a^2+5b^2\leq5$, we must have $b=0$, so
        the only integer solution would be $a=\pm1$ and $b=0$. We show
        that there are no solutions in $R$ for $3x+(2+\sqrt{-5})y=1$:
        Multiplying by $(2-\sqrt{-5})$ gives
        $3(2-\sqrt{-5}+3y)=2-\sqrt{-5}$, which has no
        solutions in $R$ because $3\not|\; (2-\sqrt{-5})$ in $R$. On the
        other hand, if $a^2+5b^2=9$, we have three cases to consider ---
        $a=3$ with $b=0$, or $a=2$ with $b=1$, or $a=2$ with $b=-1$. The
        first case is not possible because $3\not|\; (2-\sqrt{-5})$ in $R$.
        The second and third cases are also not possible because
        $2\pm\sqrt{-5}\not|\; 3$ in $R$. \\

        The argument for showing $I_3'$ is non-principal is analogous to
        the argument for $I_3$.
      \end{proof}

    \item Prove that the product of two non-principal ideals can be
      principal by showing that $I_2^2$ is the principal ideal generated by
      2, i.e., $I_2^2=\langle2\rangle$.
      \begin{proof}
        $\subseteq$: Chasing definitions, we have \[I_2^2=\langle 2\cdot2,
        2(1+\sqrt{-5}), (1+\sqrt{-5})^2 \rangle =\langle4,
        2(1+\sqrt{-5}), 2(-2+\sqrt{-5}) \rangle.\] Since
        these three generators of $I_2^2$ are multiples of 2, we have
        $I_2^2\subseteq\langle2\rangle$. \\

        $\supseteq$: We need to show that elements $4$, $2(1+\sqrt{-5})$,
        and $2(-2+\sqrt{-5})$ generate $2$. This is true since
        \[2=-2(1+\sqrt{-5})-2(-2+\sqrt{-5}).\]
      \end{proof}

    \item Prove similarly that $I_2I_3=\langle1-\sqrt{-5}\rangle$ and
      $I_2I_3'=\langle1+\sqrt{-5}\rangle$ are principal. Conclude that the
      principal ideal $\langle6\rangle$ is the product of 4 ideals:
      $\langle6\rangle=I_2^2I_3I_3'$.

      \begin{proof}
        Chasing definitions, we have
        \begin{align*}
          I_2I_3 &=\langle 2\cdot3, 2(2+\sqrt{-5}), 3(1+\sqrt{-5}),
            (1+\sqrt{-5})(2+\sqrt{-5}) \rangle
            \\
            &=\langle6, 4+2\sqrt{-5}, 3+3\sqrt{-5}, 3-3\sqrt{-5} \rangle, \\
          I_2I_3' &=\langle 2\cdot3, 2(2-\sqrt{-5}), 3(1+\sqrt{-5}),
            (1+\sqrt{-5})(2-\sqrt{-5}) \rangle
            \\
            &=\langle6, 4-2\sqrt{-5}, 3+3\sqrt{-5}, 7+\sqrt{-5} \rangle. \\
        \end{align*}

        To verify $I_2I_3\subseteq\langle1-\sqrt{-5}\rangle$, we check that
        each of the four generators of $I_2I_3$ calculated above are
        multiples of $1-\sqrt{-5}$ in $R$: The forth generator is clearly a
        multiple. As for the remaining three generators, we check that each
        generator is multiplied with $1/(1-\sqrt{-5})=(1+\sqrt{-5})/6$, the
        result is in $R$. This is clearly true for generator 6. Similarly,
        $2(2+\sqrt{-5})(1+\sqrt{-5})/6=(-3+3\sqrt{-5})/3\in R$, and
        $3(1+\sqrt{-5})(1+\sqrt{-5})/6=(-4+2\sqrt{-5})/2\in R$. \\

        To verify that $I_2I_3\supseteq\langle1-\sqrt{-5}\rangle$, we note
        that $1-\sqrt{-5}=(4+2\sqrt{-5})+(3-3\sqrt{-5})-6$. \\

        To verify $I_2I_3'\subseteq\langle1-\sqrt{-5}\rangle$, we check that
        each of the four generators of $I_2I_3'$ calculated above are
        multiples of $1+\sqrt{-5}$ in $R$: The third generator is clearly a
        multiple. As for the remaining three generators, we check that each
        generator is multiplied with $1/(1+\sqrt{-5})=(1-\sqrt{-5})/6$, the
        result is in $R$. This is clearly true for generator 6. Similarly,
        $2(2-\sqrt{-5})(1-\sqrt{-5})/6=(-3-3\sqrt{-5})/3\in R$, and
        $3(1+\sqrt{-5})(1-\sqrt{-5})/6=(1+5)/2\in R$. \\

        To verify that $I_2I_3'\supseteq\langle1+\sqrt{-5}\rangle$, we note
        that $1+\sqrt{-5}=(4-2\sqrt{-5})+(3+3\sqrt{-5})-6$. \\

        Thus $I_2I_3=\langle1-\sqrt{-5}\rangle$ and
        $I_2I_3'=\langle1+\sqrt{-5}\rangle$. Then
        \begin{align*}
          I_2^2I_3I_3' &=(I_2I_3)(I_2I_3') \\
            &=\langle1-\sqrt{-5}\rangle \langle1+\sqrt{-5}\rangle \\
            &=\langle(1-\sqrt{-5})(1+\sqrt{-5})\rangle \\
            &=\langle6\rangle. \\
        \end{align*}
      \end{proof}
  \end{enumerate}

\textbf{Section 8.3 Question 2:} Let $a$ and $b$ be non-zero elements of
  the Unique Factorization Domain $R$. Prove that $a$ and $b$ have a least
  common multiple and describe it in terms of the prime factorizations of
  $a$ and $b$ in the same fashion that Proposition 13 describes their
  greatest common divisor.

  \begin{proof}
    Let the prime factorization of $a$ and $b$ be given by
    \begin{align*}
      a &= up_1^{e_1}\cdots p_n^{e_n}, \\
      b &= vp_1^{f_1}\cdots p_n^{f_n},
    \end{align*}
    where $u$ and $v$ are units, the $p_i$'s are unique primes, and the
    exponents $e_i$ and $f_i$ are greater or equal 0. We claim that
    \begin{align*}
      c &= p_1^{\max(e_1,f_1)}\cdots p_n^{\max(e_n,f_n)}
    \end{align*}
    is a least common multiple of $a$ and $b$. $c$ clearly divides $a$ and
    $b$. Let $c'$ be a multiple of $a$ and $b$. We want to show that
    $c|c'$. \\

    We first show that that $p_i^{\max(e_i,f_i)}$ divides $c'$ for every
    $i\in\{1,\ldots,n\}$. Assume without loss of generality that $e_i\geq
    f_i$. Then $a$ divides $c'$ implies that $c'=ax$ for some $x\in R$. So
    we can write $c'=up_1^{e_1}\cdots p_n^{e_n}x$, and thus
    $p_i^{\max(e_i,f_i)}=p_i^{e_i}$ divides $c'$. \\

    Let $c'=wp_1^{g_1}\cdots p_n^{g_n}q_1^{h_1}\cdots q_m^{h_m}$ be a prime
    factor decomposition of $c'$, where $q_j$ are primes distinct from the
    $p_i$'s and $w$ is a unit. We wish to show that $g_i\geq\max(e_i,f_i)$
    for each $i\in\{1,\ldots,n\}$. Since $p_i^{\max(e_i,f_i)}$ divides
    $c'$, we have 
    \begin{align*}
      wp_1^{g_1}\cdots p_n^{g_n}q_1^{h_1}\cdots q_m^{h_m}&=c' \\
        &=p_i^{\max(e_i,f_i)}y &(\text{for some}\; y\in R) \\
        &=p_i^{\max(e_i,f_i)+t} w'r_1^{s_1}\cdots r_k^{s_k}.
          &(y=w'p_i^tr_1^{s_1}\cdots r_k^{s_k}\; \text{is
          prime decomposition of}\; y) \\
    \end{align*}
    Then from uniqueness of prime decomposition, we have
    $g_i=\max(e_i,f_i)+t$, and thus $g_i\geq\max(e_i,f_i)$.
  \end{proof}

\textbf{Section 8.3 Question 6a:} Prove that the quotient ring
  $\mathbb{Z}[i]/(1+i)$ is a field of order 2.
  \begin{proof}
    We first show that the quotient ring is a field. This is equivalent to
    showing that $(1+i)$ is a maximal ideal in $\mathbb{Z}[i]$.  It is
    known that the Gaussian integers $\mathbb{Z}[i]$ is an Euclidean domain
    with respect to the norm $N(a+bi)=a^2+b^2$. Since Euclidean domains are
    PIDs, and every non-zero prime ideal in a PID is a maximal ideal
    (Proposition 7, Section 8.2), it suffices to show that $(1+i)$ is a
    prime ideal in $\mathbb{Z}[i]$, or equivalently, that $1+i$ is prime in
    $\mathbb{Z}[i]$. Now since in PIDs, the prime elements are exactly the
    irreducible elements (Proposition 11, Section 8.3), it suffices to show
    that $1+i$ is an irreducible element in $\mathbb{Z}[i]$. \\

    Let $1+i=xy$, where $x,y\in\mathbb{Z}[i]$. We wish to show that either
    $x$ or $y$ is a unit. Now $2=N(1+i)=N(xy)=N(x)N(y)$, so either $N(x)=1$
    or $N(y)=1$. Assume without loss of generality that $N(x)=1$, and write
    $x=x_1+x+2i$. Then $N(x)=x_1^2+x_2^2=1$, which implies that either
    $x_1=\pm1$ with $x_2=0$, or $x_1=0$ with $x_2=\pm1$. Thus $x=\pm1$ or
    $x=\pm i$, and in all four cases, $x$ is a unit. Thus $1+i$ is
    irreducible, and therefore $\mathbb{Z}[i]/(1+i)$ is a field. \\

    Now we show that the quotient ring has order 2. Let
    $x+(1+i)\in\mathbb{Z}[i]/(1+i)$. Since $\mathbb{Z}[i]$ is a Euclidean
    domain, there exists some $q,r\in\mathbb{Z}[i]$ such that $x=q(1+i)+r$
    and $N(r)<N(1+i)=2$. Then $x+(1+i)=r+(1+i)\in\mathbb{Z}[i]/(1+i)$.
    Write $r=r_1+r_2i$. Then $N(r)=r_1^2+r_2^2\leq0$, which means $r=0$, or
    $r=\pm1$, or $r=\pm i$. Thus the quotient ring has at most five
    elements. Now $1+(1+i)=-1+(1+i)$ because $1-(-1)=2\in(1+i)$ since
    $2=(1+i)(1-i)\in\mathbb{Z}[i]$. Also, $i+(1+i)=-i+(1+i)$ because
    $i-(-i)=2i\in(1+i)$ since $2i=(1+i)(1+i)\in\mathbb{Z}[i]$. Also,
    $1+(1+i)=-i+(1+i)$ because $1-(-i)=1+i\in(1+i)$. Thus the four cosets
    $1+(1+i)$, $-1+(1+i)$, $i+(1+i)$, and $-i+(1+i)$ are equal. It remains
    to show that the coset $0+(1+i)$ is distinct from $1+(1+i)$. This is
    true because $1-0=1\not\in(1+i)$, since
    $1/(1+i)=(1-i)/2\neq\in\mathbb{Z}[i]$. Thus the quotient ring has
    exactly two elements $(1+i)$ and $1+(1+i)$.
  \end{proof}

\textbf{Section 8.3 Question 8:} Let $R$ be the quadratic integer ring
  $\mathbb{Z}[\sqrt{-5}]$ and define the ideals
  $I_2=\langle2,1+\sqrt{-5}\rangle$, $I_3=\langle3,2+\sqrt{-5}\rangle$,
  $I_3'=\langle3,2-\sqrt{-5}\rangle$.

  \begin{enumerate}[label={\bf(\alph*)}]
    \item Prove that 2, 3, $1+\sqrt{-5}$ and $1-\sqrt{-5}$ are irreducibles
      in $R$, no two of which are associate in $R$, and that
      $6=2\cdot3=(1+\sqrt{-5})(1-\sqrt{-5})$ are two distinct
      factorizations of 6 into irreducibles in $R$.

      \begin{proof}
        Define the norm $N:R\rightarrow\mathbb{N}$ of $R$ by
        $N(a+b\sqrt{-5})=a^2+5b^2$. It is known that $N(xy)=N(x)N(y)$ for
        all $x,y\in\mathbb{Z}[\sqrt{-5}]$. Thus if $x|y$ in $R$, we must
        have $N(x)|N(y)$. We use this rule to test for divisors of elements
        in $R$. \\

        To test for associativity, we first find the units of $R$.
        Assume $xy=1$ in $R$. Then $N(x)N(y)=N(1)=1$, so $N(x)=N(y)=1$.
        Write $x=x_1+x_2\sqrt{-5}$. Then $N(x)=x_1^2+5x_2^2=1$, which
        implies that $x=\pm1$. Thus the units of $R$ are only $1$ or $-1$.
        Thus the only associates of a given element $x$ in $R$ are $\pm x$.
        Therefore no pair of elements in the given four are associates in
        $R$.

        Let $2=xy$ in $R$. Then $N(x)$ and $N(y)$ divides $N(2)=4$, and so
        $N(x)$ must be 1, 2, or 4. We need to show that either $x$ or $y$
        is a unit in $R$. If $N(x)=1$ or 2, then $x=\pm1$, which is a unit.
        On the other hand, if $N(x)=4$, then $x=\pm2$, which implies
        $y=\pm1$, which is a unit. Thus 2 is irreducible in $R$. \\

        Let $3=xy$ in $R$. Then $N(x)$ and $N(y)$ divides $N(3)=9$, and so
        $N(x)$ must be 1, 3, or 9. We need to show that either $x$ or $y$
        is a unit in $R$. If $N(x)=1$ or 3, then $x=\pm1$, which is a unit.
        On the other hand, if $N(x)=9$, then $x=\pm3$ or $x=2\pm\sqrt{-5}$.
        $x=\pm3$ would give $y=\pm1$, which is a unit. So assume
        $x=2\pm\sqrt{-5}$. But such $x$ will not divide 3 in $R$. Thus 3 is
        irreducible in $R$. \\

        Let $1+\sqrt{-5}=xy$ in $R$. Then $N(x)$ and $N(y)$ divides
        $N(1+\sqrt{-5})=6$, and so $N(x)$ must be 1, 2, 3, or 6. We need to
        show that either $x$ or $y$ is a unit in $R$. If $N(x)=1$, 2, or 3,
        then $x=\pm1$, which is a unit. On the other hand, if $N(x)=6$,
        then $x=\pm(1\pm\sqrt{-5})$. $x=\pm(1+\sqrt{-5})$ would give
        $y=\mp1$, which is a unit. So assume $x=\pm(1-\sqrt{-5})$. But such
        $x$ will not divide $1+\sqrt{-5}$ in $R$. Thus $1+\sqrt{-5}$ is
        irreducible in $R$. \\

        The argument for testing that $1-\sqrt{-5}$ is irreducible in $R$
        is analogous with the argument for $1+\sqrt{-5}$. \\

        Since the four given elements are irreducible in $R$ and
        $6=2\cdot3=(1+\sqrt{-5})(1-\sqrt{-5})$, the given factorizations
        are distinct factorizations of 6 into irreducbles in $R$.
      \end{proof}
  \end{enumerate}
\end{document}
