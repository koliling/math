\documentclass{article}
\usepackage[left=3cm,right=3cm,top=3cm,bottom=3cm]{geometry}
\usepackage{amsmath,amssymb,amsthm,pgfplots,tikz}
\usepackage[inline]{enumitem}
\usepackage{color}
\setlength{\parindent}{0mm} %So that we do not indent on new paragraphs
\newcommand{\TODO}[1]{\textcolor{red}{TODO: #1}}

\begin{document}
\title{Graduate Algebra I: Homework 10}
\author{Li Ling Ko\\ lko@nd.edu}
\date{\today}
\maketitle

Let $R$ be a ring with identity $1\neq0$. \\

\textbf{Section 7.6 Question 4:} Prove that if $R$ and $S$ are nonzero
  rings then $R\times S$ is never a field.
  \begin{proof}
    $R\times S$ has non-zero divisors $(0,1_S)$ and $(1_R,0)$ whose
    product equals 0. Thus $R\times S$ cannot be an integral domain and
    therefore cannot be a field.
  \end{proof}

\textbf{Section 7.6 Question 5:} Let $n_1,\ldots,n_k$ be integers which are
  relative prime in pairs: $(n_i,n_j)=1$ for all $i\neq j$.
  \begin{enumerate}[label={\bf(\alph*)}]
    \item Show that the Chinese Remainder Theorem implies that for any
      $a_1,\ldots,a_k\in\mathbb{Z}$ there is a solution $x\in\mathbb{Z}$ to
      the simultaneous congruences
      \[\begin{array}{cccc}
        x\equiv a_1\mod{n_1}, &
        x\equiv a_2\mod{n_2}, &
        \ldots, &
        x\equiv a_k\mod{n_k} \\
      \end{array}\]
      and that the solution $x$ is unique $\mod{n}=n_1n_2,\ldots,n_k$.

      \begin{proof}
        Consider the ring homomorphism $\varphi:\mathbb{Z}\rightarrow
        \mathbb{Z}_{n_1}^\times \times\ldots \mathbb{Z}_{n_k}^\times$,
        $m\mapsto (\bar{m})$. Since each $n_i\neq n_j\in\mathbb{Z}$ are
        relatively prime, the ideals $\langle n_i\rangle$, $\langle
        n_j\rangle$ are co-maximal. Thus from the Chinese Remainder
        Theorem, $\varphi$ is surjective with kernel $\langle n_1\rangle
        \cap\ldots \cap\langle n_k\rangle =\langle n\rangle$. By
        subjectivity, $a=\varphi^{-1}((a_1,\ldots,a_k))$ exists, and will
        satisfy the given congruences. Then from the first isomorphism
        theorem, the solution $a$ of the congruences is unique modulo
        $n$.
      \end{proof}

    \item Let $n_i'=n/n_i$ be the quotient of $n$ by $n_i$, which is
      relatively prime to $n_i$ by assumption. Let $t_i$ be the inverse of
      $n_i'\mod{n_i}$. Prove that the solution $x$ in (a) is given by
      \[x=a_1t_1n_1'+\ldots+a_kt_kn_k'\mod{n}.\]
      Note that the elements $t_i$ can be quickly found by the Euclidean
      Algorithm as described in Section 2 of the Preliminaries chapter
      (writing $an_i+bn_i'=(n_i,n_i')=1$ gives $t_i=b$) and that these then
      quickly give the solutions to the system of congruences above for any
      choice of $a_1,\ldots,a_k$.

      \begin{proof}
        We show that the given $x$ is a solution to the set of congruences.
        Given any $i\in\{1,\ldots,k\}$, we want to show that $x\equiv
        a_i\mod{n_i}$. Now for each $j\in\{1,\ldots,k\}$ with $j\neq i$, we
        have $n_j|n_i'$, and so $a_jt_jn_j'\equiv0\mod{n_i}$. Therefore
        $x\equiv a_it_in_i'\mod{n_i}$. Then since
        $t_in_i'\equiv1\mod{n_i}$, we have $x\equiv a_it_in_i'\equiv
        a_i\mod{n_i}$, as required.
      \end{proof}

    \item Solve the simultaneous system of congruences
      \[x\equiv1\mod{8},\;\; x\equiv2\mod{25},\;\; x\equiv3\mod{81}\]
      and the simultaneous system
      \[y\equiv5\mod{8},\;\; y\equiv12\mod{25},\;\; y\equiv47\mod{81}.\]

      \begin{proof}
        Note that $n_1=8$, $n_2=25$, and $n_3=81$ are pairwise relatively
        prime, hence we can use the formula given in part (b) above and
        also the Euclidean algorithm to get the $t_i$'s and solve the
        system of congruences. We have
        \[\begin{array}{lrrl}
          n_1'= &25\cdot81\equiv &1  &\mod{8}, \\
          n_2'= &8\cdot81\equiv  &23 &\mod{25}, \\
          n_3'= &8\cdot25\equiv  &38 &\mod{81}. \\
        \end{array}\]
        Then from the Euclidean algorithm, we get
        \[\begin{array}{lrll}
          t_1\equiv &1  &\mod{8},  &(\because n_1'\equiv1\mod{8}) \\
          t_2\equiv &12 &\mod{25}, &(\because n_2'\equiv-2\mod{25}\;
            \text{and}\; 1\cdot25-12\cdot(2)=1) \\
          t_3\equiv &32 &\mod{81}. &(\because -15\cdot81+32\cdot(81)=1). \\
        \end{array}\]
        Thus we have
        \[\begin{array}{llcrcrcrr}
          x\equiv &1\cdot1\cdot25\cdot81 &+ &2\cdot23\cdot8\cdot81
            &+ &3\cdot38\cdot8\cdot25 &\equiv &4377 &\mod{8\cdot25\cdot81},
            \\
          y\equiv &5\cdot1\cdot25\cdot81 &+ &12\cdot23\cdot8\cdot81
            &+ &47\cdot38\cdot8\cdot25 &\equiv &15437
            &\mod{8\cdot25\cdot81}. \\
        \end{array}\]
      \end{proof}
  \end{enumerate}

\textbf{Section 8.1 Question 1c:} Determine the greatest common divisor $d$
  of $a=11391$ and $b=5673$ and write $d$ as a linear combination $ax+by$
  of $a$ and $b$.
  \begin{proof}
    We apply Euclidean's algorithm to find $x$, $y$, and $d$:
    \[\begin{array}{rcrcrcr}
      11391 &= &2   &\cdot &5673 &+ &45 \\
      5673  &= &126 &\cdot &45   &+ &3 \\
      45    &= &15  &\cdot &3    &+ &0 \\
    \end{array}\]
    Thus $d=\text{gcd}(11391,5673)=3$. To determine $x$ and $y$, we
    ``reverse'' the Euclidean algorithm as follows:
    \begin{align*}
      3 &=5673-126\cdot45 \\
        &=5673-126\cdot(11391-2\cdot5673) \\
        &=-126\cdot11391+(1+126\cdot2)\cdot5673 \\
        &=-126\cdot11391+253\cdot5673 \\
    \end{align*}
    Thus $x=-126$ and $y=253$.
  \end{proof}

\textbf{Section 8.1 Question 2c:} Show that $a=1891$ is relatively prime to
  $n=3797$ and determine the inverse of $a\mod{n}$.
  \begin{proof}
    To check for relative primeness, we apply Euclidean's algorithm to
    verify that $d=\text{gcd}(a,n)=1$:
    \[\begin{array}{rcrcrcr}
      3797 &= &2   &\cdot &1891 &+ &15 \\
      1891 &= &126 &\cdot &15   &+ &1 \\
      15   &= &15  &\cdot &1    &+ &0 \\
    \end{array}\]
    Thus $d=\text{gcd}(3797,1891)=1$, which implies that $a$ and $n$ are
    relatively prime. To determine the inverse of $a\mod{n}$, we first
    ``reverse'' the Euclidean algorithm to solve for $x$ and $y$ in the
    equation $d=ax+ny$:
    \begin{align*}
      1 &=1891-126\cdot15 \\
        &=1891-126\cdot(3797-2\cdot1891) \\
        &=-126\cdot3797+(1+126\cdot2)\cdot1891 \\
        &=-126\cdot3797+253\cdot1891 \\
    \end{align*}
    Rearranging, we get $253\cdot1891=126\cdot3797+1$, and therefore the
    inverse of $1891\mod{3797}$ is $253$.
  \end{proof}

\textbf{Section 8.1 Question 5b:} Determine all integer solutions of
  $17x+29y=31$.
  \begin{proof}
    Let $d=\text{gcd}(17,29)$. Note that if $x_0,y_0$ and $x,y$ are
    both integer solutions of the equation, then $17(x_0-x)=29(y-y_0)$,
    which gives $x=x_0-29a/d$ for some $a\in\mathbb{Z}$, and
    $y=y_0+17a/d$. Conversely, if $x_0,y_0$ are integer solutions and
    $x=x_0-29a/d$ and $y=y_0+17a/d$ for some $a\in\mathbb{Z}$, then $x,y$
    will also be integer solutions. Thus, fixing an integer solution
    $x_0,y_0$ of the given equation, the integer solutions are exactly
    $(x_0-29a/d,y_0+17a/d)$ for some $a\in\mathbb{Z}$. \\

    Now since $d=(17,29)=1$, so there are integers $x_0$ and $y_0$
    such that $17x_0'+29y_0'=\text{gcd}(17,29)=1$. Multiplying throughout by
    31 would give $17(31x_0')+29(31y_0')=31$. Thus integer solutions
    $x_0=31x_0'$, $y_0=31y_0'$ exist. We apply Euclidean's algorithm to
    find $x_0',y_0'$ such that $17x_0'+29y_0'=1$:
    \[\begin{array}{rcrcrcr}
      29 &= &1 &\cdot &17 &+ &12 \\
      17 &= &1 &\cdot &12 &+ &5 \\
      12 &= &2 &\cdot &5  &+ &2 \\
       5 &= &2 &\cdot &2  &+ &1 \\
       2 &= &2 &\cdot &1  &+ &0 \\
    \end{array}\]

    We ``reverse'' the Euclidean algorithm to solve for $x_0'$ and $y_0'$:
    \begin{align*}
      1 &=5-2\cdot2 \\
        &=5-2\cdot(12-2\cdot5) \\
        &=-2\cdot12+5\cdot5 \\
        &=-2\cdot12+5\cdot(17-12) \\
        &=-7\cdot12+5\cdot17 \\
        &=-7\cdot(29-17)+5\cdot17 \\
        &=-7\cdot29+12\cdot17 \\
    \end{align*}
    Thus $x_0'=12$ and $y_0'=-7$. Then $x_0=31\cdot x_0'=31\cdot12=372$, and
    $y_0=31\cdot y_0'=-31\cdot7=-217$. So from the above argument, the
    integer solutions are exactly
    \[\{(372-29a,-217+17a):a\in\mathbb{Z}\}.\]
  \end{proof}

\textbf{Section 8.1 Question 7:} Find a generator for the ideal
  $(85,1+13i)$ in $\mathbb{Z}[i]$, i.e., a greatest common divisor for 85
  and $1+13i$ by the Euclidean Algorithm. Do the same for the ideal
  $(47-13i,53+56i)$.

  \begin{proof}
    Note that the textbook has shown that the Gaussian integers
    $\mathbb{Z}[i]$ is an Euclidean Domain with respect to the norm
    $N(a+bi)=a^2+b^2$. Thus $\mathbb{Z}[i]$ is a PID, and so the notion of
    greatest common divisor is well-defined. \\

    Now $N(\text{gcd}(85,1+13i)) =N(\text{gcd}(N(85),N(1+13i)))
    =\text{gcd}(85,170)=85$. Thus we are looking for $a,b\in\mathbb{Z}$
    such that $a^2+b^2=85$. Since $\sqrt{85}<10$, there are at most 5
    integer solutions we need to test to satisfy $a^2+b^2=85$. We check
    that the possible solutions are $a=\pm2,b=\pm9$ and $a=\pm6,b=\pm7$. We
    check that of these possibilities, only $\pm6\mp7i$ divides $1+13i$ in
    $\mathbb{Z}[i]$. Thus a GCD of 85 and $1+13i$ is $6-7i$. \\

    For the second part of the question, we apply the Euclidean algorithm
    with the given norm to find $\text{gcd}(47-13i,53+56i)$. At each step
    of the algorithm, to solve for $q,r\in\mathbb{Z}[i]$ in the equation
    $b=qa+r$ given $a,b\mathbb{Z}[i]$, we apply the norm to get
    $N(b)=N(q)N(a)+N(r)$ and find $q,r\in\mathbb{Z}[i]$ such that the norm
    of $q$ is $N(q)$ such that $N(r)<N(a)$:
    \[\begin{array}{rcrcrcr}
      53+56i &= &(1+i)   &\cdot &(47-13i) &+ &(-7+22i) \\
      47+13i &= &(-1-2i) &\cdot &(-7-22i) &+ &(-4-5i) \\
      -7-22i &= &(-2-3i) &\cdot &(-4-5i)  &+ &0 \\
    \end{array}\]
    Thus a GCD of $53+56i$ and $47+13i$ is $4+5i$. \\
  \end{proof}
\end{document}
