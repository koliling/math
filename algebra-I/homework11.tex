\documentclass{article}
\usepackage[left=3cm,right=3cm,top=3cm,bottom=3cm]{geometry}
\usepackage{amsmath,amssymb,amsthm,pgfplots,tikz}
\usepackage[inline]{enumitem}
\usepackage{color}
\setlength{\parindent}{0mm} %So that we do not indent on new paragraphs
\newcommand{\TODO}[1]{\textcolor{red}{TODO: #1}}

\begin{document}
\title{Graduate Algebra I: Homework 11}
\author{Li Ling Ko\\ lko@nd.edu}
\date{\today}
\maketitle

\it \textbf{Question 1:} For the field of 9 elements and for the field of 8
  elements you construct in Question 6 of Section 9.4 below, find a
  generator for the (cyclic) multiplicative group of non-zero elements.
  \begin{proof}
  \end{proof}

\it \textbf{Section 9.1 Question 4:} Prove that the ideals $(x)$ and
  $(x,y)$ are prime ideals in $\mathbb{Q}[x,y]$ but only the latter ideal
  is a maximal ideal.

  \begin{proof}
    Consider the map $\varphi:\mathbb{Q}[x,y]\rightarrow\mathbb{Q}[y]$
    defined by $\varphi(xf(x,y)+g(y))=g(y)$. This map is a surjective ring
    homomorphism with kernel $(x)$, and thus from the first isomorphism
    theorem of rings, $\mathbb{Q}[x,y]/(x)\cong\mathbb{Q}[y]$. Now since
    $\mathbb{Q}$ is an integral domain, $\mathbb{Q}[y]$ will also be an
    integral domain, and thus $(x)$ is a prime ideal in $\mathbb{Q}[x,y]$.
    However $(x)$ is not a maximal ideal in $\mathbb{Q}[x,y]$ because
    $\mathbb{Q}[y]$ is not a field - the polynomial $y\in\mathbb{Q}[y]$
    does not have a multiplicative inverse. \\

    Consider the map $\phi:\mathbb{Q}[x,y]\rightarrow\mathbb{Q}$
    defined by $\phi(xf(x,y)+yg(y)+c)=c$. This map is a surjective ring
    homomorphism with kernel $(x,y)$, and thus from the first isomorphism
    theorem of rings, $\mathbb{Q}[x,y]/(x,y)\cong\mathbb{Q}$. Now since
    $\mathbb{Q}$ is field, $(x,y)$ is a maximal and also prime ideal in
    $\mathbb{Q}[x,y]$.
  \end{proof}

\it \textbf{Section 9.1 Question 5:} Prove that $(x,y)$ and $(2,x,y)$ are
  prime ideals in $\mathbb{Z}[x,y]$ but only the latter ideal is a maximal
  ideal.

  \begin{proof}
    Consider the map $\phi:\mathbb{Z}[x,y]\rightarrow\mathbb{Z}$
    defined by $\phi(xf(x,y)+yg(y)+c)=c$. This map is a surjective ring
    homomorphism with kernel $(x,y)$, and thus from the first isomorphism
    theorem of rings, $\mathbb{Z}[x,y]/(x,y)\cong\mathbb{Z}$. Now since
    $\mathbb{Z}$ is an integral domain, $(x,y)$ is a prime ideal in
    $\mathbb{Z}[x,y]$. However, because $\mathbb{Z}$ is not a field,
    $(x,y)$ is not a maximal ideal in $\mathbb{Z}[x,y]$. \\

    Consider the map $\phi:\mathbb{Z}[x,y]\rightarrow\mathbb{Z}_2$
    defined by $\phi(xf(x,y)+yg(y)+c)=\bar{c}$. This map is a surjective ring
    homomorphism with kernel $(2,x,y)$, and thus from the first isomorphism
    theorem of rings, $\mathbb{Z}[x,y]/(2,x,y)\cong\mathbb{Z}_2$. Now since
    $\mathbb{Z}_2$ is a field, $(2,x,y)$ is a maximal and also prime ideal in
    $\mathbb{Z}[x,y]$.
  \end{proof}

\it \textbf{Section 9.1 Question 6:} Prove that $(x,y)$ is not a principal
  ideal in $\mathbb{Q}[x,y]$.

  \begin{proof}
    If $(x,y)$ is principal, then $(x,y)=(p(x,y))$ for some polynomial
    $p(x,y)\in\mathbb{Q}[x,y]$. Then $x\in(p(x,y))$ implies $p(x,y)|x$ in
    $\mathbb{Q}[x,y]$. Now in $\mathbb{Q}[x,y]$, the only divisors of the
    polynomial $x$ are of the form $q$ or $qx$ for non-zero rationals
    $q\in\mathbb{Q}\setminus\{0\}$, because if the degree of $x$ in
    $p(x,y)$ is larger than 1, then the degree of $x$ in $p(x,y)f(x,y)$ for
    any non-zero polynomial $f(x,y)\in\mathbb{Q}[x,y]$ will remain larger
    than 1. Thus $p(x,y)$ must be of the form $q$ or $qx$.  Similarly,
    since $y\in(p(x,y))$, $p(x,y)$ must be of a form $q$ or $qy$ for
    non-zero rationals $q\in\mathbb{Q}\setminus\{0\}$. Thus $p(x,y)$ must
    be a non-zero rational $q$. But $(q)=\mathbb{Q}[x,y]$ for non-zero
    $q\in\mathbb{Q}$, yet $(x,y)$ is strictly contained in
    $\mathbb{Q}[x,y]$ because the $\mathbb{Q}[x,y]$ contains $1$ but
    $(x,y)$ does not.
  \end{proof}

\it \textbf{Section 9.2 Question 3:} Let $F$ be a field and let $x$ be an
  indeterminate over $F$. Let $f(x)$ be a polynomial in $F[x]$. Prove that
  $F[x]/(f(x))$ is a field if and only if $f(x)$ is irreducible.

  \begin{proof}
    $\Rightarrow$: If $F[x]/(f(x))$ is a field, $(f(x))$ must be a prime
    ideal, which means $f(x)$ is prime in $F(x)$. Now  since $F(x)$ is a
    UFD, prime elements are equivalent to the irreducible elements, thus
    $f(x)$ is irreducible. \\

    $\Leftarrow$: Assume $f(x)$ is irreducible. Then since $F(x)$ is a UFD,
    irreducible elements are equivalent to prime ones, so $f(x)$ is a prime
    element. Then $(f(x))$ will be a prime ideal. Now since $F[x]$ is a PID
    (Corollary 4, Section 9.2), its prime ideals are maximal ideals
    (Proposition 7, Section 8.2), thus $(f(x))$ is a maximal ideal, which
    makes $F[x]/(f(x))$ a field.
  \end{proof}

\it \textbf{Section 9.2 Question 5:} Let $F$ be a field and let $x$ be an
  indeterminate over $F$. Exhibit all the ideals in the ring $F[x]/(p(x))$,
  where $F$ is a field and $p(x)$ is a polynomial in $F[x]$ (describe them
  in terms of the factorization of $p(x)$).

  \begin{proof}
    Since $F$ is a field, $F[x]$ is a Euclidean domain, thus $p(x)$ factors
    uniquely (up to associates) into a product of primes
    $p(x)=u\prod_{i=1}^nf_i^{\alpha_i}(x)$, where the $f_i(x)$'s are unique
    primes, $u$ is a unit (which is a unit in $R$, by Proposition 1 of
    Section 9.1), and $\alpha_i>0$. \\

    Now from the Lattice Isomorphism Theorem for rings, the ideals of
    $F[x]/(p(x))$ are exactly the ideals $I/(p(x))$, where $I$ is an ideal
    of $F[x]$ that contains $(p(x))$. Since $F[x]$ is a PID, we can write
    $I=(q(x))$ for some $q(x)\in F[x]$. Now because $(q(x))$ contains
    $(p(x))$ if and only if $q(x)$ divides $p(x)$. Therefore the ideals of
    $F[x]/(p(x))$ are exactly those of the form $(q(x))/(p(x))$ for
    divisors $q(x)$ of $p(x)$. \\

    Furthermore, for principle ideals, $(q_1(x))$ equals $(q_2(x))$ if and
    only if $q_1(x)$ and $q_2(x)$ are associates. Thus from the factor
    decomposition of $p(x)$, the ideals of $F[x]/(p(x))$ are exactly those
    of the form $(\prod_{i=1}^nf_i^{\beta_i}(x))/(p(x))$, where
    $\beta_i\leq\alpha_i$, giving a total of $\prod_{i=1}^n\alpha_i$
    unique ideals.
  \end{proof}

\it \textbf{Section 9.2 Question 6:} Describe (briefly) the ring structure
  of the following rings:
  \begin{enumerate}[label={(\alph*)}]
    \item $\mathbb{Z}[x]/(2)$
    \item $\mathbb{Z}[x]/(x)$
    \item $\mathbb{Z}[x]/(x^2)$
    \item $\mathbb{Z}[x,y]/(x^2,y^2,2)$
  \end{enumerate}
  Show that $\alpha^2=0$ or 1 for every $\alpha$ in the last ring and
  determine those elements with $\alpha^2=0$. Determine the characteristics
  of each ring.

  \begin{proof}
    \begin{enumerate}[label={(\alph*)}]
      \item $\mathbb{Z}[x]/(2)$: By Proposition 2 of Section 9.1,
        $\mathbb{Z}[x]/(2)$ is isomorphic to $\mathbb{Z}/(2)[x]$, which is
        isomorphic to $\mathbb{Z}_2[x]$ since
        $\mathbb{Z}/(2)\cong\mathbb{Z}_2$. Thus $\mathbb{Z}[x]/(2)$ has
        characteristic 2.

      \item $\mathbb{Z}[x]/(x)$: The map
        $\varphi:\mathbb{Z}[x]\rightarrow\mathbb{Z}$ defined by
        $\varphi(xf(x)+c)=c$ is a surjective ring-homomorphism with kernel
        $(x)$. Thus by the first isomorphism theorem, $\mathbb{Z}[x]/(x)$
        is isomorphic to $\mathbb{Z}$, which has characteristic 0.

      \item $\mathbb{Z}[x]/(x^2)$: We show that $\mathbb{Z}[x]/(x^2)
        =\{\overline{ax+b}:a,b\in\mathbb{Z}\}$. Since every
        $f(x)\in\mathbb{Z}[x]$ can be written in the form
        $f(x)=x^2g(x)+ax+b$ where $g(x)\in\mathbb{Z}[x]$, the given set of
        cosets contain all elements of $\mathbb{Z}[x]/(x^2)$. Also, since
        $x^2|(ax+b)$ if and only if $a=b=0$, the given set of cosets are
        distinct. Thus $\mathbb{Z}[x]/(x^2)
        =\{\overline{ax+b}:a,b\in\mathbb{Z}\}$, which has characteristic 0.

      \item $\mathbb{Z}[x,y]/(x^2,y^2,2)$: We show that
        $\mathbb{Z}[x,y]/(x^2,y^2,2)
        =\{\overline{axy+bx+cy+d}:a,b,c,d\in\mathbb{Z}_2\}$. Since every
        $f(x)\in\mathbb{Z}[x]$ can be written in the form
        \[f(x)=x^2g(x)+y^2h(x)+(2a_0+a)xy+(2b_0+b)x+(2c_0+c)y+(2d_0+d),\]
        where $g(x),h(x)\in\mathbb{Z}[x,y]$, and $a,b,c,d\in\{0,1\}$, the
        given set of cosets contain all elements of
        $\mathbb{Z}[x,y]/(x^2,y^2,2)$. Next, we show that these cosets are
        distinct. If $axy+bx+cy+d\in(x^2,y^2,2)$, since the coefficients of
        $xy$, $x$, $y$, and $1$ must divide 2, we get $a=b=c=d=0$. Thus the
        cosets are distinct. Then since $\bar{1}+\bar{1}=\bar{0}$, this
        ring has characteristic 2.
    \end{enumerate}
  \end{proof}

\it \textbf{Section 9.2 Question 9:} Determine the greatest common divisor
  of $a(x)=x^5+2x^3+x^2+x+1$ and the polynomial
  $b(x)=x^5+x^4+2x^3+2x^2+2x+1$ in $\mathbb{Q}[x]$ and write it as a linear
  combination (in $\mathbb{Q}[x]$) of $a(x)$ and $b(x)$.

  \begin{proof}
    Note that since $\mathbb{Q}$ is a field, $\mathbb{Q}$ is a Euclidean
    domain, thus the notion of greatest common divisor is well-defined. We
    perform the Euclidean algorithm to find the GCD,
    and backtrack to express the GCD in terms of $a(x)$ and $b(x)$. The
    Euclidean algorithm gives:

    \[\begin{array}{rll}
      a(x)= &b(x) &+(x^4+x^2+x) \\
      b(x)= &x(x^4+x^2+x) &+(x^3+x+1) \\
      (x^4+x^2+x)= &x(x^3+x+1) &+0 \\
    \end{array}\]

    Thus the GCD is $(x^3+x+1)$. Backtracking, we have
    \begin{align*}
      (x^3+x+1) &=b(x)-x(x^4+x^2+x) \\
      &=b(x)-x(a(x)-b(x)) \\
      &=-xa(x)+(x+1)b(x). \\
    \end{align*}
  \end{proof}

\it \textbf{Section 9.3 Question 1:} Let $R$ be an integral domain with
  quotient field $F$ and let $p(x)$ be a monic polynomial in $R[x]$. Assume
  that $p(x)=a(x)b(x)$ where $a(x)$ and $b(x)$ are monic polynomials in
  $F[x]$ of smaller degree than $p(x)$. Prove that if $a(x)\not\in R[x]$
  then $R$ is not a UFD. Deduce that $\mathbb{Z}[2\sqrt{2}]$ is not a UFD.

  \begin{proof}
    Assume that $R$ is a UFD. Then by Gauss's lemma, there are non-zero
    elements $a_0,b_0\in F$ such that $a_1(x)=a_0a(x),b_1(x)=b_0b(x)\in
    R[x]$ and $p(x)=a_1(x)b_1(x)$. Comparing leading coefficients, we must
    have $a_0b_0=1$. Also, since $a_0a(x)\in R[x]$ and $a(x)$ is monic,
    $a_0$ must lie in $R$. Then $a(x)=a_0^{-1}a_1(x)$ which will be in
    $R[x]$ since $a_0^{-1},a_1(x)\in R[x]$. \\

    Let $\mathbb{Z}[2\sqrt{2}]$ play the role of $R$ in the question. Then
    the field of fractions of $R$ will be
    $F=\mathbb{Q}[2\sqrt{2}]=\mathbb{Q}[\sqrt{2}]$, since this is a field,
    and the field of fractions of $R$ must contain both $\mathbb{Q}$ and
    $2\sqrt{2}$. Consider $p(x)=x^2-2\in R[x]$. Then $p(x)=a(x)b(x)$, where
    $a(x)=x-\sqrt{2}\in F[x]$ and $b(x)=x+\sqrt{2}\in F[x]$. However
    $a(x)\not\in R[x]$. Thus $R$ is not a UFD.
  \end{proof}

\it \textbf{Section 9.3 Question 4:} Let
  $R=\mathbb{Z}+x\mathbb{Q}[x]\subset\mathbb{Q}[x]$ be the set of
  polynomials in $x$ with rational coefficients whose constant term is an
  integer.

  \begin{enumerate}[label={(\alph*)}]
    \item Prove that $R$ is an integral domain and its units are $\pm1$.
      \begin{proof}
        $\mathbb{Q}[x]$ is a Euclidean domain because $\mathbb{Q}$ is a
        field, therefore $\mathbb{Q}[x]$ is an integral domain. Since any
        zero-divisors of $R$ is also a zero-divisor of $\mathbb{Q}[x]$, $R$
        must be an integral domain. \\

        Since $R$ is a sub-ring of $\mathbb{Q}[x]$ whose units are
        $\mathbb{Q}^+$ (Proposition 1, Section 9.1), the units of $R$ must
        be a subset of $\mathbb{Q}^+\cap R=\{1,-1\}$. Clearly $\pm1$ are
        units of $R$, thus the units of $R$ are $\pm1$.
      \end{proof}

    \item Show that the irreducibles in $R$ are $\pm p$ where $p$ is prime
      in $\mathbb{Z}$ and the polynomials $f(x)$ are irreducible in
      $\mathbb{Q}[x]$ and have constant term $\pm1$. Prove that these
      irreducibles are prime in $R$.
      \begin{proof}
        We first check that the elements given in the question are
        irreducibles in $R$. Primes $p$ are clearly irreducibles in $R$.
        Also, let $f(x)=xg(x)\pm1\in R$ where $g(x)\in\mathbb{Q}[x]$ and
        $f(x)$ is irreducible in $\mathbb{Q}[x]$. Assume by contradiction
        that $f(x)$ is reducible. If the factor $r\in\mathbb{Z}$ witnesses
        the irreducibility, then since the units of $R$ are only $\pm1$,
        $|r|$ must be greater or equal 2. However, such $r$ cannot divide
        $f(x)$ in $R$. So the factors of $f(x)$ must have degree between 1
        and below the degree of $f(x)$. But such a factor would witness the
        irreducibility of $f(x)$ in $\mathbb{Q}[x]$, since the units of
        $\mathbb{Q}[x]$ are $\mathbb{Q}^+$. Thus $f(x)$ is irreducible in
        $R$. \\

        Next we show that there are no other irreducibles in $R$. Consider
        elements in $R$ of degree 0. These are elements in $\mathbb{Z}$,
        which are irreducible in $R$ if and only if they are prime, since
        the units of $R$ are only $\pm1$. Now consider elements in $R$ of
        degree greater or equal 1 and that are not of the form given in the
        question. Such elements can be written as $f(x)=xg(x)+n\in R$ where
        $g(x)\in\mathbb{Q}[x]$, the degree of $g(x)$ is greater or equal 1,
        and $n\in\mathbb{Z}\setminus\{1,-1\}$. Then $n$ is a non-unit and
        non-associate factor of $f(x)$, thus $f(x)$ is reducible.
      \end{proof}

    \item Show that $x$ cannot be written as the product of irreducibles in
      $R$ (in particular, $x$ is not irreducible) and conclude that $R$ is
      not a UFD.

      \begin{proof}
        The factors of $x$ in $R$ are only $\pm1$ and $\pm x$, which
        are either units or associates with $x$. Thus $x$ must be
        irreducible by definition. However, from the first part of the
        question, $x$ is not a unit. So $x$ cannot be decomposed into a
        product of irreducibles, and therefore $R$ is not a UFD by
        definition.
      \end{proof}

    \item Show that $x$ is not prime in $R$ and describe the quotient ring
      $R/(x)$.
      \begin{proof}
        $x$ divides $x=2\cdot(0.5x)$ in $R$, where $2,0.5x\in R$. However,
        $x$ does not divide $2$ or $0.5x$ in $R$. Thus $x$ is not prime in
        $R$. \\

        We claim that
        \[R/(x)= \{\overline{qx+n}:q\in\mathbb{Q},0\leq q<1,
          n\in\mathbb{Z}\}.\]
        Clearly the set of cosets is contained in the quotient ring. Given
        any $f(x)=xf_0(x)+n\in R$, where
        $f_0(x)=xf_1(x)+q_0\in\mathbb{Q}[x]$ and $n\in\mathbb{Z}$, we can
        rearrange to get
        \begin{align*}
          f(x) &=x(f_1(x)+q_0)+n \\
          &=xf_1(x)+q_0x+n \\
          &=xf_1(x)-n_0x+(q_0-n_0)x+n, &(\text{where}\; n_0=\lfloor
            q_0\rfloor) \\
          &=x(f_1(x)-n_0)+(q_0-n_0)x+n. \\
        \end{align*}
        Then since $f_1(x)-n_0\in R$, $0\leq q_0-n_0<1$, and
        $n\in\mathbb{Z}$, we have
        $\overline{f(x)}=\overline{(q_0-n_0)x+n}$, which is contained in
        the set of cosets we defined. Thus the set of cosets contain the
        quotient ring. \\

        Finally, we show that the set of cosets we defined are distinct.
        Let $qx+n\in(x)$ where $q\in\mathbb{Q}$, $0\leq q<1$, and
        $n\in\mathbb{Z}$. We need to show that $q=n=0$. Given any
        $f(x)\in(x)$, the constant of $f(x)$ must be 0, thus $n=0$. Also,
        the coefficient of $x$ in $f(x)$ must be an integer, thus $q=0$.
      \end{proof}
  \end{enumerate}

\it \textbf{Section 9.4 Question 1:} Determine whether the following
  polynomials are irreducible in the rings indicated. For those that are
  reducible, determine their factorization into irreducibles. The notation
  $\mathbb{F}_p$ denotes the finite field $\mathbb{Z}_p$, $p$ a prime.

  \begin{enumerate}[label={(\alph*)}]
    \item $f(x)=x^2+x+1$ in $\mathbb{F}_2[x]$
      \begin{proof}
        $f(x)$ is irreducible. $f(x)$ is a polynomial of degree two over
        field $\mathbb{F}_2$ with no roots, thus is irreducible
        (Proposition 10 Section 9.3).
      \end{proof}

    \item $f(x)=x^3+x+1$ in $\mathbb{F}_3[x]$
      \begin{proof}
        $f(x)$ is reducible. $f(x)$ is a polynomial of degree three over
        the field $\mathbb{F}_3$ with root $x=1$, thus it is reducible
        (Proposition 10 Section 9.3).
      \end{proof}

    \item $f(x)=x^4+1$ in $\mathbb{F}_5[x]$
      \begin{proof}
        $f(x)$ is reducible. We first check that $f(x)$ has no roots in
        $\mathbb{F}_5$, and thus has no factors of degree 1. So if $f(x)$
        is reducible, it can only have roots of degree 2. We check that
        $f(x)=x^4+1=x^4-4=(x^2-2)(x^2+2)$, thus $f(x)$ is reducible.
      \end{proof}

    \item $f(x)=x^4+10x^2+1$ in $\mathbb{Z}[x]$
      \begin{proof}
        $f(x)$ has no factors of degree 1 since $\pm1$, the factors of the
        constant term of $f(x)$, are not roots of $f(x)$. Thus if $f(x)$ is
        reducible, it can only have factors of degree 2. We first try
        \[f(x)= (x^2+ax+1)(x^2+bx+1) =x^4+10x^2+1.\]
        Comparing coefficients give us $a+b=0$ and $ab+2=10$, which does
        not have integer solutions for $a$ or $b$. Next, we try
        \[f(x)= (x^2+ax-1)(x^2+bx-1) =x^4+10x^2+1.\]
        Comparing coefficients give us $a+b=0$ and $ab-2=10$, which also
        does not have integer solutions for $a$ or $b$. Thus $f(x)$ is
        irreducible in $\mathbb{Z}[x]$.
      \end{proof}
  \end{enumerate}

\it \textbf{Section 9.4 Question 2:} Prove that the following polynomials
  are irreducible in $\mathbb{Z}[x]$.
  \begin{enumerate}[label={(\alph*)}]
    \item $x^4-4x^3+6$
      \begin{proof}
        Eisenstein's criteria with $p=2$ shows that the polynomial is
        irreducible.
      \end{proof}

    \item $x^6+30x^5-15x^3+6x-120$
      \begin{proof}
        Eisenstein's criteria with $p=3$ shows that the polynomial is
        irreducible.
      \end{proof}

    \item $x^4+4x^3+6x^2+2x+1$ [Substitute $x-1$ for $x$.]
      \begin{proof}
        After substituting $x-1$ for $x$, we get $g(x)=x^4-2x+2$, which is
        irreducible from Eisenstein's criteria with $p=2$. Thus the
        original polynomial is irreducible.
      \end{proof}

    \item $f(x)=\frac{(x+2)^p-2^p}{x}$, where $p$ is an odd prime.
      \begin{proof}
        Expanding with binomial's theorem, we get \[f(x) = \sum_{i=0}^{p-1}
        \binom{p}{i}2^ix^{p-1-i}.\] Note that for each
        $i\in\{1,\ldots,p-1\}$, since $p$ is an odd prime,
        $\binom{p}{i}=\frac{p!}{i!(p-i)!}$ is a multiple of $p$ because
        $i!$ and $(p-i)!$ are not multiples of $p$. Note also that the
        constant term $\binom{p}{p-1}2^{p-1}=p2^{p-1}$ is not a multiple of
        $p^2$ since $p$ is an odd prime. Thus $f(x)$ is irreducible from
        Eisenstein's criteria with prime $p$ as witness.
      \end{proof}
  \end{enumerate}

\it \textbf{Section 9.4 Question 6:} Construct fields of order 9 and 8.
  \begin{proof}
    We first prove that given a finite field $F$ of order $q$ and $f(x)$ a
    polynomial in $F[x]$ of degree $n\geq1$, then $F[x]/(f(x))$ has $q^n$
    elements (Question 2, Section 9.2): We show that the quotient ring is
    the set of cosets given by
    \[\{\overline{a_{n-1}x^{n-1}+\ldots+a_1x+a_0}: a_i\in F\},\]
    which is defined by the set of polynomials in $F$ with degree smaller
    or equal $n-1$. Since every polynomial $g(x)$ in $F(x)$ can be written
    as $g(x)=h(x)f(x)+r(x)$ where $r(x)$ is a polynomial with degree
    smaller than $n$, the set of cosets contains the quotient ring. Also,
    the quotient ring clearly contains the given set of cosets. Finally,
    since $F$ is an integral domain, no polynomial of degree less than $n$
    can divide $f(x)$, thus the given set of cosets is distinct. Thus the
    given set of cosets is the quotient ring, and this set has $q^n$
    elements. \\

    Thus, if $f(x)$ is irreducible in $F[x]$, then $F[x]/(f(x))$ will be a
    field (Question 3, Section 9.2) of size $q^n$. We use this rule,
    together with the fact that fields $\mathcal{F}_p$ have order $p$ for
    all primes $p\in\mathbb{N}$, to find fields of the desired order. \\

    From the above argument, to find a field of order 9, it suffices to
    find a irreducible polynomial $f(x)\in\mathbb{F}_3$ of degree 2. Try
    $f(x)=x^2+1$. Then $f(x)$ has no roots in $\mathbb{F}_3$, thus it is
    irreducible in $\mathbb{F}_3[x]$, and so $\mathbb{F}_3[x]/(x^2+1)$ is a
    field of order 9. \\

    Similarly, to find a field of order 8, it suffices to
    find a irreducible polynomial $f(x)\in\mathbb{F}_2$ of degree 3. Try
    $f(x)=x^3+x+1$. Then $f(x)$ has no roots in $\mathbb{F}_2$, thus it is
    irreducible in $\mathbb{F}_2[x]$, and so $\mathbb{F}_2[x]/(x^3+x+1)$ is
    a field of order 8. \\
  \end{proof}

\it \textbf{Section 9.4 Question 7:} Prove that $\mathbb{R}[x]/(x^2+1)$ is
  a field which is isomorphic to the complex numbers.

  \begin{proof}
    Consider the map $\alpha:\mathbb{R}[x]\rightarrow\mathbb{C}$ given by
    $\alpha(f(x))=f(i)$. Then $\alpha$ is surjective since every
    $a+bi\in\mathbb{C}$ has an inverse image $a+bx\in\mathbb{R}[x]$. The
    map is also a ring homomorphism by the commutative, associative, and
    distributive properties of $\mathbb{R}[x]$. Thus from the first ring
    isomorphism theorem, $\mathbb{C}$ is isomorphic to
    $\mathbb{R}[x]/\ker(\alpha)$. It remains to prove that
    $\ker(\alpha)=(x^2+1)$. Clearly $(x^2+1)\subseteq\ker(\alpha)$ since
    $x^2+1\in\ker(\alpha)$. Also, $(x^2+1)$ is a prime ideal since $x^2+1$
    is irreducible and thus prime in $\mathbb{R}[x]$. Then since
    $\mathbb{R}[x]$ is a PID and $(x^2+1)$ is prime, $(x^2+1)$ must also be
    maximal. Thus if $\ker(\alpha)\supsetneq(x^2+1)$ then
    $\ker(\alpha)=\mathbb{R}[x]$, which is not true since
    $1\not\in\ker(\alpha)$. Thus $\ker(\alpha)=(x^2+1)$.
  \end{proof}
\end{document}
