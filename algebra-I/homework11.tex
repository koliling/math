\documentclass{article}
\usepackage[left=3cm,right=3cm,top=3cm,bottom=3cm]{geometry}
\usepackage{amsmath,amssymb,amsthm,pgfplots,tikz}
\usepackage[inline]{enumitem}
\usepackage{color}
\setlength{\parindent}{0mm} %So that we do not indent on new paragraphs
\newcommand{\TODO}[1]{\textcolor{red}{TODO: #1}}

\begin{document}
\title{Graduate Algebra I: Homework 11}
\author{Li Ling Ko\\ lko@nd.edu}
\date{\today}
\maketitle

\it \textbf{Question 1:} For the field of 9 elements and for the field of 8
  elements you construct in Question 6 of Section 9.4 below, find a
  generator for the (cyclic) multiplicative group of non-zero elements.
  \begin{proof}
  \end{proof}

\it \textbf{Section 9.1 Question 4:} Prove that the ideals $(x)$ and
  $(x,y)$ are prime ideals in $\mathbb{Q}[x,y]$ but only the latter ideal
  is a maximal ideal.

  \begin{proof}
    Consider the map $\varphi:\mathbb{Q}[x,y]\rightarrow\mathbb{Q}[y]$
    defined by $\varphi(xf(x,y)+g(y))=g(y)$. This map is a surjective ring
    homomorphism with kernel $(x)$, and thus from the first isomorphism
    theorem of rings, $\mathbb{Q}[x,y]/(x)\cong\mathbb{Q}[y]$. Now since
    $\mathbb{Q}$ is an integral domain, $\mathbb{Q}[y]$ will also be an
    integral domain, and thus $(x)$ is a prime ideal in $\mathbb{Q}[x,y]$.
    However $(x)$ is not a maximal ideal in $\mathbb{Q}[x,y]$ because
    $\mathbb{Q}[y]$ is not a field - the polynomial $y\in\mathbb{Q}[y]$
    does not have a multiplicative inverse. \\

    Consider the map $\phi:\mathbb{Q}[x,y]\rightarrow\mathbb{Q}$
    defined by $\phi(xf(x,y)+yg(y)+c)=c$. This map is a surjective ring
    homomorphism with kernel $(x,y)$, and thus from the first isomorphism
    theorem of rings, $\mathbb{Q}[x,y]/(x,y)\cong\mathbb{Q}$. Now since
    $\mathbb{Q}$ is field, $(x,y)$ is a maximal and also prime ideal in
    $\mathbb{Q}[x,y]$.
  \end{proof}

\it \textbf{Section 9.1 Question 5:} Prove that $(x,y)$ and $(2,x,y)$ are
  prime ideals in $\mathbb{Z}[x,y]$ but only the latter ideal is a maximal
  ideal.

  \begin{proof}
    Consider the map $\phi:\mathbb{Z}[x,y]\rightarrow\mathbb{Z}$
    defined by $\phi(xf(x,y)+yg(y)+c)=c$. This map is a surjective ring
    homomorphism with kernel $(x,y)$, and thus from the first isomorphism
    theorem of rings, $\mathbb{Z}[x,y]/(x,y)\cong\mathbb{Z}$. Now since
    $\mathbb{Z}$ is an integral domain, $(x,y)$ is a prime ideal in
    $\mathbb{Z}[x,y]$. However, because $\mathbb{Z}$ is not a field,
    $(x,y)$ is not a maximal ideal in $\mathbb{Z}[x,y]$. \\

    Consider the map $\phi:\mathbb{Z}[x,y]\rightarrow\mathbb{Z}_2$
    defined by $\phi(xf(x,y)+yg(y)+c)=\bar{c}$. This map is a surjective ring
    homomorphism with kernel $(2,x,y)$, and thus from the first isomorphism
    theorem of rings, $\mathbb{Z}[x,y]/(2,x,y)\cong\mathbb{Z}_2$. Now since
    $\mathbb{Z}_2$ is a field, $(2,x,y)$ is a maximal and also prime ideal in
    $\mathbb{Z}[x,y]$.
  \end{proof}

\it \textbf{Section 9.1 Question 6:} Prove that $(x,y)$ is not a principal
  ideal in $\mathbb{Q}[x,y]$.

  \begin{proof}
    If $(x,y)$ is principal, then $(x,y)=(p(x,y))$ for some polynomial
    $p(x,y)\in\mathbb{Q}[x,y]$. Then $x\in(p(x,y))$ implies $p(x,y)|x$ in
    $\mathbb{Q}[x,y]$. Now in $\mathbb{Q}[x,y]$, the only divisors of the
    polynomial $x$ are of the form $q$ or $qx$ for non-zero rationals
    $q\in\mathbb{Q}\setminus\{0\}$, because if the degree of $x$ in
    $p(x,y)$ is larger than 1, then the degree of $x$ in $p(x,y)f(x,y)$ for
    any non-zero polynomial $f(x,y)\in\mathbb{Q}[x,y]$ will remain larger
    than 1. Thus $p(x,y)$ must be of the form $q$ or $qx$.  Similarly,
    since $y\in(p(x,y))$, $p(x,y)$ must be of a form $q$ or $qy$ for
    non-zero rationals $q\in\mathbb{Q}\setminus\{0\}$. Thus $p(x,y)$ must
    be a non-zero rational $q$. But $(q)=\mathbb{Q}[x,y]$ for non-zero
    $q\in\mathbb{Q}$, yet $(x,y)$ is strictly contained in
    $\mathbb{Q}[x,y]$ because the $\mathbb{Q}[x,y]$ contains $1$ but
    $(x,y)$ does not.
  \end{proof}

\it \textbf{Section 9.2 Question 3:} Let $F$ be a field and let $x$ be an
  indeterminate over $F$. Let $f(x)$ be a polynomial in $F[x]$. Prove that
  $F[x]/(f(x))$ is a field if and only if $f(x)$ is irreducible.

  \begin{proof}
    $\Rightarrow$: If $F[x]/(f(x))$ is a field, $(f(x))$ must be a prime
    ideal, which means $f(x)$ is prime in $F(x)$. Now  since $F(x)$ is a
    UFD, prime elements are equivalent to the irreducible elements, thus
    $f(x)$ is irreducible. \\

    $\Leftarrow$: Assume $f(x)$ is irreducible. Then since $F(x)$ is a UFD,
    irreducible elements are equivalent to prime ones, so $f(x)$ is a prime
    element. Then $(f(x))$ will be a prime ideal. Now since $F[x]$ is a PID
    (Corollary 4, Section 9.2), its prime ideals are maximal ideals
    (Proposition 7, Section 8.2), thus $(f(x))$ is a maximal ideal, which
    makes $F[x]/(f(x))$ a field.
  \end{proof}
\end{document}
