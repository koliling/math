\documentclass{article}
\usepackage[left=3cm,right=3cm,top=3cm,bottom=3cm]{geometry}
\usepackage{amsmath,amssymb,amsthm,pgfplots,tikz}
\usepackage[inline]{enumitem}
\usepackage{color}
\setlength{\parindent}{0mm} %So that we do not indent on new paragraphs
\newcommand{\TODO}[1]{\textcolor{red}{TODO: #1}}

\begin{document}
\title{Graduate Algebra II: Homework 1}
\author{Li Ling Ko\\ lko@nd.edu}
\date{\today}
\maketitle

In these exercises $R$ is a ring with 1 and $M$ is a left $R$-module. \\

\it \textbf{Section 10.1 Q3:} Assume that $rm=0$ for some $r\in R$ and some
  $m\in M$ with $m\neq0$. Prove that $r$ does not have a left inverse
  (i.e., there is no $s\in R$ such that $sr=1$).

  \begin{proof}
    Assume $r$ has a left inverse $s$. Then
    \begin{align*}
      m &= 1m \\
      &= (sr)m \\
      &= s(rm) \\
      &= s0. \\
    \end{align*}
    Also, we have $s0=s(0+0)=s0+s0$, so by cancellation, we get $s0=0$,
    which implies $m=0$, a contradiction.
  \end{proof}

\it \textbf{Section 10.1 Q5:} For any left ideal $I$ of $R$ define
  \[IM = \left\{\sum_{\text{finite}}a_im_i| a_i\in I, m_i\in M\right\}\]
  to be the collection of all finite sums of elements of the form $am$
  where $a\in I$ and $m\in M$. Prove that $IM$ is a submodule of $M$.

  \begin{proof}
    Let $r\in R$ and $x,y\in IM$. Write $x=a_1m_1+\ldots+a_pm_p$ and
    $y=b_1n_1+\ldots+b_qn_q$ for some $a_i,b_j\in I$ and $m_i,n_j\in M$.
    By the submodule criterion (Proposition 1, Section 10.1), it suffices
    to show that $x+ry\in IM$. Now
    \begin{align*}
      x+ry &=a_1m_1+\ldots+a_pm_p + r(b_1n_1+\ldots+b_qn_q) \\
      &=a_1m_1+\ldots+a_pm_p + (rb_1)n_1+\ldots+(rb_q)n_q, \\
    \end{align*}
    which is contained in $IM$ since $rb_j\in I$ because $I$ is a left
    ideal of $R$.
  \end{proof}

\it \textbf{Section 10.1 Q8:} An element $m$ of the $R$-module $M$ is
  called a torsion element if $rm=0$ for some nonzero element $r\in R$. The
  set of torsion elements is denoted
  \[\text{Tor}(M) = \{m\in M| rm=0\; \text{for some nonzero}\; r\in R\}.\]

  \begin{enumerate}[label={(\alph*)}]
    \item \it Prove that if $R$ is an integral domain then $\text{Tor}(M)$
      is a submodule of $M$ (called the torsion submodule of $M$).

      \begin{proof}
        Let $x,y\in\text{Tor}(M)$ with $r_xx=r_yy=0$ for some nonzero
        $r_x,r_y\in R$. Let $r\in R$. By the submodule criterion, we need
        to show that $x+ry\in\text{Tor}(M)$. Now
        \begin{align*}
          r_xr_y(x+ry) &= r_xr_yx+r_xr_yry \\
          &=r_yr_xx+r_xrr_yy &(\text{from commutativity of integral domain}) \\
          &=r_y0+r_xr0 \\
          &=0+0 &(s0=0\; \text{for all}\; s\in R,\; \text{shown in Q3}) \\
          &=0.
        \end{align*}
        Also, because $R$ is an integral domain, $r_xr_y\neq0$, thus
        $x+ry\in\text{Tor}(M)$.
      \end{proof}

    \item \it Give an example of a ring $R$ and an $R$-module $M$ such that
      $\text{Tor}(M)$ is not a submodule. [Consider the torsion elements in
      the $R$-module $R$.]

      \begin{proof}
        Consider $R=M=\mathbb{Z}_6$, where $M$ is a left module over
        itself. Then $\text{Tor}(M)=\{\bar{0},\bar{2},\bar{3}\}$, but this
        set is not closed under addition since $\bar{2}+\bar{3}=\bar{5}$ is
        not in the set. Thus $\text{Tor}(M)$ is not a submodule.
      \end{proof}

    \item \it If $R$ has zero divisors show that every nonzero $R$-module
      has nonzero torsion elements.
      \begin{proof}
        Let $r,s\in R$ be nonzero elements such that $rs=0$. Let $M$ be
        a nonzero $R$-module. If $sm=0$ for all $m\in M$, pick any
        nonzero $m\in M$. This $m$ will be a torsion element with $s$ as
        witness. If no such $m$ exists, then pick any nonzero $sm\in M$;
        then we have $r(sm)=(rs)m=0m=0$, which implies that $sm$ is a
        torsion element. In either cases we get a nonzero torsion element.
      \end{proof}
  \end{enumerate}

\it \textbf{Section 10.1 Q9:} If $N$ is a submodule of $M$, the annihilator
  of $N$ in $R$ is defined to be
  \[\text{Ann}_R(N):= \{r\in R|rn=0\; \text{for all}\; n\in N\}.\]
  Prove that the annihilator of $N$ in $R$ is a 2-sided ideal of $R$.

  \begin{proof}
    Write the annihilator of $N$ in $R$ as $\text{Ann}_R(N)$. Let
    $a,b\in\text{Ann}_R(N)$. Then $aN=bN=0$, so $(a-b)N=aN-bN=0-0=0$ (here
    the 0 means the subset of $M$ containing only the zero element). Thus
    $a-b\in\text{Ann}_R(N)$, which implies that $\text{Ann}_R(N)$ is closed
    under subtraction. Also, given arbitrary $r\in R$,
    $z\in\text{Ann}_R(N)$, and $n\in N$, we have $(rz)n=r(zn)=r0=0$. Thus
    $\text{Ann}_R(N)$ is a left ideal. Furthermore, $(zr)n=z(rn)=0$, where
    the last equality holds because $rn\in N$ and $zN=0$. Thus
    $\text{Ann}_R(N)$ is also a right ideal.
  \end{proof}

\it \textbf{Section 10.1 Q10:} If $I$ is a right ideal of $R$, the
  annihilator of $I$ in $M$ is defined to be
  \[\text{Ann}_M(I):= \{m\in M|am=0\; \text{for all}\; a\in I\}.\]
  Prove that the annihilator of $I$ in $M$ is a submodule of $M$.

  \begin{proof}
    Let $x,y\in M$, and $r\in R$. Then $Ix=Iy=0$, where $0$ here denotes
    the zero set. Then
    \begin{align*}
      I(x+ry) &=Ix+I(ry) \\
      &=0+(Ir)y \\
      &=(Ir)y \\
    \end{align*}
    Now since $I$ is a right ideal, we have $Ir\subseteq I$, so
    $(Ir)y\subseteq Iy=0$. Thus $x+ry$ is also in the annihilator, which
    shows that the annihilator is a submodule, by the submodule criterion.
  \end{proof}

\it \textbf{Section 10.1 Q11:} Let $M$ be the abelian group (i.e.,
  $\mathbb{Z}$-module
  $\mathbb{Z}_{24}\times\mathbb{Z}_{15}\times\mathbb{Z}_{50}$).

  \begin{enumerate}[label={(\alph*)}]
    \item \it Find the annihilator of $M$ in $\mathbb{Z}$ (i.e., a
      generator for this principal ideal).

      \begin{proof}
        An element $r\in\mathbb{Z}$ in the annihilator must annihilate
        $(\bar{1},\bar{1},\bar{1})\in M$. Thus $r(\bar{1},\bar{1},\bar{1})
        =(\bar{r},\bar{r},\bar{r}) =(\bar{0},\bar{0},\bar{0})$, which is
        satisfied if and only if $r$ is a multiple of 24, 15, and 50. Hence
        the annihilator is a subset of $(\text{lcm}(24,15,50))\mathbb{Z}
        =600\mathbb{Z} =\langle600\rangle$. Conversely, any multiple of 600
        will annihilate $M$. Thus the annihilator is exactly
        $\langle600\rangle$.
      \end{proof}

    \item \it Let $I=2\mathbb{Z}$. Describe the annihilator of $I$ in $M$
      as a direct product of cyclic groups.

      \begin{proof}
        Any element $(\bar{x},\bar{y},\bar{z})$ in the annihilator must
        annihilate $2\in2\mathbb{Z}$. Thus $2(\bar{x},\bar{y},\bar{z})
        =(\bar{2x},\bar{2y},\bar{2z}) =(\bar{0},\bar{0},\bar{0})$, which
        holds if and only if $2x|24$, $2y|15$, and $2z|50$. Equivalently,
        we have $x|12$, $y|15$, and $z|25$. Thus the annihilator is a
        subset of $\langle12\rangle \times\langle15\rangle \times
        \langle25\rangle$. Conversely, since $2\mathbb{Z}$ is generated by
        2, any element of $2\mathbb{Z}$ will be annihilated by
        $\langle12\rangle \times\langle15\rangle \times \langle25\rangle$.
        Thus the annihilator is $\langle12\rangle \times\langle15\rangle
        \times \langle25\rangle$.
      \end{proof}
  \end{enumerate}

\it \textbf{Section 10.1 Q12:} In the notation of the preceding exercises
  prove the following facts about annihilators.
  \begin{enumerate}[label={(\alph*)}]
    \item \it Let $N$ be a submodule of $M$ and let $I$ be its annihilator
      in $R$. Prove that the annihilator of $I$ in $M$ contains $N$. Give
      an example where the annihilator of $I$ in $M$ does not equal $N$.

      \begin{proof}
        Let $n\in N$. We need to show that $In$ is the zero set. Now since
        elements in $I$ annihilate $n$, $In$ is the zero set by definition.
        \\

        Consider the case where $R=\mathbb{Z}$,
        $M=\mathbb{Z}_2\times\mathbb{Z}_6$, and $N=\langle\bar{0}\rangle
        \times\langle\bar{3}\rangle$. Then $I=2\mathbb{Z}$, but the
        annihilator of $I$ in $M$ is
        $\mathbb{Z}_2\times\langle\bar{3}\rangle$, which is not equals to
        $N$.
      \end{proof}

    \item \it Let $I$ be a right ideal of $R$ and let $N$ be its
      annihilator in $M$. Prove that the annihilator of $N$ in $R$ contains
      $I$. Give an example where the annihilator of $N$ in $R$ does not
      equal $I$.

      \begin{proof}
        Let $a\in I$. We need to show that $aN$ is the zero set. Now since
        elements in $N$ annihilate $a$, $aN$ is the zero set by definition.
        \\

        Consider the case where $R=\mathbb{Z}$,
        $M=\mathbb{Z}_2\times\mathbb{Z}_6$, and $I=4\mathbb{Z}$. Then
        $N=\mathbb{Z}_2\times\langle\bar{3}\rangle$, but the annihilator of
        $N$ in $R$ is $2\mathbb{Z}$, which is not equals to $I$.
      \end{proof}
  \end{enumerate}
\end{document}
