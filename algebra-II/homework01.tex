\documentclass{article}
\usepackage[left=3cm,right=3cm,top=3cm,bottom=3cm]{geometry}
\usepackage{amsmath,amssymb,amsthm,pgfplots,tikz}
\usepackage[inline]{enumitem}
\usepackage{color}
\setlength{\parindent}{0mm} %So that we do not indent on new paragraphs
\newcommand{\TODO}[1]{\textcolor{red}{TODO: #1}}

\begin{document}
\title{Graduate Algebra II: Homework 1}
\author{Li Ling Ko\\ lko@nd.edu}
\date{\today}
\maketitle

In these exercises $R$ is a ring with 1 and $M$ is a left $R$-module. \\

\it \textbf{Section 10.1 Q3:} Assume that $rm=0$ for some $r\in R$ and some
  $m\in M$ with $m\neq0$. Prove that $r$ does not have a left inverse
  (i.e., there is no $s\in R$ such that $sr=1$).

  \begin{proof}
    Assume $r$ has a left inverse $s$. Then
    \begin{align*}
      m &= 1m \\
      &= (sr)m \\
      &= s(rm) \\
      &= s0. \\
    \end{align*}
    Also, we have $s0=s(0+0)=s0+s0$, so by cancellation, we get $s0=0$,
    which implies $m=0$, a contradiction.
  \end{proof}

\it \textbf{Section 10.1 Q5:} For any left ideal $I$ of $R$ define
  \[IM = \left\{\sum_{\text{finite}}a_im_i| a_i\in I, m_i\in M\right\}\]
  to be the collection of all finite sums of elements of the form $am$
  where $a\in I$ and $m\in M$. Prove that $IM$ is a submodule of $M$.

  \begin{proof}
    Let $r\in R$ and $x,y\in IM$. Write $x=a_1m_1+\ldots+a_pm_p$ and
    $y=b_1n_1+\ldots+b_qn_q$ for some $a_i,b_j\in I$ and $m_i,n_j\in M$.
    By the submodule criterion (Proposition 1, Section 10.1), it suffices
    to show that $x+ry\in IM$. Now
    \begin{align*}
      x+ry &=a_1m_1+\ldots+a_pm_p + r(b_1n_1+\ldots+b_qn_q) \\
      &=a_1m_1+\ldots+a_pm_p + (rb_1)n_1+\ldots+(rb_q)n_q, \\
    \end{align*}
    which is contained in $IM$ since $rb_j\in I$ because $I$ is a left
    ideal of $R$.
  \end{proof}

\it \textbf{Section 10.1 Q8:} An element $m$ of the $R$-module $M$ is
  called a torsion element if $rm=0$ for some nonzero element $r\in R$. The
  set of torsion elements is denoted
  \[\text{Tor}(M) = \{m\in M| rm=0\; \text{for some nonzero}\; r\in R\}.\]

  \begin{enumerate}[label={(\alph*)}]
    \item \it Prove that if $R$ is an integral domain then $\text{Tor}(M)$
      is a submodule of $M$ (called the torsion submodule of $M$).

      \begin{proof}
        Let $x,y\in\text{Tor}(M)$ with $r_xx=r_yy=0$ for some nonzero
        $r_x,r_y\in R$. Let $r\in R$. By the submodule criterion, we need
        to show that $x+ry\in\text{Tor}(M)$. Now
        \begin{align*}
          r_xr_y(x+ry) &= r_xr_yx+r_xr_yry \\
          &=r_yr_xx+r_xrr_yy &(\text{from commutativity of integral domain}) \\
          &=r_y0+r_xr0 \\
          &=0+0 &(s0=0\; \text{for all}\; s\in R,\; \text{shown in Q3}) \\
          &=0.
        \end{align*}
        Also, because $R$ is an integral domain, $r_xr_y\neq0$, thus
        $x+ry\in\text{Tor}(M)$.
      \end{proof}

    \item \it Give an example of a ring $R$ and an $R$-module $M$ such that
      $\text{Tor}(M)$ is not a submodule. [Consider the torsion elements in
      the $R$-module $R$.]

      \begin{proof}
        Consider $R=M=\mathbb{Z}_6$, where $M$ is a left module over
        itself. Then $\text{Tor}(M)=\{\bar{0},\bar{2},\bar{3}\}$, but this
        set is not closed under addition since $\bar{2}+\bar{3}=\bar{5}$ is
        not in the set. Thus $\text{Tor}(M)$ is not a submodule.
      \end{proof}

    \item \it If $R$ has zero divisors show that every nonzero $R$-module
      has nonzero torsion elements.
      \begin{proof}
        Let $r,s\in R$ be nonzero elements such that $rs=0$. Let $M$ be
        a nonzero $R$-module. If $sm=0$ for all $m\in M$, pick any
        nonzero $m\in M$. This $m$ will be a torsion element with $s$ as
        witness. If no such $m$ exists, then pick any nonzero $sm\in M$;
        then we have $r(sm)=(rs)m=0m=0$, which implies that $sm$ is a
        torsion element. In either cases we get a nonzero torsion element.
      \end{proof}
  \end{enumerate}

\it \textbf{Section 10.1 Q9:} If $N$ is a submodule of $M$, the annihilator
  of $N$ in $R$ is defined to be
  \[\text{Ann}_R(N):= \{r\in R|rn=0\; \text{for all}\; n\in N\}.\]
  Prove that the annihilator of $N$ in $R$ is a 2-sided ideal of $R$.

  \begin{proof}
    Write the annihilator of $N$ in $R$ as $\text{Ann}_R(N)$. Let
    $a,b\in\text{Ann}_R(N)$. Then $aN=bN=0$, so $(a-b)N=aN-bN=0-0=0$ (here
    the 0 means the subset of $M$ containing only the zero element). Thus
    $a-b\in\text{Ann}_R(N)$, which implies that $\text{Ann}_R(N)$ is closed
    under subtraction. Also, given arbitrary $r\in R$,
    $z\in\text{Ann}_R(N)$, and $n\in N$, we have $(rz)n=r(zn)=r0=0$. Thus
    $\text{Ann}_R(N)$ is a left ideal. Furthermore, $(zr)n=z(rn)=0$, where
    the last equality holds because $rn\in N$ and $zN=0$. Thus
    $\text{Ann}_R(N)$ is also a right ideal.
  \end{proof}

\it \textbf{Section 10.1 Q10:} If $I$ is a right ideal of $R$, the
  annihilator of $I$ in $M$ is defined to be
  \[\text{Ann}_M(I):= \{m\in M|am=0\; \text{for all}\; a\in I\}.\]
  Prove that the annihilator of $I$ in $M$ is a submodule of $M$.

  \begin{proof}
    Let $x,y\in M$, and $r\in R$. Then $Ix=Iy=0$, where $0$ here denotes
    the zero set. Then
    \begin{align*}
      I(x+ry) &=Ix+I(ry) \\
      &=0+(Ir)y \\
      &=(Ir)y \\
    \end{align*}
    Now since $I$ is a right ideal, we have $Ir\subseteq I$, so
    $(Ir)y\subseteq Iy=0$. Thus $x+ry$ is also in the annihilator, which
    shows that the annihilator is a submodule, by the submodule criterion.
  \end{proof}

\it \textbf{Section 10.1 Q11:} Let $M$ be the abelian group (i.e.,
  $\mathbb{Z}$-module
  $\mathbb{Z}_{24}\times\mathbb{Z}_{15}\times\mathbb{Z}_{50}$).

  \begin{enumerate}[label={(\alph*)}]
    \item \it Find the annihilator of $M$ in $\mathbb{Z}$ (i.e., a
      generator for this principal ideal).

      \begin{proof}
        An element $r\in\mathbb{Z}$ in the annihilator must annihilate
        $(\bar{1},\bar{1},\bar{1})\in M$. Thus $r(\bar{1},\bar{1},\bar{1})
        =(\bar{r},\bar{r},\bar{r}) =(\bar{0},\bar{0},\bar{0})$, which is
        satisfied if and only if $r$ is a multiple of 24, 15, and 50. Hence
        the annihilator is a subset of $(\text{lcm}(24,15,50))\mathbb{Z}
        =600\mathbb{Z} =\langle600\rangle$. Conversely, any multiple of 600
        will annihilate $M$. Thus the annihilator is exactly
        $\langle600\rangle$.
      \end{proof}

    \item \it Let $I=2\mathbb{Z}$. Describe the annihilator of $I$ in $M$
      as a direct product of cyclic groups.

      \begin{proof}
        Any element $(\bar{x},\bar{y},\bar{z})$ in the annihilator must
        annihilate $2\in2\mathbb{Z}$. Thus $2(\bar{x},\bar{y},\bar{z})
        =(\bar{2x},\bar{2y},\bar{2z}) =(\bar{0},\bar{0},\bar{0})$, which
        holds if and only if $2x|24$, $2y|15$, and $2z|50$. Equivalently,
        we have $x|12$, $y|15$, and $z|25$. Thus the annihilator is a
        subset of $\langle12\rangle \times\langle15\rangle \times
        \langle25\rangle$. Conversely, since $2\mathbb{Z}$ is generated by
        2, any element of $2\mathbb{Z}$ will be annihilated by
        $\langle12\rangle \times\langle15\rangle \times \langle25\rangle$.
        Thus the annihilator is $\langle12\rangle \times\langle15\rangle
        \times \langle25\rangle$.
      \end{proof}
  \end{enumerate}

\it \textbf{Section 10.1 Q12:} In the notation of the preceding exercises
  prove the following facts about annihilators.
  \begin{enumerate}[label={(\alph*)}]
    \item \it Let $N$ be a submodule of $M$ and let $I$ be its annihilator
      in $R$. Prove that the annihilator of $I$ in $M$ contains $N$. Give
      an example where the annihilator of $I$ in $M$ does not equal $N$.

      \begin{proof}
        Let $n\in N$. We need to show that $In$ is the zero set. Now since
        elements in $I$ annihilate $n$, $In$ is the zero set by definition.
        \\

        Consider the case where $R=\mathbb{Z}$,
        $M=\mathbb{Z}_2\times\mathbb{Z}_6$, and $N=\langle\bar{0}\rangle
        \times\langle\bar{3}\rangle$. Then $I=2\mathbb{Z}$, but the
        annihilator of $I$ in $M$ is
        $\mathbb{Z}_2\times\langle\bar{3}\rangle$, which is not equals to
        $N$.
      \end{proof}

    \item \it Let $I$ be a right ideal of $R$ and let $N$ be its
      annihilator in $M$. Prove that the annihilator of $N$ in $R$ contains
      $I$. Give an example where the annihilator of $N$ in $R$ does not
      equal $I$.

      \begin{proof}
        Let $a\in I$. We need to show that $aN$ is the zero set. Now since
        elements in $N$ annihilate $a$, $aN$ is the zero set by definition.
        \\

        Consider the case where $R=\mathbb{Z}$,
        $M=\mathbb{Z}_2\times\mathbb{Z}_6$, and $I=4\mathbb{Z}$. Then
        $N=\mathbb{Z}_2\times\langle\bar{3}\rangle$, but the annihilator of
        $N$ in $R$ is $2\mathbb{Z}$, which is not equals to $I$.
      \end{proof}
  \end{enumerate}

\it \textbf{Section 10.1 Q22:} Suppose that $A$ is a ring with identity
  $1_A$ that is a (unital) left $R$-module satisfying $r\cdot(ab)=(r\cdot
  a)b=a(r\cdot b)$ for all $r\in R$ and $a,b\in A$. Prove that the map
  $f:R\rightarrow A$ defined by $f(r)=r\cdot1_A$ is a ring homomorphism
  mapping $1_R$ to $1_A$ and that $f(R)$ is contained in the center of $A$.
  Conclude that $A$ is an $R$-algebra and that the $R$-module structure on
  $A$ induced by its algebra structure is precisely the original $R$-module
  structure.

  \begin{proof}
    Since $A$ is a unital $R$-module, we have $f(1_R)=1_R\cdot1_A=1_A$ as
    required. Also, given $r_1,r_2\in R$, we have
    \begin{align*}
      f(r_1+r_2) &=(r_1+r_2)\cdot1_A \\
      &=r_1\cdot1_A + r_2\cdot1_A \\
      &=f(r_1)+f(r_2), \\
      f(r_1r_2) &=(r_1r_2)\cdot1_A \\
      &=r_1\cdot(r_2\cdot1_A) \\
      &=r_1\cdot f(r_2) \\
      &=(r_1\cdot1_A)f(r_2) \\
      &=f(r_1)f(r_2). \\
    \end{align*}
    Thus $f$ is a ring homomorphism. \\

    Given arbitrary $r\in R$ and $a\in A$, we want to show that
    $f(r)a=af(r)$. We have
    \begin{align*}
      f(r)a &=(r\cdot1_A)a \\
      &=r\cdot(1_Aa) \\
      &=r\cdot(a) \\
      &=r\cdot(a1_A) \\
      &=a(r\cdot1_A) \\
      &=af(r). \\
    \end{align*}
    Thus $f(R)$ is contained in the center of $A$, which implies $A$ is an
    $R$-algebra by definition. \\

    The induced algebra structure $\circ$ on $A$ is given by
    \[r\circ a:=f(r)a=(r\cdot1_A)a=r\cdot(1_Aa)=r\cdot a,\]
    which is precisely the original $R$-module structure.
  \end{proof}

\it \textbf{Section 10.2 Q4:} Let $A$ be any $\mathbb{Z}$-module, let $a$
  be any element of $A$ and let $n$ be a positive integer. Prove that the
  map $\varphi_a:\mathbb{Z}_n\rightarrow A$ given by $\varphi_a(\bar{k})=ka$
  is a well defined $\mathbb{Z}$-module homomorphism if and only if
  $na=0$. Prove that $\text{Hom}_{\mathbb{Z}}(\mathbb{Z}_n,A)\cong A_n$,
  where $A_n=\{a\in A|na=0\}$.

  \begin{proof}
    To respect $\mathbb{Z}$-module structure, any $\mathbb{Z}$-module
    homomorphism $\varphi$ must satisfy $\varphi(\bar{k})
    =\varphi(k\bar{1}) =k\varphi(\bar{1})$. Write $a=\varphi(\bar{1})$, all
    $\mathbb{Z}$-module homomorphisms must be of the form
    $\varphi(\bar{k})=ka$ for some $a\in A$. \\

    Now the homomorphism is well defined if and only if
    $\varphi_a(\overline{k_0+sn})=\varphi_a(\overline{k_0})$ for any
    $k_0,s\in\mathbb{Z}$. Now
    \begin{align*}
      \varphi_a(\overline{k_0+sn}) &=(k_0+sn)a \\
      &=k_0a+(sn)a \\
      &=k_0a+s(na) \\
      &=\varphi_a(\overline{k_0})+s(na), \\
    \end{align*}
    thus the map is well defined if and only if $s(na)=0$ for any
    $s\in\mathbb{Z}$. So if $na=0$, the map will be well defined.
    For the converse, if the map is well defined, then setting $s=1$ will
    give us $1(na)=0$, which implies that $na=0$. \\

    We now show that if $na=0$, then $\varphi_a$ is a
    $\mathbb{Z}$-module homomorphism. Given
    $\overline{k_1},\overline{k_2}\in\mathbb{Z}_n$ and $r\in\mathbb{Z}$, we
    have
    \begin{align*}
      \varphi_a(r\overline{k_1}+\overline{k_2})
        &=\varphi_a(\overline{rk_1+k_2}) \\
      &=(rk_1+k_2)a \\
      &=(rk_1)a+k_2a \\
      &=r(k_1a)+k_2a \\
      &=r\varphi_a(k_1)+\varphi_a(k_2). \\
    \end{align*}

    Now we prove that $\text{Hom}_{\mathbb{Z}}(\mathbb{Z}_n,A)\cong A_n$,
    where $A_n=\{a\in A|na=0\}$. Note that since $A_n$ is the annihilator
    of the ideal $(n)$ of $\mathbb{Z}$, it is a submodule of $A$ and is
    therefore also a $\mathbb{Z}$-module. Also, since $\mathbb{Z}$ is a
    commutative ring with identity,
    $\text{Hom}_{\mathbb{Z}}(\mathbb{Z}_n,A)$ is a $\mathbb{Z}$-module.
    Consider the map $\Theta:A_n\rightarrow
    \text{Hom}_{\mathbb{Z}}(\mathbb{Z}_n,A)$ defined by
    $\Theta(a)=\varphi_a$. Note that from argument in the first paragraph,
    this map is surjective. Also, from earlier argument, this map is well
    defined. If $a\in\ker(\Theta)$, then $\varphi_a$ is the zero map, where
    $\varphi_a(\bar{1})=1a=0$, implying that $a=0$. Thus $\ker(\Theta)$ is
    the zero submodule, which means $\Theta$ is injective. It remains to
    show that $\Theta$ respects $\mathbb{Z}$-module structure. Given
    $a_1,a_2\in A_n$, $r\in\mathbb{Z}$, and $\bar{k}\in\mathbb{Z}_n$, we
    have
    \begin{align*}
      \varphi_{ra_1+a_2}(\bar{k}) &=k(ra_1+a_2) \\
      &=rka_1+ka_2 \\
      &=r\varphi_{a_1}(\bar{k})+\varphi_{a_2}(\bar{k}), \\
      &=(r\varphi_{a_1})(\bar{k})+\varphi_{a_2}(\bar{k}), \\
    \end{align*}
    thus $\Theta(ra_1+a_2) =\varphi_{ra_1+a_2} =r\Theta(a_1)+\Theta(a_2)$,
    as required.
  \end{proof}

\it \textbf{Section 10.2 Q5:} Exhibit all $\mathbb{Z}$-module homomorphisms
  from $\mathbb{Z}_{30}$ to $\mathbb{Z}_{21}$.

  \begin{proof}
    From Section 10.2 Question 4, we know that the $\mathbb{Z}$-module
    homomorphisms are exactly those of the form
    $\varphi_a:\mathbb{Z}_{30}\rightarrow\mathbb{Z}_{21}$, given by
    $\varphi_a(\bar{k})=ka$, where $a\in
    A_{30}:=\{a\in\mathbb{Z}_{21}|30a=0\}$. Since $21=3\cdot7$ and
    $30=2\cdot3\cdot5$, we get $A_{30}=(\bar{7})\cong\mathbb{Z}_3$.
    Summarizing, there are exactly three $\mathbb{Z}$-module homomorphisms
    $\varphi_0$, $\varphi_7$, and $\varphi_{14}$, where
    $\varphi_a(\bar{k})=ka$.
  \end{proof}

\it \textbf{Section 10.2 Q6:} Prove that
  $\text{Hom}_{\mathbb{Z}}(\mathbb{Z}_n,\mathbb{Z}_m)
  \cong\mathbb{Z}_{(n,m)}$.

  \begin{proof}
    From Section 10.2 Question 4, we know that
    $\text{Hom}_{\mathbb{Z}}(\mathbb{Z}_n,\mathbb{Z}_m) \cong A_n$, where
    $A_n:=\{a\in\mathbb{Z}_{m}|na=0\}$. 
    Write $d=(n,m)$, $n=n_0d$, and $m=m_0d$. Then $(m_0,n_0)=1$. We show
    that $A_n=(\overline{m_0})$, which is isomorphic to $\mathbb{Z}_{d}$.
    If $\bar{a}\in A_n$, then $m|na$, which implies that $m_0d|n_0da$,
    which implies that $m_0|n_0a$. Then since $(m_0,n_0)=1$, we have
    $m_0|a$. Hence $A_n\subseteq(\overline{m_0})$. Conversely, if $a=a_0m_0$
    for some $a_0\in\mathbb{Z}$, then $n\overline{a} =n\overline{a_0m_0}
    =n_0d\overline{a_0m_0} =\overline{n_0a_0(m_0d)} =\overline{n_0am} =0$.
    Thus $A_n=(\overline{m/(n,m)}) \cong\mathbb{Z}_{(n,m)}$.
  \end{proof}

\it \textbf{Section 10.2 Q9:} Let $R$ be a commutative ring. Prove that
  $\text{Hom}_R(R,M)$ and $M$ are isomorphic as left $R$-modules. [Show
  that each element of $\text{Hom}_R(R,M)$ is determined by its value on
  the identity of $R$.]

  \begin{proof}
    By Proposition 2 of Section 10.2, we know that $\text{Hom}_R(R,M)$ is
    an $R$-module since $R$ is commutative. Consider the map
    $\Theta:\text{Hom}_R(R,M)\rightarrow M$ defined by
    $\Theta(\varphi)=\varphi(1)$. We show that $\Theta$ is an $R$-module
    isomorphism. Let $\varphi,\phi\in\text{Hom}_R(R,M)$, and $r\in R$. Then
    we have
    \begin{align*}
      \Theta(\varphi+r\phi) &=(\varphi+r\phi)(1) \\
      &=\varphi(1)+r\phi(1) \\
      &=\Theta(\varphi)+r\Theta(phi),
    \end{align*}
    and so $\Theta$ is an $R$-module homomorphism. \\

    Let $\varphi\in\ker(\Theta)$. Then $\Theta(\varphi)=\varphi(1)=0$, so
    given arbitrary $r\in R$, we have $\varphi(r) =\varphi(r1)
    =r\varphi(1) =r0$, thus $\varphi$ is the zero $R$-homomorphism. Hence
    $\Theta$ is injective. \\

    Let $m\in M$ be arbitrary, and consider the map $\varphi_m:R\rightarrow
    M$ defined by $\varphi_m(r)=r\cdot m$. Then given $x,y\in R$ and $r\in
    R$, we have
    \begin{align*}
      \varphi_m(x+r\cdot y) &=(x+r\cdot y)\cdot m \\
      &=x\cdot m + r\cdot(y\cdot m) \\
      &=\varphi_m(x) + r\varphi_m(y),
    \end{align*}
    which shows that $\varphi_m$ is an $R$-module homomorphism. Since $m$
    was arbitrary, and $\varphi_m(1)=1$, $\Theta$ is surjective. \\

    Thus $\Theta$ is bijective, and so $\text{Hom}_R(R,M)\cong M$.
  \end{proof}

\it \textbf{Section 10.2 Q10:} Let $R$ be a commutative ring. Prove that
  $\text{Hom}_R(R,R)$ and $R$ are isomorphic as rings.

  \begin{proof}
    From Section 10.2 Question 9, we showed that
    $\Theta:\text{Hom}_R(R,R)\rightarrow R$ defined by
    $\Theta(\varphi)=\varphi(1)$ is an $R$-module isomorphism. Thus
    $\Theta(\varphi+\phi)=\Theta(\varphi)+\Theta(\phi)$. It remains to show
    that $\Theta(\varphi\circ\phi)=\Theta(\varphi)\Theta(\phi)$. We have
    \begin{align*}
      \Theta(\varphi\circ\phi) &=(\varphi\circ\phi)(1) \\
      &=\varphi(\phi(1)) \\
      &=\phi(1)\cdot\varphi(1) \\
      &=\varphi(1)\phi(1) \\
      &=\Theta(\varphi)\Theta(\phi),
    \end{align*}
    from commutativity of $R$. Thus $\text{Hom}_R(R,R)$ and $R$ are
    isomorphic as rings.
  \end{proof}
\end{document}
