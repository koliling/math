\documentclass{article}
\usepackage[left=3cm,right=3cm,top=3cm,bottom=3cm]{geometry}
\usepackage{amsmath,amssymb,amsthm,pgfplots,tikz}
\usepackage[inline]{enumitem}
\usepackage{color}
\setlength{\parindent}{0mm} %So that we do not indent on new paragraphs
\newcommand{\TODO}[1]{\textcolor{red}{TODO: #1}}

\begin{document}
\title{Graduate Algebra II: Homework 1}
\author{Li Ling Ko\\ lko@nd.edu}
\date{\today}
\maketitle

In these exercises $R$ is a ring with 1 and $M$ is a left $R$-module. \\

\it \textbf{Section 10.1 Q3:} Assume that $rm=0$ for some $r\in R$ and some
  $m\in M$ with $m\neq0$. Prove that $r$ does not have a left inverse
  (i.e., there is no $s\in R$ such that $sr=1$).

  \begin{proof}
    Assume $r$ has a left inverse $s$. Then
    \begin{align*}
      m &= 1m \\
      &= (sr)m \\
      &= s(rm) \\
      &= s0. \\
    \end{align*}
    Also, we have $s0=s(0+0)=s0+s0$, so by cancellation, we get $s0=0$,
    which implies $m=0$, a contradiction.
  \end{proof}

\it \textbf{Section 10.1 Q5:} For any left ideal $I$ of $R$ define
  \[IM = \left\{\sum_{\text{finite}}a_im_i| a_i\in I, m_i\in M\right\}\]
  to be the collection of all finite sums of elements of the form $am$
  where $a\in I$ and $m\in M$. Prove that $IM$ is a submodule of $M$.

  \begin{proof}
    Let $r\in R$ and $x,y\in IM$. Write $x=a_1m_1+\ldots+a_pm_p$ and
    $y=b_1n_1+\ldots+b_qn_q$ for some $a_i,b_j\in I$ and $m_i,n_j\in M$.
    By the submodule criterion (Proposition 1, Section 10.1), it suffices
    to show that $x+ry\in IM$. Now
    \begin{align*}
      x+ry &=a_1m_1+\ldots+a_pm_p + r(b_1n_1+\ldots+b_qn_q) \\
      &=a_1m_1+\ldots+a_pm_p + (rb_1)n_1+\ldots+(rb_q)n_q, \\
    \end{align*}
    which is contained in $IM$ since $rb_j\in I$ because $I$ is a left
    ideal of $R$.
  \end{proof}

\it \textbf{Section 10.1 Q8:} An element $m$ of the $R$-module $M$ is
  called a torsion element if $rm=0$ for some nonzero element $r\in R$. The
  set of torsion elements is denoted
  \[\text{Tor}(M) = \{m\in M| rm=0\; \text{for some nonzero}\; r\in R\}.\]

  \begin{enumerate}[label={(\alph*)}]
    \item \it Prove that if $R$ is an integral domain then $\text{Tor}(M)$
      is a submodule of $M$ (called the torsion submodule of $M$).

      \begin{proof}
        Let $x,y\in\text{Tor}(M)$ with $r_xx=r_yy=0$ for some nonzero
        $r_x,r_y\in R$. Let $r\in R$. By the submodule criterion, we need
        to show that $x+ry\in\text{Tor}(M)$. Now
        \begin{align*}
          r_xr_y(x+ry) &= r_xr_yx+r_xr_yry \\
          &=r_yr_xx+r_xrr_yy &(\text{from commutativity of integral domain}) \\
          &=r_y0+r_xr0 \\
          &=0+0 &(s0=0\; \text{for all}\; s\in R,\; \text{shown in Q3}) \\
          &=0.
        \end{align*}
        Also, because $R$ is an integral domain, $r_xr_y\neq0$, thus
        $x+ry\in\text{Tor}(M)$.
      \end{proof}

    \item \it Give an example of a ring $R$ and an $R$-module $M$ such that
      $\text{Tor}(M)$ is not a submodule. [Consider the torsion elements in
      the $R$-module $R$.]

      \begin{proof}
      \end{proof}

    \item \it If $R$ has zero divisors show that every nonzero $R$-module
      has nonzero torsion elements.
      \begin{proof}
        Let $r,s\in R$ be nonzero elements such that $rs=0$. Let $M$ be
        a nonzero $R$-module. If $sm=0$ for all $m\in M$, pick any
        nonzero $m\in M$. This $m$ will be a torsion element with $s$ as
        witness. If no such $m$ exists, then pick any nonzero $sm\in M$;
        then we have $r(sm)=(rs)m=0m=0$, which implies that $sm$ is a
        torsion element. In either cases we get a nonzero torsion element.
      \end{proof}
  \end{enumerate}
\end{document}
