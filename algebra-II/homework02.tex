\documentclass{article}
\usepackage[left=3cm,right=3cm,top=3cm,bottom=3cm]{geometry}
\usepackage{amsmath,amssymb,amsthm,pgfplots,tikz}
\usepackage[inline]{enumitem}
\usepackage{color}
\setlength{\parindent}{0mm} %So that we do not indent on new paragraphs
\newcommand{\TODO}[1]{\textcolor{red}{TODO: #1}}

\begin{document}
\title{Graduate Algebra II: Homework 2}
\author{Li Ling Ko\\ lko@nd.edu}
\date{\today}
\maketitle

In these exercises $R$ is a ring with 1 and $M$ is a left $R$-module. \\

\it \textbf{Section 10.3 Q4:} An $R$-module $M$ is called a torsion module
  if for each $m\in M$ there is a nonzero element $r\in R$ such that
  $rm=0$, where $r$ may depend on $m$. Prove that every finite abelian
  group is a torsion $\mathbb{Z}$-module. Give an example of an infinite
  abelian group that is a torsion $\mathbb{Z}$-module.

  \begin{proof}
    Let $(A,+)$ be a finite abelian group. Then $|A|\cdot a=0$ for each
    $a\in A$, therefore $A$ is a torsion $\mathbb{Z}$-module. Consider the
    infinite abelian group $\prod_{\mathbb{N}}\mathbb{Z}_2$. This is a
    $\mathbb{Z}$-module, and every element $m$ in it satisfies $2\cdot
    m=0$.
  \end{proof}
\end{document}
