\documentclass{article}
\usepackage[left=3cm,right=3cm,top=3cm,bottom=3cm]{geometry}
\usepackage{amsmath,amssymb,amsthm,pgfplots,tikz}
\usepackage[inline]{enumitem}
\usepackage{color}
\setlength{\parindent}{0mm} %So that we do not indent on new paragraphs
\newcommand{\TODO}[1]{\textcolor{red}{TODO: #1}}

\begin{document}
\title{Graduate Algebra II: Homework 2}
\author{Li Ling Ko\\ lko@nd.edu}
\date{\today}
\maketitle

\it \textbf{Q1:} Let $V$ be a vector space over $\mathbb{F}$, and let
  $\text{End}_{\mathbb{F}}(V)$ denote the associative $\mathbb{F}$-algebra
  of linear transformations from $V$ to $V$ under the usual composition of
  linear transformations. Define the bracket on $V$ to be $[x,y]=xy-yx$,
  (also called the commutator on $\text{End}_{\mathbb{F}}(V)$). Show that
  with this bracket $\text{End}_{\mathbb{F}}(V)$) is a Lie algebra.

  \begin{proof}
    Skew-symmetry: $[x,y]=xy-yx=-(yx-xy)=-[y,x]$. \\
    Jacobi identity: $[x,[y,z]]+[y,[z,x]]+[z,[x,y]]$
    \begin{align*}
      &=x(yz-zy)-(yz-zy)x +y(zx-xz)-(zx-xz)y +z(xy-yx)-(xy-yx)z\\
      &=0.\\
    \end{align*}
  \end{proof}

\it \textbf{Q2:} Now let $A$ be a (not necessarily associative)
  $\mathbb{F}$-algebras. Prove that the set $\text{Der}_{\mathbb{F}}(A)$ is
  a Lie subalgebra of $\text{End}_{\mathbb{F}}(A)$, under the bracket
  operation $[x,y]=xy-yx$. (Recall that $x\in\text{End}_{\mathbb{F}}(A)$ is
  a derivation if $x(ab)=x(a)b+ax(b)$ for all $a,b\in A$.)

  \begin{proof}
    Let $\alpha,\beta\in\text{Der}_{\mathbb{F}}(A)$ and $a,b\in A$. Then
    \begin{align*}
      [\alpha,\beta](ab) &=(\alpha\beta)(ab) -(\beta\alpha)(ab)\\
      &=\alpha(\beta(a)b+a\beta(b)) -\beta(\alpha(a)b+a\alpha(b))\\
      &=\alpha(\beta(a))b +\beta(a)\alpha(b) +\alpha(a)\beta(b)
        +a\alpha(\beta(b))\\
      &-\beta(\alpha(a))b -\alpha(a)\beta(b) -\beta(a)\alpha(b)
        -a\beta(\alpha(b))\\
      &=\alpha(\beta(a))b -\beta(\alpha(a))b + a\alpha(\beta(b))
        -a\beta(\alpha(b))\\
      &=(\alpha\beta-\beta\alpha)(a)b + a(\alpha\beta-\beta\alpha)(b)\\
      &=[\alpha,\beta](a)b + a[\alpha,\beta](b).\\
    \end{align*}
    Thus $f\text{Der}_{\mathbb{F}}(A)$ is a Lie subalgebra.
  \end{proof}

In these exercises $R$ is a ring with 1 and $M$ is a left $R$-module. \\

\it \textbf{Section 10.3 Q4:} An $R$-module $M$ is called a torsion module
  if for each $m\in M$ there is a nonzero element $r\in R$ such that
  $rm=0$, where $r$ may depend on $m$. Prove that every finite abelian
  group is a torsion $\mathbb{Z}$-module. Give an example of an infinite
  abelian group that is a torsion $\mathbb{Z}$-module.

  \begin{proof}
    Let $(A,+)$ be a finite abelian group. Then $|A|\cdot a=0$ for each
    $a\in A$, therefore $A$ is a torsion $\mathbb{Z}$-module. Consider the
    infinite abelian group $\prod_{\mathbb{N}}\mathbb{Z}_2$. This is a
    $\mathbb{Z}$-module, and every element in it is annihilated by
    $2\in\mathbb{Z}$.
  \end{proof}

\it \textbf{Section 10.3 Q5:} Let $R$ be an integral domain. Prove that
  every finitely generated torsion $R$-module has a nonzero annihilator
  i.e., there is a nonzero element $r\in R$ such that $rm=0$ for all $m\in
  M$ -- here $r$ does not depend on $m$. Give an example of a torsion
  $R$-module whose annihilator is the zero ideal.

  \begin{proof}
    Let $F=RA$ be a finitely generated torsion $R$-module, where
    $A=\{a_1,\ldots,a_n\}$. For $i=1,\ldots,n$, let $r_i\in R-\{0\}$
    annihilate $a_i$. Consider $r=r_1\ldots r_n$. Note that $r\neq0$ since
    $R$ has no zero divisors. Also, $r$ annihilates every element of $A$
    from commutativity of $R$. Finally, because $F$ is generated by $A$,
    $r$ would also annihilate every element of $RA$: every $m\in F$ can be
    written as $s_1\cdot a_1+\ldots s_n\cdot a_n$ for some $s_i\in R$. Then
    \begin{align*}
      r\cdot m &=r\cdot(s_1\cdot a_1+\ldots s_n\cdot a_n)\\
      &=(rs_1)\cdot a_1 +\ldots +(rs_n)\cdot a_n\\
      &=(s_1r)\cdot a_1 +\ldots +(s_nr)\cdot a_n\\
      &=s_1\cdot(r\cdot a_1) +\ldots +s_n\cdot(r\cdot a_n)\\
      &=s_1\cdot0 +\ldots +s_n\cdot0\\
      &=0 +\ldots +0\\
      &=0.\\
    \end{align*}

    Consider the $\mathbb{Q}$-module $\mathbb{Q}/\mathbb{Z}$. This is a
    torsion module because every nonzero $\bar{q}\in\mathbb{Q}/\mathbb{Z}$
    is annihilated by $1/q\in\mathbb{Q}$. However, $\mathbb{Q}/\mathbb{Z}$
    does not have a nonzero annihilator, because every nonzero
    $r\in\mathbb{Q}$ does not annihilate
    $\overline{1/(2r)}\in\mathbb{Q}/\mathbb{Z}$.
  \end{proof}

\it \textbf{Section 10.3 Q9:} An $R$-module $M$ is called irreducible if
  $M\neq0$ and if 0 and $M$ are the only submodules of $M$. Show that $M$
  is irreducible if and only if $M\neq0$ and $M$ is a cyclic module with
  any nonzero element as generator. Determine all the irreducible
  $\mathbb{Z}$-modules.

  \begin{proof}
    $\Rightarrow$: Let $M$ be an irreducible $R$-module. Then any nonzero
    $m\in M$ must generate the whole of $M$, otherwise $Rm$ will be a
    nontrivial submodule of $M$. \\

    $\Leftarrow$: Assume $M$ is a nonzero cyclic $R$-module with any
    nonzero element as generator, and let $N$ be an arbitrary submodule of
    $M$. Choose any $n\in N$ be any nonzero element of $N$. Then
    $M=Rn\subseteq N$, and also clearly $N\subseteq M$, thus $N=M$. \\

    Let $M$ be an irreducible $\mathbb{Z}$-module. We first show that $M$
    must be torsion. Otherwise let $m$ be a nonzero element in $M$. Then
    $M=\mathbb{Z}m=\mathbb{Z}(2\cdot m)$, so $m=(2k)\cdot m$ for some
    $k\in\mathbb{Z}$. Rearranging, we get $(2k-1)\cdot m=0$, and since
    $2k-1\neq0$ for any $k\in\mathbb{Z}$, $m$ must be torsion, a
    contradiction. \\

    Let $m\in M$. Then $m$ is a torsion element.  Let $k\in\mathbb{Z}$ be
    the smallest positive integer that annihilates $m$. Note that such an
    integer exists because if $m$ is annihilated by some negative integer
    $k$, then $(-k)\cdot m=-k\cdot m=0$. Now given arbitrary nonzero
    $m'=k'\cdot m\in M$, we have $m'=(k'\pmod{k})\cdot m$ since $k\cdot
    m=0$. Thus the unique elements of $M$ are exactly
    \[M =\{0,1\cdot m,2\cdot m,\ldots,k\cdot m\} =\mathbb{Z}_km
    \cong\mathbb{Z}_k.\]

    Now since irreducible $\mathbb{Z}$-modules are exactly the cyclic ones
    with any nonzero element as generator, every nonzero element in
    $\mathbb{Z}_k$ must generate $\mathbb{Z}_k$, which holds if and only if
    $k$ is prime. Thus the irreducible $\mathbb{Z}$-modules are exactly all
    $\mathbb{Z}_p$ for prime $p$. \\
  \end{proof}

\it \textbf{Section 10.3 Q10:} Assume $R$ is commutative. Show that an
  $R$-module $M$ is irreducible if and only if $M$ is isomorphic (as an
  $R$-module) to $R/I$ where $I$ is a maximal ideal of $R$. [By the
  previous exercise, if $M$ is irreducible there is a natural map
  $R\rightarrow M$ defined by $r\mapsto rm$, where $m$ is any fixed nonzero
  element of $M$.]

  \begin{proof}
    Let $M$ be an irreducible $R$-module, and $m$ a nonzero element in $M$.
    Note from the previous question that $m$ generates $M$. Consider the
    natural map $\pi_m:R\rightarrow M$, $r\mapsto r\cdot m$. This map is
    a $R$-module homomorphism: For any $r,x,y\in R$,
    $\pi_m(rx+y)=(rx+y)\cdot m=rx\cdot m+y\cdot m =r\cdot(x\cdot m)+y\cdot
    m =r\cdot\pi_m(x)+\pi_m(y)$. This map is also surjective because $m$
    generates $M$. Thus from the first isomorphism theorem, $M\cong
    R/\ker(\pi_m)$ as a $R$-module. It suffices to show that $\ker(\pi_m)$
    is a maximal ideal of $R$. Let $r\in R$ be an arbitrary element outside
    $\ker(\pi_m)$. Then since $M$ is irreducible, $r\cdot m$ generates $M$,
    from the claim in the previous question. Therefore, $m=r'\cdot(r\cdot
    m)=(r'r)\cdot m$ for some nonzero $r'\in R$. Equivalently,
    $(r'r-1)\cdot m=0$, so $r'r-1\in\ker(\pi_m)$, which implies
    $\bar{r}\bar{r'}=\bar{1}\in R\ker(\pi_m)$. Thus $R\ker(\pi_m)$ is a
    field, which is equivalent to $\ker(\pi_m)$ being a maximal ideal of
    $R$. \\

    Assume $M\cong R/I$, where $I$ is a maximal ideal of $R$. Then $R/I$ is
    a field. Let $\bar{r}\in R/I$ be an arbitrary nonzero element. By the
    previous question, it suffices to show that $\bar{r}$ generates $M$ as
    an $R$-module. Now since $R/I$ is a field, there exists some nonzero
    $\bar{r'}\in R/I$ such that $\bar{r'}\bar{r}=\overline{r'r}=\bar{1}$. Let
    $\bar{m}\in R/I$ be arbitrary. Then
    $\bar{m}=\overline{r'rm}=\overline{r'mr}=(r'm)\cdot\bar{r}$, thus
    $\bar{r}$ generates $\bar{m}$.
  \end{proof}

\it \textbf{Section 10.3 Q11:} Show that if $M_1$ and $M_2$ are irreducible
  $R$-modules, then any nonzero $R$-module homomorphism from $M_1$ to $M_2$
  is an isomorphism. Deduce that if $M$ is irreducible then
  $\text{End}_R(M)$ is a division ring (this result is called Schur's
  Lemma). [Consider the kernel and the image]

  \begin{proof}
    Let $\varphi:M_1\rightarrow M_2$ be nonzero $R$-module homomorphism
    between the two irreducible $R$-modules. Since $\ker(\varphi)$ is a
    submodule of $M_1$ and $M_1$ has no non-trivial submodules (Section
    10.3 Question 9), $\ker(\varphi)$ must be 0, otherwise $\varphi$ would
    be the zero homomorphism. Also, since $\varphi(M_1)$ is a submodule of
    $M_2$ which also has no non-trivial submodules, $\varphi(M_1)$ must be
    $M_2$, otherwise $\varphi$ would be the zero homomorphism. Thus
    $\varphi$ is surjective and injective, which makes it an isomorphism.
    \\

    Thus, any nonzero $\varphi\in\text{End}_R(M)$ will have a
    multiplicative inverse which sends $m\in M_1$ to $\varphi^{-1}(m)$.
  \end{proof}

\it \textbf{Section 10.3 Q12:} Let $R$ be a commutative ring and let $A,B$
  and $M$ be $R$-modules Prove the following isomorphism of $R$-modules:
  \begin{enumerate}[label={\bf(\alph*)}]
    \item $\text{Hom}_R(A\times B,M) \cong\text{Hom}_R(A,M)
      \times\text{Hom}_R(B,M)$

      \begin{proof}
        Consider the map $\Theta:\text{Hom}_R(A,M)\times\text{Hom}_R(B,M)
        \rightarrow \text{Hom}_R(A\times B,M)$, defined by
        \[\Theta(\alpha,\beta)(a,b) :=\alpha(a)+\beta(b).\]
        We first show that $\Theta(\alpha,\beta) \in\text{Hom}_R(A\times
        B,M)$: Given $r\in R$, $x=(a_1,b_1),y=(a_2,b_2)\in A\times B$, we
        have
        \begin{align*}
          \Theta(\alpha,\beta)(rx+y) &=\Theta(\alpha,\beta)(r\cdot
            a_1+a_2,r\cdot b_1+b_2)\\
          &=\alpha(r\cdot a_1+a_2) +\beta(r\cdot b_1+b_2)\\
          &=\alpha(r\cdot a_1)+\beta(r\cdot b_1) +\alpha(a_2)+\beta(b_2)\\
          &=r\cdot(\alpha(a_1)+\beta(b_1)) +\alpha(a_1)+\beta(b_2)\\
          &=r\cdot\Theta(\alpha,\beta)(x) +\Theta(\alpha,\beta)(y).\\
        \end{align*}

        Next, we show that $\Theta$ is an $R$-module. Let $r\in R$,
        $\gamma_1=(\alpha_1,\beta_1), \gamma_2=(\alpha_2,\beta_2)\in
        \text{Hom}_R(A,M)\times\text{Hom}_R(B,M)$, and $(a,b)\in A\times
        B$. Then
        \begin{align*}
          \Theta(r\gamma_1+\gamma_2)(a,b)
            &=\Theta((r\alpha_1+r\alpha_2,\beta_1+\beta_2))(a,b)\\
          &=(r\alpha_1+r\alpha_2)(a) +(\beta_1+\beta_2)(b)\\
          &=r\cdot\alpha_1(a)+r\cdot\alpha_2(a) +\beta_1(b)+\beta_2(b)\\
          &=r\cdot\Theta(\gamma_1)(a,b) +\Theta(\gamma_2)(a,b).\\
        \end{align*}

        Now we show that $\Theta$ is injective. Let
        $(\alpha,\beta)\in\ker(\Theta)$. Then $\alpha(a)+\beta(b)=0$ for
        all $a\in A$, $b\in B$. In particular, setting $b=0$, $\alpha(a)=0$
        for all $a\in A$, which means $\alpha\in\text{Hom}_R(A,M)$ is the
        zero homomorphism. Similarly, setting $a=0$, we get
        $\beta\in\text{Hom}_R(B,M)$ is the zero homomorphism. Thus
        $(\alpha,\beta)=0$. Since $(\alpha,\beta)\in\ker(\Theta)$ was
        arbitrary, $\Theta$ is injective. \\

        Finally we show that $\Theta$ is surjective. Let $\gamma:A\times
        B\rightarrow M$ be an $R$-module homomorphism. Consider the map
        $\alpha:A\rightarrow M$ defined by $\alpha(a)=\gamma((a,0))$. Then
        $\alpha$ is an $R$-module homomorphism, because given $r\in R$ and
        $a_1,a_2\in A$,
        \begin{align*}
          \alpha(r\cdot a_1+a_2) &=\gamma((r\cdot a_1+a_2,0)\\
          &=r\cdot\gamma((a_1,0)) +\gamma((a_2,0))\\
          &=r\cdot\alpha(a_1) +\alpha(a_2).\\
        \end{align*}

        Similarly, the map $\beta:B\rightarrow M$ defined by
        $\beta(b)=\gamma((0,b))$ is an $R$-module homomorphism. We show
        that $(\alpha,\beta)$ is a pre-image of $\gamma$: Given any $a\in
        A$ and $b\in B$,
        \begin{align*}
          \Theta((\alpha,\beta))(a,b) &=\alpha(a)+\beta(b)\\
          &=\gamma((a,0)) +\gamma((0,b))\\
          &=\gamma((a,b)).\\
        \end{align*}
      \end{proof}

    \item $\text{Hom}_R(M,A\times B) \cong\text{Hom}_R(M,A)
      \times\text{Hom}_R(M,B)$

      \begin{proof}
        Consider the map $\Theta:\text{Hom}_R(M,A) \times\text{Hom}_R(M,B)
        \rightarrow \text{Hom}_R(M,A\times B)$ defined by
        \[\Theta((\alpha,\beta))(m) :=(\alpha(m),\beta(m)).\]

        We first show that $\Theta((\alpha,\beta))\in\text{Hom}_R(M,A\times
        B)$: Given $r\in R$ and $x,y\in M$,
        \begin{align*}
           \Theta((\alpha,\beta))(r\cdot x+y) &=(\alpha(r\cdot x+y),
            \beta(r\cdot x+y))\\
          &=(r\cdot\alpha(x)+\alpha(y), r\cdot\beta(x)+\beta(y))\\
          &=r\cdot(\alpha(x),\beta(x)) +(\alpha(y),\beta(y))\\
          &=r\cdot\Theta((\alpha,\beta))(x) +\Theta((\alpha,\beta))(y).\\
        \end{align*}

        Next, we show that $\Theta$ is an $R$-module homomorphism: Given
        $r\in R$, $\gamma_1=(\alpha_1,\beta_1),
        \gamma_2=(\alpha_2,\beta_2)\in \text{Hom}_R(M,A)
        \times\text{Hom}_R(M,B)$, and $m\in M$,
        \begin{align*}
          \Theta(r\cdot\gamma_1+\gamma_2)(m)
            &=\Theta((r\cdot\alpha_1+\beta_1, r\cdot\alpha_2+\beta_2))(m)\\
          &=((r\cdot\alpha_1+\beta_1)(m), (r\cdot\alpha_2+\beta_2)(m))\\
          &=((r\cdot\alpha_1(m)+\beta_1(m), r\cdot\alpha_2(m)+\beta_2(m))\\
          &=(r\cdot\alpha_1(m), r\cdot\alpha_2(m)) +(\beta_1(m),\beta_2(m))\\
          &=r\cdot(\alpha_1(m),\alpha_2(m)) +(\beta_1(m),\beta_2(m))\\
          &=r\cdot\Theta(\gamma_1)(m) +\Theta(\gamma_2)(m)\\
          &=(r\cdot\Theta(\gamma_1)+\Theta(\gamma_2))(m).\\
        \end{align*}

        Next, we show that $\Theta$ is injective. Let
        $(\alpha,\beta)\in\ker(\Theta)$. Then for all $m\in M$,
        $(\alpha(m),\beta(m))=(0,0)$, which means $\alpha$ and $\beta$ are
        the zero $R$-homomorphisms. Thus $(\alpha,\beta)=0$, and $\Theta$
        is injective. \\

        Finally, we show that $\Theta$ is surjective. Let
        $\gamma:M\rightarrow A\times B$ be an arbitrary $R$-homomorphism.
        Consider the map $\alpha:M\rightarrow A$ defined by
        $\alpha(m)=\pi_1(\gamma(m))$, where $\pi_i$ is the $i$-th
        projection map. Then $\gamma$ is an $R$-module homomorphism.
        Similarly, the map $\beta:M\rightarrow B$ defined by
        $\beta(m)=\pi_2(\gamma(m))$ is an $R$-module homomorphism. Also,
        $(\alpha,\beta)$ is a pre-image of $\gamma$ under $\Theta$: Given
        $m\in M$,
        \begin{align*}
          \Theta((\alpha,\beta))(m) &=(\alpha(m),\beta(m))\\
          &=\gamma(m).\\
        \end{align*}
      \end{proof}
  \end{enumerate}

\it \textbf{Section 10.3 Q14:} Let $R$ be a commutative ring and let $F$ be
  the free $R$-module of rank $n$. Prove that $\text{Hom}_R(F,M) \cong
  M\times\cdots\times M$ ($n$ times).

  \begin{proof}
    From Theorem 6 of Section 10.3, we can assume that $F=R^n$. Now by
    induction on $n$ and by Section 10.3 Question 12a, we get
    \[\text{Hom}_R(R^n,M) \cong\text{Hom}_R(R,M)^n.\]
    Then from Section 10.2 Question 9, the formula on the right is
    isomorphic as an $R$-module to $M^n$.
  \end{proof}
\end{document}
