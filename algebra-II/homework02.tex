\documentclass{article}
\usepackage[left=3cm,right=3cm,top=3cm,bottom=3cm]{geometry}
\usepackage{amsmath,amssymb,amsthm,pgfplots,tikz}
\usepackage[inline]{enumitem}
\usepackage{color}
\setlength{\parindent}{0mm} %So that we do not indent on new paragraphs
\newcommand{\TODO}[1]{\textcolor{red}{TODO: #1}}

\begin{document}
\title{Graduate Algebra II: Homework 2}
\author{Li Ling Ko\\ lko@nd.edu}
\date{\today}
\maketitle

In these exercises $R$ is a ring with 1 and $M$ is a left $R$-module. \\

\it \textbf{Section 10.3 Q4:} An $R$-module $M$ is called a torsion module
  if for each $m\in M$ there is a nonzero element $r\in R$ such that
  $rm=0$, where $r$ may depend on $m$. Prove that every finite abelian
  group is a torsion $\mathbb{Z}$-module. Give an example of an infinite
  abelian group that is a torsion $\mathbb{Z}$-module.

  \begin{proof}
    Let $(A,+)$ be a finite abelian group. Then $|A|\cdot a=0$ for each
    $a\in A$, therefore $A$ is a torsion $\mathbb{Z}$-module. Consider the
    infinite abelian group $\prod_{\mathbb{N}}\mathbb{Z}_2$. This is a
    $\mathbb{Z}$-module, and every element in it is annihilated by
    $2\in\mathbb{Z}$.
  \end{proof}

\it \textbf{Section 10.3 Q5:} Let $R$ be an integral domain. Prove that
  every finitely generated torsion $R$-module has a nonzero annihilator
  i.e., there is a nonzero element $r\in R$ such that $rm=0$ for all $m\in
  M$ -- here $r$ does not depend on $m$. Give an example of a torsion
  $R$-module whose annihilator is the zero ideal.

  \begin{proof}
    Let $F=RA$ be a finitely generated torsion $R$-module, where
    $A=\{a_1,\ldots,a_n\}$. For $i=1,\ldots,n$, let $r_i\in R-\{0\}$
    annihilate $a_i$. Consider $r=r_1\ldots r_n$. Note that $r\neq0$ since
    $R$ has no zero divisors. Also, $r$ annihilates every element of $A$
    from commutativity of $R$. Finally, because $F$ is generated by $A$,
    $r$ would also annihilate every element of $RA$: every $m\in F$ can be
    written as $s_1\cdot a_1+\ldots s_n\cdot a_n$ for some $s_i\in R$. Then
    \begin{align*}
      r\cdot m &=r\cdot(s_1\cdot a_1+\ldots s_n\cdot a_n)\\
      &=(rs_1)\cdot a_1 +\ldots +(rs_n)\cdot a_n\\
      &=(s_1r)\cdot a_1 +\ldots +(s_nr)\cdot a_n\\
      &=s_1\cdot(r\cdot a_1) +\ldots +s_n\cdot(r\cdot a_n)\\
      &=s_1\cdot0 +\ldots +s_n\cdot0\\
      &=0 +\ldots +0\\
      &=0.\\
    \end{align*}

    Consider the $\mathbb{Q}$-module $\mathbb{Q}/\mathbb{Z}$. This is a
    torsion module because every nonzero $\bar{q}\in\mathbb{Q}/\mathbb{Z}$
    is annihilated by $1/q\in\mathbb{Q}$. However, $\mathbb{Q}/\mathbb{Z}$
    does not have a nonzero annihilator, because every nonzero
    $r\in\mathbb{Q}$ does not annihilate
    $\overline{1/(2r)}\in\mathbb{Q}/\mathbb{Z}$.
  \end{proof}
\end{document}
