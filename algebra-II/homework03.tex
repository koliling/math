\documentclass{article}
\usepackage[left=3cm,right=3cm,top=3cm,bottom=3cm]{geometry}
\usepackage{amsmath,amssymb,amsthm,pgfplots,tikz}
\usepackage[inline]{enumitem}
\usepackage{color}
\setlength{\parindent}{0mm} %So that we do not indent on new paragraphs
\newcommand{\TODO}[1]{\textcolor{red}{TODO: #1}}

\begin{document}
\title{Graduate Algebra II: Homework 3}
\author{Li Ling Ko\\ lko@nd.edu}
\date{\today}
\maketitle

\it \textbf{Q1:} Let $A$ and $B$ be abelian groups, and $m,n$ positive
  integers.
  \begin{enumerate}[label={(\alph*)}]
    \item Show that $A\otimes_\mathbb{Z}\mathbb{Z}_m \cong A/mA$.
      \begin{proof}
        Consider the map $\varphi:A\times\mathbb{Z}_m
        \rightarrow A/mA$ defined by
        $(a,\bar{z})\mapsto\overline{z\cdot a}$. We first show this map
        is well-defined and $\mathbb{Z}$-bilinear. To show that the map is
        well-defined, we have
        \begin{align*}
          \varphi((a,\overline{z+m})) &=\overline{(z+m)\cdot a}\\
          &=\overline{z\cdot a+m\cdot a}\\
          &=\overline{z\cdot a}\\
          &=\varphi((a,\overline{z})).\\
        \end{align*}

        To show $\mathbb{Z}$-bilinearlity, we have
        \begin{align*}
          \varphi(((a_1+a_2),\overline{z})) &=\overline{z\cdot
            (a_1+a_2)}\\
          &=\overline{z\cdot a_1 +z\cdot a_2}\\
          &=\overline{z\cdot a_1} +\overline{z\cdot a_2}\\
          &=\varphi((a_1,\overline{z}))
            +\varphi((a_2,\overline{z})),\\
        \end{align*}

        and
        \begin{align*}
          \varphi((a,(\overline{z_1}+\overline{z_2})))
            &=\varphi((a,(\overline{z_1+z_2})))\\
          &=\overline{(z_1+z_2)\cdot a}\\
          &=\overline{z_1\cdot a +z_2\cdot a}\\
          &=\overline{z_1\cdot a} +\overline{z_2\cdot a}\\
          &=\varphi((a,\overline{z_1}))
            +\varphi((a,\overline{z_2})),\\
        \end{align*}

        and
        \begin{align*}
          \varphi((n\cdot a,\overline{z})))
          &=\overline{z\cdot (n\cdot a)}\\
          &=\overline{(nz)\cdot a}\\
          &=\overline{(n\cdot\overline{z})\cdot a}\\
          &=\varphi((a, (n\cdot \overline{z}))),\\
        \end{align*}

        and
        \begin{align*}
          \varphi((a,n\cdot\overline{z}))
          &=\varphi((a,\overline{nz}))\\
          &=\overline{(nz)\cdot a}\\
          &=\overline{(n\cdot\overline{z})\cdot a}\\
          &=\varphi((a, (n\cdot \overline{z}))).\\
        \end{align*}

        Thus from Corollary 12, $\varphi$ induces a $\mathbb{Z}$-module
        homomorphism $\Phi:A\otimes_\mathbb{Z}\mathbb{Z}_m \rightarrow
        A/mA$ defined by $\Phi(a\otimes\bar{z})=\overline{z\cdot a}$. Now
        $\Phi$ is clearly surjective since every $\overline{a}\in
        A/mA$ has a pre-image $a\otimes\bar{1}$. It remains to show that
        $\Phi$ is injective. Now by definition, every element $x$ of
        $A\otimes_\mathbb{Z}\mathbb{Z}_m$ can be written
        \[x=\sum_{i=1}^n a_i\otimes\overline{z_i}\]
        for some $a_i\in A$, $\overline{z_i}\in\mathbb{Z}_m$. Thus
        \begin{align*}
          \;&x\in\ker(\Phi)\\
          \Leftrightarrow\;&\sum_{i=1}^n a_i\otimes\overline{z_i}
            \in\ker(\Phi)\\
          \Leftrightarrow\;&\sum_{i=1}^n \overline{z_i\cdot
            a_i}=\overline{0}\in A/mA\\
          \Leftrightarrow\;&\overline{\sum_{i=1}^n z_i\cdot
            a_i}=\overline{0}\in A/mA\\
          \Leftrightarrow\;&\sum_{i=1}^n z_i\cdot a_i\in mA\\
          \Leftrightarrow\;&\sum_{i=1}^n z_i\cdot a_i =m\cdot b\in A\;
            \text{for some}\; b\in A\\
          \Rightarrow\;&\left(\sum_{i=1}^n z_i\cdot a_i\right)
            \otimes\bar{1}=(m\cdot b)\otimes\bar{1}\; \text{for some}\;
            b\in A.\\
          \Rightarrow\;&\sum_{i=1}^n (z_i\cdot a_i)\otimes\bar{1} =(m\cdot
            b)\otimes\bar{1}\; \text{for some}\; b\in A.\\
        \end{align*}

        Now by $\mathbb{Z}$-bilinear property of tensors, given arbitrary
        $a\in A$ and $z\in\mathbb{Z}_m$,
        \begin{align*}
          &\;(z\cdot a)\otimes\bar{1}\\
          =&\;a\otimes(z\cdot\bar{1})\\
          =&\;a\otimes\bar{z},\\
        \end{align*}
        thus we can simplify the above conditions to get
        \[\sum_{i=1}^n a_i\otimes\overline{z_i} =b\otimes\overline{m}=0,\]
        which implies $x=0$. Thus $\ker(\Phi)=0$.
      \end{proof}

    \item Show that $\mathbb{Z}_m\otimes_\mathbb{Z}\mathbb{Z}_n
      \cong\mathbb{Z}_c$, where $c=(m,n)$.
      \begin{proof}
        By replacing $m$ with $n$ in part (a) of this question letting
        $A=\mathbb{Z}_m$, it suffices to show that
        $\mathbb{Z}_m/n\mathbb{Z}_m\cong \mathbb{Z}_c$ as
        $\mathbb{Z}$-modules. First, observe that
        $n\mathbb{Z}_m=c\mathbb{Z}$: $c$ is the smallest positive integer
        such that there exists $x_0,y_0\in\mathbb{Z}$ with $c=x_0m+y_0n$.
        Thus in $\mathbb{Z}_m$,
        $\overline{c}=\overline{x_0m+y_0n}=\overline{y_0n}=n\overline{y_0}
        \in n\mathbb{Z}_m$. Also, since all elements in $n\mathbb{Z}_m$
        should be of the form $xm+yn$ for some $x,y\in\mathbb{Z}$, and $c$
        is the smallest positive integer of such form, and $n\mathbb{Z}_m$
        is closed under multiplication modulo $m$, $n\mathbb{Z}_m$ must
        equal
        \[n\mathbb{Z}_m =\{\overline{0}, \overline{c},
        \overline{2c}, \ldots, \overline{m-c}\}=c\mathbb{Z}_m.\]

        Therefore
        \[\mathbb{Z}_m/n\mathbb{Z}_m =\mathbb{Z}_m/c\mathbb{Z}_m
        \cong\mathbb{Z}_c.\]
      \end{proof}

    \item If $A$ is a torsion group (i.e. a torsion $\mathbb{Z}$-module),
      then $A\otimes_\mathbb{Z}\mathbb{Q}=0$.

      \begin{proof}
        Every element in $x\in A\otimes_\mathbb{Z}\mathbb{Q}$ can be
        written as a finite sum
        \[x=\sum_{i=1}^k a_i\otimes q_i\]
        for some $a_i\in A$, $q_i\in \mathbb{Q}$ by definition of torsion
        products. Since $A$ is torsion, for each $i=\{1,\ldots,k\}$,
        there exists $n_i$ such that $n_i\cdot a_i=0$. Let $n=n_1\cdots
        n_k$. Then $n\cdot a_i=0$ for all $i=\{1,\ldots,k\}$. So
        \begin{align*}
          &\;a_i\otimes q_i\\
          =&\;a_i\otimes(n\cdot(q_i/n))\\
          =&\;(n\cdot a_i)\otimes(q_i/n)\\
          =&\;0\otimes(q_i/n)\\
          =&\;(0\cdot0)\otimes(q_i/n)\\
          =&\;0\otimes(0\cdot q_i/n)\\
          =&\;0\otimes0\\
          =&\;0,\\
        \end{align*}
        thus
        \[x=\sum_{i=1}^k a_i\otimes q_i =\sum_{i=1}^k 0=0.\]
      \end{proof}
  \end{enumerate}

\it \textbf{Q2:} What is the difference between the homomorphism $f\otimes
  g$ as defined in class, and the element $f\otimes g\in\text{Hom}_R(A,A')
  \otimes_\mathbb{Z}\text{Hom}_R(B,B')$?
  \begin{proof}
  \end{proof}

\it \textbf{Q3:}
  \begin{enumerate}[label={(\alph*)}]
    \item From Exercise 1, determine what
      $\mathbb{Z}_2\otimes_\mathbb{Z}\mathbb{Z}_2$ is as an abelian group,
      and what is $\mathbb{Z}_2\otimes_\mathbb{Z}\mathbb{Z}_4$ as an
      abelian group.

      \begin{proof}
        From Question 1b, $\mathbb{Z}_2\otimes_\mathbb{Z}\mathbb{Z}_2
        \cong\mathbb{Z}_2$ as a $\mathbb{Z}$-module, thus it is isomorphic
        to $\mathbb{Z}_2$ as an abelian group. Similarly,
        $\mathbb{Z}_2\otimes_\mathbb{Z}\mathbb{Z}_4$ is isomorphic to
        $\mathbb{Z}_2$ as an abelian group.
      \end{proof}

    \item Let $\phi:\mathbb{Z}_2\rightarrow\mathbb{Z}_4$ be the usual
      injection of abelian groups. Show that $1\otimes\phi:
      \mathbb{Z}_2\otimes_\mathbb{Z}\mathbb{Z}_2
      \rightarrow\mathbb{Z}_2\otimes_\mathbb{Z}\mathbb{Z}_4$, is the zero
      map, even though $\mathbb{Z}_2\otimes_\mathbb{Z}\mathbb{Z}_2$ and
      $\mathbb{Z}_2\otimes_\mathbb{Z}\mathbb{Z}_4$ are nontrivial.

      \begin{proof}
        Given arbitrary $a,b\in\mathbb{Z}_2$,
        \begin{align*}
          &\;(1\otimes\phi)(\overline{a}\otimes \overline{b})\\
          =&\;\overline{a}\otimes \overline{2b}
            \in\mathbb{Z}_2\otimes_\mathbb{Z}\mathbb{Z}_4\\
          =&\;\overline{2a}\otimes \overline{b}\\
          =&\;\overline{0}\otimes \overline{b} &(\because\; 2|2a)\\
          =&\;\overline{0}\otimes \overline{0}\\
          =&\;0\in\mathbb{Z}_2\otimes_\mathbb{Z}\mathbb{Z}_4.\\
        \end{align*}

        So since elements of $\mathbb{Z}_2\otimes_\mathbb{Z}\mathbb{Z}_2$
        are finite sums of elements of the form $a\otimes b$ with
        $a,b\in\mathbb{Z}_2$ and homomorphisms preserve addition, the image
        of any element in $\mathbb{Z}_2\otimes_\mathbb{Z}\mathbb{Z}_2$ will
        be a finite sum of $0$'s, which will remain 0.
      \end{proof}
  \end{enumerate}

\textbf{Let $R$ be a ring with 1.} \\

\it \textbf{Section 10.4 Q3:} Show that
  $\mathbb{C}\otimes_\mathbb{R}\mathbb{C}$ and
  $\mathbb{C}\otimes_\mathbb{C}\mathbb{C}$ are both left
  $\mathbb{R}$-modules but are not isomorphic as $\mathbb{R}$-modules.

  \begin{proof}
    $\mathbb{C}\otimes_\mathbb{R}\mathbb{C}$ is a left $\mathbb{R}$-module
    because $\mathbb{C}$ is trivally a $\mathbb{R}$-module. Similarly,
    $\mathbb{C}\otimes_\mathbb{C}\mathbb{C}$ is a left $\mathbb{C}$-module
    because $\mathbb{C}$ is trivially a $\mathbb{C}$-module. Then since
    $\mathbb{R}$ is a subring of $\mathbb{C}$ with the same unity,
    $\mathbb{C}\otimes_\mathbb{C}\mathbb{C}$ will also be a left
    $\mathbb{R}$-module. \\

    We first show that $\mathbb{C}\otimes_\mathbb{R}\mathbb{C}$ is
    isomorphic to $\mathbb{R}^4$ as $\mathbb{R}$-modules. Consider the map
    $\varphi:\mathbb{C}\times\mathbb{C}\rightarrow\mathbb{R}^4$ defined by
    $\varphi((a_0+b_0i,a_1+b_1i)) =(a_0a_1,a_0b_1,a_1b_0,b_0b_1)$, where
    $a_0,b_0,a_1,b_1\in\mathbb{R}$. It is routine to show that $\varphi$ is
    a $\mathbb{R}$-bilinear map: Given
    $r,a_0,b_0,a_1,b_1,c_0,d_0,c_1,d_1\in\mathbb{R}$,
    \begin{align*}
      \;&\varphi((ra_0+rb_0i,a_1+b_1i))\\
      =\;&(ra_0a_1,ra_0b_1,ra_1b_0,rb_0b_1)\\
      =\;&r\cdot(a_0a_1,a_0b_1,a_1b_0,b_0b_1)\\
      =\;&\varphi((a_0+b_0i,ra_1+rb_1i)),\\
      =\;&r\cdot(a_0a_1,a_0b_1,a_1b_0,b_0b_1)\\
      =\;&r\cdot\varphi((a_0+b_0i,a_1+b_1i)),\\
    \end{align*}

    and,
    \begin{align*}
      \;&\varphi((a_0+b_0i+c_0+d_0i, a_1+b_1i))\\
      =\;&(a_0a_1+c_0a_1, a_0b_1+c_0b_1, a_1b_0+a_1d_0, b_0b_1+d_0b_1)\\
      =\;&(a_0a_1,a_0b_1,a_1b_0,b_0b_1) +(c_0a_1,c_0b_1,a_1d_0,d_0b_1)\\
      =\;&\varphi((a_0+b_0i, a_1+b_1i)) +\varphi((c_0+d_0i, a_1+b_1i)).\\
    \end{align*}
    Similarly, we can show that
    \[\varphi((a_0+b_0i, a_1+b_1i+c_1+d_1i))
    =\varphi((a_0+b_0i, a_1+b_1i)) +\varphi((a_0+b_0i, c_1+c_1i)).\]

    Thus from Corollary 12, $\varphi$ induces an $\mathbb{R}$-module
    homomorphism $\Phi:\mathbb{C}\otimes_\mathbb{R}\mathbb{C}\rightarrow
    \mathbb{R}^4$ defined by
    $\Phi((a_0+b_0i)\otimes(a_1+b_1i))=(a_0,b_0,a_1,b_1)$. This map is
    trivially surjective, since arbitrary
    $(r_0,r_1,r_2,r_3)\in\mathbb{R}^4$ as pre-image
    $(r_0+r_1i)\otimes(r_2+r_3i)$. Thus the dimension of
    $\mathbb{C}\otimes_\mathbb{R}\mathbb{C}$ as an $\mathbb{R}$-module must
    be at least the dimension of $\mathbb{R}^4$, which is 4. \\

    Next, observe that $\mathbb{C}\otimes_\mathbb{C}\mathbb{C}$ can be
    spanned as an $\mathbb{R}$-module by two elements
    $\{1\otimes1,1\otimes i\}$, because given $a,b,c,d\in\mathbb{R}$, we can
    write
    \begin{align*}
      &\;(a+bi)\otimes(c+di)\\
      =&\;a\otimes c+a\otimes di +bi\otimes c +bi\otimes di\\
      =&\;ac\cdot(1\otimes1) +ad\cdot(1\otimes i) +bc\cdot(i\otimes1)
        +bd\cdot(i\otimes i)\\
      =&\;ac\cdot(1\otimes1) +ad\cdot(1\otimes i) +bc\cdot(1\otimes i)
        -bd\cdot(1\otimes1)\\
      =&\;(ac-bd)\cdot(1\otimes1) +(ad+bc)\cdot(1\otimes i),\\
    \end{align*}
    where the second last equality follows because $1\otimes i=i\otimes1$
    and $i\otimes i=-1\otimes1$ as the tensor is defined with respect to
    $\mathbb{C}$. \\

    Thus $\mathbb{C}\otimes_\mathbb{R}\mathbb{C}$ has $\mathbb{R}$-module
    dimension no greater than 2, while
    $\mathbb{C}\otimes_\mathbb{C}\mathbb{C}$ has $\mathbb{R}$-module
    dimension at least 4, so they cannot be isomorphic as
    $\mathbb{R}$-modules.
  \end{proof}

\it \textbf{Section 10.4 Q5:} Let $A$ be a finite abelian group of order
  $n$ and let $p^k$ be the largest power of the prime $p$ dividing $n$.
  Prove that $\mathbb{Z}_{p^k}\otimes_\mathbb{Z}A$ is isomorphic to
  the Sylow $p$-subgroup of $A$.

  \begin{proof}
    Let the elementary divisor decomposition of $A$ be
    \[A\cong P\oplus_{i=1}^m\mathbb{Z}_{q_i^{\alpha_i}},\]
    where the $q_i$'s are distinct primes from $p$, and
    \[P=\mathbb{Z}_{p^{\alpha_1}}
    \oplus\ldots\oplus\mathbb{Z}_{p^{\alpha_n}}\]
    is the Sylow $p$-subgroup of $A$. Note that the decomposition is
    expressed as direct sum of $\mathbb{Z}$-modules. Then
    \begin{align*}
      &\;\mathbb{Z}_{p^k}\otimes_{\mathbb{Z}}A\\
      =&\;\mathbb{Z}_{p^k}\otimes_{\mathbb{Z}}
        (P\oplus_{i=1}^m \mathbb{Z}_{q_i^{\alpha_i}})\\
      =&\;(\mathbb{Z}_{p^k}\otimes_{\mathbb{Z}}P)
        \oplus(\mathbb{Z}_{p^k}
        \otimes_{\mathbb{Z}}\mathbb{Z}_{q_1^{\alpha_1}}) \oplus\ldots
        \oplus(\mathbb{Z}_{p^k}
        \otimes_{\mathbb{Z}}\mathbb{Z}_{q_m^{\alpha_m}}).
        &(\text{distributive property of tensors})\\
    \end{align*}

    Now for each $i\in\{1,\ldots,m\}$, since $p$ and $q_i$ are distinct
    primes, we have from Question 1b that
    \[\mathbb{Z}_{p^k} \otimes_{\mathbb{Z}}\mathbb{Z}_{q_m^{\alpha_m}}
    \cong\mathbb{Z}_1.\]

    Therefore
    \begin{align*}
      &\;\mathbb{Z}_{p^k}\otimes_{\mathbb{Z}}A\\
      =&\;(\mathbb{Z}_{p^k}\otimes_{\mathbb{Z}}P)
        \oplus(\mathbb{Z}_{p^k}
        \otimes_{\mathbb{Z}}\mathbb{Z}_{q_1^{\alpha_1}}) \oplus\ldots
        \oplus(\mathbb{Z}_{p^k}
        \otimes_{\mathbb{Z}}\mathbb{Z}_{q_m^{\alpha_m}})\\
      =&\;(\mathbb{Z}_{p^k}\otimes_{\mathbb{Z}}P) \oplus\mathbb{Z}_1
        \oplus\ldots \oplus\mathbb{Z}_1\\
      =&\;\mathbb{Z}_{p^k}\otimes_{\mathbb{Z}}P.\\
    \end{align*}
  \end{proof}

\it \textbf{Section 10.4 Q7:} If $R$ is any integral domain with quotient
  field $Q$ and $N$ is a left $R$-module, prove that every element of the
  tensor product $Q\otimes_R N$ can be written as a simple tensor of the
  form $(1/d)\otimes n$ for some nonzero $d\in R$ and some $n\in N$.

  \begin{proof}
    Let $q\in Q$ and $n'\in N$ be arbitrary. Then $q=c/d$ for some $c,d\in
    R$ and $d\neq0$, thus
    \begin{align*}
      &\;q\otimes n'\\
      =&\;(c/d)\otimes n'\\
      =&\;c\cdot((1/d)\otimes n')\\
      =&\;(1/d)\otimes (c\cdot n'),\\
    \end{align*}
    which is of the desired form. \\

    Now since elements of $Q\otimes_R N$ can
    be written as a finite sum of elements of the form $q\otimes n$ for
    some $q\in Q$ and $n\in N$, by induction on the number of sums, it
    suffices to show the sum of two elements $(1/d_1)\otimes n_1$ and
    $(1/d_2)\otimes n_2$ is of the desired form:
    \begin{align*}
      &\;(1/d_1)\otimes n_1 +(1/d_2)\otimes n_2\\
      =&\;(1/(d_1d_2))\otimes d_2n_1 +(1/(d_1d_2))\otimes d_1n_2
      &(\text{possible since}\; d_1,d_2\neq0)\\
      =&\;(1/(d_1d_2))\otimes(d_2n_1+d_1n_2),\\
    \end{align*}
    which is of the desired form since $d_1d_2\neq0$ because $R$ is an
    integral domain.
  \end{proof}

\it \textbf{Section 10.4 Q11:} Let $\{e_1,e_2\}$ be a basis of
  $V=\mathbb{R}^2$. Show that the element $e_1\otimes e_2+e_2\otimes e_1$
  in $V\otimes_\mathbb{R}V$ cannot be written as a simple tensor $v\otimes
  w$ for any $v,w\in\mathbb{R}^2$.

  \begin{proof}
    Consider the map $\varphi:V\times V\rightarrow\mathbb{R}^2$ defined by
    $\varphi\left(\left(\binom{r}{s},\binom{x}{y}\right)\right)
    =\binom{rx}{ry}$. It is routine to check that $\varphi$ is an
    $\mathbb{R}$-bilinear map. Thus by Corollary 12, $\varphi$ induces an
    $\mathbb{R}$-homomorphism
    $\Phi:V\otimes_\mathbb{R}V\rightarrow\mathbb{R}^2$ defined by
    $\Phi\left(\binom{r}{s}\otimes\binom{x}{y}\right) =\binom{rx}{ry}$.
    Thus if $\binom{r_1}{s_1}\otimes\binom{x_1}{y_1}
    =\binom{r_2}{s_2}\otimes\binom{x_2}{y_2} \in V\otimes V$, their image
    under $\Phi$ should be the same. In other words, we must have
    $r_1\binom{x_1}{y_1} =r_2\binom{x_2}{y_2}$. By a similar argument,
    swapping the roles of $s$ and $r$, we also get that if
    $\binom{r_1}{s_1}\otimes\binom{x_1}{y_1}
    =\binom{r_2}{s_2}\otimes\binom{x_2}{y_2} \in V\otimes V$ then
    $s_1\binom{x_1}{y_1} =s_2\binom{x_2}{y_2}$. We use these observations
    to check if two elements in $V\otimes V$ can be the same. \\

    In particular, if $\binom{r_1}{s_1}\otimes e_1 =\binom{r_2}{s_2}\otimes
    e_2$, then we must have $r_1\cdot e_1=r_2\cdot e_2$ and $s_1\cdot
    e_1=s_2\cdot e_2$. However, since $e_1$ and $e_2$ are linearly
    independent, we must have $r_1=r_2=s_1=s_2=0$. Thus $a\otimes e_1
    =b\otimes e_2$ if and only if $a=b=0$. Equivalently, $a\otimes e_1
    +b\otimes e_2=0$ has only trivial solutions $a=b=0\in V$. Thus, if
    \[r_{1,1}(e_1\otimes e_1) +r_{1,2}(e_1\otimes e_2) +r_{2,1}(e_2\otimes
      e_1) +r_{2,2}(e_2\otimes e_2)=0,\]
    for some $r_{1,1}=r_{1,2}=r_{2,1}=r_{2,2}\in\mathbb{R}$, then
    rearranging will give us
    \[(r_{1,1}e_1 +r_{2,1}e_2)\otimes e_1 +(r_{1,2}e_1 +r_{2,2}e_2)\otimes
      e_2=0\]
    Then from the above observation, we can only have trivial solutions
    \[r_{1,1}e_1 +r_{2,1}e_2 =r_{1,2}e_1 +r_{2,2}e_2 =0,\]
    which in turn only has trivial solutions
    $r_{1,1}=r_{1,2}=r_{2,1}=r_{2,2}=0$ since $e_1,e_2$ are linearly
    independent. \\

    Now assume $e_1\otimes e_2+e_2\otimes e_1 =v\otimes w$ for some $v,w\in
    V$. Then since $\{e_1,e_2\}$ is a basis of $\mathbb{R}^2$, we can write
    $v=r_1e_1+r_2e_2$ and $w=s_1e_1+s_2e_2$ for some
    $r_1,r_2,s_1,s_2\in\mathbb{R}$. Then
    \begin{align*}
      e_1\otimes e_2+e_2\otimes e_1 &=v\otimes w\\
      &=(r_1e_1+r_2e_2) \otimes(s_1e_1+s_2e_2)\\
      &=r_1s_1(e_1\otimes e_1) +r_1s_2(e_1\otimes e_2) +r_2s_1(e_2\otimes
        e_1) +r_2s_2(e_2\otimes e_2).\\
    \end{align*}
    Then from the previous argument, $v=w=0\in V$, thus $e_1\otimes
    e_2+e_2\otimes e_1=0$. But this would give a non-trivial solution for
    $a\otimes e_1+b\otimes e_2=0$, a contradiction.
  \end{proof}

\it \textbf{Section 10.4 Q12:} Let $V$ be a vector space over the field $F$
  and let $v,v'$ be nonzero elements of $V$. Prove that $v\otimes
  v'=v'\otimes v$ in $V\otimes_F V$ if and only if $v=av'$ for some $a\in
  F$.

  \begin{proof}
    The converse is trivial:
    \begin{align*}
      v\otimes v' &=av'\otimes v'\\
      &=a\cdot(v'\otimes v')\\
      &=v'\otimes av'\\
      &=v'\otimes v.\\
    \end{align*}

    For the forward direction. Let $\{e_i:i\in I\}$ be a
    basis of $V$. Assuming axiom of Choice, such a basis exists. Given
    $w\in V$, let $\alpha_i(w)\in F$ denote the coefficient of $e_i$ in the
    unique expression of $w$ as a linear sum of the basis elements. \\

    For each $i\in I$, consider the map $\varphi_i:V\times V\rightarrow V$
    defined by $\varphi_i((x,y))=\alpha_i(x)y$. It is routine to prove that
    $\varphi_i$ is an $F$-bilinear map. Thus by Corollary 12, $\varphi_i$
    induces an $F$-module homomorphism $\Phi_i:V\otimes_F V\rightarrow V$
    defined by $\Phi_i(x\otimes y)=\alpha_i(x)y$. Thus if $x_1\otimes
    y_1=x_2\otimes y_2\in V\otimes_F V$, then their images under $\Phi_i$
    for all $i\in I$ should be the same. In other words, we need
    $\alpha_i(x_1)y_1=\alpha_i(x_2)y_2$ for all $i\in I$. \\

    So assume $v\otimes v'=v'\otimes v \in V\otimes_F V$ where $v,v'\in V$
    are nonzero. Since $v'\neq0$, $\alpha_k(v')\neq0$ for some $k\in I$. So
    from the above argument, we have $\alpha_k(v)v'=\alpha_k(v')v$, which
    implies $v'$ and $v$ are linearly dependent since $\alpha_k(v)\neq0$.
    Then since $F$ is a field, $\alpha_k(v)$ has a multiplicative inverse.
    Rearranging, we get
    \[v=\alpha_k(v)\alpha_k(v')^{-1}v',\]
    where $\alpha_k(v)\alpha_k(v')^{-1} \in F$ as required.
  \end{proof}

\it \textbf{Section 10.4 Q16:} Suppose $R$ is commutative and let $I$ and
  $J$ be ideals of $R$, so $R/I$ and $R/J$ are naturally $R$-modules.
  \begin{enumerate}[label={(\alph*)}]
    \item Prove that every element of $R/I\otimes_R R/J$ can be written as
      a simple tensor of the form $(1\mod{I})\otimes(r\mod{J})$.
      \begin{proof}
        Given $\overline{i}\otimes\overline{j} \in R/I\otimes_R R/J$ with a
        simple form, we can write
        \begin{align*}
          &\;\overline{i}\otimes\overline{j}\\
          =&\;(i\cdot\overline{1})\otimes\overline{j}\\
          =&\;i\cdot(\overline{1}\otimes\overline{j})\\
          =&\;\overline{1}\otimes(i\cdot\overline{j})\\
          =&\;\overline{1}\otimes\overline{ij},\\
        \end{align*}
        which is of the desired form. \\

        Now since every element of $R/I\otimes_R R/J$ is a finite sum of
        elements of the simple form, by induction on the number of elements
        in the sum and the above, it suffices to show that the sum of two
        elements of the desired form is equal to an element of the desired
        form. To that end, we have
        \begin{align*}
          &\;\overline{1}\otimes\overline{r_1}
            +\overline{1}\otimes\overline{r_2}\\
          &\;\overline{1}\otimes (\overline{r_1}+\overline{r_2})\\
          &\;\overline{1}\otimes (\overline{r_1+r_2}),\\
        \end{align*}
        which is also of the desired form.
      \end{proof}

    \item Prove that there is an $R$-module isomorphism $R/I\otimes_R R/J
      \cong R/(I+J)$ mapping $(r\mod{I})\otimes(r'\mod{J})$ to
      $rr'\mod{(I+J)}$.

      \begin{proof}
        Define the map $\varphi:R/I\times_R R/J \rightarrow R/(I+J)$ by
        $\varphi((\overline{r},\overline{r'})) =\overline{rr'}$. This map
        is well-defined because given $i\in I$ and $j\in J$, within
        $R/(I+J)$, we have $\overline{(r+i)(r'+j)}
        =\overline{rr'+rj+r'i+ij} =\overline{rr'} \in R/(I+J)$ because
        $rj+r'i+ij\in I+J$. Also, it is routine to show that this map is
        $R$-bilinear. Thus by Corollary 12, $\varphi$ induces an $R$-module
        homomorphism $\Phi:R/I\otimes_R R/J \rightarrow R/(I+J)$ given by
        $\Phi(\overline{r}\otimes\overline{r'}) =\overline{rr'}$. $\Phi$ is
        trivially surjective since every $\overline{r}\in R/(I+J)$ has
        pre-image $\overline{1}\otimes\overline{r} \in R/I\otimes_R R/J$.
        It remains to show that $\Phi$ is injective. Let $x\in\ker(\Phi)$.
        From part (a), we can assume $x=\overline{1}\otimes\overline{r} \in
        R/I\otimes_R R/J$. Then $\Phi(x) =\overline{r}=0\in R/(I+J)$,
        which implies that $r\in I+J$. Thus $r=i+j$ for some $i\in I$ and
        $j\in J$. Then
        \begin{align*}
          x &=\overline{1}\otimes\overline{r} &\in R/I\otimes_R R/J\\
          &=\overline{1}\otimes\overline{i+j}\\
          &=\overline{1}\otimes\overline{i} &(\because j\in J)\\
          &=\overline{1}\otimes(i\cdot\overline{1})\\
          &=\overline{i}\otimes\overline{1}\\
          &=\overline{0}\otimes\overline{1} &(\because i\in I)\\
          &=0. &\in R/I\otimes_R R/J\\
        \end{align*}
        Thus $\Phi$ is injective.
      \end{proof}
  \end{enumerate}

\it \textbf{Section 10.4 Q17:} Let $I=(2,x)$ be the ideal generated by 2
  and $x$ in the ring $R=\mathbb{Z}[x]$. The ring $\mathbb{Z}_2=R/I$ is
  naturally an $R$-module annihilated by both 2 and $x$.

  \begin{enumerate}[label={(\alph*)}]
    \item Show that the map $\varphi:I\times I\rightarrow\mathbb{Z}_2$
      defined by
      \[\varphi(a_0+a_1x+\ldots+a_nx^n, b_0+b_1x+\ldots+b_mx^m)
      =\frac{a_0}{2}b_1\mod{2}\]
      is $R$-bilinear.

      \begin{proof}
        We have
        \begin{align*}
          &\;\varphi((2a_0+xf_0(x))+(2b_0+xf_1(x)), c_0+c_1x+x^2f_2(x))\\
          =&\;(a_0+b_0)c_1\mod{2}\\
          =&\;a_0c_1+b_0c_1\mod{2}\\
          =&\;\varphi(2a_0+xf_0(x), c_0+c_1x+x^2f_2(x))
            +\varphi(2b_0+xf_1(x), c_0+c_1x+x^2f_2(x))\\
        \end{align*}

        Similarly,
        \begin{align*}
          &\;\varphi(2a_0+xf_0(x), (b_0+b_1x+x^2f_1(x))
            +(c_0+c_1x+x^2f_2(x)))\\
          =&\;a_0(b_1+c_1)\mod{2}\\
          =&\;a_0b_1+a_0c_1\mod{2}\\
          =&\;\varphi(2a_0+xf_0(x), b_0+b_1x+x^2f_1(x)
            +\varphi(2a_0+xf_0(x), c_0+c_1x+x^2f_2(x)).\\
        \end{align*}

        Also,
        \begin{align*}
          &\;\varphi((c_0+c_1x+x^2f_2(x))(2a_0+xf_0(x)),
            b_0+b_1x+x^2f_1(x))\\
          =&\;c_0a_0b_1\mod{2}\\
          =&\;c_0\cdot(a_0b_1\mod{2})\\
          =&\;c_0\cdot (a_0b_1\mod{2}) +(xc_1+c^2f_2(x))\cdot (a_0b_1\mod{2})
            &(\text{annihilate}\;
            a_0b_0(xc_1+c^2f_2(x)))\\
          =&\;(c_0+c_1x+x^2f_2(x))\cdot (a_0b_1\mod{2})\\
          =&\;(c_0+c_1x+x^2f_2(x))\cdot \varphi(2a_0+xf_0(x),
            b_0+b_1x+x^2f_1(x)),\\
        \end{align*}

        and
        \begin{align*}
          &\;\varphi(2a_0+xf_0(x),
            (c_0+c_1x+x^2f_2(x))(b_0+b_1x+x^2f_1(x))).\\
          =&\;a_0(c_0b_1+b_0c_1)\mod{2}\\
          =&\;a_0c_0b_1\mod{2} +a_0b_0c_1\mod{2}\\
          =&\;a_0c_0b_1\mod{2} &(\because 2|b_0)\\
          =&\;\varphi((c_0+c_1x+x^2f_2(x))(2a_0+xf_0(x)),
            b_0+b_1x+x^2f_1(x)).\\
        \end{align*}
      \end{proof}

    \item Show that there is an $R$-module homomorphism from $I\otimes_R
      I\rightarrow\mathbb{Z}_2$ mapping $p(x)\otimes q(x)$ to
      $\frac{p(0)}{2}q'(0)$ where $q'$ denotes the usual polynomial
      derivative of $q$.

      \begin{proof}
        The map $\varphi$ defined in part (a) is $R$-bilinear, so by
        Corollary 12, induces an $R$-module homomorphism $\Phi:I\otimes_R
        I\mapsto\mathbb{Z}_2$ given by $p(x)\otimes q(x)$ to
        $\frac{p(0)}{2}q'(0)$.
      \end{proof}

    \item Show that $2\otimes x \neq x\otimes2$ in $I\otimes I$.
      \begin{proof}
        If these two elements are equal, then $\Phi$ defined above must map
        them to the same element in $\mathbb{Z}_2$. However $\Phi(2\otimes
        x) =\bar{1}$ while $\Phi(x\otimes2) =\bar{0}$.
      \end{proof}
  \end{enumerate}

\it \textbf{Section 10.4 Q18:} Suppose $I$ is a principal ideal in the
  integral domain $R$. Prove that the $R$-module $I\otimes_R I$ has no
  nonzero torsion elements (i.e., $rm=0$ with $0\neq r\in R$ and $m\in
  I\otimes_R I$ implies that $m=0$).

  \begin{proof}
    Write $I=(a)$ for some $a\in I$. First, we show that the elements of
    $I\otimes_RI$ are all simple, and of the form $ra\otimes a$ for some
    $r\in R$. For simple elements $r_1a\otimes r_2a$, this is equal to
    $r_1r_2a\otimes a$. For two simple elements of such form, their sum
    can be written $r_1a\otimes a+r_2a\otimes a =(r_1+r_2)a\otimes a$.
    Thus by induction on the number of elements in a finite sum of simple
    elements, all elements of $I\otimes_RI$ can be expressed as
    $ra\otimes a$ for some $r\in R$. \\

    Consider the map $\varphi:I\times I\rightarrow R$ defined by
    $\varphi(r_1a,r_2a)=r_1r_2$. It is routine to show that $\varphi$ is
    $R$-bilinear. Thus from Corollary 12, $\varphi$ induces an $R$-module
    homomorphism $\Phi:I\otimes_R I\rightarrow R$ defined by
    $\Phi(r_1a\otimes r_2a)=r_1r_2$. Let $m\in I\otimes_R I$ be torsion
    with $r\neq0$ as witness. Then from above, we can write $m=sa\otimes a$
    for some $s\in R$, so $rm=rsa\otimes a=0\otimes0$. Thus
    \[0=\Phi(0)=\Phi(m)=\Phi(rsa\otimes a)=rs.\]

    Then since $r\neq0$ and $R$ is an integral domain, we must have $s=0$,
    so $m=sa\otimes a=0\otimes a=0$.
  \end{proof}

\it \textbf{Section 10.4 Q19:} Let $I=(2,x)$ be the ideal generated by 2
  and $x$ in the ring $R=\mathbb{Z}[x]$ as in Exercise 17. Show that the
  nonzero element $2\otimes x-x\otimes2$ in $I\otimes_RI$ is a torsion
  element. Show in fact that $2\otimes x-x\otimes2$ is annihilated by both
  2 and $x$ and that the submodule of $I\otimes_RI$ generated by $2\otimes
  x-x\otimes2$ is isomorphic to $R/I$.

  \begin{proof}
    We have shown in Question 17 that $2\otimes x-x\otimes2$ is non-zero in
    $I\otimes_RI$. Now
    \begin{align*}
      2\cdot(2\otimes x-x\otimes2) &=2\otimes 2x-2x\otimes2\\
      &=x\cdot(2\otimes2) -x\cdot(2\otimes2)\\
      &=0,\\
    \end{align*}

    and similarly
    \begin{align*}
      x\cdot(2\otimes x-x\otimes2) &=2x\otimes x-x\otimes2x\\
      &=2\cdot(x\otimes x) -2\cdot(x\otimes x)\\
      &=0,\\
    \end{align*}
    thus $2$ and $x$ annihilate $2\otimes x-x\otimes2$, making it a torsion
    element. \\
  \end{proof}
\end{document}
