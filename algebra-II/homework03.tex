\documentclass{article}
\usepackage[left=3cm,right=3cm,top=3cm,bottom=3cm]{geometry}
\usepackage{amsmath,amssymb,amsthm,pgfplots,tikz}
\usepackage[inline]{enumitem}
\usepackage{color}
\setlength{\parindent}{0mm} %So that we do not indent on new paragraphs
\newcommand{\TODO}[1]{\textcolor{red}{TODO: #1}}

\begin{document}
\title{Graduate Algebra II: Homework 3}
\author{Li Ling Ko\\ lko@nd.edu}
\date{\today}
\maketitle

\it \textbf{Q1:} Let $A$ and $B$ be abelian groups, and $m,n$ positive
  integers.
  \begin{enumerate}[label={(\alph*)}]
    \item Show that $A\otimes_\mathbb{Z}\mathbb{Z}_m \cong A/mA$.
      \begin{proof}
        Consider the map $\varphi:A\times\mathbb{Z}_m
        \rightarrow A/mA$ defined by
        $(a,\bar{z})\mapsto\overline{z\cdot a}$. We first show this map
        is well-defined and $\mathbb{Z}$-bilinear. To show that the map is
        well-defined, we have
        \begin{align*}
          \varphi((a,\overline{z+m})) &=\overline{(z+m)\cdot a}\\
          &=\overline{z\cdot a+m\cdot a}\\
          &=\overline{z\cdot a}\\
          &=\varphi((a,\overline{z})).\\
        \end{align*}

        To show $\mathbb{Z}$-bilinearlity, we have
        \begin{align*}
          \varphi(((a_1+a_2),\overline{z})) &=\overline{z\cdot
            (a_1+a_2)}\\
          &=\overline{z\cdot a_1 +z\cdot a_2}\\
          &=\overline{z\cdot a_1} +\overline{z\cdot a_2}\\
          &=\varphi((a_1,\overline{z}))
            +\varphi((a_2,\overline{z})),\\
        \end{align*}

        and
        \begin{align*}
          \varphi((a,(\overline{z_1}+\overline{z_2})))
            &=\varphi((a,(\overline{z_1+z_2})))\\
          &=\overline{(z_1+z_2)\cdot a}\\
          &=\overline{z_1\cdot a +z_2\cdot a}\\
          &=\overline{z_1\cdot a} +\overline{z_2\cdot a}\\
          &=\varphi((a,\overline{z_1}))
            +\varphi((a,\overline{z_2})),\\
        \end{align*}

        and
        \begin{align*}
          \varphi((n\cdot a,\overline{z})))
          &=\overline{z\cdot (n\cdot a)}\\
          &=\overline{(nz)\cdot a}\\
          &=\overline{(n\cdot\overline{z})\cdot a}\\
          &=\varphi((a, (n\cdot \overline{z}))),\\
        \end{align*}

        and
        \begin{align*}
          \varphi((a,n\cdot\overline{z}))
          &=\varphi((a,\overline{nz}))\\
          &=\overline{(nz)\cdot a}\\
          &=\overline{(n\cdot\overline{z})\cdot a}\\
          &=\varphi((a, (n\cdot \overline{z}))).\\
        \end{align*}

        Thus from Corollary 12, $\varphi$ induces a $\mathbb{Z}$-module
        homomorphism $\Phi:A\otimes_\mathbb{Z}\mathbb{Z}_m \rightarrow
        A/mA$ defined by $\Phi(a\otimes\bar{z})=\overline{z\cdot a}$. Now
        $\Phi$ is clearly surjective since every $\overline{a}\in
        A/mA$ has a pre-image $a\otimes\bar{1}$. It remains to show that
        $\Phi$ is injective. Now by definition, every element $x$ of
        $A\otimes_\mathbb{Z}\mathbb{Z}_m$ can be written
        \[x=\sum_{i=1}^n a_i\otimes\overline{z_i}\]
        for some $a_i\in A$, $\overline{z_i}\in\mathbb{Z}_m$. Thus
        \begin{align*}
          \;&x\in\ker(\Phi)\\
          \Leftrightarrow\;&\sum_{i=1}^n a_i\otimes\overline{z_i}
            \in\ker(\Phi)\\
          \Leftrightarrow\;&\sum_{i=1}^n \overline{z_i\cdot
            a_i}=\overline{0}\in A/mA\\
          \Leftrightarrow\;&\sum_{i=1}^n z_i\cdot a_i\in mA\\
          \Leftrightarrow\;&\sum_{i=1}^n z_i\cdot a_i =m\cdot b\in A\;
            \text{for some}\; b\in A\\
          \Rightarrow\;&\left(\sum_{i=1}^n z_i\cdot a_i\right)
            \otimes\bar{1}=(m\cdot b)\otimes\bar{1}\; \text{for some}\;
            b\in A.\\
          \Rightarrow\;&\sum_{i=1}^n (z_i\cdot a_i)\otimes\bar{1} =(m\cdot
            b)\otimes\bar{1}\; \text{for some}\; b\in A.\\
        \end{align*}

        Now by $\mathbb{Z}$-bilinear property of tensors, given arbitrary
        $a\in A$ and $z\in\mathbb{Z}_m$,
        \begin{align*}
          &\;(z\cdot a)\otimes\bar{1}\\
          =&\;a\otimes(z\cdot\bar{1})\\
          =&\;a\otimes\bar{z},\\
        \end{align*}
        thus we can simplify the above conditions to get
        \[\sum_{i=1}^n a_i\otimes\overline{z_i} =b\otimes\overline{m}=0,\]
        which implies $x=0$. Thus $\ker(\Phi)=0$.
      \end{proof}

    \item Show that $\mathbb{Z}_m\otimes_\mathbb{Z}\mathbb{Z}_n
      \cong\mathbb{Z}_c$, where $c=(m,n)$.
      \begin{proof}
        By replacing $m$ with $n$ in part (a) of this question letting
        $A=\mathbb{Z}_m$, it suffices to show that
        $\mathbb{Z}_m/n\mathbb{Z}_m\cong \mathbb{Z}_c$ as
        $\mathbb{Z}$-modules. First, observe that $n\mathbb{Z}_m=\langle
        c\rangle$: $c$ is the smallest positive integer such that there
        exists $x_0,y_0\in\mathbb{Z}$ with $c=x_0m+y_0n$. Thus in
        $\mathbb{Z}_m$,
        $\overline{c}=\overline{x_0m+y_0n}=\overline{y_0n}=n\overline{y_0}
        \in n\mathbb{Z}_m$. Also, since all elements in $n\mathbb{Z}_m$
        should be of the form $xm+yn$ for some $x,y\in\mathbb{Z}$, and $c$
        is the smallest positive integer of such form, and $n\mathbb{Z}_m$
        is closed under multiplication modulo $m$, $n\mathbb{Z}_m$ must
        equal
        \[n\mathbb{Z}_m =\{\overline{0}, \overline{c},
        \overline{2c}, \ldots, \overline{m-c}\}.\]

        Therefore
        \[\mathbb{Z}_m/n\mathbb{Z}_m
        =\mathbb{Z}_m/\langle\overline{c}\rangle
        \cong\mathbb{Z}_c.\]
      \end{proof}

    \item If $A$ is a torsion group (i.e. a torsion $\mathbb{Z}$-module),
      then $A\otimes_\mathbb{Z}\mathbb{Q}=0$.

      \begin{proof}
        Every element in $x\in A\otimes_\mathbb{Z}\mathbb{Q}$ can be
        written as a finite sum
        \[x=\sum_{i=1}^k a_i\otimes q_i\]
        for some $a_i\in A$, $q_i\in \mathbb{Q}$ by definition of torsion
        products. Since $A$ is torsion, for each $i=\{1,\ldots,k\}$,
        there exists $n_i$ such that $n_i\cdot a_i=0$. Let $n=n_1\cdots
        n_k$. Then $n\cdot a_i=0$ for all $i=\{1,\ldots,k\}$. So
        \begin{align*}
          &\;a_i\otimes q_i\\
          =&\;a_i\otimes(n\cdot(q_i/n))\\
          =&\;(n\cdot a_i)\otimes(q_i/n)\\
          =&\;0\otimes(q_i/n)\\
          =&\;(0\cdot0)\otimes(q_i/n)\\
          =&\;0\otimes(0\cdot q_i/n)\\
          =&\;0\otimes0\\
          =&\;0,\\
        \end{align*}
        thus
        \[x=\sum_{i=1}^k a_i\otimes q_i =\sum_{i=1}^k 0=0.\]
      \end{proof}
  \end{enumerate}

\it \textbf{Q2:} What is the difference between the homomorphism $f\otimes
  g$ as defined in class, and the element $f\otimes g\in\text{Hom}_R(A,A')
  \otimes_\mathbb{Z}\text{Hom}_R(B,B')$?
  \begin{proof}
  \end{proof}

\it \textbf{Q3:}
  \begin{enumerate}[label={(\alph*)}]
    \item From Exercise 1, determine what
      $\mathbb{Z}_2\otimes_\mathbb{Z}\mathbb{Z}_2$ is as an abelian group,
      and what is $\mathbb{Z}_2\otimes_\mathbb{Z}\mathbb{Z}_4$ as an
      abelian group.

      \begin{proof}
        From Question 1b, $\mathbb{Z}_2\otimes_\mathbb{Z}\mathbb{Z}_2
        \cong\mathbb{Z}_2$ as a $\mathbb{Z}$-module, thus it is isomorphic
        to $\mathbb{Z}_2$ as an abelian group. Similarly,
        $\mathbb{Z}_2\otimes_\mathbb{Z}\mathbb{Z}_4$ is isomorphic to
        $\mathbb{Z}_2$ as an abelian group.
      \end{proof}

    \item Let $\phi:\mathbb{Z}_2\rightarrow\mathbb{Z}_4$ be the usual
      injection of abelian groups. Show that $1\otimes\phi:
      \mathbb{Z}_2\otimes_\mathbb{Z}\mathbb{Z}_2
      \rightarrow\mathbb{Z}_2\otimes_\mathbb{Z}\mathbb{Z}_4$, is the zero
      map, even though $\mathbb{Z}_2\otimes_\mathbb{Z}\mathbb{Z}_2$ and
      $\mathbb{Z}_2\otimes_\mathbb{Z}\mathbb{Z}_4$ are nontrivial.

      \begin{proof}
        Given arbitrary $a,b\in\mathbb{Z}_2$,
        \begin{align*}
          &\;(1\otimes\phi)(\overline{a}\otimes \overline{b})\\
          =&\;\overline{a}\otimes \overline{2b}
            \in\mathbb{Z}_2\otimes_\mathbb{Z}\mathbb{Z}_4\\
          =&\;\overline{2a}\otimes \overline{b}\\
          =&\;\overline{0}\otimes \overline{b} &(\because\; 2|2a)\\
          =&\;\overline{0}\otimes \overline{0}\\
          =&\;0\in\mathbb{Z}_2\otimes_\mathbb{Z}\mathbb{Z}_4.\\
        \end{align*}

        So since elements of $\mathbb{Z}_2\otimes_\mathbb{Z}\mathbb{Z}_2$
        are finite sums of elements of the form $a\otimes b$ with
        $a,b\in\mathbb{Z}_2$ and homomorphisms preserve addition, the image
        of any element in $\mathbb{Z}_2\otimes_\mathbb{Z}\mathbb{Z}_2$ will
        be a finite sum of $0$'s, which will remain 0.
      \end{proof}
  \end{enumerate}

\textbf{Let $R$ be a ring with 1.} \\

\it \textbf{Section 10.4 Q3:} Show that
  $\mathbb{C}\otimes_\mathbb{R}\mathbb{C}$ and
  $\mathbb{C}\otimes_\mathbb{C}\mathbb{C}$ are both left
  $\mathbb{R}$-modules but are not isomorphic as $\mathbb{R}$-modules.

  \begin{proof}
    $\mathbb{C}\otimes_\mathbb{R}\mathbb{C}$ is a left $\mathbb{R}$-module
    because $\mathbb{C}$ is trivally a $\mathbb{R}$-module. Similarly,
    $\mathbb{C}\otimes_\mathbb{C}\mathbb{C}$ is a left $\mathbb{C}$-module
    because $\mathbb{C}$ is trivially a $\mathbb{C}$-module. Then since
    $\mathbb{R}$ is a subring of $\mathbb{C}$ with the same unity,
    $\mathbb{C}\otimes_\mathbb{C}\mathbb{C}$ will also be a left
    $\mathbb{R}$-module. \\

    We first show that $\mathbb{C}\otimes_\mathbb{R}\mathbb{C}$ is
    isomorphic to $\mathbb{R}^4$ as $\mathbb{R}$-modules. Consider the map
    $\varphi:\mathbb{C}\times\mathbb{C}\rightarrow\mathbb{R}^4$ defined by
    $\varphi((a_0+b_0i,a_1+b_1i))=(a_0,b_0,a_1,b_1)$, where
    $a_0,b_0,a_1,b_1\in\mathbb{R}$. It is routine to show that $\varphi$ is
    a $\mathbb{R}$-bilinear map: Given
    $r,a_0,b_0,a_1,b_1,c_0,c_1,d_0,d_1\in\mathbb{R}$,
    \begin{align*}
      \;&\varphi(r(a_0+b_0i,a_1+b_1i)+(c_0+d_0i,c_1+d_1i))\\
      =\;&\varphi(ra_0+c_0+(rb_0+d_0)i, ra_1+c_1+(rb_1+d_1)i)\\
      =\;&(ra_0+c_0, rb_0+d_0, ra_1+c_1, rb_1+d_1)\\
      =\;&(ra_0,rb_0,ra_1,rb_1) +(c_0,d_0,c_1,d_1)\\
      =\;&r(a_0,b_0,a_1,b_1) +(c_0,d_0,c_1,d_1)\\
      =\;&r\varphi((a_0+b_0i,a_1+b_1i)) +\varphi((c_0+d_0i,c_1+d_1i)).\\
    \end{align*}
    Similarly, we can show that
    \[\varphi((a_0+b_0i,a_1+b_1i)+r(c_0+d_0i,c_1+d_1i))
    =\varphi((a_0+b_0i,a_1+b_1i)) +r\varphi((c_0+d_0i,c_1+d_1i)).\]

    Thus from Corollary 12, $\varphi$ induces an $\mathbb{R}$-module
    homomorphism $\Phi:\mathbb{C}\otimes_\mathbb{R}\mathbb{C}\rightarrow
    \mathbb{R}^4$ defined by
    $\Phi((a_0+b_0i)\otimes(a_1+b_1i))=(a_0,b_0,a_1,b_1)$. This map is
    trivially surjective, since arbitrary
    $(r_0,r_1,r_2,r_3)\in\mathbb{R}^4$ as pre-image
    $(r_0+r_1i)\otimes(r_2+r_3i)$. Thus the dimension of
    $\mathbb{C}\otimes_\mathbb{R}\mathbb{C}$ as an $\mathbb{R}$-module must
    be at least the dimension of $\mathbb{R}^4$, which is 4. \\

    Next, observe that $\mathbb{C}\otimes_\mathbb{C}\mathbb{C}$ can be
    spanned as an $\mathbb{R}$-module by two elements
    $\{1\otimes1,1\otimes i\}$, because given $a,b,c,d\in\mathbb{R}$, we can
    write
    \begin{align*}
      &\;(a+bi)\otimes(c+di)\\
      =&\;a\otimes c+a\otimes di +bi\otimes c +bi\otimes di\\
      =&\;ac\cdot(1\otimes1) +ad\cdot(1\otimes i) +bc\cdot(i\otimes1)
        +bd\cdot(i\otimes i)\\
      =&\;ac\cdot(1\otimes1) +ad\cdot(1\otimes i) +bc\cdot(1\otimes i)
        -bd\cdot(1\otimes1)\\
      =&\;(ac-bd)\cdot(1\otimes1) +(ad+bc)\cdot(1\otimes i).\\
    \end{align*}

    Thus $\mathbb{C}\otimes_\mathbb{R}\mathbb{C}$ has $\mathbb{R}$-module
    dimension no greater than 2, while
    $\mathbb{C}\otimes_\mathbb{C}\mathbb{C}$ has $\mathbb{R}$-module
    dimension at least 4, so they cannot be isomorphic as
    $\mathbb{R}$-modules.
  \end{proof}

\it \textbf{Section 10.4 Q5:} Let $A$ be a finite abelian group of order
  $n$ and let $p^k$ be the largest power of the prime $p$ dividing $n$.
  Prove that $\mathbb{Z}_{p^k}\otimes_\mathbb{Z}A$ is isomorphic to
  the Sylow $p$-subgroup of $A$.

  \begin{proof}
    Let the elementary divisor decomposition of $A$ be
    \[A\cong P\oplus_{i=1}^m\mathbb{Z}_{q_i^{\alpha_i}},\]
    where the $q_i$'s are distinct primes from $p$, and
    \[P=\mathbb{Z}_{p^{\alpha_1}}
    \oplus\ldots\oplus\mathbb{Z}_{p^{\alpha_n}}\]
    is the Sylow $p$-subgroup of $A$. Note that the decomposition is
    expressed as direct sum of $\mathbb{Z}$-modules. Then
    \begin{align*}
      &\;\mathbb{Z}_{p^k}\otimes_{\mathbb{Z}}A\\
      =&\;\mathbb{Z}_{p^k}\otimes_{\mathbb{Z}}
        (P\oplus_{i=1}^m \mathbb{Z}_{q_i^{\alpha_i}})\\
      =&\;(\mathbb{Z}_{p^k}\otimes_{\mathbb{Z}}P)
        \oplus(\mathbb{Z}_{p^k}
        \otimes_{\mathbb{Z}}\mathbb{Z}_{q_1^{\alpha_1}}) \oplus\ldots
        \oplus(\mathbb{Z}_{p^k}
        \otimes_{\mathbb{Z}}\mathbb{Z}_{q_m^{\alpha_m}}).
        &(\text{distributive property of tensors})\\
    \end{align*}

    Now for each $i\in\{1,\ldots,m\}$, since $p$ and $q_i$ are distinct
    primes, we have from Question 1b that
    \[\mathbb{Z}_{p^k} \otimes_{\mathbb{Z}}\mathbb{Z}_{q_m^{\alpha_m}}
    \cong\mathbb{Z}_1.\]

    Therefore
    \begin{align*}
      &\;\mathbb{Z}_{p^k}\otimes_{\mathbb{Z}}A\\
      =&\;(\mathbb{Z}_{p^k}\otimes_{\mathbb{Z}}P)
        \oplus(\mathbb{Z}_{p^k}
        \otimes_{\mathbb{Z}}\mathbb{Z}_{q_1^{\alpha_1}}) \oplus\ldots
        \oplus(\mathbb{Z}_{p^k}
        \otimes_{\mathbb{Z}}\mathbb{Z}_{q_m^{\alpha_m}})\\
      =&\;(\mathbb{Z}_{p^k}\otimes_{\mathbb{Z}}P) \oplus\mathbb{Z}_1
        \oplus\ldots \oplus\mathbb{Z}_1\\
      =&\;\mathbb{Z}_{p^k}\otimes_{\mathbb{Z}}P.\\
    \end{align*}
  \end{proof}

\it \textbf{Section 10.4 Q7:} If $R$ is any integral domain with quotient
  field $Q$ and $N$ is a left $R$-module, prove that every element of the
  tensor product $Q\otimes_R N$ can be written as a simple tensor of the
  form $(1/d)\otimes n$ for some nonzero $d\in R$ and some $n\in N$.

  \begin{proof}
    Let $q\in Q$ and $n'\in N$ be arbitrary. Then $q=c/d$ for some $c,d\in
    R$ and $d\neq0$, thus
    \begin{align*}
      &\;q\otimes n'\\
      =&\;(c/d)\otimes n'\\
      =&\;c\cdot((1/d)\otimes n')\\
      =&\;(1/d)\otimes (c\cdot n'),\\
    \end{align*}
    which is of the desired form. \\

    Now since elements of $Q\otimes_R N$ can
    be written as a finite sum of elements of the form $q\otimes n$ for
    some $q\in Q$ and $n\in N$, by induction on the number of sums, it
    suffices to show the sum of two elements $(1/d_1)\otimes n_1$ and
    $(1/d_2)\otimes n_2$ is of the desired form:
    \begin{align*}
      &\;(1/d_1)\otimes n_1 +(1/d_2)\otimes n_2\\
      =&\;(1/(d_1d_2))\otimes d_2n_1 +(1/(d_1d_2))\otimes d_1n_2
      &(\text{possible since}\; d_1,d_2\neq0)\\
      =&\;(1/(d_1d_2))\otimes(d_2n_1+d_1n_2),\\
    \end{align*}
    which is of the desired form since $d_1d_2\neq0$ because $R$ is an
    integral domain.
  \end{proof}
\end{document}
