\documentclass{article}
\usepackage[left=3cm,right=3cm,top=3cm,bottom=3cm]{geometry}
\usepackage{amsmath,amssymb,amsthm,pgfplots,tikz}
\usepackage[inline]{enumitem}
\usepackage{color}
\setlength{\parindent}{0mm} %So that we do not indent on new paragraphs
\newcommand{\TODO}[1]{\textcolor{red}{TODO: #1}}

\begin{document}
\title{Graduate Algebra II: Homework 3}
\author{Li Ling Ko\\ lko@nd.edu}
\date{\today}
\maketitle

\it \textbf{Q1:} Let $A$ and $B$ be abelian groups, and $m,n$ positive
  integers.
  \begin{enumerate}[label={(\alph*)}]
    \item Show that $A\otimes_\mathbb{Z}\mathbb{Z}_m \cong A/mA$.
      \begin{proof}
        Consider the map $\varphi:A\times\mathbb{Z}_m
        \rightarrow A/mA$ defined by
        $(a,\bar{z})\mapsto\overline{z\cdot a}$. We first show this map
        is well-defined and $\mathbb{Z}$-bilinear. To show that the map is
        well-defined, we have
        \begin{align*}
          \varphi((a,\overline{z+m})) &=\overline{(z+m)\cdot a}\\
          &=\overline{z\cdot a+m\cdot a}\\
          &=\overline{z\cdot a}\\
          &=\varphi((a,\overline{z})).\\
        \end{align*}

        To show $\mathbb{Z}$-bilinearlity, we have
        \begin{align*}
          \varphi(((a_1+a_2),\overline{z})) &=\overline{z\cdot
            (a_1+a_2)}\\
          &=\overline{z\cdot a_1 +z\cdot a_2}\\
          &=\overline{z\cdot a_1} +\overline{z\cdot a_2}\\
          &=\varphi((a_1,\overline{z}))
            +\varphi((a_2,\overline{z})),\\
        \end{align*}

        and
        \begin{align*}
          \varphi((a,(\overline{z_1}+\overline{z_2})))
            &=\varphi((a,(\overline{z_1+z_2})))\\
          &=\overline{(z_1+z_2)\cdot a}\\
          &=\overline{z_1\cdot a +z_2\cdot a}\\
          &=\overline{z_1\cdot a} +\overline{z_2\cdot a}\\
          &=\varphi((a,\overline{z_1}))
            +\varphi((a,\overline{z_2})),\\
        \end{align*}

        and
        \begin{align*}
          \varphi((n\cdot a,\overline{z})))
          &=\overline{z\cdot (n\cdot a)}\\
          &=\overline{(nz)\cdot a}\\
          &=\overline{(n\cdot\overline{z})\cdot a}\\
          &=\varphi((a, (n\cdot \overline{z}))),\\
        \end{align*}

        and
        \begin{align*}
          \varphi((a,n\cdot\overline{z}))
          &=\varphi((a,\overline{nz}))\\
          &=\overline{(nz)\cdot a}\\
          &=\overline{(n\cdot\overline{z})\cdot a}\\
          &=\varphi((a, (n\cdot \overline{z}))).\\
        \end{align*}

        Thus from Corollary 12, $\varphi$ induces a $\mathbb{Z}$-module
        homomorphism $\Phi:A\otimes_\mathbb{Z}\mathbb{Z}_m \rightarrow
        A/mA$ defined by $\Phi(a\otimes\bar{z})=\overline{z\cdot a}$. Now
        $\Phi$ is clearly surjective since every $\overline{a}\in
        A/mA$ has a pre-image $a\otimes\bar{1}$. It remains to show that
        $\Phi$ is injective:
        \begin{align*}
          \;&a\otimes\overline{z}\in\ker(\Phi)\\
          \Leftrightarrow\;&\overline{z\cdot a}=\overline{0}\in A/mA\\
          \Leftrightarrow\;&z\cdot a\in mA\\
          \Leftrightarrow\;&z\cdot a =m\cdot b\in A\; \text{for some}\;
            b\in A\\
          \Rightarrow\;&(z\cdot a)\otimes\bar{1}=(m\cdot
            b)\otimes\bar{1}\; \text{for some}\; b\in A.\\
        \end{align*}

        Now $(m\cdot b)\otimes\bar{1} =b\otimes(m\cdot\bar{1})
        =b\otimes\overline{m} =b\otimes\bar{0} =b\otimes(0\cdot\bar{1})
        =(0\cdot b)\otimes\overline{0} =0\otimes\overline{0}$. Also,
        $(z\cdot a)\otimes\bar{1} =a\otimes(z\cdot\bar{1})
        =a\otimes\overline{z}$. Thus we have
        \[a\otimes\overline{z} =0\otimes\overline{0},\]
        which means $\ker(\Phi)=0$.
      \end{proof}

    \item Show that $\mathbb{Z}_m\otimes_\mathbb{Z}\mathbb{Z}_n
      \cong\mathbb{Z}_c$, where $c=(m,n)$.
      \begin{proof}
        This follows immediately from part (a), by replacing $m$ with $n$,
        letting $A=\mathbb{Z}_m$, and observing that
        $\mathbb{Z}_m/n\mathbb{Z}_m\cong \mathbb{Z}_c$.
      \end{proof}

    \item If $A$ is a torsion group (i.e. a torsion $\mathbb{Z}$-module),
      then $A\otimes_\mathbb{Z}\mathbb{Q}=0$.
      \begin{proof}
        Let $a\in A$ and $q\in\mathbb{Q}$ be arbitrary. Then $n\cdot a=0\in
        A$ for some $n\in\mathbb{Z}$, since $A$ is a torsion group. Then
        \begin{align*}
          &\;a\otimes q\\
          =&\;a\otimes(n\cdot(q/n))\\
          =&\;(n\cdot a)\otimes(q/n)\\
          =&\;0\otimes(q/n)\\
          =&\;(0\cdot0)\otimes(q/n)\\
          =&\;0\otimes(0\cdot q/n)\\
          =&\;0\otimes0\\
          =&\;0.\\
        \end{align*}
      \end{proof}
  \end{enumerate}

\it \textbf{Q2:} What is the difference between the homomorphism $f\otimes
  g$ as defined in class, and the element $f\otimes g\in\text{Hom}_R(A,A')
  \otimes_\mathbb{Z}\text{Hom}_R(B,B')$?
  \begin{proof}
  \end{proof}

\it \textbf{Q3:}
  \begin{enumerate}[label={(\alph*)}]
    \item From Exercise 1, determine what
      $\mathbb{Z}_2\otimes_\mathbb{Z}\mathbb{Z}_2$ is as n abelian group,
      and what $\mathbb{Z}_2\otimes_\mathbb{Z}\mathbb{Z}_4$  is as an
      abelian group.

      \begin{proof}
        From Question 1b, $\mathbb{Z}_2\otimes_\mathbb{Z}\mathbb{Z}_2
        \cong\mathbb{Z}_2$ as a $\mathbb{Z}$-module, thus it is isomorphic
        to $\mathbb{Z}_2$ as an abelian group. Similarly,
        $\mathbb{Z}_2\otimes_\mathbb{Z}\mathbb{Z}_4$ is isomorphic to
        $\mathbb{Z}_2$ as an abelian group.
      \end{proof}

    \item Let $\phi:\mathbb{Z}_2\rightarrow\mathbb{Z}_4$ be the usual
      injection of abelian groups. Show that $1\otimes\phi:
      \mathbb{Z}_2\otimes_\mathbb{Z}\mathbb{Z}_2
      \rightarrow\mathbb{Z}_2\otimes_\mathbb{Z}\mathbb{Z}_4$, is the zero
      map, even though $\mathbb{Z}_2\otimes_\mathbb{Z}\mathbb{Z}_2$ and
      $\mathbb{Z}_2\otimes_\mathbb{Z}\mathbb{Z}_4$ are nontrivial.

      \begin{proof}
        Given arbitrary $a,b\in\mathbb{Z}_2$,
        \begin{align*}
          &\;(1\otimes\phi)(\overline{a}\otimes \overline{b})\\
          =&\;\overline{a}\otimes \overline{2b}
            \in\mathbb{Z}_2\otimes_\mathbb{Z}\mathbb{Z}_4\\
          =&\;\overline{2a}\otimes \overline{b}\\
          =&\;\overline{0}\otimes \overline{b} &(\because\; 2|2a)\\
          =&\;\overline{0}\otimes \overline{0}.\\
        \end{align*}
      \end{proof}
  \end{enumerate}

\it \textbf{Section 10.4 Q3:} Show that
  $\mathbb{C}\otimes_\mathbb{R}\mathbb{C}$ and
  $\mathbb{C}\otimes_\mathbb{C}\mathbb{C}$ are both left
  $\mathbb{R}$-modules but are not isomorphic as $\mathbb{R}$-modules.

  \begin{proof}
    $\mathbb{C}\otimes_\mathbb{R}\mathbb{C}$ is a left $\mathbb{R}$-module
    because $\mathbb{C}$ is trivally a $\mathbb{R}$-module. Similarly,
    $\mathbb{C}\otimes_\mathbb{C}\mathbb{C}$ is a left $\mathbb{C}$-module
    because $\mathbb{C}$ is trivially a $\mathbb{C}$-module. Then since
    $\mathbb{R}$ is a subring of $\mathbb{C}$ with the same unity,
    $\mathbb{C}\otimes_\mathbb{C}\mathbb{C}$ will also be a left
    $\mathbb{R}$-module. \\

    We first show that $\mathbb{C}\otimes_\mathbb{R}\mathbb{C}$ is
    isomorphic to $\mathbb{R}^4$ as $\mathbb{R}$-modules. Consider the map
    $\varphi:\mathbb{C}\times\mathbb{C}\rightarrow\mathbb{R}^4$ defined by
    $\varphi((a_0+b_0i,a_1+b_1i))=(a_0,b_0,a_1,b_1)$, where
    $a_0,b_0,a_1,b_1\in\mathbb{R}$. It is routine to show that $\varphi$ is
    a $\mathbb{R}$-bilinear map: Given
    $r,a_0,b_0,a_1,b_1,c_0,c_1,d_0,d_1\in\mathbb{R}$,
    \begin{align*}
      \;&\varphi(r(a_0+b_0i,a_1+b_1i)+(c_0+d_0i,c_1+d_1i))\\
      =\;&\varphi(ra_0+c_0+(rb_0+d_0)i, ra_1+c_1+(rb_1+d_1)i)\\
      =\;&(ra_0+c_0, rb_0+d_0, ra_1+c_1, rb_1+d_1)\\
      =\;&(ra_0,rb_0,ra_1,rb_1) +(c_0,d_0,c_1,d_1)\\
      =\;&r(a_0,b_0,a_1,b_1) +(c_0,d_0,c_1,d_1)\\
      =\;&r\varphi((a_0+b_0i,a_1+b_1i)) +\varphi((c_0+d_0i,c_1+d_1i)).\\
    \end{align*}
    Similarly, we can show that
    \[\varphi((a_0+b_0i,a_1+b_1i)+r(c_0+d_0i,c_1+d_1i))
    =\varphi((a_0+b_0i,a_1+b_1i)) +r\varphi((c_0+d_0i,c_1+d_1i)).\]

    Thus from Corollary 12, $\varphi$ induces an $\mathbb{R}$-module
    homomorphism $\Phi:\mathbb{C}\otimes_\mathbb{R}\mathbb{C}\rightarrow
    \mathbb{R}^4$ defined by
    $\Phi((a_0+b_0i)\otimes(a_1+b_1i))=(a_0,b_0,a_1,b_1)$. This map is
    clearly a bijection, thus $\mathbb{C}\otimes_\mathbb{R}\mathbb{C}
    \cong_\mathbb{R}\mathbb{R}^4$ as $\mathbb{R}$-modules. Since
    $\mathbb{R}^4$ has dimension 4 as an $\mathbb{R}$-module,
    $\mathbb{C}\otimes_\mathbb{R}\mathbb{C}$ also has dimension 4 as an
    $\mathbb{R}$-module. \\

    Next, observe that $\mathbb{C}\otimes_\mathbb{C}\mathbb{C}$ can be
    spanned as an $\mathbb{R}$-module by two elements
    $\{1\otimes1,1\otimes i\}$, because given $a,b,c,d\in\mathbb{R}$, we can
    write
    \begin{align*}
      &\;(a+bi)\otimes(c+di)\\
      =&\;a\otimes c+a\otimes di +bi\otimes c +bi\otimes di\\
      =&\;ac\cdot(1\otimes1) +ad\cdot(1\otimes i) +bc\cdot(i\otimes1)
        +bd\cdot(i\otimes i)\\
      =&\;ac\cdot(1\otimes1) +ad\cdot(1\otimes i) +bc\cdot(1\otimes i)
        -bd\cdot(1\otimes1)\\
      =&\;(ac-bd)\cdot(1\otimes1) +(ad+bc)\cdot(1\otimes i).\\
    \end{align*}

    Thus $\mathbb{C}\otimes_\mathbb{R}\mathbb{C}$ has $\mathbb{R}$-module
    dimension no greater than 2, while
    $\mathbb{C}\otimes_\mathbb{C}\mathbb{C}$ has $\mathbb{R}$-module
    dimension equal 4, so they cannot be isomorphic as
    $\mathbb{R}$-modules.
  \end{proof}
\end{document}
