\documentclass{article}
\usepackage[left=3cm,right=3cm,top=3cm,bottom=3cm]{geometry}
\usepackage{amsmath,amssymb,amsthm,pgfplots,tikz}
\usepackage[inline]{enumitem}
\usepackage{color}
\setlength{\parindent}{0mm} %So that we do not indent on new paragraphs
\newcommand{\TODO}[1]{\textcolor{red}{TODO: #1}}

\begin{document}
\title{Graduate Algebra II: Homework 3}
\author{Li Ling Ko\\ lko@nd.edu}
\date{\today}
\maketitle

\it \textbf{Q1:} Let $A$ and $B$ be abelian groups, and $m,n$ positive
  integers.
  \begin{enumerate}[label={(\alph*)}]
    \item Show that $A\otimes_\mathbb{Z}\mathbb{Z}_m \cong A/mA$.
      \begin{proof}
        Consider the map $\varphi:A\times\mathbb{Z}_m
        \rightarrow A/mA$ defined by
        $a\times\bar{z}\mapsto\overline{z\cdot a}$. We first show this map
        is well-defined and $\mathbb{Z}$-bilinear. To show that the map is
        well-defined, we have
        \begin{align*}
          \varphi((a,\overline{z+m})) &=\overline{(z+m)\cdot a}\\
          &=\overline{z\cdot a+m\cdot a}\\
          &=\overline{z\cdot a}\\
          &=\varphi((a,\overline{z})).\\
        \end{align*}

        To show bilinearlity, we have
        \begin{align*}
          \varphi(((a_1+a_2),\overline{z})) &=\overline{z\cdot
            (a_1+a_2)}\\
          &=\overline{z\cdot a_1 +z\cdot a_2}\\
          &=\overline{z\cdot a_1} +\overline{z\cdot a_2}\\
          &=\varphi((a_1,\overline{z}))
            +\varphi((a_2,\overline{z})),\\
        \end{align*}

        and
        \begin{align*}
          \varphi((a,(\overline{z_1}+\overline{z_2})))
            &=\varphi((a,(\overline{z_1+z_2})))\\
          &=\overline{(z_1+z_2)\cdot a}\\
          &=\overline{z_1\cdot a +z_2\cdot a}\\
          &=\overline{z_1\cdot a} +\overline{z_2\cdot a}\\
          &=\varphi((a,\overline{z_1}))
            +\varphi((a,\overline{z_2})),\\
        \end{align*}

        and
        \begin{align*}
          \varphi(((n\cdot a),\overline{z})))
          &=\overline{z\cdot (n\cdot a)}\\
          &=\overline{(nz)\cdot a}\\
          &=\overline{(n\cdot\overline{z})\cdot a}\\
          &=\varphi((a, (n\cdot \overline{z}))).\\
        \end{align*}

        Thus from Corollary 12, $\varphi$ induces a $\mathbb{Z}$-module
        homomorphism $\Phi:A\otimes_\mathbb{Z}\mathbb{Z}_m \rightarrow
        A/mA$ defined by $\Phi(a\otimes\bar{z})=\overline{z\cdot a}$. Now
        $\Phi$ is clearly surjective since every $\overline{a}\in
        A/mA$ has a pre-image $a\otimes\bar{1}$. It remains to show that
        $\Phi$ is injective:
        \begin{align*}
          \;&a\otimes\overline{z}\in\ker(\Phi)\\
          \Leftrightarrow\;&\overline{z\cdot a}=\overline{0}\in A/mA\\
          \Leftrightarrow\;&z\cdot a\in mA\\
          \Leftrightarrow\;&z\cdot a =m\cdot b\in A\; \text{for some}\;
            b\in A\\
          \Rightarrow\;&(z\cdot a)\otimes\bar{1}=(m\cdot
            b)\otimes\bar{1}\; \text{for some}\; b\in A.\\
        \end{align*}

        Now $(m\cdot b)\otimes\bar{1} =b\otimes(m\cdot\bar{1})
        =b\otimes\overline{m} =b\otimes\bar{0} =b\otimes(0\cdot\bar{1})
        =(0\cdot b)\otimes\overline{0} =0\otimes\overline{0}$. Also,
        $(z\cdot a)\otimes\bar{1} =a\otimes(z\cdot\bar{1})
        =a\otimes\overline{z}$. Thus we have
        \[a\otimes\overline{z} =0\otimes\overline{0},\]
        which means $\ker(\Phi)=0$.
      \end{proof}
  \end{enumerate}
\end{document}
