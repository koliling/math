\documentclass{article}
\usepackage[left=3cm,right=3cm,top=3cm,bottom=3cm]{geometry}
\usepackage{amsmath,amssymb,amsthm,pgfplots,tikz}
\usepackage[inline]{enumitem}
\usepackage{color}
\setlength{\parindent}{0mm} %So that we do not indent on new paragraphs
\newcommand{\TODO}[1]{\textcolor{red}{TODO: #1}}

\begin{document}
\title{Graduate Algebra II: Homework 4}
\author{Li Ling Ko\\ lko@nd.edu}
\date{\today}
\maketitle

\it \textbf{Section 12.1 Q5:}
  \begin{proof}
    $\{2,x\}$ is not a basis of $M$ because $x,-2\in R$ witnesses their
    $R$-linear dependency; i.e. $x\cdot2+(-2)\cdot x=0$. \\

    Any two distinct elements $f(x)\neq g(x)\in M$ are $R$-linearly
    dependent with $g(x),-f(x)$ as witness. Thus $M$ has rank less than 2,
    and so can only be of rank 1. \\

    Since $M$ is an $R$-module with rank 1, if it is also free-module, then
    $M=Rf(x)$ for some $f(x)\in M$. In particular, $2,x\in Rf(x)$, so
    $f(x)$ divides $2$ and $x$ in $R$. Thus $f(x)$ divides a greatest
    common divisor of $2$ and $x$ which is 1, and thus is either 1 or -1.
    Yet neither of these are in $M$, so $M$ is not a free module.
  \end{proof}

\it \textbf{Section 12.1 Q6:}
\end{document}
