\documentclass{article}
\usepackage[left=3cm,right=3cm,top=3cm,bottom=3cm]{geometry}
\usepackage{amsmath,amssymb,amsthm,pgfplots,tikz}
\usepackage[inline]{enumitem}
\usepackage{color}
\setlength{\parindent}{0mm} %So that we do not indent on new paragraphs
\newcommand{\TODO}[1]{\textcolor{red}{TODO: #1}}

\begin{document}
\title{Graduate Algebra II: Homework 4}
\author{Li Ling Ko\\ lko@nd.edu}
\date{\today}
\maketitle

\it \textbf{Q2:}
  \begin{enumerate}[label={(\alph*)}]
    \item How many abelian groups of order 2704 are there, up to
      isomorphism?
      \begin{proof}
        The prime factorization of 2704 is $2^4\cdot13^2$. By the
        unique decomposition of finite abelian groups by invariant factors,
        there two finite abelian groups of order $13^2$ and five finite
        abelian groups of order $2^4$, giving a total of $2\times5=10$
        abelian groups of order 2704.
      \end{proof}

    \item For each isomorphism class of abelian groups of order 2704, give
      the elementary divisors and the invariant factors.
      \begin{proof}
        By unique decomposition theorem, we have the following 10
        decompositions:
        \begin{table}[h]
          \begin{center}
            \begin{tabular}{ll}
              elementary divisors &invariant factors\\
              \hline
              $13^2,2^4$ &$13^2\cdot2^4$\\
              $13^2,2^3,2$ &$13^2\cdot2^3,2$\\
              $13^2,2^2,2^2$ &$13^2\cdot2^2,2^2$\\
              $13^2,2^2,2,2$ &$13^2\cdot2^2,2,2$\\
              $13^2,2,2,2,2$ &$13^2\cdot2,2,2,2$\\
              $13,13,2^4$ &$13\cdot2^4,13$\\
              $13,13,2^3,2$ &$13\cdot2^3,13\cdot2$\\
              $13,13,2^2,2^2$ &$13\cdot2^2,13\cdot2^2$\\
              $13,13,2^2,2,2$ &$13\cdot2^2,13\cdot2,2$\\
              $13,13,2,2,2,2$ &$13\cdot2,13\cdot2,2,2$\\
            \end{tabular}
          \end{center}
        \end{table}
      \end{proof}
  \end{enumerate}

\it \textbf{Q3:} How many groups of order 1225 are there up to isomorphism?
  Prove your assertion and give the isomorphism type of each distinct
  isomorphism class.

  \begin{proof}
    The prime factorization of 1225 is $5^2\cdot7^2$. Let $G$ be a group of
    order 1225. By the Sylow laws, there is exactly 1 Sylow 5-subgroup
    $N_5$ and 1 Sylow 7-subgroup $N_7$. Furthermore, because 5 and 7 are
    coprime, $G$ must be the direct product $G=N_5N_7$. Now $N_5$ and $N_7$
    have orders the square of primes, which are abelian. Hence $G$ is
    abelian, with the following possible sets of elementary divisors:
    \begin{enumerate}[label={(\alph*)}]
      \item $5^2,7^2$
      \item $5,5,7^2$
      \item $5^2,7,7$
      \item $5,5,7,7$
    \end{enumerate}
  \end{proof}

\it \textbf{Section 12.1 Q5:} Let $R=\mathbb{Z}[x]$ and let $M=(2,x)$ be
  the ideal generated by 2 and $x$, considered as a submodule of $R$. Show
  that $\{2,x\}$ is not a basis of $M$. Show that the rank of $M$ is 1 but
  that $M$ is not free of rank 1.

  \begin{proof}
    $\{2,x\}$ is not a basis of $M$ because $x,-2\in R$ witnesses their
    $R$-linear dependency; i.e. $x\cdot2+(-2)\cdot x=0$. \\

    Any two distinct elements $f(x)\neq g(x)\in M$ are $R$-linearly
    dependent with $g(x),-f(x)$ as witness. Thus $M$ has rank less than 2,
    and so can only be of rank 1. \\

    Since $M$ is an $R$-module with rank 1, if it is also free-module, then
    $M=Rf(x)$ for some $f(x)\in M$. In particular, $2,x\in Rf(x)$, so
    $f(x)$ divides $2$ and $x$ in $R$. Thus $f(x)$ divides a greatest
    common divisor of $2$ and $x$ which is 1, and thus is either 1 or -1.
    Yet neither of these are in $M$, so $M$ is not a free module.
  \end{proof}

\it \textbf{Section 12.1 Q6:} Show that if $R$ is an integral domain and
  $M$ is any non-principal ideal of $R$ then $M$ is torsion free of rank 1
  but is not a free $R$-module.

  \begin{proof}
    $M$ being torsion-free follows directly from $R$ being an integral
    domain - if $r\in R$ and $m\in M$ are non-zero elements, then $r\cdot
    m\neq 0$. Let $m,n\in M$ be distinct elements in $M$. Then $m,n$ are
    $R$-linearly dependent, with coefficients $n,-m$ as witness. Thus $M$
    has rank less than 2, and can only be of rank 1. Thus, if $M$ is a free
    $R$-module, then $M=Rm$ for some $m\in M$, which contradicts
    non-principality of $M$.
  \end{proof}

\it \textbf{Section 12.1 Q8:} Let $R$ be a PID, let $B$ be a torsion
  $R$-module and let $p$ be a prime in $R$. Prove that if $pb=0$ for some
  nonzero $b\in B$, then $\text{Ann}(B)\subseteq(p)$.

  \begin{proof}
    Since $\text{Ann}(B)\subseteq\text{Ann}(b)$, it suffices to show that
    $\text{Ann}(b)=(p)$. Since $p\in\text{Ann}(b)$, and $\text{Ann}(b)$ is
    an ideal, we have $(p)\subseteq\text{Ann}(b)$. Now since $R$ is a PID,
    $\text{Ann}(b)=(r)$ for some $r\in R$. But for PIDs, an ideal is
    contained in another if and only if the generator of the larger ideal
    divides the generator of the smaller ideal. Thus $(p)\subseteq(r)$ if
    and only if $r|p$. Yet because $p$ is prime, its only divisors are the
    units and its associates. The ideal generated by the associates of $p$
    will be $(p)$, and the ideal generated by units are $R$. But because
    $b$ is nonzero, its annihilator cannot be the whole of $R$, its
    annihilator can only be $(p)$.
  \end{proof}

\it \textbf{Section 12.1 Q10:} For $p$ a prime in the PID, $R$ and $N$ a
  torsion $R$-module prove that the $p$-primary component of $N$ is a
  submodule of $N$ and prove that $N$ is the direct sum of its $p$-primary
  components (there need not be finitely many of them).

  \begin{proof}
    Let $p\in R$ be prime, and let $N_p\subseteq N$ denote the $p$-primary
    component of $N$. Then $N_p$ is non-empty since $p\cdot0=0$. Let
    $x,y\in N_p$ and $r\in R$. Then $p^\alpha\cdot x=p^\beta\cdot y=0$ for
    some $\alpha,\beta\in\mathbb{N}$, so
    \[p^{\alpha+\beta}\cdot(rx+y) =rp^\beta\cdot(p^\alpha\cdot
    x)+p^\alpha\cdot(p^\beta\cdot y) =0+0=0.\]
    So $rx+y\in N_p$, and $N_p$ is non-empty, implying that $N_p$ is a
    submodule of $N$. \\

    From easy observation, note that if primes $p,q\in N$ are associates,
    then $N_p=N_q$. Let $P\subset R$ be a set of primes in $R$, with one
    prime to represent itself and its associates in $R$, and where every
    prime in $R$ is represented in $P$. Consider the map
    $\varphi:N\rightarrow \oplus_{p\in P}N_p$, defined as follows: Let
    $n\in N$ with annihilator $\text{Ann}(n)=(a)$, where
    $a=p_1^{\alpha_1}\ldots p_k^{\alpha_k}$ where the $p_i$'s are distinct
    primes in $P$, and the $\alpha_i$'s are positive integers. Such a
    representation of $\text{Ann}(n)$ exists because $R$ is a PID. Letting
    $q_i$ denote $a/p_i^{\alpha_i}$, then $\text{gcd}(q_1,\ldots,q_k)=1$,
    or equivalently, $(q_1,\ldots,q_k)=(1)=R$. Thus there there exists
    $t_1,\ldots,t_k\in R$ such that \[q_1t_1+\ldots+q_1t_k=1.\] Note that
    the $t_i$'s and the notion of gcd exists because $R$ is a PID. Then let
    $\varphi$ map $n$ to $(n_p)_{p\in P}$, where
    \begin{equation*}
      n_p=
      \begin{cases}
        q_it_i\cdot n, &\text{if}\; p=p_i,\\
        0, &\text{otherwise}. \\
      \end{cases}
    \end{equation*}

    We show that $\varphi$ is an $R$-module isomorphism.
  \end{proof}

\it \textbf{Section 12.1 Q11:} Let $R$ be a PID, let $a$ be a nonzero
  element of $R$ and let $M=R/(a)$. For any prime $p$ of $R$ prove that
  \begin{equation*}
    p^{k-1}M/p^kM\cong
    \begin{cases}
      R/(p) &\text{if}\; k\leq n\\
      0 &\text{otherwise},\\
    \end{cases}
  \end{equation*}
  where $n$ is the power of $p$ dividing $a$ in $R$.

  \begin{proof}
    Write $a=p^nq$, where $p$ and $q$ are coprime. Now
    $p^kM=p^k(R/(p^nq))$ is the image of the ideal $(p^k)$ in the quotient
    $R/(p^nq)$, and hence
    \begin{align*}
      p^kM &=((p^k)+(p^nq))/(p^nq)\\
      &=(\text{gcd}(p^k,p^nq))/(p^nq)\\
      &=
      \begin{cases}
        (p^k)/(p^nq) &\text{if}\; k\leq n\\
        (p^n)/(p^nq) &\text{otherwise}\\
      \end{cases}
    \end{align*}

    So when $k\leq n$,
    \[p^{k-1}M/p^kM \cong\frac{(p^{k-1})/(a)}{(p^k)/(a)}
    \cong(p^{k-1})/(p^k)\]
    by the Third isomorphism theorem, which holds here because because
    $(p^{k-1})\supset(p^k)\supseteq(a)$ when $k\leq n$. Thus it suffices to
    show that $(p^{k-1})/(p^k) \cong R/(p)$. To this end, consider the
    map $\pi:R\rightarrow(p^{k-1})/(p^k)$ given by
    $\pi(r)=\overline{p^{k-1}r}$. It is routine to check that $\pi$ is a
    surjective ring isomorphism with kernel $(p)$, thus by the first
    isomorphism theorem, $(p^{k-1})/(p^k) \cong R/(p)$. \\

    On the other hand, if $k>n$, then
    \[p^{k-1}M/p^kM \cong\frac{(p^n)/(a)}{(p^n)/(a)} \cong0.\]
  \end{proof}

\it \textbf{Section 12.1 Q12:} Let $R$ be a PID and let $p$ be a prime in
  $R$.
  \begin{enumerate}[label={(\alph*)}]
    \item Let $M$ be a finitely generated torsion $R$-module. Use the
      previous exercise to prove that $p^{k-1}M/p^kM\cong F^{n_k}$ where
      $F$ is the field $R/(p)$ and $n_k$ is the number of elementary
      divisors of $M$ which are powers $p^\alpha$ with $\alpha\geq k$.

      \begin{proof}
        By the fundamental theorem, since $M$ is finitely generated
        torsion, we can assume $M=R/(p_1^{\alpha_1})
        \oplus\ldots \oplus R/(p_n^{\alpha_n})$, where the $p_i^{\alpha_i}$
        are the elementary divisors of $M$. Then
        \begin{align*}
          \frac{p^{k-1}M}{p^kM} &=\frac{p^{k-1}(R/(p_1^{\alpha_1})
            \oplus\ldots \oplus R/(p_n^{\alpha_n}))}{p^k(R/(p_1^{\alpha_1})
            \oplus\ldots \oplus R/(p_n^{\alpha_n}))}\\
          &=\frac{p^{k-1}(R/(p_1^{\alpha_1}))
            \oplus\ldots \oplus p^{k-1}(R/(p_n^{\alpha_n}))}
            {p^{k}(R/(p_1^{\alpha_1})) \oplus\ldots \oplus
            p^{k}(R/(p_n^{\alpha_n}))}\\
          &\cong\frac{p^{k-1}(R/(p_1^{\alpha_1}))}
            {p^{k}(R/(p_1^{\alpha_1}))} \oplus\ldots
            \oplus\frac{p^{k-1}(R/(p_n^{\alpha_n}))}
            {p^{k}(R/(p_n^{\alpha_n}))}\\
          &\cong F^{n_k},\\
        \end{align*}
        where the last equality follows from question 11, and the second
        and third equalities follow easily from constructing the natural
        isomorphism between the relevant rings.
      \end{proof}

    \item Suppose $M_1$ and $M_2$ are isomorphic finitely generated torsion
      $R$-modules. Use (a) to prove that, for every $k\geq0$, $M_1$ and
      $M_2$ have the same number of elementary divisors $p^\alpha$ with
      $\alpha\geq k$. Prove that this implies $M_1$ and $M_2$ have the same
      set of elementary divisors.

      \begin{proof}
        Let $n_{i,k}$ be the number of elementary divisors of $M_i$ which
        are powers of $p^\alpha$ with $\alpha>k$. Then from the previous
        exercise,
        \begin{align*}
          F^{n_{1,k}} &\cong\frac{p^{k-1}M_1}{p^kM_1}\\
          &\cong\frac{p^{k-1}M_2}{p^kM_2}\\
          &\cong F^{n_{2,k}},\\
        \end{align*}
        which implies that $n_{1,k}=n_{2,k}$. Since this holds for all
        $k\geq0$ and prime $p$, $M_1$ and $M_2$ must have the same
        elementary divisors.
      \end{proof}
  \end{enumerate}

\it \textbf{Section 12.1 Q13:} If $M$ is finitely generated module over the
  PID $R$, describe the structure of $M/\text{Tor}(M)$.
  \begin{proof}
    By the fundamental theorem, we can write
    \[M\cong R^t \oplus R/(a_1)\oplus\ldots \oplus R/(a_n),\]
    where the $a_i$'s are the invariant factors of $M$. Thus
    \[\text{Tor}(M) \cong R/(a_1)\oplus\ldots \oplus R/(a_n),\]
    so
    \begin{align*}
      M/\text{Tor}(M) &=\frac{R^t \oplus R/(a_1)\oplus\ldots \oplus R/(a_n)}
        {0\oplus R/(a_1)\oplus\ldots \oplus R/(a_n)}\\
      &\cong R^t,\\
    \end{align*}
    where the last equality follows from proving that the natural
    projection \[\varphi:R^t \oplus R/(a_1)\oplus\ldots \oplus R/(a_n)
    \rightarrow R^t\] is surjective with kernel $R/(a_1)\oplus\ldots \oplus
    R/(a_n)$.
  \end{proof}

\it \textbf{Section 12.1 Q14:} Let $R$ be a PID, and let $M$ be a torsion
  $R$-module. Prove that $M$ is irreducible if and only $M=Rm$ for any
  nonzero element $m\in M$ where the annihilator of $m$ is a nonzero prime
  ideal $(p)$.

  \begin{proof}
    The forward direction of the assertion follows trivially from Exercise
    9 of Section 10.3, which says that an $R$-module $M$ is irreducible if
    and only it is nonzero and generated by any nonzero element of $M$. \\

    For the reverse direction, assume $M=Rm$ for a nonzero element $m\in M$
    where $\text{Ann}(m)$ is a nonzero prime ideal $(p)$. Consider the map
    $\varphi:R\rightarrow(m)$ defined by $\varphi(r)=r\cdot m$. It is
    routine to show that $\varphi$ is a surjective $R$-module homomorphism
    with kernel $\text{Ann}(m)=(p)$, thus by the first isomorphism theorem,
    the $M=Rm\cong R/(p)$, which is equivalent to $M$ being irreducible
    from Exercise 10 of Section 10.3.
  \end{proof}

\it \textbf{Section 12.1 Q20:} Let $R$ be an integral domain with quotient
  field $F$ and let $M$ be any $R$-module. Prove that the rank of $M$
  equals the dimension of the vector space $F\otimes_RM$ over $F$.

  \begin{proof}
    Let $B\subseteq M$ be an $R$-linearly independent set, let
    $B_F:=\{1\otimes b:b\in B\} \subseteq F\otimes_RM$, and assume $\sum
    r_i(1\otimes b_i)=0$ for some $r_i\in F$. Clearing denominators, we can
    assume $r_i\in R$. Rearranging gives $1\otimes\sum r_ib_i=0$. Then
    $\sum r_ib_i=0$: it is routine to show that $\varphi:R\times M$ defined
    by $\varphi(r,m)=r\cdot m$ is an $R$-bilinear map, so by the universal
    property $\Phi:R\otimes_RM\rightarrow M$ defined by $\Phi(r\otimes
    m)=r\cdot m$ is an $R$-module homomorphism. Thus $0=\Phi(0)
    =\Phi(1\otimes(\sum r_ib_i)) =\sum r_ib_i\in M$. So from linear
    independency of the $b_i$'s, the $r_i$'s must be 0. Thus the elements
    in $B_F$ are $R$-linearly independent, and so
    $\text{rank}_R(M)\leq\text{dim}_F(F\otimes_RM)$. \\

    Conversely, let $b_i=\sum \alpha_{i,j}\otimes m_{i,j}\in F\otimes_RM$
    form a linearly independent set. Then clearing the denominators of
    $\alpha_{i,j}$, we can assume that $\alpha_{i,j}\in R$.  Further,
    expressing $\alpha_{i,j}\otimes m_{i,j}$ as $1\otimes(\alpha_{i,j}\cdot
    m_{i,j})$ and using the fact that $\sum 1\otimes n_i=i\otimes(\sum
    n_i)$, we can assume that the linearly independent set is of the form
    $1\otimes m_i$ for some $m_i\in M$. Thus if $\sum r_im_i=0\in M$, then
    $r_i(1\otimes m_i)=1\otimes(\sum r_im_i)=1\otimes0=0$, which implies
    $r_i=0$ by linear independence of $1\otimes m_i$. Thus
    $\text{rank}_R(M)\geq\text{dim}_F(F\otimes_RM)$, so $\text{rank}_R(M)
    =\text{dim}_F(F\otimes_RM)$.
  \end{proof}

\it \textbf{Section 12.1 Q21:} Prove that a finitely generated module over
  a PID is projective if and only if it is free.

  \begin{proof}
  \end{proof}
\end{document}
