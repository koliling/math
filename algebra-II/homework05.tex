\documentclass{article}
\usepackage[left=3cm,right=3cm,top=3cm,bottom=3cm]{geometry}
\usepackage{amsmath,amssymb,amsthm,pgfplots,tikz}
\usepackage[inline]{enumitem}
\usepackage{color}
\setlength{\parindent}{0mm} %So that we do not indent on new paragraphs
\newcommand{\TODO}[1]{\textcolor{red}{TODO: #1}}

\begin{document}
\title{Graduate Algebra II: Homework 5}
\author{Li Ling Ko\\ lko@nd.edu}
\date{\today}
\maketitle

\it \textbf{Section 12.2 Q3:} Prove that two $2\times2$ matrices over $F$
  which are not scalar matrices are similar if and only if they have the
  same characteristic polynomial.

  \begin{proof}
    We first show that similar matrices have the same characteristic
    polynomial. Given similar matrices $A$ and $P^{-1}AP$,
    \begin{align*}
      \text{char}(P^{-1}AP) &=|xI-P^{-1}AP|\\
      &=|P(xI-P^{-1}AP)P^{-1}|\\
      &=|P(xI)P^{-1}-A|\\
      &=|xI-A|\\
      &=\text{char}(A).\\
    \end{align*}

    For the converse, let $A,B\in\text{Mat}_2(F)$ be two non-scalar
    matrices over with the same characteristic polynomial. Suppose the
    characteristic polynomial of $A$ and $B$ has two distinct roots, or is
    an irreducible polynomial of degree two. Then the minimal polynomial of
    the matrices must also be the characteristic polynomial by Proposition
    20, so the matrices have the same rational canonical form, and
    are therefore similar to each other from Theorem 17. \\

    On the other hand, if the roots of the characteristic polynomial are
    not distinct, i.e.
    \[c_A(x)=c_B(x)=(x-\alpha)^2\]
    for some $\alpha\in F$, then $m_A(x)$ is either $x-\alpha$ or
    $(x-\alpha)^2$. If $m_A(x)=x-\alpha$, then the invariant factors of $A$
    are $x-\alpha$ and $x-\alpha$, which implies that the rational
    canonical form of $A$ is $\alpha I$. Now since $A$ is similar to its
    rational canonical form $\alpha I$,we have $A=Q^{-1}(\alpha I)Q=\alpha
    I$ for some invertible matrix $Q\in\text{Mat}_2(F)$, which makes $A$ a
    scalar matrix, a contradiction. Therefore we must have
    $m_A(x)=m_B(x)=(x-\alpha)^2$, and so $A$ and $B$ have the same rational
    canonical form, making them similar by Theorem 17.
  \end{proof}

\it \textbf{Section 12.2 Q4:} Prove that two $3\times3$ matrices are
  similar if and only if they have the same characteristic and same minimal
  polynomial. Give an explicit counterexample to this assertion for
  $4\times4$ matrices.

  \begin{proof}
    Assume $A,B\in\text{Mat}_3(F)$ are similar. Then from the argument in
    the previous question, the $A$ and $B$ have the same characteristic
    polynomial. Also, from Theorem 15, $A$ and $B$ have the same rational
    canonical form and thus also have the same minimal polynomial.\\

    For the converse, assume $A,B\in\text{Mat}_3(F)$ have the same
    characteristic polynomial $c(x)$ and minimal polynomial $m(x)$. If
    $c(x)$ has distinct roots, or is an irreducible polynomial of degree 3,
    or is the product of an irreducible polynomial of degree 2 with a
    linear factor, then by Proposition 20, $A$ and $B$ will have the same
    invariant factors, and therefore are similar from Theorem 17. \\

    On the other hand, if $c(x)$ has only two distinct roots, i.e.
    \[c_A(x)=c_B(x)=c(x) =(x-\alpha)(x-\beta)^2\]
    for some $\alpha\neq\beta\in F$, the minimal polynomial $m(x)$ of $A$
    and $B$ can either be $c(x)$ or $(x-\alpha)(x-\beta)$. If
    $m(x)=c(x)$, then the invariant factors of $A$ and of $B$ can only be
    $m(x)$, and then $A$ would be similar to $B$ from Theorem 17. If
    $m(x)=(x-\alpha)(x-\beta)$, then the invariant factors of both matrices
    can only be $(x-\beta)$ and $(x-\alpha)(x-\beta)$. Then again, $A$ and
    $B$ are similar from Theorem 17. \\

    Finally, if $c(x)$ has only one root, i.e.
    \[c_A(x)=c_B(x)=c(x) =(x-\alpha)^3\]
    for some $\alpha\in F$, the minimal polynomial $m(x)$ of $A$ and $B$
    can either be $x-\alpha$, $(x-\alpha)^2$, or $(x-\alpha)^3$. If it is
    the first case, then the invariant factors of $A$ and of $B$ can only
    be $x-\alpha$ three times, and then $A$ would be similar to $B$ from
    Theorem 17. If it is the second case, then the invariant factors can
    only be $x-\alpha$ and $(x-\alpha)^2$, and again $A$ would be similar
    to $B$. Finally, if it is the third case, then the invariant factors
    can only be $(x-\alpha)^3$, and then $A$ and $B$ would still be
    similar from Theorem 17. \\

    Thus two $3\times3$ matrices are similar if and only if they have the
    same characteristic and minimal polynomial. The assertion for
    $4\times4$ matrices $A$ and $B$ will fail if for both matrices, the
    characteristic polynomial is $(x-\alpha)^4$ and the minimal polynomial
    is $(x-\alpha)^2$. Then if $A$ has invariant factors $(x-\alpha)^2$ and
    $(x-\alpha)^2$ and $B$ has invariant factors $x-\alpha$, $x-\alpha$,
    and $(x-\alpha)^2$, by Theorem 17, their respective rational canonical
    forms $A'$ and $B'$ will not be similar even though they have the same
    characteristic and minimal polynomials.
  \end{proof}

\it \textbf{Section 12.2 Q6:} Prove that the constant term in the
  characteristic polynomial of the $n\times n$ matrix $A$ is
  $(-1)^n\text{det}A$ and that the coefficient of $x^{n-1}$ is the negative
  of the sum of the diagonal entries of $A$ (the sum of the diagonal
  entries of $A$ is called the trace of $A$). Prove that $\text{det}A$ is
  the product of the eigenvalues of $A$ and that the trace of $A$ is the
  sum of the eigenvalues of $A$.

  \begin{proof}
    The constant term of $c_A(x)$ is
    \[c_A(0)=\text{det}(0I-A)=\text{det}(-A)=(-1)^n\text{det}(A).\]

    To prove the assertion on the coefficient of $x^{n-1}$, first note that
    given two matrices $X$ and $Y$, we have $\text{tr}(XY)=\text{tr}(YX)$,
    and therefore similar matrices will have the same trace. Hence it
    suffices to prove that the coefficient of $x^{n-1}$ is the same as the
    trace of the rational canonical form of $A$. By definition, if the
    invariant factors of $A$ are $m_1(x),\ldots,m_k(x)$, then the trace of
    the rational canonical form of $A$ is $-\sum_{i=1}^k c_{i}$, where
    $c_{i}$ is the coefficient of the second highest degree of $m_i(x)$.
    Now since similar matrices have the same characteristic polynomial, we
    can assume that the matrix $A$ is of rational canonical form. Then the
    coefficient of $x^{n-1}$ of $c_A(x)=m_1(x)\cdots m_k(x)$ will be
    $-\sum_{i=1}^k c_{i}$, as required. \\

    The $n$ eigenvalues of $A$ (including multiplicities)
    $\lambda_1,\ldots,\lambda_n$ are exactly the set of
    roots of the characteristic polynomial of $A$, i.e.
    \[c_A(x) =(x-\lambda_1)\cdots(x-\lambda_n).\]
    Setting $x=0$ in the above equation gives us
    $c_A(0)=(-1)^n\lambda_1\cdots\lambda_n$. And since $c_A(0)$ is
    $(-1)^n\text{det}(A)$ as shown earlier, we have
    \[\text{det}(A) =\lambda_1\cdots\lambda_n.\]

    Similarly, in the expansion of $c_A(x)$, the coefficient of $x^{n-1}$
    is the negative of the sum of the eigenvalues of $A$. And since we have
    shown earlier that the former is the same as $-\text{tr}(A)$, we get
    that the sum of the eigenvalues is the same as $\text{tr}(A)$.
  \end{proof}

\it \textbf{Section 12.2 Q7:} Determine the eigenvalues of the matrix
  \[A=\begin{pmatrix}
    0&1&0&0\\
    0&0&1&0\\
    0&0&0&1\\
    1&0&0&0\\
  \end{pmatrix}.\]

  \begin{proof}
    Let $v$ be an eigenvector of $A$ with corresponding eigenvalue
    $\lambda$. Observe that $A$ describes a permutation of order four, and
    therefore satisfies $A^4=I$. Then it is routine to show by induction on
    $n$ that $A^nv=\lambda^nv$, which implies in particular that
    $\lambda^4$ is a eigenvalue of $A^4=I$. But the only eigenvalue of $A$
    is 1, thus the eigenvalues of $A$ must be from the set of fourth roots
    of unities, i.e. from the set $\{1,-1,i,-1\}$. \\

    Now from the previous question, the sum of the eigenvalues is
    $\text{tr}(A)=0$ and their product is $\text{det}(A)=-1$. Also,
    observe that $[1,1,1,1]^T$ is an eigenvector of $A$ with eigenvalue 1.
    So in order for the four eigenvalues to sum to 0 given that one of them
    is 1 and the eigenvalues must be from the set $\{1,-1,i,-1\}$, we must
    have that -1 is also one of the eigenvalues. So the remaining two
    eigenvalues must sum to 0 and multiply to 1, meaning they can either
    be 1 and -1, or $i$ and $-i$. We can easily check that
    $\text{det}(iI-A)=0$, therefore one of the eigenvalues must be $i$.
    Hence the eigenvalues are $1,-1,i,-i$.
  \end{proof}

\it \textbf{Section 12.2 Q9:} Find the rational canonical forms of
  \begin{align*}
    A=\begin{pmatrix}
      0&-1&-1\\
      0&0&0\\
      -1&0&0\\
    \end{pmatrix}, \text{and}\;
    B=\begin{pmatrix}
      c&0&-1\\
      0&c&1\\
      -1&1&c\\
    \end{pmatrix}.
  \end{align*}

  \begin{proof}
    We first find the eigenvalues of $A$. Since the products of the
    eigenvalues must equal $(-1)^3\text{det}(A)=0$, one of the eigenvalues
    must be 0. Also, the sum of the eigenvalues must equal the trace of $A$
    which is 0, so the remaining two eigenvalues must be additive inverses.
    Observing that $[1,0,1]^T$ is an eigenvector with eigenvalue -1, we get
    that the eigenvalues must be $\{0,1,-1\}$. Then from Proposition 20,
    since the eigenvalues are distinct, $A$ can only have one invariant
    factor, which is $m_A(x)=x(x-1)(x+1)=x^3-x$. Thus the rational
    canonical form of $A$ is the single companion matrix
    $\mathcal{C}_{m_A(x)}$, given by
    \[\begin{pmatrix}
      0&0&0\\
      1&0&1\\
      0&1&0\\
    \end{pmatrix}.\]

    Next, we find the eigenvalues of $C$. Note that $B=cI+C$, where
    \begin{align*}
      C=\begin{pmatrix}
        0&0&-1\\
        0&0&1\\
        -1&1&0\\
      \end{pmatrix}.
    \end{align*}

    Therefore
    \begin{align*}
      c_B(x) &=\text{det}(xI-B)\\
      &=\text{det}(xI-cI-C)\\
      &=\text{det}((x-c)I-C)\\
      &=c_C(x-c).
    \end{align*}

    Since the eigenvalues of $C$ are the roots of $c_C(x)$, the eigenvalues
    of $B$ must be the same roots but plus $c$. Thus we attempt to find the
    eigenvalues of $C$. We have shown in Question 6 the sum of the
    eigenvalues is the trace of $C$, which is 0, and their product is
    $(-1)^3\text{det}(C)=0$. Thus one of the eigenvalues must be 0, and the
    other two eigenvalues must be additive inverses. Indeed,
    \[c_C(x)=\text{det}(xI-C)=x^3-2x,\]
    which has distinct roots $\{0,\sqrt{2},-\sqrt{2}\}$.

    Then
    \[c_B(x)=c_C(x-c)=(x-c)^3-2(x-c) =x^3-3cx^2+(3c^2-2)x+(2c-c^3),\]
    and the eigenvalues of $B$ must be distinct since those of $C$ are also
    distinct. Thus from Proposition 20, $B$ can have only one invariant
    factor, which is the same as its characteristic polynomial. Hence the
    rational canonical form of $B$ is the single companion matrix
    $\mathcal{C}_{c_B(x)}$, given by
    \[\begin{pmatrix}
      0&0&c^3-2c\\
      1&0&2-3c^2\\
      0&1&3c\\
    \end{pmatrix}.\]
  \end{proof}

\it \textbf{Section 12.2 Q10:} Find all similarlity classes of $6\times6$
  matrices over $\mathbb{Q}$ with minimal polynomial $m(x)=(x+2)^2(x-1)$ (it
  suffices to give all lists of invariant factors and write out some of
  their corresponding matrices).

  \begin{proof}
    From Proposition 20, the remaining invariant factors can only have
    roots -2 or 1, and must divide $m(x)$. Furthermore, the products the
    invariant factors must have degree 6, and the first invariant factor
    must divide the second and so on. Finally, the similarity classes is
    completely characterized by its corresponding set of invariant factors.
    Thus these are the similarity classes, with some respective rational
    canonical forms:

    \begin{table}[h]\centering
    \begin{tabular}{lc}
      Invariant factors &Rational canonical form\\\hline\\
      $(x+2)^2(x-1)$, $(x+2)^2(x-1)$
        &$\begin{pmatrix}
          0&0&4&0&0&0\\
          1&0&0&0&0&0\\
          0&1&-3&0&0&0\\
          0&0&0&0&0&4\\
          0&0&0&1&0&0\\
          0&0&0&0&1&-3\\
        \end{pmatrix}$\\\\
      $x+2$, $(x+2)(x-1)$, $(x+2)^2(x-1)$
        &$\begin{pmatrix}
          -2&0&0&0&0&0\\
          0&0&2&0&0&0\\
          0&1&-1&0&0&0\\
          0&0&0&0&0&4\\
          0&0&0&1&0&0\\
          0&0&0&0&1&-3\\
        \end{pmatrix}$\\\\
      $x-1$, $(x+2)(x-1)$, $(x+2)^2(x-1)$
        &$\begin{pmatrix}
          1&0&0&0&0&0\\
          0&0&2&0&0&0\\
          0&1&-1&0&0&0\\
          0&0&0&0&0&4\\
          0&0&0&1&0&0\\
          0&0&0&0&1&-3\\
        \end{pmatrix}$\\\\
      $x+2$, $(x+2)^2$, $(x+2)^2(x-1)$&\\
      $x+2$, $x+2$, $x+2$, $(x+2)^2(x-1)$&\\
      $x-1$, $x-1$, $x-1$, $(x+2)^2(x-1)$&\\
    \end{tabular}
    \end{table}
  \end{proof}
\end{document}
