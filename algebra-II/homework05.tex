\documentclass{article}
\usepackage[left=3cm,right=3cm,top=3cm,bottom=3cm]{geometry}
\usepackage{amsmath,amssymb,amsthm,pgfplots,tikz}
\usepackage[inline]{enumitem}
\usepackage{color}
\setlength{\parindent}{0mm} %So that we do not indent on new paragraphs
\newcommand{\TODO}[1]{\textcolor{red}{TODO: #1}}

\begin{document}
\title{Graduate Algebra II: Homework 5}
\author{Li Ling Ko\\ lko@nd.edu}
\date{\today}
\maketitle

\it \textbf{Section 12.2 Q3:} Prove that two $2\times2$ matrices over $F$
  which are not scalar matrices are similar if and only if they have the
  same characteristic polynomial.
  \begin{proof}
    We first show that similar matrices have the same characteristic
    polynomial. Given similar matrices $A$ and $P^{-1}AP$,
    \begin{align*}
      \text{char}(P^{-1}AP) &=|xI-P^{-1}AP|\\
      &=|P(xI-P^{-1}AP)P^{-1}|\\
      &=|P(xI)P^{-1}-A|\\
      &=|xI-A|\\
      &=\text{char}(A).\\
    \end{align*}

    For the converse, let $A,B\in\text{Mat}_2(F)$ be two non-scalar
    matrices over with the same characteristic polynomial. Assume the
    characteristic polynomial of $A$ and $B$ has distinct roots. Write
    \[c_A(x) =c_B(x) =(x-\alpha)(x-\beta),\]
    where $\alpha\neq\beta\in F$.
    Then $A$ and $B$ must also have the same minimal polynomial which
    equals their characteristic polynomial, by Proposition 20.3. Therefore
    $A$ and $B$ have the same rational canonical form, and so $A$ and $B$
    are similar from Theorem 17.\\

    On the other hand, if the roots of the characteristic polynomial are
    not distinct, i.e.
    \[c_A(x)=c_B(x)=(x-\alpha)^2\]
    for some $\alpha\in F$, then $m_A(x)$ is either $x-\alpha$ or
    $(x-\alpha)^2$. If $m_A(x)=x-\alpha$, then the invariant factors of $A$
    are $x-\alpha$ and $x-\alpha$, which implies that the rational
    canonical form of $A$ is $\alpha I$. Now since $A$ is similar to its
    rational canonical form $\alpha I$,we have $A=Q^{-1}(\alpha I)Q=\alpha
    I$ for some invertible matrix $Q\in\text{Mat}_2(F)$, which makes $A$ a
    scalar matrix, a contradiction. Therefore we must have
    $m_A(x)=m_B(x)=(x-\alpha)^2$, and so $A$ and $B$ have the same rational
    canonical form, making them similar by Theorem 17.
  \end{proof}

\it \textbf{Section 12.2 Q4:} Prove that two $3\times3$ matrices are
  similar if and only if they have the same characteristic and same minimal
  polynomial. Give an explicit counterexample to this assertion for
  $4\times4$ matrices.

  \begin{proof}
    Assume $A,B\in\text{Mat}_3(F)$ are similar. Then from the argument in
    the previous question, the $A$ and $B$ have the same characteristic
    polynomial. Also, from Theorem 15, $A$ and $B$ have the same rational
    canonical form and thus also have the same minimal polynomial.\\

    For the converse, assume $A,B\in\text{Mat}_3(F)$ have the same
    characteristic polynomial $c(x)$ and minimal polynomial $m(x)$. If
    $c(x)$ has distinct roots, then $m(x)=c(x)$ also has distinct roots, so
    $A$ and $B$ has only one invariant factor $m(x)$, which implies that
    $A$ and $B$ have the same rational canonical form. Then $A$ and $B$
    would be similar from Theorem 17. \\

    On the other hand, if $c(x)$ has only two distinct roots, i.e.
    \[c_A(x)=c_B(x)=c(x) =(x-\alpha)(x-\beta)^2\]
    for some $\alpha\neq\beta\in F$, the minimal polynomial $m(x)$ of $A$
    and $B$ can either be $c(x)$ or $(x-\alpha)(x-\beta)$. If
    $m(x)=c(x)$, then the invariant factors of $A$ and of $B$ can only be
    $m(x)$, and then $A$ would be similar to $B$ from Theorem 17. If
    $m(x)=(x-\alpha)(x-\beta)$, then the invariant factors of both matrices
    can only be $(x-\beta)$ and $(x-\alpha)(x-\beta)$. Then again, $A$ and
    $B$ are similar from Theorem 17. \\

    Finally, if $c(x)$ has only one root, i.e.
    \[c_A(x)=c_B(x)=c(x) =(x-\alpha)^3\]
    for some $\alpha\in F$, the minimal polynomial $m(x)$ of $A$ and $B$
    can either be $x-\alpha$, $(x-\alpha)^2$, or $(x-\alpha)^3$. If it is
    the first case, then the invariant factors of $A$ and of $B$ can only
    be $x-\alpha$ three times, and then $A$ would be similar to $B$ from
    Theorem 17. If it is the second case, then the invariant factors can
    only be $x-\alpha$ and $(x-\alpha)^2$, and again $A$ would be similar
    to $B$. Finally, if it is the third case, then the invariant factors
    can only be $(x-\alpha)^3$, and then $A$ and $B$ would still be
    similar from Theorem 17. \\

    Thus two $3\times3$ matrices are similar if and only if they have the
    same characteristic and minimal polynomial. The assertion for
    $4\times4$ matrices $A$ and $B$ will fail if for both matrices, the
    characteristic polynomial is $(x-\alpha)^4$ and the minimal polynomial
    is $(x-\alpha)^2$. Then if $A$ has invariant factors $(x-\alpha)^2$ and
    $(x-\alpha)^2$ and $B$ has invariant factors $x-\alpha$, $x-\alpha$,
    and $(x-\alpha)^2$, by Theorem 17, their respective rational canonical
    forms $A'$ and $B'$ will not be similar even though they have the same
    characteristic and minimal polynomials.
  \end{proof}

\it \textbf{Section 12.2 Q6:} Prove that the constant term in the
  characteristic polynomial of the $n\times n$ matrix $A$ is
  $(-1)^n\text{det}A$ and that the coefficient of $x^{n-1}$ is the negative
  of the sum of the diagonal entries of $A$ (the sum of the diagonal
  entries of $A$ is called the trace of $A$). Prove that $\text{det}A$ is
  the product of the eigenvalues of $A$ and that the trace of $A$ is the
  sum of the eigenvalues of $A$.
\end{document}
