\documentclass{article}
\usepackage[left=3cm,right=3cm,top=3cm,bottom=3cm]{geometry}
\usepackage{amsmath,amssymb,amsthm,pgfplots,tikz}
\usepackage[inline]{enumitem}
\usepackage{color}
\setlength{\parindent}{0mm} %So that we do not indent on new paragraphs
\newcommand{\TODO}[1]{\textcolor{red}{TODO: #1}}

\begin{document}
\title{Graduate Algebra II: Homework 6}
\author{Li Ling Ko\\ lko@nd.edu}
\date{\today}
\maketitle

\it \textbf{Section 12.3 Q1:} Suppose the vector space $V$ is the direct
  sum of cyclic $F[x]$-modules whose annihilators are $(x+1)^2$,
  $(x-1)(x^2+1)^2$, $(x^4-1)$ and $(x+1)(x^2-1)$. Determine the invariant
  factors and elementary divisors of $V$.

  \begin{proof}
    There are three cases to consider. In the first case, $F$ has
    characteristic 2. Then $(x^2+1)$ splits in $F$ with repeated root 1,
    also 1 equals -1. In the second case, $F$ has characteristic different
    from 2, and $(x^2+1)$ splits in $F$ and with roots $\alpha$ and
    $-\alpha$. Note that since the characteristic is different from 2, the
    elements $1$, $-1$, $\alpha$ and $\-alpha$ must be distinct. In the
    third case, $F$ has characteristic different from 2, and $(x^2+1)$ does
    not split in $F$. Again, the elements $1$ and $-1$ are distinct since
    the characteristic is not 2. For each of these cases, we use the
    elementary divisor form of the fundamental theorem to write each of the
    four $F[x]$-modules. \\

    In the first case where $F$ has characteristic 2,
    \begin{align*}
      \frac{F[x]}{(x+1)^2} &\cong \frac{F[x]}{(x-1)^2},\\
      \frac{F[x]}{(x-1)(x^2+1)^2} &\cong \frac{F[x]}{(x-1)^5},\\
      \frac{F[x]}{x^4-1} &\cong \frac{F[x]}{(x-1)^4},\\
      \frac{F[x]}{(x+1)(x^2-1)} &\cong \frac{F[x]}{(x-1)^3}.\\
    \end{align*}

    Then the elementary divisors are
    \[(x-1)^2, (x-1)^3, (x-1)^4, (x-1)^5.\]
    So the invariant factors are
    \[(x-1)^2, (x-1)^3, (x-1)^4, (x-1)^5.\]

    In the second case where $F$ has characteristic different
    from 2 and $(x^2+1)$ splits in $F$ and with roots $\alpha$ and
    $-\alpha$,
    \begin{align*}
      \frac{F[x]}{(x+1)^2} &\cong \frac{F[x]}{(x+1)^2},\\
      \frac{F[x]}{(x-1)(x^2+1)^2} &\cong \frac{F[x]}{x-1} \oplus
        \frac{F[x]}{(x-\alpha)^2} \oplus \frac{F[x]}{(x+\alpha)^2},\\
      \frac{F[x]}{x^4-1} &\cong \frac{F[x]}{x-\alpha} \oplus
        \frac{F[x]}{x+\alpha} \oplus \frac{F[x]}{x-1} \oplus
        \frac{F[x]}{x+1},\\
      \frac{F[x]}{(x+1)(x^2-1)} &\cong \frac{F[x]}{x-1} \oplus
        \frac{F[x]}{(x+1)^2}.\\
    \end{align*}

    Then the elementary divisors are
    \begin{align*}
      (x+1), (x+1)^2, (x+1)^2,\\
      (x-1), (x-1), (x-1),\\
      (x+\alpha), (x+\alpha)^2,\\
      (x-\alpha), (x-\alpha)^2.\\
    \end{align*}
    So the invariant factors are
    \[(x+1)^2(x-1)(x+\alpha)^2(x-\alpha)^2,
    (x+1)^2(x-1)(x+\alpha)(x-\alpha), (x+1)(x-1).\]

    In the third case where $F$ has characteristic different
    from 2 and $(x^2+1)$ does not split in $F$,
    \begin{align*}
      \frac{F[x]}{(x+1)^2} &\cong \frac{F[x]}{(x+1)^2},\\
      \frac{F[x]}{(x-1)(x^2+1)^2} &\cong \frac{F[x]}{x-1} \oplus
        \frac{F[x]}{(x^2+1)^2},\\
      \frac{F[x]}{x^4-1} &\cong \frac{F[x]}{x^2+1} \oplus
        \frac{F[x]}{x-1} \oplus \frac{F[x]}{x+1},\\
      \frac{F[x]}{(x+1)(x^2-1)} &\cong \frac{F[x]}{x-1} \oplus
        \frac{F[x]}{(x+1)^2}.\\
    \end{align*}

    Then the elementary divisors are
    \begin{align*}
      (x+1), (x+1)^2, (x+1)^2,\\
      (x-1), (x-1), (x-1),\\
      (x^2+1), (x^2+1)^2.\\
    \end{align*}
    So the invariant factors are
    \[(x+1)^2(x-1)(x^2+1)^2, (x+1)^2(x-1)(x^2+1), (x+1)(x-1).\]
  \end{proof}

\it \textbf{Section 12.3 Q2:} Prove that if $\lambda_1,\ldots,\lambda_n$
  are the eigenvalues of the $n\times n$ matrix $A$ then
  $\lambda_1^k,\ldots,\lambda_n^k$ are the eigenvalues of $A^k$ for any
  $k\geq0$.

  \begin{proof}
    The $\lambda_i^k$ are eigenvalues of $A^k$ by induction on $k$, since
    $A^{k-1}\vec{v}=\lambda^{k-1}\vec{v}$ implies
    $A^{k}\vec{v}=\lambda^{k}\vec{v}$ when we multiply both sides by $A$.
    \\

    Now we show that the $\lambda_i^k$'s are the only possible eigenvalues
    of $A^k$. Let $A=P^{-1}JP$, where $J$ is the Jordan canonical form of
    $A$. So the diagonal values of $J$ are the eigenvalues
    $\lambda_1,\ldots,\lambda_n$. Then $A^k=P^{-1}J^kP$. We show that the
    eigenvalues of $J^k$ are $\lambda_1^k,\ldots,\lambda_n^k$. Now since
    $J$ is a diagonal matrix with diagonal entries
    $\lambda_1,\ldots,\lambda_n$, by induction on $k$, $J^k$ will be a
    diagonal matrix with diagonal entries $\lambda_1^k,\ldots,\lambda_n^k$.
    Then the characteristic polynomial of $J^k$ will be
    \[c_{J^k}(x) =|J^k-xI| =(\lambda_1^k-x) \cdots(\lambda_n^k-x),\]
    since the determinant of an upper-triangular matrix is the product of
    its diagonal entries. Thus the eigenvalues of $J^k$ which are the roots
    of $c_{J^k}(x)$ will be $\lambda_1^k,\ldots,\lambda_n^k$. \\

    So the Jordan canonical form of $J^k=Q^{-1}KQ$, where $K$ is the Jordan
    canonical form of $J^k$, and has diagonal entries
    $\lambda_1^k,\ldots,\lambda_n^k$. Then
    \[A^k =P^{-1}Q^{-1}KQP =(QP)^{-1}K(QP),\]
    which implies that $K$ is the Jordan canonical form of $A^k$, and so
    the eigenvalues of $A^k$ are the diagonal entries of $K$ which are
    $\lambda_1^k,\ldots,\lambda_n^k$.
  \end{proof}
\end{document}
