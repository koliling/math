\documentclass{article}
\usepackage[left=3cm,right=3cm,top=3cm,bottom=3cm]{geometry}
\usepackage{amsmath,amssymb,amsthm,pgfplots,tikz}
\usepackage[inline]{enumitem}
\usepackage{color}
\setlength{\parindent}{0mm} %So that we do not indent on new paragraphs
\newcommand{\TODO}[1]{\textcolor{red}{TODO: #1}}

\begin{document}
\title{Graduate Algebra II: Homework 6}
\author{Li Ling Ko\\ lko@nd.edu}
\date{\today}
\maketitle

\it \textbf{Section 12.3 Q1:} Suppose the vector space $V$ is the direct
  sum of cyclic $F[x]$-modules whose annihilators are $(x+1)^2$,
  $(x-1)(x^2+1)^2$, $(x^4-1)$ and $(x+1)(x^2-1)$. Determine the invariant
  factors and elementary divisors of $V$.

  \begin{proof}
    There are three cases to consider. In the first case, $F$ has
    characteristic 2. Then $(x^2+1)$ splits in $F$ with repeated root 1,
    also 1 equals -1. In the second case, $F$ has characteristic different
    from 2, and $(x^2+1)$ splits in $F$ and with roots $\alpha$ and
    $-\alpha$. Note that since the characteristic is different from 2, the
    elements $1$, $-1$, $\alpha$ and $\-alpha$ must be distinct. In the
    third case, $F$ has characteristic different from 2, and $(x^2+1)$ does
    not split in $F$. Again, the elements $1$ and $-1$ are distinct since
    the characteristic is not 2. For each of these cases, we use the
    elementary divisor form of the fundamental theorem to write each of the
    four $F[x]$-modules. \\

    In the first case where $F$ has characteristic 2,
    \begin{align*}
      \frac{F[x]}{(x+1)^2} &\cong \frac{F[x]}{(x-1)^2},\\
      \frac{F[x]}{(x-1)(x^2+1)^2} &\cong \frac{F[x]}{(x-1)^5},\\
      \frac{F[x]}{x^4-1} &\cong \frac{F[x]}{(x-1)^4},\\
      \frac{F[x]}{(x+1)(x^2-1)} &\cong \frac{F[x]}{(x-1)^3}.\\
    \end{align*}

    Then the elementary divisors are
    \[(x-1)^2, (x-1)^3, (x-1)^4, (x-1)^5.\]
    So the invariant factors are
    \[(x-1)^2, (x-1)^3, (x-1)^4, (x-1)^5.\]

    In the second case where $F$ has characteristic different
    from 2 and $(x^2+1)$ splits in $F$ and with roots $\alpha$ and
    $-\alpha$,
    \begin{align*}
      \frac{F[x]}{(x+1)^2} &\cong \frac{F[x]}{(x+1)^2},\\
      \frac{F[x]}{(x-1)(x^2+1)^2} &\cong \frac{F[x]}{x-1} \oplus
        \frac{F[x]}{(x-\alpha)^2} \oplus \frac{F[x]}{(x+\alpha)^2},\\
      \frac{F[x]}{x^4-1} &\cong \frac{F[x]}{x-\alpha} \oplus
        \frac{F[x]}{x+\alpha} \oplus \frac{F[x]}{x-1} \oplus
        \frac{F[x]}{x+1},\\
      \frac{F[x]}{(x+1)(x^2-1)} &\cong \frac{F[x]}{x-1} \oplus
        \frac{F[x]}{(x+1)^2}.\\
    \end{align*}

    Then the elementary divisors are
    \begin{align*}
      (x+1), (x+1)^2, (x+1)^2,\\
      (x-1), (x-1), (x-1),\\
      (x+\alpha), (x+\alpha)^2,\\
      (x-\alpha), (x-\alpha)^2.\\
    \end{align*}
    So the invariant factors are
    \[(x+1)^2(x-1)(x+\alpha)^2(x-\alpha)^2,
    (x+1)^2(x-1)(x+\alpha)(x-\alpha), (x+1)(x-1).\]

    In the third case where $F$ has characteristic different
    from 2 and $(x^2+1)$ does not split in $F$,
    \begin{align*}
      \frac{F[x]}{(x+1)^2} &\cong \frac{F[x]}{(x+1)^2},\\
      \frac{F[x]}{(x-1)(x^2+1)^2} &\cong \frac{F[x]}{x-1} \oplus
        \frac{F[x]}{(x^2+1)^2},\\
      \frac{F[x]}{x^4-1} &\cong \frac{F[x]}{x^2+1} \oplus
        \frac{F[x]}{x-1} \oplus \frac{F[x]}{x+1},\\
      \frac{F[x]}{(x+1)(x^2-1)} &\cong \frac{F[x]}{x-1} \oplus
        \frac{F[x]}{(x+1)^2}.\\
    \end{align*}

    Then the elementary divisors are
    \begin{align*}
      (x+1), (x+1)^2, (x+1)^2,\\
      (x-1), (x-1), (x-1),\\
      (x^2+1), (x^2+1)^2.\\
    \end{align*}
    So the invariant factors are
    \[(x+1)^2(x-1)(x^2+1)^2, (x+1)^2(x-1)(x^2+1), (x+1)(x-1).\]
  \end{proof}

\it \textbf{Section 12.3 Q2:} Prove that if $\lambda_1,\ldots,\lambda_n$
  are the eigenvalues of the $n\times n$ matrix $A$ then
  $\lambda_1^k,\ldots,\lambda_n^k$ are the eigenvalues of $A^k$ for any
  $k\geq0$.

  \begin{proof}
    The $\lambda_i^k$ are eigenvalues of $A^k$ by induction on $k$, since
    $A^{k-1}\vec{v}=\lambda^{k-1}\vec{v}$ implies
    $A^{k}\vec{v}=\lambda^{k}\vec{v}$ when we multiply both sides by $A$.
    \\

    Now we show that the $\lambda_i^k$'s are the only possible eigenvalues
    of $A^k$. Let $A=P^{-1}JP$, where $J$ is the Jordan canonical form of
    $A$. So the diagonal values of $J$ are the eigenvalues
    $\lambda_1,\ldots,\lambda_n$. Then $A^k=P^{-1}J^kP$. We show that the
    eigenvalues of $J^k$ are $\lambda_1^k,\ldots,\lambda_n^k$. Now since
    $J$ is a diagonal matrix with diagonal entries
    $\lambda_1,\ldots,\lambda_n$, by induction on $k$, $J^k$ will be a
    diagonal matrix with diagonal entries $\lambda_1^k,\ldots,\lambda_n^k$.
    Then the characteristic polynomial of $J^k$ will be
    \[c_{J^k}(x) =|J^k-xI| =(\lambda_1^k-x) \cdots(\lambda_n^k-x),\]
    since the determinant of an upper-triangular matrix is the product of
    its diagonal entries. Thus the eigenvalues of $J^k$ which are the roots
    of $c_{J^k}(x)$ will be $\lambda_1^k,\ldots,\lambda_n^k$. \\

    So the Jordan canonical form of $J^k=Q^{-1}KQ$, where $K$ is the Jordan
    canonical form of $J^k$, and has diagonal entries
    $\lambda_1^k,\ldots,\lambda_n^k$. Then
    \[A^k =P^{-1}Q^{-1}KQP =(QP)^{-1}K(QP),\]
    which implies that $K$ is the Jordan canonical form of $A^k$, and so
    the eigenvalues of $A^k$ are the diagonal entries of $K$ which are
    $\lambda_1^k,\ldots,\lambda_n^k$.
  \end{proof}

\it \textbf{Section 12.3 Q5:} Compute the Jordan canonical form for the
  matrix
  \[A =\begin{pmatrix}1&0&0\\ 0&0&-2\\ 0&1&3\\ \end{pmatrix}.\]

  \begin{proof}
    The characteristic polynomial of $A$ is
    \[c_A(x) =|A-xI| =(1-x)[(-x)(3-x)+2] =(1-x)^2(2-x).\]

    Therefore the eigenvalues are 1, 1, and 2. We check that $A$
    satisfies $(1-x)(2-x)$, so its minimal polynomial equals its
    characteristic polynomial. Then from Corollary 25, the Jordan Canonical
    form is
    \[\begin{pmatrix} 1&0&0\\ 0&1&0\\ 0&0&2\\ \end{pmatrix}.\]
  \end{proof}

\it \textbf{Section 12.3 Q7:} Determine the Jordan canonical form for the
  following matrices
  \[A =\begin{pmatrix} 5&4&1\\ -1&0&0\\ -3&-4&1\\ \end{pmatrix},\]
  \[B =\begin{pmatrix} 3&4&2\\ -2&-3&-1\\ -4&-4&-3\\ \end{pmatrix}.\]

  \begin{proof}
    We check that $c_A(x)=(2-x)^3$. Also, $\text{Rank}(A-2I)=2$, so the
    Jordan canonical form of $A$ is
    \[\begin{pmatrix} 2&1&0\\ 0&2&1\\ 0&0&2\\ \end{pmatrix}.\]

    We check that $c_B(x)=(1+x)^3$. Also, $\text{Rank}(B+I)=1$, so the
    Jordan canonical form of $B$ is
    \[\begin{pmatrix} -1&1&0\\ 0&-1&0\\ 0&0&-1\\ \end{pmatrix}.\]
  \end{proof}

\it \textbf{Section 12.3 Q10:} Find all Jordan canonical forms of
  $2\times2$, $3\times3$, and $4\times4$ matrices over $\mathbb{C}$.

  \begin{proof}
    For any distinct
    $\lambda_1,\lambda_2,\lambda_3,\lambda_4\in\mathbb{C}$, the possible
    Jordan canonical forms are

    \begin{align*}
      \begin{pmatrix} \lambda_1&0\\ 0&\lambda_2\\ \end{pmatrix},
        \begin{pmatrix} \lambda_1&0\\ 0&\lambda_1\\ \end{pmatrix},
        \begin{pmatrix} \lambda_1&1\\ 0&\lambda_1\\ \end{pmatrix},\\
      \begin{pmatrix} \lambda_1&0&0\\ 0&\lambda_2&0\\ 0&0&\lambda_3\\
        \end{pmatrix},
        \begin{pmatrix} \lambda_1&0&0\\ 0&\lambda_1&0\\ 0&0&\lambda_2\\
        \end{pmatrix},
        \begin{pmatrix} \lambda_1&1&0\\ 0&\lambda_1&0\\ 0&0&\lambda_2\\
        \end{pmatrix},
        \begin{pmatrix} \lambda_1&0&0\\ 0&\lambda_1&0\\ 0&0&\lambda_1\\
        \end{pmatrix},
        \begin{pmatrix} \lambda_1&1&0\\ 0&\lambda_1&0\\ 0&0&\lambda_1\\
        \end{pmatrix},
        \begin{pmatrix} \lambda_1&1&0\\ 0&\lambda_1&1\\ 0&0&\lambda_1\\
        \end{pmatrix},\\
      \begin{pmatrix} \lambda_1&0&0&0\\ 0&\lambda_2&0&0\\
        0&0&\lambda_3&0\\ 0&0&0&\lambda_4\\ \end{pmatrix},
        \begin{pmatrix} \lambda_1&0&0&0\\ 0&\lambda_1&0&0\\
          0&0&\lambda_2&0\\ 0&0&0&\lambda_3\\ \end{pmatrix},
        \begin{pmatrix} \lambda_1&1&0&0\\ 0&\lambda_1&0&0\\
          0&0&\lambda_2&0\\ 0&0&0&\lambda_3\\ \end{pmatrix},\\
        \begin{pmatrix} \lambda_1&0&0&0\\ 0&\lambda_1&0&0\\
          0&0&\lambda_2&0\\ 0&0&0&\lambda_2\\ \end{pmatrix},
        \begin{pmatrix} \lambda_1&1&0&0\\ 0&\lambda_1&0&0\\
          0&0&\lambda_2&0\\ 0&0&0&\lambda_2\\ \end{pmatrix},
        \begin{pmatrix} \lambda_1&1&0&0\\ 0&\lambda_1&0&0\\
          0&0&\lambda_2&1\\ 0&0&0&\lambda_2\\ \end{pmatrix},\\
        \begin{pmatrix} \lambda_1&0&0&0\\ 0&\lambda_1&0&0\\
          0&0&\lambda_1&0\\ 0&0&0&\lambda_1\\ \end{pmatrix},
        \begin{pmatrix} \lambda_1&1&0&0\\ 0&\lambda_1&0&0\\
          0&0&\lambda_1&0\\ 0&0&0&\lambda_1\\ \end{pmatrix},
        \begin{pmatrix} \lambda_1&1&0&0\\ 0&\lambda_1&1&0\\
          0&0&\lambda_1&0\\ 0&0&0&\lambda_1\\ \end{pmatrix},
        \begin{pmatrix} \lambda_1&1&0&0\\ 0&\lambda_1&1&0\\
          0&0&\lambda_1&1\\ 0&0&0&\lambda_1\\ \end{pmatrix}.\\
    \end{align*}
  \end{proof}

\it \textbf{Section 12.3 Q18:} Determine all possible Jordan canonical
  forms for a linear transformation with characteristic polynomial
  $(x-2)^3(x-3)^2$.

  \begin{proof}
    The possible Jordan canonical forms are
    \begin{align*}
      \begin{pmatrix} 2&0&0&0&0\\ 0&2&0&0&0\\ 0&0&2&0&0\\ 0&0&0&3&0\\
        0&0&0&0&3\\ \end{pmatrix},
      \begin{pmatrix} 2&1&0&0&0\\ 0&2&0&0&0\\ 0&0&2&0&0\\ 0&0&0&3&0\\
        0&0&0&0&3\\ \end{pmatrix},
      \begin{pmatrix} 2&1&0&0&0\\ 0&2&1&0&0\\ 0&0&2&0&0\\ 0&0&0&3&0\\
        0&0&0&0&3\\ \end{pmatrix},\\
      \begin{pmatrix} 2&0&0&0&0\\ 0&2&0&0&0\\ 0&0&2&0&0\\ 0&0&0&3&1\\
        0&0&0&0&3\\ \end{pmatrix},
      \begin{pmatrix} 2&1&0&0&0\\ 0&2&0&0&0\\ 0&0&2&0&0\\ 0&0&0&3&1\\
        0&0&0&0&3\\ \end{pmatrix},
      \begin{pmatrix} 2&1&0&0&0\\ 0&2&1&0&0\\ 0&0&2&0&0\\ 0&0&0&3&1\\
        0&0&0&0&3\\ \end{pmatrix}.
    \end{align*}
  \end{proof}

\it \textbf{Section 12.3 Q23:} Suppose $A$ is a $2\times2$ matrix with
  entries from $\mathbb{Q}$ for which $A^3=I$ but $A\neq I$. Write $A$ in
  rational canonical form and in Jordan canonical form viewed as a matrix
  over $\mathbb{C}$.

  \begin{proof}
    Matrix $A$ satisfies $x^3-1=(x-1)(x^2+x+1)$ but not $(x-1)$, therefore
    it must satisfy $f(x):=(x^2+x+1)$. Over $\mathbb{Q}$, the polynomial
    $f(x)$ is irreducible and has degree equal to 2, thus $f(x)$ must be the
    minimal polynomial of $A$. Thus the rational canonical form is
    \[\begin{pmatrix} 0&-1\\ 1&-1\\ \end{pmatrix}.\]

    Over $\mathbb{C}$, the eigenvalues are the distinct roots of $f(x)$
    which are $\omega$ and $\omega^2$, where $\omega$ is the 3rd root of
    unity. Since the eigenvalues are distinct, the Jordan canonical form
    is
    \[\begin{pmatrix} \omega&0\\ 0&\omega^2\\ \end{pmatrix}.\]
  \end{proof}

\it \textbf{Section 12.3 Q24:} Prove that there are no $3\times3$ matrices
  $A$ over $\mathbb{Q}$ with $A^8=I$ but $A^4\neq I$. 

  \begin{proof}
    If such a matrix exists, then it would satisfy $x^8-1=(x^4-1)(x^4+1)$
    but not $x^4-1$, therefore it would satisfy $f(x):=x^4+1$. Now
    $f(x)$ is irreducible over $\mathbb{Q}$ since it has no linear factors
    and factorizing into a product of two quadratic factors would give a
    contradiction when we calculate the coefficients. Yet the minimal
    polynomial of $A$ must have degree less than 4 and divide $f(x)$, a
    contradiction.
  \end{proof}

\it \textbf{Section 13.1 Q2:} Show that $x^3-2x-2$ is irreducible over
  $\mathbb{Q}$ and let $\theta$ be a root. Compute
  $(1+\theta)(1+\theta+\theta^2)$ and $\frac{1+\theta}{1+\theta+\theta^2}$
  in $\mathbb{Q}(\theta)$.

  \begin{proof}
    $f(x):=x^3-2x-2$ is irreducible over $\mathbb{Q}$ from Eisenstein's
    criteria with prime $2\in\mathbb{Q}$. \\

    From long division, we get
    \[(1+\theta)(1+\theta+\theta^2) \equiv 2\theta^2+4\theta+3
    \mod{f(\theta)}.\]
    Thus $(1+\theta)(1+\theta+\theta^2) =2\theta^2+4\theta+3$. \\

    To find the multiplicative inverse of $1+\theta+\theta^2$ in
    $\mathbb{Q}(\theta)$, we backtrack from Euclid's algorithm to find
    polynomials $g(x),h(x)\in\mathbb{Q}[x]$ such that
    \[g(x)(1+x+x^2) -h(x)f(x) =\text{gcd}(f(x),1+x+x^2)=1.\]

    Note that $\text{gcd}(f(x),1+x+x^2)=1$ because $f(x)$ is irreducible
    over $\mathbb{Q}$ and $1+x+x^2$ has degree less than $f(x)$. Then
    \[g(\theta)(1+\theta+\theta^2) \equiv 1 \mod{f(\theta)},\]
    so $g(\theta)$ will be the multiplicative inverse of
    $(1+\theta+\theta^2)$. Following Euclid's algorithm, we have
    \begin{align*}
      \theta^3-2\theta-2 &=(\theta-1)(\theta^2+\theta+1) -2\theta-1\\
      \theta^2+\theta+1 &=\left(-\frac{1}{2}\theta-\frac{1}{4}\right)
        (-2\theta-1) +\frac{3}{4}\\
    \end{align*}

    Working backwards, we get
    \begin{align*}
      \theta^3-2\theta-2 &=(\theta-1)(\theta^2+\theta+1)
        +\frac{(\theta^2+\theta+1)-\frac{3}{4}}
        {-\frac{1}{2}\theta-\frac{1}{4}}\\
      \left(-\frac{1}{2}\theta-\frac{1}{4}\right) (\theta^3-2\theta-2)
        &=(\theta-1) \left(-\frac{1}{2}\theta -\frac{1}{4}\right)
        (\theta^2+\theta+1) +(\theta^2+\theta+1)-\frac{3}{4}\\
      &=\left[(\theta-1) \left(-\frac{1}{2}\theta-\frac{1}{4}\right) +1
        \right] (\theta^2+\theta+1) -\frac{3}{4}\\
      \left(-\frac{1}{2}\theta-\frac{1}{4}\right) (\theta^3-2\theta-2)
        +\frac{3}{4}
        &=\left[(\theta-1) \left(-\frac{1}{2}\theta-\frac{1}{4}\right) +1
        \right] (\theta^2+\theta+1)\\
      \frac{4}{3} \left(-\frac{1}{2}\theta-\frac{1}{4}\right)
        (\theta^3-2\theta-2) +1
        &=\frac{4}{3} \left[(\theta-1)
        \left(-\frac{1}{2}\theta-\frac{1}{4}\right) +1 \right]
        (\theta^2+\theta+1)\\
    \end{align*}

    Thus the multiplicative inverse of $\theta^2+\theta+1$ is
    \[\frac{4}{3} \left[(\theta-1)
    \left(-\frac{1}{2}\theta-\frac{1}{4}\right) +1 \right]
    =-\frac{2}{3}\theta^2 +\frac{1}{3}\theta +\frac{5}{3} =-\frac{1}{3}
    \left(2\theta^2-\theta-5 \right).\]

    So multiplying with $1+\theta$ gives
    \begin{align*}
      \frac{1+\theta}{1+\theta+\theta^2} &=-\frac{1}{3} (\theta+1)
        \left(2\theta^2-\theta-5 \right) \\
      &=-\frac{1}{3} \left(2\theta^3+\theta^2-6\theta-5 \right)\\
      &=-\frac{1}{3} \left[2(\theta^3-2\theta-2) +\theta^2-2\theta-1
        \right]\\
      &=-\frac{1}{3} \left[2\cdot0 +\theta^2-2\theta-1 \right]\\
      &=-\frac{1}{3} \left(\theta^2-2\theta-1 \right).\\
    \end{align*}
  \end{proof}
\end{document}
