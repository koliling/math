\documentclass{article}
\usepackage[left=3cm,right=3cm,top=3cm,bottom=3cm]{geometry}
\usepackage{amsmath,amssymb,amsthm,pgfplots,tikz}
\usepackage[inline]{enumitem}
\usepackage{color}
\setlength{\parindent}{0mm} %So that we do not indent on new paragraphs
\newcommand{\TODO}[1]{\textcolor{red}{TODO: #1}}

\begin{document}
\title{Graduate Algebra II: Homework 7}
\author{Li Ling Ko\\ lko@nd.edu}
\date{\today}
\maketitle

\it \textbf{Section 13.2 Q3:} Determine the minimal polynomial over
  $\mathbb{Q}$ for the element $1+i$.

  \begin{proof}
    $1+i$ satisfies the polynomial $(x-1)^2=-1$. Rearranging gives
    $x^2-2x+2=0$, which is irreducible by Eisenstein's criteria with prime
    $2\in\mathbb{Z}$. Thus the minimal polynomial is $x^2-2x+2=0$.
  \end{proof}

\it \textbf{Section 13.2 Q4:} Determine the degree over $\mathbb{Q}$ of
  $2+\sqrt{3}$ and of $1+\sqrt[3]{2}+\sqrt[3]{4}$.

  \begin{proof}
    Observe that $\mathbb{Q}(2+\sqrt{3}) =\mathbb{Q}(\sqrt{3})$. Then since
    the degree of $\sqrt{3}$ over $\mathbb{Q}$ is 2, the degree of
    $\sqrt{3}$ over $\mathbb{Q}$ must also be 2. \\

    Next observe that since $\sqrt[3]{4}=(\sqrt[3]{2})^2$, therefore
    $\mathbb{Q}(1+\sqrt[3]{2}+\sqrt[3]{4})\subseteq
    \mathbb{Q}(\sqrt[3]{2})$. Now the degree of $\sqrt[3]{2}$ over
    $\mathbb{Q}$ is 3 since the minimal polynomial of $\sqrt[3]{2}$ over
    $\mathbb{Q}$ is $x^3-2$ which is irreducible from Eisenstein's criteria
    with prime $2\in\mathbb{Z}$. Thus the degree of
    $1+\sqrt[3]{2}+\sqrt[3]{4}$ over $\mathbb{Q}$ must divide 3, which
    means it can only be 1 or 3. But
    $1+\sqrt[3]{2}+\sqrt[3]{4}\not\in\mathbb{Q}$, so the degree can only be
    3.
  \end{proof}

\it \textbf{Section 13.2 Q6:} Prove directly from the definitions that the
  field $F(\alpha_1,\alpha_2,\ldots,\alpha_n)$ is the composite of the
  fields $F(\alpha_1),F(\alpha_2),\ldots F(\alpha_n)$.

  \begin{proof}
    By definition, $F(\alpha_1,\alpha_2,\ldots,\alpha_n)$ is the smallest
    field containing $F$ and $\alpha_1,\alpha_2,\ldots,\alpha_n$. Also by
    definition, $F(\alpha_1),F(\alpha_2),\ldots F(\alpha_n)$ is the
    smallest field containing $F(\alpha_1),F(\alpha_2),\ldots F(\alpha_n)$.
    Such a field must also contain each of $F(\alpha_i)$ since
    $F(\alpha_i)$ is the smallest field containing $F$ and $\alpha_i$ and
    $F(\alpha_1),F(\alpha_2),\ldots F(\alpha_n)$ contains $F$ and
    $\alpha_i$. For the reverse inclusion, $F(\alpha_1),F(\alpha_2),\ldots
    F(\alpha_n)$ clearly contains $F$ and each of
    $\alpha_1,\alpha_2,\ldots,\alpha_n$.
  \end{proof}

\it \textbf{Section 13.2 Q7:} Prove that $\mathbb{Q}(\sqrt{2}+\sqrt{3})
  =\mathbb{Q}(\sqrt{2},\sqrt{3})$. Conclude that
  $[\mathbb{Q}(\sqrt{2}+\sqrt{3}):\mathbb{Q}]=4$ Find an irreducible
  polynomial satisfied by $\sqrt{2}+\sqrt{3}$.

  \begin{proof}
    Clearly $\mathbb{Q}(\sqrt{2}+\sqrt{3})
    \subseteq\mathbb{Q}(\sqrt{2},\sqrt{3})$. For the reverse inclusion,
    observe that $\sqrt{2}-\sqrt{3} \in\mathbb{Q}(\sqrt{2}+\sqrt{3})$ since
    \[\sqrt{3}-\sqrt{2} =\frac{5}{\sqrt{2}+\sqrt{3}}.\]

    Therefore
    \[\sqrt{2} =\frac{1}{2} [(\sqrt{2}+\sqrt{3}) +(\sqrt{2}-\sqrt{3})]
    \in\mathbb{Q}(\sqrt{2}+\sqrt{3}),\]
    and similarly $\sqrt{3} \in\mathbb{Q}(\sqrt{2}+\sqrt{3})$. Thus
    $\mathbb{Q}(\sqrt{2}+\sqrt{3}) \supseteq\mathbb{Q}(\sqrt{2},\sqrt{3})$.
    \\

    Therefore
    \[[\mathbb{Q}(\sqrt{2}+\sqrt{3}):\mathbb{Q}]
    =[\mathbb{Q}(\sqrt{2},\sqrt{3}):\mathbb{Q}]
    =[\mathbb{Q}(\sqrt{2},\sqrt{3}):\mathbb{Q}(\sqrt{2})]
    [\mathbb{Q}(\sqrt{2}):\mathbb{Q}].\]

    Now $[\mathbb{Q}(\sqrt{2}):\mathbb{Q}]=2$ since the minimal polynomial
    of $\sqrt{2}$ over $\mathbb{Q}$ is $x^2-2$. Now
    $\sqrt{3}\not\in\mathbb{Q}(\sqrt{2})$ because otherwise $\sqrt{3}$ is
    spanned by $\{1,\sqrt{2}\}$ as a $\mathbb{Q}$ vector space. Thus
    $[\mathbb{Q}(\sqrt{2},\sqrt{3}):\mathbb{Q}(\sqrt{2})] >1$. Now
    $[\mathbb{Q}(\sqrt{2},\sqrt{3}):\mathbb{Q}(\sqrt{2})]
    \leq[\mathbb{Q}(\sqrt{3}):\mathbb{Q}]$, and
    $[\mathbb{Q}(\sqrt{3}):\mathbb{Q}]=2$, thus
    $[\mathbb{Q}(\sqrt{2},\sqrt{3}):\mathbb{Q}(\sqrt{2})]$ can only be 2.
    Hence $[\mathbb{Q}(\sqrt{2}+\sqrt{3}):\mathbb{Q}]=2\cdot2=4$.
  \end{proof}

\it \textbf{Section 13.2 Q10:} Determine the degree of the extension
  $\mathbb{Q}(\sqrt{3+2\sqrt{2}})$ over $\mathbb{Q}$.

  \begin{proof}
    Observe that $\sqrt{3+2\sqrt{2}}$ satisfies $(x^2-3)^2=8$, which
    rearranges to $x^4-6x^2+1=0$. This polynomial is irreducible over
    $\mathbb{Q}$ by Eisenstein's criteria using prime $2\in\mathbb{Z}$,
    thus it is the minimal polynomial of $\sqrt{3+2\sqrt{2}}$ over
    $\mathbb{Q}$. Hence the degree of the extension is 4.
  \end{proof}

\it \textbf{Section 13.2 Q11:}
  \begin{enumerate}[label={(\alph*)}]
    \item Let $\sqrt{3+4i}$ denote the square root of the complex number
      $3+4i$ that lies in the first quadrant and let $\sqrt{3-4i}$ denote
      the square root of $3-4i$ that lies in the fourth quadrant. Prove
      that $[\mathbb{Q}(\sqrt{3+4i}+\sqrt{3-4i}):\mathbb{Q}]=1$.

      \begin{proof}
        Denote $\alpha=\sqrt{3+4i}+\sqrt{3-4i}$. Then $\alpha$ satisfies
        $\alpha^2=6+2\sqrt{(3+4i)(3-4i)}=16$, where the last equality holds
        because of the choice of quadrant of the roots. Thus $\alpha=4$
        which lies in $\mathbb{Q}$. Therefore
        $[\mathbb{Q}(\sqrt{3+4i}+\sqrt{3-4i}):\mathbb{Q}]=1$.
      \end{proof}

    \item Determine the degree of the extension
      $\mathbb{Q}(\sqrt{1+\sqrt{-3}} +\sqrt{1-\sqrt{-3}})$ over
      $\mathbb{Q}$.

      \begin{proof}
        Denote $\alpha=\sqrt{1+\sqrt{-3}} +\sqrt{1-\sqrt{-3}}$. Then
        $\alpha$ satisfies
        \[x^2 =2\pm 2\sqrt{(1+\sqrt{-3})(1-\sqrt{-3})} =2\pm 4 =6\;
        \text{or}\; -2,\]
        depending on the choice of the quadrant of the roots
        $\sqrt{1+\sqrt{-3}}$ and $\sqrt{1-\sqrt{-3}}$. Rearranging gives
        either $x^2+2=0$ or $x^2-6=0$, both of which are irreducible in
        $\mathbb{Q}$ and are therefore minimal polynomials of $\alpha$
        depending on the choice of the quadrant of the roots. Thus the
        degree of the extension is 2.
      \end{proof}
  \end{enumerate}

\it \textbf{Section 13.2 Q14:} Prove that if $[F(\alpha):F]$ is odd then
  $F(\alpha)=F(\alpha^2)$.

  \begin{proof}
    Since $F\subseteq F(\alpha^2) \subseteq F(\alpha)$, we have
    \[[F(\alpha):F] =[F(\alpha):F(\alpha^2)] [F(\alpha^2):F].\]

    Now $[F(\alpha):F(\alpha^2)]\leq2$ since the minimal polynomial of
    $\alpha$ over $F(\alpha^2)$ is $x^2-\alpha^2$ if $\alpha\not\in
    F(\alpha^2)$. However if $[F(\alpha):F(\alpha^2)]=2$, then
    $[F(\alpha):F]$ will be even. Thus $[F(\alpha):F(\alpha^2)]=1$ and
    \[[F(\alpha):F] =[F(\alpha^2):F],\]
    which implies that $F(\alpha)=F(\alpha^2)$.
  \end{proof}

\it \textbf{Section 13.2 Q15:} A field $F$ is said to be formally real if
  $-1$ is not expressible as a sum of squares in $F$. Let $F$ be a formally
  real field, let $f(x)\in F[x]$ be an irreducible polynomial of odd degree
  and let $\alpha$ be a root of $f(x)$. Prove that $F(\alpha)$ is also
  formally real. [Pick $\alpha$ a counterexample of minimal degree. Show
  that $-1+f(x)g(x) =(p_1(x))^2 +\ldots +(p_m(x))^2$ for some
  $p_i(x),g(x)\in F[x]$ where $g(x)$ has odd degree $<\text{deg}\; f$. Show
  that some root $\beta$ of $g$ has odd degree over $F$ and $F(\beta)$ is
  not formally real, violating the minimality of $\alpha$.] 

  \begin{proof}
    Following the hint, assume $F$ is formally real and let $f(x)\in F[x]$
    be an irreducible polynomial of minimal odd odd degree with $\alpha$ as
    root and, where $F(\alpha)$ is not formally real. Then $-1$ must be
    expressible as a sum of squares in $F(\alpha)$. In other words, there
    must exist $p_i(\alpha)\in F(\alpha)$ such that
    \[-1 =(p_1(\alpha))^2 +\ldots +(p_m(\alpha))^2.\]

    Note that we can assume the degrees of $p_i$ are smaller than
    $\text{deg}\; f$, because otherwise we can take modulo $f(\alpha)$.
    Then the above equation is the same as having some $g(x)\in F[x]$ where
    \begin{equation}
      -1+f(x)g(x) =(p_1(x))^2 +\ldots +(p_m(x))^2.
      \label{eq:p}
    \end{equation}

    Now since the degree of each $p_i$ is less than the degree of $f$, the
    degree of $g$ must also be smaller than the degree of $f$. Furthermore,
    since the degree of each $(p_i(x))^2$ is even and the degree of $f$ is
    odd, the degree of $g$ must be odd. Pick any root $\beta$ of $g(x)$,
    and let $g_\beta(x)$ be the minimal polynomial of $\beta$. Then
    $g_\beta(x)$ divides $g(x)$, so $g_\beta(x)$ must
    have odd degree less than the degree of $f$. Now we can rewrite
    equation~\ref{eq:p} as
    \[-1+g_\beta(x)h(x) =(p_1(x))^2 +\ldots +(p_m(x))^2\]
    for some $h(x)\in F[x]$, which implies $F(\beta)$ is formally real with
    $-1$ expressed as a sum of squares
    \[-1 =(p_1(\beta))^2 +\ldots +(p_m(\beta))^2\]
    in $F(\beta)$, violating the minimality of $f(x)$.
  \end{proof}

\it \textbf{Section 13.2 Q16:} Let $K/F$ be an algebraic extension and let
  $R$ be a ring contained in $K$ and containing $F$. Show that $R$ is a
  subfield of $K$ containing $F$.

  \begin{proof}
    Since rings are closed under addition, subtraction, and multiplication,
    it suffices to show that $R$ is closed under multiplicative inverse.
    Now given $\alpha\in R$, since $K/F$ is an algebraic extension,
    $\alpha$ must be algebraic over $F$. So the minimal polynomial
    of $\alpha$ exists. Let $n$ be the degree of the minimal polynomial.
    Then $F(\alpha)$ is an $F$-subspace of $R$ with basis
    $\{1,\alpha,\alpha^2,\ldots,\alpha^{n-1}\}$. Thus $R$ contains the
    multiplicative inverse of $\alpha$.
  \end{proof}
\end{document}
