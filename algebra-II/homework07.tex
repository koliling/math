\documentclass{article}
\usepackage[left=3cm,right=3cm,top=3cm,bottom=3cm]{geometry}
\usepackage{amsmath,amssymb,amsthm,pgfplots,tikz}
\usepackage[inline]{enumitem}
\usepackage{color}
\setlength{\parindent}{0mm} %So that we do not indent on new paragraphs
\newcommand{\TODO}[1]{\textcolor{red}{TODO: #1}}

\begin{document}
\title{Graduate Algebra II: Homework 7}
\author{Li Ling Ko\\ lko@nd.edu}
\date{\today}
\maketitle

\it \textbf{Section 13.2 Q3:} Determine the minimal polynomial over
  $\mathbb{Q}$ for the element $1+i$.

  \begin{proof}
    $1+i$ satisfies the polynomial $(x-1)^2=-1$. Rearranging gives
    $x^2-2x+2=0$, which is irreducible by Eisenstein's criteria with prime
    $2\in\mathbb{Z}$. Thus the minimal polynomial is $x^2-2x+2=0$.
  \end{proof}

\it \textbf{Section 13.2 Q3:} Determine the degree over $\mathbb{Q}$ of
  $\alpha=2+\sqrt{3}$ and of $\beta=1+\sqrt[3]{2}+\sqrt[3]{4}$.

  \begin{proof}
    Observe that $\mathbb{Q}(2+\sqrt{3}) =\mathbb{Q}(\sqrt{3})$. Then since
    the degree of $\sqrt{3}$ over $\mathbb{Q}$ is 2, the degree of
    $\sqrt{3}$ over $\mathbb{Q}$ must also be 2. \\

    Next observe that since $\sqrt[3]{4}=(\sqrt[3]{2})^2$, therefore
    $\mathbb{Q}(1+\sqrt[3]{2}+\sqrt[3]{4})\subseteq
    \mathbb{Q}(\sqrt[3]{2})$. Now the degree of $\sqrt[3]{2}$ over
    $\mathbb{Q}$ is 3 since the minimal polynomial of $\sqrt[3]{2}$ over
    $\mathbb{Q}$ is $x^3-2$ which is irreducible from Eisenstein's criteria
    with prime $2\in\mathbb{Z}$. Thus the degree of
    $1+\sqrt[3]{2}+\sqrt[3]{4}$ over $\mathbb{Q}$ must divide 3, which
    means it can only be 1 or 3. But
    $1+\sqrt[3]{2}+\sqrt[3]{4}\not\in\mathbb{Q}$, so the degree can only be
    3.
  \end{proof}
\end{document}
