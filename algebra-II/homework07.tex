\documentclass{article}
\usepackage[left=3cm,right=3cm,top=3cm,bottom=3cm]{geometry}
\usepackage{amsmath,amssymb,amsthm,pgfplots,tikz}
\usepackage[inline]{enumitem}
\usepackage{color}
\setlength{\parindent}{0mm} %So that we do not indent on new paragraphs
\newcommand{\TODO}[1]{\textcolor{red}{TODO: #1}}

\begin{document}
\title{Graduate Algebra II: Homework 7}
\author{Li Ling Ko\\ lko@nd.edu}
\date{\today}
\maketitle

\it \textbf{Q1:} Let $K$ be an algebraic extension of $F$, and suppose $R$
  is a ring such that $F\subset R\subset K$. In problem 16 of Section 13.2
  assigned below, you prove that $R$ is a field. Show that if we remove the
  word ``algebraic'' above, then $R$ need not be a field.

  \begin{proof}
    Consider the field extension $\mathbb{Q}(t) \supset \mathbb{Q}$ defined
    by
    \[\mathbb{Q}(t):=\left\{\frac{f(t)}{g(t)}: f(t),g(t)\in\mathbb{Q}[t],
    g(t)\neq0\right\}.\]

    Then $\mathbb{Q}(t)$ is not an algebraic extension of $\mathbb{Q}$.
    Then the ring $\mathbb{Q}[t]$ lies between $\mathbb{Q}$ and
    $\mathbb{Q}(t)$, but it is not a field since the element $t$ does not
    have a multiplicative inverse.
  \end{proof}

\it \textbf{Q2:} Let $F(x_1,\ldots,x_n)$ be the field of rational functions
  in the formal variables $x_1,\ldots,x_n$. Show that every element of
  $F(x_1,\ldots,x_n)$ that is not in $F$ is transcendental over $F$.

  \begin{proof}
    Assume by contradiction that there is an element $\alpha\in
    F(x_1,\ldots,x_n)$ that is not in $F$ and that is algebraic over $F$.
    We can write
    \[\alpha =\frac{f(x_1,\ldots,x_n)} {g(x_1,\ldots,x_n)}\]
    where $f,g\in F[x_1,\ldots,x_n]$ and $g\neq0$.
    By rearranging the variables $x_1,\ldots,x_n$, if necessary, we can
    assume without loss of generality that $x_n$ appears in $f$ or $g$.
    Then rearranging to remove the denominator, we can write
    \begin{equation}
      x_n^mf_m(\alpha,x_1,\ldots,x_{n-1}) +\ldots
      +x_nf_1(\alpha,x_1,\ldots,x_{n-1}) +f_0(\alpha,x_1,\ldots,x_{n-1})
      =0,
      \label{eq:F}
    \end{equation}
    where $m\in\mathbb{N}$ and the $f_i$'s are polynomials over $F$ in
    $n$-variables. Thus $x_n$ is algebraic over
    $F(x_1,\ldots,x_{n-1},\alpha)$. \\

    Consider the field extension
    \[F(x_1,\ldots,x_{n-1}) \subset F(x_1,\ldots,x_{n-1},\alpha) \subset
    F(x_1,\ldots,x_{n-1},x_n).\]
    Now since $\alpha$ is algebraic over $F$, it must be algebraic over the
    larger field $F(x_1,\ldots,x_{n-1})$. Therefore
    $F(x_1,\ldots,x_{n-1},\alpha)/ F(x_1,\ldots,x_{n-1})$ is an algebraic
    extension. Also, the field extension
    \[F(x_1,\ldots,x_{n-1},x_n)/ F(x_1,\ldots,x_{n-1},\alpha)\]
    is algebraic from equation~(\ref{eq:F}).
    Therefore the field extension $F(x_1,\ldots,x_{n})/
    F(x_1,\ldots,x_{n-1})$ must be algebraic, a contradiction.
  \end{proof}

\it \textbf{Q3:} Let $K$ be an extension of $F$ (not necessarily
  algebraic). Suppose $a$ is is algebraic over $F(u)$ for some $u\in K$,
  and suppose $a$ is transcendental over $F$. Prove that $u$ must be
  algebraic over $F(a)$.

  \begin{proof}
    We first show that $u$ must be transcendental over $F$: Otherwise
    $F(u,a)/F(u)$ and $F(u)/F$ will both be algebraic extensions, which
    makes $F(u,a)/F$ an algebraic extension. Then $F(a)/F$ must also be
    algebraic, a contradiction. \\

    Now $a$ is algebraic over $F(u)$, so we can write
    \[a^n\frac{f_n(u)}{g_n(u)} +\ldots +a\frac{f_1(u)}{g_1(u)}
    +\frac{f_0(u)}{g_0(u)} =0\]
    for some $f_i,g_i\in F[x]$ and $n\in\mathbb{N}$. Then rearranging to
    remove denominators, we can write
    \[u^mh_m(a) +\ldots +uh_1(a) +h_0(a)=0\]
    for some $h_i\in F[x]$ and $m\in\mathbb{N}$. Thus $u$ is algebraic over
    $F(a)$.
  \end{proof}

\it \textbf{Q4:}
  \begin{enumerate}[label={(\alph*)}]
    \item Prove that if $K$ is a field extension of $F$ such that $u\in K$
      is transcendental over $F$, then $F(u)$ and $F(x)$ are isomorphic as
      fields, where $F(x)$ denotes the field of rational functions in the
      formal variable $x$.

      \begin{proof}
        Consider the map $\varphi:F(x)\rightarrow F(u)$ defined by
        \[\varphi \left(\frac{f(x)}{g(x)} \right) =\frac{f(u)}{g(u)},\]
        where $f,g\in F[x]$ and $g\neq0$. Chasing definitions, this map is
        a ring homomorphism. It is also surjective since every
        $\frac{f(u)}{g(u)}\in F(u)$ has inverse image $\frac{f(x)}{g(x)}\in
        F(x)$. Finally, the map is injective since maps of fields are
        always injective. Therefore $F(u)$ and $F(x)$ are isomorphic as
        fields. \\
      \end{proof}

    \item Give an example of a field extension $K$ over $F$ such that
      $u,v\in K$ are transcendental over $F$, but $F(u,v)$ is not
      isomorphic to $F(x_1,x_2)$, where $x_1$ and $x_2$ are formal
      variables.

      \begin{proof}
        Consider the case where $v=u$. Then $F(u,v)=F(u,u)=F(u)$, which is
        isomorphic to $F(x_1)$, but is not isomorphic to $F(x_1,x_2)$.
      \end{proof}
  \end{enumerate}

\it \textbf{Section 13.2 Q3:} Determine the minimal polynomial over
  $\mathbb{Q}$ for the element $1+i$.

  \begin{proof}
    $1+i$ satisfies the polynomial $(x-1)^2=-1$. Rearranging gives
    $x^2-2x+2=0$, which is irreducible by Eisenstein's criteria with prime
    $2\in\mathbb{Z}$. Thus the minimal polynomial is $x^2-2x+2=0$.
  \end{proof}

\it \textbf{Section 13.2 Q4:} Determine the degree over $\mathbb{Q}$ of
  $2+\sqrt{3}$ and of $1+\sqrt[3]{2}+\sqrt[3]{4}$.

  \begin{proof}
    Observe that $\mathbb{Q}(2+\sqrt{3}) =\mathbb{Q}(\sqrt{3})$. Then since
    the degree of $\sqrt{3}$ over $\mathbb{Q}$ is 2, the degree of
    $\sqrt{3}$ over $\mathbb{Q}$ must also be 2. \\

    Next observe that since $\sqrt[3]{4}=(\sqrt[3]{2})^2$, therefore
    $\mathbb{Q}(1+\sqrt[3]{2}+\sqrt[3]{4})\subseteq
    \mathbb{Q}(\sqrt[3]{2})$. Now the degree of $\sqrt[3]{2}$ over
    $\mathbb{Q}$ is 3 since the minimal polynomial of $\sqrt[3]{2}$ over
    $\mathbb{Q}$ is $x^3-2$ which is irreducible from Eisenstein's criteria
    with prime $2\in\mathbb{Z}$. Thus the degree of
    $1+\sqrt[3]{2}+\sqrt[3]{4}$ over $\mathbb{Q}$ must divide 3, which
    means it can only be 1 or 3. But
    $1+\sqrt[3]{2}+\sqrt[3]{4}\not\in\mathbb{Q}$, so the degree can only be
    3.
  \end{proof}

\it \textbf{Section 13.2 Q6:} Prove directly from the definitions that the
  field $F(\alpha_1,\alpha_2,\ldots,\alpha_n)$ is the composite of the
  fields $F(\alpha_1),F(\alpha_2),\ldots F(\alpha_n)$.

  \begin{proof}
    By definition, $F(\alpha_1,\alpha_2,\ldots,\alpha_n)$ is the smallest
    field containing $F$ and $\alpha_1,\alpha_2,\ldots,\alpha_n$. Also by
    definition, $F(\alpha_1),F(\alpha_2),\ldots F(\alpha_n)$ is the
    smallest field containing $F(\alpha_1),F(\alpha_2),\ldots F(\alpha_n)$.
    Such a field must also contain each of $F(\alpha_i)$ since
    $F(\alpha_i)$ is the smallest field containing $F$ and $\alpha_i$ and
    $F(\alpha_1),F(\alpha_2),\ldots F(\alpha_n)$ contains $F$ and
    $\alpha_i$. For the reverse inclusion, $F(\alpha_1),F(\alpha_2),\ldots
    F(\alpha_n)$ clearly contains $F$ and each of
    $\alpha_1,\alpha_2,\ldots,\alpha_n$.
  \end{proof}

\it \textbf{Section 13.2 Q7:} Prove that $\mathbb{Q}(\sqrt{2}+\sqrt{3})
  =\mathbb{Q}(\sqrt{2},\sqrt{3})$. Conclude that
  $[\mathbb{Q}(\sqrt{2}+\sqrt{3}):\mathbb{Q}]=4$ Find an irreducible
  polynomial satisfied by $\sqrt{2}+\sqrt{3}$.

  \begin{proof}
    Clearly $\mathbb{Q}(\sqrt{2}+\sqrt{3})
    \subseteq\mathbb{Q}(\sqrt{2},\sqrt{3})$. For the reverse inclusion,
    observe that $\sqrt{2}-\sqrt{3} \in\mathbb{Q}(\sqrt{2}+\sqrt{3})$ since
    \[\sqrt{3}-\sqrt{2} =\frac{5}{\sqrt{2}+\sqrt{3}}.\]

    Therefore
    \[\sqrt{2} =\frac{1}{2} [(\sqrt{2}+\sqrt{3}) +(\sqrt{2}-\sqrt{3})]
    \in\mathbb{Q}(\sqrt{2}+\sqrt{3}),\]
    and similarly $\sqrt{3} \in\mathbb{Q}(\sqrt{2}+\sqrt{3})$. Thus
    $\mathbb{Q}(\sqrt{2}+\sqrt{3}) \supseteq\mathbb{Q}(\sqrt{2},\sqrt{3})$.
    \\

    Therefore
    \[[\mathbb{Q}(\sqrt{2}+\sqrt{3}):\mathbb{Q}]
    =[\mathbb{Q}(\sqrt{2},\sqrt{3}):\mathbb{Q}]
    =[\mathbb{Q}(\sqrt{2},\sqrt{3}):\mathbb{Q}(\sqrt{2})]
    [\mathbb{Q}(\sqrt{2}):\mathbb{Q}].\]

    Now $[\mathbb{Q}(\sqrt{2}):\mathbb{Q}]=2$ since the minimal polynomial
    of $\sqrt{2}$ over $\mathbb{Q}$ is $x^2-2$. Now
    $\sqrt{3}\not\in\mathbb{Q}(\sqrt{2})$ because otherwise $\sqrt{3}$ is
    spanned by $\{1,\sqrt{2}\}$ as a $\mathbb{Q}$ vector space. Thus
    $[\mathbb{Q}(\sqrt{2},\sqrt{3}):\mathbb{Q}(\sqrt{2})] >1$. Now
    $[\mathbb{Q}(\sqrt{2},\sqrt{3}):\mathbb{Q}(\sqrt{2})]
    \leq[\mathbb{Q}(\sqrt{3}):\mathbb{Q}]$, and
    $[\mathbb{Q}(\sqrt{3}):\mathbb{Q}]=2$, thus
    $[\mathbb{Q}(\sqrt{2},\sqrt{3}):\mathbb{Q}(\sqrt{2})]$ can only be 2.
    Hence $[\mathbb{Q}(\sqrt{2}+\sqrt{3}):\mathbb{Q}]=2\cdot2=4$.
  \end{proof}

\it \textbf{Section 13.2 Q10:} Determine the degree of the extension
  $\mathbb{Q}(\sqrt{3+2\sqrt{2}})$ over $\mathbb{Q}$.

  \begin{proof}
    Observe that $\sqrt{3+2\sqrt{2}}$ satisfies $(x^2-3)^2=8$, which
    rearranges to $x^4-6x^2+1=0$. This polynomial is irreducible over
    $\mathbb{Q}$ by Eisenstein's criteria using prime $2\in\mathbb{Z}$,
    thus it is the minimal polynomial of $\sqrt{3+2\sqrt{2}}$ over
    $\mathbb{Q}$. Hence the degree of the extension is 4.
  \end{proof}

\it \textbf{Section 13.2 Q11:}
  \begin{enumerate}[label={(\alph*)}]
    \item Let $\sqrt{3+4i}$ denote the square root of the complex number
      $3+4i$ that lies in the first quadrant and let $\sqrt{3-4i}$ denote
      the square root of $3-4i$ that lies in the fourth quadrant. Prove
      that $[\mathbb{Q}(\sqrt{3+4i}+\sqrt{3-4i}):\mathbb{Q}]=1$.

      \begin{proof}
        Denote $\alpha=\sqrt{3+4i}+\sqrt{3-4i}$. Then $\alpha$ satisfies
        $\alpha^2=6+2\sqrt{(3+4i)(3-4i)}=16$, where the last equality holds
        because of the choice of quadrant of the roots. Thus $\alpha=4$
        which lies in $\mathbb{Q}$. Therefore
        $[\mathbb{Q}(\sqrt{3+4i}+\sqrt{3-4i}):\mathbb{Q}]=1$.
      \end{proof}

    \item Determine the degree of the extension
      $\mathbb{Q}(\sqrt{1+\sqrt{-3}} +\sqrt{1-\sqrt{-3}})$ over
      $\mathbb{Q}$.

      \begin{proof}
        Denote $\alpha=\sqrt{1+\sqrt{-3}} +\sqrt{1-\sqrt{-3}}$. Then
        $\alpha$ satisfies
        \[x^2 =2\pm 2\sqrt{(1+\sqrt{-3})(1-\sqrt{-3})} =2\pm 4 =6\;
        \text{or}\; -2,\]
        depending on the choice of the quadrant of the roots
        $\sqrt{1+\sqrt{-3}}$ and $\sqrt{1-\sqrt{-3}}$. Rearranging gives
        either $x^2+2=0$ or $x^2-6=0$, both of which are irreducible in
        $\mathbb{Q}$ and are therefore minimal polynomials of $\alpha$
        depending on the choice of the quadrant of the roots. Thus the
        degree of the extension is 2.
      \end{proof}
  \end{enumerate}

\it \textbf{Section 13.2 Q14:} Prove that if $[F(\alpha):F]$ is odd then
  $F(\alpha)=F(\alpha^2)$.

  \begin{proof}
    Since $F\subseteq F(\alpha^2) \subseteq F(\alpha)$, we have
    \[[F(\alpha):F] =[F(\alpha):F(\alpha^2)] [F(\alpha^2):F].\]

    Now $[F(\alpha):F(\alpha^2)]\leq2$ since the minimal polynomial of
    $\alpha$ over $F(\alpha^2)$ is $x^2-\alpha^2$ if $\alpha\not\in
    F(\alpha^2)$. However if $[F(\alpha):F(\alpha^2)]=2$, then
    $[F(\alpha):F]$ will be even. Thus $[F(\alpha):F(\alpha^2)]=1$ and
    \[[F(\alpha):F] =[F(\alpha^2):F],\]
    which implies that $F(\alpha)=F(\alpha^2)$.
  \end{proof}

\it \textbf{Section 13.2 Q15:} A field $F$ is said to be formally real if
  $-1$ is not expressible as a sum of squares in $F$. Let $F$ be a formally
  real field, let $f(x)\in F[x]$ be an irreducible polynomial of odd degree
  and let $\alpha$ be a root of $f(x)$. Prove that $F(\alpha)$ is also
  formally real. [Pick $\alpha$ a counterexample of minimal degree. Show
  that $-1+f(x)g(x) =(p_1(x))^2 +\ldots +(p_m(x))^2$ for some
  $p_i(x),g(x)\in F[x]$ where $g(x)$ has odd degree $<\text{deg}\; f$. Show
  that some root $\beta$ of $g$ has odd degree over $F$ and $F(\beta)$ is
  not formally real, violating the minimality of $\alpha$.] 

  \begin{proof}
    Following the hint, assume $F$ is formally real and let $f(x)\in F[x]$
    be an irreducible polynomial of minimal odd odd degree with $\alpha$ as
    root and, where $F(\alpha)$ is not formally real. Then $-1$ must be
    expressible as a sum of squares in $F(\alpha)$. In other words, there
    must exist $p_i(\alpha)\in F(\alpha)$ such that
    \[-1 =(p_1(\alpha))^2 +\ldots +(p_m(\alpha))^2.\]

    Note that we can assume the degrees of $p_i$ are smaller than
    $\text{deg}\; f$, because otherwise we can take modulo $f(\alpha)$.
    Then the above equation is the same as having some $g(x)\in F[x]$ where
    \begin{equation}
      -1+f(x)g(x) =(p_1(x))^2 +\ldots +(p_m(x))^2.
      \label{eq:p}
    \end{equation}

    Now since the degree of each $p_i$ is less than the degree of $f$, the
    degree of $g$ must also be smaller than the degree of $f$. Furthermore,
    since the degree of each $(p_i(x))^2$ is even and the degree of $f$ is
    odd, the degree of $g$ must be odd. Pick any root $\beta$ of $g(x)$,
    and let $g_\beta(x)$ be the minimal polynomial of $\beta$. Then
    $g_\beta(x)$ divides $g(x)$, so $g_\beta(x)$ must
    have odd degree less than the degree of $f$. Now we can rewrite
    equation~\ref{eq:p} as
    \[-1+g_\beta(x)h(x) =(p_1(x))^2 +\ldots +(p_m(x))^2\]
    for some $h(x)\in F[x]$, which implies $F(\beta)$ is formally real with
    $-1$ expressed as a sum of squares
    \[-1 =(p_1(\beta))^2 +\ldots +(p_m(\beta))^2\]
    in $F(\beta)$, violating the minimality of $f(x)$.
  \end{proof}

\it \textbf{Section 13.2 Q16:} Let $K/F$ be an algebraic extension and let
  $R$ be a ring contained in $K$ and containing $F$. Show that $R$ is a
  subfield of $K$ containing $F$.

  \begin{proof}
    Since rings are closed under addition, subtraction, and multiplication,
    it suffices to show that $R$ is closed under multiplicative inverse.
    Now given $\alpha\in R$, since $K/F$ is an algebraic extension,
    $\alpha$ must be algebraic over $F$. So the minimal polynomial
    of $\alpha$ exists. Let $n$ be the degree of the minimal polynomial.
    Then $F(\alpha)$ is an $F$-subspace of $R$ with basis
    $\{1,\alpha,\alpha^2,\ldots,\alpha^{n-1}\}$. Thus $R$ contains the
    multiplicative inverse of $\alpha$.
  \end{proof}

\it \textbf{Section 13.2 Q19:} Let $K$ be an extension of $F$ of degree
  $n$.

  \begin{enumerate}[label={(\alph*)}]
    \item For any $\alpha\in K$ prove that $\alpha$ acting by left
      multiplication on $K$ is an $F$-linear transformation of $K$.

      \begin{proof}
        Let $\varphi_\alpha:K\rightarrow K$ denote the action by left
        multiplication. Then given $f\in F$ and $\beta_1,\beta_2\in K$,
        \begin{align*}
          \varphi_\alpha(f\beta_1+\beta_2) &=\alpha(f\beta_1+\beta_2)\\
          &=\alpha f\beta_1 +\alpha \beta_2\\
          &=f \alpha \beta_1 +\alpha \beta_2\\
          &=f \varphi_\alpha(\beta_1) +\varphi_\alpha(\beta_2).\\
        \end{align*}
        Thus $\varphi_\alpha$ is an $F$-linear transformation of $K$.
      \end{proof}

    \item Prove that $K$ is isomorphic to a subfield of the ring of
      $n\times n$ matrices over $F$, so the ring of $n\times n$ matrices
      over $F$ contains an isomorphic copy of every extension of $F$ of
      degree $\leq n$.

      \begin{proof}
        Fix a basis of $K$ over $F$. This basis should contain $n$ elements
        since $[K:F]=n$. Then consider the map $\varphi:
        K\rightarrow\text{Mat}_n(F)$ which takes $\alpha\in K$ and returns
        the matrix of $\varphi_\alpha$ with respect to the chosen basis,
        where $\varphi_\alpha$ has been defined in the previous part of
        this question. Then since $\varphi_\alpha$ is an $F$-linear map as
        shown earlier, $\varphi$ will be a ring homomorphism. Furthermore,
        if $\alpha\in\ker(\varphi)$, then $\varphi_\alpha(\beta)=0$ for all
        $\beta\in K$, which implies that $\alpha=0$. Thus $\varphi$ is
        injective.
      \end{proof}
  \end{enumerate}

\it \textbf{Section 13.2 Q20:} Show that if the matrix of the linear
  transformation ``multiplication by $\alpha$'' considered in the previous
  exercise is $A$ then $\alpha$ is a root of the characteristic polynomial
  for $A$. This gives an effective procedure for determining an equation of
  degree $n$ satisfied by an element $\alpha$ in an extension of $F$ of
  degree $n$. Use this procedure to obtain the monic polynomial of degree 3
  satisfied by $\sqrt[3]{2}$ and by $1+\sqrt[3]{2}+\sqrt[3]{4}$.

  \begin{proof}
    Denote the minimal polynomial of $\alpha$ over $F$ by $m_\alpha(x)$.
    Note that the minimal polynomial exists because the extension has
    finite degree. Consider the linear transformation ``multiplication by
    $m_\alpha(\alpha)$''. By $F$-linearity, the matrix of this
    transformation is $m_\alpha(A)$, which is 0 since $m_\alpha(\alpha)=0$.
    Thus $A$ satisfies $m_\alpha(x)$ which is irreducible, implying that
    the minimal polynomial of $A$ is also $m_\alpha(x)$, and so the
    $m_\alpha(x)$ divides the characteristic polynomial of $A$ and $\alpha$
    is a root of this characteristic polynomial. \\

    Consider the basis $\{1,\sqrt[3]{2},\sqrt[3]{4}\}$. Then for
    $\alpha=\sqrt[3]{2}$, the matrix $A$ is
    \begin{align*}
      A =
      \begin{pmatrix}
        0&1&0\\
        0&0&1\\
        2&0&0\\
      \end{pmatrix},
    \end{align*}
    which has characteristic polynomial $c_A(x)=|A-xI|=-x^3+2$. Thus a
    monic polynomial of degree 3 satisfied by $\sqrt[3]{2}$ is $x^3-2$. \\

    Similarly, for $\alpha=1+\sqrt[3]{2}+\sqrt[3]{4}$, the matrix $A$ is
    \begin{align*}
      A =
      \begin{pmatrix}
        1&1&1\\
        2&1&1\\
        2&2&1\\
      \end{pmatrix},
    \end{align*}
    which has characteristic polynomial $c_A(x)=|A-xI|=-(x^3-3x^2-3x-1)$.
    Thus a monic polynomial of degree 3 satisfied by
    $1+\sqrt[3]{2}+\sqrt[3]{4}$ is $x^3-3x^2-3x-1$. \\
  \end{proof}

\it \textbf{Section 13.2 Q22:} Let $K_1$ and $K_2$ be two finite extensions
  of a field $F$ contained in the field $K$. Prove that the $F$-algebra
  $K_1\otimes_F F_2$ is a field if and only if $[K_1K_2:F]=[K_1:F][K_2:F]$.

  \begin{proof}
  \end{proof}
\end{document}
