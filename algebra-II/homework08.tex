\documentclass{article}
\usepackage[left=3cm,right=3cm,top=3cm,bottom=3cm]{geometry}
\usepackage{amsmath,amssymb,amsthm,pgfplots,tikz}
\usepackage[inline]{enumitem}
\usepackage{color}
\setlength{\parindent}{0mm} %So that we do not indent on new paragraphs
\newcommand{\TODO}[1]{\textcolor{red}{TODO: #1}}

\begin{document}
\title{Graduate Algebra II: Homework 8}
\author{Li Ling Ko\\ lko@nd.edu}
\date{\today}
\maketitle

\it \textbf{DF 13.3.1:} Prove that it is impossible to construct a regular
  9-gon.

  \begin{proof}
    If we can construct a regular 9-gon, then by marking the mid-point of
    the edges of the 9-gon and drawing a line through the mid-point and the
    opposite corner of the 9-gon, we can find identify the center of the
    9-gon with compass and ruler (it is known that we can use a compass and
    ruler to identify the mid-point of any given line). By drawing a line
    from the each of the 9 corners of the 9-gon to the center of the
    9-gon, we obtain partition the 9-gon into 9 identical isosceles
    triangles that meet at the center of the 9-gon. Now the angle of each
    triangle at the corner that touches the center of the 9-gon is
    $2\pi/9$. We can half the angle to construct an angle of degree
    $\pi/9$, which is not constructible, and thus a contradiction.
  \end{proof}

\it \textbf{DF 13.3.4:} The construction of the regular 7-gon amounts to
  the constructibility of $\cos(2\pi/7)$. We shall see later that
  $\alpha=2\cos(2\pi/7)$ satisfies the equation $x^3+x^2-2x-1=0$. Use this
  to prove that the regular 7-gon is not constructible by straightedge and
  compass.

  \begin{proof}
    The given polynomial has no roots in $\mathbb{Z}$ since neither 1 nor
    -1 is a root. Therefore from Gauss's lemma, the polynomial has no roots
    in $\mathbb{Q}$. So since the polynomial has degree three and has no
    linear roots, it is irreducible over $\mathbb{Q}$. Therefore
    $[\mathbb{Q}(\alpha):\mathbb{Q}]=3$. Now if the regular 7-gon is
    constructible, then $\alpha$ is constructible, yet
    $[\mathbb{Q}(\alpha):\mathbb{Q}]$ is not a power of 2, a contradiction.
  \end{proof}

\it \textbf{DF 13.3.5:} Use the fact that $\alpha=2\cos(2\pi/5)$ satisfies
  the equation $x^2+x-1=0$ to conclude that the regular 5-gon is
  constructible by straightedge and compass.

  \begin{proof}
    Since $\alpha$ is a real root of a quadratic polynomial over $\mathbb{Q}$,
    it can be expressed using a series of addition, multiplication,
    division, and/or square roots of rationals. Since these operations are
    constructible by a straightedge and compass, $\alpha$ is constructible.
    Then by bisecting angles (which is also known to be constructible), we
    can construct $\alpha/2=\cos(2\pi/5)$. \\

    With a line of length $\cos(2\pi/5)$ constructed, we construct the
    degree $2\pi/5$ as follows: draw a perpendicular line from one end of
    the line upwards (perpendicular lines from a given line are known to be
    constructible). Then setting the compass to a length of 1, mark the
    point where the other end of the line of length $\cos(2\pi/5)$ meets
    the perpendicular line. Then draw a line from this end of the line to
    the marked point. The angle at the other end of the line will have
    degree $2\pi/5$. \\

    Repeating this process 5 times we can construct the symmetrical sides
    of the 5 isosceles triangles that when put together, form a 5-gon.
  \end{proof}

\end{document}
