\documentclass{article}
\usepackage[left=3cm,right=3cm,top=3cm,bottom=3cm]{geometry}
\usepackage{amsmath,amssymb,amsthm,pgfplots,tikz}
\usepackage{tikz-cd}
\usepackage[inline]{enumitem}
\usepackage{color}
\setlength{\parindent}{0mm} %So that we do not indent on new paragraphs
\newcommand{\TODO}[1]{\textcolor{red}{TODO: #1}}

\begin{document}
\title{Graduate Algebra II: Homework 8}
\author{Li Ling Ko\\ lko@nd.edu}
\date{\today}
\maketitle

\it \textbf{DF 13.3.1:} Prove that it is impossible to construct a regular
  9-gon.

  \begin{proof}
    If we can construct a regular 9-gon, then by marking the mid-point of
    the edges of the 9-gon and drawing a line through the mid-point and the
    opposite corner of the 9-gon, we can find identify the center of the
    9-gon with compass and ruler (it is known that we can use a compass and
    ruler to identify the mid-point of any given line). By drawing a line
    from the each of the 9 corners of the 9-gon to the center of the
    9-gon, we obtain partition the 9-gon into 9 identical isosceles
    triangles that meet at the center of the 9-gon. Now the angle of each
    triangle at the corner that touches the center of the 9-gon is
    $2\pi/9$. We can half the angle to construct an angle of degree
    $\pi/9$, which is not constructible, and thus a contradiction.
  \end{proof}

\it \textbf{DF 13.3.4:} The construction of the regular 7-gon amounts to
  the constructibility of $\cos(2\pi/7)$. We shall see later that
  $\alpha=2\cos(2\pi/7)$ satisfies the equation $x^3+x^2-2x-1=0$. Use this
  to prove that the regular 7-gon is not constructible by straightedge and
  compass.

  \begin{proof}
    The given polynomial has no roots in $\mathbb{Z}$ since neither 1 nor
    -1 is a root. Therefore from Gauss's lemma, the polynomial has no roots
    in $\mathbb{Q}$. So since the polynomial has degree three and has no
    linear roots, it is irreducible over $\mathbb{Q}$. Therefore
    $[\mathbb{Q}(\alpha):\mathbb{Q}]=3$. Now if the regular 7-gon is
    constructible, then $\alpha$ is constructible, yet
    $[\mathbb{Q}(\alpha):\mathbb{Q}]$ is not a power of 2, a contradiction.
  \end{proof}

\it \textbf{DF 13.3.5:} Use the fact that $\alpha=2\cos(2\pi/5)$ satisfies
  the equation $x^2+x-1=0$ to conclude that the regular 5-gon is
  constructible by straightedge and compass.

  \begin{proof}
    Since $\alpha$ is a real root of a quadratic polynomial over $\mathbb{Q}$,
    it can be expressed using a series of addition, multiplication,
    division, and/or square roots of rationals. Since these operations are
    constructible by a straightedge and compass, $\alpha$ is constructible.
    Then by bisecting angles (which is also known to be constructible), we
    can construct $\alpha/2=\cos(2\pi/5)$. \\

    With a line of length $\cos(2\pi/5)$ constructed, we construct the
    degree $2\pi/5$ as follows: draw a perpendicular line from one end of
    the line upwards (perpendicular lines from a given line are known to be
    constructible). Then setting the compass to a length of 1, mark the
    point where the other end of the line of length $\cos(2\pi/5)$ meets
    the perpendicular line. Then draw a line from this end of the line to
    the marked point. The angle at the other end of the line will have
    degree $2\pi/5$. \\

    Repeating this process 5 times we can construct the symmetrical sides
    of the 5 isosceles triangles that when put together, form a 5-gon.
  \end{proof}

\it \textbf{DF 13.4.1:} Determine the splitting field and its degree over
  $\mathbb{Q}$ for $x^4-2$.

  \begin{proof}
    This polynomial has four distinct roots $\{\sqrt[4]{2}, -\sqrt[4]{2},
    \sqrt[4]{2}i, -\sqrt[4]{2}i\}$. These roots lie in
    $\mathbb{Q}(\sqrt[4]{2},i)$, thus the splitting field must lie in this
    field extension. Also, any field extension that contains the roots
    $\sqrt[4]{2}$ and $\sqrt[4]{2}i$ must also contain
    $i=\sqrt[4]{2}i/\sqrt[4]{2}$, therefore the splitting field is exactly
    $\mathbb{Q}(\sqrt[4]{2},i)$. \\

    To compute $[\mathbb{Q}(\sqrt[4]{2},i):\mathbb{Q}]
    =[\mathbb{Q}(\sqrt[4]{2},i):\mathbb{Q}(\sqrt[4]{2})]
    [\mathbb{Q}(\sqrt[4]{2}):\mathbb{Q}]$, we first compute
    $[\mathbb{Q}(\sqrt[4]{2}):\mathbb{Q}]$. The irreducible polynomial of
    $\sqrt[4]{2}$ over $\mathbb{Q}$ divides $x^4-2$. Now $x^4-2$ is
    irreducible over $\mathbb{Q}$ by Gauss's Lemma since $\pm1$, $\pm2$ are
    not roots of the polynomial. Therefore
    $[\mathbb{Q}(\sqrt[4]{2}):\mathbb{Q}]=4$. Also,
    $i\not\in\mathbb{Q}(\sqrt[4]{2})$, and the irreducible polynomial of
    $i$ over $\mathbb{Q}$ is $x^2+1$, thus
    $[\mathbb{Q}(\sqrt[4]{2},i):\mathbb{Q}(\sqrt[4]{2})]=2$ and
    $[\mathbb{Q}(\sqrt[4]{2},i):\mathbb{Q}]=4\cdot2=8$.
  \end{proof}

\it \textbf{DF 13.4.2:} Determine the splitting field and its degree over
  $\mathbb{Q}$ for $x^4+2$.

  \begin{proof}
    This polynomial has four distinct roots $\{\pm2^{-1/4}(1+i),
    \pm2^{-1/4}(1-i)\}$. These roots lie in $\mathbb{Q}(2^{-1/4},i)$, thus
    the splitting field must lie in this field extension. Also, any field
    extension that contains the roots $2^{-1/4}(1+i)$ and $2^{-1/4}(1-i)$
    must also contain $2^{-1/4}=(2^{-1/4}(1+i) +2^{-1/4}(1-i))/2$. Hence
    such an extension must also contain $i=2^{-1/4}(1+i)/2^{-1/4}-1$.
    Therefore the splitting field is exactly $\mathbb{Q}(2^{-1/4},i)$. \\

    To compute $[\mathbb{Q}(2^{-1/4},i):\mathbb{Q}]
    =[\mathbb{Q}(2^{-1/4},i):\mathbb{Q}(2^{-1/4})]
    [\mathbb{Q}(2^{-1/4}):\mathbb{Q}]$, we first compute
    $[\mathbb{Q}(2^{-1/4}):\mathbb{Q}]$. The irreducible
    polynomial of $2^{-1/4}$ over $\mathbb{Q}$ divides $2x^4-1$. Now
    $2x^4-1$ is irreducible over $\mathbb{Q}$ by Gauss's Lemma. Therefore
    $[\mathbb{Q}(2^{-1/4}):\mathbb{Q}]=4$. Also,
    $i\not\in\mathbb{Q}(2^{-1/4})$, and the irreducible polynomial of
    $i$ over $\mathbb{Q}$ is $x^2+1$, thus
    $[\mathbb{Q}(2^{-1/4},i):\mathbb{Q}(2^{-1/4})]=2$ and
    $[\mathbb{Q}(2^{-1/4},i):\mathbb{Q}]=4\cdot2=8$.
  \end{proof}

\it \textbf{DF 13.4.3:} Determine the splitting field and its degree over
  $\mathbb{Q}$ for $x^4+x^2+1$.

  \begin{proof}
    This polynomial has four distinct roots
    $\{\omega,-\omega,1/\omega,-1/\omega\}$, where $\omega=e^{\pi i/3}$.
    Therefore the splitting field is $\mathbb{Q}(\omega)$. Now $\omega$
    satisfies the polynomial $x^3-1$, which can be factorized as
    $(x-1)(x^2+x+1)$. Then $\omega$ satisfies $x^2+x+1$, which is
    irreducible over $\mathbb{Q}$. Therefore the splitting field
    $\mathbb{Q}(\omega)$ has degree 2.
  \end{proof}

\it \textbf{DF 13.4.4:} Determine the splitting field and its degree over
  $\mathbb{Q}$ for $x^6-4$.

  \begin{proof}
    This polynomial has six distinct roots $\{\sqrt[3]{2}\omega^k:
    k\in\{-2,-1,0,1,2,3\}\}$, where $\omega=e^{\pi/3}=1/2+\sqrt{3}i/2$.
    These roots lie in $\mathbb{Q}(\sqrt[3]{2},\sqrt{3}i)$, thus the
    splitting field must lie in this field extension. Also, any field
    extension that contains the roots $\sqrt[3]{2}$ and $\omega$
    must also contain $\sqrt{3}i$, so the splitting field is
    exactly $\mathbb{Q}(\sqrt[3]{2},\sqrt{3}i)$. \\

    Now the minimal polynomial of $\sqrt[3]{2}$ over $\mathbb{Q}$ is
    $x^3-2$, therefore $[\mathbb{Q}(\sqrt[3]{2}):\mathbb{Q}]=3$.  Also, the
    minimal polynomial of $\sqrt{2}i$ over $\mathbb{Q}$ is $x^2+3$,
    therefore $[\mathbb{Q}(\sqrt{2}i):\mathbb{Q}]=2$. Hence
    $[\mathbb{Q}(\sqrt[3]{2},\sqrt{3}i):\mathbb{Q}]$ must be a multiple of
    both 2 and 3, and be not larger than the product $2\times3=6$.
    Therefore $\mathbb{Q}(\sqrt[3]{2},\sqrt{3}i)=6$.
  \end{proof}

\it \textbf{DF 13.4.5:} Let $K$ be a finite extension of $F$. Prove that
  $K$ is a splitting field over $F$ if and only if every irreducible
  polynomial in $F[x]$ that has a root in $K$ splits completely in $K[x]$.

  \begin{proof}
    Let $K$ be a splitting field over $F$ for some polynomial $f(x)\in
    F[x]$. Then $K=F(a_1,\ldots,a_n)$, where $a_1,\ldots,a_n$ are the roots
    of $f(x)$ in $K$. Let $p(x)\in F[x]$ be an irreducible polynomial with
    a root $\alpha\in K$. Let $L\subseteq K$ be a splitting field over $K$
    for $p(x)$. Note that $L$ exists from Theorem 25. Let $\beta\in L$ be
    an arbitrary root of $p(x)$. We wish to show that $\beta$ lies in
    $K$. \\

    Consider the field extension $F\subseteq F(\alpha) \subseteq K
    \subseteq K(\beta)$, as illustrated in the diagram below. Consider also
    the field extension $F\subseteq F(\beta)$. Since $\alpha$ and $\beta$
    are roots of a minimal polynomial $p(x)$ over $F$, $F(\alpha)$ will be
    $F$-isomorphic to $F(\beta)$ from Theorem 8. Let $K'\supseteq F(\beta)$
    be the splitting field over $F(\beta)$ for $f(x)$. Then
    $K'=F(a_1,\ldots,a_n,\beta)$. Now $K$ is the splitting field over $F$
    for $f(x)$, and is therefore also the splitting field over $F(\alpha)$
    for $f(x)$. So we have $F(\alpha)\cong F(\beta)$, $K\supseteq
    F(\alpha)$ and $K'\supseteq F(\beta)$ are splitting fields over
    $F(\alpha)$ and $F(\beta)$ respectively over the same polynomial
    $f(x)$. Therefore $K\cong K'$ from Theorem 27. Then since $K=
    F(a_1,\ldots,a_n)\subseteq K(\beta)$ and $K'=
    F(a_1,\ldots,a_n,\beta)\subseteq K(\beta)$, we have the field extension
    \[F(a_1,\ldots,a_n) \subseteq F(a_1,\ldots,a_n,\beta) \subseteq
    K(\beta).\]

    And so
    \begin{align*}
      [K(\beta):F(a_1,\ldots,a_n)] &=[K(\beta):F(a_1,\ldots,a_n,\beta)]
        [F(a_1,\ldots,a_n,\beta):F(a_1,\ldots,a_n)]\\
      &=[K(\beta):F(a_1,\ldots,a_n)]
        [F(a_1,\ldots,a_n,\beta):F(a_1,\ldots,a_n)].\\
    \end{align*}

    Cancelling the $[K(\beta):F(a_1,\ldots,a_n)]$ term on both sides gives
    \[[F(a_1,\ldots,a_n,\beta):F(a_1,\ldots,a_n)]=1,\]
    which implies $\beta\in F(a_1,\ldots,a_n)=1$, as required.

    \begin{center}
      \begin{tikzcd}[column sep=small]
        & L \arrow[dash]{d} &\\
        & K(\beta) \arrow[dash]{dl} \arrow[dash]{dr} &\\
        K=F(a_1,\ldots,a_n) \arrow[dash]{d} \arrow[dash,dotted]{rr}{\cong}
          && K'=F(a_1,\ldots,a_n,\beta) \arrow[dash]{d}\\
        F(\alpha) \arrow[dash]{dr} \arrow[dash,dotted]{rr}{\cong} &&
          F(\beta) \arrow[dash]{dl}\\
        & F &\\
      \end{tikzcd}
    \end{center}
  \end{proof}
\end{document}
