\documentclass{article}
\usepackage[left=3cm,right=3cm,top=3cm,bottom=3cm]{geometry}
\usepackage{amsmath,amssymb,amsthm,pgfplots,tikz}
\usepackage{tikz-cd}
\usepackage[inline]{enumitem}
\usepackage{color}
\setlength{\parindent}{0mm} %So that we do not indent on new paragraphs
\newcommand{\TODO}[1]{\textcolor{red}{TODO: #1}}

\begin{document}
\title{Graduate Algebra II: Homework 9}
\author{Li Ling Ko\\ lko@nd.edu}
\date{\today}
\maketitle

\it \textbf{DF 13.6.3:} Prove that if a field contains the $n$-th roots of
  unity for $n$ odd then it also contains the $2n$-th roots of unity.

  \begin{proof}
    We first show that for $n$ odd,
    \[\mu_{2n} =\mu_n \cup \{-\xi:\xi\in\mu_n\}.\]
    Clearly $\supseteq$ holds since $x^n=1$ implies $x^{2n}=1=(-x)^{2n}$.
    For $\subseteq$, observe that if $x^n=1$, then $(-x)^n=-x^n\neq1$.
    Therefore the right side of the equality has exactly $2n$ elements.
    Then since $|\mu_{2n}|=2n$, the equality above holds. \\

    Then since fields are closed under negation,
    $\mathbb{Q}(\mu_n)=\mathbb{Q}(\mu_{2n})$ as required.
  \end{proof}

\it \textbf{DF 13.6.4:} Prove that if $n=p^km$ where $p$ is a prime and $m$
  is relatively prime to $p$ then there are precisely $m$ distinct $n$-th
  roots of unity over a field of characteristic $p$.

  \begin{proof}
    Over a field of characteristic $p$, by Proposition 35,
    \[x^{p^km}-1 =(x^m-1)^{p^k},\]
    so there are at most $m$ distinct $m$-th roots of unity. It suffices to
    show that the roots of $x^m-1$ are distinct. Differentiating with
    respect to $x$ gives $D_x(x^m-1)=mx^{m-1}$, which is non-zero since $p$
    does not divide $m$. Therefore for $m>1$, $x^m-1$ and its derivative
    are coprime, so the roots of $x^m-1$ are distinct by Proposition 33.
  \end{proof}

\it \textbf{DF 13.6.5:} Prove that there are only a finite number of roots
  of unity in any finite extension $K$ of $\mathbb{Q}$.

  \begin{proof}
    If $\xi_n$ lies in $K$, then the degree $\varphi(n)$ of its cyclotomic
    polynomial $\Phi_n(x)$ cannot be larger than $[K:\mathbb{Q}]$.
    Therefore it suffices to that given any $m\in\mathbb{N}$, there can
    only be a finite number of $m\in\mathbb{N}$ such that $\varphi(m)=n$.
    \\

    To that end, fix $m\in\mathbb{N}$. Let $p_1<\ldots<p_m$ be the
    first $m$ primes. For each prime $p_i$, let $k_i\in\mathbb{N}$ be the
    smallest positive integer such that $p_i^{k_i}-p_i^{k_i-1}>m$. Let $n$
    be any positive integer greater than
    \[N:=(p_1^{k_1}-p_1^{k_1-1}) \cdots (p_m^{k_m}-p_m^{k_m-1}).\]

    Then either $n$ has a prime factor $p$ smaller than $p_m$ such that
    $p^{k}|n$ for some $k>k_r-1$, or $n$ has at least one prime factor $q$
    greater than $p_m$. In the former case, $p^k-p^{k-1}$ divides
    $\varphi(n)$, so $\varphi$ will be greater than $m$. In the latter
    case, the prime factor $q$ that is greater than $p_m$ must also be
    greater than $m$, since $p_m>m$. But $q^k-q^{k-1}$ divides $\varphi(n)$
    for some $k\geq1$, so $\varphi$ is also greater than $m$. Thus all
    integers greater than $N$ will have Euler-totient value greater than
    $m$.
  \end{proof}

\it \textbf{DF 13.6.6:} Prove that for $n$ odd, $\Phi_{2n}(x)=\Phi_n(-x)$.
  \begin{proof}
    In Exercise 13.6.3, we proved that when $n$ is odd,
    \[\mu_{2n} =\mu_n \cup \{-\xi:\xi\in\mu_n\}.\]

    Therefore the $2n$-th roots of unity either satisfy $\Phi_n(x)$ or
    $\Phi_n(-x)$. Note that $\Phi_n(-x)$ is a monic polynomial because
    $\text{deg}(\Phi_n(x)) =\varphi(n)$ which is even because $n$ is odd.
    Now since $\Phi_n(x)$ is irreducible over $\mathbb{Q}$, $\Phi_n(-x)$
    must also be irreducible. Therefore $\Phi_{2n}(x)$ is either
    $\Phi_n(x)$ or $\Phi_n(-x)$. Now the roots of $\Phi_n(x)$ cannot
    contain the primitive $2n$-th root of unity since it contains exactly
    the $n$-th roots of unity, therefore the primitive $2n$-th root of
    unity must satisfy the other polynomial $\Phi_n(-x)$. Therefore
    $\Phi_{2n}(x)=\Phi_n(x)$.
  \end{proof}
\end{document}
