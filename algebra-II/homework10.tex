\documentclass{article}
\usepackage[left=3cm,right=3cm,top=3cm,bottom=3cm]{geometry}
\usepackage{amsmath,amssymb,amsthm}
%\usepackage{pgfplots,tikz,tikz-cd}
\usepackage[inline]{enumitem}
\usepackage{color}
\setlength{\parindent}{0mm} %So that we do not indent on new paragraphs
\newcommand{\TODO}[1]{\textcolor{red}{TODO: #1}}

\begin{document}
\title{Graduate Algebra II: Homework 10}
\author{Li Ling Ko\\ lko@nd.edu}
\date{\today}
\maketitle

\it \textbf{DF 14.2.3:} Determine the Galois group of
  $(x^2-2)(x^2-3)(x^2-5)$. Determine all the subfields of the splitting
  field of this polynomial.

  \begin{proof}
    The splitting field of the given polynomial over $\mathbb{Q}$ is
    $K:=\mathbb{Q}(\sqrt{2},\sqrt{3},\sqrt{5})$. By the Galois Theory, the
    Galois group $G$ of this polynomial has order
    $[K:\mathbb{Q}]$ which equals to
    8, since $\sqrt{3}\not\in\mathbb{Q}(\sqrt{2})$ and
    $\sqrt{5}\not\in\mathbb{Q}(\sqrt{2},\sqrt{3})$. Also, any automorphism
    in the Galois group must map $\sqrt{n}$ to $\pm\sqrt{n}$ for
    $n\in\{2,3,5\}$, since the minimal polynomial of $\sqrt{n}$ over
    $\mathbb{Q}$ has exactly two roots $\pm\sqrt{n}$. Furthermore, any
    automorphism in the Galois group is completely defined by its image of
    $\sqrt{2}$, $\sqrt{3}$, and $\sqrt{5}$, giving a maximum of $2^3$
    possible automorphisms in the Galois group for each combination of
    image. Thus the Galois group is isomorphic to $\mathbb{Z}_2^3$. \\

    In addition, as a $\mathbb{Q}$-vector space,
    $\mathbb{Q}(\sqrt{2})$ has basis $\{1,\sqrt{2}\}$, so
    $\mathbb{Q}(\sqrt{2},\sqrt{3})$ has basis
    $\{1,\sqrt{2},\sqrt{3},\sqrt{6}\}$, and then $K$ has basis
    \[\{1,\sqrt{2},\sqrt{3},\sqrt{6},
    \sqrt{5},\sqrt{10},\sqrt{15},\sqrt{30}\}.\]

    Now by the fundamental theorem of the Galois Theory, there is a bijection
    between the subfields of $\mathbb{Q}(\sqrt{2},\sqrt{3},\sqrt{5})$ and
    the subgroups of $G= \text{Gal}(\mathbb{Q}(\sqrt{2},\sqrt{3},\sqrt{5})
    /\mathbb{Q})$. The subfield corresponding to $G$ is $\mathbb{Q}$ and
    the subfield corresponding to $1$ is
    $\mathbb{Q}(\sqrt{2},\sqrt{3},\sqrt{5})$. Since
    $G\simeq\mathbb{Z}_2^3$, its non-trivial subgroups $H<G$ are either
    isomorphic to $\mathbb{Z}_2$ or to $\mathbb{Z}_2^2$. \\

    Now $G$ is generated by the three automorphisms
    $\{\sigma_2,\sigma_3,\sigma_5\}$, where $\sigma_n$ sends $\sqrt{n}$ to
    $-\sqrt{n}$ and fixes $\sqrt{m}$ for $m\in\{2,3,5\}\setminus\{n\}$.
    Each $\sigma_n$ has order 2, and
    \[G =\langle\sigma_2,\sigma_3,\sigma_5\rangle
    =\{1,\sigma_2,\sigma_3,\sigma_5, \sigma_2\sigma_3, \sigma_2\sigma_5,
    \sigma_3\sigma_5, \sigma_2\sigma_3\sigma_5\}.\]

    So if $H\simeq\mathbb{Z}_2$, then $H$ must be generated by an
    element $\sigma\in G$ of order 2, so it must be of the form
    $H=\langle\sigma\rangle$ where $\sigma\in G\setminus\{1\}$. By
    considering the action on of the generator on each of the basis
    elements, we can find the corresponding subfields of the above groups:
    \begin{table}[h]
      \centering
      \begin{tabular}{c|c}
        $H<G$ & Fixed field of $H$ \\
        \hline
        $\langle \sigma_2\rangle$ &$\mathbb{Q}(\sqrt{3},\sqrt{5})$\\
        $\langle \sigma_3\rangle$ &$\mathbb{Q}(\sqrt{2},\sqrt{5})$\\
        $\langle \sigma_5\rangle$ &$\mathbb{Q}(\sqrt{2},\sqrt{3})$\\
        $\langle \sigma_2\sigma_3\rangle$ &$\mathbb{Q}(\sqrt{5},\sqrt{6})$\\
        $\langle \sigma_2\sigma_5\rangle$ &$\mathbb{Q}(\sqrt{3},\sqrt{10})$\\
        $\langle \sigma_3\sigma_5\rangle$ &$\mathbb{Q}(\sqrt{2},\sqrt{15})$\\
        $\langle \sigma_2\sigma_3\sigma_5\rangle$
          &$\mathbb{Q}(\sqrt{6},\sqrt{10})$ \\
      \end{tabular}
    \end{table}

    Finally, if $H\simeq\mathbb{Z}_2$, then $H$ must be generated by any
    two distinct elements $\sigma,\tau\in G\setminus\{1\}$. Once again, by
    considering the action on of the generator on each of the basis
    elements, we can find the corresponding subfields of the above groups:
    \begin{table}[h]
      \centering
      \begin{tabular}{c|c}
        $H<G$ & Fixed field of $H$ \\
        \hline
        $\langle \sigma_2,\sigma_3\rangle$ &$\mathbb{Q}(\sqrt{5})$\\
        $\langle \sigma_2,\sigma_5\rangle$ &$\mathbb{Q}(\sqrt{3})$\\
        $\langle \sigma_3,\sigma_5\rangle$ &$\mathbb{Q}(\sqrt{2})$\\
        $\langle \sigma_2,\sigma_3\sigma_5\rangle$ &$\mathbb{Q}(\sqrt{15})$\\
        $\langle \sigma_3,\sigma_2\sigma_5\rangle$ &$\mathbb{Q}(\sqrt{10})$\\
        $\langle \sigma_5,\sigma_2\sigma_3\rangle$ &$\mathbb{Q}(\sqrt{6})$\\
        $\langle \sigma_2\sigma_3,\sigma_3\sigma_5\rangle$
          &$\mathbb{Q}(\sqrt{30})$\\
      \end{tabular}
    \end{table}
  \end{proof}
\end{document}
