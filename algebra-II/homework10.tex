\documentclass{article}
\usepackage[left=3cm,right=3cm,top=3cm,bottom=3cm]{geometry}
\usepackage{amsmath,amssymb,amsthm}
%\usepackage{pgfplots,tikz,tikz-cd}
\usepackage[inline]{enumitem}
\usepackage{color}
\setlength{\parindent}{0mm} %So that we do not indent on new paragraphs
\newcommand{\TODO}[1]{\textcolor{red}{TODO: #1}}

\begin{document}
\title{Graduate Algebra II: Homework 10}
\author{Li Ling Ko\\ lko@nd.edu}
\date{\today}
\maketitle

\textbf{Note:} This homework was completed with a significant amount of
  help from Justin Miller, without which I will take a much longer time to
  complete the problems. \\\\

\it \textbf{DF 14.2.3:} Determine the Galois group of
  $(x^2-2)(x^2-3)(x^2-5)$. Determine all the subfields of the splitting
  field of this polynomial.

  \begin{proof}
    The splitting field of the given polynomial over $\mathbb{Q}$ is
    $K:=\mathbb{Q}(\sqrt{2},\sqrt{3},\sqrt{5})$. By the Galois Theory, the
    Galois group $G$ of this polynomial has order
    $[K:\mathbb{Q}]$ which equals to
    8, since $\sqrt{3}\not\in\mathbb{Q}(\sqrt{2})$ and
    $\sqrt{5}\not\in\mathbb{Q}(\sqrt{2},\sqrt{3})$. Also, any automorphism
    in the Galois group must map $\sqrt{n}$ to $\pm\sqrt{n}$ for
    $n\in\{2,3,5\}$, since the minimal polynomial of $\sqrt{n}$ over
    $\mathbb{Q}$ has exactly two roots $\pm\sqrt{n}$. Furthermore, any
    automorphism in the Galois group is completely defined by its image of
    $\sqrt{2}$, $\sqrt{3}$, and $\sqrt{5}$, giving a maximum of $2^3$
    possible automorphisms in the Galois group for each combination of
    image. Thus the Galois group is isomorphic to $\mathbb{Z}_2^3$. \\

    In addition, as a $\mathbb{Q}$-vector space,
    $\mathbb{Q}(\sqrt{2})$ has basis $\{1,\sqrt{2}\}$, so
    $\mathbb{Q}(\sqrt{2},\sqrt{3})$ has basis
    $\{1,\sqrt{2},\sqrt{3},\sqrt{6}\}$, and then $K$ has basis
    \[\{1,\sqrt{2},\sqrt{3},\sqrt{6},
    \sqrt{5},\sqrt{10},\sqrt{15},\sqrt{30}\}.\]

    Now by the fundamental theorem of the Galois Theory, there is a bijection
    between the subfields of $\mathbb{Q}(\sqrt{2},\sqrt{3},\sqrt{5})$ and
    the subgroups of $G= \text{Gal}(\mathbb{Q}(\sqrt{2},\sqrt{3},\sqrt{5})
    /\mathbb{Q})$. The subfield corresponding to $G$ is $\mathbb{Q}$ and
    the subfield corresponding to $1$ is
    $\mathbb{Q}(\sqrt{2},\sqrt{3},\sqrt{5})$. Since
    $G\simeq\mathbb{Z}_2^3$, its non-trivial subgroups $H<G$ are either
    isomorphic to $\mathbb{Z}_2$ or to $\mathbb{Z}_2^2$. \\

    Now $G$ is generated by the three automorphisms
    $\{\sigma_2,\sigma_3,\sigma_5\}$, where $\sigma_n$ sends $\sqrt{n}$ to
    $-\sqrt{n}$ and fixes $\sqrt{m}$ for $m\in\{2,3,5\}\setminus\{n\}$.
    Each $\sigma_n$ has order 2, and
    \[G =\langle\sigma_2,\sigma_3,\sigma_5\rangle
    =\{1,\sigma_2,\sigma_3,\sigma_5, \sigma_2\sigma_3, \sigma_2\sigma_5,
    \sigma_3\sigma_5, \sigma_2\sigma_3\sigma_5\}.\]

    So if $H\simeq\mathbb{Z}_2$, then $H$ must be generated by an
    element $\sigma\in G$ of order 2, so it must be of the form
    $H=\langle\sigma\rangle$ where $\sigma\in G\setminus\{1\}$. By
    considering the action on of the generator on each of the basis
    elements, we can find the corresponding subfields of the above groups:
    \begin{table}[h]
      \centering
      \begin{tabular}{c|c}
        $H<G$ & Fixed field of $H$ \\
        \hline
        $\langle \sigma_2\rangle$ &$\mathbb{Q}(\sqrt{3},\sqrt{5})$\\
        $\langle \sigma_3\rangle$ &$\mathbb{Q}(\sqrt{2},\sqrt{5})$\\
        $\langle \sigma_5\rangle$ &$\mathbb{Q}(\sqrt{2},\sqrt{3})$\\
        $\langle \sigma_2\sigma_3\rangle$ &$\mathbb{Q}(\sqrt{5},\sqrt{6})$\\
        $\langle \sigma_2\sigma_5\rangle$ &$\mathbb{Q}(\sqrt{3},\sqrt{10})$\\
        $\langle \sigma_3\sigma_5\rangle$ &$\mathbb{Q}(\sqrt{2},\sqrt{15})$\\
        $\langle \sigma_2\sigma_3\sigma_5\rangle$
          &$\mathbb{Q}(\sqrt{6},\sqrt{10})$ \\
      \end{tabular}
    \end{table}

    Finally, if $H\simeq\mathbb{Z}_2$, then $H$ must be generated by any
    two distinct elements $\sigma,\tau\in G\setminus\{1\}$. Once again, by
    considering the action on of the generator on each of the basis
    elements, we can find the corresponding subfields of the above groups:
    \begin{table}[h]
      \centering
      \begin{tabular}{c|c}
        $H<G$ & Fixed field of $H$ \\
        \hline
        $\langle \sigma_2,\sigma_3\rangle$ &$\mathbb{Q}(\sqrt{5})$\\
        $\langle \sigma_2,\sigma_5\rangle$ &$\mathbb{Q}(\sqrt{3})$\\
        $\langle \sigma_3,\sigma_5\rangle$ &$\mathbb{Q}(\sqrt{2})$\\
        $\langle \sigma_2,\sigma_3\sigma_5\rangle$ &$\mathbb{Q}(\sqrt{15})$\\
        $\langle \sigma_3,\sigma_2\sigma_5\rangle$ &$\mathbb{Q}(\sqrt{10})$\\
        $\langle \sigma_5,\sigma_2\sigma_3\rangle$ &$\mathbb{Q}(\sqrt{6})$\\
        $\langle \sigma_2\sigma_3,\sigma_3\sigma_5\rangle$
          &$\mathbb{Q}(\sqrt{30})$\\
      \end{tabular}
    \end{table}
  \end{proof}

\it \textbf{DF 14.2.4:} Let $p$ be a prime. Determine the elements of the
  Galois group of $x^p-2$.

  \begin{proof}
    Following the example in Section 13.4, the splitting field of $x^p-2$
    over $\mathbb{Q}$ is $K:=\mathbb{Q}(\sqrt[p]{2},\zeta_p)$, where
    $\zeta_p$ is a primitive $p$-th root of unity, and
    $[K:\mathbb{Q}]=p(p-1)$. Therefore any automorphism of $K$ is fully
    defined by its image on $\sqrt[p]{2}$ and on $\zeta_p$. Now
    $m_{\sqrt[p]{2},\mathbb{Q}}(x)=x^p-2$, so $\sqrt[p]{2}$ has $p$
    distinct conjugates $\{\sqrt[p]{2}\zeta^i_p:0\leq i<p\}$ over
    $\mathbb{Q}$. Also, $\zeta_p$ has $p-1$ conjugates $\{\zeta_p^i:1\leq
    i<p\}$ over $\mathbb{Q}$. Since automorphisms must map roots to their
    conjugates, there are at most $p(p-1)$ combinations that an
    automorphism can map $\sqrt[p]{2}$ and $\zeta_p$ to. Then since
    the Galois group has exactly $p(p-1)$ elements, the combinations must
    be exactly all the elements in the Galois group:
    \[\text{Gal}(K/\mathbb{Q}) =\{\sigma\in\text{Aut}(K/\mathbb{Q}),\;
    \sigma(\sqrt[p]{2})=\sqrt[p]{2}\zeta^i_p,\;
    \sigma(\zeta_p)=\zeta_p^j:\; 0\leq i<p,\; 1\leq j<p\}.\]
  \end{proof}

\it \textbf{DF 14.2.5:} Prove that the Galois group of $x^p-2$ for $p$ a
  prime is isomorphic to the group of matrices $\begin{pmatrix} a&b\\ 0&1\\
  \end{pmatrix}$, where $a,b\in\mathbb{F}_p$ and $a\neq0$.

  \begin{proof}
    In the previous problem we showed that the splitting field of $x^p-2$
    over $\mathbb{Q}$ is $K:=\mathbb{Q}(\sqrt[p]{2},\zeta)$, where
    $\zeta$ is a primitive $p$-th root of unity, and
    $[K:\mathbb{Q}]=p(p-1)$. Also,
    \[\text{Gal}(K/\mathbb{Q}) =\{\sigma_{i,j}\in\text{Aut}(K/\mathbb{Q}),\;
    \sigma_{i,j}(\sqrt[p]{2})=\sqrt[p]{2}\zeta^i,\;
    \sigma_{i,j}(\zeta)=\zeta^j:\; 0\leq i<p,\; 1\leq j<p\}.\]

    Consider the map $\Psi:\text{Gal}(K/\mathbb{Q}) \rightarrow \left\{
    \begin{pmatrix} a&b\\ 0&1\\ \end{pmatrix}: a,b\in\mathbb{F}_p, a\neq0
    \right\}$ defined by $\Psi(\sigma_{i,j}) =\begin{pmatrix} j&i\\ 0&1\\
    \end{pmatrix}$. If $\Psi$ is a group homomorphism, then it is
    injective, since the $\ker(\Psi)$ clearly contains only the identity
    map. Then since both domain and codomain of $\Psi$ have $p(p-1)$
    elements, $\Psi$ must be an isomorphism if it is a homomorphism. \\

    It remains to show that $\Psi$ is a group homomorphism. To that end,
    first observe that
    $\sigma_{i_1,j_1}\circ\sigma_{i_2,j_2} =\sigma_{i_2j_1+i_1,j_1j_2})$,
    because
    \[\sigma_{i_1,j_1} (\sigma_{i_2,j_2}(\sqrt[p]{2}))
    =\sigma_{i_1,j_1}(\sqrt[p]{2}\zeta^{i_2})
    =\sigma_{i_1,j_1}(\sqrt[p]{2}) \sigma_{i_1,j_1}(\zeta^{i_2})
    =\sqrt[p]{2}\zeta^{i_1}\zeta^{i_2j_1}
    =\sqrt[p]{2}\zeta^{i_2j_1+i_1},\]
    and
    \[\sigma_{i_1,j_1} (\sigma_{i_2,j_2}(\zeta))
    =\sigma_{i_1,j_1}(\zeta^{j_2})
    =\zeta^{j_1j_2}.\]

    Therefore
    \begin{align*}
      \Psi(\sigma_{i_1,j_1}\circ\sigma_{i_2,j_2})
        &=\Psi(\sigma_{i_2j_1+i_1,j_1j_2})\\
      &=\begin{pmatrix} j_1j_2 &i_2j_1+i_1\\ 0&1\\ \end{pmatrix} \\
      &=\begin{pmatrix} j_1 &i_1\\ 0&1\\ \end{pmatrix} \begin{pmatrix} j_2
        &i_2\\ 0&1\\ \end{pmatrix}\\
      &=\Psi(\sigma_{i_1,j_1}) \Psi(\sigma_{i_2,j_2}).\\
    \end{align*}
  \end{proof}

\it \textbf{DF 14.2.6:} Let $K=\mathbb{Q}(\sqrt[8]{2},i)$ and let
  $F_1=\mathbb{Q}(i)$, $F_2=\mathbb{Q}(\sqrt{2})$,
  $F_3=\mathbb{Q}(\sqrt{-2})$. Prove that
  $\text{Gal}(K/F_1)\simeq\mathbb{Z}_8$, $\text{Gal}(K/F_2)\simeq D_8$, and
  $\text{Gal}(K/F_3)\simeq Q_8$.

  \begin{proof}
    $K$ has been analyzed in an example in Section 14.2, which showed that
    $[K:\mathbb{Q}]=16$, and the automorphisms in
    $\text{Gal}(K/\mathbb{Q})$ are determined by their image on
    $\sqrt[8]{2}$ and on $i$. Also, 
    \[\text{Gal}(K/\mathbb{Q}) \simeq\langle \sigma,\tau:\;
    \sigma^8=\tau^2=1, \sigma\tau=\tau\sigma^3\rangle,\]
    where $\sigma$ fixes $i$ but sends $\sqrt[8]{2}$ to $\sqrt[8]{2}\zeta$,
    where $\zeta$ is the primitive 8th root of unity, while $\tau$ fixes
    $\sqrt[8]{2}$ but sends $i$ to $-i$. \\

    Now $\tau$ does not fix $i$ but $\sigma$ does, and any subgroup larger
    than $\langle\sigma\rangle$ will contain $\sigma^k\tau$ for some
    $0<k<8$, which will not fix $i$. Therefore $\text{Gal}(K/F_1)
    =\langle\sigma\rangle \simeq\mathbb{Z}_8$. \\

    Next, observe that $\sigma^k\tau^j$ fixes $\sqrt{2}$ if and only if $k$
    is even, because $\zeta^{4k}$ is even if and only if $k$ is even.
    Therefore
    \[\text{Gal}(K/F_2) \simeq\langle \sigma^2,\tau:\;
    (\sigma^2)^4=\tau^2=1,\; \sigma^2\tau=\tau\sigma^6 \rangle \simeq
    \langle \alpha,\tau:\; \alpha^4=\tau^2=1,\;
    \alpha\tau=\tau\alpha^3\rangle \simeq D_8.\]

    Finally, observe that $\sigma^k\tau^j$ fixes $\sqrt{-2}$ if and only if
    $k\equiv j\mod2$, because
    \[\sigma^k\tau^j(\sqrt{-2}) =\sigma^k\tau^j(i\sqrt[8]{2}^4)
    =\sigma^k((-1)^ji\sqrt[8]{2}^4) =(-1)^ji\zeta^{4k}\sqrt[8]{2}^4,\]
    which is $\sqrt{-2}$ if and only if $(-1)^j=\zeta^{4k}=1$ or when
    $(-1)^j=\zeta^{4k}=-1$. Therefore
    \begin{align*}
      \text{Gal}(K/F_3) &\simeq \langle \sigma^2,\sigma\tau:
        (\sigma\tau)^8=(\sigma^2)^4=1,
        \sigma^2(\sigma\tau)\sigma^6=(\tau\sigma)^{-1},
        (\sigma\tau)^4=(\sigma^2)^2\rangle \\
      &\simeq \langle \alpha,\beta: \beta^8=\alpha^4=1,
        \alpha\beta\alpha^3=\beta, \beta^2=\alpha^2\rangle \\
      &\simeq Q_8.\\
    \end{align*}
  \end{proof}

\it \textbf{DF 14.2.7:} Determine all the subfields of the splitting field
  of $x^8-2$ which are Galois over $\mathbb{Q}$.

  \begin{proof}
    The splitting field is $K:=\mathbb{Q}(\theta,i)$, where
    $\theta:=\sqrt[8]{2}$. We use the diagram of the subfields of $K$ that
    is shown on Page 581. By Galois Theory, an intermediate field
    $K\supseteq L\supseteq\mathbb{Q}$ is Galois over $\mathbb{Q}$ if and
    only if it is the splitting field of a separable polynomial over
    $\mathbb{Q}$. So the intermediate fields $\mathbb{Q}(\sqrt{2})$,
    $\mathbb{Q}(i)$, $\mathbb{Q}(\sqrt{-2})$, $\mathbb{Q}(i,\sqrt{2})$ and
    $\mathbb{Q}(i,\sqrt[4]{2})$ are Galois over
    $\mathbb{Q}$ with witness $x^2-2$, $x^2+1$, $x^2+2$, $x^2+2$, and
    $x^4+2$ respectively. \\

    On the other hand, letting $\zeta$ denote the primitive 8th root of
    unity, the intermediate fields $\mathbb{Q}(\theta)$,
    $\mathbb{Q}(\zeta\theta)$, $\mathbb{Q}(\zeta^2\theta)$, and
    $\mathbb{Q}(\zeta^3\theta)$ are not Galois over $\mathbb{Q}$, because
    they are simple $\mathbb{Q}$-extensions of an element whose minimal
    polynomial is $x^8-2$, and this polynomial does not split completely
    over the intermediate fields. Likewise, $\mathbb{Q}(\sqrt[4]{2})$ and
    $\mathbb{Q}(\sqrt[4]{2}i)$ are not Galois over $\mathbb{Q}$ with
    minimal polynomial $x^4+2$ as witness. \\

    Similarly, $\mathbb{Q}((1+i)\sqrt[4]{2})$ and
    $\mathbb{Q}((1+i)\sqrt[4]{2})$ are not Galois over $\mathbb{Q}$ because
    they are simple extensions of an element whose minimal polynomial is
    $x^4+2$, which contains $\sqrt[4]{2}i$ as a root, but this root is not
    contained in either intermediate fields.
  \end{proof}

\it \textbf{DF 14.2.8:} Suppose $K$ is a Galois extension of $F$ and of
  degree $p^n$ for some prime $p$ and some $n\geq1$. Show there are Galois
  extensions of $F$ contained in $K$ of degrees $p$ and $p^{n-1}$.

  \begin{proof}
    By Galois theory, the intermediate field field extension $K\supset
    E\supset F$ is Galois if and only if it is the fixed field of some
    normal subgroup $H$ of $\text{Gal}(K/F)$. Furthermore,
    $\text{Gal}(E/F)\simeq G/H$. Thus it suffices to prove that every group
    $G$ of order $p^n$ for some $n\geq1$ has normal subgroups of order $p$
    and of order $p^{n-1}$. \\

    For the first claim, recall that any group $G$ of order $p^n$ has a
    nontrivial center $Z$. So $Z$ has order $p^k$ for some $1\leq k\leq n$.
    Then $Z$ will contain an element $z$ of order $p$, and then $\langle
    z\rangle$ will be a normal subgroup of $G$ order $p$. \\

    We prove the second claim by induction on $n$. The base case $n=1$ is
    trivially satisfied. Let $G$ be a group of order $p^{n+1}$. Then by
    argument in the previous paragraph, $G$ has a normal subgroup $N$ of
    order $p$. Then $G/N$ will be a group of order $p^n$, so from induction
    hypothesis, $G/N$ has a normal subgroup of order $p^{n-1}$. Then by the
    fourth isomorphism theorem for groups, $G$ has a normal subgroup of
    order $p^n$ as required.
  \end{proof}

\it \textbf{DF 14.2.9:} Give an example of fields $\mathbb{Q}\subseteq
  F_1\subseteq F_2\subseteq F_3$, $[F_3:\mathbb{Q}]=8$ and each field is
  Galois over all its subfields with the exception that $F_2$ is not Galois
  over $\mathbb{Q}$.

  \begin{proof}
    Consider $\mathbb{Q} \subset\mathbb{Q}(\sqrt{2})
    \subset\mathbb{Q}(\sqrt[4]{2}) \subset\mathbb{Q}(\sqrt[4]{2},i)$.
    Clearly $\mathbb{Q}(\sqrt{2})$ is Galois over $\mathbb{Q}$ and hence
    also over all its subfields. \\

    However, $\mathbb{Q}(\sqrt[4]{2})$ is not Galois over $\mathbb{Q}$
    because the minimal polynomial of $\sqrt[4]{2}$ over $\mathbb{Q}$ is
    $x^4-2$, which has root $\sqrt[4]{2}i$ that is not contained in
    $\mathbb{Q}(\sqrt[4]{2})$. \\

    Finally, $\mathbb{Q}(\sqrt[4]{2},i)$ is the splitting field extension
    of $x^4+2$ over $\mathbb{Q}$, so it is Galois over $\mathbb{Q}$ and
    hence also over all its subfields. Also,
    \[[\mathbb{Q}(\sqrt[4]{2},i):\mathbb{Q}]
    =[\mathbb{Q}(\sqrt[4]{2},i):\mathbb{Q}(\sqrt[4]{2})]
    [\mathbb{Q}(\sqrt[4]{2}):\mathbb{Q}] =2\times4=8.\]
  \end{proof}

\it \textbf{DF 14.2.11:} Suppose $f(x)\in\mathbb{Z}[x]$ is an irreducible
  quartic whose splitting field $L$ has Galois group $S_4$ over $\mathbb{Q}$.
  Let $\theta$ be a root of $f(x)$ and set $K=\mathbb{Q}(\theta)$. Prove
  that $K$ is an extension of $\mathbb{Q}$ of degree 4 which has no proper
  subfields. Are there any Galois extensions of $\mathbb{Q}$ of degree 4
  with no proper subfields?

  \begin{proof}
    $[K:\mathbb{Q}]=4$ follows directly from the degree of the minimal $f$
    begin 4. If $K$ has a proper subfield $F\supsetneq\mathbb{Q}$, then
    $[K:F]=[F:\mathbb{Q}]=2$ since $[K:\mathbb{Q}]=4$. Then
    $[L:F]=|S_4|/2$, which implies from Galois theory that
    $\text{Gal}(L/F)$ is a subgroup of $S_4$ of index 2. Now the only
    subgroup of $S_4$ of index 2 is $A_4$, so $\text{Gal}(L/F)=A_4$. \\

    Now to determine $\text{Gal}(L/K)$, observe that any permutation of
    three roots not equal to $\theta$ is an automorphism since
    $\text{Gal}(L/F)\simeq S_4$. Then since a permutation fixes $K$ if and
    only if $\theta$ is fixed, $\text{Gal}(L/K)$ must be $S_3$.  So from
    Galois theory, if $F$ exists, $S_3$ must be a subgroup of $A_4$, a
    contradiction. \\

    No. A group of order 4 is either isomorphic to $\mathbb{Z}_4$ or to
    $\mathbb{2}_2^2$, both of which have non-trivial subgroups. Thus but
    Galois Theory, the fixed field of the non-trivial subgroups will be
    proper subfields.
  \end{proof}

\it \textbf{DF 14.2.12:} Determine the Galois group of the splitting field
  over $\mathbb{Q}$ of $x^4-14x^2+9$.

  \begin{proof}
    $x^4-14x^2+9$ has four distinct roots $\pm\sqrt{7\pm2\sqrt{10}}$.
    Observe that $\sqrt{7+2\sqrt{10}} \cdot\sqrt{7-2\sqrt{10}} =3$,
    therefore the splitting field of the given polynomial is
    $K:=\mathbb{Q}(\sqrt{7+2\sqrt{10}})$. Then
    $[K:\mathbb{Q}]=4$ because $x^4-14x^2+9$
    has no roots in $\mathbb{Q}$ and also cannot factor into quadratic
    factors. Therefore $\text{Gal}(K/L)$ is either isomorphic to
    $\mathbb{Z}_4$ or to $\mathbb{Z}_2^2$. Now since $x^4-14x^2+9$ is
    irreducible with four distinct roots, for any distinct pair of roots,
    there must be an automorphism in $\text{Gal}(K/\mathbb{Q})$ that sends
    one root to the other. We check that the automorphism that sends
    $\sqrt{7+2\sqrt{10}}$ to $-\sqrt{7+2\sqrt{10}}$ and the automorphism
    that sends $\sqrt{7+2\sqrt{10}}$ to $\sqrt{7-2\sqrt{10}}$ both have
    order two. Thus $\text{Gal}(K/L)$ must be isomorphic to
    $\mathbb{Z}_2^2$.
  \end{proof}

\it \textbf{DF 14.2.13:} Prove that if the Galois group of the splitting
  field of a cubic over $\mathbb{Q}$ is the cyclic group of order 3 then
  all of the roots of the cubic are real.

  \begin{proof}
    Let $K$ denote the splitting field of the cubic $f(x)$ over
    $\mathbb{Q}$. Then $[K:\mathbb{Q}]=|\text{Gal}(K/\mathbb{Q})|=3$. Now
    since $f$ is cubic, it has at least one real root $r\in\mathbb{R}$.
    Then since $|\text{Gal}(K/\mathbb{Q})|=3$, $K$ must equal
    $\mathbb{Q}(r)$, and $r$ cannot be rational. Yet as the splitting
    field, $K$ must also include the remaining two roots, so those roots
    must also be real.
  \end{proof}

\it \textbf{DF 14.2.14:} Show that $\mathbb{Q}(\sqrt{2+\sqrt{2}})$ is a
  cyclic quartic field.

  \begin{proof}
    The minimal polynomial of $\sqrt{2+\sqrt{2}}$ is $x^4-4x^2+2$, which is
    irreducible because it has no rational roots or quadratic factors.  The
    four distinct roots of the polynomial are $\pm\sqrt{2\pm\sqrt{2}})$.
    Then since $\sqrt{2}\in K:=\mathbb{Q}(\sqrt{2+\sqrt{2}})$ and
    $\sqrt{2+\sqrt{2}} \cdot\sqrt{2-\sqrt{2}}=\sqrt{2}$, $K$ is the
    splitting field of $x^4-4x^2+2$. Thus $K$ is Galois over $\mathbb{Q}$,
    and has order 4. Now since $x^4-4x^2+2$ is irreducible with four
    distinct roots, for any distinct pair of roots, there must be an
    automorphism in $\text{Gal}(K/\mathbb{Q})$ that sends one root to the
    other. We check that the automorphism that sends $\sqrt{2+\sqrt{2}}$ to
    $\sqrt{2-\sqrt{2}}$ has order 4. Therefore $\text{Gal}(K/\mathbb{Q})$
    is isomorphic to $\mathbb{Z}_4$.
  \end{proof}

\it \textbf{DF 14.2.16:}
  \begin{enumerate}[label={(\alph*)}]
    \item Prove that $x^4-2x^2-2$ is irreducible over $\mathbb{Q}$.
      \begin{proof}
        The polynomial is irreducible over $\mathbb{Z}$ with prime 2 as
        witness. Thus it is also irreducible over $\mathbb{Q}$ by Gauss's
        Lemma.
      \end{proof}

    \item Show the roots of this quartic are $\alpha_1=\sqrt{1+\sqrt{3}}$,
      $\alpha_2=\sqrt{1-\sqrt{3}}$, $\alpha_3=-\sqrt{1+\sqrt{3}}$, and
      $\alpha_4=-\sqrt{1-\sqrt{3}}$.

      \begin{proof}
        This is true by the quadratic formula.
      \end{proof}

    \item Let $K_1=\mathbb{Q}(\alpha_1)$ and $K_2=\mathbb{Q}(\alpha_2)$.
      Show that $K_1\neq K_2$ and $K_1\cap K_2=\mathbb{Q}(\sqrt{3})=F$.

      \begin{proof}
        $\alpha_2$ is complex but $\alpha_1$ is real, so $K_1$ cannot be
        equal to $K_2$. Clearly $\sqrt{3}$ is contained in both $K_1$ and
        $K_2$. Then since both $K_1$ and $K_2$ have degree 4, any proper
        subfield of either must have either degree 1 or 2. So since
        $K_1\neq K_2$, their intersection must be a proper subfield. And
        since their intersection contains a non-rational $\sqrt{3}$ which
        has minimal polynomial of degree 2, the intersection
        must have degree 2, and is exactly $\mathbb{Q}(\sqrt{3})$.
      \end{proof}

    \item Prove that $K_1$, $K_2$, and $K_1K_2$ are Galois over $F$ with
      $\text{Gal}(K_1K_2/F)$ the Klein-four group. Write out the elements
      of $\text{Gal}(K_1K_2/F)$ explicitly. Determine all the subgroups of
      the Galois group and give their corresponding fixed subfields of
      $K_1K_2$ containing $F$.

      \begin{proof}
        $K_1K_2$ is the splitting field of $x^4-2x^2-2$ since it contains
        all its roots. So $K_1K_2$ is Galois over $\mathbb{Q}$, and is
        therefore also Galois over the intermediate field $F$.
        Now since $K_1\supseteq F$,
        \[[K_1K_2:K_1] =[K_1(\alpha_2):K_1] \leq[F(\alpha_2):F]
        =[K_2:F]=2.\]
        Similarly, since $K_2\supseteq F$,
        \[[K_1K_2:K_2] =[K_2(\alpha_1):K_2] \leq[F(\alpha_1):F]
        =[K_1:F]=2.\]
        But $K_1K_2$ is a proper extension of $K_1$ and $K_2$ since
        $K_1\neq K_2$, therefore
        \[[K_1K_2:K_1] =[K_1K_2:K_2] =2.\]
        In particular, $[K_1K_2:F]=[K_1K_2:K_1][K_1:F]=2\times2=4$, so
        $\text{Gal}(K_1K_2/F)$ is either $\mathbb{Z}_4$ or
        $\mathbb{Z}_2^2$. Now since $K_1K_2$ is Galois over $F$, it must
        also be Galois over the intermediate fields $K_1$ and $K_2$. Also
        from Galois theory, the automorphisms that fix $K_1$ form a strict
        subgroup of the automorphisms that fix $F$. Then
        since $[K_1K_2:K_1]=2$, there must be an automorphism in
        $\text{Gal}(K_1K_2:K_1)$ of order 2. Similarly since
        $[K_1K_2:K_2]=2$, there must be an $\text{Gal}(K_1K_2:K_2)$ of
        order 2; this automorphism must be distinct from the previous one
        otherwise $\text{Gal}(K_1K_2:K_1) =\text{Gal}(K_1K_2:K_2)$, giving
        $K_1=K_2$. Thus $\text{Gal}(K_1K_2:F)$ is a group of order 4 which
        has two distinct automorphisms of order 2, so it must be isomorphic
        to the Klein-four group. \\

        Since $K_1K_2$ is the splitting field of $x^4-2x^2-2$ over $F$, its
        Galois group over $F$ is completely determined by its action on the
        $\alpha_i$'s. From earlier argument, $\text{Gal}(K_1K_2:F)$
        contains an element of order 2 in $\text{Gal}(K_1K_2:K_1)$ and
        another of order 2 in $\text{Gal}(K_1K_2:K_1)$. Now
        $K_1K_2=K_1(\alpha_2)$, and $\alpha_2$ has minimal polynomial
        $x^2-1+\sqrt{3}$ over $K_1$. Thus the non-trivial automorphism
        $\sigma_1$ in $\text{Gal}(K_1K_2:K_1)$ maps $\alpha_1$ to
        $\alpha_3$ and fixes $\alpha_2$ and $\alpha_4$, and has fixed field
        $K_1$. \\

        Swapping the roles of $K_1$ and $K_2$,
        we have another non-trivial automophism $\sigma_2$ that maps
        $\alpha_2$ to $\alpha_4$ and fixes $\alpha_1$ and $\alpha_3$,
        giving fixed field $K_2$. \\

        Then by the property of Klein-four groups, the last non-trivial
        element $\sigma_3$ of $\text{Gal}(K_1K_2:F)$ is the composite of the
        two previous automorphisms; this element would swap $\alpha_1$ with
        $\alpha_3$ and swap $\alpha_2$ with $\alpha_4$. Observe that since
        $\alpha_1\alpha_2=\sqrt{2}$, $\sigma$ would fix
        $\sqrt{2}$. Then since
        \[[K_1K_2:F(\sqrt{2})]=[F(\sqrt{2}):F]=2=|\sigma_3|,\]
        the fixed field of $\sigma$ must be $F(\sqrt{2})$. \\

        As a Klein-four group, the only non-trivial subgroups of
        $\text{Gal}(K_1K_2/F)=\{1,\sigma_1,\sigma_2,\sigma_3\}$ are of the
        form $\langle\sigma_i\rangle$. Their fixed subfields have been
        determined in the previous paragraphs. 
      \end{proof}
  \end{enumerate}
\end{document}
