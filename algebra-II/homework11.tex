\documentclass{article}
\usepackage[left=3cm,right=3cm,top=3cm,bottom=3cm]{geometry}
\usepackage{amsmath,amssymb,amsthm}
%\usepackage{pgfplots,tikz,tikz-cd}
\usepackage[inline]{enumitem}
\usepackage{color}
\setlength{\parindent}{0mm} %So that we do not indent on new paragraphs
\newcommand{\TODO}[1]{\textcolor{red}{TODO: #1}}

\begin{document}
\title{Graduate Algebra II: Homework 11}
\author{Li Ling Ko\\ lko@nd.edu}
\date{\today}
\maketitle

\it \textbf{DF 14.3.1:} Factor $x^8-x$ into irreducibles in $\mathbb{Z}[x]$
  and in $\mathbb{F}_2[x]$.

  \begin{proof}
    Over $\mathbb{Z}$, $x^8-x =x(x-1)(x^6+\ldots+x+1)$. Then since 1 nor -1
    are roots of the polynomial $x^6+\ldots+x+1$, this polynomial is
    irreducible over $\mathbb{Z}$ by Gauss's lemma. \\

    Over the $\mathbb{F}_2$, $x^8-x$ is the product of all the distinct
    irreducible polynomials in $\mathbb{F}_2[x]$ of degree 1 and 3, by
    Proposition 18. Therefore we only need to consider factors that have
    degree 1 or 3: 
    \begin{align*}
      x^8-x &=x(x-1)(x^6+\ldots+x+1)\\
      &=x(x-1)(x^3+x+1)(x^3+x^2+1).\\
    \end{align*}
  \end{proof}

\it \textbf{DF 14.3.3:} Prove that an algebraically closed field must be
  infinite.

  \begin{proof}
    If the field has characteristic 0, then it must contain a copy of
    $\mathbb{Z}$ inside it, therefore it must be infinite, even if it is
    not algebraically closed. So let $F$ be a finite field of
    characteristic $p$. Then $F$ must be $\mathbb{F}_{p^n}$ for some
    $n\in\mathbb{N}$. However, it will not have the contain all the roots
    of $x^{p^{2n}}-x$, as the splitting field of this polynomial is
    $\mathbb{F}_{p^{2n}}$ which is a proper superset of $\mathbb{F}_{p^n}$.
    Therefore $F$ cannot be algebraically closed.
  \end{proof}

\it \textbf{DF 14.3.7:} Prove that one of 2, 3, or 6 is a square in
  $\mathbb{F}_p$ for every prime $p$. Conclude that the polynomial
  \[x^6-11x^4+36x^2-36 =(x^2-2)(x^2-3)(x^2-6)\]
  has a root modulo $p$ for every prime $p$ but has no root in
  $\mathbb{Z}$.

  \begin{proof}
    Observe that since $\mathbb{F}_p^\times$ is a cyclic group, it has a
    generator $g\in\mathbb{F}_p$. Then an element is a square if and only
    if it equals $g^{2k}$ for some $k\in\mathbb{N}$. So if neither 2 nor 3
    are squares, then there must be some $m,n\in\mathbb{N}$ such that
    $g^{2m+1}=2$ and $g^{2n+1}=3$. Then $g^{2(m+n+1)}=6$, thus 6 will be
    a square with witness $g^{m+n+1}$. \\

    Therefore the given polynomial has a root modulo $p$ for every prime
    $p$. In $\mathbb{Z}$ however, since no factor or negation of a factor
    of 36 is a root of the polynomial, by Gauss's lemma the polynomial has
    no roots in $\mathbb{Z}$.
  \end{proof}

\it \textbf{DF 14.3.11:} Prove that $x^{p^n}-x+1$ is irreducible over
  $\mathbb{F}_p$ only when $n=1$ or $n=p=2$. [Note that if $\alpha$ is a
  root, then so is $\alpha+a$ for any $a\in\mathbb{F}_{p^n}$. Show that
  this implies $\mathbb{F}_p(\alpha)$ contains $\mathbb{F}_{p^n}$ and that
  $[\mathbb{F}_p(\alpha):\mathbb{F}_{p^n}]=p$.]

  \begin{proof}
    Following the hint, let $\alpha$ be a root of the polynomial. Then
    given any $a\in\mathbb{F}_{p^n}$,
    \begin{align*}
      (\alpha+a)^{p^n}-(\alpha+a)+1 &=\alpha^{p^n}+a^{p^n}-\alpha-a+1\\
      &=(\alpha^{p^n}-\alpha+1)+(a^{p^n}-a)\\
      &=a^{p^n}-a\\
      &=0,\\
    \end{align*}
    where the last equality follows from the fact that $\mathbb{F}_{p^n}$
    is exactly all the elements that satisfy the polynomial $x^{p^n}-x$.
    Therefore if $\alpha$ is a root of the given polynomial, then the $p^n$
    distinct roots of the polynomial are exactly
    $\{\alpha+a:a\in\mathbb{F}_{p^n}\}$. \\

    Now assume by contradiction that the given polynomial is irreducible
    but $n\neq1$, and either $n\neq p$ or $n\neq2$. Then
    $[\mathbb{F}_p(\alpha):\mathbb{F}_p]=p^n$, therefore
    $|\text{Gal}(\mathbb{F}_p(\alpha)/\mathbb{F}_p)|=p^n$, since field
    extensions of a finite field over another finite field are always
    Galois. Then since $\mathbb{F}_p(\alpha)$ is a simple field extension
    over $\mathbb{F}_p$, the automorphisms in
    $\text{Gal}(\mathbb{F}_p(\alpha)/\mathbb{F}_p)$ are exactly those that
    map $\alpha$ to the roots of the given polynomial that are contained in
    $\mathbb{F}_p(\alpha)$. Thus in order for
    $|\text{Gal}(\mathbb{F}_p(\alpha)/\mathbb{F}_p)|$ to equal $p^n$,
    $\mathbb{F}_p(\alpha)$ must contain all $p^n$ roots of the
    given polynomial. Then since the $p^n$ roots are exactly
    $\{\alpha+a:a\in\mathbb{F}_{p^n}\}$ as we have shown earlier,
    $\mathbb{F}_p(\alpha)$ must contain $\mathbb{F}_{p^n}$. \\

    Now since extensions involving finite fields are always Galois and the
    Galois group is always cyclic,
    $\sigma\in\text{Gal}(\mathbb{F}_p(\alpha)/\mathbb{F}_{p^n})$ must
    contain a generator $\sigma$. Then $\sigma$ must map
    $\alpha$ to $\alpha+a$ for some $a\in\mathbb{F}_{p^n}$, since
    $\text{Gal}(\mathbb{F}_p(\alpha)/\mathbb{F}_{p^n})$ is a subset of
    $\text{Gal}(\mathbb{F}_p(\alpha)/\mathbb{F}_{p})$. Then
    $\sigma^p(\alpha)=\alpha+pa=\alpha$ where the first equality follows
    from $\sigma$ fixing $\mathbb{F}_{p^n}$. Thus $\sigma^p$ is the
    identity because automorphisms in
    $\text{Gal}(\mathbb{F}_p(\alpha)/\mathbb{F}_p)$ are defined by their
    image on $\alpha$. Then since $\sigma$ generates
    $\text{Gal}(\mathbb{F}_p(\alpha)/\mathbb{F}_{p^n})$, this Galois group
    must have order $p$, so $[\mathbb{F}_p(\alpha):\mathbb{F}_{p^n}]
    =|\text{Gal}(\mathbb{F}_p(\alpha)/\mathbb{F}_{p^n})|=p$. But
    $\mathbb{F}_p\subseteq\mathbb{F}_{p^n}\subseteq\mathbb{F}_p(\alpha)$,
    so $[\mathbb{F}_{p}(\alpha):\mathbb{F}_p]=p^n$ must equal
    $[\mathbb{F}_{p^n}:\mathbb{F}_p]
    [\mathbb{F}_{p}(\alpha):\mathbb{F}_{p^n}]=np$. Yet $p^{n-1}$ rises much
    faster than $n$, so unless $n=1$ or
    $n=p=2$, we will have $p^n\gneq np$. More precisely, since $n\neq2$,
    $p^{n-1}$ always exceeds $2^{n-1}$, which will always exceed $n$ if
    $n>2$; thus $p^{n-1}\gneq n$ which implies $p^n\gneq np$. \\

    For the case $n=1$, we have shown that $x^p-x+1$ is irreducible in
    Problem 14.3.8. So consider the case $n=p=2$. The polynomial $x^4-x+1$
    has no roots in $\mathbb{F}_2$, so if it is reducible, it must be a
    product of two quadratic factors. Then we will be able to write
    \[x^4-x+1 =(x^2+ax+1)(x^2+bx+1)\]
    where $a,b\in\mathbb{F}_2$. Then comparing coefficients gives
    $a=b$, $ab=0$, and $a+b=1$, a contradiction. Thus $x^4-x+1$ must be
    irreducible.
  \end{proof}
\end{document}
