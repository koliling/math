\documentclass{article}
\usepackage[left=3cm,right=3cm,top=3cm,bottom=3cm]{geometry}
\usepackage{amsmath,amssymb,amsthm}
%\usepackage{pgfplots,tikz,tikz-cd}
\usepackage[inline]{enumitem}
\usepackage{color}
\setlength{\parindent}{0mm} %So that we do not indent on new paragraphs
\newcommand{\TODO}[1]{\textcolor{red}{TODO: #1}}

\begin{document}
\title{Graduate Algebra II: Homework 11}
\author{Li Ling Ko\\ lko@nd.edu}
\date{\today}
\maketitle

\it \textbf{DF 14.3.1:} Factor $x^8-x$ into irreducibles in $\mathbb{Z}[x]$
  and in $\mathbb{F}_2[x]$.

  \begin{proof}
    Over $\mathbb{Z}$, $x^8-x =x(x-1)(x^6+\ldots+x+1)$. Then since 1 nor -1
    are roots of the polynomial $x^6+\ldots+x+1$, this polynomial is
    irreducible over $\mathbb{Z}$ by Gauss's lemma. \\

    Over the $\mathbb{F}_2$, $x^8-x$ is the product of all the distinct
    irreducible polynomials in $\mathbb{F}_2[x]$ of degree 1 and 3, by
    Proposition 18. Therefore we only need to consider factors that have
    degree 1 or 3: 
    \begin{align*}
      x^8-x &=x(x-1)(x^6+\ldots+x+1)\\
      &=x(x-1)(x^3+x+1)(x^3+x^2+1).\\
    \end{align*}
  \end{proof}

\it \textbf{DF 14.3.3:} Prove that an algebraically closed field must be
  infinite.

  \begin{proof}
    If the field has characteristic 0, then it must contain a copy of
    $\mathbb{Z}$ inside it, therefore it must be infinite, even if it is
    not algebraically closed. So let $F$ be a finite field of
    characteristic $p$. Then $F$ must be $\mathbb{F}_{p^n}$ for some
    $n\in\mathbb{N}$. However, it will not have the contain all the roots
    of $x^{p^{2n}}-x$, as the splitting field of this polynomial is
    $\mathbb{F}_{p^{2n}}$ which is a proper superset of $\mathbb{F}_{p^n}$.
    Therefore $F$ cannot be algebraically closed.
  \end{proof}

\it \textbf{DF 14.3.7:} Prove that one of 2, 3, or 6 is a square in
  $\mathbb{F}_p$ for every prime $p$. Conclude that the polynomial
  \[x^6-11x^4+36x^2-36 =(x^2-2)(x^2-3)(x^2-6)\]
  has a root modulo $p$ for every prime $p$ but has no root in
  $\mathbb{Z}$.

  \begin{proof}
    Observe that since $\mathbb{F}_p^\times$ is a cyclic group, it has a
    generator $g\in\mathbb{F}_p$. Then an element is a square if and only
    if it equals $g^{2k}$ for some $k\in\mathbb{N}$. So if neither 2 nor 3
    are squares, then there must be some $m,n\in\mathbb{N}$ such that
    $g^{2m+1}=2$ and $g^{2n+1}=3$. Then $g^{2(m+n+1)}=6$, thus 6 will be
    a square with witness $g^{m+n+1}$. \\

    Therefore the given polynomial has a root modulo $p$ for every prime
    $p$. In $\mathbb{Z}$ however, since no factor or negation of a factor
    of 36 is a root of the polynomial, by Gauss's lemma the polynomial has
    no roots in $\mathbb{Z}$.
  \end{proof}

\it \textbf{DF 14.3.11:} Prove that $x^{p^n}-x+1$ is irreducible over
  $\mathbb{F}_p$ only when $n=1$ or $n=p=2$. [Note that if $\alpha$ is a
  root, then so is $\alpha+a$ for any $a\in\mathbb{F}_{p^n}$. Show that
  this implies $\mathbb{F}_p(\alpha)$ contains $\mathbb{F}_{p^n}$ and that
  $[\mathbb{F}_p(\alpha):\mathbb{F}_{p^n}]=p$.]

  \begin{proof}
    Following the hint, let $\alpha$ be a root of the polynomial. Then
    given any $a\in\mathbb{F}_{p^n}$,
    \begin{align*}
      (\alpha+a)^{p^n}-(\alpha+a)+1 &=\alpha^{p^n}+a^{p^n}-\alpha-a+1\\
      &=(\alpha^{p^n}-\alpha+1)+(a^{p^n}-a)\\
      &=a^{p^n}-a\\
      &=0,\\
    \end{align*}
    where the last equality follows from the fact that $\mathbb{F}_{p^n}$
    is exactly all the elements that satisfy the polynomial $x^{p^n}-x$.
  \end{proof}
\end{document}
