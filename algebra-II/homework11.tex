\documentclass{article}
\usepackage[left=3cm,right=3cm,top=3cm,bottom=3cm]{geometry}
\usepackage{amsmath,amssymb,amsthm}
%\usepackage{pgfplots,tikz,tikz-cd}
\usepackage[inline]{enumitem}
\usepackage{color}
\setlength{\parindent}{0mm} %So that we do not indent on new paragraphs
\newcommand{\TODO}[1]{\textcolor{red}{TODO: #1}}

\begin{document}
\title{Graduate Algebra II: Homework 11}
\author{Li Ling Ko\\ lko@nd.edu}
\date{\today}
\maketitle

\it \textbf{Q1:}
  \begin{enumerate}[label={(\alph*)}]
    \item Exhibit an explicit isomorphism between the splitting fields of
      $x^3-x+1$ and $x^3-x-1$ over $\mathbb{F}_3$.

      \begin{proof}
        Observe that we can get the first polynomial from the second, and
        also the second polynomial from the first by replacing $x$ with
        $-x$ and taking negations. Therefore the splitting fields of both
        polynomials are the same, so the identity map would be an
        isomorphism between the fields.
      \end{proof}

    \item How many possible isomorphisms are there total between these two
      splitting fields?

      \begin{proof}
        Since both splitting fields are equal, the isomorphisms are exactly
        the automorphisms of the splitting field of one of the polynomials.
        Observe that $x^3-x+1$ has no roots in $\mathbb{F}_3$, so its
        splitting field must properly extend $\mathbb{F}_3$. Also observe
        that if $\alpha$ is a root of the polynomial, then $\alpha+1$ is
        also a root, since $(\alpha+1)^3-(\alpha+1)+1
        =\alpha^3+1-\alpha-1+1=0$. Therefore the roots of the given
        polynomial are exactly $\alpha,\alpha+1,\alpha+2$, which implies
        that the splitting field is $\mathbb{F}_3(\alpha)$, which has degree
        extension 3. \\

        Now extensions involving finite fields are always Galois,
        so $|\text{Aut}(\mathbb{F}_3(\alpha))|
        =|\text{Gal}(\mathbb{F}_3(\alpha)/\mathbb{F}_3)|
        =[\mathbb{F}_3(\alpha):\mathbb{F}_3] =3$, so there are three
        possible isomorphisms.
      \end{proof}
  \end{enumerate}

\it \textbf{DF 14.3.1:} Factor $x^8-x$ into irreducibles in $\mathbb{Z}[x]$
  and in $\mathbb{F}_2[x]$.

  \begin{proof}
    Over $\mathbb{Z}$, $x^8-x =x(x-1)(x^6+\ldots+x+1)$. Then since 1 nor -1
    are roots of the polynomial $x^6+\ldots+x+1$, this polynomial is
    irreducible over $\mathbb{Z}$ by Gauss's lemma. \\

    Over the $\mathbb{F}_2$, $x^8-x$ is the product of all the distinct
    irreducible polynomials in $\mathbb{F}_2[x]$ of degree 1 and 3, by
    Proposition 18. Therefore the given polynomial must be a product of
    irreducible polynomials of degree 1 or 3:
    \begin{align*}
      x^8-x &=x(x-1)(x^6+\ldots+x+1)\\
      &=x(x-1)(x^3+x+1)(x^3+x^2+1).\\
    \end{align*}
    The two cubic factors are irreducible since they do not have roots in
    $\mathbb{F}_2$.
  \end{proof}

\it \textbf{DF 14.3.3:} Prove that an algebraically closed field must be
  infinite.

  \begin{proof}
    If the field has characteristic 0, then it must contain a copy of
    $\mathbb{Z}$ inside it, therefore it must be infinite, even if it is
    not algebraically closed. So let $F$ be a finite field of
    characteristic $p$. Then $F$ must be $\mathbb{F}_{p^n}$ for some
    $n\in\mathbb{N}$. However, it will not have the contain all the roots
    of $x^{p^{2n}}-x$, as the splitting field of this polynomial is
    $\mathbb{F}_{p^{2n}}$ which is a proper superset of $\mathbb{F}_{p^n}$.
    Therefore $F$ cannot be algebraically closed.
  \end{proof}

\it \textbf{DF 14.3.7:} Prove that one of 2, 3, or 6 is a square in
  $\mathbb{F}_p$ for every prime $p$. Conclude that the polynomial
  \[x^6-11x^4+36x^2-36 =(x^2-2)(x^2-3)(x^2-6)\]
  has a root modulo $p$ for every prime $p$ but has no root in
  $\mathbb{Z}$.

  \begin{proof}
    Observe that since $\mathbb{F}_p^\times$ is a cyclic group, it has a
    generator $g\in\mathbb{F}_p$. Then an element is a square if and only
    if it equals $g^{2k}$ for some $k\in\mathbb{N}$. So if neither 2 nor 3
    are squares, then there must be some $m,n\in\mathbb{N}$ such that
    $g^{2m+1}=2$ and $g^{2n+1}=3$. Then $g^{2(m+n+1)}=6$, thus 6 will be
    a square with witness $g^{m+n+1}$. \\

    Therefore the given polynomial has a root modulo $p$ for every prime
    $p$. In $\mathbb{Z}$ however, since no factor or negation of a factor
    of 36 is a root of the polynomial, by Gauss's lemma the polynomial has
    no roots in $\mathbb{Z}$.
  \end{proof}

\it \textbf{DF 14.3.8:} Determine the splitting field of the polynomial
  $x^p-x-a$ over $\mathbb{F}_p$ where $a\neq0$, $a\in\mathbb{F}_p$. Show
  explicitly that the Galois group is cyclic. [Show
  $\alpha\rightarrow\alpha+1$ is an automorphism.] Such an extension is
  called an Artin-Schreier extension.

  \begin{proof}
    Observe that if $\alpha$ is a root of the polynomial, then $\alpha+1$
    is also a root, because
    \[(\alpha+1)^p-(\alpha+1)-a =(\alpha^p-\alpha-a) +(1-1) =0.\]
    Therefore the $p$ distinct roots of the polynomial are $\{\alpha+u:
    u\in\mathbb{F}_p\}$, which means that $\mathbb{F}_p(\alpha)$ is the
    splitting field extension of the given polynomial. Now $\alpha$ cannot
    lie in $\mathbb{F}_p$ since the elements in $\mathbb{F}_p$ satisfy
    $x^p-x$ and $a\neq0$, so $\mathbb{F}_p(\alpha)$ must properly extend
    $\mathbb{F}_p$. \\

    In particular the minimal polynomial $m_\alpha(x)$ of $\alpha$ is a
    non-linear factor of $x^p-x-a$, and must have another root $\alpha+u$
    for some $u\in\mathbb{F}_p^\times$, and $\mathbb{F}_p(\alpha)$ is the
    splitting field of $m_\alpha(x)$. Now as a simple splitting field
    extension, the elements in
    $\text{Gal}(\mathbb{F}_p(\alpha)/\mathbb{F}_p)$ are defined by their
    image on $\alpha$, and are exactly those that map $\alpha$ to the roots
    in $m_\alpha(x)$. In particular,
    $\text{Gal}(\mathbb{F}_p(\alpha)/\mathbb{F}_p)$ will contain an
    automorphism $\sigma$ that maps $\alpha$ to $\alpha+u$. Then for each
    $k\in\mathbb{Z}_p$, $\sigma^k(\alpha)=\alpha+ku$, since $\sigma$ fixes
    $\mathbb{F}_p$. But $p$ is prime and $u\in\mathbb{F}_p$, so for every
    root $\alpha+v$ of $x^p-x-a$, there must be some $k\in\mathbb{Z}_p$
    where $ku=v$ such that $\sigma^k(\alpha)=\alpha+ku=\alpha+v$. 
    Thus $\sigma$ generates
    $\text{Gal}(\mathbb{F}_p(\alpha)/\mathbb{F}_p)$. \\

    Now $\sigma^p(\alpha)=\alpha+pu=\alpha$, so $\sigma$ and therefore
    $\text{Gal}(\mathbb{F}_p(\alpha)/\mathbb{F}_p)$ has order $p$. Then
    \[[\mathbb{F}_p(\alpha):\mathbb{F}_p]
    =|\text{Gal}(\mathbb{F}_p(\alpha)/\mathbb{F}_p)| =p,\]
    so the splitting field of $x^p-x-a$ is $\mathbb{F}_{p^p}$.
  \end{proof}

\it \textbf{DF 14.3.11:} Prove that $x^{p^n}-x+1$ is irreducible over
  $\mathbb{F}_p$ only when $n=1$ or $n=p=2$. [Note that if $\alpha$ is a
  root, then so is $\alpha+a$ for any $a\in\mathbb{F}_{p^n}$. Show that
  this implies $\mathbb{F}_p(\alpha)$ contains $\mathbb{F}_{p^n}$ and that
  $[\mathbb{F}_p(\alpha):\mathbb{F}_{p^n}]=p$.]

  \begin{proof}
    Following the hint, let $\alpha$ be a root of the polynomial. Then
    given any $a\in\mathbb{F}_{p^n}$,
    \begin{align*}
      (\alpha+a)^{p^n}-(\alpha+a)+1 &=\alpha^{p^n}+a^{p^n}-\alpha-a+1\\
      &=(\alpha^{p^n}-\alpha+1)+(a^{p^n}-a)\\
      &=a^{p^n}-a\\
      &=0,\\
    \end{align*}
    where the last equality follows from the fact that $\mathbb{F}_{p^n}$
    is exactly all the elements that satisfy the polynomial $x^{p^n}-x$.
    Therefore if $\alpha$ is a root of the given polynomial, then the $p^n$
    distinct roots of the polynomial are exactly
    $\{\alpha+a:a\in\mathbb{F}_{p^n}\}$. \\

    Now assume by contradiction that the given polynomial is irreducible
    but $n\neq1$, and either $n\neq p$ or $n\neq2$. Then
    $[\mathbb{F}_p(\alpha):\mathbb{F}_p]=p^n$, therefore
    $|\text{Gal}(\mathbb{F}_p(\alpha)/\mathbb{F}_p)|=p^n$, since field
    extensions of a finite field over another finite field are always
    Galois. Then since $\mathbb{F}_p(\alpha)$ is a simple field extension
    over $\mathbb{F}_p$, the automorphisms in
    $\text{Gal}(\mathbb{F}_p(\alpha)/\mathbb{F}_p)$ are exactly those that
    map $\alpha$ to the roots of the given polynomial that are contained in
    $\mathbb{F}_p(\alpha)$. Thus in order for
    $|\text{Gal}(\mathbb{F}_p(\alpha)/\mathbb{F}_p)|$ to equal $p^n$,
    $\mathbb{F}_p(\alpha)$ must contain all $p^n$ roots of the
    given polynomial. Then since the $p^n$ roots are exactly
    $\{\alpha+a:a\in\mathbb{F}_{p^n}\}$ as we have shown earlier,
    $\mathbb{F}_p(\alpha)$ must contain $\mathbb{F}_{p^n}$. \\

    Now since extensions involving finite fields are always Galois and the
    Galois group is always cyclic,
    $\sigma\in\text{Gal}(\mathbb{F}_p(\alpha)/\mathbb{F}_{p^n})$ must
    contain a generator $\sigma$. Then $\sigma$ must map
    $\alpha$ to $\alpha+a$ for some $a\in\mathbb{F}_{p^n}$, since
    $\text{Gal}(\mathbb{F}_p(\alpha)/\mathbb{F}_{p^n})$ is a subset of
    $\text{Gal}(\mathbb{F}_p(\alpha)/\mathbb{F}_{p})$. Then
    $\sigma^p(\alpha)=\alpha+pa=\alpha$ where the first equality follows
    from $\sigma$ fixing $\mathbb{F}_{p^n}$. Thus $\sigma^p$ is the
    identity because automorphisms in
    $\text{Gal}(\mathbb{F}_p(\alpha)/\mathbb{F}_p)$ are defined by their
    image on $\alpha$. Then since $\sigma$ generates
    $\text{Gal}(\mathbb{F}_p(\alpha)/\mathbb{F}_{p^n})$, this Galois group
    must have order $p$, so $[\mathbb{F}_p(\alpha):\mathbb{F}_{p^n}]
    =|\text{Gal}(\mathbb{F}_p(\alpha)/\mathbb{F}_{p^n})|=p$. But
    $\mathbb{F}_p\subseteq\mathbb{F}_{p^n}\subseteq\mathbb{F}_p(\alpha)$,
    so $[\mathbb{F}_{p}(\alpha):\mathbb{F}_p]=p^n$ must equal
    $[\mathbb{F}_{p^n}:\mathbb{F}_p]
    [\mathbb{F}_{p}(\alpha):\mathbb{F}_{p^n}]=np$. Yet $p^{n-1}$ rises much
    faster than $n$, so unless $n=1$ or
    $n=p=2$, we will have $p^n\gneq np$. More precisely, since $n\neq2$,
    $p^{n-1}$ always exceeds $2^{n-1}$, which will always exceed $n$ if
    $n>2$; thus $p^{n-1}\gneq n$ which implies $p^n\gneq np$. \\

    For the case $n=1$, we have shown that $x^p-x+1$ is irreducible in
    Problem 14.3.8. So consider the case $n=p=2$. The polynomial $x^4-x+1$
    has no roots in $\mathbb{F}_2$, so if it is reducible, it must be a
    product of two quadratic factors. Then we will be able to write
    \[x^4-x+1 =(x^2+ax+1)(x^2+bx+1)\]
    where $a,b\in\mathbb{F}_2$. Then comparing coefficients gives
    $a=b$, $ab=0$, and $a+b=1$, a contradiction. Thus $x^4-x+1$ must be
    irreducible.
  \end{proof}

\it \textbf{DF 14.4.1:} Determine the Galois closure of the field
  $\mathbb{Q}(\sqrt{1+\sqrt{2}})$ over $\mathbb{Q}$.

  \begin{proof}
    Denote $\alpha=\sqrt{1+\sqrt{2}}$. Then $\alpha$ satisfies the
    polynomial $x^4-2x^2-1$, which has no linear factors over $\mathbb{Q}$
    by Gauss's Lemma since $\pm1$ do not satisfy the polynomial. Also, the
    polynomial has no quadratic factors over $\mathbb{Q}$, because writing
    \[x^4-2x^2-1 =(x^2+ax+1)(x^2+bx-1)\]
    for some $a,b\in\mathbb{Q}$ and comparing coefficients gives the
    contradiction $a=-b$, $ab=-2$, and $a=b$. Therefore $x^4-2x^2-1$ is the
    minimal polynomial $m_\alpha(x)$ of $\alpha$. Now $m_\alpha(x)$ has
    four distinct roots $\pm\sqrt{1\pm\sqrt{2}}$, two of which are not
    real. Observe that $\sqrt{1+\sqrt{2}} \cdot\sqrt{1-\sqrt{2}}=i$,
    thus the Galois closure of the given field must contain $i$, and
    $\mathbb{Q}(\sqrt{1+\sqrt{2}},i)$ will contain all roots of
    $m_\alpha(x)$. Therefore $\mathbb{Q}(\sqrt{1+\sqrt{2}},i)$ is the
    splitting field extension of $x^4-2x^2-1$, and the Galois closure
    of the given field.
  \end{proof}

\it \textbf{DF 14.4.3:} Let $F$ be a field contained in the ring of
  $n\times n$ matrices over $\mathbb{Q}$. Prove that $[F:\mathbb{Q}]\leq
  n$. (Note that, by Exercise 19 of Section 13.2, the ring of $n\times n$
  matrices over $\mathbb{Q}$ does contain fields of degree $n$ over
  $\mathbb{Q}$.)

  \begin{proof}
    Since $F$ is contained in the ring of $n\times n$ matrices over
    $\mathbb{Q}$, $[F:\mathbb{Q}]$ is at most $n^2$. Then from the
    primitive element theorem, $F$ is a simple extension, so we can write
    $F=\mathbb{Q}(\alpha)$ for some $\alpha\in F$. Now the minimal
    polynomial of the matrix representing $\alpha$ is the same as
    $m_\alpha(x)$, the minimal polynomial of $\alpha$ over $\mathbb{Q}$.
    Since the minimal polynomial of the matrix has degree not exceeding
    $n$, therefore $[F:\mathbb{Q}] =\text{deg}(m_\alpha(x))\leq n$.
  \end{proof}

\it \textbf{DF 14.4.5:} Let $p$ be a prime and let $F$ be a field. Let $K$
  be a Galois extension of $F$ whose Galois group is a $p$-group (i.e., the
  degree $[K:F]$ is a power of $p$). Such an extension is called a
  $p$-extension (note that $p$-extensions are Galois by definition).

  \begin{enumerate}[label={(\alph*)}]
    \item Let $L$ be a $p$-extension of $K$. Prove that the Galois closure
      of $L$ over $F$ is a $p$-extension of $F$.

      \begin{proof}
        Let $M\supseteq L$ denote the Galois closure of $L$ over $F$. We
        prove the assertion by induction on $[M:L]$. If $[M:L]=1$, then
        $M=L$, so $L$ is already Galois over $F$ and with degree of
        extension $[L:K][K:F]$ which is a power of $p$. \\

        For the inductive step, assume $[M:L]>1$. Then $L/K$ is not Galois,
        so there must exist an automorphism $\sigma$ in $\text{Gal}(M/F)$
        where $\sigma(L)\neq L$, otherwise $L/K$ would be Galois. Then
        since $L/K$ is Galois, $\sigma(L)/\sigma(K)$ must also be Galois.
        But $K/F$ is Galois, so $\sigma(K)=K$, so we have $\sigma(L)/K$ is
        Galois. Then $\sigma(L)/K$ is Galois and $L/K$ is Galois, which
        implies that $\sigma(L)L/K$ is Galois. Furthermore,
        $[\sigma(L)L:K]$ must divide $[\sigma(L):K][L:K]$ from Corollary
        20, therefore $\sigma(L)L$ is a $p$-extension of $K$.
        Now $\sigma(L)L$ is contained in $M$ and contains $L$, thus $M$ is
        the Galois closure of $\sigma(L)L$ over $F$. Also
        $\sigma(L)L$ properly extends $L$ because $\sigma(L)\neq L$, so
        $[M:\sigma(L)L]<[M:L]$. Thus we have a $p$-extension $\sigma(L)L$
        of $K$ whose Galois closure $M$ over $F$ satisfies
        $[M:\sigma(L)L]<[M:L]$, so by induction hypothesis $M/\sigma(L)L$
        is a $p$-extension of $F$. Then $[M:L]
        =[M:\sigma(L)L][\sigma(L)L:L]$ will also be a power of $p$.
      \end{proof}
  \end{enumerate}
\end{document}
