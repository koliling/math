\documentclass{article}
\usepackage[left=3cm,right=3cm,top=3cm,bottom=3cm]{geometry}
\usepackage{amsmath,amssymb,amsthm,pgfplots,tikz}
\usepackage{color}
%\setlength{\parindent}{0mm}

\newcommand{\TODO}[1]{\textcolor{red}{TODO: #1}}

\begin{document}
\title{Graduate Algebra I: Homework 4}
\author{Li Ling Ko\\ lko@nd.edu}
\date{\today}
\maketitle

\begin{enumerate}
  \item Assume $G$ is an abelian group of order $2n$ where $n$ is odd, and
    contains two elements $a,b$ of order 2. Then the subgroup $H=\langle
    a,b\rangle=\{1,a,b,ab\}$ has four distinct elements (can verify that
    this group is closed under group operation, and also can use
    cancellation laws to verify that the elements are distinct since $a$
    and $b$ are distinct) and has order 4, which contradicts Lagrange's
    theorem that the order of a subgroup should divide the order of the
    group.

  \item
    \begin{enumerate}
      \item We use Proposition 13 of Section 3.2. Rearranging the formula
        given of Proposition 13, we get $|HK|/|H|=|K|/|H\cap K|=[K:H\cap
        K]$. Hence we need to show that $[\langle H\cup K\rangle:H]\geq
        |HK|/|H|$. To prove that, it would suffice to show that
        $HK\subseteq\langle H\cup K\rangle$. This is true from definition:
        $\langle H\cup K\rangle$ is the smallest subgroup containing all
        elements of $H$ and $K$, and since groups are closed under group
        action, $\langle H\cup K\rangle$ should contain all elements $hk$
        for any $h\in H$ and any $k\in K$, which means it should contain
        $HK$.

      \item We use Proposition 13. Assume by contradiction that there are
        two distinct subgroups $H_1,H_2\subset G$ of order $p$. Since
        groups of prime order are cyclic, $H_1=\langle a_1\rangle$ and
        $H_2=\langle a_2\rangle$ for some $a_1,a_2\in G$ of order $p$.
        Also, since the intersection of two groups is a subgroup of each
        group, and the order of a subgroup should divide the order of the
        group, $H_1\cap H_2$ can only have order $p$ or $1$. Since $H_1\neq
        H_2$, the order can only be 1. Then from Proposition 13,
        $|H_1H_2|=|H_1||H_2|/|H_1\cap H_2|=p\times p/1=p^2$, which is not
        possible because $G$ has only $pq<p^2$ elements.
    \end{enumerate}

  \item Section 3.2 Question 13: Label the vertices of a square in
    clockwise direction by 1, 2, 3, and 4. Since $D_8$ is generated by two
    distinct elements $r$ and $s$ where $r$ is a rotation by 90 degrees and
    $s$ is a flip, we can associate $r$ with the permutation
    $\sigma_r=(1,2,3,4)$ and $s$ with the permutation
    $\sigma_s=(1,2)(3,4)$. This would identify $D_8$ as a subgroup of
    $S_4$. \\

    From Proposition 13, $|D_8\langle 1,2,3\rangle|=|D_8||\langle
    1,2,3\rangle|/|D_8\cap\langle 1,2,3\rangle|$. Now $|D_8|=8$ and
    $|\langle 1,2,3\rangle|=3$, and since intersections of subgroups are
    also subgroups whose order divides the original two subgroups by
    Lagrange's theorem, $|D_8\cap\langle 1,2,3\rangle|$ must divide 3 and
    8, implying that $D_8\cap\langle 1,2,3\rangle$ can only be the trivial
    subgroup $\{1\}$. Then $|D_8\langle 1,2,3\rangle|=8\times 3/1=24$,
    implying that $D_8\langle 1,2,3\rangle$ is the whole of $S_4$. \\

    Assume by contradiction that $D_8$ contains a non-identity element $x$
    and $D_8\langle 1,2,3\rangle$ contains a non-identity element $y$ such
    that $x$ commutes with $y$. Now since the order of an element must
    divide its subgroup, $ord(x)$ can only be 2 or 4 since $D_8$ is not
    cyclic, and $ord(y)$ can only be 3. Then from commutativity of $x$ and
    $y$, the order of $xy$ is either 6 or 12. However, no element of $S_4$
    has order 12: each element of $S_4$ can be decomposed into disjoint
    cycles of order less than or equal 4, such that the order of the
    element is the product of the orders of each cycle, hence the elements
    of $S_4$ can only have orders 1, 2, 3, or 4. Thus, non-identity
    elements of $D_8$ and $\langle 1,2,3\rangle$ do not commute.

  \item Section 3.2 Question 14: We first show that $S_4$ has no normal
    subgroup of order 8. Assume that $N\trianglelefteq S_4$ has order 8.
    Then the quotient group $S_4/N$ has order 3, and is therefore cyclic.
    So given any $g\in S_4$ outside $N$, $S_4/N$ will be generated by $gN$
    of order 3, and we have $(gN)^3=N$. In particular, $g^3=1$, so the
    order of $g$ must be a multiple of 3. Now from argument in the previous
    question, elements of $S_4$ only have orders 1, 2, 3, or 4. Hence $g$
    must have order 3. This implies there should be $|S_4|-|N|=24-8=16$
    elements of order 3. However, there are only 8 elements in $S_4$ of
    order 3: Each element in $S_4$ is a composition of disjoint cycles such
    that elements 1 to 4 appear less than once in all the cycles, and the
    order of an element is the product of the orders of the disjoint
    cycles, hence the only elements of order 3 are the 3 cycles, and there
    exactly 8 of them. Hence $S_4$ cannot have a normal subgroup of order
    8. \\

    Assume we have a normal subgroup $N$ of order 3. Since groups of prime
    order are cyclic, $N$ must be generated by some $n\in S_4$ of order 3.
    From argument in the earlier paragraph, there are exactly 8 elements in
    $S_4$ of order 3, so let $m\in S_4$ be one of them that is not in $N$.
    Now since the order of elements in a group should divide the order of
    the group, we have $ord(mN)\mid|S_4/N|=|S_4|/|N|=24/3=8$, which
    implies $(mN)^8=m^8N=N$, and in particular that $m^8=1$, which
    contradicts that the order of $m$ is 3.

  \item Section 3.2 Question 16: The multiplicative group
    $(\mathbb{Z}/p\mathbb{Z})^\times$ has $p-1$ elements
    $\{\overline{1},\ldots,\overline{p-1}\}$. By Lagrange's theorem, the
    order of any element $\overline{a}$ should divide the order of the
    group $p-1$, which implies that $\overline{a}^{p-1}=\overline{1}$. In
    other words, given any $a\in\mathbb{Z}$, we have $a^{p-1} \equiv
    1\pmod{p}$, then multiplying by $a$ we have $a^p \equiv a\pmod{p}$ as
    required.

  \item Section 3.2 Question 18: Since $N$ is a normal subgroup of $G$,
    $HN$ will be a subgroup of $G$, and so by Lagrange's theorem, $|HN|$
    divides $|G|$. Now $|G|=|N||G:N|$, so we have $|HN|$ divides
    $|N||G:N|$. From Proposition 13, $|HN|=|H||N|/|H\cap N|$, so putting
    the two equations together we get $|H||N|/|H\cap N|$ divides
    $|N||G:N|$, then dividing by $|N|$ we get $|H|/|H\cap N|$ divides
    $|G:N|$. However, since $|H|$ and $|G:N|$ are relatively prime, we must
    have $|H|/|H\cap N|=1$, which means $|H|=|H\cap N|$. Then since all the
    groups involved are finite, this implies $H=H\cap N$, which means $H$
    is contained in $N$. Finally, since $H$ is a group, we have $H\leq N$,
    as we are required to show.

  \item Section 3.2 Question 22: Given any
    $a\in(\mathbb{Z}/n\mathbb{Z})^\times$, the cyclic group $\langle
    a\rangle$ generated by $a$ is a subgroup of
    $(\mathbb{Z}/n\mathbb{Z})^\times$, and so by Lagrange's theorem, must
    have order that divides $|(\mathbb{Z}/n\mathbb{Z})^\times|=\varphi(n)$.
    Therefore, $a^{\varphi(n)}\equiv 1\pmod{n}$.

  \item Section 3.2 Question 23: To get the last two digits we consider
    $3^{3^{100}}\pmod{100}$, and apply results from the previous question
    with $n=100$. By Euler's product formula,
    $\varphi(100)=100(1-1/2)(1-1/5)=40$. From the previous question, we
    have $3^{40}\equiv1\pmod{100}$, so
    $3^{3^{100}}\equiv3^{3^{100}\pmod{40}}\pmod{100}$. Hence we first
    compute $3^{100}\pmod{40}$, and we do so by applying the results from
    the previous question again. By Euler's product formula,
    $\varphi(40)=40(1-1/2)(1-1/5)=16$. So
    $3^{100}\equiv3^{100\pmod{16}}\equiv3^4\equiv81\equiv1\pmod{40}$. Hence
    $3^{3^{100}}\equiv3^{1}\equiv3\pmod{100}$. So the last two digits is
    03.

  \item Section 3.3 Question 3: Let $H$ be a normal subgroup of $G$ of
    prime index $p$ and $K$ be a subgroup of $G$. Since $H$ is normal, $HK$
    is a subgroup of $G$, and so by Lagrange's theorem, $|HK|$ should
    divide $|G|$. Now $|G|=|H|p$, and
    $|HK|=|H||K|/|H\cap K|$ by Proposition 13 of Section 3.2. So we have
    $|H||K|/|H\cap K|$ should divide $|H|p$, and after dividing by $|H|$ on
    both sides, gives us $|K|/|H\cap K|$ should divide $p$. Since $p$ is
    prime, $|K|/|H\cap K|$ can only be $p$ or 1. Note that since
    intersections of groups are subgroups of the groups, we have $H\cap K$
    is a subgroup of $K$, so by Lagrange's theorem, $|H\cap K|$ should
    divide $K$. If it is the case that $|K|/|H\cap K|=p$, then $|K:K\cap
    H|=p$, and also $|HK|=|H||K|/|H\cap K|=|H|p=|G|$, implying that $G=HK$.
    In the second case where $|K|/|H\cap K|=1$, then since $H\cap K\leq K$,
    this implies $H\cap K=K$, which means $K\subseteq H$. Then since $K$
    and $H$ are subgroups, this means $K\leq H$ as we are required to show.

  \item Section 3.3 Question 4: Consider the map $\phi:A\times
    B\rightarrow(A/C)\times(B/D)$ defined by $(a,b)\mapsto(aC,bD)$. We show
    that $\phi$ is a surjective homomorphism with $\ker(\phi)=C\times D$,
    which would imply that $(C\times D)\trianglelefteq(A\times B)$ and
    $(A\times B)/(C\times D)\cong(A/C)\times(B/D)$, as we are required to
    show. \\

    We first show $\phi$ is a homomorphism. Given $(a_1,b_1),(a_2,b_2)\in
    A\times B$, we have
    \begin{align*}
      \phi((a_1,b_1)(a_2,b_2))  &= \phi((a_1a_2,b_1b_2))  \\
                                &= (a_1a_2C,b_1b_2D)      \\
                                &= (a_1Ca_2C,b_1Db_2D)    \\
                                &= (a_1C,b_1D)(a_2C,b_2D) \\
                                &= \phi((a_1,b_1))\phi((a_2,b_2)),  \\
    \end{align*}
    and so $\phi$ is a homomorphism. $\phi$ is also surjective since every
    $(aC,bD)\in(A/C)\times(B/D)$ has pre-image $(a,b)$. It remains to show
    that $\ker(\phi)=C\times D$. If $\phi((a,b))=(aC,bD)=(1_AC,1_BD)$, then
    $aC=1_AC$ and $bD=1_BD$, which implies $a\in C$ and $b\in D$, hence
    $\ker(\phi)\supseteq C\times D$. Also, if $a\in C$ and $b\in D$, then
    $\phi((a,b))=(aC,bD)=(1_AC,1_BD)$, so $\ker(\phi)\subseteq C\times D$.
    Hence $\ker(\phi)=C\times D$, which completes our proof.

  \item Section 3.3 Question 7: Consider the map $\phi:G
    \rightarrow(G/M)\times(G/N)$ defined by $g\mapsto(gM,gN)$. We show
    that $\phi$ is a surjective homomorphism with $\ker(\phi)=M\cap N$,
    which would imply that $(M\cap N)\trianglelefteq G$ and $G/(M\cap
    N)\cong(G/M)\times(G/N)$, as we are required to show. \\

    We first show $\phi$ is a homomorphism. Given $g_1,g_2\in G$, we have
    \begin{align*}
      \phi(g_1g_2)  &= (g_1g_2M,g_1g_2N)      \\
                    &= (g_1Mg_2M,g_1Ng_2N)    \\
                    &= (g_1M,g_1N)(g_2M,g_2N) \\
                    &= \phi(g_1)\phi(g_2),    \\
    \end{align*}
    and so $\phi$ is a homomorphism. $\phi$ is also surjective since every
    $(gM,gN)\in(G/M)\times(G/N)$ has pre-image $g$. It remains to show
    that $\ker(\phi)=M\cap N$. If $\phi(g)=(gM,gN)=(1_gM,1_gN)$, then
    $gM=1_gM$ and $gN=1_gN$, which implies $g\in M$ and $g\in N$, which is
    the same as $g\in M\cap N$. Hence
    $\ker(\phi)\supseteq M\cap N$. Also, if $g\in M\cap N$, then
    $\phi(g)=(gM,gN)=(1_gM,1_gN)$, so $\ker(\phi)\subseteq M\cap N$.
    Hence $\ker(\phi)=M\cap N$, which completes our proof.

  \item Section 3.3 Question 9: Since $N$ is normal in $G$, $PN$ will be a
    subgroup of $G$, and so by Lagrange's theorem, $|PN|$ divides
    $|G|=p^am$. Also, by Proposition 13 of Section 3.2,
    $|PN|=|P||N|/|P\cap N|=p^a\cdot p^bn/|P\cap N|$. Putting these
    equations together and dividing by $p^a$, we get $p^bn/|P\cap N|$
    divides $m$. Since $p$ does not divide $m$, $p^b$ must divide $|P\cap
    N|$. Also, $P\cap N$ is a subgroup of $N$ and of $P$, so by Lagrange's
    theorem, $|P\cap N|$ must divide both $|N|=p^bn$ and $|P|=p^a$. $|P\cap
    N|$ divides $p^a$ means $|P\cap N|$ must be a power of $p$, and then
    $|P\cap N|$ divides $p^bn$ where $(p,n)=1$ means that $|P\cap N|$ must
    divide $p^b$. Then since $p^b$ divides $|P\cap N|$, we have $|P\cap N|$
    equals $p^b$. \\

    It remains to show that $|PN/N|=p^{a-b}$. First, note from the Second
    Isomorphism theorem that because $N$ is normal in $G$, $N$ will also be
    normal in $PN$, so $PN/N$ makes sense. Then from Proposition 12 of
    Section 3.2, $|PN|=|P||N|/|P\cap N|=p^a\cdot p^bn/p^b=p^an$, and so
    $|PN/N|=|PN|/|N|=p^an/(p^bn)=p^{a-b}$, as we need to show.

  \item Section 3.4 Question 2: The 3 composition series for $Q_8$ are:
    \begin{align*}
      1 \triangleleft \langle-1\rangle \triangleleft \langle i\rangle
      \triangleleft Q_8 \\
      1 \triangleleft \langle-1\rangle \triangleleft \langle j\rangle
      \triangleleft Q_8 \\
      1 \triangleleft \langle-1\rangle \triangleleft \langle k\rangle
      \triangleleft Q_8 \\
    \end{align*}
    The composition factors are all $\mathbb{Z}_2$. \\

    By using the lattice of $D_8$, the 7 composition series of $D_8$ are:
    \begin{align*}
      1 \triangleleft \langle s\rangle
        \triangleleft \langle s,r^2\rangle
        \triangleleft D_8 \\
      1 \triangleleft \langle r^2s\rangle
        \triangleleft \langle s,r^2\rangle
        \triangleleft D_8 \\
      1 \triangleleft \langle r^2\rangle
        \triangleleft \langle r\rangle
        \triangleleft D_8 \\
      1 \triangleleft \langle r^2\rangle
        \triangleleft \langle sr^2\rangle
        \triangleleft D_8 \\
      1 \triangleleft \langle rs\rangle
        \triangleleft \langle rs,r^2\rangle
        \triangleleft D_8 \\
      1 \triangleleft \langle r^3s\rangle
        \triangleleft \langle rs,r^2\rangle
        \triangleleft D_8 \\
      1 \triangleleft \langle r^2\rangle
        \triangleleft \langle rs,r^2\rangle
        \triangleleft D_8 \\
    \end{align*}
    The composition factors are all $\mathbb{Z}_2$.

  \item Section 3.4 Question 8i-iii: (iii)$\rightarrow$(ii): This follows
    directly from the fact that groups of prime order are cyclic. \\

    (ii)$\rightarrow$(i): This follows directly from the fact that cyclic
    groups are abelian. \\

    (i)$\rightarrow$(iii): Let $G$ be solvable. Then there is a chain of
    subgroups $1=G_0\triangleleft\ldots\triangleleft G_s=G$ such that
    $G_{i+1}/G_i$ is abelian for $i=0,\ldots,s-1$. For a given $i$,
    $G_{i+1}/G_i$ is finite since $G$ is finite, so by the Jordan-Holder
    theorem, $G_{i+1}/G_i$ has a composition series. Now the composition
    series $1=H_0\triangleleft\ldots\triangleleft H_r=H$ of a finite
    abelian group $H$ must have prime composition factors: Otherwise
    suppose $|H_{j+1}/H_j|$ is composite, then by Proposition 21 of Section
    3.4, since $H_{j+1}/H_j$ is finite abelian, it contains an element
    $\bar{h}$ of order $p$, where $p$ divides $|H_{j+1}/H_j|$. Then
    $\langle\bar{h}\rangle\lneq H_{j+1}/H_j$, and is a normal subgroup of
    $H_{j+1}/H_j$ because all abelian subgroups are normal, which
    contradicts the simpleness of $H_{j+1}/H_j$. \\

    For each $i=0,\ldots,s-1$, let $1=N_0\triangleleft\ldots\triangleleft
    N_{r_i}=G_{i+1}/G_i$ be a composition series of $G_{i+1}/G_i$. We just
    showed that the composition factors of this series have prime order.
    Now by the Lattice Isomorphism Theorem, for each $j=0,\ldots,r_i$,
    there are subgroups $H_{i,j}\leq G_{i+1}$ and containing $G_i$ such that
    $H_{i,j}/G_i=N_j$. Then by the Third Isomorphism Theorem,
    \begin{equation*}
      H_{i,j+1}/H_{i,j} \cong (H_{i,j+1}/G_i)/(H_{i,j}/G_i) = N_{j+1}/N_j,
    \end{equation*}
    implying that $H_{i,j+1}/H_{i,j}$ has prime order. Then $G$ has the
    following composition series where all composition factors are of prime
    order:
    \begin{align*}
      1         &=H_{0,0}\triangleleft\ldots\triangleleft
                  H_{0,r_0-1}\triangleleft  \\
      H_{0,r_0} &=H_{1,0}\triangleleft\ldots\triangleleft
                  H_{1,r_1-1}\triangleleft\ldots  \\
      H_{s-1,r_{s-1}} &=H_{s-1,0}\triangleleft\ldots\triangleleft
                  H_{s-1,r_{s-1}}=G \\
    \end{align*}

  \item Section 3.4 Question 11: Do not know how to do.
    %Let $G$ be a solvable group with
    %non-trivial normal subgroup $N$. Let
    %$1=N_0\triangleleft\ldots\triangleleft N_s=G$ be a chain of subgroups
    %that witness the solvability of $G$. Consider the intersection of this
    %chain with $N$, i.e. consider the series $1=N_0\cap
    %N\subseteq\ldots\subseteq N_s\cap N=N$. We show that this series
    %witnesses the solvability of $N$. Given $i=0,\ldots,s-1$, consider
    %$N\cap N_{i+1}$ as a subgroup of $N_{i+1}$, which has a normal
    %subgroup $N_i$. From the Second Isomorphism Theorem, $H\cap
    %N_i=(H\cap N_{i+1})\cap N_i\trianglelefteq H\cap N_{i+1}$. Also from
    %the Second Isomorphism Theorem, $(H\cap N_{i+1})/(H\cap N_i)\cong
    %HN_i/N_i$. Now from the Fourth Isomorphism Theorem, $HN_i/N_i$ is
    %a subgroup of $N_{i+1}/N_i$, which is abelian. Hence $HN_i/N_i$ is also
    %abelian, which implies that $(H\cap N_{i+1})/(H\cap N_i)$ is abelian.
    %Hence $1=N_0\cap N\subseteq\ldots\subseteq N_s\cap N=N$ witnesses the
    %solvability of $N$. \\

    %Let $k\in\{1,\ldots,s\}$ be the smallest integer such that $N_k\cap
    %N=N$. Such a $k$ must exist since $N_s=N$. Let $A=N_{k-1}$. Then
    %$A=N_{k-1}\triangleleft N_k=N$, and by choice of $k$, $A$ is a
    %non-trivial subgroup of $N$. Also,

  \item Section 3.4 Question 12: (i)$\rightarrow$(ii): Assume (i) is true,
    and let $G$ be a simple group of odd order. We want to show that $G$
    has prime order. By assumption, $G$ is solvable. However, because $G$
    is simple, it cannot have non-trivial sub-groups, hence the chain of
    subgroups that witness the solvability of $G$ can only be the trivial
    chain $1\trianglelefteq G$. This implies that $G/1\cong G$ is abelian.
    Hence $G$ is a simple abelian group. If $|G|$ is composite with prime
    divisor $p$, then by Proposition 21 of Section 3.4, there is an element
    $g\in G$ of order $p$. Then $\langle g\rangle$ will be a strict subgroup
    of $G$, and by abelianess of $G$, $\langle g\rangle$ will be a
    non-trivial normal subgroup of $G$, which contradicts the simpleness of
    $G$. Hence $|G|$ must be prime, which completes the proof. \\

    (ii)$\rightarrow$(i): Assume (ii) is true, and let $G$ be a group of
    odd order. By the Jordan-Holder theorem, since $G$ is finite, it has a
    composition series $1=N_0\triangleleft\ldots\triangleleft N_s=G$. Now
    for each $i=0,\ldots,s$, $N_i$ is subgroup of $G$, and must have odd
    order since $G$ has odd order and the order of subgroups must divide
    the order of the group. Therefore, $|N_{i+1}/N_{i}|=|N_{i+1}|/|N_i|$
    must also be odd. Since the quotient groups $N_{i+1}/N_{i}$ are from
    the composition series, they must also be simple, so by assumption,
    they must have prime order. Then, since groups of prime order are
    cyclic and hence abelian, each $N_{i+1}/N_{i}$ is abelian, and so the
    composition series is also a chain of subgroups that witness the
    solvability of $G$, which completes the proof.
\end{enumerate}
\end{document}
