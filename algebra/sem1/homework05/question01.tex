Show that $A_n$ is the only subgroup of $S_n$ of index 2.
\begin{proof}
  We first verify that $A_n$ has index 2. Let
  $\text{sgn}:S_n\rightarrow\mathbb{Z}_2$ be the surjective homomorphism
  that sends permutations to their parity. Then $\ker(\text{sgn})=A_n$,
  so $[S_n:A_n]=|\text{Im}(\text{sgn})|=|\mathbb{Z}_2|=2$, which
  verifies that $A_n$ is a subgroup of $S_n$ of index 2. \\

  Let $N$ be a subgroup of index 2. Then $N$ is a normal subgroup of
  $S_n$, because its non-trivial left coset $gN$ equals $S_n-N$ which
  also equals its non-trivial right coset $Ng$. Hence from the second
  isomorphism theorem, $N\cap A_n$ will be normal in $A_n$. \\

  If $n\geq 5$, $N\cap A_n$ must be trivial because $A_n$ is simple.
  Hence either $N=A_n$ as required, or $N\cap A_n=\{1\}$. Now if $N\cap
  A_n=\{1\}$, then $N$ must contain all odd permutations of $S_n$ except
  for one, because $S_n$ is finite, both $A_n$ and $N$ have index 2, and
  $A_n$ contains exactly all even permutations of $S_n$. Hence of the
  three odd permutations $(12)$, $(13)$, and $(14)$, $N$ must contain
  at least two of them, and therefore contain the their composition,
  which would be a non-identity even permutation, a contradiction. \\

  Hence we can assume $n\leq4$. If $n=2$, then $A_2=\{1\}$ and the
  statement holds trivially. \\

  If $n=3$, then $A_3=\langle(123)\rangle$, which has index 2 in $S_3$.
  Let $N$ be a subgroup of $S_3$ of index 2. Then $N$ must have only
  order 3. Now the only non-identity permutations in $S_3$ that are even
  are the transpositions, which have order 2. No transposition
  $\sigma_t$ can be contained in a group of order 3 because otherwise
  $\langle\sigma_t\rangle=\{1,\sigma_t\}$ will be a subgroup,
  contradicting Lagrange's theorem. Hence $N$ cannot contain
  transpositions, and can only be the subgroup generated by $(123)$ if
  it is to contain three elements. \\

  Finally, if $n=4$, let $N\subset S_4$ be a subgroup of index 2 that is
  different from $A_4$. Then $N$ must contain an odd permutation. Now from
  Proposition 25 of Section 3.5, odd permutations are exactly those with an
  odd number of even cycles in its cycle decomposition. Hence in $S_4$, the
  only odd permutations are the transpositions and the four cycles. Now if
  $N$ contains a transposition, then from Proposition 11 of Section 4.3 and
  from the normality of $N$ in $S_4$, $N$ must contain all transpositions
  of $S_4$. However transpositions generate the entire group $S_4$ because
  all permutations can be decomposed into a finite product of
  transpositions, which would imply that $N$ is the whole of $S_4$, a
  contradiction. Hence $N$ cannot contain any transposition. Since $N$ must
  contain at least one odd permutation, it must contain a four-cycle, which
  means it contains all three four-cycles of $S_4$. Then because the only
  odd permutations of $S_4$ are transpositions and the three four-cycles,
  $N$ must contain exactly three four-cycles and nine even permutations.
  Thus $|N\cap A_4|=9$. But $N\cap A_4$ is a subgroup of $A_4$, so by
  Lagrange's theorem $|N\cap A_4|=9$ must divide $|A_4|=12$, a
  contradiction.

  %Finally, if $n=4$, let $H=N\cap A_4$. Then $H$ is a subgroup of $A_4$,
  %and so by Lagrange's theorem, must have order that divides $|A_4|=12$.
  %If $N\neq A_4$, then $H$ can only have orders 1, 2, 3, 4, or 6. In
  %other words, $N$ has no more than 6 even permutations, and must have
  %6 or more odd permutations. However, the product of two odd
  %permutations is an even permutation, so fixing any odd permutation
  %$\sigma$ in $N$ and applying left group action on the 6 or more odd
  %permutations in $N$ will give us at least 6 distinct even permutations
  %in $N$. Hence $N$ must contain exactly 6 even permutations. Then $A_4$
  %would have a subgroup $H$ of order 6, a contradiction to Section 3.5
  %Question 14.
\end{proof}
