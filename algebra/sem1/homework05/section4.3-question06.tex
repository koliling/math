Section 4.3 Question 6:
Assume $G$ is a non-abelian group of order 15. Prove that $Z(G)=1$.

\begin{proof}
  Let $G$ be non-abelian of order 15, and assume by contradiction that
  $Z(G)\neq 1$. \\

  The class equation of a group is given by
  \begin{equation*}
    |G| = |Z(G)| + \sum_{i=1}^r|\kappa(g_i)|,
  \end{equation*}
  where $\kappa(g_i)$ are the disjoint conjugacy classes of $G$, and
  $|\kappa(g_i)|=|G:C_G(g_i)|$. By Lagrange's theorem, $|Z(G)|$ and each
  $|\kappa(g_i)|$ should divide $|G|$. Also by definition,
  $|\kappa(g_i)|\neq 1$ otherwise $g_i\in Z(G)$. Furthermore, $|Z(G)|\geq
  1$ since $1\in Z(G)$, which also implies $|\kappa(g_i)|\neq|G|$. Also,
  $|Z(G)|\neq|G|$ since $G$ is abelian. \\

  So since the factors of $|G|=15$ are 1, 3, 5, or 15, from the above
  observations, each $|Z(G)|$ and $|\kappa(g_i)|$ can only be 3 or 5. \\

  If $|Z(G)|=3$, then $\sum_{i=1}^r|\kappa(g_i)|=15-3=12$. Now the only
  non-negative integer solutions for $3a+5b=12$ is $a=4$ and $b=0$, hence
  each $\kappa(g_i)$ must have 3 elements. Then
  $|C_G(g_i)|=|G|/|\kappa(g_i)|=15/3=5$. However $Z(G)$ is a subgroup of
  $C_G(g_i)$, so by Lagrange's theorem $|Z(G)=3$ should divide
  $|C_G(g_i)|=5$, a contradiction. \\

  If $|Z(G)|=5$, then $\sum_{i=1}^r|\kappa(g_i)|=15-5=10$. Now the only
  non-negative integer solutions for $3a+5b=10$ is $a=0$ and $b=2$, hence
  each $\kappa(g_i)$ must have 5 elements. Then
  $|C_G(g_i)|=|G|/|\kappa(g_i)|=15/5=3$. However $Z(G)$ is a subgroup of
  $C_G(g_i)$, so by Lagrange's theorem $|Z(G)=5$ should divide
  $|C_G(g_i)|=3$, again a contradiction. Thus $|Z(G)|$ can only be 1. \\
\end{proof}
