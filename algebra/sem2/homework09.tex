\documentclass{article}
\usepackage[left=3cm,right=3cm,top=3cm,bottom=3cm]{geometry}
\usepackage{amsmath,amssymb,amsthm}
%\usepackage{pgfplots,tikz,tikz-cd}
\usepackage[inline]{enumitem}
\usepackage{color}
\setlength{\parindent}{0mm} %So that we do not indent on new paragraphs
\newcommand{\TODO}[1]{\textcolor{red}{TODO: #1}}

\begin{document}
\title{Graduate Algebra II: Homework 9}
\author{Li Ling Ko\\ lko@nd.edu}
\date{\today}
\maketitle

\it \textbf{DF 13.6.3:} Prove that if a field contains the $n$-th roots of
  unity for $n$ odd then it also contains the $2n$-th roots of unity.

  \begin{proof}
    We first show that for $n$ odd,
    \[\mu_{2n} =\mu_n \cup \{-\zeta:\zeta\in\mu_n\}.\]
    Clearly $\supseteq$ holds since $x^n=1$ implies $x^{2n}=1=(-x)^{2n}$.
    \\

    For $\subseteq$, observe that $x^{2n}-1=(x^n-1)(x^n+1)$. Hence if $a$
    is an $2n$-th root of unity, then it either satisfies $(x^n-1)$, or it
    satisfies $(x^n+1)$. In the former case, $a$ would be an $n$-th root of
    unity. In the latter case, $-a$ would satisfy $(x^n-1)$, so $-a$ is an
    $n$-th root of unity, which implies that $a$ is the negation of some
    $n$-th root of unity. \\

    Then since fields are closed under negation,
    $\mathbb{Q}(\mu_n)=\mathbb{Q}(\mu_{2n})$ as required.
  \end{proof}

\it \textbf{DF 13.6.4:} Prove that if $n=p^km$ where $p$ is a prime and $m$
  is relatively prime to $p$ then there are precisely $m$ distinct $n$-th
  roots of unity over a field of characteristic $p$.

  \begin{proof}
    Over a field of characteristic $p$, by Proposition 35,
    \[x^{p^km}-1 =(x^m-1)^{p^k},\]
    so there are at most $m$ distinct $m$-th roots of unity. It suffices to
    show that the roots of $x^m-1$ are distinct. Differentiating with
    respect to $x$ gives $D_x(x^m-1)=mx^{m-1}$, which is non-zero since $p$
    does not divide $m$. Therefore for $m>1$, $x^m-1$ and its derivative
    are coprime, so the roots of $x^m-1$ are distinct by Proposition 33.
  \end{proof}

\it \textbf{DF 13.6.5:} Prove that there are only a finite number of roots
  of unity in any finite extension $K$ of $\mathbb{Q}$.

  \begin{proof}
    If $\zeta_n$ lies in $K$, then the degree $\varphi(n)$ of its cyclotomic
    polynomial $\Phi_n(x)$ cannot be larger than $[K:\mathbb{Q}]$.
    Therefore it suffices to that given any $m\in\mathbb{N}$, there can
    only be a finite number of $m\in\mathbb{N}$ such that $\varphi(m)=n$.
    \\

    To that end, fix $m\in\mathbb{N}$. Let $p_1<\ldots<p_m$ be the
    first $m$ primes. For each prime $p_i$, let $k_i\in\mathbb{N}$ be the
    smallest positive integer such that $p_i^{k_i}-p_i^{k_i-1}>m$. Let $n$
    be any positive integer greater than
    \[N:=(p_1^{k_1}-p_1^{k_1-1}) \cdots (p_m^{k_m}-p_m^{k_m-1}).\]

    Then either $n$ has a prime factor $p$ smaller than $p_m$ such that
    $p^{k}|n$ for some $k>k_r-1$, or $n$ has at least one prime factor $q$
    greater than $p_m$. In the former case, $p^k-p^{k-1}$ divides
    $\varphi(n)$, so $\varphi$ will be greater than $m$. In the latter
    case, the prime factor $q$ that is greater than $p_m$ must also be
    greater than $m$, since $p_m>m$. But $q^k-q^{k-1}$ divides $\varphi(n)$
    for some $k\geq1$, so $\varphi$ is also greater than $m$. Thus all
    integers greater than $N$ will have Euler-totient value greater than
    $m$.
  \end{proof}

\it \textbf{DF 13.6.6:} Prove that for $n$ odd, $\Phi_{2n}(x)=\Phi_n(-x)$.
  \begin{proof}
    In Exercise 13.6.3, we proved that when $n$ is odd,
    \[\mu_{2n} =\mu_n \cup \{-\zeta:\zeta\in\mu_n\}.\]

    Therefore the $2n$-th roots of unity either satisfy $\Phi_n(x)$ or
    $\Phi_n(-x)$. Note that $\Phi_n(-x)$ is a monic polynomial because
    $\text{deg}(\Phi_n(x)) =\varphi(n)$ which is even because $n$ is odd.
    Now since $\Phi_n(x)$ is irreducible over $\mathbb{Q}$, $\Phi_n(-x)$
    must also be irreducible. Therefore $\Phi_{2n}(x)$ is either
    $\Phi_n(x)$ or $\Phi_n(-x)$. Now the roots of $\Phi_n(x)$ cannot
    contain the primitive $2n$-th root of unity since it contains exactly
    the $n$-th roots of unity, therefore the primitive $2n$-th root of
    unity must satisfy the other polynomial $\Phi_n(-x)$. Therefore
    $\Phi_{2n}(x)=\Phi_n(x)$.
  \end{proof}

\it \textbf{DF 13.6.8:} Let $l$ be a prime and let $\Phi_l(x)
  =\frac{x^l-1}{x-1} =x^{l-1}+x^{l-2}+\ldots+x+1 \in\mathbb{Z}[x]$ be the
  $l$-th cyclotomic polynomial, which is irreducible over $\mathbb{Z}$ by
  Theorem 41. This exercise determines the factorization of $\Phi_l(x)$
  modulo $p$ for any prime $p$. Let $\zeta$ denote any fixed primitive $l$-th
  root of unity.

  \begin{enumerate}[label={(\alph*)}]
    \item Show that if $p=l$ then $\Phi_l(x)=(x-1)^{l-1}
      \in\mathbb{F}_l[x]$.

      \begin{proof}
        Over $\mathbb{Q}$ and therefore also over $\mathbb{F}_l$, we have
        \[(x-1)\Phi_l(x) =x^l-1.\]
        Now over $\mathbb{F}_l$, the right hand side is $x^l-1=(x-1)^l$.
        Then dividing by the $x-1$ term on both sides give
        \[\Phi_l(x)=(x-1)^{l-1} \in\mathbb{F}_l.\]
      \end{proof}

    \item Suppose $p\neq l$ and let $f$ denote the order of $p\mod l$, i.e.
      $f$ is the smallest power of $p$ with $p^f\equiv1\mod l$. Use the
      fact that $\mathbb{F}_{p^n}^\times$ is a cyclic group to show that
      $n=f$ is the smallest power $p^n$ of $p$ with
      $\zeta\in\mathbb{F}_{p^n}$. Conclude that the minimal polynomial of
      $\zeta$ over $\mathbb{F}_p$ has degree $f$.

      \begin{proof}
        Let $n\in\mathbb{N}$ a positive integer such that
        $\zeta\in\mathbb{F}_{p^n}$. Since $\mathbb{F}_{p^n}^\times$ is a
        cyclic group, it must be generated by some element $a$ with order
        (as a group under multiplication) $p^n-1$. Then $\zeta=a^k$ for some
        $k<p^n$. Now $\zeta^l=1$, so $a^{kl}=1$. But $a$ has order $p^n-1$,
        so $p^n-1$ must divide $kl$, which implies $p^n\equiv1\mod l$. \\

        Conversely, if $p^n\equiv1\mod l$, then $p^n-1=kl$ for some
        $k\in\mathbb{N}$. Then in the group $\mathbb{F}_{p^n}^\times$ with
        generator $a$, element $a^k$ will satisfy $x^l=x^{p^n-1}=1$.
        Therefore $\mathbb{F}_{p^n}^\times$ will contain $\zeta$. \\

        Since $\mathbb{F}_{p^n}$ contains $\zeta$ if and only if
        $p^n\equiv1\mod l$, the given $n=f$ is the smallest power $p^n$ of
        $p$ with $\zeta\in\mathbb{F}_{p^n}$. \\

        Therefore, $\mathbb{F}_{p^f}=\mathbb{F}_p(\zeta)$, so the degree of
        the minimal polynomial of $\zeta$ over $\mathbb{F}_p$ is
        \[[\mathbb{F}_p(\zeta):\mathbb{F}_p] =[\mathbb{F}_{p^f}:\mathbb{F}_p]
        =f.\]
      \end{proof}

    \item Show that $\mathbb{F}_p(\zeta)=\mathbb{F}_p(\zeta^a)$ for any
      integer $a$ not divisible by $l$. Conclude using Part (b) that, in
      $\mathbb{F}_p[x]$, $\Phi_l(x)$ is the product of $\frac{l-1}{f}$
      distinct irreducible polynomials of degree $f$.

      \begin{proof}
        Clearly $\mathbb{F}_p(\zeta) \supseteq\mathbb{F}_p(\zeta^a)$. For
        the reverse inclusion, since $l$ is prime, if $a$ is not divisible
        by $l$, then $(a,l)=1$, and $\zeta^a$ is also a primitive $l$-th
        root of unity. In particular, $\zeta=(\zeta^a)^k$ for some integer
        $k$, therefore $\mathbb{F}_p(\zeta)
        \subseteq\mathbb{F}_p(\zeta^a)$. \\

        Because $l$ is prime, there are exactly $(l-1)$ $l$-th roots of
        unity, given by $\zeta,\zeta^2,\ldots,\zeta^{l-1}$. These roots are
        the $l-1$ distinct roots of $\Phi_l(x)$. Since the irreducible
        polynomial of each of these roots has degree $f$ given in part (b),
        $\Phi_l(x)$ must be the product of $\frac{l-1}{f}$ distinct
        irreducible polynomials of degree $f$. Note that the polynomials
        must be distinct otherwise the roots would of $\Phi_l(x)$ would
        repeat.
      \end{proof}

    \item In particular, prove that, viewed in $\mathbb{F}_p[x]$,
      $\Phi_7(x) =x^6+x^5+\ldots+x+1$ is $(x-1)^6$ for $p=7$, a product of
      distinct linear factors for $p\equiv1\mod 7$, a product of 3
      irreducible quadratics for $p=6\mod7$, a product of 2 irreducible
      cubics for $p=2,4\mod7$, and is irreducible for $p=3,5\mod7$.

      \begin{proof}
        Given that $l=7$, we first find the smallest $f\in\mathbb{N}$ such
        that $p^f\equiv1\mod l$ for the given $p$'s:
        \begin{align*}
          1^1\equiv 1\mod7 \\
          2^3\equiv 1\mod7 \\
          3^6\equiv 1\mod7 \\
          4^3\equiv 1\mod7 \\
          5^6\equiv 1\mod7 \\
          6^2\equiv 1\mod7 \\
        \end{align*}

        Therefore from part (c), we get the assertions given in the
        question.
      \end{proof}
  \end{enumerate}

\it \textbf{DF 13.6.9:} Suppose $A$ is an $n\times n$ matrix over
  $\mathbb{C}$ for which $A^k=I$ for some integer $k\geq1$. Show that $A$
  can be diagonalized. Show that the matrix $A=\begin{pmatrix} 1&\alpha\\
  0&1\\ \end{pmatrix}$ where $\alpha$ is an element of a field of
  characteristic $p$ satisfies $A^p=I$ and cannot be diagonalized if
  $\alpha\neq0$.

  \begin{proof}
    Let $A$ be an $n\times n$ matrix over $\mathbb{C}$ for which $A^k=I$
    for some integer $k\geq1$. Since this is a matrix over $\mathbb{C}$,
    all the roots of its characteristic polynomial will be contained in
    $\mathbb{C}$. Also, matrix $A$ satisfies the polynomial
    $x^k-1=0$, which has distinct roots by Proposition 33, therefore the
    minimal polynomial of $A$ must have distinct roots. Then from
    Corollary 12.3.25, $A$ must be diagonalizable. \\

    Let $A=\begin{pmatrix} 1&\alpha\\ 0&1\\ \end{pmatrix}$ over
    $\mathbb{F}_p$ and where $\alpha$ is an element of a field of
    characteristic $p$. By induction on $n$, we have
    \[A^n =\begin{pmatrix} 1&n\alpha\\ 0&1\\ \end{pmatrix}.\]

    Therefore $A^p=I$. Since $\alpha\neq0$, the matrix $B=A/\alpha$ is
    well-defined, and is a Jordan block for the eigenvalue $1/\alpha$.
    Then by uniqueness of the Jordan canonical form (Theorem 12.3.23),
    $A$ is not diagonalizable.
  \end{proof}

\it \textbf{DF 13.6.10:} Let $\varphi$ denote the Frobenius map $x\mapsto
  x^p$ on the finite field $\mathbb{F}_{p^n}$. Prove that $\varphi$ gives
  an isomorphism of $\mathbb{F}_{p^n}$ to itself. A Prove that $\varphi^n$
  is the identity map and that no lower power of $\varphi$ is the identity.

  \begin{proof}
    Clearly $\varphi$ preserves addition and multiplication from
    Proposition 35, thus it is a ring homomorphism.
    Recall that $\mathbb{F}_{p^n}^\times \cong\mathbb{Z}_{p^n-1}^+$.
    Then $\varphi$ must be injective since homomorphism of fields are
    injective. Then since $\mathbb{F}_{p^n}$ is finite, $\varphi$ must also
    be surjective, therefore $\varphi$ is an automorphism. \\

    $\varphi^n$ maps $x$ to $x^{p^n}$. Then since $\mathbb{F}_{p^n}^\times
    \cong\mathbb{Z}_{p^n-1}^+$, all non-zero elements in $\mathbb{F}_{p^n}$
    have multiplicative order $p^n-1$, therefore $\varphi$ is the identity
    map. \\

    Any lower power $m$ of $\varphi$ will map $x$ to $x^{p^n}$. Let
    $a\in\mathbb{F}_{p^n}^\times$ be a generator. Then
    $\varphi^m(a)=a^{p^m}$, which will equal to $a$ for the first time when
    $p^m-1$ equals $p^n-1$ for the first time, since $a$ has multiplicative
    order $p^n-1$. Therefore $n$ is the lowest such that $\varphi^n$ gives
    the identity. 
  \end{proof}

\it \textbf{DF 13.6.11:} Let $\varphi$ denote the Frobenius map $x\mapsto
  x^p$ on the finite field $\mathbb{F}_{p^n}$ as in the previous exercise.
  Determine the rational canonical form over $\mathbb{F}_p$ for $\varphi$
  considered as an $\mathbb{F}_p$-linear transformation of the
  $n$-dimensional $\mathbb{F}_p$-vector space $\mathbb{F}_{p^n}$.

  \begin{proof}
    In the previous exercise, we showed that $\varphi$ satisfies the
    polynomial $x^n-1=0$. We show that this is polynomial is also the
    minimal polynomial $m_\varphi(x)$ of $\varphi$: Assume not. Then
    $m_\varphi(x)$ has degree smaller than $n$, which implies that
    all $p^n-1$ elements in $\mathbb{F}_{p^n}^\times$ satisfy a polynomial
    smaller than $p^n$. But such a polynomial cannot has less than $p^n-1$
    distinct roots, a contradiction. Therefore $m_\varphi(x)=x^n-1$. \\

    Then $\varphi$ can has only one invariant factor $x^n-1$, thus its
    rational canonical form is the $n\times n$ matrix
    \[\begin{pmatrix}
      0&0&\ldots&0&1\\
      1&0&\ldots&0&0\\
       & &\ldots& & \\
      0&0&\ldots&1&0\\
    \end{pmatrix}.\]
  \end{proof}

\it \textbf{DF 13.6.12:} Let $\varphi$ denote the Frobenius map $x\mapsto
  x^p$ on the finite field $\mathbb{F}_{p^n}$ as in the previous exercise.
  Determine the Jordan canonical form (over a field containing all the
  eigenvalues) for $\varphi$ considered as an $\mathbb{F}_p$-linear
  transformation of the $n$-dimensional $\mathbb{F}_p$-vector space
  $\mathbb{F}_{p^n}$.

  \begin{proof}
    From the previous exercise, we showed that the invariant factor
    of the transformation is is $x^n-1$. Write $n=mp^k$, where $(m,p)=1$.
    Then from the proof of Exercise 13.6.4, $x^n-1=(x^m-1)^{p^k}$, and the
    roots of $x^m-1$ are distinct. Therefore the invariant factor is
    $x^n-1 =(x-a_1)^{p^k}\cdots(x-a_m)^{p^k}$, where $a_1,\ldots,a_m$ are
    the $m$ distinct $n$-th roots of unity. So the transformation must have
    $m$ elementary divisors each of multiplicity $p^k$. Thus the Jordan
    canonical form of the transformation is composed of the $m$ respective
    Jordan blocks.
  \end{proof}

\it \textbf{DF 14.1.4:} Prove that $\mathbb{Q}(\sqrt{2})$ and
  $\mathbb{Q}(\sqrt{3})$ are not isomorphic.

  \begin{proof}
    Assume that $\sigma$ is an from $\mathbb{Q}(\sqrt{2})$ to
    $\mathbb{Q}(\sqrt{3})$. Now all isomorphisms of fields of
    characteristic 0 must fix $\mathbb{Q}$, therefore
    \[2=\sigma(2) =\sigma((\sqrt{2})^2) =(\sigma(\sqrt{2}))^2.\]

    But the solutions of $x^2=2$ are only $\pm\sqrt{2}$, therefore
    $\sigma(\sqrt{2})$ must be $\pm\sqrt{2}$. However, $\pm\sqrt{2}$ does
    not lie in $\mathbb{Q}(\sqrt{3})$: otherwise we can write
    \[\pm\sqrt{2} =a\sqrt{3}+b\]
    for some $a,b\in\mathbb{Q}$, then squaring on both sides and
    rearranging give $\sqrt{3}\in\mathbb{Q}$, a contradiction.
  \end{proof}

\it \textbf{DF 14.1.4:} Determine the automorphisms of the extension
  $\mathbb{Q}(\sqrt[4]{2})/\mathbb{Q}(\sqrt{2})$ explicitly.

  \begin{proof}
    $\mathbb{Q}(\sqrt[4]{2})$ is the splitting field of $x^2-\sqrt{2}$ over
    $\mathbb{Q}(\sqrt{2})$. Therefore by Corollary 6, the extension is
    Galois, so there are exactly 2 automorphisms of the extension. Then
    from Proposition 2, the automorphisms must send $\sqrt[4]{2}$ to either
    itself or its negation. So the two automorphisms are defined by their
    image of  $\sqrt[4]{2}$ to either itself, which would give the identity
    automorphism, or to its negation.
  \end{proof}

\it \textbf{DF 14.1.7:} This exercise determines
  $\text{Aut}(\mathbb{R}/\mathbb{Q})$.

  \begin{enumerate}[label={(\alph*)}]
    \item Prove that any $\sigma\in\text{Aut}(\mathbb{R}/\mathbb{Q})$ takes
      squares to squares and takes positive reals to positive reals.
      Conclude that $a<b$ implies that $\sigma a<\sigma b$ for every
      $a,b\in\mathbb{R}$.

      \begin{proof}
        Given $r\in\mathbb{R}$, $\sigma(r^2)=(\sigma(r))^2$, therefore
        automorphisms take squares to squares. So if $r>0$, then
        \[\sigma(r)=\sigma((\sqrt{r})^2)=(\sigma(\sqrt{r}))^2>0.\]

        In particular, if $a<b$, then $\sigma(b-a)>0$, then since
        $\sigma(b-a)=\sigma(b)-\sigma(a)$, rearranging gives
        $\sigma(a)<\sigma(b)$.
      \end{proof}

    \item Prove that $-\frac{1}{m}<a-b<\frac{1}{m}$ implies
      $-\frac{1}{m}<\sigma a-\sigma b<\frac{1}{m}$ for every
      positive integer $m$. Conclude that $\sigma$ is a continuous map on
      $\mathbb{R}$.

      \begin{proof}
        The first assertion follows directly from part (a), since any
        automorphism of fields of characteristic 0 must fix $\mathbb{Q}$.
        To prove the second assertion from the first, fix any $\epsilon>0$,
        and $a\in\mathbb{R}$. Let $m\in\mathbb{N}^+$ be such that
        $1/m<\epsilon$. Then for all $b\in\mathbb{R}$ such that
        $|a-b|<1/m$, the first assertion will give $|\sigma a-\sigma
        b|<\epsilon$. Thus $\sigma$ is continuous.
      \end{proof}

    \item Prove that any continuous map on $\mathbb{R}$ which is the
      identity on $\mathbb{Q}$ is the identity map, hence
      $\text{Aut}(\mathbb{R}/\mathbb{Q})=1$.

      \begin{proof}
        Let $r\in\mathbb{R}$. Then $r$ is the limit of an infinite series
        of rational numbers
        \[r =\lim_{n\rightarrow\infty} q_n.\]
        
        Then from continuity of $\sigma$,
        \[\sigma r =\lim_{n\rightarrow\infty} \sigma q_n
        =\lim_{n\rightarrow\infty} q_n =r,\]
        as required.
      \end{proof}
  \end{enumerate}

\it \textbf{DF 14.1.10:} Let $K$ be an extension of the field $F$. Let
  $\varphi:K\rightarrow K'$ be an isomorphism of $K$ with a field $K'$
  which maps $F$ to the subfield $F'$ of $K'$. Prove that the map $\psi$
  defined by $\psi(\sigma) =\varphi\sigma\varphi^{-1}$ defines a group
  isomorphism $\text{Aut}(K/F)\rightarrow\text{Aut}(K'/F')$. 

  \begin{proof}
    Let $\sigma,\tau\in\text{Aut}(K/F)$. Then
    $\varphi(\sigma),\varphi(\tau) \in\text{Aut}(K'/F')$ since compositions
    of isomorphisms are isomorphisms, and if $s\in F'$, then
    $\sigma(\varphi^{-1}(s))=\varphi^{-1}(s)$, and
    $\psi(\sigma)(s)=\varphi(\varphi^{-1}(s))=s$. Also,

    \[\psi(\sigma\circ\tau) =\varphi\circ\sigma\circ\tau\varphi^{-1}
    =\varphi\circ\sigma\circ\varphi^{-1}
    \circ\varphi\circ\tau\circ\varphi^{-1} =\psi(\sigma)\circ\psi(\tau).\]
    Thus $\psi$ is a group homomorphism from $\text{Aut}(K/F)$ to
    $\text{Aut}(K'/F')$. \\

    Next we show that $\psi$ is injective. If $\psi(\sigma)=\text{id}$,
    then given $k\in K$, $\varphi(\sigma(\varphi^{-1}(\varphi(k))))
    =\varphi(k)$ since $\varphi\circ\varphi^{-1}=\psi(\sigma) =\text{id}$.
    Then composing with $\varphi^{-1}$ on both sides give
    $\varphi^{-1}(\varphi(k)) =k=\sigma(\varphi^{-1}(\varphi(k)))
    =\sigma(k)$, so $\sigma(k)=k$, implying $\sigma=\text{id}$ as required.
    \\

    Finally, we show by a similar argument that $\psi$ is surjective: Since
    $\text{Im}(\psi) \subseteq\text{Aut}(K'/F')$,
    $\varphi^{-1}\circ\tau\circ\varphi \in\text{Aut}(K/F)$. Also, since
    $\psi(\varphi^{-1}\circ\tau\circ\varphi)
    =\varphi\circ\varphi^{-1}\circ\tau\circ\varphi^{-1}\circ\varphi =\tau$,
    we have $\tau\in\text{Im}(\psi)$, so $\psi$ is surjective.
  \end{proof}
\end{document}
