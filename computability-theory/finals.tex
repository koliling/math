\documentclass{article}
\usepackage[left=3cm,right=3cm,top=3cm,bottom=3cm]{geometry}
\usepackage{amsmath,amssymb,amsthm,tikz,mathtools}
\usepackage{stmaryrd} % For double square bracket [[]]
\usepackage{bm} % For bold vectors
\usepackage{color}
\usepackage{cancel} % To use the cancel function
\usepackage[inline]{enumitem}
\usetikzlibrary{patterns}
\setlength{\parindent}{0mm}
\newcommand{\TODO}[1]{\textcolor{red}{TODO: #1}}

\begin{document}
\title{Basic Logic II: Finals}
\author{Li Ling Ko\\ lko@nd.edu}
\date{\today}
\maketitle

\textbf{Kunen I.9.6:} \it Derive the axioms of Infinity and Replacement
  from (2) of Lemma I.9.5. Hint. For Infinity, let $A$ be the
  (possibly proper) class of all natural numbers, and let $xRy$ iff
  $x=y+1$.

  \begin{proof}
    We follow the hint to derive Infinity. Let $A$ be the
    possibly proper class of all natural numbers, and let $xRy$ iff
    $x=y+1$. Formally, $A$ is the class of sets $n$ where $n$
    is transitive, well-ordered by $\in$, and contains only elements that
    are either the empty set or a successor. Also, relation $xRy$ is
    defined by $x=S(y)$, where $S$ denotes the formula for successor. Then
    $R$ is set-like on $A$, because every natural number $n$ has only one
    successor $S(n)$ which is also a natural number. Therefore
    $\text{pred}_{A,R}(n) =\{S(n)\}$, which is a set since $S(n)$ is a set
    and singletons of sets are sets by the Pairing Axiom. Thus from Lemma
    I.9.5 the transitive closure relation $R^*$ is also set-like. \\

    Consider the transitive closure $B$ of $\emptyset$ in $A$. $B$ is a set
    since $R^*$ is set-like. Also, $B$ will contain exactly all the natural
    numbers, and is therefore a witness for the Axiom of Infinity. \\

    To derive Replacement, fix any set $X$ and formula $\varphi(x,y,w)$
    such that for a fixed set $w$, given any $x\in X$ there is a unique set
    $y$ such that $\varphi(x,y,w)$ holds. We want to show that the range of
    $\varphi$, defined as $Y :=\{y: (\exists x\in X)\; \varphi(x,y,w)\}$,
    is a set. We define a relation $R$ on the possibly proper class $X\cup
    Y$ such that if $y\in Y$, then $xRy$ iff $x\in X$ and
    $\neg\varphi(x,y,w)$. Otherwise if $x\in X$, then $yRx$ iff
    $\varphi(x,y,w)$. \\

    Now $R$ is set-like on $X\cup Y$, because if $y\in Y$, then
    $\text{pred}_{X\cup Y,R}(y) =\{x\in X: \neg\varphi(x,y,w)\}$, which is
    a set by the axiom of Comprehension since it is a subset of $X$ and can
    be defined by the formula $\neg\varphi(x,y,w)$. Also, if $x\in X$,
    then $\text{pred}_{X\cup Y,R}(x)=\{y\}$, where $y$ is the unique set
    that $\varphi(x,y,w)$ holds; this will also be a set by the Pairing
    axiom since $y$ is a set. Thus by Lemma I.9.5, the transitive closure
    relation $R^*$ is also set-like. \\

    Fix any $x\in X$, and consider its transitive closure
    $S=\text{pred}_{X\cup Y,R^*}(x)$. Then $S$ is a set from Lemma I.9.5,
    and $S$ is exactly $X\cup Y$ by definition. Then using Comprehension,
    we can extract $Y$ as a set from $S$ as follows:
    \[Y =\{y\in S:\; (\exists x)\; \varphi(x,y,w)\}.\]
  \end{proof}

\textbf{Kunen I.9.24:} \it Suppose that $x,y\in\text{WF}$. Then:
  \begin{enumerate}
    \item \it $\{x,y\}\in\text{WF}$ and $\text{rank}(\{x,y\})
      =\max(\text{rank}(x), \text{rank}(y))+1$.

      \begin{proof}
        The first assertion follows from I.9.24.3 which says a set lies in
        WF iff all its elements lie in WF. The second assertion follows
        from I.9.24.4.
      \end{proof}

    \item \it $\langle x,y\rangle\in\text{WF}$ and $\text{rank}(\langle
      x,y\rangle) =\max(\text{rank}(x), \text{rank}(y))+2$.

      \begin{proof}
        The first assertion follows from I.9.24.3 and the fact that both
        $\{x\}$ and $\{x,y\}$ lie in WF from part (1) of this question.
        The second assertion follows from I.9.24.4 and the fact that
        $\text{rank}(\{x,y\}) =\max(\text{rank}(x), \text{rank}(y))+1$ as
        shown in the first part of this question.
      \end{proof}

    \item \it If $\mathcal{P}(x)$ exists, then $\mathcal{P}(x)\in\text{WF}$
      and $\text{rank}(\mathcal{P}(x)) =\text{rank}(x)+1$.

      \begin{proof}
        $x$ lies in WF, so each of its members must also lie in WF from
        Lemma I.9.24.1. Then any subset of the $x$ also lies in WF from
        Lemma I.9.24.3, therefore $\mathcal{P}(x)$ also lies in WF from the
        same lemma. Since $x$ lies in $\mathcal{P}(x)$, the rank of
        $\mathcal{P}(x)$ is at least $\text{rank}(x)+1$ from Lemma
        I.9.24.4. Also from the same lemma, the rank of any subset of $x$
        cannot exceed the rank of $x$. Therefore the rank of
        $\mathcal{P}(x)$ cannot exceed $\text{rank}(x)+1$, and must equal
        $\text{rank}(x)+1$.
      \end{proof}

    \item \it $\bigcup x\in\text{WF}$ and $\text{rank}(\bigcup x)
      \leq\text{rank}(x)$.

      \begin{proof}
        The members of $\bigcup x$ are the members of members of $x$. Since
        each member of $x$ is in WF from Lemma I.9.24.1, the members of
        members of $x$ also lie in WF from the same lemma. Then from Lemma
        I.9.24.3, $\bigcup x$ lies in WF. For the second assertion in the
        question, assume by contradiction that $\text{rank}(\bigcup
        x)>\text{rank}(x)$. Then from Lemma I.9.24.4, there must be a
        member $y$ of $\bigcup x$ whose rank is at least the rank of $x$. But
        then $y$ is a member $z$ of a member of $x$, so the rank of $z$ is
        at least $\text{rank}(x)+1$ from Lemma I.9.24.4, and then from the
        same lemma the rank of $x$ is at least $\text{rank}(x)+2$, a
        contradiction.
      \end{proof}

    \item \it $x\cup y\in\text{WF}$ and $\text{rank}(x\cup y)
      =\max(\text{rank}(x), \text{rank}(y))$.

      \begin{proof}
        Each member of $x$ or of $y$ lie in WF from Lemma I.9.24.1,
        therefore $x\cup y$ lies in WF from Lemma I.9.24.3. Then from Lemma
        I.9.24.4, the rank of $x\cup y$ is at least the rank of $x$ and of
        $y$ since $x\cup y$ is a superset of $x$ and $y$. Also, the rank
        cannot exceed the rank of $x$ or the rank of $y$, because each
        element of $x\cup y$ is contained in either $x$ or $y$. Therefore
        $\text{rank}(x\cup y) =\max(\text{rank}(x), \text{rank}(y))$.
      \end{proof}

    \item \it $\text{trcl}(x)\in\text{WF}$ and $\text{rank}(\text{trcl}(x))
      =\text{rank}(x)$.

      \begin{proof}
        We prove by transfinite induction on the rank of $x$. If $x$ has
        rank 0, then from Lemma I.9.24.4 $x$ must be $\emptyset$, and the
        assertion holds trivially. \\

        For induction on successor ordinals, let $x\in\text{WF}$ with rank
        $\alpha+1$ for some ordinal
        $\alpha$. Then from Lemma I.9.24.4, all members of $x$ have rank
        less than or equal $\alpha$, and there must be at least one member
        $y$ of $x$ that has rank $\alpha$. Now
        \begin{equation}
          \text{trcl}(x) =x\cup\bigcup_{y\in x} \text{trcl}(y).
          \label{eqn:trcl}
        \end{equation}
        For each $y\in x$,
        since $\text{rank}(y)\leq\alpha$, by induction hypothesis
        $\text{trcl}(y)$ lies in WF and has rank equal to the rank of $y$.
        Also each $y\in x$ lies in WF from Lemma I.9.24.1. Therefore
        from equation~\eqref{eqn:trcl} and Lemma I.9.24.1, all elements of
        $\text{trcl}(x)$ lie in WF, so by Lemma I.9.24.3 $\text{trcl}(x)$
        also lies in WF. Furthermore, since
        $\text{rank}(\text{trcl}(y)) \leq\alpha$ for each $y\in x$, and
        there exists a member of $x$ whose rank is $\alpha$, therefore from
        from Lemma I.9.24.4 and equation~\eqref{eqn:trcl}, $\text{trcl}(x)$
        has rank $\alpha+1=\text{rank}(x)$. \\

        For induction on limit ordinals, let $x\in\text{WF}$ with rank
        $\gamma$ where $\gamma$ is a limit ordinal. Then from Lemma
        I.9.24.4, all members of $x$ have rank smaller than $\gamma$, and
        for every ordinal $\alpha$ less than $\gamma$, there must exist a
        member in $x$ with rank equal to $\alpha$. For each
        $y\in x$, since $\text{rank}(y)\leq\gamma$, by induction
        hypothesis $\text{trcl}(y)$ lies in WF and has rank less than
        $\gamma$. Also each $y\in x$ lies in WF from Lemma I.9.24.1.
        Therefore from equation~\eqref{eqn:trcl} and Lemma I.9.24.1, all
        elements of $\text{trcl}(x)$ lie in WF, so by Lemma I.9.24.3
        $\text{trcl}(x)$ also lies in WF. Furthermore, since
        $\text{rank}(\text{trcl}(y)) \leq\gamma$ for each $y\in x$, and for
        every ordinal less than $\gamma$
        there exists a member of $x$ whose rank equals that ordinal,
        therefore from from Lemma I.9.24.4 and equation~\eqref{eqn:trcl},
        $\text{trcl}(x)$ has rank $\sup\{\alpha+1: \alpha<\gamma\}
        =\gamma=\text{rank}(x)$. \\
      \end{proof}
  \end{enumerate}

\textbf{Kunen I.9.50:} \it Assume that $R$ is set-like on $A$ and not
  well-founded. Define a $G(x,s)$ that is an explicit counter-example to
  Theorem I.9.11.

  \begin{proof}
    If the Axiom of Foundation does not hold, there must exist an infinite
    chain of reverse-memberships:
    \[a_0 \ni a_1 \ni a_2 \ni \ldots\]
    Using the Axiom of Comprehension, we can assume that for each
    $i\in\omega$, $a_{i+1}$ is the only element contained in $a_i$. Then
    the membership relation is trivially set-like but not well-founded in
    $A:=\{a_i\}_{i\in\omega}$. Consider the rank function which can be
    defined recursively as
    \[\text{rank}(x) =\bigcup_{y\in x} \hat{S}(\text{rank}(y)),\]
    where
    \begin{align*}
      \hat{S}(x) :=
      \begin{cases}
        x\cup\{x\} &\text{if}\; x\; \text{is an ordinal}\\
        \emptyset &\text{otherwise}.\\
      \end{cases}
    \end{align*}

    Then following the notation of Theorem I.9.11, in order to define the
    rank function as
    \[\text{rank}(x)=G(x,\text{rank}\restriction(x\downarrow)),\]
    we need to define $G(x,s)$ as
    \[G(x,s) =\bigcup_{\langle y,r\rangle\in s} \hat{S}(r).\]

    Then $G(x,s)$ is clearly a 1-1 and well-defined function on the
    universe. However, then rank of the $a_i$'s are not well-defined: It is
    routine to prove by induction on $i\in\omega$ that for all
    $n\in\omega$,
    \[\text{rank}(a_i) =\hat{S}^n(\text{rank}(a_{i+n})).\]

    Now if there is some $n\in\omega$ such that $\text{rank}(a_i)$ is an
    ordinal for all $i\geq n$, then
    \[\text{rank}(a_i) =S(\text{rank}(a_{i+1})) =\text{rank}(a_{i+1})+1,\]
    which will give us an infinite decreasing chain of ordinals
    \[\text{rank}(a_n) \ni\text{rank}(a_{n+1}) \ni\text{rank}(a_{n+2})
    \ni\ldots\]
    Then $\{\text{rank}(a_i)\}_{i\geq n} \subseteq\text{rank}(a_n)$ will
    be a subset of ordinal $\text{rank}(a_n)$ that does not have an
    $\in$-minimal element, contradicting well-ordering of ordinals. \\

    Therefore there must be infinite $i\in\omega$ such that
    $\text{rank}(a_i)$ is not an ordinal. Let $n,k\in\omega$ be natural
    numbers such that $k>0$ and neither $\text{rank}(a_n)$ nor
    $\text{rank}(a_{n+k})$ are ordinals. Then
    \begin{align*}
      \emptyset &=\hat{S}(\text{rank}(a_{n})) &\text{since}\;
        \text{rank}(a_n)\; \text{is not an ordinal}\\
      &=\text{rank}(a_{n+1})\\
      &=\hat{S}^{k}(\text{rank}(a_{n+k+1}))\\
      &=\hat{S}^{k}(\hat{S}(\text{rank}(a_{n+k})))\\
      &=\hat{S}^{k}(\emptyset) &\text{since}\; \text{rank}(a_{n+k})\;
        \text{is not an ordinal}\\
      &=k,\\
    \end{align*}
    a contradiction.

    %Assume our universe $V$ contains two distinct sets $a$ and $b$ which
    %contain each other, i.e. $a=\{b\}$ and $b=\{a\}$. Let $R$ be the usual
    %containment relation $\in$, and let $A=\{a,b\}$. Then $\in$ is clearly
    %set-like since for any set $x$, $\text{pred}(x)$ is $x$ itself. However
    %$\in$ is not well-founded on $A$ since there is no $\in$-minimal
    %element in that set. \\

    %Following the notation of Theorem I.9.11, we define $G(x,s)$ such that
    %if the associated recursive function $F(x)$ exists, then $F(x)$ returns
    %the set of $\in^*$-minimal elements of the input set $x$. If designed
    %properly, we can ensure $F$ runs into an infinite loop when it
    %computes the $\in^*$-minimal elements of an element in $A$, because
    %$\text{trcl}(a) =\text{trcl}(b) =\{a,b\}$ has no $\in$-minimal element.
    %To that end, observe that the set of $\in^*$-minimal elements of a set
    %$x$ can be defined recursively as the union of the $\in^*$-minimal
    %elements of the members of $x$. Thus we have the recursive relation
    %\[F(x) :=\bigcup_{y\in x} \left\{z\in F(y): \left(\forall z'\in
    %\bigcup_{y'\in x}F(y')\right) z'\not\in z \right\}.\]
    %%Formally, $F(x)$ can be defined recursively as a function over $\{F(y):
    %%y\in x\}$:
    %%\[F(x) :=\{z|\; (\exists y) [y\in x \wedge z\in F(y)]\; \wedge\;
    %%(\forall y',z')[y'\in x \wedge z'\in F(y') \rightarrow z'\not\in
    %%z]\}.\]
    %%\[G(x,s) :=\underset{\langle y,z\rangle \in s}{R\text{-min}}\; z.\]

    %Therefore, when $x\not\in A$, to get the desired $F(x)$ we define
    %\[G(x,s) :=\bigcup_{\langle y,F\rangle\in s} \left\{z\in F:
    %\left(\forall z'\in \bigcup_{\langle y',F'\rangle\in s}F'\right)
    %z'\not\in z \right\}.\]

    %When $x$ lies in $A$, to ensure that we run into an infinite loop when
    %we compute $F(b)$, we set $G(a,s)=\{a\}$ for all $s$. Formally, define
    %\begin{align*}
    %  G(x,s) :=
    %  \begin{cases*}
    %    \{a\} &\text{if}\; x=a,\\
    %    \{z|\;\; (\exists y,Z)(\langle y,Z\rangle \in s \wedge z\in
    %    Z)\; \wedge\; [(\forall\; \langle y',Z'\rangle \in s)(\forall z'\in
    %    Z')\; z'\not\in z]\} &\text{otherwise}.\\
    %  \end{cases*}
    %\end{align*}

    %Since $G(x,s)$ is defined by $\Delta_0$-formulas, it is recursive.
    %Also, it is well-defined as the only free variables in the formula are
    %the inputs $x$ and $s$.

    %Consider the example where $A=\text{ON}$, and $R$ functions almost like
    %a successor cardinal operator:
    %\[\beta R\alpha \Leftrightarrow \beta\leq\alpha^+.\]
    %Then $R$ is set-like on ON because
    %\[\text{pred}_{R,\text{ON}}(\alpha) =\{\beta: \beta\leq\alpha^+\}
    %=\alpha^++1\in\text{ON}.\]
    %However $R$ is not not well-founded on $A$ since every ordinal has an
    %ordinal larger than it. \\

    %We define $G(\alpha,s)$ in a way that if the desired recursive function
    %$F(\alpha) =G(\alpha, F\restriction(\alpha\downarrow))$ exists, then it
    %would be computing the $R$-transitive closure of the input ordinal.
    %Intuitively, such $F$ cannot be defined on some large input ordinal
    %$\alpha$, because to compute $F(\alpha)$ we would need to keep
    %accessing cardinals that are larger than 

    %To
    %that end, recall that the usual recursive method of defining an
    %$R$-transitive-closure is
    %\[\text{trcl}_R(x) :=x \cup \bigcup_{yRx} \text{trcl}_R(y).\]

    %Thus we define our $G$ function as 
    %\[G(\alpha,s) :=\alpha \cup \bigcup_{\langle \beta,\gamma\rangle \in s}
    %\gamma.\]
  \end{proof}

\textbf{Kunen I.9.54:} \it Prove the Cantor Normal Form Theorem: Each
  ordinal $\alpha>0$ can be represented uniquely in the form:
  \[\alpha =\omega^{\beta_1}\cdot n_1 +\ldots +\omega^{\beta_k}\cdot n_k,\]
  where $k,n_1,\ldots,n_k\in\omega\setminus\{0\}$ and
  $\alpha\geq\beta_1>\cdots>\beta_k$.

  \begin{proof}
    We prove existence by transfinite induction on $\alpha$. The base case
    $\alpha=1$ can be written as $\alpha=\omega^0\cdot1$. For the inductive
    step, let $\alpha$ be an arbitrary ordinal. Then by the Logarithm rule
    of ordinal arithmetic (Table I.1), there must be some greatest $\beta$
    such that $\omega^\beta\leq\alpha$. Then by Division rule of oridnal
    arithmetic (Table I.1), there must be unique $\delta>0$ and
    $\rho<\omega^\beta\leq\alpha$ such that $\alpha=\omega^\beta\cdot\delta
    +\rho$. Now $\delta$ must be finite, otherwise $\alpha\geq\omega^\beta
    \cdot\delta \geq\omega^\beta\cdot\omega =\omega^{\beta+1}$,
    contradicting minimality of $\beta$. Then if $\rho=0$, $\alpha$ would
    be in Cantor Normal Form. So assume $\rho>0$. \\

    Then $\rho<\alpha$, so by induction hypothesis $\rho$ has a Cantor
    Normal Form 
    \[\rho =\omega^{\beta_1}\cdot n_1 +\ldots +\omega^{\beta_n}\cdot n_k\]
    for some $k,n_1,\ldots,n_k\in\omega\setminus\{0\}$ and
    $\rho\geq\beta_1>\cdots>\beta_k$. But since $\rho<\omega^\beta$, we
    have $\omega^{\beta_1}\leq\rho<\omega^\beta$ and therefore
    $\beta>\beta_1$. So $\alpha$ will also have Cantor Normal Form 
    \[\alpha =\omega^{\beta}\cdot \delta +\omega^{\beta_1}\cdot n_1 +\ldots
    +\omega^{\beta_k}\cdot n_k.\]

    Next we prove uniqueness of the Cantor Normal Form by transfinite
    induction on $\alpha$. For $\alpha=1$, the form $\alpha=\omega^0\cdot1$
    is clearly unique. For the inductive step, let
    \[\alpha =\omega^{\beta_1}\cdot n_1 +\ldots +\omega^{\beta_k}\cdot
    n_k =\omega^{\gamma_1}\cdot m_1 +\ldots +\omega^{\gamma_l}\cdot
    m_l\]
    be two Cantor Normal Forms for $\alpha$. Now observe that if
    $\beta<\gamma$, then $\omega^\beta\cdot k<\omega^\gamma$ for all finite
    $k$, because $\omega^\beta\cdot k<\omega^\beta\cdot\omega
    <\leq\omega^\gamma$. Therefore if $\gamma_1>\beta_1$, then
    $\alpha<\omega^{\gamma_1}$, a contradiction. By symmetrical argument,
    $\beta_1\not>\gamma_1$, thus $\beta_1=\gamma_1$. Then from uniqueness
    of quotient and remainder in the Division rule for ordinal arithmetic,
    with dividend $\alpha$ and divisor $\omega^{\beta_1}=\omega^{\gamma_1}$,
    the respective quotients $n_1$ and $m_1$ must be equal, and the
    remainders $\omega^{\beta_2}\cdot n_2 +\ldots +\omega^{\beta_k}\cdot
    n_k$ and $\omega^{\gamma_2}\cdot m_2 +\ldots +\omega^{\gamma_l}\cdot
    m_l$ must also be equal, since both remainders are smaller than the
    divisor. Then from induction hypothesis, the Cantor Normal Forms
    $\omega^{\beta_2}\cdot n_2 +\ldots +\omega^{\beta_k}\cdot
    n_k$ and $\omega^{\gamma_2}\cdot m_2 +\ldots +\omega^{\gamma_l}\cdot
    m_l$ must also be identical, so the Cantor Normal Form of $\alpha$ is
    unique.
  \end{proof}

\textbf{Kunen I.16.6:} \it (ZF$^-$) Let $\text{pow}(x,y)$ be $\forall z
  [z\in y \leftrightarrow z\subseteq x]$, asserting that $y$ is the power
  set of $x$. Let $\gamma$ be a limit ordinal, with $a,b\in R(\gamma)$.
  Prove that $R(\gamma) \models\text{pow}[a,b]$ iff $b=\mathcal{P}(a)$;
  that is, $R(\gamma) \preceq_{\text{pow}} V$.

  \begin{proof}
    $b=\mathcal{P}(a) \Rightarrow R(\gamma) \models\text{pow}[a,b]$ holds
    trivially since $R(\gamma)\subset V$. Assume
    $(b=\mathcal{P}(a))^{R(\gamma)}$. For $b=\mathcal{P}(a)$ to also hold
    in the universe, we need to show that every subset of $a$ in $V$ is
    also contained in $R(\gamma)$. Fix an arbitrary subset $c$ of $a$. Then
    $\text{rank}(c)$ cannot exceed $\text{rank}(a)$ by Lemma I.9.24.4.
    Also, $c$ is well-founded since it is a subset of a well-founded set
    $a$. Therefore $c$ must be contained in $R(\gamma)$. \\
  \end{proof}

\textbf{Kunen I.16.7:} \it (ZFC$^-$) Let $\gamma>\omega_1$ be a limit
  ordinal. Prove that there is a countable transitive $M$ and ordinals
  $\alpha,\beta\in M$ such that $M\equiv R(\gamma)$ and
  $(\alpha\approx\beta)^M$ is false but $(\alpha\approx\beta)^{R(\gamma)}$
  is true.

  \begin{proof}
    Since $\gamma>\omega_1$ and the rank of any given ordinal is itself and
    ordinals are well-founded, $R(\gamma)$ must contain $\omega$ and
    $\omega_1$. Thus from the the Downward Lowenheim-Skolem-Tarski theorem,
    $R(\gamma)$ must contain a countable elementary substructure $A\preceq
    R(\gamma)$ that contains $\omega$ and $\omega_1$. As an elementary
    substructure, $A\equiv R(\gamma)$. Let $M$ be the Mostowski collapse of
    $A$. \\

    We first show $M$ is transitive. Observe that $R(\gamma)$ is
    well-founded since it is a subset of WF by definition, therefore $A$ is
    also well-founded since it is a subset of $R(\gamma)$. Also, $\in$ is
    trivially set-like on $A$. Thus from Lemma I.9.22, $M$ is
    transitive. \\

    Also, because $A$ is well-founded, from Lemma I.9.35 we have $(a\in
    b)^M \Leftrightarrow (a\in b)^A$. Therefore $M\equiv A$, which implies
    $M\equiv R(\gamma)$. Next, observe that since each natural number is
    constructed from a finite application of the Pairing axiom on the empty
    set, and $A$ satisfies the Pairing axiom and Foundation, $A$ must
    contain all the natural numbers. Then $\omega\cup\{\omega\}$ is a
    transitive subset of $A$, and $A$ is well-founded, so from Lemma I.9.36
    $\text{mos}_{A,\in}(\omega)=\omega$ in $M$. On the other hand,
    $|\text{mos}_{A,\in}(\omega_1)|$ must be countable otherwise from
    transitivity of $M$, all its elements will be in $M$, thus making $M$
    uncountable. Furthermore, since $A$ is well-founded, the isomorphism
    between $(A,\in)$ and $(M,\in)$ from Lemma I.9.35 will ensure that
    $\omega_1$ remains transitive and well-founded after taking Mostowski's
    collapse with respect to $A$. Therefore $\text{mos}_{A,\in}(\omega_1)$
    is an ordinal. \\

    Let $\alpha=\text{mos}_{A,\in}(\omega)=\omega$ and
    $\beta=\text{mos}_{A,\in}(\omega_1)$. Then from the above argument,
    $\beta$ is a countable ordinal, so $(\alpha\approx\beta)^{R(\gamma)}$
    holds. However, within $M$, the isomorphism between $(A,\in)$ and
    $(M,\in)$ from Lemma I.9.35 ensures that if there is no bijection
    between $\omega$ and $\omega_1$ in $A$, then there is no bijection
    between $\alpha$ and $\beta$ in $M$. Thus
    $(\alpha\approx\beta)^{M}$ is false.
  \end{proof}

\textbf{Kunen II.4.8:} \it The notions ``$R$ well-orders $A$'' and ``$R$ is
  well-founded on $A$'' are absolute for $R(\gamma)$ for any limit
  $\gamma$.

  \begin{proof}
    The fact that ``$R$ totally-orders $A$'' is absolute for $R(\gamma)$
    follows from transitivity of $R(\gamma)$ and from the fact that the
    given statement is $\Delta_0$ (Lemma II.4.2). Thus it suffices to prove
    that the statement
    \[\varphi(A,R) :=(\forall B\subseteq A)(\exists b\in B)(\forall b'\in
    B) [b'\neq b\rightarrow \langle b,b'\rangle \in R]\]
    is absolute for $R(\gamma)$. Let $R,A\in R(\gamma)$. Clearly if $R$ is
    a well-ordering of $A$ in $V$, then it must be a well-ordering of $A$
    in $R(\gamma)$, since $R(\gamma)$ is a subset of $V$. For the converse,
    assume that $(R\; \text{well-orders}\; A)^{R(\gamma)}$. Let $B$ be an
    arbitrary subset of $A$. We need to show that $R$ picks an $R$-minimal
    element in $B$. From Exercise I.16.6, $B$ lies in $R(\gamma)$. In
    particular, $(B\subseteq A)^{R(\gamma)}$, so since the remaining
    sub-formula of $\varphi(A,R)$ is $(\exists b\in B)(\forall b'\in B)
    [b'\neq b\rightarrow \langle b,b'\rangle \in R]$ which is $\Delta_0$
    over variables $R$ and $B$, the subformula also holds in $R(\gamma)$
    for the given subset $B$. Therefore $R$ picks a minimal element for $B$
    in $R(\gamma)$. \\

    The assertion in the question for well-foundedness is just a special
    case of $R$ well-ordering $A$ but with $R$ representing the membership
    relation. Therefore the formula representing ``$R$ is well-founded on
    $A$'' is a conjunction of the formula for ``$R$ well-orders $A$'' and
    the following formula
    \[\phi(A,R):= (\forall a,a'\in A) [a\in a'\rightarrow \langle
    a,a'\rangle \in R]\; \wedge\; (\forall \langle b,b'\rangle\in R) [b\in
    A \wedge b'\in A \wedge b\in b'].\]
    Since this is a $\Delta_0$ formula, and $R(\gamma)$ is transitive, the
    formula is absolute for $R(\gamma)$. Thus ``$R$ is well-founded on
    $A$'' is absolute for $R(\gamma)$.
  \end{proof}

\textbf{Kunen II.4.8:} \it Prove that $\text{Th}(HF) \in R(\omega+1)
  \setminus L(\omega+1)$, and $\rho(\text{Th}(HF)) =\omega+1$, while
  $\text{rank}(\text{Th}(HF)) =\omega$.

  \begin{proof}
    First, observe that $L(n)$ is finite for each $n\in\omega$ because
    $L(0)=\emptyset$ and each $L(n+1)$ is at most the powerset of $L(n)$.
    In particular, each element of $L(\omega)$ is finite, therefore
    $L(\omega)$ cannot contain the infinite set $\text{Th}(HF)$. Then since
    $R(\omega)=L(\omega)$, $R(\omega)$ also cannot contain $\text{Th}(HF)$.
    Thus the rank and the $L$-rank of $\text{Th}(HF)$ must be at least
    $\omega$. \\

    Then as a subset of $\omega$, $\text{Th}(HF)$ is well-founded, and has
    rank not exceeding the rank of $\omega$, which is $\omega$. Therefore
    $\text{rank}(\text{Th}(HF)) =\omega$. \\

    Assume by contradiction that $\text{Th}(HF) \in L(\omega+1)$. Then
    since $L(\omega+1) =\mathcal{D}(HF)$, $\text{Th}(HF)$ must be definable
    over $HF$. More precisely, there must be a formula $\phi(x)$ in the
    language of set theory such that
    \[\text{Th}(HF) =\{\varphi: \phi(\ulcorner\varphi\urcorner)\},\]
    where $\ulcorner\varphi\urcorner$ is derived from a fixed recursive
    coding of formulas in the language of set theory. But since $HF$
    contains all the natural numbers and is powerful enough to perform
    arithmetic, $\phi(x)$ would contradict Tarski's theorem on the
    undefinability of truth. \\ 

    It remains to prove that $\rho(\text{Th}(HF)) =\omega+1$. From
    Tarski's definability of truth, one can use recursion to define
    rigorously the notion of $\mathfrak{A} \models\varphi$. Then since
    $L(\omega+1)$ contains the natural numbers and is powerful enough to
    perform arithmetic, by the Church-Turing thesis, there there must be a
    formula $\theta(A,p)$ in the language of set theory such that given any
    set $A\in L(\omega+1)$ and sentence $\varphi$,
    \[A\models\varphi\; \Leftrightarrow\;
    \theta(A,\ulcorner\varphi\urcorner).\]
    Then since $L(\omega+1)$ contains $HF=L(\omega)$, we can define
    $\text{Th}(HF)$ in $L(\omega+2)$ as
    \[\text{Th}(HF) =\{\varphi:
    \theta(L(\omega),\ulcorner\varphi\urcorner)\}.\]
    Note that we only need to consider sentences $\varphi$ because every
    element of $HF$ is definable in $HF$. Therefore $\rho(\text{Th}(HF))
    =\omega+1$. \\

    %Next, observe that for each $n\in\omega$, $L(n+1)$ is exactly the
    %powerset of $L(n)$, because any finite subset of elements in $L(n)$ is
    %definable. Therefore we can recursively index each set in $L(\omega)$
    %by a tuple in $\omega^{<\omega}$, where the first coordinate represents
    %the $L$-rank of the set and the second coordinate represents the
    %indices of the members of the set. In short, there is a recursive
    %bijection between $\omega$ and the elements in $L(\omega)$.

    %Observe that the sets in $L(n)$ for each $n\in\omega$ are finite (by
    %easy induction on $n\in\omega$). Also observe that the language of set
    %theory is countable, and each set in $L(n)$ is obtained from smaller
    %sets in $L(n-1)$ from enumerating formulas in the language. Therefore
    %we can recursively enumerate all the elements in $L(\omega)$, indexing
    %each element by the code of the formula that defined it and the finite
    %set of codes of the sets of lower $L$-rank that were used to define the
    %new element.
  \end{proof}
\end{document}
