\documentclass{article}
\usepackage[left=3cm,right=3cm,top=3cm,bottom=3cm]{geometry}
\usepackage{amsmath,amssymb,amsthm,tikz,mathtools}
\usepackage{color}
\usepackage[inline]{enumitem}
\usetikzlibrary{patterns}
\setlength{\parindent}{0mm}
\newcommand{\TODO}[1]{\textcolor{red}{TODO: #1}}

\begin{document}
\title{Basic Logic II: Problem Set 1}
\author{Li Ling Ko\\ lko@nd.edu}
\date{\today}
\maketitle

\begin{enumerate}[label={\bf Q\arabic*:}]
  \item \it (See Section 3 for notations and definitions). Let
    $\mathcal{L}$ be a countable language and $T$ an $\mathcal{L}$-theory.

    \begin{enumerate}[label={\bf(\arabic*)}]
      \item Assume $T$ is model-complete, and let $\mathcal{M}$ be a model
        of $T$. Show that $T\cup\text{Diag}(\mathcal{M})\models
        T_\mathcal{M}$. (Notice that both $\text{Diag}(\mathcal{M})$ and
        $T_\mathcal{M}$ are $\mathcal{L}(\mathcal{M})$-theories.)

        \begin{proof}
          Let $\varphi(\bar{m})$ be an arbitrary formula in
          $T_\mathcal{M}$, where $\bar{m}\in M$. We wish to show that
          \[T\cup\text{Diag}(\mathcal{M})\vdash \varphi(\bar{m}).\]

          By model-completeness of $T$, there is a quantifier-free formula
          $\theta(\bar{x},\bar{y})$ such that
          \[T\models \forall\bar{x}\; (\varphi(\bar{x})\leftrightarrow
            \exists\bar{y}\; \theta(\bar{x},\bar{y})).\]

          Thus since $\mathcal{M}\models\varphi(\bar{m})$, we must have
          $\theta(\bar{m},\bar{a})$ for some $\bar{a}\in M$, which means
          $\theta(\bar{m},\bar{a})\in\text{Diag}(\mathcal{M})$. Thus we
          have
          \begin{align*}
            T\cup\text{Diag}(\mathcal{M}) &\models \forall\bar{x}\;
              (\varphi(\bar{x})\leftrightarrow \exists\bar{y}\;
              \theta(\bar{x},\bar{y}))\; \wedge\; \theta(\bar{m},\bar{a})
              \\
              &\models \varphi(\bar{m}). \\
          \end{align*}

          Since $\varphi(\bar{m})\in T_\mathcal{M}$ was arbitrary, we have
          $T\cup\text{Diag}(\mathcal{M})\models T_\mathcal{M}$.
        \end{proof}

      \item Prove that $T$ is model-complete if and only if for every model
        $\mathcal{M}$ of $T$ the $\mathcal{L}(\mathcal{M})$-theory
        $\text{Diag}(\mathcal{M})\cup T$ is complete.

        \begin{proof}
          $\Rightarrow$: Follows directly from part 1 above, since
          $T_\mathcal{M}$ is a complete theory in the language of
          $\mathcal{L}(\mathcal{M})$. \\

          $\Leftarrow$: Assume the $\mathcal{L}(\mathcal{M})$-theory
          $\text{Diag}(\mathcal{M})\cup T$ is complete for every model
          $\mathcal{M}$ of $T$. To show that $T$ is model-complete is
          equivalent to showing that every model of $T$ is ec-closed in
          $T_\forall$ (Theorem 11.14.3). Now the models of $T_\forall$ are
          exactly the substructures of the models of $T$ (Lemma 11.2).
          Also, if every model of $T$ is ec-closed in $T$, then given
          models $\mathcal{M},\mathcal{N}\models T$ such that
          $\mathcal{M}<\mathcal{N}_0<\mathcal{N}$, we will have
          $\mathcal{M}<\mathcal{N}$ is ec-closed and thus
          $\mathcal{M}<\mathcal{N}_0$ is ec-closed. Therefore it suffices
          to show that models of $T$ are ec-closed in $T$. \\

          Let $\mathcal{M},\mathcal{N}\models T$ with
          $\mathcal{M}<\mathcal{N}$. Assume by contradiction that
          $\mathcal{M}$ is not ec-closed in $\mathcal{N}$. Then there is a
          quantifier-free formula $\varphi(\bar{x},\bar{y})$ such that
          $\mathcal{N}\models\exists\bar{x}\; \varphi(\bar{x},\bar{m})$ for
          some $\bar{m}\in M$ but $\mathcal{M}\not\models\exists\bar{x}\;
          \varphi(\bar{x},\bar{m})$. Then we have

          \[\begin{array}{lrlr}
            &\mathcal{M} &\models\forall\bar{x}\;
              \neg\varphi(\bar{x},\bar{m}) \\
            \Rightarrow &T_\mathcal{M} &\models\forall\bar{x}\;
              \neg\varphi(\bar{x},\bar{m}) \\
            \Rightarrow &\text{Diag}(\mathcal{M})\cup T
              &\models\forall\bar{x}\; \neg\varphi(\bar{x},\bar{m})
              &(\because\text{Diag}(\mathcal{M})\cup T\models
              T_\mathcal{M}\; \text{by completeness assumption}) \\
            \Rightarrow &\text{Diag}(\mathcal{N})\cup T
              &\models\forall\bar{x}\; \neg\varphi(\bar{x},\bar{m})
              &(\because\text{Diag}(\mathcal{M})\subseteq
              \text{Diag}(\mathcal{N})\; \text{from}\;
              \mathcal{M}<\mathcal{N}) \\
            \Rightarrow &\mathcal{N} &\models\forall\bar{x}\;
              \neg\varphi(\bar{x},\bar{m}) \\
            \Rightarrow &\mathcal{N} &\not\models\exists\bar{x}\;
              \varphi(\bar{x},\bar{m}), \\
          \end{array}\]
          a contradiction.
        \end{proof}
    \end{enumerate}

  \item \it Let $\varphi$ be a sentence in the language of abelian groups.
    As usual for a prime $p$ we denote by $\mathbb{Z}_p$ the additive group
    $\mathbb{Z}/p\mathbb{Z}$.

    \begin{enumerate}[label={\bf(\arabic*)}]
      \item Show that if $\mathbb{Q}\models\varphi$ then there is a prime
        $p$ such that $\mathbb{Z}_p\models\varphi$. (Hint: $(\mathbb{Q},+)$
        is a torsion-free divisible abelian group.)

        \begin{proof}
          The language of abelian groups is $\mathcal{L}=\{+,-,0\}$.
          Also, the theory of abelian groups $T_\text{Ab}$ can be
          considered a conjunction of a finite set of $\forall$-sentences.
          Then the theory of torsion-free divisible abelian groups
          $T_\text{tor-free}$ is $T_\text{Ab}\cup T_0 \cup T_1$, where
          \begin{align*}
            T_0 &:= \bigcup_{n\in\omega}
              \{\underbrace{x+\ldots+x}_{n\text{-times}}=0 \rightarrow
              x=0\}, \\
            T_1 &:= \bigcup_{n\in\omega}
              \{\forall x\exists y\;
              \underbrace{y+\ldots+y}_{n\text{-times}}=x\}. \\
          \end{align*}

          Now $T_\text{tor-free}$ is $\kappa$-categorical for all
          $\kappa>\omega$ (Theorem 4.35), so since $T_0$ ensures that there
          are no finite-models, $T_\text{tor-free}$ is a complete theory by
          Vaught's test (Theorem 4.34). \\

          Let $\varphi$ be a sentence in the language such that
          $\mathbb{Q}\models\varphi$. Then
          $T_\text{tor-free}\models\varphi$ from completeness of
          $T_\text{tor-free}$, thus from compactness theorem, there must be
          a finite subset of formulas $T_\text{fin}\subset
          T_\text{tor-free}$ such that $T_\text{fin}\models\varphi$.
          Without loss of generality, we can assume
          \begin{align*}
            T_\text{fin} &= T_\text{Ab}\cup \bigcup_{n=1}^N
              \{\underbrace{x+\ldots+x}_{n\text{-times}}=0 \rightarrow
              x=0\} \cup\bigcup_{n=1}^N
              \{\forall x\exists y\;
              \underbrace{y+\ldots+y}_{n\text{-times}}=x\}. \\
          \end{align*}
          Then by choosing prime $p>N$, we will have $\mathbb{Z}_p\models
          T_\text{fin}$. Then $\mathbb{Z}_p\models\varphi$, as required.
        \end{proof}

      \item Write a sentence $\psi$ such that $\mathbb{Z}\models\psi$ and
        $\mathbb{Q}\models\neg\psi$.
        \begin{proof}
          Let $\psi$ be the sentence that says ``There is an element that
          is not divisible by 2''. Formally,
          \[\psi:= \exists x\forall y\; (y+y\neq x).\]
        \end{proof}
    \end{enumerate}

  \item \it Let $\varphi$ be an $\forall\exists$-sentence in the language
    of rings. Assume $\varphi$ holds in every finite field. Prove that
    $\varphi$ holds in every algebraically-closed field.

    \begin{proof}
      If $\varphi$ holds in every algebraically-closed field of non-zero
      characteristic, then it will also hold in every algebraically-closed
      field of characteristic 0 (Corollary 4.42). Thus it suffices to prove
      that $\varphi$ holds in the former. Also, for an arbitrary prime $p$,
      if $\varphi$ holds in one algebraically-closed field of
      characteristic $p$, then $\varphi$ will also hold in all
      algebraically-closed fields of characteristic $p$, because
      $\text{ACF}_p$ is $\kappa$-categorical for all $\kappa>\omega$ and
      does not have finite models from closure property, and thus is a
      complete theory from Vaught's test. Thus, it suffices to show that
      given arbitrary prime $p$, $\varphi$ holds in one model of
      $\text{ACF}_p$. \\

      In particular, we show that $\varphi$ holds in
      $\overline{\mathbb{F}}_p$, the algebraic-closure of the field
      $\mathbb{F}_p$ of $p$ elements. Now \[\overline{\mathbb{F}}_p =
      \bigcup_{n\in\mathbb{N}^+} \mathbb{F}_{p^n},\] where
      $\mathbb{F}_{p^n}$ is the unique (up to ring-isomorphism) field with
      $p^n$ elements that embeds $\mathbb{F}_{p^{n-1}}$. Since each
      $\mathbb{F}_{p^n}$ is finite, $\varphi$ holds in each
      $\mathbb{F}_{p^n}$. So we have a chain of models
      \[\mathbb{F}_{p^1}< \mathbb{F}_{p^2}< \ldots\] of the
      $\forall\exists$-theory $\{\varphi\}$. Then from inductiveness of
      $\{\varphi\}$ (Corollary 11.8), the union $\overline{\mathbb{F}}_p$
      will also be a model of $\{\varphi\}$, thus
      $\overline{\mathbb{F}}_p\models\varphi$, as we wished to show.
    \end{proof}

  \item \it Let $T$ be the theory of $\langle\mathbb{R};<,Q\rangle$, where
    $Q$ is a unary predicate for rational numbers. Does $T$ have a prime
    model?

    \begin{proof}
      We prove the stronger assertion that $T$ is $\omega$-categorical.
      Then $S_n(T)$ is finite for every $n\in\mathbb{N}^+$
      (Ryll-Nardjewski), so the isolated types are dense in $S_n(T)$ for
      every $n\in\mathbb{N}^+$ (Theorem 7.17), thus $T$ has a prime model
      (Theorem 7.12). \\

      To prove $\omega$-categoricity, notice that $T$ includes the
      following sets of sentences:
      \begin{enumerate}
        \item $T_\text{DLO}$, which are the sentences what say ``I am a
          dense linear order (DLO) without endpoints''. $T$ includes
          $T_\text{DLO}$ since $\langle\mathbb{R},<\rangle$ is a DLO
          without endpoints.

        \item $T_{\mathbb{Q}\text{-DLO}}$, which are the sentences what say
          ``the set of rationals is a DLO without endpoints''.
          $T_{\mathbb{Q}\text{-DLO}}$ can be obtained from $T_\text{DLO}$
          by replacing each occurrence of any variable $x$ with
          ``$Q(x)\rightarrow$''.

        \item $T_{\text{dense}}$, which is the conjunction of eight
          sentences saying between a rational (or irrational) and a
          rational (or irrational) lies another rational (or irrational).
      \end{enumerate}

      We show that $T':=T_\text{DLO}\cup T_{\mathbb{Q}\text{-DLO}}\cup
      T_{\text{dense}}$ is $\omega$-categorical, which would imply that $T$
      is $\omega$-categorical, as desired. Let $\mathcal{M}$, $\mathcal{N}$
      be countable models of $T'$. Note that from density of the rationals
      as expressed in $T_{\mathbb{Q}\text{-DLO}}$,
      $|Q(M)|=|Q(N)|=\omega$. Also, from density of the irrationals as
      expressed in $T_{\text{dense}}$, we also have $|\neg Q(M)|=|\neg
      Q(N)|=\omega$. Write
      \[\begin{array}{ll}
        M=\{m_k\}_{k\in\omega},\; &N=\{n_k\}_{k\in\omega}.
        \\
      \end{array}\]

      We use back-and-forth theorem to construct an isomorphism
      $i=\cup_{r\in\omega}i_r$ between $M$ and $N$. Set $i_0=\emptyset$.
      The idea is to cycle through four steps, where at step $4r$ for
      $r\in\mathbb{N}^+$, we map an unmapped rational in $M$ to an unmapped
      rational in $N$; at step $4r+1$, we map an unmapped rational in $N$
      to an unmapped rational in $M$; at step $4r+2$ we map an unmapped
      irrational in $M$ to an unmapped irrational in $N$; at step $4r+3$,
      we map an unmapped irrational in $N$ to an unmapped irrational in
      $M$. In extending the partial-isomorphism, we take care to preserve
      ordering. \\

      More precisely, at step $4r$ where $r\in\mathbb{N}^+$, let
      $k\in\mathbb{N}$ be the smallest index such that $m_k$ is a rational,
      $m_k\not\in\text{dom}(i_r)$, and let $k_0,k_1\in\mathbb{N}$ be the
      smallest indices whose elements are are in $\text{dom}(i_r)$ and
      which are closest to $m_k$, i.e. $m_{k_0}<m_k<m_{k_1}$ and
      $m_{k_0},m_{k_1}\in\text{dom}(i_r)$. Then by construction, none of
      the elements between $i_r(m_{k_0})$ and $i_r(m_{k_1})$ will be in the
      range of $i_r$. Pick any rational $n_{k'}$ between them to be the
      image of $m_k$; such a rational must exist from density of rationals.
      Extend $i_{4r+1}=i_{4r}\cup\{(m_k,n_{k'})\}$. \\

      At step $4r+1$, we exchange the roles of $M$ and $N$ as described in
      step $4r$, and build $i_{4r+2}$ over $i_{4r+1}$. At step $4r+2$, we
      replace ``rational'' by ``irrational'' in step $4r$, building
      $i_{4r+3}$ over $i_{4r+2}$. Finally, at step $4r+3$, we replace
      ``rational'' by ``irrational'' in step $4r+2$, building $i_{4r+4}$
      over $i_{4r+3}$. \\

      Taking union $i=\cup_{r\in\omega}i_r$ will give us an
      order and rationality-preserving isomorphism between $M$ and $N$,
      thus $T$ is $\omega$-categorical.
    \end{proof}

  \item \it Let $\mathcal{L}$ be a language. Assume $\mathcal{L}$ has a
    binary predicate symbol $P(x,y)$. Let $T$ be a universal
    $\mathcal{L}$-theory such that $T\models\forall x\exists y\; P(x,y)$.
    Prove that there exists $\mathcal{L}$-terms $t_1(x),\ldots,t_n(x)$ such
    that \[T\models\forall x\; \bigvee_{m=1}^n P(x,t_m(x)).\]
    \begin{proof}
      Assume by contradiction that for any finite set of terms
      $t_1(x),\ldots,t_m(x)$ in variable $x$,
      \begin{equation}
        \tag{$*$}
        T\models\exists x\; \bigwedge_{m=1}^n \neg P(x,t_m(x).
        \label{eqn:neg}
      \end{equation}
      Then there is a model $\mathcal{M}$ of $T$ satisfying all the
      existential sentences given at equation~\eqref{eqn:neg}. Note that
      because $T\models\forall x\exists y\; P(x,y)$, every element $a\in M$
      has a witness $a'\in M$ such that $P(a,a')$. \\

      There are two possible cases for a model $\mathcal{M}$ satisfying all
      the equations in \eqref{eqn:neg}. In the first case, there is an
      element $a\in M$ whose witness $a'\in M$ of $P(a,a')$ is not any of the
      terms $t(a)$ of $a$. In this case, consider the substructure $\langle
      a\rangle<\mathcal{M}$, which is the smallest substructure of
      $\mathcal{M}$ containing $a$, defined rigorously as
      \begin{align*}
        \langle a\rangle &=\{t^\mathcal{M}(a):
          t^\mathcal{M}(x)\; \text{is an}\; \mathcal{L}\text{-term}\}. \\
      \end{align*}

      Note that $\langle a\rangle$ defined above will be closed under
      functions since the composition of terms in variable $x$ remains a
      term in variable $x$. Thus $\langle a\rangle$ is a substructure of
      $\mathcal{M}$. Then since $T$ is a universal theory, it is closed
      under substructures (Theorem 11.3), thus $\langle a\rangle$ is also a
      model of $T$. However, $a\in\langle a\rangle$ will have no witness to
      $P(a,y)$ because its witness is not $t(a)$ for any $\mathcal{L}$-term
      $t(x)$. Thus $\langle a\rangle\models\forall y\; \neg P(a,y)$,
      contradicting $T\models\forall x\exists y\; P(x,y)$. \\

      In the second case, every element $a\in M$ has a witness $a'\in M$
      for $P(a,a')$ such that $a'=t(a)$ for some $\mathcal{L}$-term $t(x)$.
      We construct an elementary extension $\mathcal{N}\succ\mathcal{M}$
      such that $\mathcal{N}$ falls under case 1 above. Add to language
      $\mathcal{L}(\mathcal{M})$ a new constant symbol $c$, i.e.
      $\mathcal{L}':=\mathcal{L}(\mathcal{M})\cup\{c\}$. In this extended
      language, consider the theory
      \[T':=T_\mathcal{M}\cup \{\neg P(c,t(c)):t^\mathcal{M}(x)\;
        \text{is an}\; \mathcal{L}\text{-term}\}.\]

      Then $T'$ is finitely satisfiable by $\mathcal{M}$: Clearly
      $\mathcal{M}$ satisfies $T_\mathcal{M}$. Given any finite set of
      $\mathcal{L}$-terms $t_1(x),\ldots,t_m(x)$ in variable $x$, because
      $\mathcal{M}$ satisfies equation~\eqref{eqn:neg}, there must be an
      element $a\in M$ that satisfies $\neg P(a,t_i(a))$ for all
      $i\in\{1,\ldots,m\}$. Assigning $c$ to this element $a$, we will have
      $\mathcal{M}\models T'$. Thus by Compactness theorem, $T'$
      has a model $\mathcal{N}$, which will also be a model of $T$ since
      $\mathcal{N}\models T_\mathcal{M}\supset T$. However in
      $\mathcal{N}$, the constant element $c$ does not have a witness $c'$
      of $P(c,c')$ which is a term $t(c)$ of $c$. Thus $\mathcal{N}$ is the
      same as case 1 above, where we defined a substructure that
      gives a contradiction.
    \end{proof}

  \item \it Let $\mathcal{M}$ be an infinite structure. Show that there is
    a structure $\mathcal{N}$ elementarily equivalent to $\mathcal{M}$ such
    that every definable subset $A\subseteq N$ is either finite or
    uncountable.

    \begin{proof}
      We construct an elementary chain of structures
      \[\mathcal{M}=\mathcal{M}_0\preceq \mathcal{M}_1\preceq\ldots,\]
      such that for each $n\in\omega$, every subset $A\subseteq M_{n+1}$
      that is definable over elements in $M_n$ is either finite or
      uncoutable. More precisely, given $n\in\omega$, formula
      $\varphi(\bar{x},\bar{y})$, and $\bar{m}\in M_n$, we have
      $|\varphi(M_{n+1},\bar{m})|$ is either finite or uncountable. Take
      \[\mathcal{N}=\bigcup_{n\in\omega}\mathcal{M}_n.\]

      Then $\mathcal{M}\preceq\mathcal{N}$ by Tarski's Chain Lemma, thus
      $\mathcal{N}$ is elementarily equivalent to $\mathcal{M}$. Also,
      every definable subset $A\subseteq N$ is either finite or
      uncountable: Given arbitrary formula $\varphi(\bar{x},\bar{y})$ and
      $\bar{a}\in N$, we want to show that $|\varphi(N,\bar{a})|$ is
      either finite or uncountable. Now all elements in $\bar{a}$ must be
      constructed at some stage $n\in\omega$, i.e. $\bar{a}\in M_n$ for
      some $n\in\omega$. Then $|\varphi(M_{n+1},\bar{a})|$ is either finite
      or uncountable. Now since $\mathcal{M}_{n+1}\preceq\mathcal{N}$ by
      Tarski's Chain Lemma, if $|\varphi(M_{n+1},\bar{a})|=k$ is finite, then
      $\mathcal{M}_{n+1}$ and thus $\mathcal{N}$ will satisfy the first
      order formula that says $\varphi(x,\bar{a})$ has exactly $k$
      solutions, and therefore $|\varphi(N,\bar{a})|=k$ is also finite. On
      the other hand, if $|\varphi(M_{n+1},\bar{a})|>\omega$, then there
      will be uncountably many elements in $M_{n+1}$ and thus in $N$ that
      satisfy $\varphi(x,\bar{a})$, thus $|\varphi(N,\bar{a})|$ is also
      uncountable. Hence $|\varphi(N,\bar{a})|$ is either finite or
      uncountable, as required. \\

      We now construct the chain. Given $\mathcal{M}_n$, formula
      $\varphi(\bar{x},\bar{y})$, and $\bar{a}\in\mathcal{M}_n$, we wish to
      construct $\mathcal{M}_{n+1}\succeq\mathcal{M}_n$ with the above
      described properties. Note that if $|\varphi(M_n,\bar{a})|$ is finite
      or uncountable, then like in the argument in the previous paragraph,
      $|\varphi(M_{n+1},\bar{a})|$ will remain finite or uncountable. Thus
      in constructing $\mathcal{M}_{n+1}$, we need not worry about formulas
      with finite or uncountable solutions. \\

      We consider the formulas $\varphi(\bar{x},\bar{a})$ with countable
      solutions in $\mathcal{M}_n$. We wish to add more solutions to make
      the total number uncountable in $\mathcal{M}_{n+1}$. For each such
      formula, add to the language an uncoutable set of constants
      $\{c_{\varphi,\bar{a},i}\}_{i\in\omega_1}$, and consider the theorem
      \[T_{\varphi,\bar{a}}:= \{\varphi(c_{\varphi,\bar{a},i}):
        i\in\omega_1\} \cup \{c_{\varphi,\bar{a},i}\neq
        c_{\varphi,\bar{a},j}: i<j\in\omega_1\}.\]

      Let $T$ be the theorem
      \[T:= T_{\mathcal{M}_n}\cup \bigcup_{|\varphi(M_n,\bar{a})|=\omega}
        T_{\varphi,\bar{a}}.\]
      Then $\mathcal{M}_n$ finitely satisfies $T$, because each involved
      $\varphi(\bar{x},\bar{a})$ has countable solutions in
      $\mathcal{M}_n$. Thus by compactness theorem, $T$ has a model
      $\mathcal{M}_{n+1}$. This model elementary extends $\mathcal{M}_n$
      since $T_{\mathcal{M}_n}\subseteq T$. Also, each formula
      $\varphi(x,\bar{a})$ will now have either finite solutions, or
      uncountable distinct solutions
      $\{c_{\varphi,\bar{a},i}\}_{i\in\omega_1}$, as desired.
    \end{proof}

  \item \it Let $\mathcal{L}=\{<\}$ be the language for orders.
    \begin{enumerate}[label={\bf(\arabic*)}]
      \item Show that there is a structure $(A,<)$ elementarily equivalent
        to $\omega$ that is not well-ordered.

        \begin{proof}
          We construct an elementary extension of $\omega$. We add to the
          language $\mathcal{L}$ constants for $\omega$ and also for a new
          chain $\mathbb{Q}$:
          \begin{align*}
            \mathcal{L}' &:=\mathcal{L}(\omega)\cup\{c_q':q\in\mathbb{Q}\}
              \\
            &=\mathcal{L}\cup\{c_n:n\in\omega\}\cup\{c_q':q\in\mathbb{Q}\}
              \\
          \end{align*}
          where the $c_n$'s are the constants for $\omega$ and the $c_q'$
          are the constants for the new $\mathbb{Q}$-chain. \\

          Consider the theory
          \begin{align*}
            T &:= T_\omega \\
            &\cup \{c_n\neq c_q':n\in\omega, q\in\mathbb{Q}\} \\
            &\cup \{c_{q_0}'\neq c_{q_1}':q_0<q_1\in\mathbb{Q}\} \\
            &\cup \{c_{q_0}'<c_{q_1}':q_0<q_1\in\mathbb{Q}\} \\
          \end{align*}
          Then $T$ is finitely satisfied by $\omega$: Given a finite subset
          of formulas in $T$, we can assume without loss of generality that
          $c_{n_1}<\ldots<c_{n_k}$ and $c_{q_{k+1}}'<\ldots<c_{q_{k+m}}'$
          are the only constants that appear in the formulas. For each
          $i\in\{1,\ldots,m\}$, assign the constant $c_{q_{k+i}}'$ to
          $n_{k}+i\in\omega$. This assignment will ensure that $\omega$
          satisfies the subset of formulas. \\

          Thus by compactness theorem, $T$ is satisfiable by some model
          $\mathcal{A}$. Since $T$ contains $T_\omega$, $\mathcal{A}$ will
          be an elementary extension of $\omega$, and will therefore be
          elementarily equivalent to $\omega$. However, $A$ is not
          well-ordered because $\langle\mathbb{Q},<\rangle$ is not well-ordered. In
          particular, the elements associated with the set of constants
          \[\left\{c_{1/n}':n\in\omega\right\}\] does not have a minimal
          element.

          %We construct an elementary extension of $\omega$. The idea of the
          %extension is to append to $\omega$ a countable chain of integers,
          %where the set of chains are indexed by positive rationals. The
          %extension will not be well-orderable because positive rationals
          %are not well-ordered. More precisely, we would like the
          %elementary extension to contain
          %\[\omega\; \bigsqcup_{q\in\mathbb{Q}^+}\; \mathbb{Z}.\]

          %We write the elements of $\omega$ as
          %$\omega:=\{a_{0,n}\}_{n\in\omega}$, and the elements in the $q$th
          %chain of integers $\mathbb{Z}_q$ for each $q\in\mathbb{Q}^+$ as
          %\[\mathbb{Z}_q :=\{a_{q,z}\}_{z\in\mathbb{Z}}.\]
          %Then we define the ordering on $\omega\;
          %\sqcup_{q\in\mathbb{Q}^+}\; \mathbb{Z}$ to be
          %\[a_{q,z}<a_{q',z'} \Leftrightarrow (q<q')\vee(q=q'\wedge z<z').\]

          %To construct an elementary extension with such a structure, we
          %add to the language new constants
          %\[\mathcal{L}':= \mathcal{L}\cup\]
        \end{proof}

      \item Let $\aleph_1$ be the first uncountable cardinal. Show that
        $\aleph_1$ and $\omega$ are not elementarily equivalent (as ordered
        sets).
        \begin{proof}
          Consider the sentence $\varphi$ which says that ``I have no limit
          elements''. More rigorously, this sentence says ``every element
          other than the smallest one has an immediate predecessor''. This
          sentence is satisfied by $\omega$ but not by $\aleph_1$, because
          $\aleph_0\in\aleph_1$ does not have a immediate predecessor. We
          can write $\varphi$ in first-order language as
          \[\varphi:= \forall x\; (\exists y\; y<x)\rightarrow\;
            \left[\exists z\; (z<x)\wedge \forall w\; (w<x\rightarrow w\leq
            z) \right].\]
        \end{proof}

      \item Show that some countable ordinal $\beta$ is an elementary
        substructure of $\aleph_1$. (Notice that $\beta$ must be an initial
        segment of $\aleph_1$.

        \begin{proof}
          We repeatedly apply Downward Lowenheim-Skolem to construct a
          chain of countable substructures
          \begin{equation*}
            \tag{$**$}
            \mathcal{M}_0\leq \mathcal{M}_1\leq\ldots
            \label{eqn:chain}
          \end{equation*}
          of $\aleph_1$ such that for all $n\in\mathbb{N}$,
          $\mathcal{M}_n\prec\aleph_1$, and $M_{n+1}\supseteq\beta_n$,
          where $\beta_n=\sup M_n$. We start by taking $\mathcal{M}_0$ to
          be any countable elementary substructure of $\aleph_1$. Such
          $\mathcal{M}_0$ exists from Downward Lowenheim-Skolem. Then for
          each $n\in\mathbb{N}$, let $\mathcal{M}_{n+1}$ be a countable
          elementary substructure of $\aleph_1$ that contains the ordinal
          $\beta_n$. Again, $\mathcal{M}_{n+1}$ exists from Downward
          Lowenheim-Skolem. \\

          Take $\beta=\cup_{n\in\omega}\mathcal{M}_n$, the union of this
          chain. To show that this $\beta$ works, note that by
          construction, $\beta=\cup_{n\in\omega}\beta_n$, so $\beta$ will
          be an initial segment of $\aleph_1$ since each $\beta_n$ is an
          initial segment of $\aleph_1$. Also, because each $\mathcal{M}_n$
          is countable, each $\beta_n$ must also be countable by
          regularlity of $\aleph_1$, otherwise the enumeration of elements
          in $M_n$ will be cofinal in $\aleph_1$. Thus, $\beta$ is a union
          of a countable set of countable initial segments, and is
          therefore a countable ordinal. \\

          Next, we show that because $\mathcal{M}_n\prec\aleph_1$ for all
          $n\in\mathbb{N}$, the chain of structures given in
          \eqref{eqn:chain} is an elementary chain: Given an arbitrary
          formula $\varphi(\bar{x})$ and $\bar{a}\in M_n$, we have
          \[\begin{array}{crll}
            &\mathcal{M}_n &\models\varphi(\bar{a}) & \\
            \Leftrightarrow &\aleph_1 &\models\varphi(\bar{a})
              &(\because\mathcal{M}_n\prec\aleph_1) \\
            \Leftrightarrow &\mathcal{M}_{n+1} &\models\varphi(\bar{a})
              &(\because\mathcal{M}_{n+1}\prec\aleph_1), \\
          \end{array}\]
          thus $\mathcal{M}_n\preceq\mathcal{M}_{n+1}$ for all
          $n\in\mathbb{N}$. Hence by Tarski's Chain Lemma,
          $\mathcal{M}_n\preceq\beta$ for all $n\in\mathbb{N}$. \\

          Finally, we show that $\beta$ is an elementary substructure of
          $\aleph_1$. Let $\varphi(\bar{x})$ be an arbitrary formula and
          $\bar{b}\in\beta$ an arbitrary tuple of elements in $\beta$.  All
          elements in the tuple $\bar{b}\in\beta$ must be constructed at
          some stage $n\in\mathbb{N}$, i.e. there is some $n\in\mathbb{N}$
          such that $\bar{b}\in M_n$. Then we have
          \[\begin{array}{crll}
            &\beta &\models\varphi(\bar{b}) & \\
            \Leftrightarrow &\mathcal{M}_n &\models\varphi(\bar{b})
              &(\because\mathcal{M}_n\prec\beta) \\
            \Leftrightarrow &\aleph_1 &\models\varphi(\bar{b})
              &(\because\mathcal{M}_{n}\prec\aleph_1), \\
          \end{array}\]
          thus $\beta\preceq\aleph_1$ as desired. \\
        \end{proof}
    \end{enumerate}
\end{enumerate}
\end{document}
