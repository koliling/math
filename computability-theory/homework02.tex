\documentclass{article}
\usepackage[left=3cm,right=3cm,top=3cm,bottom=3cm]{geometry}
\usepackage{amsmath,amssymb,amsthm,tikz,mathtools}
\usepackage{color}
\usepackage[inline]{enumitem}
\usetikzlibrary{patterns}
\setlength{\parindent}{0mm}
\newcommand{\TODO}[1]{\textcolor{red}{TODO: #1}}

\begin{document}
\title{Basic Logic II: Problem Set 2}
\author{Li Ling Ko\\ lko@nd.edu}
\date{\today}
\maketitle

\begin{enumerate}[label={\bf Q\arabic*:}]
  \item \it Show that $H\leq_1K$.
    \begin{proof}
      Consider the function
      \begin{equation*}
        \psi(e,x,y) :=
        \begin{cases}
          0, &\text{if}\; (e,x)\in H\\
          \uparrow, &\text{otherwise}\\
        \end{cases}.
      \end{equation*}

      Note that $\psi(e,x,y)$ is recursive because $H$ is recursively
      enumerable. Thus by the s-m-n theorem, there is a primitive
      recursive function $f(e,x)$ such that
      \[\psi(e,x,y)=\varphi_{f(e,x)}(y).\]

      By Lemma 25 we can assume that $f(e,x)$ is injective. Note that
      $\varphi_{f(e,x)}$ is the 0 function if and only if $(e,x)\in H$,
      otherwise it is the function that never halts on any input. Thus
      \begin{align*}
        (e,x)\in H &\Leftrightarrow \varphi_{f(e,x)}\; \text{is the 0
          function}\\
        &\Leftrightarrow f(e,x)\in K.\\
      \end{align*}
      Since $f(e,x)$ is injective, we have $H\leq_1K$.
    \end{proof}
\end{enumerate}
\end{document}
