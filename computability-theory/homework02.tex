\documentclass{article}
\usepackage[left=3cm,right=3cm,top=3cm,bottom=3cm]{geometry}
\usepackage{amsmath,amssymb,amsthm,tikz,mathtools}
\usepackage{color}
\usepackage[inline]{enumitem}
\usetikzlibrary{patterns}
\setlength{\parindent}{0mm}
\newcommand{\TODO}[1]{\textcolor{red}{TODO: #1}}

\begin{document}
\title{Basic Logic II: Problem Set 2}
\author{Li Ling Ko\\ lko@nd.edu}
\date{\today}
\maketitle

\begin{enumerate}[label={\bf Q\arabic*:}]
  \item \it Show that $H\leq_1K$.
    \begin{proof}
      Consider the function
      \begin{equation*}
        \psi(e,x,y) :=
        \begin{cases}
          0, &\text{if}\; (e,x)\in H\\
          \uparrow, &\text{otherwise}\\
        \end{cases}.
      \end{equation*}

      Note that $\psi(e,x,y)$ is recursive because $H$ is recursively
      enumerable. Thus by the s-m-n theorem, there is a primitive
      recursive function $f(e,x)$ such that
      \[\psi(e,x,y)=\varphi_{f(e,x)}(y).\]

      By Lemma 25 we can assume that $f(e,x)$ is injective. Note that
      $\varphi_{f(e,x)}$ is the 0 function if and only if $(e,x)\in H$,
      otherwise it is the function that never halts on any input. Thus
      \begin{align*}
        (e,x)\in H &\Leftrightarrow \varphi_{f(e,x)}\; \text{is the 0
          function}\\
        &\Leftrightarrow f(e,x)\in K.\\
      \end{align*}
      Since $f(e,x)$ is injective, we have $H\leq_1K$.
    \end{proof}

  \item \it Disjoint sets $A$ and $B$ are computably inseparable if there
    is no computable set $C$ such that $A\subseteq C$ and $C\cap
    B=\emptyset$. Show that $A=\{e|\varphi_e(e)\downarrow=0\}$ and
    $B=\{e|\varphi_e(e)\downarrow\neq0\}$ are computably inseparable. Prove
    that $K\equiv_1A\equiv_1B$. $\text{Ext}$ is the index set of all
    partial computable functions which can be extended to a total
    computable function. Show that $\text{Ext}$ is not $\emptyset$ or
    $\omega$.

    \begin{proof}
      Assume that $C$ computably separates $A$ from $B$. We show that this
      would make $K$ computable, a contradiction. Consider the function
      \begin{equation*}
        \psi(e,x) :=
        \begin{cases}
          0, &\text{if}\; e\in K\\
          \uparrow, &\text{otherwise}\\
        \end{cases}.
      \end{equation*}

      Note that $\psi(e,x)$ is recursive because $K$ is recursively
      enumerable. Thus by the s-m-n theorem, there is a primitive
      recursive function $f(e)$ such that
      \[\psi(e,x)=\varphi_{f(e)}(x).\]

      Observe that $\varphi_{f(e)}$ is the 0 function if and only if $e\in
      K$, otherwise $\varphi_{f(e)}$ is the function that never halts on
      any input. Thus given any $e\in\mathbb{N}$, we can decide if $e$ lies
      in $K$ by checking if $f(e)$ lies in $C$. More precisely,
      \[e\in K \Leftrightarrow \chi_C(f(e))=1.\]
      Since $f$ and $\chi_C$ are total computable functions, $K$ will be
      decidable, a contradiction. \\

      To show $K\leq_1A$, consider the function
      $\psi(e,x)=\varphi_{f(e)}(x)$ defined above. We can assume that $f$
      is injective from Lemma 25. Then
      \[e\in K \Leftrightarrow f(e)\in A,\]
      thus $K\leq_1A$. \\

      To show $A\leq_1B$, consider the function
      \begin{equation*}
        \psi(e,x) :=
        \begin{cases}
          1, &\text{if}\; e\in A\\
          \uparrow, &\text{otherwise}\\
        \end{cases}.
      \end{equation*}
      $\psi(e,x)$ is recursive because $A$ is recursively enumerable. Thus
      by the s-m-n theorem and Lemma 25, there is a primitive recursive
      injective function $f(e)$ such that $\psi(e,x)=\varphi_{f(e)}(x)$.
      Then
      \[e\in A \Leftrightarrow f(e)\in B,\]
      thus $A\leq_1B$.

      Finally $B\leq_1K$ because $B$ is recursively enumerable and $K$ is
      1-complete. So from transitivity of $\leq_1$, the sets $A$, $B$, and
      $K$ have the same 1-degree. \\

      Clearly $\text{Ext}$ is non-empty because the partial-recursive
      function that never halts on any input can be extended to the
      total computable 0 function. We show that the partial-recursive
      function $\phi_x(x)$ cannot be extended to a total computable
      function. Otherwise let $f(x)$ be an extension. Consider the set
      $C:=\{e\in\mathbb{N}|\; f(e)=0\}$. Since $f(x)$ is total, set $C$
      will be computable. Also, $A\subset C$ and $B\cap C=\emptyset$ by
      definition. Thus $C$ computably separates $A$ and $B$, a
      contradiction.
    \end{proof}

  \item \it Show that $\text{Fin}\leq_1\text{Cof}$, where $\text{Fin}$ is
    the index set of all finite sets and $\text{Cof}$ is the index set of
    all co-finite sets.

    \begin{proof}
      Given $e,s\in\mathbb{N}$, let $W_{e,s}$ denote the set of indices $i$
      in $\{0,\ldots,s-1\}$ such that $\varphi_i$ halts after $s$ steps.
      Consider the function
      \begin{equation*}
        \psi(e,s) :=
        \begin{cases}
          0, &\text{if}\; |W_{e,s+1}|>|W_{e,s}|\\
          \uparrow, &\text{otherwise}\\
        \end{cases}.
      \end{equation*}
      Note that $\psi(e,s)$ is recursive since $W_{e,s}$ is computable.
      Thus by the s-m-n theorem and Lemma 25, there is a primitive
      recursive function and injective function $f(e)$ such that
      \[\psi(e,s)=\varphi_{f(e)}(s).\]

      Also, note that for a fixed $e\in\mathbb{N}$, $W_{e,s}$ is
      nondecreasing in $s$. Furthermore, $|W_{e,s}|$ converges in $s$ if
      and only if $e\in\text{Fin}$. Thus
      \begin{align*}
        e\in\text{Fin} &\Leftrightarrow f(e)\in\text{Cof}.\\
      \end{align*}
      Then since $f$ is injective, we have $\text{Fin}\leq_1\text{Cof}$.
    \end{proof}

  \item \it (Soare 1.6.27) A set $A$ is a cylinder if $(\forall B)\;
    [B\leq_m B \Rightarrow B\leq_1A]$.
    \begin{enumerate}[label={\bf (\roman*)}]
      \item Show that any index set is a cylinder.
        \begin{proof}
        \end{proof}

      \item Show that any set of the form $A\times\omega$ is a cylinder.
        \begin{proof}
        \end{proof}

      \item Show that $A$ is a cylinder iff $A\equiv_1B\times\omega$ for
        some set $B$.
        \begin{proof}
        \end{proof}
    \end{enumerate}
\end{enumerate}
\end{document}
