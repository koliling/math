\documentclass{article}
\usepackage[left=3cm,right=3cm,top=3cm,bottom=3cm]{geometry}
\usepackage{amsmath,amssymb,amsthm,tikz,mathtools}
\usepackage{color}
\usepackage[inline]{enumitem}
\usetikzlibrary{patterns}
\setlength{\parindent}{0mm}
\newcommand{\TODO}[1]{\textcolor{red}{TODO: #1}}

\begin{document}
\title{Basic Logic II: Problem Set 3}
\author{Li Ling Ko\\ lko@nd.edu}
\date{\today}
\maketitle

\begin{enumerate}[label={\bf Q\arabic*:}]
  \item \it Working in $\mathbb{N}$ show that for every $k\in\mathbb{N}$ and
    for every subset $F\subseteq[0,k]$, there are $m,n\in\mathbb{N}$ such
    that for every $i\leq k$, $i\in F \Leftrightarrow m(i+1)+1|n$.

    \begin{proof}
      Choose $m=k!$. We first show that $m+1,\ldots,m(k+1)+1$ are pairwise
      coprime: Assume that $mx+1$ and $my+1$ are not coprime for some
      distinct $x,y\in\{1,\ldots,k+1\}$. We can assume without loss of
      generality that $x>y$. Let $p$ be the smallest prime that divides
      $mx+1$ and $my+1$. Then $p$ must also divide $(mx+1)-(my+1)=m(x-y)$.
      But $x-y\in\{1,\ldots,k\}$ and $m=k!$, so $p$ must divide one of
      $\{2,\ldots,k\}$, and therefore divide $mx$ and $my$. Thus $p$,
      cannot divide $mx+1$ and $my+1$, a contradiction. \\

      So the integers $m+1,\ldots,m(k+1)+1$ are pairwise coprime. Consider
      the system of $k+1$ equations in $n$:
      \[\begin{array}{llll}
        n &\equiv &\delta(0\not\in F) &\mod{m+1}\\
        n &\equiv &\delta(1\not\in F) &\mod{2m+1}\\
        \ldots &&&\\
        n &\equiv &\delta(k\not\in F) &\mod{(k+1)m+1},\\
      \end{array}\]
      where
      \[\delta(x)
      :=\begin{cases}1,\text{if}\;x,\\0,\text{otherwise}\end{cases}.\]

      By the Chinese remainder theorem, there is a solution for $n$. This
      $n$ will satisfy the requirement we are looking for.
    \end{proof}

  \item \it Let $\varphi$ be $\sum_1$ in $\mathbb{N}$. Consider in
    $\mathbb{N}$ the definable set $W=\{a|\varphi(a)\}$. Is $W$ computably
    enumerable? Why or why not?

    \begin{proof}
      Yes $W$ is computably enumerable. If $W$ is finite then it is
      trivially computably enumerable. So assume $W$ is infinite. Now since
      $\varphi$ is $\sum_1$, $\varphi(x)$ is equivalent to $\exists
      y\;\phi(x,y)$ for some $\Delta_0$-formula $\phi$. Fix any
      computable permutation
      $p:\mathbb{N}\rightarrow\mathbb{N}\times\mathbb{N}$, and let $\pi_i$
      denote the $i$th projection function. Consider the function
      $f(n):\mathbb{N}\rightarrow\mathbb{N}$ defined recursively as
      follows: \\

      At input $0$, we find the first $s_0\in\mathbb{N}$ such that
      $\phi(\pi_1(p(s_0)),\pi_2(p(s_0)))$ holds, and output
      $\pi_1(p(s_0))$. Then, at input $n+1$, we find the next
      $s_{n+1}>s_{n}$ such that $\phi(\pi_1(p(s_{n+1})),\pi_2(p(s_{n+1})))$
      holds, and output $\pi_1(p(s_{n+1}))$. Note that the $s_{n}$'s exists
      because $W$ is infinite. Also, given $m,n\in\mathbb{N}$, checking if
      $\phi(m,n)$ holds is al algorithmic procedure that always terminates
      since $\phi$ is $\Delta_0$.  So $f$ is defined by an
      algorithmic procedure and produces an output for every input,
      therefore it is total recursive by the Church-Turing thesis. Also,
      since $p$ is surjective, $f$ will enumerate all elements of $W$. 
    \end{proof}

  \item \it Assume $\varphi(x,y)$ is $\Delta_1$ in $\mathbb{N}$ and
    in $\mathbb{N}$, $\varphi$ is total (so for all $x$ there is a unique
    $y$ such that $\varphi(x,y)$). As a function is $\varphi$ computable?
    Why or why not?

    \begin{proof}
      Given a formula $\phi(\bar{x})$, let $\phi(\mathbb{N})$ denote
      $\{\bar{n}\in\mathbb{N}^n|\phi(\bar{n})\}$. We have shown in
      question 2 that $\varphi(\mathbb{N})$ is computably enumerable since
      $\varphi$ is $\sum_1$. Hence for a given $x_0\in\mathbb{N}$, we
      enumerate the $\varphi(\mathbb{N})$'s to find the first tuple where
      the first coordinate is $x_0$, and output the second coordinate.
      This procedure is computable by the Church-Turing thesis, and
      describes $\varphi$ as a function.
    \end{proof}

  \item \it Show that for every computably enumerable theory $T$ there is
    a computable theory $T^*$ such that
    \[\{\varphi:T\vdash\varphi\} = \{\varphi:T^*\vdash\varphi\}.\]

    \begin{proof}
      Let $f:\mathbb{N}\rightarrow\mathcal{L}\text{-formulas}$ be an
      injective and total recursive function that enumerates $T$, and
      consider
      \[T^*:=\{f(0)\wedge\ldots\wedge f(n): n\in\omega\}.\]

      Then clearly $\{\varphi:T\vdash\varphi\}
      =\{\varphi:T^*\vdash\varphi\}$. To check if a given formula $\phi$
      lies in $T^*$, we first decompose $\phi$ into its finite conjunction
      of formulas $\phi=\phi_0\wedge\ldots\wedge\phi_n$, and check if each
      $\phi_i$ is the same as $f(i)$. We know that $\phi\in T^*$ if and
      only if $\phi_i=f(i)$ for all $i\in\{0,\ldots,n\}$. Since the
      described procedure is algorithmic, by the Church-Turing thesis,
      $T^*$ is computable.
    \end{proof}
\end{enumerate}
\end{document}
