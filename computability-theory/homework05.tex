\documentclass{article}
\usepackage[left=3cm,right=3cm,top=3cm,bottom=3cm]{geometry}
\usepackage{amsmath,amssymb,amsthm,tikz,mathtools}
\usepackage{stmaryrd} % For double square bracket [[]]
\usepackage{color}
\usepackage[inline]{enumitem}
\usetikzlibrary{patterns}
\setlength{\parindent}{0mm}
\newcommand{\TODO}[1]{\textcolor{red}{TODO: #1}}

\begin{document}
\title{Basic Logic II: Homework 5}
\author{Li Ling Ko\\ lko@nd.edu}
\date{\today}
\maketitle

\begin{enumerate}[label={\bf Q\arabic*:}]
  \item \it Work in Cantor Space. Construct an open class which is not
    closed. Then a $\prod_2^0$ class which is not open or closed. You might
    want to use compactness.

    \begin{proof}
      Consider the class
      \[\mathcal{C} :=2^\omega-\{0\},\]
      which is the class of all functions from $\omega$ to $\{0,1\}$ minus
      the zero function. This class is open because it is the union of the
      basic clopen classes generated by initial segments with at least one
      value that is 1, i.e.
      \[\mathcal{C} =\llbracket\{\sigma\in2^{<\omega}: \exists n\;
      \sigma(n)=1\}\rrbracket.\]

      However, this class is not closed because its complement is the zero
      function, which cannot be open because any initial segment of the
      zero function will generate a clopen class that includes a nonzero
      function. \\

      Now consider the class of functions $f\in2^\omega$ where
      $|f^{-1}(0)|=\infty$, i.e.
      \[\mathcal{D} :=\{f\in2^\omega: |f^{-1}(0)|=\infty\}.\]

      This class is not open because the clopen class generated by any
      initial segment in $2^\omega$ will always include a function with
      only a finite number of elements mapping to 0.  Similarly, this class
      is not closed because its complement cannot be open - the clopen
      class generated by any initial segment in $2^\omega$ will always
      include a function with an infinite number of elements mapping to 0.
      Finally, this class is $\prod_2^0$ because it can be defined by a
      $\prod_2^0$ sentence as follows:
      \[\mathcal{D} =\{f\in2^\omega:\; \forall n\exists m\; (m>n \wedge
      f(m)=0)\}.\]
    \end{proof}

  \item \it Show that there is no partial computable function $f$ such that
    if $\varphi_e$ is the characteristic function of a finite set $F$ then
    $f(e)=y$ where $F=D_y$.

    \begin{proof}
      Assume such a partial computable function $f$ exists. By the SMN
      theorem, there is a total recursive function $g(e)$ such that
      \begin{align*}
        \varphi_{g(e)}(s) :=
        \begin{cases}
          0, &\text{if}\; \varphi_{e}(e)\; \text{has not halted after}\;
            s\; \text{steps}\\
          1, &\text{otherwise}.
        \end{cases}
      \end{align*}

      Then if $e\in K$ halts, $\varphi_{g(e)}$ will be the characteristic
      function of the finite set $\{1\}$, otherwise $\varphi_{g(e)}$ will
      be the characteristic function of the empty set. Thus $e\in K$ if and
      only if $f(g(e))$ is the index corresponding to the finite set
      $\{1\}$, which contradicts the incomputability of $K$.
    \end{proof}
\end{enumerate}
\end{document}
