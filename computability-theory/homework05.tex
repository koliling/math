\documentclass{article}
\usepackage[left=3cm,right=3cm,top=3cm,bottom=3cm]{geometry}
\usepackage{amsmath,amssymb,amsthm,tikz,mathtools}
\usepackage{stmaryrd} % For double square bracket [[]]
\usepackage{color}
\usepackage[inline]{enumitem}
\usetikzlibrary{patterns}
\setlength{\parindent}{0mm}
\newcommand{\TODO}[1]{\textcolor{red}{TODO: #1}}

\begin{document}
\title{Basic Logic II: Homework 5}
\author{Li Ling Ko\\ lko@nd.edu}
\date{\today}
\maketitle

\begin{enumerate}[label={\bf Q\arabic*:}]
  \item \it Work in Cantor Space. Construct an open class which is not
    closed. Then a $\Pi_2^0$ class which is not open or closed. You might
    want to use compactness.

    \begin{proof}
      Consider the class
      \[\mathcal{C} :=2^\omega-\{0\},\]
      which is the class of all functions from $\omega$ to $\{0,1\}$ minus
      the zero function. This class is open because it is the union of the
      basic clopen classes generated by initial segments with at least one
      value that is 1, i.e.
      \[\mathcal{C} =\llbracket\{\sigma\in2^{<\omega}: \exists n\;
      \sigma(n)=1\}\rrbracket.\]

      However, this class is not closed because its complement is the zero
      function, which cannot be open because any initial segment of the
      zero function will generate a clopen class that includes a nonzero
      function. \\

      Now consider the class of functions $f\in2^\omega$ where
      $|f^{-1}(0)|=\infty$, i.e.
      \[\mathcal{D} :=\{f\in2^\omega: |f^{-1}(0)|=\infty\}.\]

      This class is not open because the clopen class generated by any
      initial segment in $2^\omega$ will always include a function with
      only a finite number of elements mapping to 0.  Similarly, this class
      is not closed because its complement cannot be open - the clopen
      class generated by any initial segment in $2^\omega$ will always
      include a function with an infinite number of elements mapping to 0.
      Finally, this class is $\Pi_2^0$ because it can be defined by a
      $\Pi_2^0$ sentence as follows:
      \[\mathcal{D} =\{f\in2^\omega:\; \forall n\exists m\; (m>n \wedge
      f(m)=0)\}.\]
    \end{proof}

  \item \it Show that there is no partial computable function $f$ such that
    if $\varphi_e$ is the characteristic function of a finite set $F$ then
    $f(e)=y$ where $F=D_y$.

    \begin{proof}
      Assume such a partial computable function $f$ exists. By the SMN
      theorem, there is a total recursive function $g(e)$ such that
      \begin{align*}
        \varphi_{g(e)}(s) :=
        \begin{cases}
          1, &\text{if}\; \varphi_{e}(e)\; \text{halts at step}\; s,\\
          0, &\text{otherwise}.
        \end{cases}
      \end{align*}

      Then if $e\in K$ halts, $\varphi_{g(e)}$ will be the characteristic
      function of some single element set $\{s\}$, otherwise
      $\varphi_{g(e)}$ will be the characteristic function of the empty
      set. Thus $e\in K$ if and only if $f(g(e))$ is the index
      corresponding to a single set $\{s\}$, which contradicts the
      incomputability of $K$.
    \end{proof}

  \item \it The computable code for a computable set $W_e$ is a pair
    $\langle a,b\rangle$ such that $W_e=W_a$ and $W_a\sqcup
    W_b=\mathbb{N}$. Show there is no partial computable function $f$ such
    that if $W_e$ is computable then $f(e)$ is the computable code for
    $W_e$.

    \begin{proof}
      Assume such a partial computable function $f$ exists. By the SMN
      theorem, there is a total recursive function $g(e)$ such that
      \begin{align*}
        \varphi_{g(e)}(x) :=
        \begin{cases}
          0 &\text{if}\; x=0\; \text{and}\; \varphi_e(e)\downarrow,\\
          \uparrow &\text{otherwise}.\\
        \end{cases}
      \end{align*}

      Observe that if $e\in K$, then $W_{g(e)}=\{0\}$, otherwise
      $W_{g(e)}=\emptyset$. In either cases, $W_{g(e)}$ will be computable.
      Therefore to decide if $e\in K$, we first compute $\langle
      e_a,e_b\rangle=f(g(e))$. Then if $e\in K$, 0 will be contained in
      $W_{e_a}$ but not $W_{e_b}$. Otherwise 0 will be contained in
      $W_{e_b}$ but not $W_{e_a}$. Thus we enumerate $W_{e_a}$ and
      $W_{e_b}$ simultaneously and wait for one of them to output 0. If the
      first one outputs 0, then $e\in K$, otherwise the $e\not\in K$. Thus
      $K$ is computable, a contradiction.
    \end{proof}

  \item \it Soare 3.6.9: (Iterated Limit Lemma) Prove that if $n\geq1$,
    then $f\leq_T\emptyset^{(n)}$ iff there is a computable function
    $\hat{f}$ of $(n+1)$ variables such that
    \[f(x) =\lim_{y_1} \lim_{y_2} \ldots \lim_{y_n}\;
    \hat{f}(x,y_1,\ldots,y_n).\]

    \begin{proof}
      We prove by induction on $n$. The base case for $n=1$ is the limit
      lemma (Lemma 3.6.2). For the inductive step $(n+1)$, we first show
      that $\emptyset^{(n+1)}$ has the required form. From induction
      hypothesis, there is a computable function $\hat{g}$ such that
      \[\emptyset^{(n)}(e) =\lim_{y_1} \lim_{y_2} \ldots \lim_{y_{n}}\;
      \hat{g}(e,y_1,\ldots,y_n).\]
      Here we are considering $\emptyset^{(n)}(e)$ as its characteristic
      function $\chi_{\emptyset^{(n)}}(e)$. Consider the computable
      function
      \begin{align*}
        \hat{f}(e,y_1,\ldots,y_{n+1}) :=
        \begin{cases}
          1 &\text{if}\; \Phi_e^{\bar{g}(y_1,\ldots,y_n)}(e)\; \text{halts
            after}\; y_{n+1}\; \text{steps},\\
          0 &\text{otherwise},
        \end{cases}
      \end{align*}
      where $\bar{g}(y_1,\ldots,y_n)(x)$ denotes
      $\hat{g}(x,y_1,\ldots,y_n)$. Note that $\hat{f}$ is total computable
      since $\hat{g}$ is. We show that $\hat{f}$ gives $\emptyset^{(n+1)}$
      the required form. \\

      First, observe that since $\lim_{y_n}
      \hat{g}(x,y_1,\ldots,y_n)$ exists for arbitrary
      $x,y_1,\ldots,y_{n-1}\in\omega$, therefore
      \begin{align*}
        \lim_{y_n} \hat{f}(e,y_1,\ldots,y_{n+1}) =
        \begin{cases}
          1 &\text{if}\; \Phi_e^{\lim_{y_n} \bar{g}(y_1,\ldots,y_n)} (e)\;
            \text{halts after}\; y_{n+1}\; \text{steps},\\
          0 &\text{otherwise},
        \end{cases}
      \end{align*}
      because the left hand side equals $\hat{f}(e,y_1,\ldots,y_{n-1},m)$
      for some large enough $m\in\omega$, and similarly the right hand
      side equals 1 if $\Phi_{e,y_{n+1}}^{\lim_{y_n}
      \bar{g}(y_1,\ldots,y_{n-1},m)}(e)\downarrow$ and 0 otherwise, which
      is the same as $\hat{f}(e,y_1,\ldots,y_{n-1},m)$. Repeating the
      argument, by induction on $n$, we get
      \begin{align*}
        \lim_{y_1}\ldots\lim_{y_n} \hat{f}(e,y_1,\ldots,y_{n+1}) &=
        \begin{cases}
          1 &\text{if}\; \Phi_e^{\lim_{y_1}\ldots\lim_{y_n}
            \bar{g}(y_1,\ldots,y_n)} (e)\; \text{halts after}\; y_{n+1}\;
            \text{steps},\\
          0 &\text{otherwise}.
        \end{cases}\\
        &=
        \begin{cases}
          1 &\text{if}\; \Phi_e^{\emptyset^{(n)}} (e)\; \text{halts
            after}\; y_{n+1}\; \text{steps},\\
          0 &\text{otherwise}.
        \end{cases}\\
      \end{align*}

      Then
      \[e\in\emptyset^{(n+1)} \Leftrightarrow
      \lim_{y_{n+1}}\lim_{y_1}\ldots\lim_{y_n}
      \hat{f}(e,y_1,\ldots,y_{n+1}) =1,\]
      so $\emptyset^{(n+1)}$ has the required form. Thus if
      $h\leq_T\emptyset^{(n+1)}$, then there exists some $e\in\omega$ such
      that
      \begin{align*}
        h(x) &=\Phi_e^{\emptyset^{(n+1)}}(x)\\
        &=\Phi_e^{\lim_{y_{n+1}}\lim_{y_1}\ldots\lim_{y_n}
          \bar{f}(y_1,\ldots,y_{n+1})}(x),\\
      \end{align*}
      where $\bar{f}(y_1,\ldots,y_{n+1})(t)$ denotes
      $\hat{f}(t,y_1,\ldots,y_{n+1})$. Then repeating the earlier argument,
      we can shift the limits out repeatedly to get
      \begin{align*}
        h(x) &=\Phi_e^{\lim_{y_{n+1}}\lim_{y_1}\ldots\lim_{y_n}
          \bar{f}(y_1,\ldots,y_{n+1})}(x)\\
        &=\lim_{y_{n}}\lim_{y_1}\ldots\lim_{y_{n+1}}
          \Phi_e^{\bar{f}(y_1,\ldots,y_{n+1})}(x),\\
      \end{align*}
      which makes $h(x)$ in the required form. \\

      For the converse, assume
      \[h(x) =\lim_{y_{1}}\ldots\lim_{y_{n+1}}
      \hat{h}(x,y_1,\ldots,y_{n+1})\]
      for some recursive $\hat{h}$. Then
      \begin{align*}
        h(x) &=\lim_{y_1} \left(\lim_{y_{1}}\ldots\lim_{y_{n+1}}
          \hat{h}(x,y_1,\ldots,y_{n+1}) \right)\\
        &=\lim_{y_1} \left(\Phi_e^{\emptyset^{(n)}}(x,y_1) \right)\;
          \text{for some}\; e\in\omega &(\text{by induction
          hypothesis}).\\
      \end{align*}

      Now by relativized SMN theorem, there is a total recursive function
      $g(e,x,y)$ such that
      \begin{align*}
        \Phi^{\emptyset^{(n)}}_{g(x,y)}(z) :=
        \begin{cases}
          0 &\text{if}\; \exists z>y\;
            \Phi_e^{\emptyset^{(n)}}(x,z)\downarrow\neq
            \Phi_e^{\emptyset^{(n)}}(x,y),\\
          \uparrow &\text{otherwise}.\\
        \end{cases}
      \end{align*}

      Observe that $\Phi^{\emptyset^{(n)}}_{g(x,y)}$ is the zero function
      if and only if $y$ is smaller than the least modulus of
      $\Phi_e^{\emptyset^{(n)}}(x,y_1)$ at $x$, seen as a function of
      $y_1$. Otherwise, $\Phi^{\emptyset^{(n)}}_{g(x,y)}$ is the function
      that never halts at any input. Therefore
      \[g(x,y) \in\emptyset^{(n+1)} \Rightarrow
      \Phi^{\emptyset^{(n)}}_{e}(x,y) =\lim_{y_1}
      \Phi^{\emptyset^{(n)}}_{e}(x,y_1).\]

      Hence
      \[h(x) =\Phi^{\emptyset^{(n)}}_{e}(x,y),\]
      where
      \[y =\mu y\; g(x,y)\in\emptyset^{(n+1)},\]
      and so
      \[h(x) \leq_T \emptyset^{(n+1)}.\]
    \end{proof}

  \item \it Soare 3.8.10: Let $f$ be a one-to-one computable function with
    range $A$. Define the deficiency set for this enumeration $f$ to be:
    \[D= \{s: (\exists t>s)\; \left[f(t)<f(s)\right]\}.\]
    Prove that $A\leq_TD$ and $D\leq_{tt}A$.

    \begin{proof}
      We first show that because $f$ is one-to-one,
      $|\overline{D}|=\infty$. Assume by contradiction that $\overline{D}$
      contains a largest element $x_0\in\omega$. Then $f(x_0+1)>f(x_0)$
      since $x_0\in\overline{D}$. But $x_0+1\in D$, so there must be some
      $x>x_0+1$ such that $f(x_0)<f(x)<f(x_0+1)$. Now since $f$ is
      one-to-one, and there are only a finite number of elements between
      $f(x_0)$ and $f(x_0+1)$, therefore there can only be a finite number
      of distinct elements $a_n>\ldots>a_1>x_0+1$ such that
      $f(x_0)<f(a_i)<f(x_0+1)$. Then we have $a_n\in D$, yet there is no
      $x>a_n$ such that $f(x)<f(a_n)$, because $f(x)>f(x_0+1)>f(a_n)$ for
      all $x>a_n$. This is a contradiction, hence $|\overline{D}|=\infty$.
      \\

      Assume we know $D$ and want to decide if a given $y\in\omega$ lies in
      $A$. Since we know $D$, we can find the smallest $x_0\in\omega$ such
      that $x_0\not\in D$ and $f(x_0)\geq y$. Such $x_0$ must exist because
      $|\overline{D}|=\infty$ and $f$ is one-to-one. Then we know that
      $y\in A$ if and only if $f(x)=y$ for some $x\leq x_0$. Thus $A\leq_T
      D$.\\

      Assume we know $A$ and want to decide if a given $x\in\omega$ lies in
      $D$. Since $f$ is one-to-one, we know that $x\in D$ if and only if
      there is an element smaller than $f(x)$ that lies $A$ but which has
      not yet been enumerated, i.e.
      \[x\in D \Leftrightarrow A\restriction f(x) \not\subseteq
      \{f(0),\ldots,f(x-1)\},\]
      where $A\restriction k$ denotes $A\cap\{0,\ldots,k-1\}$. \\

      Consider the Turing functional $\Phi_e(x)$ such that given arbitrary
      $X\in2^\omega$ and $x\in\omega$, first computes the set
      $\{a<f(x): X(a)=1\}$, and then the set $\{f(0),\ldots,f(x-1)\}$, and
      returns 1 if the first set is not contained in the second, and
      returns 0 otherwise. Then $\Phi_e$ is total because $f$ is total
      recursive. Also, $D=\Phi_e^A$. Thus from the Truth-table theorem
      (Nerode, Theorem 3.8.5), $D\leq_{tt}A$.
    \end{proof}

  \item \it Soare 3.8.16: Show that for each $n$ there is an $(n+1)$-c.e.
    set which is not $n$-c.e. Hence, the d.c.e. sets are not closed under
    union. \\

    Hint. Fix $n$. First specify a method to effectively list all $n$-c.e.
    sets $\{Z_e^n\}_{e\in\omega}$. Next, treat this as a game in which
    Player 1 can insert or delete an element $x_e$ from his $(n+1)$-c.e.
    set $D$ in order to arrange that $D(x_e)\neq Z_e^n(x_e)$. The
    $D$-player has one more move for $x_e$ than the $Z_e^n$-player does.
    Hence, this closely resembles and generalizes Theorem 1.6.5 that $K$ is
    not computable because as a 1-c.e. set $K$ has one more move than any
    computable function $\varphi_e$.

    \begin{proof}
      We follow the hint and first find an effective list of all $n$-c.e.
      sets. We have shown that the $n$-c.e. sets are characterized by
      differences and unions of $n$ c.e. sets (shown in \ref{q:ce}).
      Specifically, the $(2k+1)$-c.e. sets are exactly the union of $k$
      d.c.e. sets with a c.e. set, and the $(2k)$-c.e. sets are exactly the
      union of $k$ d.c.e. sets. Therefore, an effective enumeration of
      $\omega^n$ would effectively list the $n$-tuples of the indices of
      the c.e. sets involved in constructing each the $n$-c.e. sets. In
      particular, we can associate with an arbitrary $(2k+1)$-tuple
      $e:=\langle e_0,\ldots,e_{2k+1}\rangle\in \omega^{2k+1}$ the
      $(2k+1)$-c.e. set
      \[Z_{e}^{2k+1} :=W_{2k+1} \cup \bigcup_{i=0}^k (W_{e_i}-W_{e_{i+1}}).\]
      Similarly, we can associate with an arbitrary
      $(2k)$-tuple $e:=\langle e_0,\ldots,e_{2k}\rangle\in
      \omega^{2k}$ the $(2k)$-c.e. set
      \[Z_{e}^{2k} :=\bigcup_{i=0}^k (W_{e_i}-W_{e_{i+1}}).\]

      Furthermore, given $e:=\langle e_0,\ldots,e_{n}\rangle\in
      \omega^{n}$, we can effectively find the computable sequence
      $Z_{e,s}^n(x)$ that witnesses the $n$-c.e.-ness of $Z_e^n$, as
      follows: For each $i\in\{0,\ldots,\lfloor n/2\rfloor\}$, construct
      the computable sequence $d_{i,s}(x)$ that witnesses the d.c.e.-ness
      of $W_{e_i}-W_{e_{i+1}}$:
      \begin{align*}
        d_{i,s}(x) :=
        \begin{cases}
          0 &\text{if}\; s=0\\
          0 &\text{if}\; s>0\; \text{and}\; \varphi_{e_{i+1}}(x)\downarrow\;
            \text{after}\; s\; \text{steps},\\
          1 &\text{if}\; s>0\; \text{and}\; \varphi_{e_{i+1}}(x)\uparrow\;
            \text{after}\; s\; \text{steps and}\; \varphi_{e_i}(x)\downarrow\;
            \text{after}\; s\; \text{steps}.\\
        \end{cases}
      \end{align*}

      Then if $n$ is odd, construct the computable sequence $c_s(x)$ that
      witnesses the 1-c.e.-ness of $W_{e_n}$, i.e. let
      \begin{align*}
        c_{s}(x) :=
        \begin{cases}
          0 &\text{if}\; s=0,\\
          1 &\text{if}\; s>0\; \text{and}\; \varphi_{e_n}(x)\downarrow\;
            \text{after}\; s\; \text{steps},\\
          0 &\text{otherwise}.\\
        \end{cases}
      \end{align*}

      Finally, we can construct the computable sequence
      $Z_{e,s}^n(x)$ that witnesses the $n$-c.e.-ness of $Z_e^n$, as
      \begin{align*}
        Z_{e,s}^n(x) :=
        \begin{cases}
          c_s(x) \vee \bigvee_{i=0}^{\lfloor n/2\rfloor} d_{i,s}(x)
            &\text{if}\; n\; \text{is odd},\\
          \bigvee_{i=0}^{\lfloor n/2\rfloor} d_{i,s}(x)
            &\text{if}\; n\; \text{is even}.\\
        \end{cases}
      \end{align*}

      Hence, we have an effective enumeration of all the $n$-c.e. sets
      $\{Z_e^n\}_{e\in\omega}$, and also an effective enumeration of their
      corresponding computable sequences $\{Z_{e,s}^n\}_{e\in\omega}$ that
      witness their $n$-c.e.-ness. Following the hint, we apply a
      diagonalization argument to construct a $(n+1)$-c.e. set which is
      not $n$-c.e. Using the enumeration $\{Z_{e,s}^n\}_{e\in\omega}$, we
      construct a recursive sequence $D_s(x)$ that witness the
      $(n+1)$-c.e.-ness of a set $D$, such that $D(e)\neq Z^n_e(e)$ for each
      $e\in\omega$:
      \begin{align*}
        D_s(e) :=
        \begin{cases}
          0 &\text{if}\; s=0,\\
          0 &\text{if}\; s>0\; \text{and}\; Z_{e,s}(e)=1,\\
          1 &\text{if}\; s>0\; \text{and}\; Z_{e,s}(e)=0.\\
        \end{cases}
      \end{align*}

      Then $D=\lim_sD_s$ is $(n+1)$-c.e. because each $Z_e^n$ is $n$-c.e.
      Also, from construction, $D(e)\neq Z_e^n(e)$ for any $e\in\omega$. \\

      The assertion that d.c.e. sets are not closed under union follows
      directly from the characterization of $n$-c.e. sets, which we have
      shown in (\ref{q:ce}). In particular, the 4-c.e. sets are exactly the
      unions of d.c.e. sets, but the first part of this question has shown
      that the 4-c.e. sets are a strict superset of the d.c.e. sets.
    \end{proof}

  \item \it Soare 4.1.12: Prove that
    \[S:= \{\langle x,y\rangle: W_x\; \text{and}\; W_y\; \text{are
    computably separable}\} \in \Sigma_3.\]

    \begin{proof}
      Recall that a c.e. set $W_e$ is computable if and only if its
      complement is also c.e., i.e.
      \[\exists f\; \left[W_{e}\cup W_{f}=\omega \wedge
      W_{e}\cap W_{f}=\emptyset\right].\]

      Hence $\langle x,y\rangle\in S$ if and only if it satisfies the
      formula
      \[\varphi(x,y):= \exists e,f\; \left[W_{e}\cup W_{f}=\omega \wedge
      W_{e}\cap W_{f}=\emptyset \wedge W_x\subseteq W_e \wedge
      W_y\subseteq W_f\right].\]

      Thus it suffices to show that the following three sub-formulas
      are $\Sigma_3$: $W_e\cup W_f=\omega$, $W_e\cap W_f$, $W_x\subseteq
      W_e$. Recall that the formula $x\in W_e$ is $\Sigma_1$ because
      \[x\in W_e \Leftrightarrow \exists s,y\; \Phi_{e,s}(x)=y,\]
      and $\Phi_{e,s}(x)=y$ is a $\Delta_0$ formula. Then $x\not\in
      W_e$ is a $\Pi_1$ formula. Hence
      \begin{align*}
        W_e\cup W_f=\omega\; &\Leftrightarrow \forall x\; (x\in W_e \vee
          x\in W_f) &\in\Pi_2,\\
        W_e\cap W_f=\emptyset\; &\Leftrightarrow \forall x\; (x\not\in W_e
          \vee x\not\in W_f) &\in\Pi_1,\\
        W_x\subseteq W_e=\emptyset\; &\Leftrightarrow \forall x\; (x\in W_e
          \vee x\not\in W_x) &\in\Pi_2.\\
      \end{align*}
      Therefore $\varphi(x,y)$ is $\Sigma_3$.
    \end{proof}

  \item \it Soare 4.1.13: Define $A\subseteq^*B$ if $A-B$ is finite, i.e.,
    if $A\subseteq B$ except for at most finitely many elements. Define
    $A=^*B$ if $A\subseteq^*B$ and $B\subseteq^*A$. Prove that the
    following two sets are $\Sigma_3$:
    \[\{\langle x,y\rangle: W_x\subseteq^* W_y\};\]
    \[\{\langle x,y\rangle: W_x=^* W_y\}.\]

    \begin{proof}
      Since $A=^*B$ is the same as $A\subseteq^*B$ and $B\subseteq^*A$, and
      the conjunction of two $\Sigma_3$ formulas is still $\Sigma_3$, thus
      it suffices to show that the first set is $\Sigma_3$. Observe that
      $W_x\subseteq^*W_y$ if and only if $W_x$ contains a largest element
      $n$ that is not in $W_y$. Thus
      \begin{align*}
        W_x\subseteq^* W_y &\Leftrightarrow \exists n\forall t\;
          \left[(t>n \wedge t\in W_x) \rightarrow t\in W_y\right]\\
        &\Leftrightarrow \exists n\forall t\;
          \left[t\in W_y \vee (t\leq n \vee t\not\in W_x)\right]\\
        &\Leftrightarrow \exists n\forall t\;
          \left[t\leq n \vee t\in W_y \vee t\not\in W_x\right].\\
      \end{align*}

      Now from argument in the previous question, $t\in W_e$ is $\Sigma_1$.
      Therefore the above formula is $\Sigma_3$.
    \end{proof}

  \item \it Soare 8.7.7: Prove that Definition 8.7.6 for the
    Cantor-Bendixson derivative of a closed set does not depend on the
    choice of the tree such that $[T]=\mathcal{C}$. Take any two trees
    $T_1$ and $T_2$ such that $[T_1]=[T_2]=\mathcal{C}$ and prove that the
    tree derivative of Definition 8.7.5 gives the same rank in both trees
    for any $f\in\mathcal{C}$. Hint: Keep applying the fact that
    $T_1^{\text{ext}}=T_2^{\text{ext}}$.

    \begin{proof}
      Let $T_1$ and $T_2$ be trees such that $[T_1]=[T_2]=\mathcal{C}$
      where $\mathcal{C}$ is a closed set, and let $f\in\mathcal{C}$.
      We first show that $D^1(T_1)=D^1(T_2)$. By symmetrical argument, it
      suffices to show $D^1(T_1)\subseteq D^1(T_2)$. Let $\sigma\in
      D^1(T_1)$. Then $\sigma\in T_1$ and $\sigma\not\in\Gamma(T_1)$, which
      implies that there are infinite paths through $\sigma$ that are
      contained in $[T_1]$. Each of these paths will also be in $[T_2]$
      since $[T_2]=[T_1]$, which implies that $\sigma\in T_2$ and also
      $\sigma\not\in\Gamma(T_2)$. Therefore $\sigma$ is also contained in
      $D^1(T_2)$, and so $D^1(T_1)=D^1(T_2)$.  \\

      Therefore from transfinite induction, $D^\alpha(T_1)=D^\alpha(T_2)$
      for all ordinals $\alpha\geq1$. So if the ranks of $f$ with respect
      to $T_1$ and in $T_2$ are both larger than 0, then the ranks
      must be equal, because the rank of $f$ with respect to $T_i$ is the
      largest ordinal $\alpha$ at which all initial segments $f\restriction
      n$ of $f$ are contained in $D^\alpha(T_i)$. \\

      So assume that the rank of $f$ with respect to $T_1$ is 0. Then there
      must be an initial segment $f\restriction n$ of $f$ such that
      $f\restriction n \in\Gamma(T_1)$. Therefore there can only be a
      finite number of paths through $f\restriction n$ that is contained in
      $[T_1]$. Since $[T_2]=[T_1]$, these paths must also be contained in
      $[T_2]$, and must be the only paths that pass through $f\restriction
      n$ in $T_2$. Thus $f\restriction n \in\Gamma(T_2)$, which means that
      the rank of $f$ with respect to $T_2$ is also 0.
    \end{proof}

  \item \it Soare 8.7.8:
    \begin{enumerate}[label={(\roman*)}]
      \item Prove that $D^\alpha(T)$ is a tree and hence $[D^\alpha(T)]$ is
        closed subset of $\mathcal{A}=[T]$.

        \begin{proof}
          We prove by transfinite induction on $\alpha$ that $D^\alpha(T)$
          is a tree. The base case $D^0(T)=T$ is a tree by definition.
          At the inductive step for successor ordinals $\alpha+1$, let
          $\sigma\in D^{\alpha+1}(T)$ and $\tau\preceq\sigma$. Then
          $\sigma$ is contained in $D^\alpha(T)$ which is a tree by
          inductive hypothesis, and thus $\tau$ is also contained in
          $D^\alpha(T)$. Also, $\sigma$ is not contained in
          $\Gamma(D^\alpha(T))$, which means there are inifinite paths that
          pass through $\sigma$. This implies that there are infinite paths
          that pass through $\tau$, so $\tau$ is also not contained in
          $\Gamma(D^\alpha(T))$. Hence $\tau$ is contained in
          $D^{\alpha+1}(T)$, which implies that $D^{\alpha+1}(T)$ is closed
          under initial initial segments and is therefore a tree. For the
          inductive step where $\alpha$ is a limit ordinal, $D^\alpha(T)$
          is the intersection of trees by the induction hypothesis.
          Therefore if $\sigma\in D^\alpha(T)$ and $\tau\preceq\sigma$,
          then $\tau$ must be contained in those trees whose intersection
          is $D^\alpha(T)$, so $\tau$ is also contaiend in $D^\alpha(T)$.
          Thus $D^\alpha(T)$ is closed under initial segments, making it a 
          tree. \\

          Clearly $D^\alpha(T)\subseteq T$ by definition. Then since
          $D^\alpha(T)$ is a tree, $[T]$ is well-defined. Hence
          $[D^\alpha(T)]\subseteq[T]$.
        \end{proof}

      \item Show that there is an ordinal $\beta$ such that
        $D^\beta(T)=D^\alpha(T)$ for all $\alpha>\beta$. Define
        $D^\infty(T)=D^\beta(T)$. Prove that there is an $\alpha<\omega_1$
        such that $D^\alpha(T)=D^\infty(T)$. We call $D^\infty(T)$ and
        $[D^\infty(T)]$ the perfect kernel.

        \begin{proof}
          Assume by contradiction that
          $D^{\alpha+1}(T)\subsetneq D^\alpha(T)$ for all
          $\alpha<\omega_1$. Then we will be able to construct an injective
          function from $\omega_1$ to $2^{<\omega}$, which would contradict
          the uncountability of $\omega_1$: Given $\alpha<\omega_1$, we map
          $\alpha$ to an initial segment in $\Gamma(D^\alpha(T))$. Such a
          segment exists since $D^{\alpha+1}(T)\subsetneq D^\alpha(T)$.
          The constructed map will be injective for the same reason.
          Finally, the constructed map exists from the Axiom of Choice.
          Therefore, there must be an $\alpha<\omega_1$ such that
          $D^\alpha(T)=D^\infty(T)$.
        \end{proof}

      \item Prove that either $D^\infty(T)=\emptyset$ or else $D^\infty(T)$
        is a perfect tree, namely every $\sigma\in D^\infty(T)$ splits as
        defined above. In this case $D^\infty(T)$ has $2^{\aleph_0}$ many
        infinite paths.

        \begin{proof}
          Let $\beta<\omega_1$ be the smallest ordinal such that
          $D^\infty(T)=D^\beta(T)$. If $D^\infty(T)$ is not a perfect tree,
          it would contain an initial segment $\sigma$ that does not split.
          Then there cannot be more than one path in $D^\infty(T)$ that
          extends $\sigma$, so $\sigma$ will be contained in
          $\Gamma(D^\beta(T))$, and so $D^{\beta+1}(T)\neq
          D^\infty(T)$, a contradiction. \\

          The number of paths in $D^\infty(T)$ cannot be greater than
          $2^{\aleph_0}$ since $[D^\infty(T)]$ is a subset of
          $2^{\aleph_0}$. For the reverse inclusion, we construct an
          injective map between from $2^{\aleph_0}$ to $D^\infty(T)$. Given
          a path $f\in2^{\aleph_0}$, map $f$ to a path
          $p=\cup_{n\in\omega}\sigma_n$ in $[D^\infty(T)]$ recursively, as
          follows: For each $n\in\omega$, $\sigma_n$ will always split
          since $D^\infty(T)$ is a perfect tree. Let $\sigma_{n+1}$ be a
          left extension of $\sigma_{n}$ if $f(n)=0$, otherwise let
          $\sigma_{n+1}$ be a right extension. Furthermore, we extend
          $\sigma_n$ up to the point before $\sigma_{n+1}$ splits.
        \end{proof}

      \item Let $\beta$ be as in (ii). Prove that
        $[D^\alpha(T)]-[D^{\alpha+1}(T)]$ is countable for every
        $\alpha<\beta$. Therefore, $\cup_{\alpha<\beta}[D^\alpha(T)]$ is
        countable, namely $[T]-[D^\infty(T)]$ is countable.

        \begin{proof}
          By definition, every path in $[D^\alpha(T)]-[D^{\alpha+1}(T)]$ is
          an extension of some initial segment $\sigma$ in
          $\Gamma(D^\alpha(T))$ that has only a finite number of paths
          extending from it in $D^\alpha(T)$. Since the maximum number of
          initial segments is countable, and each initial segment has only
          a finite number of paths, hence the maximum number of paths that
          can be removed from $[D^\alpha(T)]$ to form $[D^{\alpha+1}(T)]$
          can only be $\omega^2$ which is still countable. \\

          Then since $\beta$ is countable from (ii), and a countable union of
          countable objects is still countable,
          $\cup_{\alpha<\beta}[D^\alpha(T)]$ is also countable.
        \end{proof}
    \end{enumerate}

  \item \label{q:ce} \it Soare 3.8.12 (Not assigned but was used to solve
    Soare 3.8.16):
    \begin{enumerate}[label={(\roman*)}]
      \item \it Show that $A$ is $(2n+1)$-c.e. iff $A$ is the union of $n$
        d.c.e. sets and a c.e. set.
      \item \it Show that $A$ is $(2n+2)$-c.e. iff $A$ is the union of $n+1$
        d.c.e. sets.
    \end{enumerate}

    \begin{proof}
      We prove by induction on $n$. We begin with the first base case
      $n=1$. Let $A=\lim_s A_s$ be a 1-c.e. set, with $A_s$ being a
      computable sequence witnessing the 1-c.e.-ness of $A$. Then $A=W_e$,
      where $\varphi_e$ is the partial computable function
      \begin{align*}
        \varphi_e(x) :=
        \begin{cases}
          1 &\text{if}\; 1\in\{A_s(x):s\in\omega\}\\
          \uparrow &\text{otherwise}.\\
        \end{cases}
      \end{align*}
      Conversely, if $A=W_e$ for some $e\in\omega$, then the following
      computable sequence $A_s$ will witness the 1-c.e.-ness of $A$:
      \begin{align*}
        A_s(x) :=
        \begin{cases}
          0 &\text{if}\; s=0\\
          1 &\text{if}\; s>0\; \text{and}\; \varphi_e(x)\downarrow\;
          \text{after}\; s\; \text{steps}.\\
        \end{cases}
      \end{align*}

      Next, we prove the second base case $n=2$. Let $A=\lim_s A_s$ be a
      2-c.e. set, with $A_s$ being a computable sequence witnessing the
      2-c.e.-ness of $A$. Then $A=W_e-W_f$, where $\varphi_e$ and
      $\varphi_f$ are the partial computable functions
      \begin{align*}
        \varphi_e(x) &:=
        \begin{cases}
          1 &\text{if}\; 1\in\{A_s(x):s\in\omega\}\\
          \uparrow &\text{otherwise},\\
        \end{cases}\\
        \varphi_f(x) &:=
        \begin{cases}
          1 &\text{if}\; \exists s<t\; \left[A_s(x)=1\wedge
            A_t(x)=0\right]\\
          \uparrow &\text{otherwise}.\\
        \end{cases}\\
      \end{align*}

      Conversely, if $A=W_e-W_f$ for some $e,f\in\omega$, then the
      following computable sequence $A_s$ will witness the 2-c.e.-ness of
      $A$:
      \begin{align*}
        A_s(x) :=
        \begin{cases}
          0 &\text{if}\; s=0\\
          0 &\text{if}\; s>0\; \text{and}\; \varphi_f(x)\downarrow\;
            \text{after}\; s\; \text{steps},\\
          1 &\text{if}\; s>0\; \text{and}\; \varphi_f(x)\uparrow\;
            \text{after}\; s\; \text{steps and}\; \varphi_e(x)\downarrow\;
            \text{after}\; s\; \text{steps}.\\
        \end{cases}
      \end{align*}

      Now consider the $k+2$ inductive step. Let $A=B\cup C$ be the union
      of sets $B$ and $C$, where $C$ is d.c.e., and where $B$ is a union of
      $k/2$ d.c.e. sets if $k$ is even, and otherwise a union of $(k-1)/2$
      d.c.e. sets with a c.e. set if $k$ is odd. By induction hypothesis,
      $B$ is $k$-c.e. and $C$ is 2-c.e. Let $B_s$ and $C_s$ be recursive
      functions witnessing the $k$-c.e.-ness of $B$ and the 2-c.e.-ness of
      $C$ respectively. Define the recursive sequence $A_s(x) :=(B_s(x)\vee
      C_s(x))$. Then since $A$ is the union of $B$ and $C$, we have
      $A=\lim_s A_s$, and $A_s$ witnesses the $(k+2)$-c.e.-ness of $A$. \\

      For the converse, let $A$ be $(k+2)$-c.e. with recursive sequence
      $A_s$ as witness. We ``split'' $A_s$ into a union of a $k$-c.e.
      set $B$ and a 2-c.e. set $B$, such that $A_s(x)=(B_s(x)\vee C_s(x))$,
      $B=\lim_sB_s$, and $C=\lim_sC_s$, by defining the recursive sequences
      $B_s$ and $C_s$ recursively as follows:
      \begin{align*}
        B_s(x) &:=
        \begin{cases}
          A_s(x) &\text{if}\; s=0\; \text{or}\;
            \{A_0(x),\ldots,A_{s-1}(x)\}\; \text{changes less than}\; k\;
            \text{times},\\
          B_{s-1}(x) &\text{otherwise},\\
        \end{cases}\\
        C_s(x) &:=
        \begin{cases}
          A_s(x) &\text{if}\; s=0\; \text{or}\;
            \{A_0(x),\ldots,A_{s-1}(x)\}\; \text{changes more than}\; k\;
            \text{times},\\
          0 &\text{otherwise}.\\
        \end{cases}\\
      \end{align*}

      Note that $B$ will be $k$-c.e. by construction, and $C$ will be
      2-c.e. because $A$ is $k+2$-c.e. Now by induction hypothesis, $C$
      is d.c.e., and $B$ is a union of $k/2$ d.c.e. sets if $k$ is even and
      a union of $(k-1)/2$ d.c.e. sets with a c.e. set if $k$ is odd.
      Therefore $A$ is a union of $k/2+1$ d.c.e. sets if $k$ is even and
      a union of $(k+1)/2$ d.c.e. sets with a c.e. set if $k$ is odd.
    \end{proof}
\end{enumerate}
\end{document}
