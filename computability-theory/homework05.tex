\documentclass{article}
\usepackage[left=3cm,right=3cm,top=3cm,bottom=3cm]{geometry}
\usepackage{amsmath,amssymb,amsthm,tikz,mathtools}
\usepackage{stmaryrd} % For double square bracket [[]]
\usepackage{color}
\usepackage[inline]{enumitem}
\usetikzlibrary{patterns}
\setlength{\parindent}{0mm}
\newcommand{\TODO}[1]{\textcolor{red}{TODO: #1}}

\begin{document}
\title{Basic Logic II: Homework 5}
\author{Li Ling Ko\\ lko@nd.edu}
\date{\today}
\maketitle

\begin{enumerate}[label={\bf Q\arabic*:}]
  \item \it Work in Cantor Space. Construct an open class which is not
    closed. Then a $\pi_2^0$ class which is not open or closed. You might
    want to use compactness.

    \begin{proof}
      Consider the class
      \[\mathcal{C} :=2^\omega-\{0\},\]
      which is the class of all functions from $\omega$ to $\{0,1\}$ minus
      the zero function. This class is open because it is the union of the
      basic clopen classes generated by initial segments with at least one
      value that is 1, i.e.
      \[\mathcal{C} =\llbracket\{\sigma\in2^{<\omega}: \exists n\;
      \sigma(n)=1\}\rrbracket.\]

      However, this class is not closed because its complement is the zero
      function, which cannot be open because any initial segment of the
      zero function will generate a clopen class that includes a nonzero
      function. 
    \end{proof}
\end{enumerate}
\end{document}
