\documentclass{article}
\usepackage[left=3cm,right=3cm,top=3cm,bottom=3cm]{geometry}
\usepackage{amsmath,amssymb,amsthm,tikz,mathtools}
\usepackage{stmaryrd} % For double square bracket [[]]
\usepackage{color}
\usepackage[inline]{enumitem}
\usetikzlibrary{patterns}
\setlength{\parindent}{0mm}
\newcommand{\TODO}[1]{\textcolor{red}{TODO: #1}}

\begin{document}
\title{Basic Logic II: Homework 6}
\author{Li Ling Ko\\ lko@nd.edu}
\date{\today}
\maketitle

\it \textbf{Soare 3.4.7:}
  \begin{enumerate}[label={\bf (\alph*):}]
    \item Let $\{A_y\}_{y\in\omega}$ be any countable sequence of sets.
      Define the infinite join $\oplus_y A_y$ as in (2.29). Prove that
      $\text{deg}(\oplus_y A_y)$ is the uniform least upper bound for
      $\{\text{deg}(\oplus_y A_y)\}_{y\in\omega}$ in the sense that if
      there exist a set $C$ and a computable function $f$ such that
      $A_y=\Phi_{f(y)}^C$ for all $y$, then $\oplus_y A_y\leq_T C$.

      \begin{proof}
        Let $C$ be a set and $f$ a computable function such that
        $A_y=\Phi_{f(y)}^C$ for all $y$. Then to decide if $\langle
        x,y\rangle$ lies in $\oplus_y A_y$, we just check if
        $\Phi_{f(y)}^C(x)=1$. If it is then $\langle x,y\rangle$ lies in
        $\oplus_y A_y$, otherwise $\Phi_{f(y)}^C(x)=0$ and then $\langle
        x,y\rangle$ does not lie in $\oplus_y A_y$. Therefore $\oplus_y
        A_y\leq_T C$.
      \end{proof}

    \item Prove that this operation is not well defined on degrees. Namely,
      define $\{A_y\}_{y\in\omega}$ and  $\{B_y\}_{y\in\omega}$ such that
      $A_y \equiv_T B_y$ but $A\not\equiv_T B$ for $A=\oplus_y A_y$ and
      $B=\oplus_y B_y$.

      \begin{proof}
        Consider the case where $A_y(x)=0$ for all $x,y\in\omega$, and
        where
        \begin{align*}
          B_y(x) =
          \begin{cases}
            1 &\text{if}\; x=0\; \text{and}\; y\in K,\\
            0 &\text{otherwise}.\\
          \end{cases}
        \end{align*}
        Then for each $y\in\omega$, $B_y$ only differs from $A_y$ at no
        more than a single $x$ value, therefore $A_y\equiv_T B_y$. However,
        $B\not\leq_T A$, because $A=\emptyset$ is clearly computable,
        while $K\leq_T B$, since
        \[e\in K \Leftrightarrow \langle 0,e\rangle \in B.\]
      \end{proof}
  \end{enumerate}

\it \textbf{Soare 6.1.4:} Modify the proof of Theorem 6.1.1 to build an
  independent sequence $\{A_j\}_{j\in\omega}$ of sets each computable in
  $\emptyset'$.

  \begin{proof}
    We follow Soare's hint to construct the sequence. Fix any enumeration
    of $\omega^2$ such that for any given $j\in\omega$, $\langle
    s,j\rangle$ always gets enumerated before $\langle s+1,j\rangle$. We
    construct the sequence in stages $\langle s,j\rangle \in\omega^2$, so
    that $A_j=\cup_{s\in\omega} \sigma_{j}^s$, where $\sigma_j^s\prec
    \sigma_j^{s+1}$, and $\sigma_j^0=\emptyset$. At stage $\langle
    s,j\rangle$, we extend $\sigma_j^s$ to become $\sigma_j^{s+1}$ to satisfy
    the restriction
    \begin{equation}
      R_{s,j}: A_j \neq \Phi_s^{\oplus_{i\neq j} A_i}.
      \label{eq:join}
    \end{equation}

    More specifically, at stage $\langle s,j\rangle$, there are only a
    finite number indices $i_1,\ldots,i_k$ that are not equal to $j$ and
    whose initial segment $\sigma_{i_l}^s$ is non-empty. We ask the
    $\emptyset'$-question
    \[(\exists t) (\exists \rho_1 \succ \sigma_{i_1}^s) \ldots (\exists
    \rho_k \succ \sigma_{i_k}^s) \Phi_{s,t}^{\oplus_{l=1}^k
    \rho_{i_l}^s} \downarrow?\]
    If such $\rho_1,\ldots,\rho_k$ exist, then set $\sigma_j^{s+1}$, and to
    be $\sigma_j^s$ concatenated by the first value in $\{0,1\}$ that is
    different from $\Phi_{s,t}^{\oplus_{l=1}^k \rho_{i_l}^s}$, and set
    $\sigma_{i_l}^{s+1}=\rho_l$ for each $l\in\{1,\ldots,k\}$.
    Otherwise set $\sigma_j^{s+1}=\sigma_j^{s\frown}0$ (to ensure that
    $A_j$ is total eventually), and $\sigma_{i_l}^{s+1} =\sigma_{i_l}^{s}$.
    \\

    Clearly this construction will satisfy restriction~(\ref{eq:join})
    for all $s,j\in\omega$. Also, restriction~(\ref{eq:join}) will ensure
    in particular that $A_j\neq \Phi_e^{A_i}$ for all $i\neq j$ and
    $e\in\omega$. 
  \end{proof}

\it \textbf{Soare 6.1.5:} A partially ordered set $\mathcal{P}=(P,\leq_P)$
  is countably universal if every countable partially ordered set is order
  isomorphic to a subordering of $\mathcal{P}$. Prove that there is a
  computable partial ordering $\leq_R$ of $\omega$ which is countably
  universal.

  \begin{proof}
    Consider the Boolean algebra
    $\mathcal{P}=(P,\leq_P)$ where
    \[P:= \{\overline{x}\in \{0,1\}^\omega: x\in \{0,1\}^{<\omega}\},\]
    where $\overline{x}$ is the infinite string in $\{0,1\}^\omega$ when we
    repeat $x$ infinitely, and where
    \[\overline{x} \leq_B\overline{y}\; \Leftrightarrow\; \{n\in\omega:
    \overline{x}(n)=1\} \subseteq \{n\in\omega: \overline{y}(n)=1\}.\]

    This Boolean algebra is computable since $\{0,1\}^{<\omega}$ can be
    effectively enumerated. Therefore $\mathcal{P}$ can be effectively
    enumerated. Furthermore, $\mathcal{P}$ is clearly computably presented,
    and does not have atoms because given any non-zero
    $\overline{a_0,\ldots,a_{n-1}}\in P$, there is a non-zero element
    $\overline{a_0,\ldots,a_{n-1},0,\ldots,0}$ ($n$ number of zeros) which
    is smaller. By a similar argument, $P$ is dense in the sense that
    between any two distinct comparable elements in in $P$ lie an element
    between them. Finally, given any non-zero and non-one element in $P$,
    there are infinite distinct elements that are incomparable with it. \\

    Let $\mathcal{R}=(\omega,\leq_R)$ be countable partial order. We
    construct an embedding $\varphi:\mathcal{R}\rightarrow\mathcal{P}$ that
    preserves order. We construct $\varphi$ in stages, where we define
    $\varphi(s)$ at stage $s$, such that the subordering
    $(\{0,\ldots,s\},\leq_R)\subset\mathcal{R}$ is isomorphic to its image
    $\varphi(\{0,\ldots,s\},\leq_P)\subset\mathcal{P}$. At stage 0, let
    $\varphi(0)$ be an arbitrary non-zero non-one element in $P$. At stage
    $s+1$, consider the subordering
    $(\{0,\ldots,s+1\},\leq_R)\subset\mathcal{R}$. \\

    %To define the image of
    %the new unmapped element $s+1$, we find the elements in
    %$\{0,\ldots,s\}$ that are comparable with $s+1$. If such elements
    %exist, let $l$ be the largest comparable element that is smaller than
    %$s+1$, and let $u$ be the smallest comparable element that is larger
    %than $s+1$, if such elements exist. If both $l$ and $u$ exist, then we
    %have $l<s+1<u$, so by density of $\mathcal{B}$, we can find an element
    %$\overline{x}\in P$ between $\varphi(l)$ and $\varphi(u)$ to map $s+1$
    %to. If only $u$ exists, we can also similarly from density of
    %$\mathcal{B}$ map $s+1$ to an element $\overline{x}$ where
    %$0<_P\overline{x}<\varphi(u)$. A similar argument holds if only $l$
    %exists. \\

    %On the other hand, if $s+1$ is incomparable with any element in
    %$\{0,\ldots,s\}$, map

    %we define $\varphi(s)$
    %by considering the element in $\{0,\ldots,s-1\}$ that are the
    %``closest'' to $s$. Specifically, let $l_s$ be the element in
    %$\{0,\ldots,s-1\}$ that is the largest element which is smaller than
    %$s$, if such an element exists, and let $u_s$ be the element in
    %$\{0,\ldots,s-1\}$ that is the smallest element which is larger than
    %$s$, if such an element exists. So if both $l_s$ and $u_s$ exist, we
    %would have $l_s<_Q s<_Q u_s$, and no element in $\{0,\ldots,s-1\}$ lies
    %in between. \\

    %There are three cases to consider. In the first case, both $l_s$ and
    %$u_s$ exist. Then we check if there are infinite elements between $l_s$
    %and $s$, and if there are infinite elements between $u_s$ and $s$. If

    %There are three cases to consider. In the first case, neither $l_s$ nor
    %$u_s$ exists. Then we know that $s$ is $\leq_Q$ incomparable with all
    %elements in $\{0,\ldots,s-1\}$. We check if $s$ has a common ancestor
    %with any of the elements in $\{0,\ldots,s-1\}$.

    %We follow the proof of Godel's Compactness theorem, where we
    %effectively construct a model for a finitely satisfiable set of
    %sentences using Henkin constants. We work in the language
    %$\mathcal{L}=(\leq_P)$. Fix an enumeration
    %$\{(k_n,\leq_{n})\}_{n\in\omega}$ of the order types of
    %finite elements. For each order type
    %$(k_n,\leq_{n})$, let $\varphi_n$ be the sentence that
    %describes the order type in the given language. Specifically, let
    %$c_1,\ldots,c_{k_n}$ be new constants in the language, and define
    %\[\varphi_n := \bigwedge_{i\leq_{n}j} (c_i\leq_P c_j)
    %\bigwedge_{i\not\leq_{n}j} (c_i\not\leq_P c_j).\]

    %Now the set of sentences $\{\varphi_n\}_{n\in\omega}$ is finitely
    %satisfiable, since we can always add new constants to 

    %We follow the first hint given by Soare and construct $\mathcal{P}$ in
    %stages, where at stage $s+1$, given a finite set $P_s$ of elements in
    %$\leq_R$, we obtain $P_{s+1}$ by adding a new point for each possible
    %order type of $P_s$. We start with $P_0=\emptyset$, and let
    %$\mathcal{P}=\cup_{s\in\omega} P_s$. Then $\mathcal{P}$ is computable
    %since every stage $s$ is computable. \\

    %More specifically, at stage $s$, there are only a finite number of
    %points in $P_s$. Consider all possible partial orderings of $P_s$;
    %there can only be a finite number of them since $|P_s|$ is finite. Each
    %partial ordering can be coded as an element in $\mathcal{P}(|P_s|^2)$,
    %the powerset of $|P_s|^2$. Fixing an ordering on this powerset, we can
    %check which order types of $|P_s|$ elements are not already in the
    %$P_s$, and add points $p_{s,0},\cdots,p_{s,k}$ to $P_s$ to construct the
    %missing order types. \\
  \end{proof}

\it \textbf{Soare 6.1.6:} Show that for a countable partially ordered set
  $\mathcal{P}=(P,\leq_P)$ there is a 1:1 order-preserving map from $P$
  into $\mathcal{D}(\leq\emptyset')$, the degrees $\leq\emptyset'$.

  \begin{proof}
  \end{proof}

\it \textbf{Q3:} Show that every partial computable function $f$ has a low
  total function $g$ extending $f$. So if $f(x)$ exists then $f(x)=g(x)$.

  \begin{proof}
    We $\emptyset'$-construct $g(x)$ in stages $s$, such that
    $g(x)=\cup_{s\in\omega} g_s(x)$ for all $x\in\omega$, where $g_s\in
    \omega^{<\omega}$, $g_s\prec g_{s+1}$, and $g_s(x)$ is an extension of
    $f(x)$ when $x<|g_s|$. \\

    To ensure that the resulting $g$ is low, we force the jump by
    making sure that for all $s$, we satisfy the jump requirement
    \[J_s: (\exists \sigma\succ g_s)\; \left\{ \left[
    \Phi_s^{\sigma}(s)\downarrow\; \text{or}\;\; (\forall
    \rho\succ\sigma)\; \Phi_s^{\rho}(s)\uparrow \right]\; \text{and}\;
    \sigma\; \text{is consistent with}\; f\right\}.\]

    At stage 0, set $g_0=\emptyset$. At stage $s+1$, we satisfy $J_s$ by
    asking the $\emptyset'$-question:
    \begin{center}
      \textit{Is there a finite extension $\sigma\in\omega^{<\omega}$ of
      $g_s$ that is consistent with $f$ such that $\Phi^\sigma_s(s)$
      converges?}
    \end{center}

    Observe that this is a $\emptyset'$-question because we can
    effectively enumerate all finite extensions
    $\sigma\in\omega^{<\omega}$ of $g_s$, and for each of these extensions,
    use $\emptyset'$ to check of it is consistent with $f$. If it is
    consistent, we can use $\emptyset'$ again to check if
    $\Phi^\sigma_s(s)$ converges. Formally, this question is a
    $\Sigma_1$-question
    \[(\exists \sigma,t)\; \left\{ g_s\prec\sigma \wedge (\forall
    x<|\sigma|)\; \left[ (\exists u\; f_u(x)\downarrow)
    \rightarrow\sigma(x)=f(x) \right]\; \wedge\;
    \Phi^\sigma_{s,t}(s)\downarrow \right\},\]
    and therefore can be answered by $\emptyset'$. If such $\sigma$ exists,
    set $g_{s+1}=\sigma$. Otherwise, let $g_{s+1}$ be the first strict
    extension of $g_s$ that is consistent with $f$ (extension needs to be
    strict to ensure that $g$ is total). Observe that in the latter case,
    every finite extension $\rho$ of $g_s$ that is consistent with $f$
    gives $\Phi_s^\rho(s)\uparrow$, so the jump requirement $J_s$ is
    satisfied. \\

    Our construction ensures that $g$ is total and extends $f$. It remains
    to verify that $J_s$ ensures that $g$ is low. We show that
    $g'\leq_T\emptyset'$. Now $g\leq_T\emptyset'$ since the construction
    asked only $\emptyset'$ questions. To decide if $e\in g'$, or
    equivalently, if $\Phi_e^g(e)$ converges, we know that at stage $e$,
    $g_{e+1}\prec g$ satisfies the jump requirement $J_e$. Thus from
    construction, \[e\in g'\; \Leftrightarrow\; \Phi_e^g(e)\downarrow\;
    \Leftrightarrow\; \Phi_e^{g_{e+1}}(e)\downarrow,\] and $\emptyset'$ can
    check if $\Phi_e^{g_{e+1}}(e)\downarrow$ holds, since $g_{e+1}\leq_T
    \emptyset'$. Therefore $g'\leq_T\emptyset'$ as required.
  \end{proof}
\end{document}
