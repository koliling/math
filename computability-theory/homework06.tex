\documentclass{article}
\usepackage[left=3cm,right=3cm,top=3cm,bottom=3cm]{geometry}
\usepackage{amsmath,amssymb,amsthm,tikz,mathtools}
\usepackage{stmaryrd} % For double square bracket [[]]
\usepackage{bm} % For bold vectors
\usepackage{color}
\usepackage[inline]{enumitem}
\usetikzlibrary{patterns}
\setlength{\parindent}{0mm}
\newcommand{\TODO}[1]{\textcolor{red}{TODO: #1}}

\begin{document}
\title{Basic Logic II: Homework 6}
\author{Li Ling Ko\\ lko@nd.edu}
\date{\today}
\maketitle

\it \textbf{Soare 3.4.7:}
  \begin{enumerate}[label={\bf (\alph*):}]
    \item Let $\{A_y\}_{y\in\omega}$ be any countable sequence of sets.
      Define the infinite join $\oplus_y A_y$ as in (2.29). Prove that
      $\text{deg}(\oplus_y A_y)$ is the uniform least upper bound for
      $\{\text{deg}(\oplus_y A_y)\}_{y\in\omega}$ in the sense that if
      there exist a set $C$ and a computable function $f$ such that
      $A_y=\Phi_{f(y)}^C$ for all $y$, then $\oplus_y A_y\leq_T C$.

      \begin{proof}
        Let $C$ be a set and $f$ a computable function such that
        $A_y=\Phi_{f(y)}^C$ for all $y$. Then to decide if $\langle
        x,y\rangle$ lies in $\oplus_y A_y$, we just check if
        $\Phi_{f(y)}^C(x)=1$. If it is then $\langle x,y\rangle$ lies in
        $\oplus_y A_y$, otherwise $\Phi_{f(y)}^C(x)=0$ and then $\langle
        x,y\rangle$ does not lie in $\oplus_y A_y$. Therefore $\oplus_y
        A_y\leq_T C$.
      \end{proof}

    \item Prove that this operation is not well defined on degrees. Namely,
      define $\{A_y\}_{y\in\omega}$ and  $\{B_y\}_{y\in\omega}$ such that
      $A_y \equiv_T B_y$ but $A\not\equiv_T B$ for $A=\oplus_y A_y$ and
      $B=\oplus_y B_y$.

      \begin{proof}
        Consider the case where $A_y(x)=0$ for all $x,y\in\omega$, and
        where
        \begin{align*}
          B_y(x) =
          \begin{cases}
            1 &\text{if}\; x=0\; \text{and}\; y\in K,\\
            0 &\text{otherwise}.\\
          \end{cases}
        \end{align*}
        Then for each $y\in\omega$, $B_y$ only differs from $A_y$ at no
        more than a single $x$ value, therefore $A_y\equiv_T B_y$. However,
        $B\not\leq_T A$, because $A=\emptyset$ is clearly computable,
        while $K\leq_T B$, since
        \[e\in K \Leftrightarrow \langle 0,e\rangle \in B.\]
      \end{proof}
  \end{enumerate}

\it \textbf{Soare 6.1.4:} Modify the proof of Theorem 6.1.1 to build an
  independent sequence $\{A_j\}_{j\in\omega}$ of sets each computable in
  $\emptyset'$. Hint: Use a finite extension $\emptyset'$-computable
  construction to build at stage $s$ strings $\{p_j^s\}_{s\in\omega}$ such
  that if $A_j=\cup_s \rho_j^s$ then we meet for each $e$ and $i$ the
  requirement
  \[R_{\langle e,i\rangle} :A_i\neq \Phi_e^{A_j:j\neq i}.\]

  \begin{proof}
    We follow Soare's hint to construct the sequence. Fix any enumeration
    of $\omega^2$ such that for any given $j\in\omega$, $\langle
    s,j\rangle$ always gets enumerated before $\langle s+1,j\rangle$. We
    construct the sequence in stages $\langle s,j\rangle \in\omega^2$, so
    that $A_j=\cup_{s\in\omega} \sigma_{j}^s$, where $\sigma_j^s\prec
    \sigma_j^{s+1}$, and $\sigma_j^0=\emptyset$. At stage $\langle
    s,j\rangle$, we extend $\sigma_j^s$ to become $\sigma_j^{s+1}$ to satisfy
    the restriction
    \begin{equation}
      R_{s,j}: A_j \neq \Phi_s^{\oplus_{i\neq j} A_i}.
      \label{eq:join}
    \end{equation}

    More specifically, at stage $\langle s,j\rangle$, there are only a
    finite number indices $i_1,\ldots,i_k$ that are not equal to $j$ and
    whose initial segment $\sigma_{i_l}^s$ is non-empty. We ask the
    $\emptyset'$-question
    \[(\exists t) (\exists \rho_1 \succ \sigma_{i_1}^s) \ldots (\exists
    \rho_k \succ \sigma_{i_k}^s) \Phi_{s,t}^{\oplus_{l=1}^k
    \rho_{i_l}^s}(|\sigma_j^s|) \downarrow?\]
    If such $\rho_1,\ldots,\rho_k$ exist, then set $\sigma_j^{s+1}$, and to
    be $\sigma_j^s$ concatenated by the first value in $\{0,1\}$ that is
    different from $\Phi_{s,t}^{\oplus_{l=1}^k \rho_{i_l}^s}$, and set
    $\sigma_{i_l}^{s+1}=\rho_l$ for each $l\in\{1,\ldots,k\}$.
    Otherwise set $\sigma_j^{s+1}=\sigma_j^{s\frown}0$ (to ensure that
    $A_j$ is total eventually), and $\sigma_{i_l}^{s+1} =\sigma_{i_l}^{s}$.
    \\

    Clearly this construction will satisfy restriction~(\ref{eq:join})
    for all $s,j\in\omega$. Also, restriction~(\ref{eq:join}) will ensure
    in particular that $A_j\neq \Phi_e^{A_i}$ for all $i\neq j$ and
    $e\in\omega$. 
  \end{proof}

\it \textbf{Soare 6.1.5:} A partially ordered set $\mathcal{P}=(P,\leq_P)$
  is countably universal if every countable partially ordered set is order
  isomorphic to a subordering of $\mathcal{P}$. Prove that there is a
  computable partial ordering $\leq_R$ of $\omega$ which is countably
  universal. Hint: This can be done either by considering a computably
  presented atomless Boolean algebra, or by a direct construction where at
  stage $s+1$, given a finite set $P_s$ of elements in $\leq_R$, one
  obtains $P_{s+1}$ by adding a new point for each possible order type over
  $P_s$. A boolean algebra
  $\mathcal{B}=(\{b_i\}_{i\in\omega};\leq,\vee,\wedge,')$ is computably
  presented if there exist a computable relation $P(i,j)$ and computable
  functions $f$, $g$, and $h$ such that $P(i,j)$ holes iff $b_i\leq b_j$,
  and such that $b_{f(i,j)}=b_i\vee b_j$, $b_{g(i,j)}=b_i\wedge b_j$, and
  $b_{h(i)}=b_i'$.

  \begin{proof}
    We follow Soare's first hint of using a computably presented atomless
    Boolean algebra. Consider the Boolean algebra $\mathcal{P}=(P,\leq_P)$
    where
    \[P:= \{\overline{x}\in \{0,1\}^\omega: x\in \{0,1\}^{<\omega}\},\]
    where $\overline{x}$ is the infinite string in $\{0,1\}^\omega$ when we
    repeat $x$ infinitely, and where
    \[\overline{x} \leq_B\overline{y}\; \Leftrightarrow\; \{n\in\omega:
    \overline{x}(n)=1\} \subseteq \{n\in\omega: \overline{y}(n)=1\}.\]

    This Boolean algebra is computable since $\{0,1\}^{<\omega}$ can be
    effectively enumerated. In particular $\mathcal{P}$ can be effectively
    enumerated. Furthermore, $\mathcal{P}$ is clearly computably presented,
    and does not contain atoms because given any non-zero
    $\overline{a_0,\ldots,a_{n-1}}\in P$, there is a non-zero element
    \[\overline{a_0,\ldots,a_{n-1},\underbrace{0,\ldots,0}_{n\;
    \text{zeros}}}\]
    which is smaller. \\

    Also observe that $P$ is dense: given two comparable
    elements, we can effectively construct an element that lies between
    them by considering elements whose periods are a common multiple of the
    periods of two given elements. Specifically, by taking multiples of
    periods, we can assume that the two given elements
    $\overline{a_0,\ldots,a_{n-1}} <_P\overline{b_0,\ldots,b_{n-1}}\in P$
    have the same period $n$. Let $i$ be the first index such that $b_i=1$
    but $a_i=0$; then the element
    \[\overline{a_0,\ldots,a_{i-1},1,a_{i+1},\ldots,a_{n-1},
    a_0,\ldots,a_{n-1}}\]
    lies strictly between the two given elements. \\

    Also by a considering multiples of periods, we can show
    that given any finite number of incomparable elements from $P$, we can
    effectively construct a new element that is incomparable from all those
    elements: Like before, we can assume that the given $k$ elements
    \[a_0=\overline{a_{0,0},\ldots,a_{0,n-1}},\; \ldots,\;
    a_{k-1}=\overline{a_{k-1,0},\ldots,a_{k-1,n-1}}\]
    have the same period $n$. For each element $a_i$, let $u_i$ be the
    first index such that $a_{i,u_i}=0$, and let $v_i$ be the
    first index such that $a_{i,v_i}=1$. Then the element with period $kn$
    that is 0 everywhere except at indices $ni+iu_i$ and $ni+v_i$ for all
    $i\in\{0,\ldots,k-1\}$ will be incomparable with all the elements
    $a_0,\ldots,a_{k-1}$. \\

    Let $\mathcal{R}=(\omega,\leq_R)$ be an arbitrary countable partial
    order. We construct an order-preserving embedding
    $\varphi:\mathcal{R}\rightarrow\mathcal{P}$. We construct this map in
    stages, where we define $\varphi(s)$ at stage $s$, such that the
    subordering $(\{0,\ldots,s\},\leq_R)\subset\mathcal{R}$ is isomorphic
    to its image $(\varphi(\{0,\ldots,s\}),\leq_P)\subset\mathcal{P}$. At
    stage 0, let $\varphi(0)$ be an arbitrary non-zero non-one element in
    $P$. At stage $s+1$, consider the subordering $\mathcal{R}_{s}
    :=(\{0,\ldots,s\},\leq_R)\subset\mathcal{R}$. Let
    $A=\{a_0,\ldots,a_{m-1}\}$ denote the set of direct ancestors of $s+1$
    from $\{0,\ldots,s\}$, and similarly let $D=\{d_0,\ldots,a_{n-1}\}$ be
    the set of direct descendants of $s+1$ from the same set. \\

    If $A$ and $D$ are non-empty, then map $s+1$ to any element that lies
    strictly between $\varphi(a_0)\wedge\ldots\wedge\varphi(a_{m-1})$ and
    $\varphi(d_0)\vee\ldots\vee\varphi(d_{n-1})$. If only $A$ is empty, map
    $s+1$ to any element that lies strictly between 0 and
    $\varphi(d_0)\vee\ldots\vee\varphi(d_{n-1})$. If only $D$ is empty, map
    $s+1$ to any element that lies strictly between
    $\varphi(d_0)\vee\ldots\vee\varphi(d_{n-1})$ and 1. Note that for these
    three cases the image of $s+1$ exists from the density of the
    $\mathcal{P}$. Finally, if both $A$ and $D$ are empty, then $s+1$ must
    be incomparable with all elements in $\{0,\ldots,s\}$, so map $s+1$ to
    an element that is incomparable with $\varphi(0),\ldots,\varphi(s)$.
    Again the image of $s+1$ exists from the above argument, where we
    showed that we can always find an element that is incomparable with any
    finite set of elements. \\

    By construction, $(\{0,\ldots,s+1\},\leq_R)\subset\mathcal{R}$ will be
    isomorphic to its image
    $(\varphi(\{0,\ldots,s+1\}),\leq_P)\subset\mathcal{P}$. Thus
    $\mathcal{R}$ will be isomorphic to
    $(\varphi(\omega),\leq_P)\subseteq\mathcal{P}$.
  \end{proof}

\it \textbf{Soare 6.1.6:} Show that for a countable partially ordered set
  $\mathcal{P}=(P,\leq_P)$ there is a 1:1 order-preserving map from $P$
  into $\bm{D}(\leq\emptyset')$, the degrees $\leq\emptyset'$. Hint:
  By Exercise 6.1.5 we may assume $P=\omega$ and $\leq_P$ is a computable
  relation. Let $\{A\}_{i\in\omega}$ be as in Exercise 6.1.4. Define
  $f:\omega\rightarrow \bm{D}(\leq\emptyset')$ by
  $f(i)=\bm{a}_i=\text{deg}(\oplus A_j:j\leq_P i)$. Show that if $i\leq_P
  j$ then $\bm{a}_i\leq\bm{a}_j$ (by definition and the fact that $\leq_P$
  is computable), and if $i\not\leq_P j$ then $\bm{a}_i\not\leq\bm{a}_j$
  (by the computable independence of $\{A_i\}_{i\in\omega}$).

  \begin{proof}
    We follow Soare's hint. By the previous exercise, any countable
    partially ordered set $\mathcal{P}=(P,\leq_P)$ can be embedded in some
    computable partial ordering. Thus it suffices to show that we can embed
    a computable partial ordering into $\bm{D}(\leq\emptyset')$.
    Equivalently, we assume that $P=\omega$ and $\leq_P$ is computable. \\

    Let $\{A\}_{i\in\omega}$ be as in Exercise 6.1.4. Define
    $f:\omega\rightarrow \bm{D}(\leq\emptyset')$ by
    $f(i)=\bm{a}_i=\text{deg}(\oplus A_j:j\leq_P i)$. We first show that
    each $f(i)\leq_T\emptyset'$: To decide if $(x,j)$ lies in $\bm{a}_i$,
    we first check if $j\leq_P i$; this is decidable since $\leq_P$ is
    computable. If $j\not\leq_P i$, then $(x,j)\not\in\bm{a}_i$. On the
    other hand, if $j\leq_P i$, then we ask $\emptyset'$ if $x\in A_j$;
    $\emptyset'$ can answer this question since $A_j\leq_T\emptyset'$.
    Therefore $f(i)\leq_T\emptyset'$. \\

    Next, we show that if $i\leq_P j$ then $\bm{a}_i\leq_T\bm{a}_j$: To
    decide if $(x,k)$ lies in $\bm{a}_i$, we first check if $k\leq_P i$;
    this is decidable since $\leq_P$ is computable. If $k\not\leq_P i$,
    then $(x,k)\not\in\bm{a}_i$. On the other hand, if $k\not\leq_P i$,
    then $(x,k)$ lies in $\bm{a}_i$ if and only if it also lies in
    $\bm{a}_j$ by transitivity of $\leq_P$ and the fact that $i\leq_P j$.
    Therefore $\bm{a}_j$ can decide if $(x,k)$ lies in $\bm{a}_i$. \\

    Finally, we show that if $i\not\leq_P j$ then
    $\bm{a}_i\not\leq_T\bm{a}_j$: Assume $i\not\leq_P j$, but
    $\bm{a}_i\leq_T\bm{a}_j$ by contradiction. Since $i\not\leq_P j$, there
    must be some $k\in\omega$ such that $k\leq_P i$ but $k\not\leq_P j$.
    Then since $k\leq_P i$, $A_k\leq_T\bm{a}_i$, which implies that
    $A_k\leq_T\bm{a}_j$. Then since $k\not\leq_P j$, we have $\bm{a}_j\leq_T
    \oplus\{A_s:s\neq k\}$. Hence $A_k\leq_T \oplus\{A_s:s\neq k\}$,
    a contradiction. \\

    So $i\leq_P j$ if and only if $\bm{a}_i\leq_T\bm{a}_j$ as required.
  \end{proof}

\it \textbf{Soare 6.1.7:} Show that there are $2^{\aleph_0}$ mutually
  incomparable degrees.

  \begin{proof}
    We follow Soare's hint and construct a tree $T\subseteq2^{<\omega}$ such
    $|[T]|=2^{\aleph_0}$ and for every $f,g\in[T]$ with $f\neq g$, we have
    $f|_Tg$. Let $T=\cup_eT_e$ where $T_{e+1}\supset T_e$ and $T_{e+1}$ is
    defined by induction as follows. Let $T_0=\{\emptyset\}$, the tree with
    the empty node as its only member. Given $T_e$ define $L_e$ to be the
    leaves of tree $T_e$, namely
    \[L_e =\{\sigma: \sigma\in T_e\; \text{and}\; (\forall \tau \succ
    \sigma)\; [\tau\not\in T_e]\}.\]

    Next define the successors to leaves
    \[S_e =\{\sigma^\frown0: \sigma\in L_e\} \cup \{\sigma^\frown1:
    \sigma\in L_e\}.\]

    Suppose $S_e=\{\rho_i: i\leq2^{e+1}\}$. Fix $i,j\leq2^{e+1}$, $i\neq
    j$. We find extensions $\sigma_i\succ\rho_i$, and $\sigma_j\succ\rho_j$
    to satisfy the requirement
    \[R_{\langle e,i,j\rangle}:\; (\forall f\succ \sigma_i) (\forall g\succ
    \sigma_j)\; [\Phi_e^f\neq g\; \text{and}\; \Phi_e^g\neq f],\]
    as follows: We ask the $\emptyset'$-question
    \begin{center}
      \textit{Does there exist an extension $\sigma$ of $\rho_i$ such that
      $\Phi_e^{\sigma}(|\rho_j|)$ converges?}
    \end{center}

    Formally, this is a $\Sigma_1$-question 
    \[(\exists \sigma,t,y)\;
    [\Phi_{e,t}^\sigma (|\rho_j|)\downarrow=y\; \wedge\;
    \sigma\succ\rho_i].\]

    If such $\sigma,t,y$ exist, replace $\rho_i$ with $\sigma$ and $\rho_j$
    with $\rho_j$ concatenated with the first element in $\{0,1\}$ that is
    different from $y$. Then we know that for any extension $f\succ\rho_i$
    and $g\succ\rho_j$, we will have $\Phi_e^f(|\rho_j|-1)\downarrow\neq
    g(|\rho_j|-1)$. On the other hand, if no such $\sigma,t,y$ exist then
    $\rho_i$ and $\rho_j$ remain as is. Then we know that for any extension
    $f\succ\rho_i$ and $g\succ\rho_j$, we will have
    $\Phi_e^f(|\rho_j|-1)\uparrow$, and therefore $\Phi_e^f(|\rho_j|-1)\neq
    g(|\rho_j|-1)$. Next, we ask the question again but with the roles of
    $i$ and $j$ swapped, replacing $\rho_i$ and $\rho_j$ with respective
    appropriate extensions if necessary. Observe that the extensions will
    ensure that the requirement $R_{\langle e,i,j\rangle}$ is satisfied.
    We repeat the procedure of finding extensions $\sigma_i\succ\rho_i$,
    and $\sigma_j\succ\rho_j$ for all $i,j\leq2^{e+1}$ with $i\neq j$. \\

    We show that this construction ensures that for any $f,g\in[T]$ with
    $f\neq g$, we have $f|_Tg$. By symmetrical argument it suffices to show
    that $f\not\leq_T g$. Given $e\in\omega$, at the $e$-th stage of the
    construction, requirement $R_{\langle e,i,j\rangle}$ will ensure that
    that $f\neq\Phi_e^g$, where $i,j\in\omega$ are the indices for the
    successors to the leaves that are initial segments of $f$ and $g$
    respectively. 
  \end{proof}

\it \textbf{Q3:} Show that every partial computable function $f$ has a low
  total function $g$ extending $f$. So if $f(x)$ exists then $f(x)=g(x)$.

  \begin{proof}
    We $\emptyset'$-construct $g(x)$ in stages $s$, such that
    $g(x)=\cup_{s\in\omega} g_s(x)$ for all $x\in\omega$, where $g_s\in
    \omega^{<\omega}$, $g_s\prec g_{s+1}$, and $g_s(x)$ is an extension of
    $f(x)$ when $x<|g_s|$. \\

    To ensure that the resulting $g$ is low, we force the jump by
    making sure that for all $s$, we satisfy the jump requirement
    \[J_s: (\exists \sigma\succ g_s)\; \left\{ \left[
    \Phi_s^{\sigma}(s)\downarrow\; \text{or}\;\; (\forall
    \rho\succ\sigma)\; \Phi_s^{\rho}(s)\uparrow \right]\; \text{and}\;
    \sigma\; \text{is consistent with}\; f\right\}.\]

    At stage 0, set $g_0=\emptyset$. At stage $s+1$, we satisfy $J_s$ by
    asking the $\emptyset'$-question:
    \begin{center}
      \textit{Is there a finite extension $\sigma\in\omega^{<\omega}$ of
      $g_s$ that is consistent with $f$ such that $\Phi^\sigma_s(s)$
      converges?}
    \end{center}

    Observe that this is a $\emptyset'$-question because we can
    effectively enumerate all finite extensions
    $\sigma\in\omega^{<\omega}$ of $g_s$, and for each of these extensions,
    use $\emptyset'$ to check if it is consistent with $f$. If it is
    consistent, we can use $\emptyset'$ again to check if
    $\Phi^\sigma_s(s)$ converges. Formally, this is a $\Sigma_1$-question
    \[(\exists \sigma,t)\; \left\{ g_s\prec\sigma \wedge (\forall
    x<|\sigma|)\; \left[ (\exists u\; f_u(x)\downarrow)
    \rightarrow\sigma(x)=f(x) \right]\; \wedge\;
    \Phi^\sigma_{s,t}(s)\downarrow \right\},\]
    and therefore can be answered by $\emptyset'$. If such $\sigma$ exists,
    set $g_{s+1}=\sigma$. Otherwise, let $g_{s+1}$ be the first strict
    extension of $g_s$ that is consistent with $f$ (extension needs to be
    strict to ensure that $g$ is total). Observe that in the latter case,
    every finite extension $\rho$ of $g_s$ that is consistent with $f$
    gives $\Phi_s^\rho(s)\uparrow$, so the jump requirement $J_s$ is
    satisfied. \\

    Our construction ensures that $g$ is total and extends $f$. It remains
    to verify that $J_s$ ensures that $g$ is low. We show that
    $g'\leq_T\emptyset'$. Now $g\leq_T\emptyset'$ since the construction
    asked only $\emptyset'$ questions. To decide if $e\in g'$, or
    equivalently, if $\Phi_e^g(e)$ converges, we know that at stage $e$,
    $g_{e+1}\prec g$ satisfies the jump requirement $J_e$. Thus from
    construction, \[e\in g'\; \Leftrightarrow\; \Phi_e^g(e)\downarrow\;
    \Leftrightarrow\; \Phi_e^{g_{e+1}}(e)\downarrow,\] and $\emptyset'$ can
    check if $\Phi_e^{g_{e+1}}(e)\downarrow$ holds, since $g_{e+1}\leq_T
    \emptyset'$. Therefore $g'\leq_T\emptyset'$ as required.
  \end{proof}
\end{document}
