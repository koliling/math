\documentclass{article}
\usepackage[left=3cm,right=3cm,top=3cm,bottom=3cm]{geometry}
\usepackage{amsmath,amssymb,amsthm,tikz,mathtools}
\usepackage{stmaryrd} % For double square bracket [[]]
\usepackage{color}
\usepackage[inline]{enumitem}
\usetikzlibrary{patterns}
\setlength{\parindent}{0mm}
\newcommand{\TODO}[1]{\textcolor{red}{TODO: #1}}

\begin{document}
\title{Basic Logic II: Homework 6}
\author{Li Ling Ko\\ lko@nd.edu}
\date{\today}
\maketitle

\it \textbf{Soare 3.4.7:}
  \begin{enumerate}[label={\bf (\alph*):}]
    \item Let $\{A_y\}_{y\in\omega}$ be any countable sequence of sets.
      Define the infinite join $\oplus_y A_y$ as in (2.29). Prove that
      $\text{deg}(\oplus_y A_y)$ is the uniform least upper bound for
      $\{\text{deg}(\oplus_y A_y)\}_{y\in\omega}$ in the sense that if
      there exist a set $C$ and a computable function $f$ such that
      $A_y=\Phi_{f(y)}^C$ for all $y$, then $\oplus_y A_y\leq_T C$.

      \begin{proof}
        Let $C$ be a set and $f$ a computable function such that
        $A_y=\Phi_{f(y)}^C$ for all $y$. Then to decide if $\langle
        x,y\rangle$ lies in $\oplus_y A_y$, we just check if
        $\Phi_{f(y)}^C(x)=1$. If it is then $\langle x,y\rangle$ lies in
        $\oplus_y A_y$, otherwise $\Phi_{f(y)}^C(x)=0$ and then $\langle
        x,y\rangle$ does not lie in $\oplus_y A_y$. Therefore $\oplus_y
        A_y\leq_T C$.
      \end{proof}

    \item Prove that this operation is not well defined on degrees. Namely,
      define $\{A_y\}_{y\in\omega}$ and  $\{B_y\}_{y\in\omega}$ such that
      $A_y \equiv_T B_y$ but $A\not\equiv_T B$ for $A=\oplus_y A_y$ and
      $B=\oplus_y B_y$.

      \begin{proof}
        Consider the case where $A_y(x)=0$ for all $x,y\in\omega$, and
        where
        \begin{align*}
          B_y(x) =
          \begin{cases}
            1 &\text{if}\; x=0\; \text{and}\; y\in K,\\
            0 &\text{otherwise}.\\
          \end{cases}
        \end{align*}
        Then for each $y\in\omega$, $B_y$ only differs from $A_y$ at no
        more than a single $x$ value, therefore $A_y\equiv_T B_y$. However,
        $B\not\leq_T A$, because $A=\emptyset$ is clearly computable,
        while $K\leq_T B$, since
        \[e\in K \Leftrightarrow \langle 0,e\rangle \in B.\]
      \end{proof}
  \end{enumerate}

\it \textbf{Soare 6.1.4:} Modify the proof of Theorem 6.1.1 to build an
  independent sequence $\{A_j\}_{j\in\omega}$ of sets each computable in
  $\emptyset'$.

  \begin{proof}
    We follow Soare's hint to construct the sequence. Fix any enumeration
    of $\omega^2$ such that for any given $j\in\omega$, $\langle
    s,j\rangle$ always gets enumerated before $\langle s+1,j\rangle$. We
    construct the sequence in stages $\langle s,j\rangle \in\omega^2$, so
    that $A_j=\cup_{s\in\omega} \sigma_{j}^s$, where $\sigma_j^s\prec
    \sigma_j^{s+1}$, and $\sigma_j^0=\emptyset$. At stage $\langle
    s,j\rangle$, we extend $\sigma_j^s$ to become $\sigma_j^{s+1}$ to satisfy
    the restriction
    \begin{equation}
      R_{s,j}: A_j \neq \Phi_s^{\oplus_{i\neq j} A_i}.
      \label{eq:join}
    \end{equation}

    More specifically, at stage $\langle s,j\rangle$, there are only a
    finite number indices $i_1,\ldots,i_k$ that are not equal to $j$ and
    whose initial segment $\sigma_{i_l}^s$ is non-empty. We ask the
    $\emptyset'$-question
    \[(\exists t) (\exists \rho_1 \succ \sigma_{i_1}^s) \ldots (\exists
    \rho_k \succ \sigma_{i_k}^s) \Phi_{s,t}^{\oplus_{l=1}^k
    \rho_{i_l}^s} \downarrow?\]
    If such $\rho_1,\ldots,\rho_k$ exist, then set $\sigma_j^{s+1}
    =\sigma_j^{s\frown}(\Phi_{s,t}^{\oplus_{l=1}^k \rho_{i_l}^s}+1)$, and
    $\sigma_{i_l}^{s+1}=\rho_l$ for each $l\in\{1,\ldots,k\}$.
    Otherwise set $\sigma_j^{s+1}=\sigma_j^{s}$, and
    $\sigma_{i_l}^{s+1} =\sigma_{i_l}^{s}$. \\

    Clearly this construction will satisfy restriction~(\ref{eq:join})
    for all $s,j\in\omega$. Also, restriction~(\ref{eq:join}) will ensure
    in particular that $A_j\neq \Phi_e^{A_i}$ for all $i\neq j$ and
    $e\in\omega$. 
  \end{proof}
\end{document}
