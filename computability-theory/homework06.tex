\documentclass{article}
\usepackage[left=3cm,right=3cm,top=3cm,bottom=3cm]{geometry}
\usepackage{amsmath,amssymb,amsthm,tikz,mathtools}
\usepackage{stmaryrd} % For double square bracket [[]]
\usepackage{bm} % For bold vectors
\usepackage{color}
\usepackage[inline]{enumitem}
\usetikzlibrary{patterns}
\setlength{\parindent}{0mm}
\newcommand{\TODO}[1]{\textcolor{red}{TODO: #1}}

\begin{document}
\title{Basic Logic II: Homework 6}
\author{Li Ling Ko\\ lko@nd.edu}
\date{\today}
\maketitle

\it \textbf{Soare 3.4.7:}
  \begin{enumerate}[label={\bf (\alph*):}]
    \item Let $\{A_y\}_{y\in\omega}$ be any countable sequence of sets.
      Define the infinite join $\oplus_y A_y$ as in (2.29). Prove that
      $\text{deg}(\oplus_y A_y)$ is the uniform least upper bound for
      $\{\text{deg}(\oplus_y A_y)\}_{y\in\omega}$ in the sense that if
      there exist a set $C$ and a computable function $f$ such that
      $A_y=\Phi_{f(y)}^C$ for all $y$, then $\oplus_y A_y\leq_T C$.

      \begin{proof}
        Let $C$ be a set and $f$ a computable function such that
        $A_y=\Phi_{f(y)}^C$ for all $y$. Then to decide if $\langle
        x,y\rangle$ lies in $\oplus_y A_y$, we just check if
        $\Phi_{f(y)}^C(x)=1$. If it is then $\langle x,y\rangle$ lies in
        $\oplus_y A_y$, otherwise $\Phi_{f(y)}^C(x)=0$ and then $\langle
        x,y\rangle$ does not lie in $\oplus_y A_y$. Therefore $\oplus_y
        A_y\leq_T C$.
      \end{proof}

    \item Prove that this operation is not well defined on degrees. Namely,
      define $\{A_y\}_{y\in\omega}$ and  $\{B_y\}_{y\in\omega}$ such that
      $A_y \equiv_T B_y$ but $A\not\equiv_T B$ for $A=\oplus_y A_y$ and
      $B=\oplus_y B_y$.

      \begin{proof}
        Consider the case where $A_y(x)=0$ for all $x,y\in\omega$, and
        where
        \begin{align*}
          B_y(x) =
          \begin{cases}
            1 &\text{if}\; x=0\; \text{and}\; y\in K,\\
            0 &\text{otherwise}.\\
          \end{cases}
        \end{align*}
        Then for each $y\in\omega$, $B_y$ only differs from $A_y$ at no
        more than a single $x$ value, therefore $A_y\equiv_T B_y$. However,
        $B\not\leq_T A$, because $A=\emptyset$ is clearly computable,
        while $K\leq_T B$, since
        \[e\in K \Leftrightarrow \langle 0,e\rangle \in B.\]
      \end{proof}
  \end{enumerate}

\it \textbf{Soare 6.1.4:} Modify the proof of Theorem 6.1.1 to build an
  independent sequence $\{A_j\}_{j\in\omega}$ of sets each computable in
  $\emptyset'$. Hint: Use a finite extension $\emptyset'$-computable
  construction to build at stage $s$ strings $\{p_j^s\}_{s\in\omega}$ such
  that if $A_j=\cup_s \rho_j^s$ then we meet for each $e$ and $i$ the
  requirement
  \[R_{\langle e,i\rangle} :A_i\neq \Phi_e^{A_j:j\neq i}.\]

  \begin{proof}
    We construct the sequence in stages $s$ so that for
    each $i\in\omega$, that $A_i=\cup_{s\in\omega} A_{i}^s$, where
    $A_i^0=\emptyset$, $A_i^s\in2^{<\omega}$, and $A_i^s\prec A_i^{s+1}$.
    At stage $s=\langle e,i\rangle$, we extend a finite number of initial
    segments $A_j^s$'s to satisfy the restriction
    \begin{equation}
      R_{e,i}: A_{i} \neq \Phi_{e}^{\oplus_{j\neq i} A_j},
      \label{eq:join}
    \end{equation}
    and to preserve the initial segment of $\oplus_{j\neq i}A_j$ that is
    involved in the computation. \\

    At stage 0, initialize $A_i^0=\emptyset$ for all $i\in\omega$. \\

    At stage $s+1=\langle e,i\rangle$, we want to satisfy restriction
    $R_{e,i}$. Now there are only a finite number indices
    $k_0,\ldots,k_{n-1}$ that are not equal to $i$ and whose initial
    segment $A_{k_j}^s$ is non-empty. We ask the $\emptyset'$-question
    \begin{center}
      \textit{Is there an extension $\sigma \succ A_{k_0}^s\oplus \ldots
      \oplus A_{k_{n-1}}^s\in 2^{<\omega}$ that is consistent with each
      $A_{k_m}^s$ and with $A_{i}=\emptyset$ such that $\Phi_{e}^\sigma
      (|A_{i}^s|)$ converges?}
    \end{center}

    If no such $\sigma$ exists, set $A_{i}^{s+1}=A_{i}^{s\frown}0$ (we use
    a strict extension to ensure that $A_{i}$ is total eventually), and
    set $A_{k_m}^{s+1}=A_{k_m}^{s}$ for each $m\in\{0,\ldots,n-1\}$. \\

    On the other hand, if $\sigma$ exists, we extend both
    $A_{i}^s$ and a finite number of $A_{j}^s\subseteq \sigma$ where
    $j\neq i_s$ so that we can preserve the computation of
    \[\Phi_{e}^\sigma (|A_{i}^s|) \downarrow \neq
    A_{i}(|A_{i}^s|).\]

    To preserve $A_{i}(|A_{i}^s|)$, set $A_{i}^{s+1}$ to be
    $A_{i}^s$ concatenated by the first value in $\{0,1\}$ that is
    different from $\Phi_{e}^\sigma (|A_{i}^s|)$. \\

    To preserve $\sigma$ used in the computation of $\Phi_{e_s}^\sigma
    (|A_{i_s}^s|)$, let $l_1,\ldots,l_m$ be the indices different from $i$
    such that $\langle x,l_r\rangle <|\sigma|$ for some $x\in\omega$.
    Observe that $\{k_1,\ldots,k_n\}\subseteq \{l_1,\ldots,l_m\}$ by
    construction. For each such index $l_r$, let $x_r\in\omega$ be the
    largest value such that $\langle x_r,l_r\rangle <|\sigma|$. Then set
    $A_{l_r}^{s+1}\in 2^{<\omega}$ to be the smallest (with respect to
    lexicographical order) string of length $\max(x_r+1,|A_{l_v}^s|)$ that
    is consistent with both $\sigma$ and with $A_{l_r}^s$; such string must
    exist since $\sigma$ was chosen to be consistent with $A_{l_r}^s$. \\

    Clearly this construction will satisfy restriction~(\ref{eq:join})
    for all $e,i\in\omega$. Also, restriction~(\ref{eq:join}) will ensure
    in particular that $A_j\neq \Phi_e^{A_i}$ for all $i\neq j$ and
    $e\in\omega$. 

    %We follow Soare's hint to construct the sequence. Fix any enumeration
    %of $\omega^2$ such that for any given $j\in\omega$, $\langle
    %s,j\rangle$ always gets enumerated before $\langle s+1,j\rangle$. We
    %construct the sequence in stages $\langle s,j\rangle \in\omega^2$, so
    %that $A_j=\cup_{s\in\omega} \sigma_{j}^s$, where $\sigma_j^s\prec
    %\sigma_j^{s+1}$, and $\sigma_j^0=\emptyset$. At stage $\langle
    %s,j\rangle$, we extend $\sigma_j^s$ to become $\sigma_j^{s+1}$ to satisfy
    %the restriction
    %\begin{equation}
    %  R_{s,j}: A_j \neq \Phi_s^{\oplus_{i\neq j} A_i}.
    %  \label{eq:join}
    %\end{equation}

    %More specifically, at stage $\langle s,j\rangle$, there are only a
    %finite number indices $i_1,\ldots,i_k$ that are not equal to $j$ and
    %whose initial segment $\sigma_{i_l}^s$ is non-empty. We ask the
    %$\emptyset'$-question
    %\[(\exists t) (\exists \rho_1 \succ \sigma_{i_1}^s) \ldots (\exists
    %\rho_k \succ \sigma_{i_k}^s) \Phi_{s,t}^{\oplus_{l=1}^k
    %\rho_{i_l}^s}(|\sigma_j^s|) \downarrow?\]
    %If such $\rho_1,\ldots,\rho_k$ exist, then set $\sigma_j^{s+1}$, and to
    %be $\sigma_j^s$ concatenated by the first value in $\{0,1\}$ that is
    %different from $\Phi_{s,t}^{\oplus_{l=1}^k \rho_{i_l}^s}$, and set
    %$\sigma_{i_l}^{s+1}=\rho_l$ for each $l\in\{1,\ldots,k\}$.
    %Otherwise set $\sigma_j^{s+1}=\sigma_j^{s\frown}0$ (to ensure that
    %$A_j$ is total eventually), and $\sigma_{i_l}^{s+1} =\sigma_{i_l}^{s}$.
    %\\

    %Clearly this construction will satisfy restriction~(\ref{eq:join})
    %for all $s,j\in\omega$. Also, restriction~(\ref{eq:join}) will ensure
    %in particular that $A_j\neq \Phi_e^{A_i}$ for all $i\neq j$ and
    %$e\in\omega$. 
  \end{proof}

\it \textbf{Soare 6.1.5:} A partially ordered set $\mathcal{P}=(P,\leq_P)$
  is countably universal if every countable partially ordered set is order
  isomorphic to a subordering of $\mathcal{P}$. Prove that there is a
  computable partial ordering $\leq_R$ of $\omega$ which is countably
  universal. Hint: This can be done either by considering a computably
  presented atomless Boolean algebra, or by a direct construction where at
  stage $s+1$, given a finite set $P_s$ of elements in $\leq_R$, one
  obtains $P_{s+1}$ by adding a new point for each possible order type over
  $P_s$. A boolean algebra
  $\mathcal{B}=(\{b_i\}_{i\in\omega};\leq,\vee,\wedge,')$ is computably
  presented if there exist a computable relation $P(i,j)$ and computable
  functions $f$, $g$, and $h$ such that $P(i,j)$ holes iff $b_i\leq b_j$,
  and such that $b_{f(i,j)}=b_i\vee b_j$, $b_{g(i,j)}=b_i\wedge b_j$, and
  $b_{h(i)}=b_i'$.

  \begin{proof}
    We follow Soare's first hint of using a computably presented atomless
    Boolean algebra. Consider the Boolean algebra $\mathcal{P}=(P,\leq_P)$
    where
    \[P:= \{\overline{x}\in \{0,1\}^\omega: x\in \{0,1\}^{<\omega}\},\]
    where $\overline{x}$ is the infinite string in $\{0,1\}^\omega$ when we
    repeat $x$ infinitely, and where
    \[\overline{x} \leq_B\overline{y}\; \Leftrightarrow\; \{n\in\omega:
    \overline{x}(n)=1\} \subseteq \{n\in\omega: \overline{y}(n)=1\}.\]

    This Boolean algebra is computable since $\{0,1\}^{<\omega}$ can be
    effectively enumerated. In particular $\mathcal{P}$ can be effectively
    enumerated. Furthermore, $\mathcal{P}$ is clearly computably presented,
    and does not contain atoms because given any non-zero
    $\overline{a_0,\ldots,a_{n-1}}\in P$, there is a non-zero element
    \[\overline{a_0,\ldots,a_{n-1},\underbrace{0,\ldots,0}_{n\;
    \text{zeros}}}\]
    which is smaller. \\

    Also observe that $P$ is dense: given two comparable
    elements, we can effectively construct an element that lies between
    them by considering elements whose periods are a common multiple of the
    periods of two given elements. Specifically, by taking multiples of
    periods, we can assume that the two given elements
    $\overline{a_0,\ldots,a_{n-1}} <_P\overline{b_0,\ldots,b_{n-1}}\in P$
    have the same period $n$. Let $i$ be the first index such that $b_i=1$
    but $a_i=0$; then the element
    \[\overline{a_0,\ldots,a_{i-1},1,a_{i+1},\ldots,a_{n-1},
    a_0,\ldots,a_{n-1}}\]
    lies strictly between the two given elements. \\

    Also by a considering multiples of periods, we can show that given any
    finite number of elements from $P$, we can effectively construct a new
    element that is incomparable from all those elements: Like before, we
    can assume that the given $k$ elements
    \[a_0=\overline{a_{0,0},\ldots,a_{0,n-1}},\; \ldots,\;
    a_{k-1}=\overline{a_{k-1,0},\ldots,a_{k-1,n-1}}\]
    have the same period $n$. For each element $a_i$, let $u_i$ be the
    first index such that $a_{i,u_i}=0$, and let $v_i$ be the
    first index such that $a_{i,v_i}=1$. Then the element with period $kn$
    that is 0 everywhere except at indices $ni+iu_i$ and $ni+v_i$ for all
    $i\in\{0,\ldots,k-1\}$ will be incomparable with all the elements
    $a_0,\ldots,a_{k-1}$. \\

    Let $\mathcal{R}=(\omega,\leq_R)$ be an arbitrary countable partial
    order. We construct an order-preserving embedding
    $\varphi:\mathcal{R}\rightarrow\mathcal{P}$. We construct this map in
    stages, where we define $\varphi(s)$ at stage $s$, such that the
    subordering $(\{0,\ldots,s\},\leq_R)\subset\mathcal{R}$ is isomorphic
    to its image $(\varphi(\{0,\ldots,s\}),\leq_P)\subset\mathcal{P}$. At
    stage 0, let $\varphi(0)$ be an arbitrary non-zero non-one element in
    $P$. At stage $s+1$, consider the subordering $\mathcal{R}_{s}
    :=(\{0,\ldots,s\},\leq_R)\subset\mathcal{R}$. Let
    $A=\{a_0,\ldots,a_{m-1}\}$ denote the set of direct ancestors of $s+1$
    from $\{0,\ldots,s\}$, and similarly let $D=\{d_0,\ldots,a_{n-1}\}$ be
    the set of direct descendants of $s+1$ from the same set. \\

    If $A$ and $D$ are non-empty, then map $s+1$ to any element that lies
    strictly between $\varphi(a_0)\wedge\ldots\wedge\varphi(a_{m-1})$ and
    $\varphi(d_0)\vee\ldots\vee\varphi(d_{n-1})$. If only $A$ is empty, map
    $s+1$ to any element that lies strictly between 0 and
    $\varphi(d_0)\vee\ldots\vee\varphi(d_{n-1})$. If only $D$ is empty, map
    $s+1$ to any element that lies strictly between
    $\varphi(d_0)\vee\ldots\vee\varphi(d_{n-1})$ and 1. Note that for these
    three cases the image of $s+1$ exists from the density of the
    $\mathcal{P}$. Finally, if both $A$ and $D$ are empty, then $s+1$ must
    be incomparable with all elements in $\{0,\ldots,s\}$, so map $s+1$ to
    an element that is incomparable with $\varphi(0),\ldots,\varphi(s)$.
    Again the image of $s+1$ exists from the above argument, where we
    showed that we can always find an element that is incomparable with any
    finite set of elements. \\

    By construction, $(\{0,\ldots,s+1\},\leq_R)\subset\mathcal{R}$ will be
    isomorphic to its image
    $(\varphi(\{0,\ldots,s+1\}),\leq_P)\subset\mathcal{P}$. Thus
    $\mathcal{R}$ will be isomorphic to
    $(\varphi(\omega),\leq_P)\subseteq\mathcal{P}$. More specifically, if
    $i|_R j$, then at stage $\max(i,j)$, the construction would map the
    element that is enumerated later to an element that is $\leq_P$
    incomparable with the image of the earlier element. On the other hand,
    if $i\leq_R j$, then at stage $\max(i,j)$, the construction ensure that
    $\varphi(i)\leq_P\varphi(j)$.
  \end{proof}

\it \textbf{Soare 6.1.6:} Show that for a countable partially ordered set
  $\mathcal{P}=(P,\leq_P)$ there is a 1:1 order-preserving map from $P$
  into $\bm{D}(\leq\emptyset')$, the degrees $\leq\emptyset'$. Hint:
  By Exercise 6.1.5 we may assume $P=\omega$ and $\leq_P$ is a computable
  relation. Let $\{A\}_{i\in\omega}$ be as in Exercise 6.1.4. Define
  $f:\omega\rightarrow \bm{D}(\leq\emptyset')$ by
  $f(i)=\bm{a}_i=\text{deg}(\oplus A_j:j\leq_P i)$. Show that if $i\leq_P
  j$ then $\bm{a}_i\leq\bm{a}_j$ (by definition and the fact that $\leq_P$
  is computable), and if $i\not\leq_P j$ then $\bm{a}_i\not\leq\bm{a}_j$
  (by the computable independence of $\{A_i\}_{i\in\omega}$).

  \begin{proof}
    We follow Soare's hint. By the previous exercise, any countable
    partially ordered set $\mathcal{P}=(P,\leq_P)$ can be embedded in some
    computable partial ordering. Thus it suffices to show that we can embed
    a computable partial ordering into $\bm{D}(\leq\emptyset')$.
    Therefore we can assume that $P=\omega$ and $\leq_P$ is computable. \\

    Let $\{A\}_{i\in\omega}$ be as in Exercise 6.1.4. Define
    $f:\omega\rightarrow \bm{D}(\leq\emptyset')$ by
    $f(i)=\bm{a}_i=\text{deg}(\oplus A_j:j\leq_P i)$. We first show that
    each $f(i)\leq_T\emptyset'$: To decide if $(x,j)$ lies in $\bm{a}_i$,
    we first check if $j\leq_P i$; this is decidable since $\leq_P$ is
    computable. If $j\not\leq_P i$, then $(x,j)\not\in\bm{a}_i$. On the
    other hand, if $j\leq_P i$, then we ask $\emptyset'$ if $x\in A_j$;
    $\emptyset'$ can answer this question since $A_j\leq_T\emptyset'$.
    Therefore $f(i)\leq_T\emptyset'$. \\

    Next, we show that if $i\leq_P j$ then $\bm{a}_i\leq_T\bm{a}_j$: To
    decide if $(x,k)$ lies in $\bm{a}_i$, we first check if $k\leq_P i$;
    this is decidable since $\leq_P$ is computable. If $k\not\leq_P i$,
    then $(x,k)\not\in\bm{a}_i$. On the other hand, if $k\not\leq_P i$,
    then $(x,k)$ lies in $\bm{a}_i$ if and only if it also lies in
    $\bm{a}_j$ by transitivity of $\leq_P$ and the fact that $i\leq_P j$.
    Therefore $\bm{a}_j$ can decide if $(x,k)$ lies in $\bm{a}_i$. \\

    Finally, we show that if $i\not\leq_P j$ then
    $\bm{a}_i\not\leq_T\bm{a}_j$: Assume $i\not\leq_P j$, but
    $\bm{a}_i\leq_T\bm{a}_j$ by contradiction. Since $i\not\leq_P j$, there
    must be some $k\in\omega$ such that $k\leq_P i$ but $k\not\leq_P j$.
    Then since $k\leq_P i$, $A_k\leq_T\bm{a}_i$, which implies that
    $A_k\leq_T\bm{a}_j$. Then since $k\not\leq_P j$, we have $\bm{a}_j\leq_T
    \oplus\{A_s:s\neq k\}$. Hence $A_k\leq_T \oplus\{A_s:s\neq k\}$,
    a contradiction. \\

    So $i\leq_P j$ if and only if $\bm{a}_i\leq_T\bm{a}_j$ as required.
  \end{proof}

\it \textbf{Soare 6.1.7:} Show that there are $2^{\aleph_0}$ mutually
  incomparable degrees.

  \begin{proof}
    We follow Soare's hint and construct a tree $T\subseteq2^{<\omega}$ such
    $|[T]|=2^{\aleph_0}$ and for every $f,g\in[T]$ with $f\neq g$, we have
    $f|_Tg$. Let $T=\cup_eT_e$ where $T_{e+1}\supset T_e$ and $T_{e+1}$ is
    defined by induction as follows. Let $T_0=\{\emptyset\}$, the tree with
    the empty node as its only member. Given $T_e$ define $L_e$ to be the
    leaves of tree $T_e$, namely
    \[L_e =\{\sigma: \sigma\in T_e\; \text{and}\; (\forall \tau \succ
    \sigma)\; [\tau\not\in T_e]\}.\]

    Next define the successors to leaves
    \[S_e =\{\sigma^\frown0: \sigma\in L_e\} \cup \{\sigma^\frown1:
    \sigma\in L_e\}.\]

    Suppose $S_e=\{\rho_i: i\leq2^{e+1}\}$. Fix $i,j\leq2^{e+1}$, $i\neq
    j$. We find extensions $\sigma_i\succ\rho_i$, and $\sigma_j\succ\rho_j$
    to satisfy the requirement
    \[R_{\langle e,i,j\rangle}:\; (\forall f\succ \sigma_i) (\forall g\succ
    \sigma_j)\; [\Phi_e^f\neq g\; \text{and}\; \Phi_e^g\neq f],\]
    as follows: We ask the $\emptyset'$-question
    \begin{center}
      \textit{Does there exist an extension $\sigma$ of $\rho_i$ such that
      $\Phi_e^{\sigma}(|\rho_j|)$ converges?}
    \end{center}

    Formally, this is a $\Sigma_1$-question 
    \[(\exists \sigma,t,y)\;
    [\Phi_{e,t}^\sigma (|\rho_j|)\downarrow=y\; \wedge\;
    \sigma\succ\rho_i].\]

    If such $\sigma,t,y$ exist, replace $\rho_i$ with $\sigma$ and $\rho_j$
    with $\rho_j$ concatenated with the first element in $\{0,1\}$ that is
    different from $y$. Then we know that for any extension $f\succ\rho_i$
    and $g\succ\rho_j$, we will have $\Phi_e^f(|\rho_j|-1)\downarrow\neq
    g(|\rho_j|-1)$. On the other hand, if no such $\sigma,t,y$ exist then
    $\rho_i$ and $\rho_j$ remain as is. Then we know that for any extension
    $f\succ\rho_i$ and $g\succ\rho_j$, we will have
    $\Phi_e^f(|\rho_j|-1)\uparrow$, and therefore $\Phi_e^f(|\rho_j|-1)\neq
    g(|\rho_j|-1)$. Next, we ask the question again but with the roles of
    $i$ and $j$ swapped, replacing $\rho_i$ and $\rho_j$ with respective
    appropriate extensions if necessary. Observe that the extensions will
    ensure that the requirement $R_{\langle e,i,j\rangle}$ is satisfied.
    We repeat the procedure of finding extensions $\sigma_i\succ\rho_i$,
    and $\sigma_j\succ\rho_j$ for all $i,j\leq2^{e+1}$ with $i\neq j$. \\

    We show that this construction ensures that for any $f,g\in[T]$ with
    $f\neq g$, we have $f|_Tg$. By symmetrical argument it suffices to show
    that $f\not\leq_T g$. Given $e\in\omega$, at the $e$-th stage of the
    construction, requirement $R_{\langle e,i,j\rangle}$ will ensure that
    that $f\neq\Phi_e^g$, where $i,j\in\omega$ are the indices for the
    successors to the leaves that are initial segments of $f$ and $g$
    respectively. 
  \end{proof}

\it \textbf{Soare 7.3.6:}
  \begin{enumerate}[label={(\roman*)}]
    \item Show that there is a u.c.e. sequence of c.e. sets
      $\{A_i\}_{i\in\omega}$ such that for every $i$, $A_i\not\leq_T \oplus
      \{A_j\}_{j\neq i}$. Hint: Modify the construction of the
      Friedberg-Muchnik Theorem 7.3.1 to meet the requirements of
      $R_{\langle e,i\rangle}$ of Exercise 6.1.4.

      \begin{proof}
        We follow Soare's hint. For each index $i\in\omega$, we construct
        $A_i$ in stages $s$ such that $A_i=\cup_{s\in\omega}A_i^s$, where
        $A_i^s\subset\omega$, $A_i^0=\emptyset$, and $A_i^s\subseteq
        A_i^{s+1}$. For each $e,i\in\omega$, we want to satisfy the
        requirement \[R_{\langle e,i\rangle}: A_i\neq\Phi_e^{\oplus_{j\neq
        i}A_j},\] giving higher priority to requirements $R_{\langle
        e,i\rangle}$ with a smaller $\langle e,i\rangle\in\omega$ value. \\

        To avoid conflicts, for each $e,i\in\omega$, we choose witnesses
        from amongst $w_{e,i}:=\{\langle e,i,x\rangle: x\in\omega\}$ to
        satisfy requirement $R_{\langle e,i\rangle}$. Also, at a given
        stage $s\in\omega$ and for each $e,i\in\omega$, we define the
        restraint function $r(e,i,s)$ to preserve the computation involved
        in satisfying $R_{\langle e,i\rangle}$. We reset $r(e,i,s+1)=0$ if
        $R_{\langle e,i\rangle}$ is injured at stage $s+1$. Thus
        $r(e,i,s)=0$ indicates that $R_{\langle e,i\rangle}$ has either
        never been acted, or acted, injured, and never acted
        since then until now. \\

        \textit{Stage} $s=0$: Initialize $r(e,i,0)=0$ and $A_i=\emptyset$
        for all $e,i\in\omega$. \\

        \textit{Stage} $s+1$: For each requirement $R_{\langle e,i\rangle}$
        where $\langle e,i\rangle \leq s$ and where requirement $R_{\langle
        e,i\rangle}$ is not being acted on (i.e. $r(e,i,s)=0$), let $x\in
        w_{e,i}-A_i^s$ be the smallest witness that is greater than the
        restraints imposed by higher priority requirements that are
        being acted on. Formally, given $e,i\in\omega$, consider the higher
        priority requirements $R_{\langle e',i'\rangle}$ where
        $\langle e',i'\rangle <\langle e,i\rangle$, $i'\neq i$, and
        $r(e',i',s)>0$. Then by construction, $R_{\langle e',i'\rangle}$
        was acted at stage $s'=r(e',i',s)$, and the computation
        involved in acting $R_{\langle e',i'\rangle}$ is of the form
        \[\Phi_{e',s'}^{(\oplus_{j\neq i'} A_{j}^{s'}) \restriction s'}
        (x')\downarrow\neq A_{i'}(x')\]
        for some $x'\in\omega$. We need to ensure that we choose a
        sufficiently large $x$ as potential witness for $R_{\langle
        e,i\rangle}$ so that should $x$ be enumerated into $A_i^{s+1}$ at
        this stage $s+1$, we would preserve
        \[(\oplus_{j\neq i'} A_{j}^{s'}) \restriction s' =(\oplus_{j\neq
        i'} A_{j}^{s+1}) \restriction (s+1).\]

        To that end, choose $x\in w_{e,i}-A_i^s$ to be larger than all $y$
        where $\langle y,i\rangle <s'$. Such $x$ ensures that the
        computations of the higher priority requirements that have been
        acted will not be injured. \\

        Let $k_0,\ldots,k_{n-1}$ be the indices not equal to $i$ and whose
        associated string $A_{k_m}^s$ is non-empty at this stage. Check if
        \begin{equation}
          \Phi_{e,s}^{(\oplus_{j<n} A_{k_j}^s) \restriction s}(x)
          \downarrow=0.
          \label{eq:converge}
        \end{equation}

        Let $R_{\langle e,i\rangle}$ where $0\leq \langle e,i\rangle \leq
        s$ be the highest priority requirement that is not being acted and
        for which equation~\eqref{eq:converge} is satisfied. If no such
        requirement $R_{\langle e,i\rangle}$ exists, set
        $A_j^{s+1}=A_j^s$ for all indices $j$ where $A_j^s$ is non-empty,
        and $r(e',i',s+1)=r(e',i',s)$ for all $e',i'\in\omega$,
        then go to the next stage. Otherwise we say that $R_{\langle
        e,i\rangle}$ acts at stage $s+1$, and we perform the following
        steps. \\

        \textit{Step 1.} Enumerate $x$ into $A_i^s$. In order words, set
        $A_i^{s+1}=A_i^s\cup\{x\}$. Observe that from our choice of $x$
        satisfying the restraint conditions of higher priority
        requirements, the higher priority requirements will not be injured
        from the enumeration. \\

        \textit{Step 2.} Update the restraint function $r(e,i,s+1)=s$. \\

        \textit{Step 3.} Update the restraint functions for lower priority
        requirements - set $r(e',i',s+1)=0$ for all $\langle e',i'\rangle
        >\langle e,i\rangle$. These lower priority requirements
        $\{R_{\langle e',i'\rangle}\}_{\langle e',i'\rangle >\langle
        e,i\rangle}$ are injured at stage $s+1$ and are reset. \\

        \textit{Step 4.} Preserve the restraint functions for higher
        priority requirements - set $r(e',i',s+1)=r(e',i',s)$ for all
        $\langle e',i'\rangle <\langle e,i\rangle$. \\

        \textbf{Verification:} We now verify that the construction works.
        $\{A_i\}_{i\in\omega}$ is clearly u.c.e from construction. We first
        show by induction on $n\in\omega$ that all
        requirements $R_{n}$ are act and are injured
        only a finite number of times. For the base case, $R_0$ is never
        injured since requirements can only be injured when higher order
        requirements act, and there are no requirements that are
        of higher priority than $R_0$. Also, $R_0$ can only act at
        most once, because once it has acted, then it will never be
        injured and therefore never needs to act again. \\

        For the inductive step $n$, assume that $R_0,\ldots,R_{n-1}$ are
        all only injured and acted a finite number of times. Let
        $s\in\omega$ be the first stage after which all the these
        requirements have been injured and acted for the last time.
        Then $R_n$ will never be injured after stage $s$ since the
        requirements of higher priority are never injured again.
        Also, once $R_n$ is acts stage $s$, it will never be
        injured since the higher priority requirements never act
        again. Therefore $R_n$ will only act or be injured a finite
        number of times. \\

        \textit{Base case:} Now we verify by induction on $\langle
        e,i\rangle\in\omega$ that all requirements $R_{\langle e,i\rangle}$
        are satisfied. For the base case $\langle e,i\rangle=0$, assume by
        contradiction that $A_i(x)=\Phi_e^{\oplus_{j\neq i} A_j}(x)$ for
        all $x\in\omega$. \\

        Now from construction, if $R_{\langle e,i\rangle}$ acts at stage
        $s$, then $R_{\langle e,i\rangle}$ will never be injured
        because it is the requirement of the highest priority and the
        restraint function $r(e,i,s)$ preserves the computations involved
        in acting $R_{\langle e,i\rangle}$. Therefore, if $R_{\langle
        e,i\rangle}$ is not satisfied, it must never have acted.
        Then from our choice of potential witness, none of the witnesses in
        $w_{e,i}$ was ever enumerated into $A_i$, so the at every stage
        $s>0$, we would consider trying to act $R_{\langle e,i\rangle}$
        using the same witness $x\in w_{e,i}$, and $x\not\in A_i$. \\

        Now since $\Phi_e^{\oplus_{j\neq i} A_j}(x)\downarrow$ by
        assumption, and $\{A_j\}_{j\in\omega}$ is u.c.e., there must be
        some stage $s\in\omega$ where
        \[\oplus_{j\neq i} A_j^s \restriction s =\oplus_{j\neq i} A_j,\]
        so that
        \begin{equation}
          \Phi_{e,s}^{\oplus_{j\neq i} A_j^s \restriction s}(x)
          =\Phi_e^{\oplus_{j\neq i} A_j}(x) \downarrow.
          \label{eq:sat}
        \end{equation}

        Then since $x$ was never enumerated into $A_i$, at stage $s$,
        $A_i^s(x)=0$, so from assumption equation~\eqref{eq:sat} also
        converges to 0 at stage $s$. Then since the $R_{\langle
        e,i\rangle}$ is of highest priority and never acted, at
        stage $s$ we must have chosen to act $R_{\langle
        e,i\rangle}$ and enumerated $x$ into $A_i$, a contradiction. \\

        \textit{Inductive step:} Assume all higher priority requirements
        $\{R_n\}_{n <\langle e,i\rangle}$ have been satisfied but
        $A_i(x)=\Phi_e^{\oplus_{j\neq i} A_j}(x)$ for all $x\in\omega$ by
        contradiction. By our earlier claim, all requirements are injured
        and acted finitely many times, so wait for a stage
        $s'$ where all requirments $R_0,\ldots,R_{\langle e,i\rangle}$ have
        been injured and acted for the last time. Then $R_{\langle
        e,i\rangle}$ cannot be acted stage $s'$, and is never
        acted after stage $s'$. \\

        Consider the potential witness that for
        $R_{\langle e,i\rangle}$ at stages greater than $s'$. Note that we
        will always consider acting $R_{\langle
        e,i\rangle}$ because the higher prioity requirements that are not
        acted at stage $s'$ never act after stage $s'$.
        Observe that since the restraints imposed by the higher priority
        requirements never change after stage $s'$, the potential witness
        $x\in w_{e,i}-A_i^{s'}$ that we consider will always be the same
        one. Our argument from here will then be similar to the argument
        for the base case - since $\Phi_e^{\oplus_{j\neq i} A_j}(x)$
        converges by assumption, we wait for the first stage $s\geq s'$
        where the sequences involved in the computation stabilizes. Now
        $x\not\in A_i^{s'}$ and $x$ is never enumerated into $A_i$
        because $R_{\langle e,i\rangle}$ never acts after stage
        $s'$, so we must have $\Phi_e^{\oplus_{j\neq i} A_j}(x)=A_i(x)=0$,
        and $R_{\langle e,i\rangle}$ must have been selected to be acted,
        then $x$ will be enumerated into $A_i$ at stage $s$, a
        contradiction.
      \end{proof}

    \item Show that any countable partially ordereed set can be embedded in
      the c.e. degrees $(\bm{C},\leq_T)$ by an order-preserving map.
      (See the hint for Exercise 6.1.6.)

      \begin{proof}
        Similar to Exercise 6.1.6, we may assume that the given partially
        ordered set is of the form $\mathcal{P}=(\omega,\leq_P)$, where
        $\leq_P$ is computable. Let $\{A\}_{i\in\omega}$ be the u.c.e.
        sequence of c.e. sets constructed in the first part of this
        question. Define the map $f:\omega\rightarrow \bm{C}$ by
        $f(i)=\bm{a}_i=\text{deg}(\oplus A_j:j\leq_P i)$. The argument that
        $i\leq_P j$ if and only if $\bm{a}_i\leq_T\bm{a}_j$ is the same as
        for Exercise 6.1.6. Thus it remains to show that map $f$ is
        well-defined, i.e. given any $i\in\omega$, $\bm{a}_i$ is c.e. \\

        Let $i\in\omega$ be arbitrary, and $g(e)$ be a computable function
        such that $A_e=W_{g(e)}$ for all $e\in\omega$. We show that there
        is a $\Sigma_1$-formula that decides if a given $(x,j)$ lies in
        $\bm{a}_i$. We first check if $j\leq_P i$; this is decidable since
        $\leq_P$ is computable. If $j\not\leq_P i$, then
        $(x,j)\not\in\bm{a}_i$. On the other hand, if $j\leq_P i$, then we
        wait for $A_j$ to enumerate $x$. Formally, $(x,j)\in\bm{a}_i$ if
        and only if $(x,j)$ satisfies the $\Sigma_1$-formula
        \[j\leq_Pi\;\; \wedge\;\; (\exists s)\; [x\in W_{g(j),s}].\]
      \end{proof}
  \end{enumerate}

\it \textbf{Soare 7.3.7:} (Cooper-Epstein-Lachlan) Construct a pair of c.e.
  sets $A$ and $B$ such that the d.c.e. set $D=A-B$ does not have a c.e.
  degree.

  \begin{proof}
    We follow Soare's hint. We construct $A$ and $B$ in stages $s$ such
    that $A=\cup_{s\in\omega}A_s$ and $B=\cup_{s\in\omega}B_s$, where
    $A_0=B_0=\emptyset$, $A_s,B_s\subset\omega$, $A_s\subseteq A_{s+1}$ and
    $B_s\subseteq B_{s+1}$. For each $e,i,j\in\omega$, we want to satisfy
    the requirement
    \begin{equation}
      R_{\langle e,i,j\rangle}: D\neq \Phi_i^{W_e}\; \vee\;
      W_e\neq\Phi_j^D,
      \label{eq:requirement}
    \end{equation}
    giving higher priority to requirements $R_{\langle e,i,j\rangle}$ with
    smaller $\langle e,i,j\rangle\in\omega$ value. \\

    For each $e,i,j\in\omega$, we always try to act at the first clause
    of $R_{\langle e,i,j\rangle}$ first, by waiting for a stage
    $s\in\omega$ and finding a witness $x\in\omega$ for which
    \[D(x) \neq \Phi_{i,s}^{W_{e,s}}(x).\]
    However this clause can be injured at later stages when a new element
    $y$ gets enumerated into $W_e$ and affecting the computation
    $\Phi_{i,s}^{W_{e,s}}$. When that happens, we act at the second clause
    of $R_{\langle e,i,j\rangle}$ using witness $y$ instead: 
    \[W_e(y) \neq \Phi_j^D(y).\]
    By our construction, once the second clause is acted, $R_{\langle
    e,i,j\rangle}$ will never be injured except by higher priority
    requirements. \\

    To avoid conflicts, for each $e,i,j\in\omega$, we choose witnesses from
    amongst $w_{e,i,j}:=\{\langle e,i,j,x\rangle: x\in\omega\}$ to satisfy
    requirement $R_{\langle e,i,j\rangle}$. Also, at a given stage
    $s\in\omega$ and for each $e,i,j\in\omega$, we define use functions
    $u(e,i,j,s)$ and $v(e,i,j,s)$ to keep track of the length of $W_{e,s}$
    and of $D_s=A_s-B_s$ respectively that has been used to act on
    requirement $R_{\langle e,i,j\rangle}$. The use function $v(e,i,j,s)$
    for $D_s$ is also used to preserve the computation involved in
    acting $R_{\langle e,i,j\rangle}$ - lower priority requirements are
    not allowed to enumerate any values smaller or equal to $v(e,i,j,s)$
    into $A$ or $B$. We reset the use functions
    $u(e,i,j,s+1)=v(e,i,j,s+1)=0$ if $R_{\langle e,i,j\rangle}$ is injured
    at stage $s+1$. Thus $u(e,i,j,s)=v(e,i,j,s)=0$ indicates that
    $R_{\langle e,i,j\rangle}$ has either never been acted, or
    acted, injured, and never acted since then until now. \\

    Similarly, at a given stage $s\in\omega$ and for each $e,i,j\in\omega$,
    we define the witness parameter $x(e,i,j,s)$ to keep track of
    the witness that was used to act at the first clause of $R_{\langle
    e,i,j\rangle}$, if it is the first clause (and not the second one) that
    is acted. By convention we initialize $x(e,i,j,s)=-1$, and also
    reset $x(e,i,j,s)=-1$ to indicate that the first clause is injured and
    the second clause is being used to act $R_{\langle e,i,j\rangle}$
    instead. \\

    \textit{Stage} $s=0$: Initialize $r(e,i,j,0)=u(e,i,j,0)=v(e,i,j,0)=0$,
    $x(e,i,j,0)=-1$ for all $e,i,j\in\omega$, and set $A_0=B_0=\emptyset$.
    \\

    \textit{Stage} $s+1$: We first check if any of the higher priority
    requirements $R_0,\ldots,R_{s-1}$ that were acted at the first
    clause in equation~\eqref{eq:requirement} is no longer satisfied by the
    first clause, and needs to be changed to be acted at the second
    clause. Formally, let $\langle e',i',j'\rangle <s$ be the smallest
    index (if it exists) such that all of the following holds:

    \begin{enumerate}
      \item $v(e',i',j',s)>0$.
      \item $x(e',i',j',s)\neq-1$.
      \item There exists some $y\in W_{e',s+1}-W_{e',s}$ such that
        $y<u(e',i',j',s)$.
    \end{enumerate}

    The first two conditions ensure that $R_{\langle e',i',j'\rangle}$ was
    acted at the first clause, and the third condition ensures that an
    element that was newly enumerated into $W_e$ injured the first clause.
    If such $e',i',j'$ exist, we say $R_{\langle e',i',j'\rangle}$ acts by
    the second clause at stage $s+1$, and perform the following steps. \\

    \textit{Step 1.1} Enumerate $x(e',i',j',s)$ into $B$, i.e. set
    $B_{s+1}=B_{s}\cup\{x(e',i',j',s)\}$. Then $R_{\langle
    e',i',j'\rangle}$ will be acted at the second clause with witness
    $y$. $A$ will remain as is - set $A_{s+1}=A_s$. \\
    
    \textit{Step 1.2} Set $x(e',i',j',s+1)=-1$ to indicate that $R_{\langle
    e',i',j'\rangle}$ is acted at the second clause. Also, set
    $u(e',i',j',s+1)=u(e',i',j',s)$, and $v(e',i',j',s+1)=v(e',i',j',s)$.
    \\

    \textit{Step 1.3} Preserve all higher priority requirements - for all
    $\langle e,i,j\rangle <\langle e',i',j'\rangle$, set
    $u(e,i,j,s+1)=u(e,i,j,s)$, $v(e,i,j,s+1)=v(e,i,j,s)$, and
    $x(e,i,j,s+1)=x(e,i,j,s)$. \\

    \textit{Step 1.4} All lower priority requirements are injured - for all
    $\langle e,i,j\rangle >\langle e',i',j'\rangle$, reset
    $u(e,i,j,s+1)=v(e,i,j,s+1)=0$, and $x(e,i,j,s+1)=-1$. Then go to
    the next stage. \\

    On the other hand, if no $e',i',j'$ can be found (i.e. no requirement
    acts by the second clause at stage $s+1$), we check for requirements
    that act by the first clause instead. For each requirement $R_{\langle
    e,i,j\rangle}$ where $\langle e,i,j\rangle \leq s$ and where
    requirement $R_{\langle e,i,j\rangle}$ is not acted by either clauses
    (i.e.  $v(e,i,j,s)=0$), let $x\in w_{e,i,j}-(A_s\cup B_s)$ be the
    smallest witness that is greater than the restraints imposed by higher
    priority requirements. More specifically, choose $x$ to be the smallest
    element in $w_{e,i,j}-(A_s\cup B_s)$ that is greater than $v(e',i',j')$
    for all $\langle e',i',j'\rangle <\langle e,i,j\rangle$. This choice of
    $x$ ensures that the computations of the higher priority requirements
    that have been acted will not be injured. \\

    Next, check if there exists some $u,v\leq s$ such that the following
    conditions hold:
    \begin{equation}
      0 =\Phi_{i,s}^{W_{e,s}\restriction (u+1)} (x)\;\; \text{and}\;\;
      W_{e,s}\restriction (u+1) =\Phi_{j,s}^{D_s\restriction (v+1)}
      \restriction (u+1).
      \label{eq:cooper}
    \end{equation}

    Let $R_{\langle e,i,j\rangle}$ where $0\leq \langle e,i,j\rangle \leq
    s$ be the highest priority requirement that is not acted and for
    which equation~\eqref{eq:cooper} is satisfied, and let $u,v\leq s$ be
    the smallest witnesses. If no such requirement
    $R_{\langle e,i,j\rangle}$ exists, set $A_{s+1}=A_s$, $B_{s+1}=B_s$,
    and for all $e,i,j\in\omega$, set $u(e,i,j,s+1)=u(e,i,j,s)$,
    $v(e,i,j,s+1)=v(e,i,j,s)$, $x(e,i,j,s+1)=x(e,i,j,s)$,
    then go to the next stage. Otherwise we say that $R_{\langle
    e,i,j\rangle}$ acts by the first clause at stage $s+1$, and we perform
    the following steps. \\

    \textit{Step 2.1.} Enumerate $x$ into $A$. In order words, set
    $A_{s+1}=A_s\cup\{x\}$. Observe that from our choice of $x$
    satisfying the restraint conditions of higher priority
    requirements, the higher priority requirements will not be injured
    from the enumeration. \\

    \textit{Step 2.2.} Update the use functions and the witness parameter -
    set $u(e,i,j,s+1)=u$, $v(e,i,j,s+1)=v$, $x(e,i,j,s+1)=x$. \\

    \textit{Step 2.3.} Reset functions for lower priority requirements
    since they are all injured - for all $\langle e',i',j'\rangle >\langle
    e,i,j\rangle$, set $u(e',i',j',s+1)=v(e',i',j',s+1)=0$,
    $x(e,i,j',s+1)=-1$. \\

    \textit{Step 2.4.} Preserve the functions for higher
    priority requirements - for all $\langle e',i',j'\rangle <\langle
    e,i,j\rangle$, set $u(e',i',j',s+1)=u(e',i',j',s)$,
    $v(e',i',j',s+1)=v(e',i',j',s)$, and $x(e',i',j',s+1)=x(e',i',j',s)$.
    \\

    \textbf{Verification:} We now verify that the construction works.
    $A$ and $B$ are both clearly c.e from construction. We first show that
    if $R_{\langle e,i,j\rangle}$ is satisfied for every $e,i,j\in\omega$,
    then $D$ does not have a c.e. degree: Because if $D$ has a c.e. degree,
    then there must exist some $i,j\in\omega$ and c.e. set $W_e$ such that
    \[D=\Phi_i^{W_e}\;\; \text{and}\;\; W_e=\Phi_j^D,\]
    thus $R_{\langle e,i,j\rangle}$ would not be satisfied. \\

    Next, we show by induction on $n\in\omega$ that there are only a finite
    number of times where $R_n$ can get injured, get acted at the first
    clause, or get acted at the second clause. For the base case, $R_0$ is
    never injured since requirements can only be injured when higher order
    requirements act, and there are no requirements that are of
    higher priority than $R_0$. Also, $R_0$ can only act at either
    clauses at most once, because once it acts by the second clause, it
    will never be injured, and by construction requirements can only
    act at the first clause after it has been injured or before it has
    ever acted at any clause. \\

    For the inductive step $n$, assume that $R_0,\ldots,R_{n-1}$ are all
    only injured and acted a finite number of times by either clauses.
    Let $s\in\omega$ be the first stage after which all the these
    requirements have been injured and acted for the last time. Then
    $R_n$ will never be injured after stage $s$ since the requirements of
    higher priority again. Also, once $R_n$ is acted at the first
    clause after stage $s$, it will either never be injured, or be
    acted at the second clause and then never injured since the higher
    priority requirements are never acted again. Therefore $R_n$ will
    only be acted or injured a finite number of times. \\

    \textit{Base case:} We verify by induction on $\langle
    e,i,j\rangle\in\omega$ that all requirements $R_{\langle e,i,j\rangle}$
    are satisfied. For the base case $\langle e,i,j\rangle=0$, assume by
    contradiction that for all $x\in\omega$,
    \[D(x)=\Phi_i^{W_e}(x)\;\; \text{and}\;\; W_e(x)=\Phi_j^D(x).\]

    By earlier argument, $R_{\langle e,i,j\rangle}$ can act a finite number
    of times. Wait for a stage after $R_{\langle e,i,j\rangle}$ acts for
    the last time. Then there are three possible cases - $R_{\langle
    e,i,j\rangle}$ never acted at either clauses; the last time $R_{\langle
    e,i,j\rangle}$ acted was by the first clause; or the last time
    $R_{\langle e,i,j\rangle}$ acted was by the second clause. \\

    Assume $R_{\langle e,i,j\rangle}$ never acted by either clauses.
    Now $R_{\langle e,i,j\rangle}$ has the highest priority, so it is never
    injured. Thus we must have always chosen the same potential witness
    $x\in w_{e,i,j}$ for $R_{\langle e,i,j\rangle}$ for all stages of the
    construction. Also, since the requirement never acted at the first
    clause, $x$ was never enumerated into $A$ or $B$. Then since
    $\Phi_i^{W_e}(x)=D(x)=0$ by assumption, at some large enough stage
    $s'\in\omega$, we must have
    \[W_e\restriction (s'+1) =W_{e,s'}\restriction (s'+1)\;\;
    \text{and}\;\; \Phi_{i}^{W_{e} \restriction (s'+1)}(x)
    =\Phi_{i,s'}^{W_{e,s'} \restriction (s'+1)}(x)=0.\]

    Similarly, since $W_e\restriction(s'+1) =\Phi_j^D\restriction(s'+1)$
    and $\Phi_j^D$ is a total function by assumption, at some large enough
    stage $s\geq s'$, we must have
    \[D\restriction(s+1) =D_{s}\restriction(s+1)\;\;
    \text{and}\;\; \Phi_{j}^{D\restriction(s+1)}
    \restriction(s+1) =\Phi_{j,s}^{D_{s}\restriction(s+1)}
    \restriction(s+1).\]
    Thus at stage $s$ condition~\eqref{eq:cooper} will be satisfied, then
    $R_{\langle e,i,j\rangle}$ will act at the first clause, a
    contradiction. \\

    Next, assume that last time $R_{\langle e,i,j\rangle}$ acted was
    at the first clause. Then $R_{\langle e,i,j\rangle}$ never acted
    at the second clause. Let $s$ be the stage at which the requirement
    acted at the first clause. Then
    \[0=D_s(x) =\Phi_{i,s}^{W_{e,s}\restriction (u+1)} (x)\;\;
    \text{and}\;\; 1=D(x)=D_{s+1}(x) \neq\Phi_{i,s}^{W_{e,s}\restriction
    (u+1)} (x)\]
    for some $u\leq s$. Yet, for all stages $t>s$, no new elements were
    enumerated into $W_e\restriction (u+1)$ since $R_{\langle
    e,i,j\rangle}$ never acted at the second clause, so the computation
    for inequality in the right clause above is preserved, i.e.
    \[\Phi_{i,s}^{W_{e,s}\restriction (u+1)} (x)
    =\Phi_{i}^{W_{e}\restriction (u+1)} (x).\]
    Thus the right clause of $R_{\langle e,i,j\rangle}$ is not satisfied, a
    contradiction. \\

    Finally, assume that last time $R_{\langle e,i,j\rangle}$ acted was at
    the second clause. From construction, if $R_{\langle e,i,j\rangle}$
    acts at the second clause at stage $s$, then $R_{\langle e,i,j\rangle}$
    will forever stay acted at the second clause because $R_{\langle
    e,i,j\rangle}$ has highest priority and the use function $v(e,i,j,s)$
    preserves the computations involved in acting at the second clause.
    More precisely, when the requirement acts at the second clause, we
    have
    \[1=W_{e,s}(y) =\Phi_{j,s}^{D_s\restriction (v(e,i,j,s)+1)}(y)\]
    for some $y\in\omega$ that is enumerated into both $A$ and $B$ at stage
    $s+1$. But $D\restriction (v(e,i,j,s+1)+1) =D_{s+1}\restriction
    (v(e,i,j,s+1)+1)$ since $v(e,i,j,s+1)$ prevents elements smaller than
    it from entering $A$ or $B$ after stage $s+1$, thus
    \[1=W_{e}(y) =\Phi_{j}^{D\restriction (v(e,i,j,s)+1)}(y),\] which
    implies $R_{\langle e,i,j\rangle}$ acted at the second clause
    with witness $y\in A\cap B$. Therefore, if $R_{\langle e,i,j\rangle}$
    is not satisfied, then $R_{\langle e,i,j\rangle}$ must have never
    have acted at the second clause. \\

    \textit{Inductive step:} Assume all higher priority requirements
    $\{R_n\}_{n <\langle e,i,j\rangle}$ have been satisfied but $R_{\langle
    e,i,j\rangle}$ is not. By our earlier claim, all requirements are
    injured and acted at only a finite number of times, so wait for a stage
    $s$ where all requirements $R_0,\ldots,R_{\langle e,i,j\rangle}$ have
    been injured and acted at for the last time. Then at stage $s$,
    $R_{\langle e,i,j\rangle}$ is either injured or has never acted, and
    after stage $s$ $R_{\langle e,i,j\rangle}$ will never act. Also, since
    higher priority requirements never act after stage $s$, the potential
    witness considered for $R_{\langle e,i,j\rangle}$ will always be the
    same after stage $s$. Therefore we can use the same argument for the
    base case to give a contradiction if $R_{\langle e,i,j\rangle}$ is not
    satisfied. \\
  \end{proof}

\it \textbf{Soare 5.3.9:} Give a direct ``movable marker'' type
  construction of an h-simple set. Hint: Modify the construction of the
  canonical simple set in Theorem 5.2.5 with requirement $P_e$ of (5.1)
  replaced by the requirement
  \[\hat{P_e}: \{D_{\varphi_e(x)}\}_{x\in\omega}\; \text{a disjoint strong
  array}\; \Rightarrow\; (\exists x)[D_{\varphi_e(x)} \subseteq A].\]

  \begin{proof}
    We follow Soare's hint. For every $e\in\omega$ we meet the requirement
    $\hat{P_e}$. We construct the h-simple set $A$ in stages $s$, so that
    $A=\cup_{s\in\omega} A_s$, where $A_s\subset\omega$, $A_0=\emptyset$,
    and $A_s\subseteq A_{s+1}$. \\

    \textit{Stage 0.} Initialize $A_0=\emptyset$. \\

    \textit{Stage $s+1$.} Write
    \[\overline{A_s}=\{a_0^s<a_1^s<\ldots\}.\]
    Given $A_s$, choose the least $e\leq s$ that meets all these
    conditions:

    \begin{enumerate}[label={(\roman*)}]
      \item There is some $t\leq s$ such that $\varphi_{e,s}(x)\downarrow$
        for all $x<t$
      \item $\{D_{\varphi_{e}(0)}, \ldots, D_{\varphi_{e}(t-1)}\}$ is a
        disjoint strong array
      \item For all $x<t$, $D_{\varphi_{e}(x)}\not\subseteq A_s$
      \item For some $x<t$, all elements of $D_{\varphi_{e}(x)}$ are
        greater than $a_e^s$.
        \label{cond:marker}
    \end{enumerate}

    Choose the least such $e\leq s$ satisfying all these conditions, and
    for this $e$, choose the least $x<t$ that satisfies
    condition~\ref{cond:marker}, and enumerate all elements of
    $D_{\varphi_{e}(x)}$ in $A_{s+1}$. We say that requirement $\hat{P_e}$
    acts at stage $s+1$. If there is no such $e$, go to the next stage. \\

    \textbf{Verification}: $A$ is clearly c.e. from construction. We first
    show that $|\overline{A}|$ is infinite. Let $n\in\omega$ be arbitrary.
    We show that $|\overline{A}|>n$. Observe that each requirement
    $\hat{P_e}$ can only act at most once, because once $\hat{P_e}$ has
    acted, condition (iii) will never be satisfied again. So wait for the
    first stage $s\in\omega$ such that all requirements
    $\hat{P_0},\ldots,\hat{P_{n}}$ have acted, if they ever act. Then after
    stage $s$, the only requirements that act must have index greater than
    $n$. So by condition (iv), none of the elements
    \[a_0^s<\ldots<a_n^s\]
    will ever be enumerated into $A$, since these elements enter $A$ only
    when a requirement of index lower than $n$ acts. In particular,
    $\{a_0^s,\ldots,a_n^s\} \subset \overline{A}$, so $|\overline{A}|>n$.
    \\

    Next, we show that $\hat{P_e}$ is satisfied for all $e\in\omega$. Let
    $e\in\omega$ be arbitrary. We can assume that
    $\{D_{\varphi_e(x)}\}_{x\in\omega}$ is a disjoint strong array,
    otherwise $\hat{P_e}$ is trivially satisfied. By earlier argument, we
    can wait for the first stage $s\in\omega$ such that all requirements
    $\hat{P_0},\ldots,\hat{P}_{e-1}$ have acted, if they ever act. We can
    assume that $\hat{P_e}$ has not acted, otherwise this requirement would
    be satisfied. Now from our choice of $s$, requirements with index
    smaller than $e$ will never act after stage $s$, so elements
    \[a_0^s<\ldots<a_e^s\] will never be enumerated into $A$. In
    particular, at all stages $s'>s$, we will have $a_i^{s'}=a_i^s$ for all
    $i\leq e$. Now since the strong array associated with $\varphi_e$ is
    disjoint, there must exist some large enough $x>s$ such that all
    elements of $D_{\varphi_e(x)}$ are greater than $a_e^s$. Then at some
    large enough stage $t>x$, all elements
    $\varphi_{e,t}(0),\ldots,\varphi_{e,t}(x)$ will converge.
    Therefore, at stage $t>s$, if $\hat{P}_e$ has not already acted, then
    conditions (i) to (iv) would be satisfied. And since requirements of
    lower index have already acted if they ever act, we would choose to
    act on requirement $\hat{P}_e$ at stage $t$, and therefore satisfy
    $\hat{P}_e$.
  \end{proof}

\it \textbf{Soare 5.3.14:} A set $S$ is introreducible if $S\leq_T T$ for
  every infinite set $T\subseteq S$. Let $A$ be an infinite c.c. set and
  $f$ a 1:1 computable function with range $A$. Let $a_s$ denote $f(s)$ and
  $A_s=\{a_v:v\leq s\}$. Let
  \[\overline{D} :=\{t: A\restriction a_t =A_t\restriction a_t\}.\]
  Prove that $\overline{D}$ is introreducible.

  \begin{proof}
    Let $T\subseteq\overline{D}$ be an arbitrary infinite subset of
    $\overline{D}$. Given $t\in\omega$, we can decide if $t$ lies in $T$
    using only $\overline{D}$, as follows. We ask $\overline{D}$ for the
    smallest index $u\in\overline{D}$ such that $u>t$ and
    $a_{u-1}>a_{t-1}$. Such $u$ must exist since $\overline{D}$ is infinite
    and since the $a_s$'s are distinct, therefore this question is
    recursive in $\overline{D}$. Since $u\in\overline{D}$, all elements
    $a_u,a_{u+1},\ldots$ must be greater than $a_{u-1}$, and therefore also
    greater than $a_{t-1}$. Thus $t\in T$ if and only if the elements
    $a_t,a_{t+1},\ldots,a_{u-2}$ are also greater than $a_{t-1}$.  Since
    there are only a finite elements to check, and $f$ is recursive, we can
    decide if $t$ lies in $T$.
  \end{proof}

\it \textbf{Soare 5.4.5:} Prove that the following are equivalent for an
  arbitrary set $A$:

  \begin{enumerate}[label={(\roman*)}]
    \item $(\exists f\leq_T A)\; (\forall e)\; [W_e\neq W_{f(e)}]$,
    \item $(\exists g\leq_T A)\; (\forall e)\; [\varphi_e\neq
      \varphi_{g(e)}]$,
    \item $(\exists h\leq_T A)\; (\forall e)\; [h(e)\neq \varphi_e(e)]$.
  \end{enumerate}

  \begin{proof}
    (i) implies (ii) trivially with witness $f$ since
    $\varphi_e=\varphi_{f(e)}$ implies $W_e=W_{f(e)}$. \\

    (ii) $\Rightarrow$ (iii): Let $d(e)$ be the total recursive function
    such that $\varphi_{d(e)} =\varphi_{\varphi_e(e)}$. Note that $d$
    exists from SMN theorem. Consider the function $h=g\circ d$. Then for
    all $e\in\omega$,
    \[\varphi_{h(e)} =\varphi_{g(d(e))} \neq\varphi_{d(e)}
    =\varphi_{\varphi_e(e)},\]
    therefore $h(e)\neq\varphi_e(e)$. \\

    (iii) $\Rightarrow$ (i): By SMN, there is a total recursive function
    $q$ such that
    \begin{align*}
      \varphi_{q(e)}(y) :=
      \begin{cases}
        \mu x\; [\varphi_e(x)\downarrow] &\text{if}\; W_e\neq\emptyset,\\
        \uparrow &\text{otherwise}.\\
      \end{cases}
    \end{align*}

    Then by relativised SMN, there is a total function $f\leq_Th\leq_TA$
    such that
    \begin{align*}
      \Phi_{f(e)}^A(y) :=
      \begin{cases}
        0 &\text{if}\; y=h(q(e)),\\
        \uparrow &\text{otherwise}.\\
      \end{cases}
    \end{align*}

    Then $W_{f(e)}=\{h(q(e))\}$ for all $e$. Now if $|W_e|\neq 1$, then
    $W_e\neq W_{f(e)}$ since $|W_{f(e)}|=1$. So assume $|W_e|=1$, and write
    $W_e=\{x\}$. Then by definition of $q(e)$, $\varphi_{q(e)}(y) =x$ for
    all $y\in\omega$. In particular, $\varphi_{q(e)}(q(e)) =x$. But
    $\varphi_{q(e)}(q(e))\neq h(q(e))$ by assumption of (iii), therefore
    $h(q(e))\neq x$, which implies that
    \[W_{f(e)}=\{h(q(e))\}\neq \{x\}=W_e.\]
  \end{proof}

\it \textbf{Q2:} Prove that if $G$ is 1-generic, then $G$ is immune,
  hyperimmune, and if $G=A_0\oplus A_1$, then $A_0$ and $A_1$ are pairwise
  Turing incomparable.

  \begin{proof}
    To show $G$ is immune, let $Z\neq\emptyset$ be a c.e. set contained in
    $G$. We show that $|Z|$ must be finite. Consider the set $V\subseteq
    2^{<\omega}$ defined by
    \[V:= \{\sigma: (\exists x\in Z)\; [\sigma(x)=0]\}.\]

    This set is c.e. because $Z$ is c.e.: at any given stage $s$, we can
    enumerate into $V$ all the strings in $2^{<s}$ satisfying $\sigma(x)=0$
    for some $x$ amongst the first $s$ elements enumerated into $Z$. Now
    since $G$ is generic, it must contain an initial segment $\sigma$
    that is either contained in $V$ or whose extensions are not
    contained in $V$. The former cannot be true since $Z\subseteq G$ and
    $\sigma(x)=1$ for all $\sigma\prec G$ and $x\in Z$. So it must be
    that all extensions of some $\sigma\prec G$ are not contained in $V$. In
    this case, if $Z$ is infinite, we will be able to pick an element $x$
    in $Z$ larger than $|\sigma|$, and extend $\sigma$ to an arbitrary
    string $\tau\succ\sigma$ with $\tau(x)=0$. This $\tau$ will be an
    extension of $\sigma$ that lies in $V$, a contradiction. Therefore $Z$
    must be finite. Then since $Z$ is an arbitrary c.e. set, $G$ must be
    immune. \\

    We use a similar approach to show $G$ is hyperimmune. Let $f(n)$ be a
    total computable function such that $\{D_{f(n)}\}_{n\in\omega}$ is a
    disjoint strong array, and assume by contradiction that $G$ intersects
    $D_{f(n)}$ for every $n\in\omega$. Consider the set $W\subseteq
    2^{<\omega}$ defined by
    \[W:= \{\sigma: (\exists n)(\forall x\in D_{f(n)})\;
    [\sigma(x)=0]\}.\]

    This set is c.e. because $f$ is recursive: at any given stage $s$, we
    can enumerate into $W$ all the strings in $2^{<s}$ satisfying
    $\sigma(D_{f(n)})=0$ for some $n\leq s$. Now since $G$ is generic, it
    must contain an initial segment $\sigma$ that is either contained in
    $W$ or whose extensions are not contained in $W$. The former cannot
    be true since $G$ intersects every $D_{f(n)}$. So it must be that all
    extensions of some $\sigma\prec G$ are not contained in $W$. Let
    $n\in\omega$ be large enough such that all elements of $D_{f(n)}$ are
    greater than $|\sigma|$. Such $n$ must exist since the $D_{f(n)}$ are
    pairwise disjoint. Extend $\sigma$ to an arbitrary string
    $\tau\succ\sigma$ with $\tau(x)=0$ for all $x\in D_{f(n)}$. Then $\tau$
    is an extension of $\sigma$ that lies in $W$, a contradiction.
    Therefore $G$ is hyperimmune. \\

    Let $G=A_0\oplus A_1$. By symmetrical argument, it suffices to show
    that $A_0\not\leq_T A_1$. Assume by contradiction that there exists
    $e\in\omega$ such that $A_0=\Phi_e^{A_1}$. Consider the set
    $V_e\subseteq 2^{<\omega}$ defined by
    \[V_e :=\{\sigma: (\exists x)\; [\Phi_e^{\sigma_1}(x) \downarrow\neq
    \sigma_0(x)]\},\]
    where $\sigma_0(x):=\sigma(2x)$ and $\sigma_1(x):=\sigma(2x+1)$. This
    set is c.e. since the sentence defining the elements in it is
    $\Sigma_1$. Now since $G$ is generic, it
    must contain an initial segment $\sigma$ that is either contained in
    $V_e$ or whose extensions are not contained in $V_e$. The former cannot
    be true since $A_0=\Phi_e^{A_1}$ by assumption. So it must be that all
    extensions of some $\sigma\prec G$ are not contained in $V_e$. Let
    $\tau$ be an initial segment of $G$ of length $2n$ that strictly extends
    $\sigma$. Then let $\tau'$ be a long enough extension of $\tau$ that is
    also an initial segment of $G$ such that $\tau'_1$ is longer than the
    use of compute $\Phi_e^{A_1}(2n)$; i.e. $\tau'_1$ is long enough to
    compute $\Phi_e^{A_1}(2n) =\Phi_e^{\tau'}(2n)$. Finally, let $\tau''$
    be the same as $\tau'$ except $\tau''(2n)=1-\tau'(2n)$. Then $\tau''$
    belongs in $V_e$ with $2n$ as witness, yet $\tau''$ is an extension of
    $\sigma$, a contradiction.
  \end{proof}

\it \textbf{Q3:} Show that every partial computable function $f$ has a low
  total function $g$ extending $f$. So if $f(x)$ exists then $f(x)=g(x)$.

  \begin{proof}
    We $\emptyset'$-construct $g(x)$ in stages $s$, such that
    $g(x)=\cup_{s\in\omega} g_s(x)$ for all $x\in\omega$, where $g_s\in
    \omega^{<\omega}$, $g_s\prec g_{s+1}$, and $g_s(x)$ is an extension of
    $f(x)$ when $x<|g_s|$. \\

    To ensure that the resulting $g$ is low, we force the jump by
    making sure that for all $s$, we satisfy the jump requirement
    \[J_s: (\exists \sigma\succ g_s)\; \left\{ \left[
    \Phi_s^{\sigma}(s)\downarrow\; \text{or}\;\; (\forall
    \rho\succ\sigma)\; \Phi_s^{\rho}(s)\uparrow \right]\; \text{and}\;
    \sigma\; \text{is consistent with}\; f\right\}.\]

    At stage 0, set $g_0=\emptyset$. At stage $s+1$, we satisfy $J_s$ by
    asking the $\emptyset'$-question:
    \begin{center}
      \textit{Is there a finite extension $\sigma\in\omega^{<\omega}$ of
      $g_s$ that is consistent with $f$ such that $\Phi^\sigma_s(s)$
      converges?}
    \end{center}

    Observe that this is a $\emptyset'$-question because we can
    effectively enumerate all finite extensions
    $\sigma\in\omega^{<\omega}$ of $g_s$, and for each of these extensions,
    use $\emptyset'$ to check if it is consistent with $f$. If it is
    consistent, we can use $\emptyset'$ again to check if
    $\Phi^\sigma_s(s)$ converges. Formally, this is a $\Sigma_1$-question
    \[(\exists \sigma,t)\; \left\{ g_s\prec\sigma \wedge (\forall
    x<|\sigma|)\; \left[ (\exists u\; f_u(x)\downarrow)
    \rightarrow\sigma(x)=f(x) \right]\; \wedge\;
    \Phi^\sigma_{s,t}(s)\downarrow \right\},\]
    and therefore can be answered by $\emptyset'$. If such $\sigma$ exists,
    set $g_{s+1}=\sigma$. Otherwise, let $g_{s+1}$ be the first strict
    extension of $g_s$ that is consistent with $f$ (extension needs to be
    strict to ensure that $g$ is total). Observe that in the latter case,
    every finite extension $\rho$ of $g_s$ that is consistent with $f$
    gives $\Phi_s^\rho(s)\uparrow$, so the jump requirement $J_s$ is
    satisfied. \\

    Our construction ensures that $g$ is total and extends $f$. It remains
    to verify that $J_s$ ensures that $g$ is low. We show that
    $g'\leq_T\emptyset'$. Now $g\leq_T\emptyset'$ since the construction
    asked only $\emptyset'$ questions. To decide if $e\in g'$, or
    equivalently, if $\Phi_e^g(e)$ converges, we know that at stage $e$,
    $g_{e+1}\prec g$ satisfies the jump requirement $J_e$. Thus from
    construction, \[e\in g'\; \Leftrightarrow\; \Phi_e^g(e)\downarrow\;
    \Leftrightarrow\; \Phi_e^{g_{e+1}}(e)\downarrow,\] and $\emptyset'$ can
    check if $\Phi_e^{g_{e+1}}(e)\downarrow$ holds, since $g_{e+1}\leq_T
    \emptyset'$. Therefore $g'\leq_T\emptyset'$ as required.
  \end{proof}
\end{document}
