\documentclass{article}
\usepackage[left=3cm,right=3cm,top=3cm,bottom=3cm]{geometry}
\usepackage{amsmath,amssymb,amsthm,tikz,mathtools}
\usepackage{stmaryrd} % For double square bracket [[]]
\usepackage{color}
\usepackage[inline]{enumitem}
\usetikzlibrary{patterns}
\setlength{\parindent}{0mm}
\newcommand{\TODO}[1]{\textcolor{red}{TODO: #1}}

\begin{document}
\title{Basic Logic II: Homework 6}
\author{Li Ling Ko\\ lko@nd.edu}
\date{\today}
\maketitle

\it \textbf{Q1:} Soare 3.4.7:
  \begin{enumerate}[label={\bf (\alph*):}]
    \item Let $\{A_y\}_{y\in\omega}$ be any countable sequence of sets.
      Define the infinite join $\oplus_y A_y$ as in (2.29). Prove that
      $\text{deg}(\oplus_y A_y)$ is the uniform least upper bound for
      $\{\text{deg}(\oplus_y A_y)\}_{y\in\omega}$ in the sense that if
      there exist a set $C$ and a computable function $f$ such that
      $A_y=\Phi_{f(y)}^C$ for all $y$, then $\oplus_y A_y\leq_T C$.

      \begin{proof}
        Let $C$ be a set and $f$ a computable function such that
        $A_y=\Phi_{f(y)}^C$ for all $y$. Then to decide if $\langle
        x,y\rangle$ lies in $\oplus_y A_y$, we just check if
        $\Phi_{f(y)}^C(x)=1$. If it is then $\langle x,y\rangle$ lies in
        $\oplus_y A_y$, otherwise $\Phi_{f(y)}^C(x)=0$ and then $\langle
        x,y\rangle$ does not lie in $\oplus_y A_y$. Therefore $\oplus_y
        A_y\leq_T C$.
      \end{proof}
  \end{enumerate}
\end{document}
