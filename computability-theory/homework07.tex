\documentclass{article}
\usepackage[left=3cm,right=3cm,top=3cm,bottom=3cm]{geometry}
\usepackage{amsmath,amssymb,amsthm,tikz,mathtools}
\usepackage{stmaryrd} % For double square bracket [[]]
\usepackage{bm} % For bold vectors
\usepackage{color}
\usepackage[inline]{enumitem}
\usetikzlibrary{patterns}
\setlength{\parindent}{0mm}
\newcommand{\TODO}[1]{\textcolor{red}{TODO: #1}}

\begin{document}
\title{Basic Logic II: Homework 7}
\author{Li Ling Ko\\ lko@nd.edu}
\date{\today}
\maketitle

\begin{enumerate}
  \item \it A set $A$ is REA iff $A=W^B\oplus B$ for some c.e. set $B$.
    Show that every non-computable REA set has a hyperimmune (Turing
    degree). It is enough to show that there is a modulus $g\leq_TA$ such
    that every $h>g$ computes $A$. Show that every REA set is not a
    hyperimmune set.

    \begin{proof}
      Given $A$, we can compute $W^B$ and $B$. From these two subsets of
      $\omega$, we can compute the modulus $g\leq_T A$ of $A$ as follows.
      Given arbitrary $x\in\omega$, let $g(x)$ be the smallest integer
      $s\in\omega$ such that $B_s\restriction s$ can compute $A\restriction
      x$. That is, $s$ is the least integer such that
      \[B_s\restriction \lceil x/2\rceil =B\restriction \lceil x/2\rceil,\]
      and
      \[W^{B_s\restriction s}_{s} \restriction \lfloor x/2\rfloor
      =W^B \restriction \lfloor x/2\rfloor.\]

      Then $g$ can compute $B$ and $W^B$, and therefore also $A$. So
      $g\equiv_T A$.  Now by construction, any function that dominates $g$
      will also compute $A$. Therefore $g$ cannot be computably dominated
      otherwise the computable function that dominates $g$ will also
      compute $A$, making $A$ computable. Thus $g$ is hyperimmune. \\

      However, $A$ is not hyperimmune, because its principal function can
      be computably dominated as follows. If $B$ is infinite, let $f(x)$ be
      a total-recursive and strictly increasing function that enumerates a
      subset of $B$; such a function can be defined recursively by letting
      $f(0)$ be the first element of $B$ in the enumeration of $B$, and
      letting $f(x+1)$ be the first element of $B$ that is greater than
      $f(0),\ldots,f(x)$, using the same enumeration. Then the function
      $2f(x)+1$ will computably dominate $A$. \\

      On the other hand, if $B$ is finite, $W^B$ will be c.e., so
      $A=W^B\oplus B$ will also be c.e., and we can use a similar argument
      to computably dominate $A$ through constructing a computable strictly
      increasing function that enumerates a subset of $A$. 
    \end{proof}

  \item \it Call $f$ diagonal non-recursive (DNR) iff for all $e$,
    $f(e)\neq\varphi_e(e)$. Show there is a $\Pi_1^0$ class of DNR
    functions in Cantor space. Show there is a low DNR function.

    \begin{proof}
      Define the set of strings $T\subset2^{<\omega}$ such that $\sigma$
      lies in $T$ if and only if
      \[(\forall x<|\sigma|)\; [\varphi_{x,|\sigma|}(x)\downarrow
      \Rightarrow \sigma(x) \neq\varphi_{x,|\sigma|}(x)].\]

      Since this is a $\Delta_0$-formula, and $T$ is closed under initial
      segments, $[T]$ will be a $\Pi_1^0$ class. Also, by definition, the
      paths in $[T]$ are DNR functions. \\

      We show that $[T]$ is non-empty by using $\emptyset'$ to construct a
      function $f$ in $[T]$. Given arbitrary $x\in\omega$, ask $\emptyset'$
      if $\varphi_x(x)$ converges; if it does, set $f(x)$ to the first
      value in $\{0,1\}$ different from $\varphi_x(x)$, and if it
      doesn't, set $f(x)$ to be 0. Then $f$ will be a DNR function in
      $[T]$. \\

      Since $[T]$ is a non-empty $\Pi_1^0$ class, by the low-basis theorem,
      $[T]$ contains a low function $g$. Then $g$ would be a low DNR
      function.
    \end{proof}

  \item \it Show there is a low 1-random real. (Think trees)
    \begin{proof}
      The class of paths outside a fixed universal Martin-Lof test is a
      non-empty $\Pi_1^0$ class of 1-randoms (Corollary 11.2.2). Therefore
      by the low-basis theorem, this class contains a low element, which
      would be a low 1-random real.
    \end{proof}

  \item \it Show there is a hyperimmune-free degree. Hint: you have to
    prove a basis theorem like the low basis theorem. For each $\Phi^X_e$
    first try to force divergence. If this cannot happen show $\Phi_e^X$ is
    computably dominated for all $X\in[T_e]$.

    \begin{proof}
      We construct a series of trees $2^{<\omega}=T_0\supseteq T_1\supseteq
      T_2\supseteq\ldots$, such that any path $Y\in
      T:=\cap_{e\in\omega}T_e$ has a hyperimmune-free degree. Given $T_e$,
      we try to construct $T_{e+1}$ by finding the paths $Y$ in $T_e$ such
      that $\Phi^Y_e$ is not total. This way $\Phi^Y_e$ cannot be the
      principal function of a hyperimmune set. If no such path can be
      found, we set $T_{e+1}$ to equal $T_e$, which will be computable.
      Then using the computability of $T_{e+1}$ and the fact that
      $\Phi^Y_e$ is total for all paths $Y$ in $T_{e+1}$, we will be able
      to computably dominate all $\Phi^Y_e$. \\

      Formally, given $T_e$, define $T_{e+1}\subseteq T_e$ as follows.
      For each $n\in\omega$, define
      \[U_{e,n} :=\{\sigma\in T_e:\;
      \Phi_{e,|\sigma|}^\sigma(n)\uparrow\}.\]
      Then $U_{e,n}$ is computable and closed under initial segments. Let
      $n\in\omega$ be the first integer such that $U_{e,n}$ is infinite.
      Note that $\emptyset''$ can determine if such $n$ exists, and find
      the first one if it does. If no such $n$ exists, set $T_{e+1}=T_e$;
      then for all paths $Y$ in $T_{e+1}$, $\Phi^Y_e$ will be total.
      If $n$ exists, set $T_{e+1}=U_{e,n}$; then for all paths $f$ in
      $T_{e+1}$, $\Phi^Y_e$ will not converge at input $n$. \\

      Observe that by induction on $e$, each $T_e$ will be computable tree
      since $U_{e,n}$ is computable for each $e,n\in\omega$. Also by
      induction on $e$, each $T_e$ is infinite and computable, therefore by
      Compactness theorem $T=\cap_{e\in\omega}T_e$ will be infinite. Then
      by Konig's lemma $T$ must contain a path. Fix any path $X$ in $T$. We
      show that $X$ has a hyperimmune-free degree. \\

      Fix an arbitrary $e\in\omega$. We want to show that if $\Phi_e^X$ is
      the principal function of some set, then that set cannot be
      hyperimmune. So assuming $\Phi_e^X$ is a principal function,
      we construct a computable function that dominates
      $\Phi_e^X$. As a principal function, $\Phi_e^X$ is total, so
      $U_{e,n}$ must be finite for all $n\in\omega$. Then by our
      construction, $T_{e+1}=T_e$, is computable, and for every path $Y$ in
      $T_{e+1}$, $\Phi^Y_e$ is total. Treating each $\Phi^Y_e$ as a
      principal function of some set, we recursively construct a function
      $h$ that will computably dominate $\Phi^Y_e$ for all paths $Y$ in
      $T_{e+1}$. Then in particular, $h$ will computably dominate
      $\Phi^X_e$. \\

      Set $h(0)$ to be the first value that is larger than $\Phi^Y_e(0)$
      for all paths $Y$ in $T_{e+1}$. Then, set $h(n+1)$ to be the first
      value that is larger than $h(n)$ and $\Phi^Y_e(n)$ for all paths
      $Y$ is $T_{e+1}$. Formally, to set $h(0)$, find the first
      $k\in\omega$ such that all paths $\sigma$ of length $k$ in
      $T_{e+1}$ satisfy $\Phi^\sigma_{e,|\sigma|}(0)\downarrow$. Such $k$
      must exist since $U_{e,0}$ is finite. Then set
      \[h(0) =1+\max_{\sigma\in 2^k\cap T_{e+1}} \Phi^\sigma_e(0).\]
      Similarly, given $h(n)$, find the first $k\in\omega$ such that all
      paths of length $k$ in $T_{e+1}$ satisfy
      $\Phi^\sigma_{e,|\sigma|}(n+1)\downarrow$. Once again such $k$ exists
      since $U_{e,n+1}$ is finite. Then set
      \[h(n+1) =1+ \max \left(h(n), \max_{\sigma\in 2^k\cap T_{e+1}}
      \Phi^\sigma_e(n+1) \right).\]
    \end{proof}

  \item \it Show no 1-random is hyperimmune. (Again, think trees.)
    \begin{proof}
      Fix a universal Martin-Lof test
      $\mathcal{A}_0\supset\mathcal{A}_1\supset\mathcal{A}_2\supset\ldots$,
      where $\mathcal{A}_n\subseteq2^\omega$. Then $\mathcal{A}_n
      =\llbracket A_n\rrbracket$ where $A_0,A_1,\ldots$ is a u.c.e.
      sequence of subsets of $2^{<\omega}$, and
      $\mu(\mathcal{A}_n)\leq2^{-n}$ for all $n\in\omega$. Let
      $f\in2^\omega$ be 1-random real. We construct a disjoint strong array
      $\{D_n\}_{n\in\omega}$ that witnesses the non-hyperimmunity of $f$.
      We construct $D_n$ in stages $n$, defining $D_n$ at stage $n$. Each
      $D_n$ we construct will be an interval, such that
      $D_0=\{0,\ldots,a_0\}$ for some $a_0\in\omega$, and for each
      $n\in\omega$, $D_{n+1}=\{a_n+1,\ldots,a_{n+1}\}$ for some
      $a_{n+1}>a_n$. \\

      At stage 0, let $k\in\omega$ be the first integer such that
      $f\not\in \mathcal{A}_k$. Note that such $k$ must exist since $f$
      passes the Martin-Lof test. Now the zero function $z\in2^\omega$ is
      computable, so it fails the Martin-Lof test and should be contained
      in $\mathcal{A}_k$. Equivalently, there must be an initial segment
      $z\restriction a_0$ of $z$ that is contained in $A_k$, and the length
      $a_0$ of this initial segment must be greater than $k$ since the
      measure $\mu(\mathcal{A}_k)$ of $\mathcal{A}_k$ is smaller than
      $2^{-k}$. We enumerate the strings of $A_k$ until we find the first
      such an initial segment $z\restriction a_0$. Now this segment cannot be
      an initial segment of $f$ since $f\not\in\mathcal{A}_k$, therefore
      $f(x)=1$ for some $x<m$. Thus
      $\{x:f(x)=1\}\cap\{0,\ldots,m-1\}\neq\emptyset$, so set
      $D_0=\{0,\ldots,a_0\}$. \\

      At stage $n+1$, we have $D_n=\{a_{n-1}+1,\ldots,a_n\}$ for some
      $a_{n-1}<a_n\in\omega$. We work within $\mathcal{A}_{a_n+1}$.
      Consider each string $\sigma\in2^{a_n}$ of length $a_n$. For each
      such string $\sigma$, the function $z_\sigma\in2^\omega$ that extends
      $\sigma$ with zeros (i.e. $z_\sigma\restriction|\sigma|=\sigma$ and
      $z_\sigma(x)=0$ if $x\geq|\sigma|$) is computable, hence must have an
      initial segment $\tau_\sigma$ contained in $A_{a_n+1}$. We enumerate
      the strings of $A_{a_n+1}$ until we get all initial segments
      $\tau_\sigma$ for each $\sigma$ in $2^{a_n}$. Now none of these
      strings $\tau_\sigma$ can be an initial segment of $f$ since
      $f\not\in\mathcal{A}_{a_n+1}$. Yet one of these strings
      $\tau_{f\restriction a_n}$ has an initial segment of length $a_n$ is
      the same as $f\restriction a_n$. But since $f$ does not extend
      $\tau_{f\restriction a_n}$, $f(x)$ must equal 1 at some
      $x\in\{a_n,\ldots,|\tau_{f\restriction a_n}|\}$. In particular, 
      $\{x:f(x)=1\}\cap\{a_n+1,\ldots,a_{n+1}\} \neq\emptyset$, where
      $a_{n+1}$ is defined as
      \[a_{n+1} =\max\{|\tau_\sigma|: \sigma\in2^{a_n}\}.\]
      So set $D_{n+1}=\{a_n+1,\ldots,a_{n+1}\}$. \\
    \end{proof}

  \item \it (Solovay) $R$ is 1-random if and only if for every computable
    sequence of $\Sigma^0_1$-classes $\{\mathcal{V}_i\}_{i\in\omega}$ with
    $\sum_i \mu(\mathcal{V}_i)<\infty$, $R$ is in only finitely many of the
    sets $\mathcal{V}_i$.

    \begin{proof}
      The converse holds trivially since a Martin-Lof test is a computable
      sequence of $\Sigma_1^0$-class $\{\mathcal{V}_i\}_{i\in\omega}$ with
      $\sum_i \mu(\mathcal{V}_i) \leq\sum_i 2^{-i} =2 <\infty$. \\

      Let $R$ be 1-random, and $\{\mathcal{V}_i\}_{i\in\omega}$ be a
      computable sequence of $\Sigma_1^0$-class with $\sum_i
      \mu(\mathcal{V}_i)=s<\infty$. We effectively construct a Martin-Lof
      test $\{\mathcal{M}_i\}_{i\in\omega}$ from
      $\{\mathcal{V}_i\}_{i\in\omega}$ such that $R$ passing the Martin-Lof
      test implies it can only be in finitely many $\mathcal{V}_i$. \\

      For each $i\in\omega$, let $\mathcal{M}_i\subseteq2^\omega$ be the
      set of paths that appear in more than $2^is$ of the $\mathcal{V}_j$'s.
      Formally, define
      \[\mathcal{M}_i :=\{f\in2^\omega:\; |\{j\in\omega:\;
      f\in\mathcal{V}_j\}| >2^is\}.\]

      Then $\{\mathcal{M}_i\}_{i\in\omega}$ is also a computable sequence
      of $\Sigma_1^0$-class, because to enumerate the elements in
      $\mathcal{M}_i$, we just need to enumerate the strings in the
      $\mathcal{V}_j$'s and list those that appear in at least $2^is$ of
      the $\mathcal{V}_j$'s. We show that the Lesbegue measure of each
      $\mathcal{M}_i$ cannot exceed $2^{-i}$:
      \begin{align*}
        \mu(\mathcal{M}_i) &\leq \frac{1}{2^is}
          \sum_{j\in\omega} \mu(\mathcal{M}_i\cap \mathcal{V}_j)
          &\text{($\because$ each $f\in\mathcal{M}_i$
          appears in more than $2^is$ $V_j$'s)}\\
        &\leq \frac{1}{2^is} \sum_{j\in\omega} \mu(\mathcal{V}_j)
          &(\because \mathcal{M}_i\cap\mathcal{V}_j \subseteq
          \mathcal{V}_j)\\
        &= \frac{1}{2^i}.\\
      \end{align*}

      Note that in the first inequality, the measure $\mu(\mathcal{M}_i\cap
      \mathcal{V}_j)$ is well-defined since both $\mathcal{M}_i$ and
      $\mathcal{V}_j$ are $\Sigma_1^0$-classes. So the
      $\{\mathcal{M}_i\}_{i\in\omega}$ we have constructed is a Martin-Lof
      test. Now since $R$ is ML-random, it can only be contained in
      finitely many of the $\mathcal{M}_i$'s. Let $k\in\omega$ be the
      smallest integer such that $\mathcal{M}_k$ does not contain $R$. Then
      by definition of $\mathcal{M}_k$, $R$ can only be in at most $2^ks$
      $\mathcal{V}_j$'s.
    \end{proof}

  \item \it There are an infinite number of $\sigma$ such that
    $K(\sigma)>|\sigma|+\log(|\sigma|)$.

    \begin{proof}
      Assume not. Then there exists some $k\in\omega$ such that all strings
      $\sigma$ of length greater $k$ will satisfy
      $K(\sigma)\leq|\sigma|+\log(|\sigma|)$. For each such string
      $\sigma\in 2^{<\omega}$ with $|\sigma|>k$, let
      $\tau_\sigma\in2^{<\omega}$ denote the string that witnesses the
      value of $K(\sigma)$, i.e.
      $|\tau_\sigma|=K(\sigma)$ and $\mathcal{U}(\tau_\sigma)=\sigma$, where
      $\mathcal{U}$ is some fixed prefix-free universal Turing machine.
      Then the set of all $\tau_\sigma$'s is prefix-free, so from
      the Kraft inequality, $\sum_{|\sigma|>k} 2^{-|\tau_\sigma|} \leq 1$.
      However $\tau_\sigma=K(\sigma) \leq |\sigma|+\log(|\sigma|)$, so
      \begin{align*}
        \sum_{|\sigma|>k} 2^{-|\tau_\sigma|} &=\sum_{|\sigma|>n}
          2^{-K(\sigma)} \\
        &=\sum_{\sigma\in 2^{<\omega}} 2^{-K(\sigma)} -\sum_{|\sigma|\leq
          k} 2^{-K(\sigma)} \\
        &\geq \sum_{\sigma\in 2^{<\omega}} 2^{-|\sigma|-\log(|\sigma|)}
          -\sum_{|\sigma|\leq k} 2^{-K(\sigma)}\\
        &=\sum_{\sigma\in 2^{<\omega}} \frac{1}{|\sigma|2^{|\sigma|}}
          -\sum_{|\sigma|\leq k} 2^{-K(\sigma)}\\
        &=\sum_{n\in\omega} \sum_{|\sigma|=n}
          \frac{1}{|\sigma|2^{|\sigma|}} -\sum_{|\sigma|\leq k}
          2^{-K(\sigma)}\\
        &=\sum_{n\in\omega} \sum_{|\sigma|=n}
          \frac{1}{n2^{n}} -\sum_{|\sigma|\leq k} 2^{-K(\sigma)}\\
        &=\sum_{n\in\omega}
          \frac{2^n}{n2^{n}} -\sum_{|\sigma|\leq k} 2^{-K(\sigma)}
          &(\text{since there are $2^n$ strings of length $n$})\\
        &=\sum_{n\in\omega} \frac{1}{n} -\sum_{|\sigma|\leq k}
          2^{-K(\sigma)}.\\
      \end{align*}

      Now the left term is the Harmonic series which diverges to $\infty$,
      while the right term is a finite sum, therefore their difference is
      $\infty$, contradicting $\sum_{|\sigma|>k} 2^{-|\tau_\sigma|} \leq
      1$.
    \end{proof}

  \item \it (Demuth and Kucera) A 1-generic does not compute a DNR
    function.

    \begin{proof}
      Let $A\subset\omega$ be a 1-generic. Fix arbitrary $e\in\omega$. We
      want to show $\Phi^A_e(x)$ is not a DNR function. We can assume that
      $\Phi^A_e(x)$ is total, because otherwise $\Phi^A_e(x)$ will not be
      DNR by definition. Consider the c.e. set of strings
      \[W :=\{\sigma\in2^{<\omega}:\; (\exists x)\;
      \Phi_e^\sigma(x) \downarrow= \varphi_x(x)\}.\]
      More precisely, $W$ is c.e. because membership is defined
      by the $\Sigma_1$-formula
      \[(\exists s,x,y)\; \Phi_{e,s}^\sigma(x) \downarrow=
      \varphi_{x,s}(x)=y.\]

      If an initial segment of $A$ lies in $W$, then $\Phi_A(x)$ will not
      be DNR, and we would be done. So assume that no initial segment of
      $A$ lies in $W$. Then by 1-genericity of $A$, there must be an
      initial segment $\tau\prec A$ of $A$ such that every extension
      $\sigma\succ\tau$ of this segment does not lie in
      $W$. From this we construct a total-recursive DNR function $f$,
      which would contradict the fact that DNR functions cannot be
      recursive (because the index of a given total-recursive function
      would witness that the function is not DNR). \\

      Given arbitrary $x\in\omega$, we effectively
      compute $f(x)$ as follows. Look for the first extension $\sigma$ of
      $\tau$ such that $\Phi^\sigma_e(x)$ converges, and set $f(x)$ to be
      the value that it converges to. Now such a $\sigma$ must exist
      since $A$ extends $\tau$ and $\Phi^A_e$ is total by assumption.
      Furthermore, such $\sigma$ will satisfy
      $\Phi^\sigma_e(x)\neq\varphi_x(x)$ because $\sigma$ does not lie in
      $W$. Since this procedure is effective and always terminates, $f$ is
      a total-recursive DNR function, a contradiction.
    \end{proof}

  \item \it (Scott and Tennenbaaun) A PA degree is not minimal. Hint: use
    Theorem 10.3.3 and the right $\Pi_1^0$ class.

    \begin{proof}
      Let $\bm{d}$ be a PA degree, and fix a $\Pi_1^0$ class of 1-randoms.
      Note that we can define this class as the set of paths that lie
      outside a universal Martin-Lof test; this class exists from Corollary
      11.2.2. Now this class must contain a path $R$ with degree less than
      $\bm{d}$, from Theorem 10.3.3. Write $R=R^+\oplus R^-$. We use
      1-randomness of $R$ to show that $R^+$ and $R^-$ are Turing
      incomparable. Then $R^+$ and $R^-$ will have degrees that lie
      strictly between $\emptyset'$ and $\bm{d}$. \\

      Assume by contradiction that $R^+=\Phi^{R^-}_e$ for some
      $e\in\omega$. We construct a Martin-Lof test
      $\mathcal{M}_0\supseteq\mathcal{M}_1\supseteq\ldots$ that $R$ fails.
      Given an arbitrary string $\sigma\in2^{<\omega}$ (or path
      $f\in2^\omega$), let $\sigma^+,\sigma^-\in2^{<\omega}$
      (or $f^+,f^-\in2^\omega$) denote the strings (or paths) such that
      $\sigma=\sigma^+\oplus\sigma^-$ (or $f=f^+\oplus f^-$).
      For each $i\in\omega$, let $\mathcal{M}_i$ be the $\Sigma_1^0$-class
      of paths with an initial segment that witnesses $f^+=\Phi_e^{f^-}$
      at the first $i$ bits. Formally, define
      \[\mathcal{M}_i :=\{f\in2^\omega:\; (\exists \sigma\prec f)\;
      R_i(\sigma)\},\]
      where $R_i(\sigma)$ is the relation defined as
      \begin{equation}
        R_i(\sigma):= |\sigma|>2i\; \wedge\; \Phi_{e,|\sigma|}^{\sigma^+}
        \restriction i = \sigma^-\restriction i.
        \label{eqn:R}
      \end{equation}

      Then since $R_i(\sigma)$ is a $\Delta_0$-formula in $i$ and $\sigma$,
      $\{\mathcal{M}_i\}_{i\in\omega}$ is a computable sequence of
      $\Sigma_1^0$-classes. We show that the Lesbegue measure of each
      $\mathcal{M}_i$ cannot exceed $2^{-i}$. Observe that for every
      $\sigma$ that satisfies $R_i$, there will be at least $2^{i}-1$
      strings of equal length that do not satisfy $R_i$ based on the second
      subformula in equation~\eqref{eqn:R}; these are the strings $\tau$
      where $\tau^-\restriction i$ differ from $\sigma^-\restriction i$.
      So if $M_i\subset2^{<\omega}$ is a prefix-free set of strings that
      generate $\mathcal{M}_i$, then we can use $M_i$ to
      effectively construct a prefix-free set of strings that generate at
      least part of $2^\omega-\mathcal{M}_i$; this set will have at least
      $2^i-1$ more strings than $M_i$. Therefore
      \[\mu(2^\omega-\mathcal{M}_i) \leq (2^i-1)\mu(\mathcal{M}_i).\]
      Rearranging and using the fact that $1=\mu(2^\omega)
      =\mu(2^\omega-\mathcal{M}_i) +\mu(\mathcal{M}_i)$, we get
      $\mu(\mathcal{M}_i) \leq 2^{-i}$. Thus
      $\{\mathcal{M}_i\}_{i\in\omega}$ is a Martin-Lof test. \\

      Now $R$ clearly lies in all $\mathcal{M}_i$ since all sufficiently
      long initial segments of $R$ satisfy equation~\eqref{eqn:R}. So $R$
      fails the constructed Martin-Lof test, a contradiction.
    \end{proof}
\end{enumerate}
\end{document}
