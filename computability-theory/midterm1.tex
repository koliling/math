\documentclass{article}
\usepackage[left=3cm,right=3cm,top=3cm,bottom=3cm]{geometry}
\usepackage{amsmath,amssymb,amsthm,tikz,mathtools}
\usepackage{color}
\usepackage[inline]{enumitem}
\usetikzlibrary{patterns}
\setlength{\parindent}{0mm}
\newcommand{\TODO}[1]{\textcolor{red}{TODO: #1}}

\begin{document}
\title{Basic Logic II: Midterm I}
\author{Li Ling Ko\\ lko@nd.edu}
\date{\today}
\maketitle

\begin{enumerate}[label={\bf Q\arabic*:}]
  \item \it A tree $T$ is a set of finite strings of 0 and 1 and closed
    under initial segments, $\prec$. So $T\subseteq2^{<\omega}$. If
    $f\in2^\omega$ and, for all $n$, $f\restriction n\in T$, then $f$ is a
    path through $T$ and $f\in[T]$. A path $p$ is isolated in $T$ if and
    only if there is a $\sigma\in2^{<\omega}$ such that $\{p\}=[\tau\in
    T:\sigma\preceq\tau\; \text{or}\; \tau\preceq\sigma]$. Assume that $T$
    is computable. Show that all isolated paths of $T$ are computable. Show
    that if $[T]$ is finite then all paths are isolated.

    \begin{proof}
      Let $p$ be an isolated path of a computable tree $T$ as described in
      the question, with $\sigma\in T$ witnessing the isolation. Let
      $\tau\in2^{<\omega}$. To decide if $\tau$ lies in $p$, we first check
      if $\tau\in T$. Next, we check if $\tau$ is comparable with
      $\sigma$. If either of these two conditions fail, then $\tau$ is not
      in $p$. Otherwise, if $\tau$ is an initial segment of $\sigma$, then
      $\tau$ is definitely in $p$. \\

      But if $\tau$ strictly extends $\sigma$, we need to check for one
      more condition to decide if $\tau$ lies on $p$. We check by induction
      on the length of $\tau$. Assume without loss of generality that
      $|\tau|=n+1$, where $n\geq|\sigma|$. We can also assume that
      $\tau(n)=0$. Then $\tau\in p$ if and only if $\tau\restriction n\in
      p$ and $(\tau\restriction n)^\frown1 \not\in p$, where $^\frown$
      denotes concatenation. The second condition follows because $p$ is
      isolated. We can check for the first condition recursively on the
      length of the $\tau$, where the base case $|\tau|=|\sigma|$ is
      decided by $\tau\in p$ if and only if $\tau=\sigma$. To check for the
      second condition, note that from the isolation of $p$, exactly one of
      the subtrees $T_{(\tau\restriction n)^\frown0}$ or
      $T_{(\tau\restriction n)^\frown1}$ contains a path. Furthermore, from
      Konig's lemma, the subtree that does not contain a path must be
      finite. Thus there exists some large $k\in\omega$ such that either
      $T_{(\tau\restriction n)^\frown0}$ is contained in $(\tau\restriction
      n)^\frown0^\frown 2^{\leq k}$, or $T_{(\tau\restriction n)^\frown1}$
      is contained in $(\tau\restriction n)^\frown1^\frown 2^{\leq k}$, and
      exactly one of these two conditions holds. Note that deciding if
      $T_{(\tau\restriction n)^\frown i}$ is contained in
      $(\tau\restriction n)^\frown i^\frown 2^{\leq k}$ is computable
      because $T$ is computable. Thus if the former is true, then
      $\tau\not\in p$, otherwise $\tau\in p$. \\

      Observe that all checks we performed to determine if $\tau\in p$ are
      algorithmic in nature: checking if $\tau\in T$ is algorithmic since
      $T$ is computable; checking if $\tau$ is an initial segment or an
      extension of $\sigma$ is also algorithmic since $\sigma$ and $\tau$
      are finite in length; checking if $p$ extends $\tau$ as described in
      the previous paragraph is also recursive in nature and thus
      algorithmic. Thus by Church-Turing thesis, $p$ is computable. \\

      Assume $T$ contains a path $p$ that is not isolated. Then we can find
      countably infinite distinct paths $\{q_n:n\in\omega\}$ in $T$
      different from $p$ as follows: Staring at $n=0$, since the empty
      string does not isolate $p$, there must be a path $q_0\in T$ that is
      distinct from $p$. For $n+1$, let $\sigma_n\in2^{<\omega}$ denote the
      longest initial segment shared by $p$ and $q_n$. Note that $\sigma_n$
      exists since $p$ and $q_n$ are distinct. Also from construction,
      $|\sigma_n|$ is strictly increasing in $n$. Then let $q_{n+1}$ be a
      path in $T$ that extends $p\restriction(|\sigma_n|+1)$ and that is
      distinct from $p$. Such a $q_{n+1}$ must exist since $\sigma_n$ does
      not isolate $p$. Also, note $q_{n+1}$ will be distinct from
      $\{q_0,\ldots,q_n\}$ from construction. Thus $[T]$ is infinite.
    \end{proof}

  \item \it Work in the language of $Q$, Show that there is a computable
    tree $T\subseteq2^{<\omega}$ such that
    \begin{itemize}
      \item If $f\in[T]$, then
        \[\Gamma_f =\{\varphi_i| f(i)=1\} \cup\{\neg\varphi_i| f(i)=0\}\]
        is a complete consistent extension of PA.

      \item If $\Gamma$ is a complete consistent extension of PA then there
        is an $f\in[T]$ such that $\Gamma=\Gamma_f$.
    \end{itemize}

    Show $T$ has no computable paths. Show that $T$ has no isolated paths.

    \begin{proof}
      Given $\sigma\in2^{<\omega}$, denote
      \[\Gamma_\sigma:= \{\varphi_i| \sigma(i)=1\} \cup\{\neg\varphi_i|
      \sigma(i)=0\}.\]

      Also, given a finite set of formulas $\Gamma=\{\phi_0,\ldots,\phi_k\}$,
      let $\text{En}_\Gamma:\omega\rightarrow\text{Th}(\Gamma)$ be a recursive
      enumeration of the theorems of $\Gamma\cup\text{PA}$. By coding
      $\Gamma$ recursively, $\text{En}_\Gamma(k)$ function is recursive in
      both $\Gamma$ and $k$. \\

      To construct $T$, put $\sigma$ into $T$ if and only if
      $\sigma\in\text{PA}$, or
      \begin{align*}
        \text{``0=1''}\notin &\{\text{En}_{\Gamma_{\sigma\restriction1}}(0),
          \ldots,\text{En}_{\Gamma_{\sigma\restriction1}}(|\sigma|)\}\\
        \cup &\{\text{En}_{\Gamma_{\sigma\restriction2}}(0),
          \ldots,\text{En}_{\Gamma_{\sigma\restriction2}}(|\sigma|)\}\\
        \cup &\ldots\\
        \cup &\{\text{En}_{\Gamma_\sigma}(0),
          \ldots,\text{En}_{\Gamma_\sigma}(|\sigma|)\}.\\
      \end{align*}

      Note that from construction, $T$ is closed under initial segment, and
      is therefore a tree. Also, if $\Gamma_f$ is a complete and consistent
      extension of PA, $f$ will be contained in $[T]$.  Conversely, if
      $\Gamma_f$ is inconsistent, then by the compactness theorem, there
      must be some initial segment $\sigma\prec f$ and $n\in\omega$ such
      that $\text{En}_\sigma(n)$ is the contradiction ``0=1''. So from
      construction, all large enough extensions of $\sigma$ will not be
      placed in $T$, and thus $\Gamma_f$ will not be contained in $[T]$.
      Thus $[T]$ contains exactly all complete and consistent extensions of
      PA. \\

      The fact that $T$ has no computable paths follows directly from the
      fact that that all paths of $T$ are complete and consistent
      extensions of PA, and also from Godel's incompleteness theorem, which
      says all computable extensions of PA are incomplete. \\

      The fact that $T$ has no isolated paths follows directly from the
      fact that $T$ is computable, $T$ has no computable paths, and that
      all isolated paths in computable trees are computable, as we have
      shown in question 1.
    \end{proof}

  \item \it Fix the above tree $T$. Show for every infinite computable
    subtree $\tilde{T}\subseteq T$ we can effectively find an computable
    consistent theory $H_{\tilde{T}}$ extending PA such that if
    $f\in[\tilde{T}]$ then $\Gamma_f$ is a complete consistent extension of
    $H_{\tilde{T}}$ and if $\Gamma$ is a complete consistent extension of
    $H_{\tilde{T}}$ then there is $f\in[\tilde{T}]$ such that
    $\Gamma=\Gamma_f$.

    \begin{proof}
      Let $H_{\tilde{T}}$ be the theory of PA plus $\Omega_{\tilde{T}}$,
      the negation of the initial segments that are not in $\tilde{T}$. \\
      %plus $\Phi_{\tilde{T}}$, the sentences that all segments that
      %$\tilde{T}$ agree with

      More precisely, let
      \[H_{\tilde{T}} :=\text{PA} \cup \{\neg(\theta_{\sigma,0}
      \wedge\ldots \wedge\theta_{\sigma,|\sigma|-1}): \sigma\in
      2^{<\omega}-\tilde{T}\},\]
      where
      \begin{equation*}
        \theta_{\sigma,i} :=
        \begin{cases}
          \varphi_i, &\text{if}\; \sigma(i)=1,\\
          \neg\varphi_i, &\text{otherwise}.
        \end{cases}
      \end{equation*}
      %\[\Phi_{\tilde{T}} :=\{\varphi_i:i\in\omega\; \wedge\; (\sigma(i)=1\;
      %\forall \sigma\in \tilde{T}_{i})\},\]
      %and

      Note that $H_{\tilde{T}}$ is computable because $\tilde{T}$ is.
      Clearly, all paths in $\tilde{T}$ will satisfy $H_{\tilde{T}}$ by
      definition of $H_{\tilde{T}}$. For the reverse inclusion, let $p$ be
      a path in $T$ but not in $\tilde{T}$. Then there must be some initial
      segment $\sigma\prec p$ that is not in $\tilde{T}$. Then the theorem
      associated with $p$ will satisfy the sentence \[\theta_{\sigma,0}
      \wedge\ldots \wedge\theta_{\sigma,|\sigma|-1},\] and therefore cannot
      satisfy the negation of this sentence which is contained in
      $H_{\tilde{T}}$.
    \end{proof}

  \item \it Soare 3.7.3:

  \item \it Use Problem 3 and Rosser's Theorem to show there is a
    computable tree $T$ with the same two properties in Problem 2 and there
    is an infinite computable set $M$ such that for all $m\in M$, for all
    $\sigma\in T^{\text{ext}}\cap2^m$, $\sigma0,\sigma1\in T^{\text{ext}}$.
    (Hint: rearrange the $\varphi_i$.)

    \begin{proof}
      Rosser's trick uses the diagonal lemma to effectively construct a
      sentence that is independent of a given computable extension $T$ of
      PA. We first generalize Rosser's trick to effectively construct a
      sentence that is independent of a given finite set of computable
      extensions $T_1,\ldots,T_n$ of PA. The original Rosser's sentence
      says:
      \begin{center}
        ``\textit{For any proof of me from $T$, there is a shorter proof of my
        negation from $T$.}''
      \end{center}

      To generalize Rosser's trick, we change Rosser's sentence to
      $\varphi$ that says:
      \begin{center}
        ``\textit{For any proof of me from any consistent theory in
        $T_1,\ldots,T_n$, there is a shorter proof of my
        negation from one of those consistent theories.}''
      \end{center}

      We construct this sentence $\varphi$ effectively using the diagonal
      lemma. We denote formulas similarly to what was used in the notes:
      \begin{align*}
        \text{Prov}_T(p,s) :=&\;\text{``Proof}\; p\; \text{proves
          sentence}\; s\; \text{in theory}\; T\text{''},\\
        \text{Con}_T :=&\;\text{``}T\; \text{is consistent''},\\
        \text{Prov}_{T_i}^R(p,s) :=&\;\text{``}p\; \text{proves}\;
          T_i\vdash s\; \text{and no shorter proof can prove its
          negation}\\
          &\;\text{from consistent theories of}\; T_1,\ldots,T_n
          \text{''},\\ :=&\;\text{Prov}_{T_i}(p,s)\\
          \bigwedge_{j=1}^n&\; \text{Con}_{T_j} \rightarrow [(\forall
          s'\forall p'<p)\; \text{Prov}_{T_j}(p',s') \rightarrow
          (s'\neq\text{neg}(s) \wedge s\neq\text{neg}(s'))], \\
        \text{Pr}(s) :=&\;\exists p\; \bigvee_{i=1}^n
          \text{Prov}_{T_i}^R(p,s),\\
        \varphi :=&\;\text{``Diagonalization of}\; \neg\text{Pr}(s)
          \text{''}.\\
      \end{align*}

      Now we show that this sentence $\varphi$ works, i.e., that it is
      independent of all consistent $T_1,\ldots,T_n$. Assume that one of
      these consistent theories $T_1$ proves $\varphi$. Then there is a
      ``real'' proof coded as some natural number $n_1\in\omega$ that
      proves $T_1\vdash\varphi$. Choose $n_1$ to be the length of the
      shortest such proof. So we have
      \[\exists\; \text{shortest proof of}\; T_1\vdash\varphi\; \text{with
      length}\; n_1\in\omega.\]

      Now from the assumption of $\varphi$, one of the consistent theories
      of $T_1,\ldots,T_n$ also proves $\neg\varphi$ via some proof whose
      code $n_2$ is smaller than $n_1$. The theory that proves
      $\neg\varphi$ cannot be $T_1$ since $T_1$ is consistent and already
      proves $\varphi$. We can assume that it is $T_2$ that proves
      $\neg\varphi$, and choose $n_2$ to be the length of the shortest such
      proof. So we also have
      \[\exists\; \text{shortest proof of}\; T_2\vdash\neg\varphi\;
      \text{with length}\; n_2<n_1\in\omega.\]

      Then from the assumption of $\neg\varphi$, there is a proof of
      $\varphi$ from one of the consistent theories $T_1,\ldots,T_n$, which
      is shorter than all proofs of $\neg\varphi$ from any consistent
      theory in $T_1,\ldots,T_n$. Now we already have
      $T_2\vdash\neg\varphi$ via a proof of length $n_2$, so from the
      assumption of $\neg\varphi$, the proof of $\varphi$ we are seeking
      must have length $n_3<n_2$, and in particular, is a ``real'' proof,
      not just a proof from a non-standard number. Then the theory that
      proves $\varphi$ cannot be $T_1$ from the minimality of $n_1$ and
      from $n_3<n_1$. Also, the theory cannot be $T_2$ from consistency of
      $T_2$. Assume it is $T_3$ that proves $\varphi$, and let $n_3$ be the
      length of shortest proof of $\varphi$ from $T_3$. So we have
      \[\exists\; \text{shortest proof of}\; T_3\vdash\varphi\;
      \text{with length}\; n_3<n_2<n_1\in\omega.\]

      Continuing in this manner, we get a series of proofs of decreasing
      lengths from distinct consistent theories, where the sentence proved
      alternates between $\varphi$ and $\neg\varphi$. Furthermore, the
      length of each proof is the shortest possible from each theory, i.e.
      \[\begin{array}{l}
        \exists\; \text{shortest proof of}\; T_1\vdash\varphi\; \text{with
        length}\; n_1\in\omega\\
        \exists\; \text{shortest proof of}\; T_2\vdash\neg\varphi\;
        \text{with length}\; n_2<n_1\in\omega\\
        \exists\; \text{shortest proof of}\; T_3\vdash\varphi\;
        \text{with length}\; n_3<n_2<n_1\in\omega\\
        \ldots\\
      \end{array}\]

      Since the number of theories considered is finite, we will eventually
      exhaust all possible consistent theories, and be unable to find an
      unused consistent theory to prove either $\varphi$ or $\neg\varphi$.
      Thus $\varphi$ is independent in all consistent $T_1,\ldots,T_n$. \\

      Note that if we begin by assuming that $T_1$ proves $\neg\varphi$
      instead of $\varphi$, we can also use the same alternating series
      argument to get a contradiction. \\

      Fix an enumeration of the sentences $\Phi$. We build the tree $T$ in
      steps, such that $T=\cup_{n\in\omega} T_n$, where $T_n=T\cap2^n$. At
      each step $n$, we choose an appropriate sentence $\varphi_n$ and
      extend the tree $T_{n-1}$ to $T_n$ in the way that was described in
      Question 2. At even steps, the sentence we choose is the first one in
      the enumeration $\Phi$ that has not been used yet. At odd steps
      $2n+1$, the sentence we choose is the one described earlier, where we
      effectively construct a sentence that is independent of all the
      consistent theories of the branches of of length $2n$. Note that the
      theory of each branch is computable since it is the finite
      conjunction of the sentences in of the branch together with PA. So
      setting $M$ to be all the even numbers gives us an infinite
      computable set that satisfies the requirements.
    \end{proof}

  \item \it Prove there is a computable listing of computably enumerable
    sets which contains all the computable sets and the only sets appearing
    on the list are computable.

    \begin{proof}
      We first construct a total recursive function $f$ that enumerates
      indices $e$ where $\text{range}(\varphi_e)$ is increasing. We
      construct $f$ by constructing another function $\varphi_{f(e)}(s)$
      recursively: let $\varphi_{f(e)}(s+1)$ be the first element in
      $\text{range}(\varphi_e)$ that is larger than all of
      $\{\varphi_{f(e)}(0), \ldots,\varphi_{f(e)}(s)\}$. If no such element
      exists, then $\varphi_{f(e)}(s+1)$ will not halt. Note that $f(e)$
      exists from the SMN theorem, and is total recursive from
      construction. \\

      Now since a set is computable if and only if it is the range of some
      strictly increasing partial recursive function, $f$ will enumerate
      only indices whose range is a computable set, and also all computable
      sets will have a partial recursive function whose range enumerates it
      and whose index is enumerated by $f$. Thus by letting $g(e)$ be a
      total recursive function that outputs the index whose domain equals
      the range of $\varphi_e$, the composition $g\circ f$ will give a
      computable listing of all computable sets. \\

      Once again, we can construct $g$ via the SMN theorem, by first
      constructing another function $\varphi_{g(e)}(x)$ that outputs
      0 if $x\in\text{range}(\varphi_e)$ and that does not halt otherwise.
    \end{proof}

  \item \it Soare 2.4.17: Assume the disjoint pair $(A,B)$ of c.e. sets is
    effectively inseparable via $\psi$.
    \begin{enumerate}[label={(\roman*)}]
      \item Use the method of Exercise 2.4.11 to prove that we may assume
        $\psi$ is total and 1-1.

        \begin{proof}
          Since $A$ and $B$ are computably enumerable, we can write $A=W_a$
          and $B=W_b$ for some $a,b\in\omega$. We first make $\psi$ total.
          The idea is to first use the hint of Exercise 2.4.11 to
          ``convert'' both $W_{e_0}$ and $W_{e_1}$ to $\emptyset$ if and
          only if $\psi(\langle e_0,e_1\rangle)$ diverges. Then, we append
          $W_a$ to $W_{e_0}$ and $W_b$ to $W_{e_1}$, so that $W_{e_0}$ is
          converted to $W_a$ and $W_{e_1}$ is converted to $W_b$ if and
          only if $\psi(\langle e_0,e_1\rangle)$ diverges. If $\psi(\langle
          e_0,e_1\rangle)$ converges, $W_{e_0}$ and $W_{e_1}$ will remain
          the same, since $W_a\subseteq W_{e_0}$ and $W_b\subseteq
          W_{e_1}$. \\

          Specifically, by the SMN theorem, there are total recursive
          functions $f_0,f_1$ that convert $W_{e_0}$ and $W_{e_1}$ to the
          empty set if and only if $\psi(\langle e_0,e_1\rangle)$ does not
          halt:
          \begin{align*}
            W_{f_0(\langle e_0,e_1\rangle)} :=
            \begin{cases}
              W_{e_0}, &\text{if}\; \psi(\langle e_0,e_1\rangle)\downarrow,\\
              \emptyset, &\text{otherwise},\\
            \end{cases}
          \end{align*}

          and
          \begin{align*}
            W_{f_1(\langle e_0,e_1\rangle)} :=
            \begin{cases}
              W_{e_1}, &\text{if}\; \psi(\langle e_0,e_1\rangle)\downarrow,\\
              \emptyset, &\text{otherwise}.\\
            \end{cases}
          \end{align*}

          By SMN again, there are total recursive functions $g_0,g_1$ that
          append $W_a$ to $W_{e_0}$ and append $W_b$ to $W_{e_1}$:
          \begin{align*}
            W_{g_0f_0(\langle e_0,e_1\rangle)} :=
            \begin{cases}
              W_{e_0}\cup W_a=W_{e_0}, &\text{if}\; \psi(\langle
                e_0,e_1\rangle)\downarrow,\\
              W_a\cup\emptyset =W_a, &\text{otherwise},\\
            \end{cases}
          \end{align*}

          and
          \begin{align*}
            W_{g_1f_1(\langle e_0,e_1\rangle)} :=
            \begin{cases}
              W_{e_1}\cup W_b=W_{e_1}, &\text{if}\; \psi(\langle
                e_0,e_1\rangle)\downarrow,\\
              W_b\cup\emptyset =W_b, &\text{otherwise}.\\
            \end{cases}
          \end{align*}

          Then
          \[\psi'(\langle e_0,e_1\rangle) :=\psi(\langle g_0f_0(\langle
            e_0,e_1\rangle), g_1f_1(\langle e_0,e_1\rangle)\rangle)\]
          will give us the required total function. \\

          Now we can assume that $\psi$ is total, and we make $\psi'$
          injective. We construct $\psi'$ recursively, starting with
          $\psi'(0)=\psi(0)$. Then to compute $\psi'(n+1)$ where
          $n+1=\langle e_0,e_1\rangle$, if $\psi(n+1)$ is not in
          $\{\psi'(0),\ldots,\psi'(n)\}$, then we set
          $\psi'(n+1)=\psi(n+1)$. Otherwise, we add $\psi(n+1)$ to
          $W_{e_0}$ and ask $\psi'$ for an element that is not
          $W_{e_0}\cup\{\psi'(n+1)\}$ and $W_{e_1}$. If the element
          produced is still in $\{\psi'(0),\ldots,\psi'(n)\}$, then we add
          that element to $W_{e_0}$ and repeat the process. \\

          Observe that if $W_{e_0}$ and $W_{e_1}$ are disjoint sets that
          separate $A$ and $B$, then this procedure will eventually
          terminate from effective inseparability of $(A,B)$: Otherwise
          $W_{e_0}$ union with a finite set is the complement of $W_{e_1}$,
          which will make $(A,B)$ computably separable, a contradiction. On
          the other hand, if $W_{e_0}$ and $W_{e_1}$ do not separate $A$
          and $B$ or are not disjoint, then because
          $\{\psi'(0),\ldots,\psi'(n)\}$ is finite, the procedure described
          will either eventually produce an output that has been produced
          before, or produce an output not in
          $\{\psi'(0),\ldots,\psi'(n)\}$. \\

          The above procedure can be defined formally as follows: Let
          $h(x)$ be a recursive function such that $W_{h(x)}=W_x\cup
          \{\psi(x)\}$. $h$ exists from SMN. To compute $\psi'(n+1)$,
          enumerate
          \[\{\psi(n+1), \psi h(n+1), \psi h^2(n+1),\ldots\}\]
          until either some $y$ not in $\{\psi'(0),\ldots,\psi'(n)\}$ is
          found, or a repetition occurs. Then set $\psi'(n+1)=y$. Note that
          $\psi'$ will be total recursive from the Church-Turing thesis.
        \end{proof}

      \item Prove that the following c.e. sets are effectively inseparable:
        $A=\{x:\varphi_x(x)=0\}$ and $B=\{x:\varphi_x(x)=1\}$. Hint: Define
        $\psi$ by
        \begin{align*}
          \varphi_{\psi(x,y)}(z)=
          \begin{cases}
            1 &\text{if}\; z\in W_x\backslash W_y\\
            0 &\text{if}\; z\in W_y\backslash W_z\\
            \uparrow &\text{otherwise}.\\
          \end{cases}
        \end{align*}

        \begin{proof}
          We show that the $\psi$ given in the hint works. Assume that
          $W_x\supseteq A$ and $W_y\subseteq B$ are disjoint, however
          $\psi(x,y)\in W_x$. Since $W_x$ and $W_y$ are disjoint, we have
          \begin{align*}
            \varphi_{\psi(x,y)}(z)=
            \begin{cases}
              1 &\text{if}\; z\in W_x\\
              0 &\text{if}\; z\in W_y\\
              \uparrow &\text{otherwise}.\\
            \end{cases}
          \end{align*}

          So since $\varphi(x,y)\in W_x$,
          $\varphi_{\psi(x,y)}(\psi(x,y))=1$, which implies that
          $\psi(x,y)\in B$, which contradicts $\psi(x,y)\in W_x$ where
          $B\cap W_x=\emptyset$. By a symmetrical argument,
          $\psi(x,y)\not\in W_y$, thus $\psi(x,y)\in\overline{W_x\cup W_y}$
          as required.
        \end{proof}
    \end{enumerate}

  \item \it Soare 2.4.18: (Smullyan). Assume that $(A_1,A_2)$ is a pair of
    effectively inseparable c.e. sets with productive function $p$. Prove
    that if $(B_1,B_2)$ is a pair of disjoint c.e. cets then
    $(B_1,B_2)\leq_1(A_1,A_2)$, as in Definition 2.4.9. Hint. By the Double
    Recursion Theorem with one parameter Excercise 2.3.14, define 1:1
    computable functions $g(z)$, $h(z)$ such that
    \begin{align*}
      W_{g(z)}=
      \begin{cases}
        A_1\cup\{p(\langle g(z),h(z)\rangle)\} &\text{if}\; z\in B_2\\
        A_1 &\text{otherwise},
      \end{cases}
    \end{align*}

    and
    \begin{align*}
      W_{h(z)}=
      \begin{cases}
        A_2\cup\{p(\langle g(z),h(z)\rangle)\} &\text{if}\; z\in B_1\\
        A_2 &\text{otherwise}.
      \end{cases}
    \end{align*}

    Let $f(z)=p(\langle g(z),h(z)\rangle)$ and show that
    $(B_1,B_2)\leq_1(A_1,A_2)$ via $f$.

    \begin{proof}
      We follow Smullyan's proof to prove the Double Recursion Theorem,
      which states that given any c.e. functions
      $f_1(x,y_1,y_2,z_1,z_2)$ and $f_2(x,y_1,y_2,z_1,z_2)$, there are
      injective and total recursive functions $t_1(y_1,y_2)$ and
      $t_2(y_1,y_2)$ such that
      \begin{align*}
        \varphi_{t_1(y_1,y_2)}(x)
          &=f_1(x,y_1,y_2,t_1(y_1,y_2),t_2(y_1,y_2)),\\
        \varphi_{t_2(y_1,y_2)}(x)
          &=f_2(x,y_1,y_2,t_1(y_1,y_2),t_2(y_1,y_2)).\\
      \end{align*}

      From SMN and the padding theorem, there is an injective and total
      recursive function $t(y,z)$ such that
      \[\varphi_{t(y,z)}(x) =\varphi_y(x,y,z).\]

      By SMN and padding again, let $\phi_1(y_1,y_2)$, $\phi_2(y_1,y_2)$ be
      injective and total recursive functions such that
      \begin{align*}
        \varphi_{\phi_1(y_1,y_2)}(x,y,z) &=f_1(x,y_1,y_2,t(y,z),t(z,y)),\\
        \varphi_{\phi_2(y_1,y_2)}(x,y,z) &=f_2(x,y_1,y_2,t(z,y),t(y,z)).\\
      \end{align*}

      Let
      \begin{align*}
        t_1(y_1,y_2) &:=t(\phi_1(y_1,y_2),\phi_2(y_1,y_2)),\\
        t_2(y_1,y_2) &:=t(\phi_2(y_1,y_2),\phi_1(y_1,y_2)).\\
      \end{align*}

      We show that these $t_1$ and $t_2$ work:
      \begin{align*}
        \varphi_{t_1(y_1,y_2)}(x)
          &=\varphi_{t(\phi_1(y_1,y_2),\phi_2(y_1,y_2))}(x)\\
        &=\varphi_{\phi_1(y_1,y_2)} (x,\phi_1(y_1,y_2),\phi_2(y_1,y_2))\\
        &=f_1(x,y_1,y_2, t(\phi_1(y_1,y_2),\phi_2(y_1,y_2)),
          t(\phi_2(y_1,y_2),\phi_1(y_1,y_2)))\\
        &=f_1(x,y_1,y_2,t_1(y_1,y_2),t_2(y_1,y_2)).\\
      \end{align*}

      By symmetrical argument, we also get
      \[\varphi_{t_2(y_1,y_2)}(x) =f_2(x,y_1,y_2,t_1(y_1,y_2),t_2(y_1,y_2)),\]
      as required. \\

      To get the functions $g(z)$ and $h(z)$ given in the hint, we let
      \begin{align*}
        f_1(x,y_1,y_2,z_1,z_2) &:=\varphi_{r_1(z_1,z_2,\langle
          y_1,y_2\rangle)}(x),\\
        f_2(x,y_1,y_2,z_1,z_2) &:=\varphi_{r_2(z_1,z_2,\langle
          y_1,y_2\rangle)}(x),\\
      \end{align*}

      where
      \begin{align*}
        W_{r_1(z_1,z_2,y)}(x) &:=
        \begin{cases}
          A_1\cup\{p(\langle z_1,z_2\rangle)\}, &\text{if}\; y\in B_2,\\
          A_1, &\text{otherwise},
        \end{cases}\\
      \end{align*}

      and
      \begin{align*}
        W_{r_2(z_1,z_2,y)}(x) &:=
        \begin{cases}
          A_2\cup\{p(\langle z_1,z_2\rangle)\}, &\text{if}\; y\in B_1,\\
          A_2, &\text{otherwise},
        \end{cases}\\
      \end{align*}

      where $r_1$ and $r_2$ exist from the SMN theorem, and apply the
      double recursion theorem to get
      \begin{align*}
        g(z) &:= t_1(\pi_1(z),\pi_2(z))\\
        h(z) &:= t_2(\pi_1(z),\pi_2(z)),\\
      \end{align*}
      where $\pi_i$ are the inverse pairing functions. \\

      Now we show that $(B_1,B_2)\leq_1(A_1,A_2)$ via $f(z)=p(\langle
      g(z),h(z)\rangle)$. $f$ is injective since it is a composition of
      injective functions.\\

      Assume $z\in B_1$, but $f(z)\not\in A_1$. Now
      since $z\not\in B_2$, $W_{g(z)}=A_1$ by definition. And since $z\in
      B_1$, $W_{h(z)}=A_2\cup\{f(z)\}$. Thus we have $A_1\subseteq
      W_{g(z)}$, $A_2\subset W_{h(z)}$, and $W_{g(z)}\cap
      W_{h(z)}=\emptyset$. Thus $\psi(g(z),h(z))=f(z)$ should lie outside
      $A_1\cup A_2$, a contradiction. Hence $f(z)\in A_1$. By symmetrical
      argument, $f(B_2)\subseteq A_2$. \\

      Finally, let $z\in\overline{B_1\cup B_2}$. Then $W_{g(z)}=A_1$ and
      $W_{h(z)}=A_2$, so $\psi(g(z),h(z))=f(z)$ lies outside
      $A_1\cup A_2$, as we are required to show.

      %We first prove the Double Recursion Theorem, which states that given
      %computable functions $f(x,y,z)$ and $g(x,y,z)$, there are injective
      %and computable functions $a(z)$ and $b(z)$ such that
      %\begin{align*}
      %  \varphi_{a(z)} &=\varphi_{f(a(z),b(z),z)},\\
      %  \varphi_{b(z)} &=\varphi_{g(a(z),b(z),z)}.
      %\end{align*}

      %We follow Smullyan's proof to prove the Double Recursion Theorem,
      %which states that given any c.e. functions
      %$f_1(x,y,z_1,z_2)$ and $f_2(x,y,z_1,z_2)$, there are
      %recursive functions $t_1(y)$ and $t_2(y)$ such that
      %\begin{align*}
      %  \varphi_{t_1(y)}(x)
      %    &=f_1(x,y,t_1(y),t_2(y)),\\
      %  \varphi_{t_2(y)}(x)
      %    &=f_2(x,y,t_1(y),t_2(y)).\\
      %\end{align*}

      %From SMN and the pading theorem, there is an injective and total
      %recursive function $t(y,z)$ such that
      %\[\varphi_{t(y,z)}(x) =\varphi_y(x,z).\]

      %By SMN and padding again, let $\phi_1(y)$, $\phi_2(y)$ be
      %injective and total recursive functions such that
      %\begin{align*}
      %  \varphi_{\phi_1(y)}(x,z) &=f_1(x,y,t(y,z),t(z,y)),\\
      %  \varphi_{\phi_2(y)}(x,z) &=f_2(x,y,t(z,y),t(y,z)).\\
      %\end{align*}

      %Let
      %\begin{align*}
      %  t_1(y) &:=t(\phi_1(y),\phi_2(y)),\\
      %  t_2(y) &:=t(\phi_2(y),\phi_1(y)).\\
      %\end{align*}

      %We show that these $t_1$ and $t_2$ work:
      %\begin{align*}
      %  \varphi_{t_1(y)}(x)
      %    &=\varphi_{t(\phi_1(y),\phi_2(y))}(x)\\
      %  &=\varphi_{\phi_1(y)} (x,\phi_2(y))\\
      %  &=f_1(x,y,t(y,\phi_2(y)),t(\phi_2(y),y))\\
      %  &=f_1(x,y_1,y_2,t_1(y_1,y_2),t_2(y_1,y_2)).\\
      %\end{align*}

      %By symmetrical argument, we also get
      %\[\varphi_{t_2(y_1,y_2)}(x) =f_2(x,y_1,y_2,t_1(y_1,y_2),t_2(y_1,y_2)),\]
      %as required.
    \end{proof}
\end{enumerate}
\end{document}
