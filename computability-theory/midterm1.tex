\documentclass{article}
\usepackage[left=3cm,right=3cm,top=3cm,bottom=3cm]{geometry}
\usepackage{amsmath,amssymb,amsthm,tikz,mathtools}
\usepackage{color}
\usepackage[inline]{enumitem}
\usetikzlibrary{patterns}
\setlength{\parindent}{0mm}
\newcommand{\TODO}[1]{\textcolor{red}{TODO: #1}}

\begin{document}
\title{Basic Logic II: Midterm I}
\author{Li Ling Ko\\ lko@nd.edu}
\date{\today}
\maketitle

\begin{enumerate}[label={\bf Q\arabic*:}]
  \item \it A tree $T$ is a set of finite strings of 0 and 1 and closed
    under initial segments, $\prec$. So $T\subseteq2^{<\omega}$. If
    $f\in2^\omega$ and, for all $n$, $f\restriction n\in T$, then $f$ is a
    path through $T$ and $f\in[T]$. A path $p$ is isolated in $T$ if and
    only if there is a $\sigma\in2^{<\omega}$ such that $\{p\}=[\tau\in
    T:\sigma\preceq\tau\; \text{or}\; \tau\preceq\sigma]$. Assume that $T$
    is computable. Show that all isolated paths of $T$ are computable. Show
    that if $[T]$ is finite then all paths are isolated.

    \begin{proof}
      Let $p$ be an isolated path of a computable tree $T$ as described in
      the question, with $\sigma\in T$ witnessing the isolation. Let
      $\tau\in2^{<\omega}$. To decide if $\tau$ lies in $p$, we first check
      if $\tau\in T$. Next, we check if $\tau$ is comparable with
      $\sigma$. If either of these two conditions fail, then $\tau$ is not
      in $p$. Otherwise, if $\tau$ is an initial segment of $\sigma$, then
      $\tau$ is definitely in $p$. \\

      But if $\tau$ strictly extends $\sigma$, we need to check for one
      more condition to decide if $\tau$ lies on $p$. We check by induction
      on the length of $\tau$. Assume without loss of generality that
      $|\tau|=n+1$, where $n\geq|\sigma|$. We can also assume that
      $\tau(n)=0$. Then $\tau\in p$ if and only if $\tau\restriction n\in
      p$ and $(\tau\restriction n)^\frown1 \not\in p$, where $^\frown$
      denotes concatenation. The second condition follows because $p$ is
      isolated. We can check for the first condition recursively on the
      length of the $\tau$, where the base case $|\tau|=|\sigma|$ is
      decided by $\tau\in p$ if and only if $\tau=\sigma$. To check for the
      second condition, note that from the isolation of $p$, exactly one of
      the subtrees $T_{(\tau\restriction n)^\frown0}$ or
      $T_{(\tau\restriction n)^\frown1}$ contains a path. Furthermore, from
      Konig's lemma, the subtree that does not contain a path must be
      finite. Thus there exists some large $k\in\omega$ such that either
      $T_{(\tau\restriction n)^\frown0}$ is contained in $(\tau\restriction
      n)^\frown0^\frown 2^{\leq k}$, or $T_{(\tau\restriction n)^\frown1}$
      is contained in $(\tau\restriction n)^\frown1^\frown 2^{\leq k}$, and
      exactly one of these two conditions holds. Note that deciding if
      $T_{(\tau\restriction n)^\frown i}$ is contained in
      $(\tau\restriction n)^\frown i^\frown 2^{\leq k}$ is computable
      because $T$ is computable. Thus if the former is true, then
      $\tau\not\in p$, otherwise $\tau\in p$. \\

      Observe that all checks we performed to determine if $\tau\in p$ are
      algorithmic in nature: checking if $\tau\in T$ is algorithmic since
      $T$ is computable; checking if $\tau$ is an initial segment or an
      extension of $\sigma$ is also algorithmic since $\sigma$ and $\tau$
      are finite in length; checking if $p$ extends $\tau$ as described in
      the previous paragraph is also recursive in nature and thus
      algorithmic. Thus by Church-Turing thesis, $p$ is computable. \\

      Assume $T$ contains a path $p$ that is not isolated. Then we can find
      countably infinite distinct paths $\{q_n:n\in\omega\}$ in $T$
      different from $p$ as follows: Staring at $n=0$, since the empty
      string does not isolate $p$, there must be a path $q_0\in T$ that is
      distinct from $p$. For $n+1$, let $\sigma_n\in2^{<\omega}$ denote the
      longest initial segment shared by $p$ and $q_n$. Note that $\sigma_n$
      exists since $p$ and $q_n$ are distinct. Also from construction,
      $|\sigma_n|$ is strictly increasing in $n$. Then let $q_{n+1}$ be a
      path in $T$ that extends $p\restriction(|\sigma_n|+1)$ and that is
      distinct from $p$. Such a $q_{n+1}$ must exist since $\sigma_n$ does
      not isolate $p$. Also, note $q_{n+1}$ will be distinct from
      $\{q_0,\ldots,q_n\}$ from construction. Thus $[T]$ is infinite.
    \end{proof}

  \item \it Work in the language of $Q$, Show that there is a computable
    tree $T\subseteq2^{<\omega}$ such that
    \begin{itemize}
      \item If $f\in[T]$, then
        \[\Gamma_f =\{\varphi_i| f(i)=1\} \cup\{\neg\varphi_i| f(i)=0\}\]
        is a complete consistent extension of PA.

      \item If $\Gamma$ is a complete consistent extension of PA then there
        is an $f\in[T]$ such that $\Gamma=\Gamma_f$.
    \end{itemize}

    Show $T$ has no computable paths. Show that $T$ has no isolated paths.

    \begin{proof}
      Given $\sigma\in2^{<\omega}$, denote
      \[\Gamma_\sigma:= \{\varphi_i| \sigma(i)=1\} \cup\{\neg\varphi_i|
      \sigma(i)=0\}.\]

      Also, given a finite set of formulas $\Gamma=\{\phi_0,\ldots,\phi_k\}$,
      let $\text{En}_\Gamma:\omega\rightarrow\text{Th}(\Gamma)$ be a recursive
      enumeration of the theorems of $\Gamma\cup\text{PA}$. By coding
      $\Gamma$ recursively, $\text{En}_\Gamma(k)$ function is recursive in
      both $\Gamma$ and $k$. \\

      To construct $T$, put $\sigma$ into $T$ if and only if
      $\sigma\in\text{PA}$, or
      \begin{align*}
        \text{``0=1''}\notin &\{\text{En}_{\Gamma_{\sigma\restriction1}}(0),
          \ldots,\text{En}_{\Gamma_{\sigma\restriction1}}(|\sigma|)\}\\
        \cup &\{\text{En}_{\Gamma_{\sigma\restriction2}}(0),
          \ldots,\text{En}_{\Gamma_{\sigma\restriction2}}(|\sigma|)\}\\
        \cup &\ldots\\
        \cup &\{\text{En}_{\Gamma_\sigma}(0),
          \ldots,\text{En}_{\Gamma_\sigma}(|\sigma|)\}.\\
      \end{align*}

      Note that from construction, $T$ is closed under initial segment, and
      is therefore a tree. Also, if $\Gamma_f$ is a complete and consistent
      extension of PA, $f$ will be contained in $[T]$.  Conversely, if
      $\Gamma_f$ is inconsistent, then by the compactness theorem, there
      must be some initial segment $\sigma\prec f$ and $n\in\omega$ such
      that $\text{En}_\sigma(n)$ is the contradiction ``0=1''. So from
      construction, all large enough extensions of $\sigma$ will not be
      placed in $T$, and thus $\Gamma_f$ will not be contained in $[T]$.
      Thus $[T]$ contains exactly all complete and consistent extensions of
      PA. \\

      The fact that $T$ has no computable paths follows directly from the
      fact that that all paths of $T$ are complete and consistent
      extensions of PA, and also from Godel's incompleteness theorem, which
      says all computable extensions of PA are incomplete. \\

      The fact that $T$ has no isolated paths follows directly from the
      fact that $T$ is computable, $T$ has no computable paths, and that
      all isolated paths in computable trees are computable, as we have
      shown in question 1.
    \end{proof}

  \item \it Fix the above tree $T$. Show for every infinite computable
    subtree $\tilde{T}\subseteq T$ we can effectively find an computable
    consistent theory $H_{\tilde{T}}$ extending PA such that if
    $f\in[\tilde{T}]$ then $\Gamma_f$ is a complete consistent extension of
    $H_{\tilde{T}}$ and if $\Gamma$ is a complete consistent extension of
    $H_{\tilde{T}}$ then there is $f\in[\tilde{T}]$ such that
    $\Gamma=\Gamma_f$.

    \begin{proof}
      Let $H_{\tilde{T}}$ be the theory of PA plus $\Omega_{\tilde{T}}$,
      the negation of the initial segments that are not in $\tilde{T}$. \\
      %plus $\Phi_{\tilde{T}}$, the sentences that all segments that
      %$\tilde{T}$ agree with

      More precisely, let
      \[H_{\tilde{T}} :=\text{PA} \cup \{\neg(\theta_{\sigma,0}
      \wedge\ldots \wedge\theta_{\sigma,|\sigma|-1}): \sigma\in
      2^{<\omega}-\tilde{T}\},\]
      where
      \begin{equation*}
        \theta_{\sigma,i} :=
        \begin{cases}
          \varphi_i, &\text{if}\; \sigma(i)=1,\\
          \neg\varphi_i, &\text{otherwise}.
        \end{cases}
      \end{equation*}
      %\[\Phi_{\tilde{T}} :=\{\varphi_i:i\in\omega\; \wedge\; (\sigma(i)=1\;
      %\forall \sigma\in \tilde{T}_{i})\},\]
      %and

      Note that $H_{\tilde{T}}$ is computable because $\tilde{T}$ is.
      Clearly, all paths in $\tilde{T}$ will satisfy $H_{\tilde{T}}$ by
      definition of $H_{\tilde{T}}$. For the reverse inclusion, let $p$ be
      a path in $T$ but not in $\tilde{T}$. Then there must be some initial
      segment $\sigma\prec p$ that is not in $\tilde{T}$. Then the theorem
      associated with $p$ will satisfy the sentence \[\theta_{\sigma,0}
      \wedge\ldots \wedge\theta_{\sigma,|\sigma|-1},\] and therefore cannot
      satisfy the negation of this sentence which is contained in
      $H_{\tilde{T}}$.
    \end{proof}

  \item \it Soare 3.7.3:

  \item \it Use Problem 3 and Rosser's Theorem to show there is a
    computable tree $T$ with the same two properties in Problem 2 and there
    is an infinite computable set $M$ such that for all $m\in M$, for all
    $\sigma\in T^{\text{ext}}\cap2^m$, $\sigma0,\sigma1\in T^{\text{ext}}$.
    (Hint: rearrange the $\varphi_i$.)

    \begin{proof}
      Rosser's trick uses the diagonal lemma to effectively construct a
      sentence that is independent of a given computable extension $T$ of
      PA. We first generalize Rosser's trick to effectively construct a
      sentence that is independent of a given finite set of computable
      extensions $T_1,\ldots,T_n$ of PA. The sentence constructed from
      Rosser's trick says:\\
      \begin{center}
        ``For any proof of me from $T$, there is a shorter proof of my
        negation from $T$.''
      \end{center}

      To generalize Rosser's trick, we change the sentence to say\\
      \begin{center}
        ``For any proof of me from any consistent theories in
        $T_1,\ldots,T_n$, there is a shorter proof of my
        negation from one of those consistent theories.''
      \end{center}

      %We first use Rosser's theorem to obtain a computable and infinite set
      %of formulas $\Phi_\text{ind}$ that are independent in the theory
      %$T=\text{PA}$: By Rosser's theorem, by applying the diagonal lemma to
      %the formula $\neg\text{Pr}_{\text{PA}}(x)$, we can obtain a formula
      %$\text{Diag}(\neg\text{Pr}_{\text{PA}})$ that is independent in PA.
      %By the same argument, replacing $\text{Pr}_\text{PA}(x)$ by $n$
      %conjunctions of itself for some $n\in\omega$, the formula
      %$\text{Diag}(\neg(\text{Pr}_{\text{PA}} \wedge\ldots
      %\wedge\text{Pr}_{\text{PA}}))$ is also independent in PA.
    \end{proof}

  \item \it Prove there is a computable listing of computably enumerable
    sets which contains all the computable sets and the only sets appearing
    on the list are computable.

    \begin{proof}
    \end{proof}

  \item \it Soare 2.4.17:

  \item \it Soare 2.4.18:
\end{enumerate}
\end{document}
