\documentclass{article}
\usepackage[left=3cm,right=3cm,top=3cm,bottom=3cm]{geometry}
\usepackage{amsmath,amssymb,amsthm,tikz,mathtools}
\usepackage{color}
\usepackage[inline]{enumitem}
\usetikzlibrary{patterns}
\setlength{\parindent}{0mm}
\newcommand{\TODO}[1]{\textcolor{red}{TODO: #1}}

\begin{document}
\title{Basic Logic II: Midterm I}
\author{Li Ling Ko\\ lko@nd.edu}
\date{\today}
\maketitle

\begin{enumerate}[label={\bf Q\arabic*:}]
  \item \it A tree $T$ is a set of finite strings of 0 and 1 and closed
    under initial segments, $\prec$. So $T\subseteq2^{<\omega}$. If
    $f\in2^\omega$ and, for all $n$, $f\restriction n\in T$, then $f$ is a
    path through $T$ and $f\in[T]$. A path $p$ is isolated in $T$ if and
    only if there is a $\sigma\in2^{<\omega}$ such that $\{p\}=[\tau\in
    T:\sigma\preceq\tau\; \text{or}\; \tau\preceq\sigma]$. Assume that $T$
    is computable. Show that all isolated paths of $T$ are computable. Show
    that if $[T]$ is finite then all paths are isolated.

    \begin{proof}
      Let $p$ be an isolated path of a computable tree $T$ as described in
      the question, with $\sigma\in T$ witnessing the isolation. Let
      $\tau\in2^{<\omega}$. To decide if $\tau$ lies in $p$, we first check
      if $\tau\in T$. Next, we check if $\tau$ is comparable with
      $\sigma$. If either of these two conditions fail, then $\tau$ is not
      in $p$. Otherwise, if $\tau$ is an initial segment of $\sigma$, then
      $\tau$ is definitely in $p$. \\

      But if $\tau$ strictly extends $\sigma$, we need to check for one
      more condition to decide if $\tau$ lies on $p$. We check by induction
      on the length of $\tau$. Assume without loss of generality that
      $|\tau|=n+1$, where $n\geq|\sigma|$. We can also assume that
      $\tau(n)=0$. Then $\tau\in p$ if and only if $\tau\restriction n\in
      p$ and $(\tau\restriction n)^\frown1 \not\in p$, where $^\frown$
      denotes concatenation. The second condition follows because $p$ is
      isolated. We can check for the first condition recursively on the
      length of the $\tau$, where the base case $|\tau|=|\sigma|$ is
      decided by $\tau\in p$ if and only if $\tau=\sigma$. To check for the
      second condition, note that from the isolation of $p$, exactly one of
      the subtrees $T_{(\tau\restriction n)^\frown0}$ or
      $T_{(\tau\restriction n)^\frown1}$ contains a path. Furthermore, from
      Konig's lemma, the subtree that does not contain a path must be
      finite. Thus, letting $T_k$ denote the complete binary tree of length
      $k\in\omega$, there exists some large $k\in\omega$ such that either
      of $T_{(\tau\restriction n)^\frown0}$ is contained in
      $\tau\restriction n^\frown0^\frown T_k$, or $T_{(\tau\restriction
      n)^\frown1}$ is contained in $\tau\restriction n^\frown1^\frown T_k$.
      Note that deciding if $T_{(\tau\restriction n)^\frown i}$ is
      contained in $\tau\restriction n^\frown i^\frown T_k$ is computable
      because $T$ is computable. Thus if $T_{(\tau\restriction n)^\frown1}$
      is contained in $\tau\restriction n^\frown1^\frown T_k$ for some
      large enough $k\in\omega$, then we know $\tau\in p$. On the other
      hand, if $T_{(\tau\restriction n)^\frown0}$ is contained in
      $\tau\restriction n^\frown0^\frown T_k$ for some large enough
      $k\in\omega$, then $\tau\not\in p$. \\

      Observe that all checks we performed to determine if $\tau\in p$ are
      algorithmic in nature: checking if $\tau\in T$ is algorithmic since
      $T$ is computable; checking if $\tau$ is an initial segment or an
      extension of $\sigma$ is also algorithmic since $\sigma$ and $\tau$
      are finite in length; checking if $p$ extends $\tau$ as described in
      the previous paragraph is also recursive in nature and thus
      algorithmic. Thus by Church-Turing thesis, $p$ is computable. \\

      Assume $T$ contains a path $p$ that is not isolated. Then we can find
      countably infinite distinct paths $\{q_n:n\in\omega\}$ in $T$
      different from $p$ as follows: Staring at $n=0$, since the empty
      string does not isolate $p$, there must be a path $q_0\in T$ that is
      distinct from $p$. For $n+1$, let $\sigma_n\in2^{<\omega}$ denote the
      longest initial segment shared by $p$ and $q_n$. Note that $\sigma_n$
      exists since $p$ and $q_n$ are distinct. Also from construction,
      $|\sigma_n|$ is strictly increasing in $n$. Then let $q_{n+1}$ be a
      path in $T$ that extends $p\restriction(|\sigma_n|+1)$ and that is
      distinct from $p$. Such a $q_{n+1}$ must exist since $\sigma_n$ does
      not isolate $p$. Also, note $q_{n+1}$ will be distinct from
      $\{q_0,\ldots,q_n\}$ from construction. Thus $[T]$ is infinite.
    \end{proof}

  \item \it Work in the language of $Q$, Show that there is a computable
    tree $T\subseteq2^{<\omega}$ such that
    \begin{itemize}
      \item If $f\in[T]$, then
        \[\Gamma_f =\{\varphi_i| f(i)=1\} \cup\{\neg\varphi_i| f(i)=0\}\]
        is a complete consistent extension of PA.

      \item If $\Gamma$ is a complete consistent extension of PA then there
        is an $f\in[T]$ such that $\Gamma=\Gamma_f$.
    \end{itemize}

    Show $T$ has no computable paths. Show that $T$ has no isolated paths.

    \begin{proof}
      Given $\sigma\in2^{<\omega}$, denote
      \[\Gamma_\sigma:= \{\varphi_i| \sigma(i)=1\} \cup\{\neg\varphi_i|
      \sigma(i)=0\}.\]

      We put $\sigma$ into $T$ if and only if in the enumeration
      $\text{Th}(\Gamma_\sigma)$ of all the theorems of $\Gamma_\sigma$,
      none of the first $|\sigma|$ theorems are a contradiction. Then if
      $\Gamma_f$ is a complete and consistent extension of PA, $f$ will be
      contained in $[T]$ by construction of $T$. Conversely, if $\Gamma_f$
      is inconsistent, then by the compactness theorem, there must be some
      initial segment $\sigma\prec f$ and $n\in\omega$ such that
      $\text{Th}(\Gamma_\sigma)$ we derive a contradiction within the first
      $n$ theorems that are derived. So from construction, $\sigma$ and all
      extensions of $\sigma$ will not be placed in $T$. Thus $[T]$ contains
      exactly all complete and consistent extensions of PA.
    \end{proof}
\end{enumerate}
\end{document}
