\documentclass{article}
\usepackage[left=3cm,right=3cm,top=3cm,bottom=3cm]{geometry}
\usepackage{amsmath,amssymb,amsthm}
\usepackage{color}
%\setlength{\parindent}{0mm}

\newcommand{\TODO}[1]{\textcolor{red}{TODO: #1}}

\begin{document}
\title{Descriptive Set Theory Exercises}
\author{Li Ling Ko\\ lko@nd.edu}
\date{\today}
\maketitle

\begin{enumerate}
  \item Exercise 2.2: Prove that the Cantor space is compact without using
    Tychonoff's Theorem. You could use K\"{o}nig's Lemma or Compactness Theorem
    for propositional logic.
    \begin{proof}
      We can prove the statement directly without using Tychonoff's
      Theorem. Let $\mathcal{C}=(\sigma_i)_{i\in I}\subseteq 2^{<\omega}$
      be an infinite open cover of the Cantor space. Assume by
      contradiction that no finite subset of $\mathcal{C}$ covers the
      Cantor space. We construct an infinite path $f\in 2^\omega$
      recursively as follows: Given $f\upharpoonright n$, define $f(n+1)$
      by the child node of $f\upharpoonright n$ that is not finitely
      covered by $\mathcal{C}$.  Such a child must exist since at each
      level $n$, the sub-tree below $f\upharpoonright n$ is not finitely
      covered. Then by construction, $f$ cannot be covered by
      $\mathcal{C}$, a contradiction. \\

      We can also prove the statement using Compactness Theorem. Let
      $\mathcal{C}=(\sigma_i)_{i\in I}\subseteq 2^{<\omega}$ be an infinite
      open cover of the Cantor space. We wish to show that a finite subset
      of $\mathcal{C}$ covers $2^{\omega}$. For each
      $\sigma_i\in\mathcal{C}$, define a propositional sentence $s_i$ that
      says ``I am a path in $2^\omega$ that does not extend $\sigma_i$.''
      Each sentence can be constructed specifically as follows: Given
      $\sigma_i=(a_0,\ldots,a_{n-1})\in\mathcal{C}$, let $s_i$ be
      $w_0\vee\ldots\vee w_{n-1}$, where $w_j$ is $v_j$ if $a_j=0$, and
      $\neg v_j$ otherwise. Here $v_j$ is the variable used to represent
      the value of the path at level $j$. Now $\mathcal{S}=(s_i)_{i\in
      I}$ cannot be satisfied because $\mathcal{C}$ covers the Cantor
      space. Then by Compactness Theorem, a finite subset of sentences
      $\{s_0,\ldots,s_{k-1}\}$ in $\mathcal{S}$ cannot be satisfied. This
      means all paths in $2^\omega$ must extend at least one of the finite
      set $\sigma_0,\ldots,\sigma_{k-1}$, which implies that this finite set
      covers the Cantor space. \\

      Finally, we can prove the statement using K\"{o}nig's Lemma. Let
      $\mathcal{C}=(\sigma_i)_{i\in\omega}\subseteq 2^{<\omega}$ define an open
      cover of the Cantor space, and assume by contradiction that no finite
      subset of set of covers the space. Let $\mathcal{T}$ be a sub-tree
      of $2^\omega$ consisting of exactly the nodes whose children are not
      covered by any finite subset of $\mathcal{C}$. This sub-tree is
      infinite because each level of $2^\omega$ must contain at least one
      node in $\mathcal{T}$ otherwise $\mathcal{C}$ would finitely cover
      the Cantor space. The sub-tree is also connected because for each
      node in $\mathcal{T}$, its parent node must also be in $\mathcal{T}$.
      Hence by K\"{o}nig's Lemma, $\mathcal{T}$ must contain an infinite
      path. From our construction of $\mathcal{T}$, this path cannot be
      covered by any cover in $\mathcal{C}$, a contradiction.
    \end{proof}

  \item Exercise 3.4: Supposing that $X$ is complete with the metric $d'$ ,
    determine whether it is complete with the metric $d$.
    \begin{proof}
      We show that $X$ is also complete with metric $d$. \\

      Assume $X$ is complete with metric $d'$. Let $(x_n)_{n\in\omega}$ be
      a Cauchy sequence with respect to $d$. We first show that this
      sequence is also Cauchy with respect to $d'$. Given $\epsilon>0$, by
      Cauchy property with respect to $d$, there exists $n_0$ such that for
      all $m,n\geq n_0$, $d(x_m,x_n)<\epsilon$. Then
      $d'(x_m,x_n)\leq d(x_m,x_n)<\epsilon$, hence the sequence is also Cauchy
      with respect to $d'$. \\

      By completeness of $d'$, $X$ must contain some $d'$ limit $x$. We
      show that $x$ is also the limit according to $d$. Given $\epsilon>0$,
      take $0<\epsilon'\leq f(\epsilon)$. By completeness of $d'$, there
      must be some $n_0$ such that for all $n\geq n_0$,
      $d'(x_n,x)<\epsilon'$, and then $d(x_n,x)<\epsilon$.
    \end{proof}
\end{enumerate}

\end{document}
