\documentclass{article}
\usepackage[left=3cm,right=3cm,top=3cm,bottom=3cm]{geometry}
\usepackage{amsmath,amssymb,amsthm,tikz,mathtools}
\usepackage{color}
\usepackage[inline]{enumitem}
\usetikzlibrary{patterns}
\setlength{\parindent}{0mm}
\newcommand{\TODO}[1]{\textcolor{red}{TODO: #1}}

\begin{document}
\title{Discrete Mathematics: Finals}
\author{Li Ling Ko\\ lko@nd.edu}
\date{\today}
\maketitle

\begin{enumerate}
  \item \it

  \item \it Let $T$ be a tree with $k$ edges.
    \begin{enumerate}
      \item Let $G$ be a graph with $n$ vertices and more than $(k-1)n$
        edges. Show that $G$ contains $T$ as a subgraph.

        \begin{proof}
          Following some part of the proof of the Theorem 57.3, we can
          assume that each vertex of $G$ has at least $(k-1)$ edges, and
          that $G$ is connected: Let $G$ with $n$ vertices and $m$ edges
          that satisfy $m>(k-1)n$. Construct $G'$ from $G$ by iteratively
          deleting vertices of degree at most $(k-1)$, as long as there are
          such vertices. Suppose the resulting graph has $n'$ vertices and
          $m'$ edges. Since no more than $(k-1)$ edges were deleted at each
          iteration, we have
          \[m' >(k-1)n-(k-1)(n-n') =(k-1)n'.\]
          So $m'>0$, and therefore $n'>0$, and the resulting graph is not
          empty, and satisfies the desired edge-to-vertex ratio. Thus we
          may restrict ourself to considering the subgraph $G'$ of $G$.
          Also, if the graph is not connected, the taking the densest
          component would satisfy the desired edge-to-vertex ratio, and we
          may restrict ourselves to considering that component. \\

          We prove the assertion by induction on $k$. The case of $k=1$ is
          trivial. Let $T$ be a tree with $k>1$ edges, $G$ a graph with $n$
          vertices and more than $(k-1)n$ edges, and where each edge has
          degree at least $(k-1)$. Note that $n\geq k+1$ because $T$ has
          $k+1$ vertices. Now $T$ can be either a path $P_{k+1}$, a star
          $S_{k+1}$, or neither. We consider the three cases separately. \\

          \textit{Case $T=P_{k+1}$ is a path:} The result follows directly
          from the proof of Theorem 57.3, which showed that every graph
          with $n$ vertices and more than $(k-1)n/2$ edges embeds a path
          with $k$ edges. \\

          \textit{Case $T=S_{k+1}$ is a star:} $G$ has $n$ vertices but
          at least $(k-1)n+1$ edges. So by the pigeonhole principle, $G$
          must have a vertex with $k$ or more neighbors. Making this vertex
          the center of the star, we can embed $S_{k+1}$ onto $G$. \\

          \textit{Case $T$ is neither a path nor a star:} Let $v_0v_1\ldots
          v_t$ be the longest path in $T$. Then $t\leq k-1$ since $T$ is
          not a path. Now either $v_0$ or $v_t$ must be a leaf, otherwise
          we can extend the path to a longer one. Assume without loss of
          generality that $v_t$ is a leaf. Then by inductive hypothesis,
          after removing leaf $v_t$ we can embed the resulting tree $T'$
          into $G$, because $T'$ will have $k-1$ edges and
          $m=(k-1)n>(k-2)n$ since $k>1$. For each $i\in\{1,\ldots,t-1\}$,
          let $u_i$ denote the vertex in $G$ that $v_i$ is mapped to.
          Now in $T$, the degree of $v_{t-1}$ cannot exceed $k-1$ because
          $T$ is not the star $S_{k+1}$. Then since $u_{t-1}$ has
          at least $k-1$ neighbors in $G$ by assumption, and there are only
          $k$ vertices in $T'$, and trees cannot have cycles, it must be
          that at least one of the neighbors of $u_{t-1}$ is is not an
          image of any $v_i$ in $T'$. So mapping $v_t$ to this neighbor
          will embed $T$ onto $G$. \\
        \end{proof}

      \item Show that there are infinitely many $n$ for which there is a
        graph $G$ with $n$ vertices and at least $(k-1)n/2$ edges for which
        $G$ does not contain $T$ as a subgraph.

        \begin{proof}
          For each $n\in\mathbb{N}$ such that $k|n$, the graph consisting
          of $n/k$ copies of $K_k$, the complete graph on $k$ vertices,
          does not have $T$ as a subgraph, since there are not enough
          vertices on each $K_k$. The number of edges in such a graph is
          \[m =\binom{k}{2}\cdot\frac{n}{k} =(k-1)n/2.\]
          Since there are infinite $n$ which are multiples of $k$, there
          are infinite such $G$'s.
        \end{proof}
    \end{enumerate}
\end{enumerate}
\end{document}
