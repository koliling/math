\documentclass{article}
\usepackage[left=3cm,right=3cm,top=3cm,bottom=3cm]{geometry}
\usepackage{amsmath,amssymb,amsthm,tikz,mathtools}
\usepackage{color}
\usepackage[inline]{enumitem}
\usetikzlibrary{patterns}
\setlength{\parindent}{0mm}
\newcommand{\TODO}[1]{\textcolor{red}{TODO: #1}}

\begin{document}
\title{Basic Logic II: Problem Set 1}
\author{Li Ling Ko\\ lko@nd.edu}
\date{\today}
\maketitle

\begin{enumerate}[label={\bf Q\arabic*:}]
  \item \it Show the relation $R(x,y,p):=\{(x,y,p)|xy\neq p\}$ is primitive
    recursive.

    \begin{proof}
      We work top-down to get the derivation. First, note that
      \[\chi_R(\bar{x})=\chi_{\neq}(\text{mult}(\bar{x}_1,
        \bar{x}_2), \bar{x}_3),\]
      where $\text{mult}(x_1,x_2)$ denotes the multiplication function,
      $\bar{x}_i$ denotes the $i$th projection function, and
      $\chi_{\neq}(x_1,x_2)$ denotes the graph of the inequality relation.
      Thus by closure under projection and composition, it suffices to show
      that the constituent functions $\text{mult}(x_1,x_2)$ and
      $\chi_{\neq}(x_1,x_2)$ are primitive recursive. \\

      The multiplication function can be derived as:
      \[\begin{array}{lcl}
        \text{mult}(0,x_2) &= &0, \\
        \text{mult}(x_1+1,x_2) &= &\text{mult}(x_1,x_2)+x_2. \\
      \end{array}\]
      Now the zero
      function is primitive recursive by definition. Also, the function
      $\text{mult}(x_1,x_2)+x_2$ is primitive recursive by closure under
      projection and composition, and also by the fact that the addition
      function is primitive recursive (example on page 2 of notes).
      Therefore, $\text{mult}(x_1,x_2)$ is primitive recursive by closure
      under primitive recursion. \\

      It remains to show that the inequality relation $\chi_{\neg}$ is
      primitive recursive. This relation can be expressed as a composition
      of functions:
      \[\begin{array}{lcl}
        \chi_{\neg}(x_1,x_2) &= &\chi_{>0}(\text{absDiff}(x_1,x_2)), \\
      \end{array}\]
      where $\chi_{>0}(x)$ is the relation for non-zero elements, and
      $\text{absDiff}(x_1,x_2):=|x_1-x_2|$ is the function of absolute
      difference. $\chi_{>0}(x)$ can be easily derived as follows:
      \[\begin{array}{lcl}
        g_1(0,x_2) &= &0, \\
        g_1(x_1+1,x_2) &= &1, \\
        \chi_{>0}(x) &= &g_1(x,0). \\
      \end{array}\]

      To derive $\text{absDiff}(x_1,x_2)$, notice that absolute difference
      is the sum
      \[\begin{array}{lcl}
        \text{absDiff}(x_1,x_2) &=
        &\text{diff}(x_1,x_2)+\text{diff}(x_2,x_1), \\
      \end{array}\]
      where $\text{diff}(x_1,x_2):=\max(x_2-x_1,0)$.
      Thus by closure under composition and the fact that the addition
      function is primitive recursive, it suffices to show that
      $\text{diff}(x_1,x_2)$ is primitive recursive.
      We can derive $\text{diff}(x_1,x_2)$ from primitive recursion:
      \[\begin{array}{lcl}
        \text{diff}(0,x_2) &= &x_2, \\
        \text{diff}(x_1+1,x_2) &= &\text{pred}(\text{diff}(x_1,x_2)), \\
      \end{array}\]
      where $\text{pred}(x)$ is the predecessor function. \\
      
      Now $\text{pred}(x)$ can be derived from primitive recursion as
      follows:
      \[\begin{array}{lcl}
        \text{pred}(x) &= &p_1(x,0), \\
        p_1(0,x_2) &= &0, \\
        p_1(x_1+1,x_2) &= &h(x_1,p_1(x_1,x_2)), \\
      \end{array}\]
      where $h(x,y)$ is defined as
      \[h(x,y):=
        \begin{cases}
          y+1, &\text{if}\; x\geq1\\
          0, &\text{otherwise}\\
        \end{cases},
      \]
      and can be derived via primitive recursion as:
      \[\begin{array}{lcl}
        h(0,y) &= &0, \\
        h(x+1,y) &= &y+1. \\
      \end{array}\]
      Since addition is primitive recursive, $h(x,y)$ will also be primitive
      recursive. This completes the derivation
    \end{proof}

  \item \it What is the code of the primitive recursive used in your third
    step of the above derivation?

    \begin{proof}
      We work backwards with the solution of the previous question to
      get the first three steps of the derivation. In our solution, we
      first derived addition function. Thus the first three steps are:
      \begin{enumerate}[label={\arabic*.}]
        \item $f_1(x) = x+1$
        \item $f_2(x) = x$
        \item $f_3(x_1,x_2,x_3) = x_2$
      \end{enumerate}
      The code of the third primitive recursive is therefore
      $p_1^{3+1}p_2^{2+1}=2^4\cdot3^3=1296$.
    \end{proof}

  \item \it Find a ``computable'' function $f$ that dominates all primitive
    recursive functions.

    \begin{proof}
      We use diagonal argument. Letting $p_e$ denote the $e$th primitive
      recursive function, we define $f(n)$ as
      \begin{equation*}
        f(n) := \max(p_1(n),\ldots,p_n(n))+1.
      \end{equation*}
      Then for any given $e$, we will have $f(n)>p_e(n)$ for all $n\geq e$.
    \end{proof}
\end{enumerate}
\end{document}
