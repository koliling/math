\documentclass{article}
\usepackage[left=3cm,right=3cm,top=3cm,bottom=3cm]{geometry}
\usepackage{amsmath,amssymb,amsthm,tikz,mathtools}
\usepackage{color}
\usepackage[inline]{enumitem}
\usetikzlibrary{patterns}
\setlength{\parindent}{0mm}
\newcommand{\TODO}[1]{\textcolor{red}{TODO: #1}}

\begin{document}
\title{Discrete Mathematics: Problem Set 1}
\author{Li Ling Ko\\ lko@nd.edu}
\date{\today}
\maketitle

\begin{enumerate}[label={\bf Q\arabic*:}]
  \item \it Let $G$ be a graph on vertex set $V$. Argue that the following
    are equivalent.
    \begin{enumerate}[label={(\alph*)}]
      \item \it $G$ is connected, but is no longer connected on the
        deletion of any edge.
      \item \it $G$ is acyclic (has no cycles), but is no longer acyclic
        with the addition of any edge.
      \item \it $G$ has $n-1$ edges and is connected.
      \item \it $G$ has $n-1$ edges and is acyclic.
      \item \it $G$ is connected and acyclic.
    \end{enumerate}

    \it The first of these is our definition of a tree, so this exercise
    gives lots of different characterisations of a tree.

    \begin{proof}
      Consider also the condition (f) given by:
      \[\text{(f): There is exactly one path between any two vertices.}\]
      We show that conditions (a) to (f) are equivalent. \\

      (a)$\Rightarrow$(f): If $G$ is connected, then there must be at least
      one path between any pair of vertices. Also, there cannot be more
      than one path, otherwise the graph would have a cycle, and deleting
      an edge of the cycle will ensure that the graph remains connected,
      contradicting minimal connectedness. \\

      (f)$\Rightarrow$(a): If condition (f) holds, then $G$ must be
      connected since there is a path between any two vertices. Assume by
      contradiction that the graph remains connected after deleting an edge
      between vertices $u$ and $v$. Then there are at least two paths from
      $u$ to $v$. \\

      (f)$\Rightarrow$(b): If $G$ is cyclic, then there must be a pair of
      vertices with at least two paths between them, thus condition (f)
      will not hold. Also, when condition (f) holds, then adding any
      non-existent edge will give at least two paths between the involved
      vertices and thus make the graph cyclic. Hence $G$ is also maximally
      acyclic. \\

      (b)$\Rightarrow$(f): If condition (b) holds, then the graph must be
      connected, otherwise adding an edge to connect two distinct
      components will not make the graph cyclic. Thus every pair of
      vertices must be connected by at least one path. Now if a pair of
      vertices has more than one path between them, then the graph must be
      cyclic, so (b) will not hold. \\

      (a)$\Rightarrow$(c): $G$ having $n-1$ edges follows from Theorem 3.1.
      $G$ being connected follows directly from assumption of (a). \\

      (c)$\Rightarrow$(a): Assume (c) holds but the graph $G$ is not
      minimally connected. Then we can remove edges from $G$ until the
      resulting graph $G$' becomes minimally connected. The edges can be
      removed such that the number of vertices remain unchanged; this is
      possible since the graph is not minimally connected. Thus $G'$ will
      be a minimally connected graph with $n$ vertices but with less than
      $n-1$ edges, which contradicts Theorem 3.1. \\

      (a)$\Rightarrow$(d): $G$ having $n-1$ edges follows from Theorem 3.1.
      If $G$ is cyclic, then deleting an edge in a cycle will not change
      the connectedness of $G$, thus $G$ cannot be minimally connected. \\

      (a)$\Rightarrow$(e): If $G$ is cyclic, then deleting an edge in a
      cycle will not change the connectedness of $G$, thus $G$ cannot be
      minimally connected. \\

      (e)$\Rightarrow$(f): Since $G$ is connected, every pair of vertices
      must have at least one path between them. If there is a pair of
      vertices with more than one path between them, then the graph cannot
      be acyclic. \\

      (d)$\Rightarrow$(c): Assume (d) holds but the graph $G$ is not
      connected. Then $G$ is the disjoint union of finitely many connected
      subgraphs, each being acyclic since $G$ is acyclic. Then from (e) and
      the equivalence of (e) with (c), the number of edges of each subgraph
      is one less than its number of vertices. Then $G$ would have $n-k$
      edges, where $k$ is the number of disjoint subgraphs. This
      contradicts $G$ having $n-1$ edges.
    \end{proof}
\end{enumerate}
\end{document}
