\documentclass{article}
\usepackage[left=3cm,right=3cm,top=3cm,bottom=3cm]{geometry}
\usepackage{amsmath,amssymb,amsthm,tikz,mathtools}
\usepackage{color}
\usepackage[inline]{enumitem}
\usetikzlibrary{patterns}
\setlength{\parindent}{0mm}
\newcommand{\TODO}[1]{\textcolor{red}{TODO: #1}}

\begin{document}
\title{Discrete Mathematics: Problem Set 2}
\author{Li Ling Ko\\ lko@nd.edu}
\date{\today}
\maketitle

\begin{enumerate}[label={\bf Q\arabic*:}]
  \item \it \textbf{Section 8 Problem 13:} In how many ways can one choose
    a pair of subsets $S,T\subseteq\{1,\ldots,n\}$, subject to the
    condition that $S$ is a subset of $T$?

    \begin{proof}
      We first choose the larger subset $T$, then consider all subsets $S$
      that can be chosen from the larger subset $T$. There are
      $\binom{n}{k}$ ways to choose $T$ such that $T$ contains $k$
      elements. For each such $T$, there are $2^k$ ways to choose $S$. This
      gives a total of
      \begin{align*}
        \binom{n}{0}2^0+\ldots+\binom{n}{n}2^n
          &=\sum_{k=0}^n\binom{n}{k}2^k\\
          &=(1+2)^n &(\text{by Binomial expansion})\\
          &=3^n\\
      \end{align*}
      ways to choose $S\subseteq T\subseteq[n]$.
    \end{proof}

  \item \it \textbf{Section 8 Problem 14:} Let $n$, $p$, and $q$ be fixed
    positive integers with $p\leq n$ and $q\leq n$. Prove that
    \[\sum_{k=0}^n \binom{n}{k}\binom{n-k}{p-k}\binom{n-p}{q-k}
    =\binom{n}{p}\binom{n}{q}.\]

    \begin{proof}
      Consider the question of how many ways there are to choose a pair of
      subsets $P,Q\subseteq[n]$ such that $|P|=p$ and $|Q|=q$. We can
      choose the subsets independently to get $\binom{n}{p}\binom{n}{q}$
      choices. Alternatively, we first select $k$ elements from $[n]$ to
      represent $P\cap Q$; there are $\binom{n}{k}$ ways to do this for
      $k=0,\ldots n$. From the remaining $n-k$ unchosen elements, we pick
      $p-k$ of them to be in $P\setminus Q$. Finally, from the remaining
      $n-p$ elements, we pick $q-k$ of them to be in $Q\setminus P$. So
      there are $\binom{n}{k}\binom{n-k}{p-k}\binom{n-p}{q-k}$ choices
      if $|P\cap Q|=k$, giving a total of 
      \[\sum_{k=0}^n \binom{n}{k}\binom{n-k}{p-k}\binom{n-p}{q-k}\]
      choices.
    \end{proof}
\end{enumerate}
\end{document}
