\documentclass{article}
\usepackage[left=3cm,right=3cm,top=3cm,bottom=3cm]{geometry}
\usepackage{amsmath,amssymb,amsthm,tikz,mathtools}
\usepackage{color}
\usepackage[inline]{enumitem}
\usetikzlibrary{patterns}
\setlength{\parindent}{0mm}
\newcommand{\TODO}[1]{\textcolor{red}{TODO: #1}}

\begin{document}
\title{Discrete Mathematics: Problem Set 2}
\author{Li Ling Ko\\ lko@nd.edu}
\date{\today}
\maketitle

In the problems in this section that merely state an identity without given
a question, the question is always to exhibit a combinatorial proof of the
identity.

\begin{enumerate}[label={\bf Q\arabic*:}]
  \item \it From the binomial theorem we find that for $n\geq1$,
    \[\binom{n}{0}+\binom{n}{2}+\ldots =\binom{n}{1}+\binom{n}{3}+\ldots.\]
    Give a bijective combinatorial proof of this fact. That is, construct
    sets $\mathcal{S}_1$ and $\mathcal{S}_2$ with
    $|\mathcal{S}_1|=\binom{n}{0}+\binom{n}{2}+\ldots$ and
    $|\mathcal{S}_2|=\binom{n}{1}+\binom{n}{3}+\ldots$, and exhibit a
    bijection between $\mathcal{S}_1$ and $\mathcal{S}_2$.

    \begin{proof}
      Let $\mathcal{S}_1$ be the set of all subsets of $\{1,\ldots,n\}$
      of even cardinality, and $\mathcal{S}_2$ the set of all subsets odd
      cardinality. Consider the map
      $f:\mathcal{S}_1\rightarrow\mathcal{S}_2$ defined by
      $f(A)=A\bigtriangleup\{1\}$, where $\bigtriangleup$ denotes symmetric
      difference. Then $f$ is a bijection, thus the formula holds.
    \end{proof}

  \item \it (Parallel summation identity) For $m,n\geq0$,
    \[\sum_{k=0}^n\binom{m+k}{k} =\binom{n+m+1}{n}.\]

    \begin{proof}
      $\binom{n+m+1}{n}$ is the number of subsets $\mathcal{S}$ of
      $A=\{a_1,\ldots,a_n,b_1,\ldots,b_m,c\}$ of size $n$. For
      $k\in\{0,\ldots,n\}$, denote by $\mathcal{S}_{k}$ the set of subsets
      of $A$ of size $n$ such that all elements in $\{a_1,\ldots,a_k\}$
      appear in the subset but $a_{k+1}$ does not. Then
      $|\mathcal{S}_k|=\binom{m+n-k}{n-k}$. Also, $\mathcal{S}$ is the
      disjoint union of $\mathcal{S}_0,\ldots,\mathcal{S}_n$. Thus the
      parallel summation identity holds.
    \end{proof}

  \item \it Derive the parallel summation identity from the upper summation
    identity, using an early, extremely simple binomial coefficient
    identity.

    \begin{proof}
      \begin{align*}
        \binom{n+m+1}{n} &=\binom{n+m+1}{m+1} &(\text{symmetry identity}) \\
        &=\binom{m}{m}+\binom{m+1}{m}+\ldots+\binom{m+n}{m} &(\text{
          upper summation identity}) \\
        &=\binom{m}{0}+\binom{m+1}{1}+\ldots+\binom{m+n}{n} &(\text{
          symmetry identity}) \\
        &=\sum_{k=0}^n\binom{m+k}{k}.
      \end{align*}
    \end{proof}

  \item \it (Cancellation, or committee-chair identity) For $n\geq k\geq1$,
    \[\binom{n}{k} =\frac{n}{k}\binom{n-1}{k-1}\;\; \text{or}\;\;
    k\binom{n}{k}=n\binom{n-1}{k-1}.\]

    \begin{proof}
      We ask how many ways there are to choose $k$
      committee members from $n$ people, and electing 1 of the $k$ members
      to be the committee chair. On the left hand side, we first choose $k$
      members from $n$ people, and for each of these choices, we elect
      one of the $k$ members to be the committee chair. Thus we get
      $k\binom{n}{k}$ ways to answer the question. On the right hand side,
      we first choose the committee-chair from one of the $n$ people. From
      the remaining $n-1$ people, we pick $k-1$ to be in the committee, but
      none of these $k-1$ committee members are chair. This gives us
      $n\binom{n-1}{k-1}$ ways to answer the question. Equating both
      solutions give us
      \[k\binom{n}{k}=n\binom{n-1}{k-1}.\]
    \end{proof}

  \item \it (Committee-subcommittee identity) For $n\geq k\geq r\geq0$,
    \[\binom{k}{r}\binom{n}{k} = \binom{n}{r}\binom{n-r}{k-r}.\]

    \begin{proof}
      We ask how many ways there are to choose $k$
      committee members from $n$ people, and within these $k$ members, to
      choose $r$ of them to be subcommittee members. On the left hand
      side, we first choose $k$ members from $n$ people, and for each of
      these choices, we choose $r$ of them to be in the subcommittee.
      Thus we get $\binom{k}{r}\binom{n}{k}$ ways to answer the
      question. On the right hand side, we first choose $r$ subcommittee
      members from one of the $n$ people. From the remaining $n-r$ people,
      we pick $k-r$ to be in the committee, but none of these $k-r$
      committee members are in the subcommittee. This gives us
      $\binom{n}{r}\binom{n-1}{k-1}$ ways to answer the question. Equating
      both solutions give us
      \[\binom{k}{r}\binom{n}{k}=\binom{n}{r}\binom{n-1}{k-1}.\]
    \end{proof}

  \item \it (Vandermonde's identity) For $m,n,r\geq0$,
    \[\binom{m+n}{r} =\sum_{k=0}^r\binom{m}{k}\binom{n}{r-k}.\]

    \begin{proof}
      We ask how many ways there are to choose $r$ elements from a set with
      $m+n$ elements. On the left hand side, we know that the number of
      ways is $\binom{m+n}{r}$. On the right hand side, we let the set $S$ of
      $m+n$ elements be the disjoint union of sets $A$ with $m$ elements
      and set $B$ with $n$ elements. Then for case $k\in\{0,\ldots,r\}$, to
      form a subset of $S$ with $r$ elements, we first choose $k$ elements
      from $A$ before choosing $r-k$ elements from $B$. Each of these cases
      give us $\binom{m}{k}\binom{n}{r-k}$ subsets. This gives the total
      number of ways to be $\sum_{k=0}^r\binom{m}{k}\binom{n}{r-k}$.
    \end{proof}

  \item \it (Binomial theorem for falling powers) With
    $(x)_k=x(x-1)\ldots(x-k+1)$,
    \[(x+y)_n =\sum_{k=0}^n\binom{n}{k}(x)_k(y)_{n-k}.\]
    Here $x$ and $y$ are (complex) variables, and $n\geq0$.

    \begin{proof}
      We first prove for the case where $x,y\in\mathbb{N}$ and $x,y\geq n$.
      Then the left hand side computes the number of ways to draw $n$
      objects from a set of size $x+y$, and arrange them in order. This is
      the same as asking how many ordered subsets of size $n$ we can get
      from the disjoint union $S=A\sqcup B$, where $A=\{a_1,\ldots,a_x\}$
      and $B=\{b_1,\ldots,b_y\}$. Let $\mathcal{S}$ denote the set of
      ordered subsets, and let $\mathcal{S}_k$ denote the set of ordered
      subsets where $k$ elements come from set $A$ and $n-k$ elements come
      from set $B$. Then $\mathcal{S}$ is the disjoint union
      $\mathcal{S}=\mathcal{S}_0\sqcup\ldots\sqcup\mathcal{S}_n$. Thus
      \[(x+y)_n =|\mathcal{S}|
      =|\mathcal{S}_0|\sqcup\ldots\sqcup|\mathcal{S}_n|.\]

      Now for $k=0,\ldots,n$, to compute $|\mathcal{S}_k|$, we first
      consider determining the $k$ positions amongst positions $1,\ldots,n$
      that should contain elements from set $A$; there are $\binom{n}{k}$
      such positions. For each such position, we consider how we can choose
      $k$ elements from set $A$ and order them within the chosen positions;
      there are $(x)_k$ such possibilities. Similarly, for each such
      possibility, we consider how we can choose $n-k$ elements from set
      $B$ and order them within the remaining $n-k$ positions; there are
      $(y)_{n-k}$ such possibilities. Thus,
      $|\mathcal{S}_k|=\binom{n}{k}(x)_k(y)_{n-k}$. Summarizing, we have
      \begin{align*}
        (x+y)_n &=|\mathcal{S}| \\
        &=|\mathcal{S}_0|\sqcup\ldots\sqcup|\mathcal{S}_n| \\
        &=\binom{n}{0}(x)_0(y)_n +\ldots +\binom{n}{n}(x)_n(y)_0. \\
        &=\sum_{k=0}^n\binom{n}{k}(x)_k(y)_{n-k}, \\
      \end{align*}
      so the assertion in the question holds for all $x,y\in\mathbb{N}$,
      where $x,y\geq n\in\mathbb{N}$. \\

      Now we generalize our solution to the case where $x,y\in\mathbb{C}$.
      Denote \[f(x,y):=(x+y)_n-\sum_{k=0}^n\binom{n}{k}(x)_k(y)_{n-k}.\]
      Notice that $f(x,y)\in\mathbb{C}[x,y]$, thus we can write
      \[f(x,y)= g_m(y)x^m+\ldots+g_1(y)x+g_0(y),\]
      for some $m\in\mathbb{N}$, and
      $g_m(y),\ldots,g_0(y)\in\mathbb{C}[y]$. Earlier, we proved that
      $f(x,y)=0$ for all $x,y\in\mathbb{N}$ and $x,y\geq n$. Thus, at
      $y=n$, the polynomial
      \[f(x,n)= g_m(n)x^m+\ldots+g_1(n)x+g_0(n) \in\mathbb{C}[x]\]
      has infinite solutions $x=n,n+1,n+1,\ldots$. Thus from Principle 7.6,
      the coefficients $g_m(n),\ldots,g_0(n)$ must be identically 0.
      Repeating the argument at $y=n+1,n+2,\ldots$, we get
      \[g_m(n+i)=\ldots=g_0(n+i)=0\]
      for all $i\in\mathbb{N}$. In other words, for each
      $k\in\{0,\ldots,m\}$, the polynomial $g_k(y)\in\mathbb{C}[y]$ has
      infinite solutions $n,n+1,n+2,\ldots$. Applying Principle 7.6 again,
      the $g_k(y)$ must be identically 0. Thus, $f(x,y)=0$ for all
      $x,y\in\mathbb{C}$, and the assertion in the question holds in the
      general case where $x,y\in\mathbb{C}$.
    \end{proof}
\end{enumerate}
\end{document}
