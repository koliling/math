\documentclass{article}
\usepackage[left=3cm,right=3cm,top=3cm,bottom=3cm]{geometry}
\usepackage{amsmath,amssymb,amsthm,tikz,mathtools}
\usepackage{color}
\usepackage[inline]{enumitem}
\usetikzlibrary{patterns}
\setlength{\parindent}{0mm}
\newcommand{\TODO}[1]{\textcolor{red}{TODO: #1}}

\begin{document}
\title{Discrete Mathematics: Problem Set 2}
\author{Li Ling Ko\\ lko@nd.edu}
\date{\today}
\maketitle

In the problems in this section that merely state an identity without given
a question, the question is always to exhibit a combinatorial proof of the
identity.

\begin{enumerate}[label={\bf Q\arabic*:}]
  \item \it From the binomial theorem we find that for $n\geq1$,
    \[\binom{n}{0}+\binom{n}{2}+\ldots =\binom{n}{1}+\binom{n}{3}+\ldots.\]
    Give a bijective combinatorial proof of this fact. That is, construct
    sets $\mathcal{S}_1$ and $\mathcal{S}_2$ with
    $|\mathcal{S}_1|=\binom{n}{0}+\binom{n}{2}+\ldots$ and
    $|\mathcal{S}_2|=\binom{n}{1}+\binom{n}{3}+\ldots$, and exhibit a
    bijection between $\mathcal{S}_1$ and $\mathcal{S}_2$.

    \begin{proof}
      Let $\mathcal{S}_1$ be the set of all subsets of $\{1,\ldots,n\}$
      of even cardinality, and $\mathcal{S}_2$ the set of all subsets odd
      cardinality. Consider the map
      $f:\mathcal{S}_1\rightarrow\mathcal{S}_2$ defined by
      $f(A)=A\bigtriangleup\{1\}$, where $\bigtriangleup$ denotes symmetric
      difference. Then $f$ is a bijection, thus the formula holds.
    \end{proof}

  \item \it (Parallel summation identity) For $m,n\geq0$,
    \[\sum_{k=0}^n\binom{m+k}{k} =\binom{n+m+1}{n}.\]

    \begin{proof}
      $\binom{n+m+1}{n}$ is the number of subsets $\mathcal{S}$ of
      $A=\{a_1,\ldots,a_n,b_1,\ldots,b_m,c\}$ of size $n$. For
      $k\in\{0,\ldots,n\}$, denote by $\mathcal{S}_{k}$ the set of subsets
      of $A$ of size $n$ such that all elements in $\{a_1,\ldots,a_k\}$
      appear in the subset but $a_{k+1}$ does not. Then
      $|\mathcal{S}_k|=\binom{m+n-k}{n-k}$. Also, $\mathcal{S}$ is the
      disjoint union of $\mathcal{S}_0,\ldots,\mathcal{S}_n$. Thus the
      parallel summation identity holds.
    \end{proof}

  \item \it Derive the parallel summation identity from the upper summation
    identity, using an early, extremely simple binomial coefficient
    identity.

    \begin{proof}
      \begin{align*}
        \binom{n+m+1}{n} &=\binom{n+m+1}{m+1} &(\text{symmetry identity}) \\
        &=\binom{m}{m}+\binom{m+1}{m}+\ldots+\binom{m+n}{m} &(\text{
          upper summation identity}) \\
        &=\binom{m}{0}+\binom{m+1}{1}+\ldots+\binom{m+n}{n} &(\text{
          symmetry identity}) \\
        &=\sum_{k=0}^n\binom{m+k}{k}.
      \end{align*}
    \end{proof}

  \item \it (Cancellation, or committee-chair identity) For $n\geq k\geq1$,
    \[\binom{n}{k} =\frac{n}{k}\binom{n-1}{k-1}\;\; \text{or}\;\;
    k\binom{n}{k}=n\binom{n-1}{k-1}.\]

    \begin{proof}
      We ask the question of how many ways there are to choose $k$
      committee members from $n$ people, and electing 1 of the $k$ members
      to be the committee chair. On the left hand side, we first choose $k$
      members from $n$ people, and for each of these choices, we elect
      one of the $k$ members to be the committee chair. Thus we get
      $k\binom{n}{k}$ ways to answer the question. On the right hand side,
      we first choose the committee-chair from one of the $n$ people. From
      the remaining $n-1$ people, we pick $k-1$ to be in the committee, but
      none of these $k-1$ committee members are chair. This gives us
      $n\binom{n-1}{k-1}$ ways to answer the question. Equating both
      solutions give us
      \[k\binom{n}{k}=n\binom{n-1}{k-1}.\]
    \end{proof}

  \item \it (Committee-subcommittee identity) For $n\geq k\geq r\geq0$,
    \[\binom{k}{r}\binom{n}{k} = \binom{n}{r}\binom{n-r}{k-r}.\]

    \begin{proof}
      We ask the question of how many ways there are to choose $k$
      committee members from $n$ people, and within these $k$ members, to
      choose $r$ of them to be subcommittee members. On the left hand
      side, we first choose $k$ members from $n$ people, and for each of
      these choices, we choose $r$ of them to be in the subcommittee.
      Thus we get $\binom{k}{r}\binom{n}{k}$ ways to answer the
      question. On the right hand side, we first choose $r$ subcommittee
      members from one of the $n$ people. From the remaining $n-r$ people,
      we pick $k-r$ to be in the committee, but none of these $k-r$
      committee members are in the subcommittee. This gives us
      $\binom{n}{r}\binom{n-1}{k-1}$ ways to answer the question. Equating
      both solutions give us
      \[\binom{k}{r}\binom{n}{k}=\binom{n}{r}\binom{n-1}{k-1}.\]
    \end{proof}
\end{enumerate}
\end{document}
