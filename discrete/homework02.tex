\documentclass{article}
\usepackage[left=3cm,right=3cm,top=3cm,bottom=3cm]{geometry}
\usepackage{amsmath,amssymb,amsthm,tikz,mathtools}
\usepackage{color}
\usepackage[inline]{enumitem}
\usetikzlibrary{patterns}
\setlength{\parindent}{0mm}
\newcommand{\TODO}[1]{\textcolor{red}{TODO: #1}}

\begin{document}
\title{Discrete Mathematics: Problem Set 2}
\author{Li Ling Ko\\ lko@nd.edu}
\date{\today}
\maketitle

In the problems in this section that merely state an identity without given
a question, the question is always to exhibit a combinatorial proof of the
identity.

\begin{enumerate}[label={\bf Q\arabic*:}]
  \item \it From the binomial theorem we find that for $n\geq1$,
    \[\binom{n}{0}+\binom{n}{2}+\ldots =\binom{n}{1}+\binom{n}{3}+\ldots.\]
    Give a bijective combinatorial proof of this fact. That is, construct
    sets $\mathcal{S}_1$ and $\mathcal{S}_2$ with
    $|\mathcal{S}_1|=\binom{n}{0}+\binom{n}{2}+\ldots$ and
    $|\mathcal{S}_2|=\binom{n}{1}+\binom{n}{3}+\ldots$, and exhibit a
    bijection between $\mathcal{S}_1$ and $\mathcal{S}_2$.

    \begin{proof}
      Let $\mathcal{S}_1$ be the set of all subsets of $\{1,\ldots,n\}$
      of even cardinality, and $\mathcal{S}_2$ the set of all subsets odd
      cardinality. Consider the map
      $f:\mathcal{S}_1\rightarrow\mathcal{S}_2$ defined by
      $f(A)=A\bigtriangleup\{1\}$, where $\bigtriangleup$ denotes symmetric
      difference. Then $f$ is a bijection, thus the formula holds.
    \end{proof}
\end{enumerate}
\end{document}
