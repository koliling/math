\documentclass{article}
\usepackage[left=3cm,right=3cm,top=3cm,bottom=3cm]{geometry}
\usepackage{amsmath,amssymb,amsthm,tikz,mathtools}
\usepackage{color}
\usepackage[inline]{enumitem}
\usetikzlibrary{patterns}
\setlength{\parindent}{0mm}
\newcommand{\TODO}[1]{\textcolor{red}{TODO: #1}}

\begin{document}
\title{Discrete Mathematics: Problem Set 2}
\author{Li Ling Ko\\ lko@nd.edu}
\date{\today}
\maketitle

\begin{enumerate}[label={\bf Q\arabic*:}]
  \item \it \textbf{Section 8 Problem 13:} In how many ways can one choose
    a pair of subsets $S,T\subseteq\{1,\ldots,n\}$, subject to the
    condition that $S$ is a subset of $T$?

    \begin{proof}
      We first choose the larger subset $T$, then consider all subsets $S$
      that can be chosen from the larger subset $T$. There are
      $\binom{n}{k}$ ways to choose $T$ such that $T$ contains $k$
      elements. For each such $T$, there are $2^k$ ways to choose $S$. This
      gives a total of
      \begin{align*}
        \binom{n}{0}2^0+\ldots+\binom{n}{n}2^n
          &=\sum_{k=0}^n\binom{n}{k}2^k\\
          &=(1+2)^n &(\text{by Binomial expansion})\\
          &=3^n\\
      \end{align*}
      ways to choose $S\subseteq T\subseteq[n]$.
    \end{proof}

  \item \it \textbf{Section 8 Problem 14:} Let $n$, $p$, and $q$ be fixed
    positive integers with $p\leq n$ and $q\leq n$. Prove that
    \[\sum_{k=0}^n \binom{n}{k}\binom{n-k}{p-k}\binom{n-p}{q-k}
    =\binom{n}{p}\binom{n}{q}.\]

    \begin{proof}
      Consider the question of how many ways there are to choose a pair of
      subsets $P,Q\subseteq[n]$ such that $|P|=p$ and $|Q|=q$. We can
      choose the subsets independently to get $\binom{n}{p}\binom{n}{q}$
      choices. Alternatively, we first select $k$ elements from $[n]$ to
      represent $P\cap Q$; there are $\binom{n}{k}$ ways to do this for
      $k=0,\ldots n$. From the remaining $n-k$ unchosen elements, we pick
      $p-k$ of them to be in $P\setminus Q$. Finally, from the remaining
      $n-p$ elements, we pick $q-k$ of them to be in $Q\setminus P$. So
      there are $\binom{n}{k}\binom{n-k}{p-k}\binom{n-p}{q-k}$ choices
      if $|P\cap Q|=k$, giving a total of 
      \[\sum_{k=0}^n \binom{n}{k}\binom{n-k}{p-k}\binom{n-p}{q-k}\]
      choices.
    \end{proof}

  \item \it \textbf{Section 8 Problem 12:} Fix $k\geq0$ and set $n=4k+2$.
    Prove that exactly one-quarter of all subsets of $[n]$ have size
    divisible by 4.

    \begin{proof}
      The binomial expansion gives
      \[(1+x)^n =\sum_{i=0}^n\binom{n}{i}x^i\]
      for all $x\in\mathbb{C}$. Then letting $\epsilon=e^{\frac{2\pi
      i}{4}}$ denote the $4$th root of unity and substituting $x$ with
      $\epsilon,\epsilon^2,\epsilon^3,$ and 1 into the binomial expansion
      above, we get a series of $4$ equations
      \begin{align*}
        (1+1)^n         &=c_0+c_1+c_2+c_3\\
        (1+\epsilon)^n   &=c_0+c_1\epsilon  +c_2\epsilon^2
          +c_3\epsilon^3\\
        (1+\epsilon^2)^n &=c_0+c_1\epsilon^2+c_2\epsilon^4
          +c_3\epsilon^6\\
        (1+\epsilon^3)^n &=c_0+c_1\epsilon^3+c_2\epsilon^6
          +c_3\epsilon^9\\
      \end{align*}
      where
      \[c_i :=\binom{n}{i}+\binom{n}{i+4}+\binom{n}{i+8}+
      \ldots+\binom{n}{i+4k}\]
      for $i\in\{0,1,2,3\}$. \\

      Now as the $4$th root of unity, $\epsilon$ satisfies
      $\epsilon^4=1$ and also $1+\epsilon+\epsilon^2+\epsilon^{3}=0$,
      thus when we add the 4 equations together and simplify, we get
      \begin{align*}
        \text{LHS} &=(1+\epsilon_n)^n+\ldots+(1+\epsilon_n^n)^n
          =2^n+\sum_{k=1}^{n-1}(1+\epsilon_n^k)^n\\
        =\text{RHS} &=nc_0
          +(c_1+\ldots+c_{l-1})(1+\epsilon_n+\ldots+\epsilon_n^{n-1})\\
        &=nc_0 +(c_1+\ldots+c_{l-1})0\\
        &=nc_0.\\
      \end{align*}
    \end{proof}

  \item \it \textbf{Section 8 Problem 11:} Show that for each fixed
    $l\geq2$,
    \[\lim_{n\rightarrow\infty}
    \frac{\binom{n}{0}+\binom{n}{l}+\binom{n}{2l}+\ldots}{2^n}
    =\frac{1}{l}.\]

    \begin{proof}
      The binomial expansion gives
      \[(1+x)^n =\sum_{k=0}^n\binom{n}{k}x^k\]
      for all $x\in\mathbb{C}$. Then letting $\epsilon_l=e^{\frac{2\pi
      i}{l}}$ denote the $l$th root of unity and substituting $x$ with
      $\epsilon_l,\epsilon_l^2,\ldots,\epsilon_l^l$ into the binomial
      expansion above, we get a series of $l$ equations
      \begin{align*}
        (1+1)^n           &=c_0+c_1+c_2+\ldots+c_{l-1}\\
        (1+\epsilon_l)^n   &=c_0+c_1\epsilon_l  +c_2\epsilon_l^2
          +\ldots+c_{l-1}\epsilon_l^l\\
        (1+\epsilon_l^2)^n &=c_0+c_1\epsilon_l^2+c_2\epsilon_l^4
          +\ldots+c_{l-1}\epsilon_l^{2l}\\
        \ldots\\
        (1+\epsilon_l^k)^n &=c_0+c_1\epsilon_l^k+c_2\epsilon_l^{2k}
          +\ldots+c_{l-1}\epsilon_l^{kl}\\\
        \ldots\\
        (1+\epsilon_l^{l-1})^n
          &=c_0+c_1\epsilon_l^{l-1}+c_2\epsilon_l^{2(l-1)}
          +\ldots+c_{l-1}\epsilon_l^{l(l-1)},\\
      \end{align*}
      where
      \[c_k :=\binom{n}{k}+\binom{n}{k+l}+\binom{n}{k+2l}+
      \ldots+\binom{n}{k+\lfloor n/l\rfloor l}\]
      for $k\in\{0,\ldots,l-1\}$. \\

      Now when we sum the $l$ equations above together, the coefficient of
      $c_k$ for $k\in\{0,\ldots,l-1\}$ will be $f(\epsilon_l^k)$, where
      \[f(x):=1+x+x^2+\ldots+x^{l-1}.\]

      However, because $\epsilon_l$ is the $l$th root of unity,
      each $\epsilon_l^k$ is a root of $x^l-1=(x-1)f(x)$, thus if
      $\epsilon_l^k\neq1$ then $f(\epsilon_l^k)=0$. In other words, for
      $k\in\{1,\ldots,l-1\}$, we have $f(\epsilon_l^k)=0$. Therefore when
      we add the $l$ equations together and simplify, we get on the
      right hand side
      \begin{align*}
        \text{RHS} &=lc_0 +c_1f(\epsilon_l) +c_2f(\epsilon_l^2) +\ldots
          +c_{l-1}f(\epsilon_l^{l-1})\\
        &=lc_0.\\
      \end{align*}

      On the left hand side, we get
      \begin{align*}
        \text{LHS} &=2^n+(1+\epsilon_l)^n+\ldots+(1+\epsilon_l^{l-1})^n\\
        &=\sum_{k=0}^{l-1}(1+\epsilon_l^k)^n\\
        &=\sum_{k=0}^{l-1}
          \epsilon_l^{nk/2}(\epsilon_l^{k/2}+\epsilon_l^{-k/2})^n\\
        &=2^n\sum_{k=0}^{l-1}
          \epsilon_l^{nk/2}\cos^n\left(\frac{k\pi}{l}\right).\\
      \end{align*}

      Equating LHS to RHS and rearranging, we get
      \[\frac{c_0}{2^n}
      =\frac{\sum_{k=0}^{l-1}
      \epsilon_l^{nk/2}\cos^n\left(\frac{k\pi}{l}\right)}{l}.\]

      Thus it suffices to show that 
      \[\lim_{n\rightarrow\infty}\; \sum_{k=0}^{l-1}
      \epsilon_l^{nk/2}\cos^n\left(\frac{k\pi}{l}\right) =1.\]

      Observe that for $k\in\{1,\ldots,l-1\}$,
      \[\lim_{n\rightarrow\infty}
      \left|\epsilon_l^{nk/2}\cos^n\left(\frac{k\pi}{l}\right)\right|
      \leq\lim_{n\rightarrow\infty}
      \left|\cos^n\left(\frac{k\pi}{l}\right)\right|=0\]
      because $\left|\cos\left(\frac{k\pi}{l}\right)\right|<1$, therefore
      \begin{align*}
        \lim_{n\rightarrow\infty}\; \sum_{k=0}^{l-1}
          \epsilon_l^{nk/2}\cos^n\left(\frac{k\pi}{l}\right)
        &=\sum_{k=0}^{l-1} \lim_{n\rightarrow\infty}\;
          \epsilon_l^{nk/2}\cos^n\left(\frac{k\pi}{l}\right)\\
        &=1+\sum_{k=1}^{l-1} \lim_{n\rightarrow\infty}\;
          \epsilon_l^{nk/2}\cos^n\left(\frac{k\pi}{l}\right)\\
        &=1+\sum_{k=1}^{l-1}0\\
        &=1.\\
      \end{align*}
    \end{proof}
\end{enumerate}
\end{document}
