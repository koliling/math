\documentclass{article}
\usepackage[left=3cm,right=3cm,top=3cm,bottom=3cm]{geometry}
\usepackage{amsmath,amssymb,amsthm,tikz,mathtools}
\usepackage{color}
\usepackage[inline]{enumitem}
\usetikzlibrary{patterns}
\setlength{\parindent}{0mm}
\newcommand{\TODO}[1]{\textcolor{red}{TODO: #1}}

\begin{document}
\title{Discrete Mathematics: Problem Set 3}
\author{Li Ling Ko\\ lko@nd.edu}
\date{\today}
\maketitle

\begin{enumerate}[label={\bf Q\arabic*:}]
  \item \it \textbf{Section 15 Problem 5:}
    \begin{proof}
      Let $F_k(n)$ be the number of partitions of $[n]$ into $k$ blocks,
      each containing at least two elements. Find a formula for $F_k(n)$ in
      terms of Stirling numbers of the second kind.

      \begin{proof}
        We apply the inclusion-exclusion theorem to get our formula.
        Let $\mathcal{S}_k(n)$ denote the set of partitions of $[n]$ into
        $k$ non-empty blocks, and given $I\subseteq[n]$, let
        $\mathcal{S}_{k,I}(n)\subseteq \mathcal{S}_k(n)$ denote the set of
        these partitions where for each $i\in I$, the $i$-th element is
        isolated. Observe that
        \begin{align*}
          \mathcal{S}_{k,I}(n) &=\bigcap_{i\in I} \mathcal{S}_{k,\{i\}}(n),\\
          |\mathcal{S}_{k,I}(n)| &=
            \begin{Bmatrix}n-|I|\\k-|I|\end{Bmatrix}.\\
        \end{align*}

        Then,
        \begin{align*}
          F_k(n) &=|\mathcal{S}_k(n)| -\left|\bigcup_{i=1}^n
            \mathcal{S}_{k,\{i\}}(n)\right|\\
          &=\begin{Bmatrix}n\\k\\\end{Bmatrix} -\left|\bigcup_{i=1}^n
            \mathcal{S}_{k,\{i\}}(n)\right|\\
          &=\begin{Bmatrix}n\\k\\\end{Bmatrix}
            -\sum_{I\neq\emptyset\subseteq[k]} (-1)^{|I|-1}
            \left|\bigcap_{i\in I} \mathcal{S}_{k,\{i\}}(n)\right|
            &(\text{by inclusion-exclusion})\\
          &=\begin{Bmatrix}n\\k\\\end{Bmatrix}
            -\sum_{I\neq\emptyset\subseteq[k]} (-1)^{|I|-1}
            \left|\mathcal{S}_{k,I}(n)\right| &(\text{by above})\\
          &=\begin{Bmatrix}n\\k\\\end{Bmatrix}
            -\sum_{I\neq\emptyset\subseteq[k]} (-1)^{|I|-1}
            \begin{Bmatrix}n-|I|\\k-|I|\end{Bmatrix}\\
          &=\begin{Bmatrix}n\\k\\\end{Bmatrix}
            -\sum_{i=1}^n \sum_{I\subseteq[k],|I|=i} (-1)^{i-1}
            \begin{Bmatrix}n-i\\k-i\end{Bmatrix}\\
          &=\begin{Bmatrix}n\\k\\\end{Bmatrix}
            -\sum_{i=1}^n \binom{k}{i} (-1)^{i-1}
            \begin{Bmatrix}n-i\\k-i\end{Bmatrix}.\\
        \end{align*}
      \end{proof}
    \end{proof}

  \item \it \textbf{Section 22 Problem 1:} A hotel corridor has $n$ rooms
    in a row, numbered 1 through $n$. The rooms are to be painted, each one
    either red, or white, or blue, subject to the condition that a red room
    can't be immediately adjacent to a blue room. Let $p_n$ be the number
    of different ways to paint the rooms. Find a recurrence for $p_n$, and
    use generating functions to find an explicit expression for $p_n$.

    \begin{proof}
      Let $\mathcal{S}_n\subset\{R,W,B\}^n$ denote the number of such
      paintings, and for each $i\in\{0,\ldots,n\}$, let
      $\mathcal{S}_n(i)\subset\mathcal{S}_n$ denote the paintings where $i$
      is the number of the largest room painted white, with
      $\mathcal{S}_n(0)$ denoting the case where no door is painted white.
      Then
      \[\mathcal{S}_n =\bigsqcup_{i=0}^n \mathcal{S}_n(i).\]

      Also, observe that if $i$ is number of the largest room colored
      white, then the rooms numbered $i+1\ldots n$ must be either all
      colored red or all colored blue. Thus
      \[|\mathcal{S}_n(0)|=|\mathcal{S}_n(1)|=2,\]
      and for for $i\in\{2,\ldots,n-1\}$,
      \[|\mathcal{S}_n(i)| =2p_{i-1},\]
      and finally,
      \[|\mathcal{S}_n(n)| =p_{n-1}.\]

      Then using the convention that $p_0=1$, we have the following
      recurrence relation:
      \begin{align*}
        p_n &=|\mathcal{S}_n|\\
        &=\left|\bigsqcup_{i=0}^n \mathcal{S}_n(i)\right|\\
        &=\sum_{i=0}^n|\mathcal{S}_n(i)|\\
        &=2+2+\sum_{i=2}^{n-1} 2p_{i-1}+p_{n-1}\\
        &=4+2p_1+\ldots+2p_{n-2}+p_{n-1}\\
        &=2+2p_0+\ldots+2p_{n-2}+p_{n-1}.\\
      \end{align*}

      So the generating function $P(x)$ of the sequence $p_n$ is
      \begin{align*}
        P(x) &=p_0 +p_1x +p_2x^2+\ldots\\
          &=1\\
          &\;\;\;+(2+p_0)x\\
          &\;\;\;+(2+2p_0+p_1)x^2\\
          &\;\;\;+(2+2p_0+2p_1+p_2)x^3\\
          &\;\;\;+\ldots\\
        &=1+x(p_0+p_1x+p_2x^2+\ldots)\\
          &\;\;\;+2x(1+x+x^2+\ldots)\\
          &\;\;\;+2p_0x^2(1+x+x^2+\ldots)\\
          &\;\;\;+2p_1x^3(1+x+x^2+\ldots)\\
          &\;\;\;+\ldots\\
        &=1+xP(x) +\frac{2x+2x^2(p_0+p_1x+p_2x^2+\ldots)}{1-x}\\
        &=1+xP(x) +\frac{2x+2x^2P(x)}{1-x}.\\
      \end{align*}

      Rearranging to solve for $P(x)$, we get
      \begin{align*}
        P(x) &=-\frac{x+1}{x^2+2x-1}\\
        &=-\frac{x+1}{(x+1+\sqrt{2})(x+1-\sqrt{2})}.\\
      \end{align*}

      Solving for partial fractions, we get
      \begin{align*}
        P(x) &=-\frac{x+1}{(x+1+\sqrt{2})(x+1-\sqrt{2})}\\
        &=\frac{(1-\sqrt{2})/2}{1-(1-\sqrt{2})x}
          +\frac{(1+\sqrt{2})/2}{1-(1+\sqrt{2})x}\\
        &=\frac{1-\sqrt{2}}{2} \sum_{n=0}^\infty
          \left(1-\sqrt{2}\right)^nx^n +\frac{1-\sqrt{2}}{2}
          \sum_{n=0}^\infty \left(1+\sqrt{2}\right)^nx^n\\
        &=\frac{1}{2} \sum_{n=0}^\infty \left[\left(1-\sqrt{2}\right)^{n+1}
        +\left(1+\sqrt{2}\right)^{n+1}\right] x^n.\\
      \end{align*}

      Comparing coefficients, we get
      \[p_n =\frac{1}{2} \left[\left(1-\sqrt{2}\right)^{n+1}
        +\left(1+\sqrt{2}\right)^{n+1}\right].\]
    \end{proof}

  \item \it \textbf{Section 25 Problem 1:} $2n$ people sit around a
    circular table. Let $h_n$ be the number of ways that they can pair off
    into $n$ pairs, in such a way that the $n$ pairs can shake hands
    simultaneously without there being any pair of handshakers with
    crossing hands. With $h_0=1$ by definition, show that $h_n$ is the
    $n$th Catalan number.

    \begin{proof}
      We can assume the $2n$ people are numbered 1 to $2n$, and seated in
      that order, with person 1 sitting next to person 2 and person $2n$.
      We partition the legal hand-shaking pairings according to the person
      that person 1 shakes hands with. Observe that if person 1 shakes
      hands with person $k$, then these two persons will have partitioned
      the remaining people into two circular groups - the first group
      comprising of persons 2 through $k-1$, and the second group
      comprising of persons $k+1$ through $2n$. Furthermore, to avoid
      crossing hands with the 1 to $k$ pairing, members within each
      circular group can only shake hands within their groups. Thus the
      size of these groups must both be even, which forces person 1 to
      only be allowed to shake hands with even numbered persons. \\

      So we have scenarios 1 through $n$, where at the $k$th scenario,
      person 1 shakes hands with person $2k$, and partitions the remaining
      people into two circular groups: the first group of size $2k-2$ and
      comprised of members 2 through $2k-1$, and the second group of size
      $2n-2k$ and comprised of members $2k+1$ through $2n$.
      There are $h_{k-1}$ ways the first group can shake hands legally, and
      $h_{n-k}$ ways the second group can shake hands legally, giving a
      total of $h_{k-1}h_{n-k}$ ways for everyone to shake hands legally
      when person 1 shakes hands with person $2k$. Thus we have the
      recurrence relation for $h_n$:
      \[h_n =\sum_{k=0}^{n-1} h_kh_{n-k-1},\]
      which is the same recurrence as the Catalan number. \\

      Finally, to get $h_1=1$ from the recurrence relation, we need to
      assume the convention $h_0=1$.
    \end{proof}

  \item \it \textbf{Section 29 Problem 1:} Show combinatorially that for
    $n,m\geq1$ we have
    \[b_{n,m} =b_{n-1,m} +b_{n,m-1} +b_{n-1,m-1}.\]

    \begin{proof}
      Partition $B_n(m)$ into three disjoint subgroups depending on the
      last coordinate value of the elements in the subgroup:
      \[B_n(m) =B_0 \sqcup B_+ \sqcup B_-,\]
      where
      \begin{align*}
        B_0 &=\{(x_1,\ldots,x_n)\in B_n(m): x_n=0\},\\
        B_+ &=\{(x_1,\ldots,x_n)\in B_n(m): x_n\neq0\; \text{or}\; -1\},\\
        B_- &=\{(x_1,\ldots,x_n)\in B_n(m): x_n=-1\}.\\
      \end{align*}

      Then there is a bijection between $B_0$ and $B_{n-1}(m)$, which sends
      elements $(x_1,\ldots,x_{n-1},0)\in B_0$ to $(x_1,\ldots,x_{n-1})\in
      B_{n-1}(m)$. Thus
      \[|B_0|=|B_{n-1}(m)|=b_{n-1,m}.\]

      Similarly, there is a bijection between $B_-$ and $B_{n-1}(m-1)$, which
      sends elements $(x_1,\ldots,x_{n-1},-1)\in B_-$ to
      $(x_1,\ldots,x_{n-1})\in B_{n-1}(m-1)$. Thus
      \[|B_-|=|B_{n-1}(m-1)|=b_{n-1,m-1}.\]

      Finally, there is a bijection between $B_+$ and $B_{n}(m-1)$, which
      sends elements $(x_1,\ldots,x_{n-1},x_n)\in B_+$ to
      $(x_1,\ldots,x_{n-1},f(x_n))\in B_{n}(m-1)$, where $f$ is defined as
      \begin{align*}
        f(x)=\begin{cases}
          x-1 &\text{if}\; x\geq0\\
          x+1 &\text{otherwise}.\\
        \end{cases}
      \end{align*}
      Thus
      \[|B_+|=|B_{n}(m-1)|=b_{n,m-1}.\]

      So
      \[b_{n,m} =|B_n(m)| =|B_0|+|B_+|+|B_-| =b_{n-1,m} +b_{n,m-1}
      +b_{n-1,m-1}.\]
    \end{proof}

  \item \it \textbf{Section 32 Problem 1:} Let $m_{n,k}$ denote the number
    of ways of extracting $k$ pairs from a set of $n$. Find a recurrence
    relation for $m_{n,k}$ with initial conditions. Show that the
    generating function $P_n(y)=\sum_{k\geq0} m_{n,k}y^k$ always has all
    real roots.

    \begin{proof}
      Partition the set of legal pairings into two disjoint subsets, where
      the first subset comprises of pairings that exclude the largest
      element $n$, and the second subset comprises of pairings that include
      $n$. Clearly the first subset has $m_{n-1,k}$ elements. To count
      the number of elements in the second subset, observe that to
      construct $k$ pairings from $[n]$ of which one of them includes $n$,
      we can first construct $k-1$ pairings from $[n-1]$, then pick one of
      the remaining $n-2(k-1)-1$ elements to be paired with $n$. Thus the
      second subset has $(n+1-2k)m_{n-1,k-1}$ elements, giving us the
      following recurrence relation:
      \[m_{n,k} =m_{n-1,k} +(n+1-2k)m_{n-1,k-1}.\]

      The initial conditions are
      \begin{align*}
        m_{1,k}=0 &\;\;\text{for all}\; k\in\mathbb{N},\\
        m_{n,0}=1 &\;\;\text{for all}\; 1\neq n\in\mathbb{N}.\\
      \end{align*}

      To find a recurrence relation for $P_n(y)$, we have
      \begin{align*}
        P_n(y) &=\sum_{k\geq0} m_{n,k}y^k\\
        &=m_{n,0} +\sum_{k\geq1} m_{n,k}y^k\\
        &=m_{n,0} +\sum_{k\geq1} \left[m_{n-1,k}
          +(n+1-2k)m_{n-1,k-1}\right] y^k\\
        &=\sum_{k\geq0} m_{n-1,k}y^k +(n+1)\sum_{k\geq1} m_{n-1,k-1} y^k -2
          \sum_{k\geq1} km_{n-1,k-1} y^k\\
        &=P_n(y) +(n+1)yP_{n-1}(y) -2y[P_{n-1}(y)+yP_{n-1}'(y)]\\
        &=[1+(n-1)y] P_{n-1}(y) -2y^2 P_{n-1}'(y).\\
      \end{align*}

      Observe that $P_{2k}(y)$ and $P_{2k+1}(y)$ are polynomials of
      degree $k$ with positive coefficients, because $m_{a,b}>0$ for all
      $b\in\{0,\ldots,\lfloor a/2\rfloor\}$, and because $m_{a,b}=0$ for
      all $a>2b$. Also, since $m_{n,0}=1$, the constant term for all
      $P_n(y)$ is always 1, which means $P_n(0)=1$ for all
      $n\in\mathbb{N}$. \\

      We follow the proof of Theorem 31.7 to show that $P_n(y)$ always has
      all real distinct, and negative roots, and that the roots of $P_n(y)$
      and $P_{n-1}(y)$ are distinct from each other and interleave, with
      the least negative root of $P_n(y)$ being closer to 0 than the least
      negative root of $P_{n-1}(y)$. \\

      The statements hold trivially for initial case $P_2(y)=1+y$ and
      $P_3(y)=1+3y$. Assume the statements hold for $P_2(y)$ up to
      $P_{2k}(y)$, and consider the case of $P_{2k+1}(y)$. Let $0> y_1'>
      y_2'\ldots>y_k'$ be the $k$ distinct negative roots of $P_{2k}(y)$.
      Then because $P_{2k}(0)=1$ and $y_1'$ is the first negative root of
      $P_{2k}(y)$ and all roots of $P_{2k}(y)$ are distinct, we must have
      $P_{2k}'(y_1')>0$. Therefore $P_{2k+1}(y_1')\leq0$ from the
      recurrence relation of $P_{2k+1}(y)$. So since $P_{2k+1}(0)=1$,
      $P_{2k+1}(y)$ must have a negative root $y_1$ between $y_1'$ and 0.
      By a similar argument, the $k$ distinct and negative roots of
      $P_{2k+1}(y)$ $0> y_1> y_2\ldots>y_k$ interleave and are distinct
      from $0> y_1'> y_2'\ldots>y_k'$, i.e.
      \[0>y_1>y_1'>y_2>y_2'\ldots>y_k>y_k'.\]

      The same argument also holds going from the case of $P_{2k-1}(y)$ to
      $P_{2k}$, giving $P_{2k}(y)$ at least $k-1$ distinct negative roots.
      However, because the degree of $P_{2k}(y)$ is one higher than the
      degree of $P_{2k-1}$, the gradient of $P_{2k}(y)$ for extremely large
      negative values of $y$ should be the opposite of the gradient of
      $P_{2k-1}(y)$ at the same values, implying that $P_{2k}(y)$ must cut
      the $y$-axis one more time beyond the most negative root of
      $P_{2k-1}(y)$. Therefore $P_{2k}(y)$ has at $k$ distinct negative
      roots that interleave with the roots of $P_{2k-1}(y)$.
    \end{proof}
\end{enumerate}
\end{document}
