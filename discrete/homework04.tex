\documentclass{article}
\usepackage[left=3cm,right=3cm,top=3cm,bottom=3cm]{geometry}
\usepackage{amsmath,amssymb,amsthm,tikz,mathtools}
\usepackage{color}
\usepackage[inline]{enumitem}
\usetikzlibrary{patterns}
\setlength{\parindent}{0mm}
\newcommand{\TODO}[1]{\textcolor{red}{TODO: #1}}

\begin{document}
\title{Discrete Mathematics: Problem Set 4}
\author{Li Ling Ko\\ lko@nd.edu}
\date{\today}
\maketitle

\begin{enumerate}[label={\bf Q\arabic*:}]
  \item \it \textbf{Section 40 Problem 1:}
    \begin{enumerate}
      \item \label{qn:odd}
        Let $a_n$ be the number of permutations of $[n]$ with only
        odd-length cycles. Use the exponential formula to show that $A(x)$,
        the exponential generating function of $(a_n)_{n\geq0}$ satisfies
        \[A(x) = \sqrt{\frac{1+x}{1-x}}.\]

        \begin{proof}
          Following the notation of Theorem 37.1 in the notes, we have
          \begin{align*}
            c_n =
            \begin{cases}
              (n-1)! &\text{if}\; n\; \text{is odd},\\
              0 &\text{otherwise},\\
            \end{cases}
          \end{align*}

          so
          \begin{align*}
            (c_n)_{n\geq0}\xleftrightarrow{\text{egf}} &C(x)\\
            =&\sum_{n\geq0} \frac{c_nx^n}{n!}\\
            =&\sum_{k\geq0} \frac{(2k)!x^{2k+1}}{(2k+1)!}\\
            =&\sum_{k\geq0} \frac{x^{2k+1}}{2k+1}.\\
          \end{align*}

          Now the Taylor series expansion of $\ln{(1+x)}$ is
          \[\ln{(1+x)} =x -\frac{x^2}{2} +\frac{x^3}{3} -\ldots
          =\sum_{k\geq0} \frac{x^{2k+1}}{2k+1} -\sum_{k\geq1}
          \frac{x^{2k}}{2k},\]
          so
          \[\frac{1}{2}\ln{(1-x^2)} =-\frac{x^2}{4} -\frac{x^4}{6}
          -\frac{x^6}{8} -\ldots =-\sum_{k\geq1} \frac{x^{2k}}{2k},\]
          therefore
          \begin{align*}
            C(x) &=\sum_{k\geq0} \frac{x^{2k+1}}{2k+1}\\
            &=\ln{(1+x)} -\frac{1}{2}\ln{(1-x^2)}\\
            &=\ln{\frac{1+x}{\sqrt{1-x^2}}}\\
            &=\ln{\sqrt{\frac{1+x}{1-x}}}.\\
          \end{align*}

          Then
          \[A(x) =\exp{\left(C(x)\right)}
          =\exp{\left(\ln{\sqrt{\frac{1+x}{1-x}}}\right)}
          =\sqrt{\frac{1+x}{1-x}}.\]
        \end{proof}

      \item Let $b_n$ be the number of permutations of $[n]$ with an even
        number of cycles, all odd-length. Use the result of
        part (\ref{qn:odd}) to find $B(x)$, the exponential generating
        function of $(b_n)_{n\geq0}$.

        \begin{proof}
          Observe that if $n$ is odd, then there can only be an odd number
          of cycles all of odd-length. Likewise, if $n$ is even, then there
          can only be an even number of cycles all of odd-length. Therefore
          \begin{align*}
            b_n =
            \begin{cases}
              a_n &\text{if}\; n\; \text{is even},\\
              0 &\text{otherwise}.\\
            \end{cases}
          \end{align*}

          Thus
          \begin{align*}
            B(x) &=a_0+a_2x^2+a_4x^4+\ldots\\
            &=\sum_{n\geq0} a_{2n}x^{2n}\\
            &=\frac{1}{2} \left[\sum_{n\geq0} a_{n}x^{n} + \sum_{n\geq0}
              a_{n}(-x)^{n}\right]\\
            &=\frac{1}{2} \left[A(x) + A(-x)\right]\\
            &=\frac{1}{2} \left[\sqrt{\frac{1+x}{1-x}} +
              \sqrt{\frac{1-x}{1+x}}\right].\\
          \end{align*}
        \end{proof}
    \end{enumerate}
\end{enumerate}
\end{document}
