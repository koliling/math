\documentclass{article}
\usepackage[left=3cm,right=3cm,top=3cm,bottom=3cm]{geometry}
\usepackage{amsmath,amssymb,amsthm,tikz,mathtools}
\usepackage{color}
\usepackage[inline]{enumitem}
\usetikzlibrary{patterns}
\setlength{\parindent}{0mm}
\newcommand{\TODO}[1]{\textcolor{red}{TODO: #1}}

\begin{document}
\title{Discrete Mathematics: Problem Set 4}
\author{Li Ling Ko\\ lko@nd.edu}
\date{\today}
\maketitle

\it \textbf{Section 40 Problem 1}
  \begin{enumerate}[label={(\alph*)}]
    \item \label{qn:odd}
      Let $a_n$ be the number of permutations of $[n]$ with only
      odd-length cycles. Use the exponential formula to show that $A(x)$,
      the exponential generating function of $(a_n)_{n\geq0}$ satisfies
      \[A(x) = \sqrt{\frac{1+x}{1-x}}.\]

      \begin{proof}
        Following the notation of Theorem 37.1 in the notes, we have
        \begin{align*}
          c_n =
          \begin{cases}
            (n-1)! &\text{if}\; n\; \text{is odd},\\
            0 &\text{otherwise},\\
          \end{cases}
        \end{align*}

        so
        \begin{align*}
          (c_n)_{n\geq0}\xleftrightarrow{\text{egf}} &C(x)\\
          =&\sum_{n\geq0} \frac{c_nx^n}{n!}\\
          =&\sum_{k\geq0} \frac{(2k)!x^{2k+1}}{(2k+1)!}\\
          =&\sum_{k\geq0} \frac{x^{2k+1}}{2k+1}.\\
        \end{align*}

        Now the Taylor series expansion of $\ln{(1+x)}$ is
        \[\ln{(1+x)} =x -\frac{x^2}{2} +\frac{x^3}{3} -\ldots
        =\sum_{k\geq0} \frac{x^{2k+1}}{2k+1} -\sum_{k\geq1}
        \frac{x^{2k}}{2k},\]
        so
        \[\frac{1}{2}\ln{(1-x^2)} =-\frac{x^2}{4} -\frac{x^4}{6}
        -\frac{x^6}{8} -\ldots =-\sum_{k\geq1} \frac{x^{2k}}{2k},\]
        therefore
        \begin{align*}
          C(x) &=\sum_{k\geq0} \frac{x^{2k+1}}{2k+1}\\
          &=\ln{(1+x)} -\frac{1}{2}\ln{(1-x^2)}\\
          &=\ln{\frac{1+x}{\sqrt{1-x^2}}}\\
          &=\ln{\sqrt{\frac{1+x}{1-x}}}.\\
        \end{align*}

        Then
        \[A(x) =\exp{\left(C(x)\right)}
        =\exp{\left(\ln{\sqrt{\frac{1+x}{1-x}}}\right)}
        =\sqrt{\frac{1+x}{1-x}}.\]
      \end{proof}

    \item \label{qn:even-no-odd-cycles}
      Let $b_n$ be the number of permutations of $[n]$ with an even
      number of cycles, all odd-length. Use the result of
      \ref{qn:odd} to find $B(x)$, the exponential generating function
      of $(b_n)_{n\geq0}$.

      \begin{proof}
        Observe that if $n$ is odd, then there can only be an odd number
        of cycles all of odd-length. Likewise, if $n$ is even, then there
        can only be an even number of cycles all of odd-length. Therefore
        \begin{align*}
          b_n =
          \begin{cases}
            a_n &\text{if}\; n\; \text{is even},\\
            0 &\text{otherwise}.\\
          \end{cases}
        \end{align*}

        Thus
        \begin{align*}
          B(x) &=a_0+a_2x^2+a_4x^4+\ldots\\
          &=\sum_{n\geq0} a_{2n}x^{2n}\\
          &=\frac{1}{2} \left[\sum_{n\geq0} a_{n}x^{n} + \sum_{n\geq0}
            a_{n}(-x)^{n}\right]\\
          &=\frac{1}{2} \left[A(x) + A(-x)\right]\\
          &=\frac{1}{2} \left[\sqrt{\frac{1+x}{1-x}} +
            \sqrt{\frac{1-x}{1+x}}\right]\\
          &=\frac{1}{2} \frac{1+x+1-x}{\sqrt{1-x^2}}\\
          &=(1-x^2)^{-\frac{1}{2}}.\\
        \end{align*}
      \end{proof}

    \item Let $p$ be the probability that a permutation selected
      uniformly at random from all permutations of $[n]$ has an even
      number of cycles, all odd-length. Let $q$ be the probability that a
      fair coin tossed $n$ times comes up heads exactly $n/2$ times. Use
      the result of~\ref{qn:even-no-odd-cycles} to show that $p=q$.

      \begin{proof}
        There are $2^n$ possible outcomes when we toss a fair coin $n$
        times, of which $\binom{n}{n/2}$ of the outcomes have exactly
        $n/2$ heads. Therefore
        \[q= \frac{\binom{n}{n/2}}{2^n}
        =\begin{cases}
          \frac{1}{2^n} \binom{n}{n/2} &\text{if}\; n\; \text{is even},\\
          0 &\text{otherwise}.\\
        \end{cases}\]

        Also, $p=b_n/n!$, where
        \begin{align*}
          b_n &=[x^n]B(x)\\
          &=[x^n](1-x^2)^{\frac{1}{2}} &\text{(from
            part~\ref{qn:even-no-odd-cycles})}\\
          &=[x^n] \left\{ \sum_{k\geq0} \frac{(2k-1)!!}{(2k)!!}
            x^{2k} \right\} &\text{(from generalized negative binomial
            series)}\\
          &=\begin{cases}
              \frac{(n-1)!!}{n!!} &\text{if}\; n\; \text{is even},\\
              0 &\text{otherwise}.\\
            \end{cases}\\
        \end{align*}

        Thus it suffices to show that $\frac{(n-1)!!}{n!!} =\frac{1}{2^n}
        \binom{n}{n/2}$ when $n$ is even. Write $n=2k$. Then
        \begin{align*}
          \frac{(n-1)!!}{n!!} &=\frac{(2k-1)!!}{(2k)!!}\\
          &=\frac{(2k)!!(2k-1)!!}{(2k)!!(2k)!!}\\
          &=\frac{(2k)!}{((2k)!!)^2}\\
          &=\frac{(2k)!}{(2^kk!)^2}\\
          &=\frac{(2k)!}{2^{2k}k!k!}\\
          &=\frac{1}{2^{2k}} \frac{(2k)!}{k!k!}\\
          &=\frac{1}{2^{n}} \binom{n}{n/2}.\\
        \end{align*}
      \end{proof}
  \end{enumerate}

\it \textbf{Section 40 Problem 2} The Lah number $L(n,k)$ is the number of
  ways to partition $[n]$ into $k$ non-empty lists. By convention
  $L(0,0)=1$.

  \begin{enumerate}[label={(\alph*)}]
    \item Use the exponential formula to get a closed-form expression for
      the mixed generating function $L(x,y) =\sum_{n,k\geq0} L(n,k)
      \frac{x^n}{n!} y^k$, and in turn use this to find an explicit
      expression for $L(n,k)$ (which should turn out to be very simple
      involving no summation). \label{qn:Lah}

      \begin{proof}
        Using the notation of Theorem 35.1, for the Lah numbers, we have
        \begin{align*}
          c_n =
          \begin{cases}
            n! &\text{if}\; n\geq1\\
            0 &\text{if}\; n=0.\\
          \end{cases}
        \end{align*}

        Therefore
        \begin{align*}
          (c_n)_{n\geq0}\xleftrightarrow{\text{egf}} &C(x)\\
          =&\sum_{n\geq0} \frac{c_nx^n}{n!}\\
          =&\sum_{n\geq1} \frac{n!x^n}{n!}\\
          =&\sum_{n\geq1} x^n\\
          =&\frac{x}{1-x}.\\
        \end{align*}

        Then Theorem 35.1 gives
        \begin{align*}
          A(x,y) &=\exp{\left(yC(x)\right)}\\
          &=\exp{\left(\frac{xy}{1-x} \right)}\\
          &=\sum_{k\geq0} \frac{1}{k!} \frac{x^ky^k}{(1-x)^k}\\
          &=\sum_{k\geq0} \frac{x^ky^k}{k!} (1+x+x^2+\ldots)^k.\\
        \end{align*}

        Observe that $[x^n](1+x+x^2+\ldots)^k$ is the same as the number of
        $k$-tuples of non-negative integers that sum to $n$, which is the
        same as number of weak compositions of $n$ into $k$ parts, which
        equals $\binom{n+k-1}{n}$ (Proposition 13.3). Thus we can further
        simplify
        \begin{align*}
          A(x,y) &=\sum_{k\geq0} \frac{x^ky^k}{k!} (1+x+x^2+\ldots)^k\\
          &=\sum_{k\geq0} \frac{x^ky^k}{k!} \sum_{n\geq0} \binom{n+k-1}{n}
            x^n\\
          &=\sum_{n,k\geq0} \frac{1}{k!} \binom{n+k-1}{n} x^{n+k}y^k\\
          &=\sum_{n,k\geq0} \frac{(n+k-1)!}{k!n!(k-1)!}
            x^{n+k}y^k,\\
          &=\sum_{n,k\geq0} \frac{(n-1)!}{k!(n-k)!(k-1)!} x^{n}y^k.\\
        \end{align*}

        Then
        \begin{align*}
          L(n,k) &=[\frac{x^ny^k}{n!}]A(x,y)\\
          &=\frac{n!(n-1)!}{k!(n-k)!(k-1)!}.\\
        \end{align*}
      \end{proof}

    \item Give a combinatorial explanation for the formula you obtained in
      part~\ref{qn:Lah}.
      \begin{proof}
      \end{proof}
  \end{enumerate}
\end{document}
