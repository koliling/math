\documentclass{article}
\usepackage[left=3cm,right=3cm,top=3cm,bottom=3cm]{geometry}
\usepackage{amsmath,amssymb,amsthm,tikz,mathtools}
\usepackage{color}
\usepackage[inline]{enumitem}
\usetikzlibrary{patterns}
\setlength{\parindent}{0mm}
\newcommand{\TODO}[1]{\textcolor{red}{TODO: #1}}

\begin{document}
\title{Discrete Mathematics: Problem Set 4}
\author{Li Ling Ko\\ lko@nd.edu}
\date{\today}
\maketitle

\it \textbf{Section 40 Problem 1}
  \begin{enumerate}[label={(\alph*)}]
    \item \label{qn:odd}
      Let $a_n$ be the number of permutations of $[n]$ with only
      odd-length cycles. Use the exponential formula to show that $A(x)$,
      the exponential generating function of $(a_n)_{n\geq0}$ satisfies
      \[A(x) = \sqrt{\frac{1+x}{1-x}}.\]

      \begin{proof}
        Following the notation of Theorem 37.1 in the notes, we have
        \begin{align*}
          c_n =
          \begin{cases}
            (n-1)! &\text{if}\; n\; \text{is odd},\\
            0 &\text{otherwise},\\
          \end{cases}
        \end{align*}

        so
        \begin{align*}
          (c_n)_{n\geq0}\xleftrightarrow{\text{egf}} &C(x)\\
          =&\sum_{n\geq0} \frac{c_nx^n}{n!}\\
          =&\sum_{k\geq0} \frac{(2k)!x^{2k+1}}{(2k+1)!}\\
          =&\sum_{k\geq0} \frac{x^{2k+1}}{2k+1}.\\
        \end{align*}

        Now the Taylor series expansion of $\ln{(1+x)}$ is
        \[\ln{(1+x)} =x -\frac{x^2}{2} +\frac{x^3}{3} -\ldots
        =\sum_{k\geq0} \frac{x^{2k+1}}{2k+1} -\sum_{k\geq1}
        \frac{x^{2k}}{2k},\]
        so
        \[\frac{1}{2}\ln{(1-x^2)} =-\frac{x^2}{4} -\frac{x^4}{6}
        -\frac{x^6}{8} -\ldots =-\sum_{k\geq1} \frac{x^{2k}}{2k},\]
        therefore
        \begin{align*}
          C(x) &=\sum_{k\geq0} \frac{x^{2k+1}}{2k+1}\\
          &=\ln{(1+x)} -\frac{1}{2}\ln{(1-x^2)}\\
          &=\ln{\frac{1+x}{\sqrt{1-x^2}}}\\
          &=\ln{\sqrt{\frac{1+x}{1-x}}}.\\
        \end{align*}

        Then
        \[A(x) =\exp{\left(C(x)\right)}
        =\exp{\left(\ln{\sqrt{\frac{1+x}{1-x}}}\right)}
        =\sqrt{\frac{1+x}{1-x}}.\]
      \end{proof}

    \item \label{qn:even-no-odd-cycles}
      Let $b_n$ be the number of permutations of $[n]$ with an even
      number of cycles, all odd-length. Use the result of
      \ref{qn:odd} to find $B(x)$, the exponential generating function
      of $(b_n)_{n\geq0}$.

      \begin{proof}
        Observe that if $n$ is odd, then there can only be an odd number
        of cycles all of odd-length. Likewise, if $n$ is even, then there
        can only be an even number of cycles all of odd-length. Therefore
        \begin{align*}
          b_n =
          \begin{cases}
            a_n &\text{if}\; n\; \text{is even},\\
            0 &\text{otherwise}.\\
          \end{cases}
        \end{align*}

        Thus
        \begin{align*}
          B(x) &=a_0+a_2x^2+a_4x^4+\ldots\\
          &=\sum_{n\geq0} a_{2n}x^{2n}\\
          &=\frac{1}{2} \left[\sum_{n\geq0} a_{n}x^{n} + \sum_{n\geq0}
            a_{n}(-x)^{n}\right]\\
          &=\frac{1}{2} \left[A(x) + A(-x)\right]\\
          &=\frac{1}{2} \left[\sqrt{\frac{1+x}{1-x}} +
            \sqrt{\frac{1-x}{1+x}}\right]\\
          &=\frac{1}{2} \frac{1+x+1-x}{\sqrt{1-x^2}}\\
          &=(1-x^2)^{-\frac{1}{2}}.\\
        \end{align*}
      \end{proof}

    \item Let $p$ be the probability that a permutation selected
      uniformly at random from all permutations of $[n]$ has an even
      number of cycles, all odd-length. Let $q$ be the probability that a
      fair coin tossed $n$ times comes up heads exactly $n/2$ times. Use
      the result of~\ref{qn:even-no-odd-cycles} to show that $p=q$.

      \begin{proof}
        There are $2^n$ possible outcomes when we toss a fair coin $n$
        times, of which $\binom{n}{n/2}$ of the outcomes have exactly
        $n/2$ heads. Therefore
        \[q= \frac{\binom{n}{n/2}}{2^n}
        =\begin{cases}
          \frac{1}{2^n} \binom{n}{n/2} &\text{if}\; n\; \text{is even},\\
          0 &\text{otherwise}.\\
        \end{cases}\]

        Also, $p=b_n/n!$, where
        \begin{align*}
          b_n &=[x^n]B(x)\\
          &=[x^n](1-x^2)^{\frac{1}{2}} &\text{(from
            part~\ref{qn:even-no-odd-cycles})}\\
          &=[x^n] \left\{ \sum_{k\geq0} \frac{(2k-1)!!}{(2k)!!}
            x^{2k} \right\} &\text{(from generalized negative binomial
            series)}\\
          &=\begin{cases}
              \frac{(n-1)!!}{n!!} &\text{if}\; n\; \text{is even},\\
              0 &\text{otherwise}.\\
            \end{cases}\\
        \end{align*}

        Thus it suffices to show that $\frac{(n-1)!!}{n!!} =\frac{1}{2^n}
        \binom{n}{n/2}$ when $n$ is even. Write $n=2k$. Then
        \begin{align*}
          \frac{(n-1)!!}{n!!} &=\frac{(2k-1)!!}{(2k)!!}\\
          &=\frac{(2k)!!(2k-1)!!}{(2k)!!(2k)!!}\\
          &=\frac{(2k)!}{((2k)!!)^2}\\
          &=\frac{(2k)!}{(2^kk!)^2}\\
          &=\frac{(2k)!}{2^{2k}k!k!}\\
          &=\frac{1}{2^{2k}} \frac{(2k)!}{k!k!}\\
          &=\frac{1}{2^{n}} \binom{n}{n/2}.\\
        \end{align*}
      \end{proof}
  \end{enumerate}

\it \textbf{Section 40 Problem 2} The Lah number $L(n,k)$ is the number of
  ways to partition $[n]$ into $k$ non-empty lists. By convention
  $L(0,0)=1$.

  \begin{enumerate}[label={(\alph*)}]
    \item Use the exponential formula to get a closed-form expression for
      the mixed generating function $L(x,y) =\sum_{n,k\geq0} L(n,k)
      \frac{x^n}{n!} y^k$, and in turn use this to find an explicit
      expression for $L(n,k)$ (which should turn out to be very simple
      involving no summation). \label{qn:Lah}

      \begin{proof}
        Using the notation of Theorem 35.1, for the Lah numbers, we have
        \begin{align*}
          c_n =
          \begin{cases}
            n! &\text{if}\; n\geq1\\
            0 &\text{if}\; n=0.\\
          \end{cases}
        \end{align*}

        Therefore
        \begin{align*}
          (c_n)_{n\geq0}\xleftrightarrow{\text{egf}} &C(x)\\
          =&\sum_{n\geq0} \frac{c_nx^n}{n!}\\
          =&\sum_{n\geq1} \frac{n!x^n}{n!}\\
          =&\sum_{n\geq1} x^n\\
          =&\frac{x}{1-x}.\\
        \end{align*}

        Then Theorem 35.1 gives
        \begin{align*}
          A(x,y) &=\exp{\left(yC(x)\right)}\\
          &=\exp{\left(\frac{xy}{1-x} \right)}\\
          &=\sum_{k\geq0} \frac{1}{k!} \frac{x^ky^k}{(1-x)^k}\\
          &=\sum_{k\geq0} \frac{x^ky^k}{k!} (1+x+x^2+\ldots)^k.\\
        \end{align*}

        Observe that $[x^n](1+x+x^2+\ldots)^k$ is the same as the number of
        $k$-tuples of non-negative integers that sum to $n$, which is the
        same as number of weak compositions of $n$ into $k$ parts, which
        equals $\binom{n+k-1}{n}$ (Proposition 13.3). Thus we can further
        simplify
        \begin{align*}
          A(x,y) &=\sum_{k\geq0} \frac{x^ky^k}{k!} (1+x+x^2+\ldots)^k\\
          &=\sum_{k\geq0} \frac{x^ky^k}{k!} \sum_{n\geq0} \binom{n+k-1}{n}
            x^n\\
          &=\sum_{n,k\geq0} \frac{1}{k!} \binom{n+k-1}{n} x^{n+k}y^k\\
          &=\sum_{n,k\geq0} \frac{(n+k-1)!}{k!n!(k-1)!}
            x^{n+k}y^k,\\
          &=\sum_{n,k\geq0} \frac{(n-1)!}{k!(n-k)!(k-1)!} x^{n}y^k.\\
        \end{align*}

        Then
        \begin{align*}
          L(n,k) &=\left[\frac{x^ny^k}{n!}\right] A(x,y)\\
          &=\frac{n!(n-1)!}{k!(n-k)!(k-1)!}.\\
        \end{align*}
      \end{proof}

    \item Give a combinatorial explanation for the formula you obtained in
      part~\ref{qn:Lah}.
      \begin{proof}
        We first pick the first element for each of the $k$ lists, say
        $a_1,\ldots,a_k\subseteq[n]$. There
        are $\binom{n}{k}$ ways to do this. This leaves $(n-k)$ elements
        whose relative positions have yet to be decided. We arrange the
        $(n-k)$ elements in ascending order $b_1<\ldots b_{n-k}$, and
        decide the positions of the smaller elements first before deciding
        the positions of the larger elements. We only allow that each
        element be placed behind exactly one of the elements that has
        already been fixed. So for the smallest of the $(n-k)$ elements
        $b_1$, since there are $k$ elements whose relative positions have
        been decided, $b_1$ has only $k$ positions to choose from,
        depending on which element $a_1,\ldots,a_k$ it should lie behind,
        though not necessarily directly behind. Then for the next smallest
        element $b_2$, there are $(k+1)$ positions to choose from,
        depending on which element $a_1,\ldots,a_k,b_1$ it should lie behind,
        again though not necessarily directly behind. Continuing in this
        manner, there are $k(k+1)\cdots(n-1)$ ways to decide the positions
        of the remaining $(n-k)$ elements $b_1\ldots b_{n-k}$. Thus
        \[L(n,k) =\binom{n}{k} k(k+1)\cdots(n-1)
        =\frac{n!(n-1)!}{k!(n-k)!(k-1)!} =\binom{n-1}{k-1}\frac{n!}{k!}.\]
      \end{proof}

    \item Either combinatorially, or from the generating function, show
      that for all $n\geq1$
      \[x^{(n)} =\sum_{k=1}^n L(n,k)(x)_k\]
      (so that the Lah matrix $(L(n,k))_{n,k\geq0}$ moves one from the
      falling-power basis of the space of polynomials to the rising-power
      basis).

      \begin{proof}
        From the polynomial principle, it suffices to prove for the cases
        where $x$ is a positive integer.  Observe that $x^{(n)}$ is the
        same as $(x+n-1)_n$, where we count the number of ways of choosing
        $n$ elements from $[x+n-1]$ and arange them order. Consider
        $[x+n-1]$ as the disjoint union of sets $A=\{a_1,\ldots,a_{x}\}$
        and $B=\{b_1,\ldots,b_{n-1}\}$. Note that at least one of the chosen
        elements must come from set $A$ because there are less than $n$
        elements in $B$. \\

        To choose $n$ elements from $A\sqcup B$ and arrange them in order,
        we first choose $k$ unordered elements from $A$, which gives us
        \[\binom{x}{k} =\frac{(x)_k}{k!}\]
        ways to choose. For each choices of $k$ elements, there are
        \[\binom{n-1}{n-k} =\binom{n-1}{k-1}\]
        ways to choose the remaining unordered $n-k$ elements from $A$.
        Then with each choice of $n$ unordered elements, there are $n!$
        ways to arrange them in order. Thus there is a total of
        \[\sum_{k=1}^n \frac{(x)_k}{k!}\binom{n-1}{k-1} n! =\sum_{k=1}^n
        L(n,k)(x)_k\]
        ways to choose $n$ elements from $[x+n-1]$ and arrange them in
        order.
      \end{proof}

    \item Deduce that
      \[(x)_n =\sum_{k=1}^n (-1)^{n-k} L(n,k)x^{(k)}\]
      so that the matrix that moves one in the other direction is the
      signed Lah matrix
      \[((-1)^{n-k} L(n,k))_{n,k\geq0}.\]

      \begin{proof}
        Using the fact that
        \begin{align*}
          (-x)_n &=(-1)^nx^{(n)},\\
          (-x)^{(k)} &=(-1)^k(x)_k,\\
        \end{align*}
        the previous part of the question will give us
        \begin{align*}
          (x)_n &=(-1)^n (-x)^{(n)}\\
          &=(-1)^n \sum_{k=1}^n L(n,k) (-x)_{k}\\
          &=(-1)^n \sum_{k=1}^n L(n,k) (-1)^{k} x^{(k)}\\
          &=(-1)^n \sum_{k=1}^n L(n,k) (-1)^{-k} x^{(k)}\\
          &=\sum_{k=1}^n (-1)^{n-k} L(n,k) x^{(k)}.\\
        \end{align*}
      \end{proof}
  \end{enumerate}

\it \textbf{Section 40 Problem 3} In the coupon collector's problem we
  imagine that we would like to get a complete collection of photos of
  movie stars, where each time we buy a box of cereal we acquire one such
  photo, which may of course duplicate one that is already in our
  collection. Suppose there are $d$ different photos in a complete
  collection. Let $p_n$ be the probability that exactly $n$ trials are
  needed in order, for the first time, to have a complete collection.

  \begin{enumerate}[label={(\alph*)}]
    \item Find a very simple formula for $p_n$ involving Stirling numbers
      of the second kind.

      \begin{proof}
        Clearly if $n<d$ then $p_n=0$. So assume $n\geq d$. Let
        $\mathcal{S}_n\subseteq [d]^{[n]}$ be the set of $n$-tuples with
        values in $[d]$ that represent the set of $n$ trials that first
        complete the collection of $d$ photos. Notice that for each 
        $n$-tuple $\bar{a}=(a_1,\ldots,a_n)\in\mathcal{S}$, the last coupon
        number $a_n$ must have not appeared before, and the first $(n-1)$
        elements of the tuple must contain all $(d-1)$ elements
        $[d]-\{a_n\}$. \\
        
        Write $[d]-\{a_n\}=\{b_1<\ldots<b_{n-1}\}$. We can associate to
        each tuple $\bar{a}$ with the partition into $(d-1)$-parts
        \[S_1|\ldots|S_{d-1},\]
        where
        \[S_k:=\{i\in[n-1]: a_i=b_k\}.\]

        Then $\mathcal{S}_n$ will give ${n-1\brace d-1}$ partitions, where
        each partition is repeated exactly $d!$ times. Thus
        \[p_n =|\mathcal{S}_n|/d^n ={n-1\brace d-1}\frac{d!}{d^n}.\]
      \end{proof}

    \item Let $p(x)$ be the ordinary generating function of $(p_n)_{n\geq
      d}$. Show that
      \[p(x) =\frac{(d-1)!x^d}{(d-x)(d-2x)\cdots(d-(d-1)x)}.\]
      (You can use anything we know about the generating function of
      Stirling numbers of the second kind.)

      \begin{proof}
        From Section 30 of the notes, we know that
        \[B_d(x) :=\sum_{n\geq0} {n\brace d} x^n =\sum_{n\geq d} {n\brace
        d} x^n =\frac{x^d}{(1-x)(1-2x)\ldots(1-dx)}.\]

        Now the ordinary generating function we want is
        \begin{align*}
          p(x) &=\sum_{n\geq d} p_n x^n\\
          &=\sum_{n\geq d} {n-1\brace d-1}\frac{d!}{d^n} x^n\\
          &=d! \sum_{n\geq d} {n-1\brace d-1} \left(\frac{x}{d}\right)^n\\
          &=(d-1)!x \sum_{n\geq d} {n-1\brace d-1}
            \left(\frac{x}{d}\right)^{n-1}\\
          &=(d-1)!x \sum_{n\geq d-1} {n\brace d-1}
            \left(\frac{x}{d}\right)^{n}\\
          &=(d-1)!x B_{d-1}(x/d)\\
          &=(d-1)!x \frac{(x/d)^{d-1}}{(1-x/d)(1-2x/d)\cdots(1-x)}\\
          &=\frac{(d-1)!x^d}{(d-x)(d-2x)\cdots(d-(d-1)x)}.
        \end{align*}
      \end{proof}

    \item Find, directly from the generating function $p(x)$, the average
      number of trials that are needed to get a complete collection of all
      $d$ coupons.

      \begin{proof}
        The average number of trials is
        \[\mu_d :=\sum_{n\geq d} np_n =p'(1).\]
        To compute $p'(1)$ more easily, we first take logarithm before
        taking derivatives:

        \begin{align*}
          \ln{p(x)}
            &=\ln{\left[\frac{(d-1)!x^d}{(d-x)(d-2x)\cdots(d-(d-1)x)}
            \right]}\\
          &=\ln{(d-1)!} +d\ln{x} -\ln{(d-x)} -\cdots -\ln{(d-(d-1)x)}.\\
        \end{align*}

        Taking derivatives with respect to $x$ on both sides give
        \begin{align*}
          \frac{p'(x)}{p(x)} &=\frac{d}{x} +\frac{1}{d-x} +\frac{2}{d-2x}
            +\cdots +\frac{d-1}{d-(d-1)x}.\\
        \end{align*}

        Multiplying with $p(x)$ and substituting with $x=1$ give
        \begin{align*}
          \mu_d=p'(1) &=\left[\frac{d}{1} +\frac{1}{d-1} +\frac{2}{d-2}
            +\cdots +\frac{d-1}{d-(d-1)} \right] p(1)\\
          &=\left[d +\sum_{k-1}^{d-1} \frac{k}{d-k}\right] \cdot
            \frac{(d-1)!}{(d-2)(d-1)\cdots2\cdot1}\\
          &=d +\sum_{k=1}^{d-1} \frac{k}{d-k}\\
          &=d +d\sum_{k=1}^{d-1} \frac{1}{d-k} -\sum_{k=1}^{d-1}1\\
          &=1 +d\sum_{k=1}^{d-1} \frac{1}{k}\\
          &=d\sum_{k=1}^{d} \frac{1}{k}\\
          &=dH_d.\\
        \end{align*}
      \end{proof}

    \item Similarly, using $p(x)$, find the standard deviation of the
      number of trials.

      \begin{proof}
        Let $X$ be the random variable that represents the number of trials
        needed to get for the first time a complete collection of coupons.
        Then
        \[E(X^2) =\sum_{n\geq d} n^2p_n =p''(1)+p'(1).\]

        Like in the previous part of the question, to compute the
        derivatives more easily, we take logarithm before differentiating.
        From the previous part of the question, we have
        \begin{align*}
          p'(x) &=\left[\frac{d}{x} +\frac{1}{d-x} +\frac{2}{d-2x}
            +\cdots +\frac{d-1}{d-(d-1)x} \right] p(x)\\
          &=\left[\frac{d}{x} +\sum_{k=1}^{d-1} \frac{k}{d-kx}\right]
            p(x).\\
        \end{align*}

        Differentiating with respect to $x$ gives
        \begin{align*}
          p''(x) &=\left[-\frac{d}{x^2} +\sum_{k=1}^{d-1}
            \frac{k^2}{(d-kx)^2}\right] p(x) + \left[\frac{d}{x}
            +\sum_{k=1}^{d-1} \frac{k}{d-kx}\right] p'(x)\\
          &=\left[-\frac{d}{x^2} +\sum_{k=1}^{d-1}
            \frac{k^2}{(d-kx)^2}\right] p(x) + \frac{p'(x)}{p(x)} p'(x)\\
          &=\left[-\frac{d}{x^2} +\sum_{k=1}^{d-1}
            \frac{k^2}{(d-kx)^2}\right] p(x) + \frac{(p'(x))^2}{p(x)}.\\
        \end{align*}

        Substituting $x=1$ and using $p(1)=1$ and $p'(1)=dH_d$ as shown in
        the previous part of this question, we have
        \begin{align*}
          p''(1) &=\left[-d +\sum_{k=1}^{d-1}
            \frac{k^2}{(d-k)^2}\right] +d^2H_d^2\\
          &=-d +\sum_{k=1}^{d-1} \left[\frac{d^2}{(d-k)^2}
            -\frac{2d}{d-k} +1\right] +d^2H_d^2\\
          &=d^2 \sum_{k=1}^{d-1} \frac{1}{k^2} -2d\sum_{k=1}^{d-1}
            \frac{1}{k} +d^2H_d^2-1\\
          &=d^2 \sum_{k=1}^{d-1} \frac{1}{k^2} -2dH_d +d^2H_d^2+1\\
        \end{align*}

        Then
        \begin{align*}
          \sigma_d^2 &=E(X^2)-\mu_d^2\\
          &=p''(1)+p'(1) -p'(1)^2\\
          &=d^2 \sum_{k=1}^{d-1} \frac{1}{k^2} -2dH_d +d^2H_d^2+1 +dH_d
            -d^2H_d^2\\
          &=d^2 \sum_{k=1}^{d} \frac{1}{k^2} -dH_d.\\
        \end{align*}

        So
        \[\sigma_d = \sqrt{d^2 \sum_{k=1}^{d} \frac{1}{k^2} -dH_d}.\]
      \end{proof}

    \item If there are 10 different kinds of pictures, how many boxes of
      cereal would you expect to have to buy in order to collect all 10?

      \begin{proof}
        This is $p'(1)=10H_{10}$, as proven in part (c).
      \end{proof}
  \end{enumerate}
\end{document}
