\documentclass{article}
\usepackage[left=3cm,right=3cm,top=3cm,bottom=3cm]{geometry}
\usepackage{amsmath,amssymb,amsthm,tikz,mathtools}
\usepackage{color}
\usepackage[inline]{enumitem}
\usetikzlibrary{patterns}
\setlength{\parindent}{0mm}
\newcommand{\TODO}[1]{\textcolor{red}{TODO: #1}}

\begin{document}
\title{Discrete Mathematics: Problem Set 5}
\author{Li Ling Ko\\ lko@nd.edu}
\date{\today}
\maketitle

\it \textbf{Section 50 Problem 1}: Let $\mathcal{F}$ be an intersecting set
  system on ground set $[n]$. Show that there exists an intersecting set
  system $\mathcal{F}'$ on $[n]$ with $|\mathcal{F}'|=2^{n-1}$ and with
  $\mathcal{F}\subseteq\mathcal{F}'$.

  \begin{proof}
    Partition $\mathcal{B}_n$ into $2^{n-1}$ pairs of the form
    $(A,[n]\setminus A)$, and remove all pairs that contain an element of
    $\mathcal{F}$. Observe that there would be exactly
    $m:=2^{n-1}-|\mathcal{F}|$ pairs left, because any pair that contains an
    element of $\mathcal{F}$ cannot contain more than one element of
    $\mathcal{F}$ since complements do not intersect. Fix any enumeration
    of the remaining pairs
    \[(A_1,[n]\setminus A_1), \ldots, (A_m, [n]\setminus A_m).\]

    We show that we can iteratively extend the $\mathcal{F}$ to a size of
    $2^{n-1}$ by adding exactly one element from each remaining pair to
    $\mathcal{F}$. Start with $\mathcal{F}_0=\mathcal{F}$. For each
    $i\in\{1,\ldots,m\}$, select an element from $(A_i,[n]\setminus A_i)$
    to add to $\mathcal{F}_{i-1}$ such that the resulting set
    $\mathcal{F}_i$ remains intersecting. Such an element must exist
    because if $A_i$ does not intersect some element $B$ in
    $\mathcal{F}_i$, then the complement $[n]\setminus A_i$ must contain
    $B$ and therefore intersect every element in $\mathcal{F}_i$ since $B$
    intersects every element in $\mathcal{F}_i$. Thus repeating this
    process $m$ times and setting $\mathcal{F}'=\mathcal{F}_m$ gives us an
    extension of $\mathcal{F}$ that is intersecting and that contains
    $2^{n-1}$ elements.
  \end{proof}

\it \textbf{Section 50 Problem 2}: Let $n=2m$ be even. Let
  $\mathcal{F}\subseteq\binom{[2m]}{m}$ be the set of all subsets of $[2m]$
  of size $m$ that include element 1. Let $\mathcal{F}'$ be an arbitrary
  subset (possibly empty) of $\mathcal{F}$. Let
  \[\mathcal{F}'' =\mathcal{F}' \cup \{[2m]\setminus A:
  A\in\mathcal{F}\setminus\mathcal{F}'\}.\]
  Show that $\mathcal{F}''$ is an intersecting family.

  \begin{proof}
    First, observe that since each element in $\mathcal{F}''$ is exactly of
    half the length of the number of elements in the ground set, any other
    element of the same length that is not its complement must intersect
    it. \\

    Let $A,B$ be sets in $\mathcal{F}''$. From the above argument it
    suffices to show that $A$ and $B$ are not complements. If both $A$ and
    $B$ belong in $\mathcal{F}'$, then they have a common element 1 and
    cannot be complements. If only one of the two sets, say $A$,
    belongs in $\mathcal{F}'$, then by definition $B$ is the complement of
    a set that is different from $A$, thus $A$ and $B$ cannot be
    complements. Finally, if neither $A$ nor $B$ belong in $\mathcal{F}'$,
    then neither of them contain 1, so they cannot be complements.
  \end{proof}

\it \textbf{Section 50 Problem 3}: Use the results of the last two
  questions to show that there are at least $2^{\binom{2m-1}{m-1}}$
  intersecting families on ground set $[n]$ that have size $2^{n-1}$.

  \begin{proof}
    Each of the intersecting sets $\mathcal{F}''$ of Problem 2 can be
    characterized by $\mathcal{F}'$, the subsets of $\mathcal{F}''$ that
    contain element 1. Now there are exactly $\binom{2m-1}{m-1}$ possible
    subsets of $[n]$ of size $m$ and containing 1. Therefore from Problem
    2, there are at least $2^{\binom{2m-1}{m-1}}$ intersecting families.
    Each of these families have size $\binom{2m-1}{m-1}$, which is smaller
    than $2^{2m-1}$ since $2^{2m-1}=\sum_{i=0}^{2m-1} \binom{2m-1}{i}$ by
    the binomial expansion. Then from Problem 1, each intersecting family
    can be extended to an intersecting family of size $2^{n-1}$. The
    extended families remain distinct since the original families are
    distinct. Therefore we have at least $2^{\binom{2m-1}{m-1}}$
    intersecting families of size $2^{n-1}$.
  \end{proof}

\it \textbf{Section 50 Problem 4}: An ideal is a set system $\mathcal{I}$
  on ground set $[n]$ that is closed under taking subsets. Prove that if
  $\mathcal{I}$ is an ideal, then each $i\in[n]$ appears in at most half of
  the members of $\mathcal{I}$.

  \begin{proof}
    Fix an $i\in[n]$. Partition $\mathcal{I}=\mathcal{A}\sqcup
    \mathcal{B}$, where $\mathcal{A}$ contains exactly those elements that
    that contain $i$. Then the set
    \[\mathcal{A}^*:=\{A\setminus\{i\}: A\in\mathcal{A}\}\]
    must be a subset of $\mathcal{B}$ since $\mathcal{I}$ is an ideal. Also
    $|\mathcal{A}^*|=|\mathcal{A}|$, therefore
    $|\mathcal{B}|\geq|\mathcal{A}|$ as we are required to show.
  \end{proof}

\it \textbf{Section 50 Problem 5}: Let $\mathcal{I}$ be an ideal on the
  ground set $[n]$, and let $\mathcal{I}'$ be the st of complements of
  members of $\mathcal{I}$. Prove that there is a bijection
  $f:\mathcal{I}\rightarrow\mathcal{I}'$ satisfying $A\subseteq f(A)$ for
  all $A\in\mathcal{I}$.

  \begin{proof}
    We can assume $|\mathcal{I}|>0$ otherwise the result is trivial.
    Following the hint, first fix any element $i\in[n]$ that appears in at
    least one element of $\mathcal{I}$; we can assume without loss of
    generality that this element is 1. Then partition
    \[\mathcal{I}=\mathcal{A}\sqcup\mathcal{A}^*\sqcup\mathcal{B},\]
    where
    \[\mathcal{A} :=\{X\in\mathcal{I}: 1\in X\},\]
    and
    \[\mathcal{A}^* :=\{X\setminus\{i\}: X\in\mathcal{A}\}.\]

    Now observe that both $\mathcal{A}^*$ and
    $\mathcal{A}^*\sqcup\mathcal{B}$ are ideals. So since
    $|\mathcal{A}|=|\mathcal{A}^*|$, the number of elements in
    $\mathcal{A}^*$ or in $\mathcal{A}^*\sqcup\mathcal{B}$ is strictly less
    than $|\mathcal{I}|$. Thus by induction on the length of ideals, these
    ideals induce respective bijections $f_{\mathcal{A}^*}$ and
    $f_{\mathcal{A}^*\sqcup\mathcal{B}}$ satisfying the assertion given in
    the question. \\

    %Also observe that the map
    %$f_{\mathcal{A}^*}:\mathcal{A}^*\rightarrow\mathcal{A}^{*'}$ defined by
    %\[f_{\mathcal{A}^*}(X) :=f_{\mathcal{A}}(X\cup\{i\})\setminus\{i\}\]
    %gives a bijection which satisfies the containment criteria $X\subseteq
    %f_{\mathcal{A}^*}(X)$, because $f_{\mathcal{A}}$ satisfies the same
    %criteria. \\

    Consider the map $f:\mathcal{I}\rightarrow\mathcal{I}'$ defined by
    \begin{align*}
      f(X) :=
      \begin{cases}
        f_{\mathcal{A}^*}(X\setminus\{1\}) &\text{if}\; X\in \mathcal{A},\\
        f_{\mathcal{A}^*\sqcup\mathcal{B}}(X) &\text{if}\; X\in
          \mathcal{A}^*\sqcup\mathcal{B}\; \text{and}\;
          f_{\mathcal{A}^*\sqcup\mathcal{B}}(X) \in\mathcal{B},\\
        f_{\mathcal{A}^*\sqcup\mathcal{B}}(X)\setminus\{1\} &\text{if}\; X\in
          \mathcal{A}^*\sqcup\mathcal{B}\; \text{and}\;
          f_{\mathcal{A}^*\sqcup\mathcal{B}}(X) \in\mathcal{A}^*.\\
      \end{cases}
    \end{align*}
  \end{proof}
\end{document}
