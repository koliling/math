\documentclass{article}
\usepackage[left=3cm,right=3cm,top=0cm,bottom=2cm]{geometry} % page settings
\usepackage{amsmath,amssymb,amsthm}

\setlength{\parindent}{0mm}

%\renewcommand{\baselinestretch}{1.05}
%\usepackage{verbatim,amsfonts,amscd, graphicx}
%\usepackage{graphics}
%\topmargin0.0cm
%\headheight0.0cm
%\headsep0.0cm
%\oddsidemargin0.0cm
%\textheight23.0cm
%\textwidth16.5cm
%\footskip1.0cm
%\theoremstyle{plain}
%\newtheorem{theorem}{Theorem}
%\newtheorem{corollary}{Corollary}
%\newtheorem{lemma}{Lemma}
%\newtheorem{proposition}{Proposition}
%\newtheorem*{surfacecor}{Corollary 1}
%\newtheorem{conjecture}{Conjecture} 
%\newtheorem{question}{Question} 
%\theoremstyle{definition}
%\newtheorem{definition}{Definition}

\begin{document}

\title{Basic Logic I: Homework 1}
\author{Li Ling Ko\\ lko@nd.edu}
\date{\today}
\maketitle

\section{Prove $(x,y) = (u,v) \Leftrightarrow x = u \, \& \, y = v$.}
\begin{proof}
  The left implication follows immediately from the definition of $(,)$ and
  the axiom of extensionality. \\

  To show the right implication, assume $(x,y) = (u,v)$. This means the set
  $\{\{x\}, \{x,y\}\}$ has the same elements as the set $\{\{u\}, \{u,v\}\}$.
  From extensionality, there are only two possible cases - either $\{x\} =
  \{u\}$ and $\{x,y\} = \{u,v\}$, or $\{x\} = \{u,v\}$ and $\{x,y\} = \{u\}$.
  In the former, we first get $x = u$ from $\{x\} = \{u\}$, then applying
  extensionality and $x=u$ on $\{x,y\} = \{u,v\}$, we get $y=v$ as required.
  \\

  In the latter, $\{x\}$ has one element while $\{u,v\}$ has at most two, so
  from extensionality, since both sets are equal, we get $x=u=v$. Similarly,
  from $\{x,y\} = \{u\}$, we get $x=y=u$. In particular, $x=u$ and $y=v$ as
  required. \\
\end{proof}


\end{document}
