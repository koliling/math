\documentclass{article}
\usepackage[left=3cm,right=3cm,top=3cm,bottom=3cm]{geometry}
\usepackage{amsmath,amssymb,amsthm}
\usepackage{color}
\setlength{\parindent}{2mm}

\newcommand{\TODO}[1]{\textcolor{red}{TODO: #1}}

\begin{document}
\title{Geometry and Topology: Homework 1}
\author{Li Ling Ko\\ lko@nd.edu}
\date{\today}
\maketitle

\begin{enumerate}
  \item
    \begin{proof}
      We first prove the forward implication. Fix $\epsilon>0$ and $p\in X$.
      By assumption, the set $U=f^{-1}(B_\epsilon(f(p)))\subseteq X$ is open.
      By choice of $U$, it must contain $p$. Then there must exist
      some $\delta>0$ such that the $d_Y$ ball $B_\delta(p)$ is contained in
      $U$. This $\delta$ satisfies the requirement. \\

      For the converse, assume the second definition. Let $U\subseteq Y$ be
      open, and $x\in f^{-1}(U)$. We want to find an open set in $f^{-1}(U)$
      that contains $x$. Since $U$ is open, $f(x)$ must be contained in a
      ball $B_\epsilon(f(x))\subseteq U$ for some $\epsilon>0$. Let $\delta$
      be the one given by the assumption. Then $B_\delta(x)$ satisfies the
      requirements for the open set we need.
    \end{proof}
  \item
    \begin{proof}
      Since $\pi':X\rightarrow Y'$ is a continuous function such that
      $\pi'(p)\sim\pi'(q)$ for all $p,q\in X$ satisfying $p\sim q$, we can
      replace the role of $Z$ by $Y'$ in the first set of properties, and
      get a continuous function $\vec{f}:Y\rightarrow Y'$ such that
      $\pi'=\vec{f}\circ\pi$. By symmetrical argument, there is also a
      continuous function $\vec{f'}:Y'\rightarrow Y$ such that
      $\pi=\vec{f'}\circ\pi'$. We show that
      $\vec{f}\circ\vec{f'}=\text{id}$, and
      $\vec{f'}\circ\vec{f}=\text{id}$. It suffices to prove the first
      equality since the argument for the second equality will be
      symmetrical. Let $\pi(x)\in Y$. By uniqueness of $\vec{f}$, $\vec{f}$
      must send $\pi(x)$ to $\pi'(x)$. Then by uniqueness of $\vec{f'}$,
      $\vec{f'}$ must send $\pi'(x)$ back to $\pi(x)$. Hence
      $\vec{f}\circ\vec{f'}=\text{id}$.
    \end{proof}
  \item
    \begin{enumerate}
      \item
        \begin{proof}
          We show that $\mathbb{R}^2/\sim$ is homeomorphic to the real line
          $\mathbb{R}$, by proving that the map
          $f:\mathbb{R}^2/\sim\rightarrow\mathbb{R}$ defined by
          $[(r,0)]\mapsto r$ is a homeomorphism. \\

          We first show that this map is well-defined. The equivalence
          class $[(r,0)]$ comprises all solutions to the quadratic curve
          $x+y^2=r$. So if $r_1\neq r_2\in\mathbb{R}$, then $(r_1,0)$ and
          $(r_2,0)$ belong to distinct equivalence classes. Also, every
          $(a,b)\in\mathbb{R}^2$ belongs to class $[(a+b^2,0)]$. Hence, map
          $f$ is well-defined. \\

          Map $f$ is clearly bijective, with inverse $f^{-1}$ defined by
          $f^{-1}(r)=[(r,0)]$. It remains to show that $f$ and $f^{-1}$ are
          continuous. We first show that $f$ is continuous. Given an open
          interval $(r_1,r_2)\in\mathbb{R}$, we need to show that its
          pre-image $\{[(r,0)]:r_1<r<r_2\}$ is open in $\mathbb{R}^2/\sim$.
          This is the same as showing that $\{(x,y):r_1<x+y^2<r_2\}$ is
          open in $\mathbb{R}^2$, which is true. \\

          Finally, we show that $f^{-1}$ is continuous, which is equivalent
          to showing that $f$ sends open sets to open sets. Let
          $\bar{U}=\{[(r_i,0)]:i\in I\}$ be an open set in
          $\mathbb{R}^2/\sim$. Then $f(U)=\{r_i:i\in I\}$. Let $r\in f(U)$.
          We want to show that there is an open interval
          $B_\epsilon(r)\subset\mathbb{R}$ that is contained in $f(U)$.
          Since $\bar{U}$ is open in $\mathbb{R}^2/\sim$,
          $U=\{(x,y):x+y^2\in\{r_i:i\in I\}\}$ must be open in
          $\mathbb{R}^2$. Also, $(r,0)\in U$, so there must be a ball
          $B_\epsilon((r,0))\subset\mathbb{R}^2$ containing $(r,0)$ in $U$.
          The image of this ball under $f$ is an open interval
          $B_\epsilon(r)\subset\mathbb{R}$ that is contained in $f(U)$ and
          that contains $r$, which is the interval we are looking for.
        \end{proof}
      \item
        \begin{proof}
          We show that $\mathbb{R}^2/\sim$ is homeomorphic to the half real
          line $R=\mathbb{R}^+\cup\{0\}$, by proving that the map
          $f:\mathbb{R}^2/\sim\rightarrow R$ defined by
          $[(r,0)]\mapsto |r|$ is a homeomorphism. \\

          We first show that this map is well-defined. The equivalence
          class $[(r,0)]$ comprises all solutions to the circle
          $x^2+y^2=r^2$. So if $r_1\neq r_2\in\mathbb{R}$, then $(r_1,0)$
          and $(r_2,0)$ belong to distinct equivalence classes. Also, every
          $(a,b)\in\mathbb{R}^2$ belongs to class $[(a^2+b^2,0)]$. Hence,
          map $f$ is well-defined. \\

          Map $f$ is clearly bijective, with inverse $f^{-1}$ defined by
          $f^{-1}(r)=[(r,0)]$. It remains to show that $f$ and $f^{-1}$ are
          continuous. We first show that $f$ is continuous. Given an open
          interval $(r_1,r_2)\in R$ (or $(0,r_2)\in R$), we need to show
          that its pre-image $\{[(r,0)]:r_1<r<r_2\}$ (or $\{[(r,0)]:0\leq
          r<r_2\}$) is open in $R$. This is the same as showing that
          $\{(x,y):r_1<x^2+y^2<r_2\}$ (or $\{(x,y):x^2+y^2<r_2\}$) is open
          in $\mathbb{R}^2$, which is true. \\

          Finally, we show that $f^{-1}$ is continuous, which is equivalent
          to showing that $f$ sends open sets to open sets. Let
          $\bar{U}=\{[(r_i,0)]:i\in I\}$ be an open set in
          $\mathbb{R}^2/\sim$. Then $f(U)=\{|r_i|:i\in I\}$. Let $r\in f(U)$.
          We want to show that there is an open interval
          $B_\epsilon(r)\subset R$ that is contained in $f(U)$.
          Since $\bar{U}$ is open in $\mathbb{R}^2/\sim$,
          $U=\{(x,y):x^2+y^2\in\{r_i^2:i\in I\}\}$ must be open in
          $\mathbb{R}^2$. Also, $(r,0)\in U$, so there must be a ball
          $B_\epsilon((r,0))\subset\mathbb{R}^2$ containing $(r,0)$ in $U$.
          The image of this ball under $f$ is an open interval
          $B_\epsilon(r)\subset\mathbb{R}$ that is contained in $f(U)$ and
          that contains $r$, which is the interval we are looking for.

        \end{proof}
    \end{enumerate}
\end{enumerate}
\end{document}
