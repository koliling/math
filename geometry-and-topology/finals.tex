\documentclass{article}
\usepackage[left=3cm,right=3cm,top=3cm,bottom=3cm]{geometry}
\usepackage{amsmath,amssymb,amsthm,tikz,mathtools}
\usepackage[inline]{enumitem}
\usetikzlibrary{patterns}
\usepackage{color}
\setlength{\parindent}{2mm}
\newcommand{\TODO}[1]{\textcolor{red}{TODO: #1}}
\newtheorem{prob}{Problem}

\begin{document}
\title{Geometry and Topology: Final Exam}
\author{Li Ling Ko\\ lko@nd.edu}
\date{\today}
\maketitle

\begin{enumerate}[label={\bf Q\arabic*:}]
  \item \it Define $\mathbb{C}\mathbb{P}^n$ to be the set of equivalence
    classes of $\mathbb{C}^{n+1}\setminus\{0\}$ under the equivalence
    relation that identifies $x$ and $\lambda x$ for all
    $x\in\mathbb{C}^{n+1}\setminus\{0\}$ and
    $\lambda\in\mathbb{C}\setminus\{0\}$. Endow $\mathbb{C}\mathbb{P}^n$
    with the quotient topology, so a set $U\subset\mathbb{C}\mathbb{P}^n$
    is open if and only if the preimage of $U$ in
    $\mathbb{C}^{n+1}\setminus\{0\}$ under the natural surjection
    $\mathbb{C}^{n+1}\setminus\{0\}\mapsto\mathbb{C}\mathbb{P}^n$ is open.
    Problem: prove that $\mathbb{C}\mathbb{P}^n$ is a smooth orientable
    manifold of dimension $2n$.

    \begin{proof}
    \end{proof}

  \item

  \item \it Let $M^n$ be an arbitrary smooth manifold. Prove that the tangent
    bundle $TM^n$ is orientable.

    \begin{proof}
      The set 
      \[\mathcal{A} = \{T\phi:TU\rightarrow V\times\mathbb{R}^n|\;
        \phi:U\rightarrow V\; \text{a chart on}\; M^n\}\]
      has been shown to be a smooth atlas on $TM^n$ (Lemma 2.6), therefore
      the tangent bundle is a smooth manifold. So given any local chart
      $(U,\phi)$ of $M^n$, the corresponding chart on $TM^n$ is given by
      $(TU,T\phi)$, where $T\phi(p,v)=(\phi(p),d\phi(v))$. Let
      $(TU_0,T\phi_0)$ be another chart for $TM$. Then we have
      \[T\phi\circ T\phi_0^{-1}(x_1,\ldots,x_{2n})
        =(\phi\circ\phi_0^{-1}(x_1,\ldots,x_n),\; d(\phi\circ\phi_0^{-1})
        (x_{n+1},\ldots,x_{2n})),\]
      and so
      \[d(T\phi\circ T\phi_0^{-1})(x_{1},\ldots,x_{2n})
        =(d(\phi\circ\phi_0^{-1})(x_{1},\ldots,x_{n}),\;
        d(\phi\circ\phi_0^{-1})(x_{n+1},\ldots,x_{2n})).\]

      Then, the eigenvectors of $d(T\phi\circ T\phi_0^{-1})$ are $(0,v_i)$
      and $(v_1,0)$, where the $v_i$ are eigenvectors of
      $d(\phi\circ\phi_0^{-1})$. Hence each eigenvalue of
      $d(\phi\circ\phi_0^{-1})$ occurs twice in the eigenvalues of
      $d(T\phi\circ T\phi_0^{-1})$, which implies that
      \[\text{det}(d(T\phi\circ T\phi_0^{-1}))
      =\text{det}(d(\phi\circ\phi_0^{-1}))^2 >0.\]

      Thus for every change of coordinate function, the determinant of its
      Jacobian matrix is positive, which is equivalent to being orientable.
    \end{proof}

  \item \it Let $G$ be a Lie group, that is, a group $G$ that is also a smooth
    manifold such that the multiplication map $M\times M\rightarrow M$
    taking $(x,y)$ to $xy$ and the inversion map $M\rightarrow M$ taking
    $x$ to $x^{-1}$ are smooth. Letting $n$ be the dimension of $G$, prove
    that $TG\cong G\times\mathbb{R}^n$.

    \begin{proof}
      Given $g\in G$, define the left multiplication function
      $L_g:G\rightarrow G$ by $L_g(x)=gx$. Then $L_g$ is a diffeomorphism,
      and $L_g(e)=ge=g$, where $e$ is the group identity. Thus $L_g$
      induces a map on the tangent space $dL_g:T_eG\rightarrow T_gG$.  \\

      Let $v_1,\ldots,v_n$ be a basis of the tangent space $T_eG$. Then the
      $n$ vector fields $dL_g(v_1),\ldots,dL_g(v_n)\in T_gG$ are linearly
      independent at $g\in M$: First, note that each $dL_g(v_i)$ is a smooth
      vector field from the smoothness of the group operations. Now since
      $L_g$ is a diffeomorphism, $dL_g$ is an isomorphism. Then since
      $v_1,\ldots,v_n$ form a basis of $T_eG$, their images
      $dL_g(v_1),\ldots,dL_g(v_n)$ form a basis of $T_gG$. \\

      So the map $\phi:G\times \mathbb{R}^n\rightarrow TG$ defined by
      $\phi((g,V))=(g,dL_g)_e(V)$ is a smooth vector-bundle isomorphism,
      thus $TG\cong G\times\mathbb{R}^n$.
    \end{proof}

  \item

  \item
\end{enumerate}
\end{document}
