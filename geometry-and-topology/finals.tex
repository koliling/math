\documentclass{article}
\usepackage[left=3cm,right=3cm,top=3cm,bottom=3cm]{geometry}
\usepackage{amsmath,amssymb,amsthm,tikz,mathtools}
\usepackage[inline]{enumitem}
\usetikzlibrary{patterns}
\usepackage{color}
\setlength{\parindent}{2mm}
\newcommand{\TODO}[1]{\textcolor{red}{TODO: #1}}
\newtheorem{prob}{Problem}

\begin{document}
\title{Geometry and Topology: Final Exam}
\author{Li Ling Ko\\ lko@nd.edu}
\date{\today}
\maketitle

\begin{enumerate}[label={\bf Q\arabic*:}]
  \item \it Define $\mathbb{C}\mathbb{P}^n$ to be the set of equivalence
    classes of $\mathbb{C}^{n+1}\setminus\{0\}$ under the equivalence
    relation that identifies $x$ and $\lambda x$ for all
    $x\in\mathbb{C}^{n+1}\setminus\{0\}$ and
    $\lambda\in\mathbb{C}\setminus\{0\}$. Endow $\mathbb{C}\mathbb{P}^n$
    with the quotient topology, so a set $U\subset\mathbb{C}\mathbb{P}^n$
    is open if and only if the preimage of $U$ in
    $\mathbb{C}^{n+1}\setminus\{0\}$ under the natural surjection
    $\mathbb{C}^{n+1}\setminus\{0\}\mapsto\mathbb{C}\mathbb{P}^n$ is open.
    Problem: prove that $\mathbb{C}\mathbb{P}^n$ is a smooth orientable
    manifold of dimension $2n$.

    \begin{proof}
      We first prove that $\mathbb{C}\mathbb{P}^n$ is Hausdorff. Define the
      map $f:(\mathbb{C}^{n+1}\setminus\{0\})
      \times(\mathbb{C}^{n+1}\setminus\{0\}) \rightarrow\mathbb{R}$ by
      \[f(\bar{x},\bar{y}) = \sum_{i\neq j}\left|x_iy_j-x_jy_i\right|^2.\]
      Then $f(\bar{x},\bar{y})=0$ if and only if $\bar{x}$ and $\bar{y}$
      are in the same equivalence class. Thus $\mathbb{C}\mathbb{P}^n
      =f^{-1}(0)$. Note that $f$ is continuous and $\{0\}$ is closed in
      $\mathbb{R}$. Now from point-set topology, any quotient space
      $X/\sim$ of a Hausdorff space $X$ where the projection map
      $\pi:X\rightarrow X/\sim$ is open will also be Hausdorff if and only
      if $\{(x,y)\in X\times X:x\sim y\}$ is closed in $X\times X$. Thus
      $\mathbb{C}\mathbb{P}^n$ is Hausdorff. \\

      Next we prove that $\mathbb{C}\mathbb{P}^n$ is second countable. This
      projective space is an open quotient subspace of $\mathbb{R}^{2n}$,
      where the projection map is open. Now the quotient space of a second
      countable space with open projection map is also second countable.
      Then since $\mathbb{R}^{2n}$ is second countable, the complex
      projective space is also second countable. \\

      Next we show that the complex projective space is a smooth manifold
      of dimension $2n$. Define charts on $\mathbb{C}\mathbb{P}^n$ as
      follows for each $i\in\{0,\ldots,n\}$:
      \[U_i= \{[x_0:\ldots:x_n]\in\mathbb{C}\mathbb{P}^n\}\]
      Then the function
      \[\phi_i:U_i\rightarrow\mathbb{C}^n\quad [x_0:\ldots:x_n]\mapsto
        \left(\frac{x_0}{x_i},\ldots,\frac{x_{i-1}}{x_i},
        \frac{x_{i+1}}{x_i},\ldots,\frac{x_n}{x_1}\right)\]
      is a homeomorphism since it has a continuous inverse
      \[\phi_i^{-1}:\mathbb{C}^n\rightarrow U_i\quad
        (x_1,\ldots,x_n)\mapsto[x_1:\ldots:x_{i-1}:x_{i+1}\ldots:x_n].\]
      Consider the change of coordinates $\phi_i\circ\phi_j^{-1}$, where we
      can assume without loss of generality that $i<j$,
      \[(x_1,\ldots,x_n)\mapsto [x_1:\ldots:x_{i-1}:x_{i+1}\ldots:x_n]
        \mapsto\left(\frac{x_1}{x_i},\ldots,
        \frac{x_{i-1}}{x_i},\frac{x_{i+1}}{x_i},
        \ldots,\frac{x_{j-1}}{x_i},\frac{1}{x_i},\frac{x_{j+1}}{x_i},
        \ldots,\frac{x_n}{x_1}\right).\]
      This is a smooth function, thus $\phi$ gives a smooth atlas for the
      complex projective space.

      To show that the manifold is orientable, we show that for every
      change of coordinate function, the determinant of its Jacobian matrix
      is positive. Let $(U,\phi)$ be a local chart of
      $\mathbb{C}\mathbb{P}^n$. We can assume $\phi$ is a function from $U$
      to $\mathbb{C}^n$. Let $(U_0,\phi_0)$ be another chart for $TM$.
      Similarly, we can consider $\phi_0$ as a function from $U_0$ to
      $\mathbb{C}^n$. We need to show that
      $\text{det}(d(\phi\circ\phi_0^{-1}))>0$. Now given a complex vector
      space $V$ with basis $B=(u_1,\ldots,u_n)$ over $\mathbb{C}$,
      $B^*=(u_1,iu_1,\ldots,u_n,iu_n)$ will be a basis over $\mathbb{R}$.
      Thus if $B_0$ is another basis over $\mathbb{C}$ corresponding to the
      chart $(U_0,\phi_0)$ and $l=d(\phi\circ\phi_0^{-1})$ is the unique
      $\mathbb{C}$-linear map such that $lB=B_0$, then we have
      $lB^∗=B^∗_0$, and the determinant of $l$, viewed as a
      $\mathbb{R}$-linear map, will be
      $\text{det}(d(\phi\circ\phi_0^{-1}))=|l|^2>0$, as required.

      %Given $t\in\mathbb{C}\setminus\{0\}$, consider the scaling map
      %$\phi_t:\mathbb{C}^{n+1}\setminus\{0\}\rightarrow
      %\mathbb{C}^{n+1}\setminus\{0\}$ defined by $\phi_t(x)=tx$. This map
      %is invertible with inverse $\phi_{1/t}$. Then since $\phi_t$ and
      %$\phi_{1/t}$ are analytic, $\phi_t$ is a homeomorphism. Hence if
      %$U\subset\mathbb{C}^{n+1}$ is open, its image
      %$\phi_t(U)\subset\mathbb{C}^{n+1}$ will also be open.

      %We can realize the complex projective space in a similar way that
      %the real projective space was realized in the notes (Basic example in
      %Section 1.2): $\mathbb{C}\mathbb{P}^n$ is the quotient
      %$S^{2n+1}/S^1$. For $1\leq i\leq 2n+1$, define
      %$U_i\subset\mathbb{C}\mathbb{P}^n$ to be the image of
      %$U_{x_i>0}\subset S^{2n+1}$ under the quotient map
      %$S^{2n+1}\rightarrow\mathbb{C}\mathbb{P}^n$. Letting $V$ be the unit
      %disk in $\mathbb{R}^{2n}$, we can define homeomorphisms
      %$\phi_i:U_i\rightarrow V$ as the composition
      %\[U_i\cong U_{x_i>0}\xrightarrow{\phi_{x_i>0}} V.\]
      %Then the set
      %\[\mathcal{A}=\{\phi_i:U_i\rightarrow V\}\]
      %forms a smooth atlas for $\mathbb{C}\mathbb{P}^n$. Thus, the
      %complex projective space is a smooth manifold of dimension $2n$. \\

      %We know that the real projective space $\mathbb{R}\mathbb{P}^n$ is a
      %smooth manifold of dimension $n$, with coordinate chart at
      %$(x_0:\ldots:x_n)\in\mathbb{R}\mathbb{P}^n$ given by
      %\[\phi_i(x_0:\ldots:x_n)
      %  =\left(\frac{x_0}{x_i},\ldots,\frac{x_n}{x_i}\right),\]
      %where the domain of $\phi_i$ is
      %\[U_i=\{(x_0:\ldots:x_n):x^i\neq0\}.\]
      %The map in this chart is continuous because the quotient map composed
      %with it is continuous. Also, $\phi_i(U_i)=\mathbb{R}^n$ and the
      %inverse
      %\[\phi_i^{-1}(y_1,\ldots,y_n)=
      %  (y_1:\ldots:y_{i-1}:y_{i+1}:\ldots:y_n)\]
      %is continuous. Thus the charts are homeomorphisms, and will give a
      %smooth atlas for the real projective space. \\
    \end{proof}

  \item \it Let $f:M_1^{n_1}\rightarrow M_2^{n_2}$ be a smooth map between
    smooth manifolds. Also, let $i:X^k\xhookrightarrow{}M_2^{n_2}$ be an
    embedding of a smooth manifold $X^k$ into $M_2^{n_2}$. Assume that for
    all $p\in M_1^{n_1}$ and $x\in X^k$ with $f(p)=i(x)$, the span of the
    images of $D_pf:T_pM_1^{n_1}\rightarrow T_{f(p)}M_2^{n_2}$ and
    $D_xi:T_xX^k\rightarrow T_{i(x)}M_2^{n_2}$ is all of
    $T_{f(p)}M_2^{n_2}=T_{i(x)}M_2^{n_2}$. Prove that the set
    $f^{-1}(i(X^k))\subset M_1^{n_1}$ is a smooth
    $(n_1-(n_2-k))$-dimensional submanifold of $M_1^{n_1}$. Hint: One
    special case of this is where $X^k$ is a single point $x$ and $i(x)$ is
    a regular value of $f$. Use the local immersion theorem to reduce this
    (locally) to this special case.

    \begin{proof}
      Given arbitrary $x\in X^k$ with $i(x)\in f(M_1^{n_1})$, there exist
      by the immersion theorem, an open $U_x\subset X^k$ containing $x$, an
      open $U_2\subset M_2^{n_2}$, an open $V\subset\mathbb{R}^{n_2-k}$,
      $v_0\in V$, and a diffeomorphism $\phi:U_2\rightarrow U_x\times V$
      such that $\phi(i(u))=(u,v_0)$ for all $u\in U_x$. Then because
      $\phi$ is bijective, we have
      \[\phi^{-1}(U_x\times\{v_0\})=i(U_x).\]

      Consider the map $\varphi:f^{-1}(U_2)\rightarrow V$ given by
      $\varphi(p)=\pi_2(\phi(f(p)))$, where $\pi_n$ is the $n$th-projection
      map. Note that $\varphi$ is smooth because it is a composition of
      smooth functions. We show that $v_0$ is a regular value of $\varphi$.
      Note that $\varphi^{-1}(v_0)$ is non-empty because
      $\phi^{-1}(U_x\times\{v_0\})=i(U_x)$. So let $p\in\varphi^{-1}(v_0)$,
      and $y\in U_x$ such that $f(p)=i(y)$. Then the chain rule gives
      \begin{align*}
        D_p\varphi(T_pM_1^{n_1}) &=D_p(\pi_2\circ\phi\circ f)(T_pM_1^{n_1})
          \\
        &=D_{f(p)}(\pi_2\circ\phi)(D_pf(T_pM_1^{n_1})). \\
      \end{align*}

      Also, since both $\pi_2$ and $\phi$ are submersions, their
      composition is also a submersion, thus
      \[D_{f(p)}(\pi_2\circ\phi)(T_{f(p)}M_2^{n_2}) = T_{v_0}V.\]

      Furthermore, because $\pi_2\circ\phi\circ i$ is a constant map, we
      have
      \[D_y(\pi_2\circ\phi\circ i)(T_yX^k)
        =D_{f(p)}(\pi_2\circ\phi)(D_yi(T_yX^k)) =0,\]
      where $y\in U_x$ is such that $f(p)=i(y)$.

      Thus, from the assumption in the question, we have
      \[D_p\varphi(T_pM_1^{n_1})=
      D_{f(p)}(\pi_2\circ\phi)(D_pf(T_pM_1^{n_1})) =T_{v_0}W,\]
      and thus $v_0$ is a regular value of $\varphi$. \\

      Then by the regular value theorem,
      \[\varphi^{-1}(v_0) =f^{-1}\circ\phi^{-1}\circ\pi_2^{-1}(v_0)
      =f^{-1}(i(U_x))\]
      is a smooth $(n_1-(n_2-k))$-dimensional submanifold of $M_1^{n_1}$.
      Therefore
      \[f^{-1}(i(X^k))=\bigcup_{x\in X^k} f^{-1}(i(U_x))\]
      will also be a smooth $(n_1-(n_2-k))$-dimensional submanifold of
      $M_1^{n_1}$.
    \end{proof}

  \item \it Let $M^n$ be an arbitrary smooth manifold. Prove that the tangent
    bundle $TM^n$ is orientable.

    \begin{proof}
      The set 
      \[\mathcal{A} = \{T\phi:TU\rightarrow V\times\mathbb{R}^n|\;
        \phi:U\rightarrow V\; \text{a chart on}\; M^n\}\]
      has been shown to be a smooth atlas on $TM^n$ (Lemma 2.6), therefore
      the tangent bundle is a smooth manifold. So given any local chart
      $(U,\phi)$ of $M^n$, the corresponding chart on $TM^n$ is given by
      $(TU,T\phi)$, where $T\phi(p,v)=(\phi(p),d\phi(v))$. Let
      $(TU_0,T\phi_0)$ be another chart for $TM$. Then we have
      \[T\phi\circ T\phi_0^{-1}(x_1,\ldots,x_{2n})
        =(\phi\circ\phi_0^{-1}(x_1,\ldots,x_n),\; d(\phi\circ\phi_0^{-1})
        (x_{n+1},\ldots,x_{2n})),\]
      and so
      \[d(T\phi\circ T\phi_0^{-1})(x_{1},\ldots,x_{2n})
        =(d(\phi\circ\phi_0^{-1})(x_{1},\ldots,x_{n}),\;
        d(\phi\circ\phi_0^{-1})(x_{n+1},\ldots,x_{2n})).\]

      Then, the eigenvectors of $d(T\phi\circ T\phi_0^{-1})$ are $(0,v_i)$
      and $(v_i,0)$, where the $v_i$ are eigenvectors of
      $d(\phi\circ\phi_0^{-1})$. Hence each eigenvalue of
      $d(\phi\circ\phi_0^{-1})$ occurs twice in the eigenvalues of
      $d(T\phi\circ T\phi_0^{-1})$, which implies that
      \[\text{det}(d(T\phi\circ T\phi_0^{-1}))
      =\text{det}(d(\phi\circ\phi_0^{-1}))^2 >0.\]

      Thus for every change of coordinate function, the determinant of its
      Jacobian matrix is positive, which is equivalent to being orientable.
    \end{proof}

  \item \it Let $G$ be a Lie group, that is, a group $G$ that is also a
    smooth manifold such that the multiplication map $G\times G\rightarrow
    G$ taking $(x,y)$ to $xy$ and the inversion map $G\rightarrow G$ taking
    $x$ to $x^{-1}$ are smooth. Letting $n$ be the dimension of $G$, prove
    that $TG\cong G\times\mathbb{R}^n$.

    \begin{proof}
      Given $g\in G$, define the left multiplication function
      $L_g:G\rightarrow G$ by $L_g(x)=gx$. Then $L_g$ is a diffeomorphism,
      and $L_g(e)=ge=g$, where $e$ is the group identity. Thus $L_g$
      induces a map on the tangent space $dL_g:T_eG\rightarrow T_gG$.  \\

      Let $v_1,\ldots,v_n$ be a basis of the tangent space $T_eG$. Then the
      $n$ vector fields $dL_g(v_1),\ldots,dL_g(v_n)\in T_gG$ are linearly
      independent at $g\in M$: First, note that each $dL_g(v_i)$ is a smooth
      vector field from the smoothness of the group operations. Now since
      $L_g$ is a diffeomorphism, $dL_g$ is an isomorphism. Then since
      $v_1,\ldots,v_n$ form a basis of $T_eG$, their images
      $dL_g(v_1),\ldots,dL_g(v_n)$ form a basis of $T_gG$. \\

      So the map $\phi:G\times \mathbb{R}^n\rightarrow TG$ defined by
      $\phi((g,V))=(g,dL_g)_e(V)$ is a smooth vector-bundle isomorphism,
      thus $TG\cong G\times\mathbb{R}^n$.
    \end{proof}

  \item \it Let $M^3$ be a 3-manifold and let $\alpha\in\Omega^1(M^3)$ be a
    1-form such that $\alpha\wedge d\alpha$ is a volume form on $M^3$.
    Prove that there does not exist an embedding $f:X^2\rightarrow M^3$
    with $X^2$ a 2-manifold such that for all $p\in X^2$ and all
    $\vec{v}\in T_pX^2$, we have
    \[\alpha(f(p))((D_pf)(\vec{v}))=0.\]

    \begin{proof}
      Assume by contradiction that such an embedding exists.
    \end{proof}

  \item \it Let $f_0:M^k\rightarrow X^n$ and $f_1:M^k\rightarrow X^n$ be
    smoothly homotopic maps between smooth manifolds and let
    $\omega\in\Omega^k(X^n)$ be closed. Assume that $M^k$ is oriented and
    compact. Prove that
    \[\int_{M^k} f_0^*(\omega)= \int_{M^k} f_1^*(\omega).\]
    Make sure to be careful about orientations in your proof!

    \begin{proof}
      Consider the manifold $M=M^k\times[0,1]$. The boundary on $M$
      is comprised of $M^k\times\{0\}$ and $M^k\times\{1\}$, where the
      former is counted with a minus sign and the latter with a plus sign.
      Thus if $\phi:M^k\times[0,1]\rightarrow X^n$ is a homotopy between
      $f_0$ and $f_1$, then we have
      \begin{align*}
        0 &=\int_{M^k\times[0,1]} \phi^*(d\omega) \\
        &=\int_{M^k\times[0,1]} d\phi^*(\omega) \\
        &=\int_{d(M^k\times[0,1])} \phi^*(\omega) &(\text{Stoke's theorem})
          \\
        &=\int_{M^k} f_1^*(\omega) - \int_{M^k} f_0^*(\omega), \\
      \end{align*}
      thus $\int_{M^k} f_0^*(\omega)= \int_{M^k} f_1^*(\omega)$ as
      required.
    \end{proof}
\end{enumerate}
\end{document}
