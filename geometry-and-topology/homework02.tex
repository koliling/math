\documentclass{article}
\usepackage[left=3cm,right=3cm,top=3cm,bottom=3cm]{geometry}
\usepackage{amsmath,amssymb,amsthm,pgfplots,tikz}
\usetikzlibrary{patterns}
\usepackage{color}
\setlength{\parindent}{2mm}
\newcommand{\TODO}[1]{\textcolor{red}{TODO: #1}}

\begin{document}
\title{Geometry and Topology: Homework 2}
\author{Li Ling Ko\\ lko@nd.edu}
\date{\today}
\maketitle

\begin{enumerate}
  \item Let $X$ be a CW complex.
    \begin{enumerate}
      \item Prove that if $X$ has finitely may cells, then $X$ is compact.
        \begin{proof}
          We prove by induction on $n$, the number of cells of $X$. For the
          base case where $n=1$, $X$ is a single point, which is trivially
          compact. For the inductive step, assume the statement is true for
          $n$ cells, and consider the case when $X$ is composed of $(n+1)$
          cells. $X$ is obtained from gluing an $n$-disc $D^n$ to $X^{(n)}$
          at the boundary of the disk $\delta D^n$. $X^{(n)}$ is compact by
          inductive assumption. The $n$-disc is also compact since it is
          closed and bounded in $\mathbb{R}^n$. Since $X$ is a compact
          space attached to another compact space, it must also be compact
          - any open cover of $X$ must cover each of its component
          spaces, giving a finite cover from each component space by the
          compactness of those spaces.
        \end{proof}

      \item Let $C\subset X$ be a compact subset. Prove that $C$ only
        intersects finitely many cells of $X$.
        \begin{proof}
          Assume by contradiction that a compact subset $C$ intersects
          infinite number of cells of $X$. Let $\{e^i\}_{i\in\mathbb{N}}$
          be a countably infinite series of distinct cells that intersect
          with $C$. From each cell $e^i$, we pick a point $x_i\in X$ from
          their area of intersection, to obtain a series $P=\{x_0,\ldots\}$
          of points in $X$. \\

          We first prove by induction that $P$ is closed in $X$. By
          definition, this is equivalent to proving that $P\cap X^{(n)}$ is
          closed in $X^{(n)}$ for each $n\in\mathbb{N}$. We prove this
          claim by induction on $n$: The base case is trivially true since
          it consists $P\cap X^{(0)}$ consists of only one point $x_0$.
          Assume the claim is true for $n$. Then $\varphi_{n+1}^{-1}(P)$ is
          closed in the disk boundary $\delta D^{n+1}$, so after adding one
          more point $x_{n+1}$ into $D^{n+1}$, $\varphi_{n+1}^{-1}(P)$ will
          remain closed in $D^{n+1}$, implying that $P\cap X^{n+1}$ is
          closed in $X^{n+1}$. \\

          By similar argument, any subset of $P$ must also be closed in
          $X$, which means that $P$ as a subset of $X$ has the discrete
          topology. Yet, $P$ is infinite, so it cannot be compact, which is
          a contradiction because $P$ is a closed subset of a compact space
          $C$, and should be compact.
        \end{proof}
    \end{enumerate}

  \item Construct CW complex structures on the following spaces.
    \begin{enumerate}
      \item An $n$-dimensional torus.
        \begin{proof}
          An $n$-torus is formed by first creating an $n$-dimensional cube,
          then gluing opposing faces of the cube together. We observe the
          pattern of number cells needed as $n$ increases from 0 to 2. The
          0-torus is a single 0-cell. The 1-torus is a 0-cell, glued with a
          1-cell. The 2-torus is a 0-cell, glued with two 1-cells, then
          with a 2-cell on the two one-cells. So the $n$-cell is obtained
          by starting with a 0-cell, then gluing $n$ 1-cells to the 0-cell,
          where the end points of each 1-cell is glued to the 0-cell,
          forming a flower with $n$ petals. To complete the $n$-cell, we
          build a $(n-1)$-torus over each $\binom{n}{n-1}$ choice of
          $(n-1)$ petals, reusing any $(n-i)$ cells that have been built
          earlier. \\

          This system of construction can be summarized as follows
          \begin{enumerate}
            \item Start with a single 0-cell.
            \item Glue $n$ 1-cells to the 0-cell, sticking the end points
              of each 1-cell to the 0-cell, forming a flower with $n$
              petals.
            \item Repeat for $i=2$ to $n$: Between every $i$
              1-cells, glue an $i$-cell to the $i$ $(i-1)$-cells
              constructed in the earlier stage, such that opposing faces of
              the $i$-cell are glued to the same $(i-1)$-cell, and each
              $(i-1)$-cell has exactly two $i$-cell faces glued onto them.
          \end{enumerate}

          From construction, we summarize that an $n$-torus is built from
          $\binom{n}{i}$ $i$-cells, for $0\leq i\leq n$. \\
        \end{proof}

      \item Letting $\{p_1,\ldots,p_n\}$ be $n$ distinct points on $S^2$,
        the quotient space of $S^2$ that identifies all the $p_i$ to a
        single point.

        \begin{proof}
          After trial and error, we conclude that the quotient space of
          $S^2$ is constructed using one 0-cell, $n$ 1-cells glued onto the
          0-cell through gluing the end points of each 1-cell onto the
          0-cell, forming a $n$-petaled flower, then finally gluing a
          single 2-cell glued onto the flower in the following manner:
        \end{proof}
    \end{enumerate}
\end{enumerate}
\end{document}
