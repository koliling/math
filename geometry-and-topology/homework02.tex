\documentclass{article}
\usepackage[left=3cm,right=3cm,top=3cm,bottom=3cm]{geometry}
\usepackage{amsmath,amssymb,amsthm,pgfplots,tikz}
\usetikzlibrary{patterns}
\usepackage{color}
\setlength{\parindent}{2mm}
\newcommand{\TODO}[1]{\textcolor{red}{TODO: #1}}

\begin{document}
\title{Geometry and Topology: Homework 2}
\author{Li Ling Ko\\ lko@nd.edu}
\date{\today}
\maketitle

\begin{enumerate}
  \item Let $X$ be a CW complex.
    \begin{enumerate}
      \item Prove that if $X$ has finitely may cells, then $X$ is compact.
        \begin{proof}
          We prove by induction on $n$, the number of cells of $X$. For the
          base case where $n=1$, $X$ is a single point, which is trivially
          compact. For the inductive step, assume the statement is true for
          $n$ cells, and consider the case when $X$ is composed of $(n+1)$
          cells. $X$ is obtained from gluing an $n$-disc $D^n$ to $X^{(n)}$
          at the boundary of the disk $\delta D^n$. $X^{(n)}$ is compact by
          inductive assumption. The $n$-disc is also compact since it is
          closed and bounded in $\mathbb{R}^n$. Since $X$ is a compact
          space attached to another compact space, it must also be compact
          - any open cover of $X$ must cover each of its component
          spaces, giving a finite cover from each component space by the
          compactness of those spaces.
        \end{proof}

      \item Let $C\subset X$ be a compact subset. Prove that $C$ only
        intersects finitely many cells of $X$.
        \begin{proof}
          Assume by contradiction that a compact subset $C$ intersects
          infinite number of cells of $X$. Let $\{e^i\}_{i\in\mathbb{N}}$
          be a countably infinite series of distinct cells that intersect
          with $C$. From each cell $e^i$, we pick a point $x_i\in X$ from
          their area of intersection, to obtain a series $P=\{x_0,\ldots\}$
          of points in $X$. \\

          We first prove by induction that $P$ is closed in $X$. By
          definition, this is equivalent to proving that $P\cap X^{(n)}$ is
          closed in $X^{(n)}$ for each $n\in\mathbb{N}$. We prove this
          claim by induction on $n$: The base case is trivially true since
          it consists $P\cap X^{(0)}$ consists of only one point $x_0$.
          Assume the claim is true for $n$. Then $\varphi_{n+1}^{-1}(P)$ is
          closed in the disk boundary $\delta D^{n+1}$, so after adding one
          more point $x_{n+1}$ into $D^{n+1}$, $\varphi_{n+1}^{-1}(P)$ will
          remain closed in $D^{n+1}$, implying that $P\cap X^{n+1}$ is
          closed in $X^{n+1}$. \\

          By similar argument, any subset of $P$ must also be closed in
          $X$, which means that $P$ as a subset of $X$ has the discrete
          topology. Yet, $P$ is infinite, so it cannot be compact, which is
          a contradiction because $P$ is a closed subset of a compact space
          $C$, and should be compact.
        \end{proof}
    \end{enumerate}

  \item Construct CW complex structures on the following spaces.
    \begin{enumerate}
      \item An $n$-dimensional torus.
        \begin{proof}
          An $n$-torus is formed by first creating an $n$-dimensional cube,
          and glueing opposing faces of the cube together. An
          $n$-dimensional cube has $2^n$ faces.

          We prove by induction on $n$ that an $n$ dimensional torus is
          constructed using $\binom ni$ $i$-cells, for $0\leq i\leq n$. For
          the base case, the $0$-torus is a single point, so the claim is
          trivially true.
        \end{proof}

      \item Letting $\{p_1,\ldots,p_n\}$ be $n$ distinct points on $S^2$,
        the quotient space of $S^2$ that identifies all the $p_i$ to a
        single point.
    \end{enumerate}
\end{enumerate}
\end{document}
