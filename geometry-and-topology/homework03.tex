\documentclass{article}
\usepackage[left=3cm,right=3cm,top=3cm,bottom=3cm]{geometry}
\usepackage{amsmath,amssymb,amsthm,pgfplots,tikz}
\usetikzlibrary{patterns}
\usepackage{color}
\setlength{\parindent}{2mm}
\newcommand{\TODO}[1]{\textcolor{red}{TODO: #1}}

\begin{document}
\title{Geometry and Topology: Homework 3}
\author{Li Ling Ko\\ lko@nd.edu}
\date{\today}
\maketitle

\begin{enumerate}
  \item Exercise 2.16: Let $T$ be the infinite 4-valent tree and let
    $\tilde{X}$ be any connected graph which is oriented and labeled as in
    Example 2.10. Also, let $X$ be the graph from Figure 2.3, so we have
    covering maps $f:T\rightarrow X$ and $g:\tilde{X}\rightarrow X$. Let
    $v$ be any vertex of $T$ and let $w$ be any vertex of $\tilde{X}$.
    Prove that there exists a covering map $h:T\rightarrow\tilde{X}$ such
    that $f=g\circ h$.

    \begin{proof}
      Let $h$ map the middle vertex $t_m$ in $T$ to any vertex $x_m$ in
      $\tilde{X}$. Then for other vertices $t$ in $T$, we find the unique
      path $p$ from $t_m$ to $t$, which can be represented as a word from
      $\{a,-a,b,-b\}^{<\omega}$. Then let $h$ map $t$ to the unique vertex
      $x$ in $\tilde{X}$ that is reached from starting at vertex $x_m$ and
      following the same path directions given by $p$. Then, given any edge
      from vertex $t_0$ to $t_1$ in $T$, let $h$ map the edge linearly in
      the direction from $h(t_0)$ to $h(t_1)$. This map is well-defined
      since there is only one path between any two vertices on $T$ and
      every vertex in $\tilde{X}$ has valence 4 with one $a$ (resp. $b$)
      edge entering and one $a$ (resp. $b$) edge exiting. \\

      Next, we show that $f=g\circ h$. The middle vertex is mapped to the
      only vertex in $X$, as required. For the remaining nodes, it is
      routine to show by induction on the path distance between a node $t$
      to the middle node $t_m$ that $g\circ h$ will map all vertices to the
      vertex in $X$, all $a$-edges to the $a$-edge in $X$, and $b$-edges to
      the $b$-edge in $X$, in the proper direction. \\

      Finally, we show that $h$ is a covering map. We note that the map
      is surjective from the connectivity of $\tilde{X}$ and the fact that
      every word from $\{a,-a,b,-b\}^{<\omega}$ can be reached on $T$
      starting from the middle vertex of $T$. Also, $h$ inherits continuity
      from $f$ and $g$. Furthermore, given any point on $\tilde{x}\in\tilde{X}$
      that is not a vertex, consider the open edge $e\subset\tilde{X}$ that
      $x$ lies on. The pre-image of this edge will be a union of open edges
      in $T$ that are homeomorphic to $e$.  If $\tilde{x}\in\tilde{X}$ is a
      vertex, let $O$ be the union of $\tilde{x}$ with half of the open
      $a$-edge that starts at $\tilde{x}$, and half of the open $-a$-edge
      that ends at $\tilde{x}$. Then $O$ is open in $\tilde{X}$, and its
      pre-image will be a union of disjoint copies of $O$.
    \end{proof}

  \item Exercise 3.16: Let $f:\tilde{X}\rightarrow X$ be a degree 2 cover.
    Prove that $\tilde{X}$ is a regular cover.
    \begin{proof}
      Consider the map $\phi:\tilde{X}\rightarrow\tilde{X}$ which sends
      $\tilde{x}\in\tilde{X}$ to the only other pre-image of $f(\tilde{x})$
      that is not equal to $\tilde{x}$. It suffices to show that $\phi$ is
      a deck transformation of $f$. $\phi$ is bijective from chasing
      definitions. By a symmetrical argument, if $\phi$ is continuous, then
      applying the same argument to $\phi^{-1}$ which is also $\phi$, we
      know $\phi^{-1}$ is also continuous, which would imply that $\phi$ is
      a covering space isomorphism. So, we only need to show that $\phi$ is
      continuous. \\

      Let $O$ be open in $\tilde{X}$. We wish to show that $\phi^{-1}(O)$
      is open in $\tilde{X}$. For each $\tilde{x}$ in $O$, let $o\subset
      f(\tilde{x})$ be a trivialized neighborhood of $f(\tilde{x})$, and
      $O_1'$ and $O_2'$ be the disjoint open sheets lying above $o$. Assume
      without loss of generality that $\tilde{x}$ is contained in $O_1'$,
      and let $O_{\tilde{x},1}=O_1'\cap O$, and
      $O_{\tilde{x},2}=\phi^{-1}(O_1)$. Then since $O_1'$ is homeomorphic
      to $O_2'$ and $O_{\tilde{x},1}$ is an open subset of $O_1'$,
      $O_{\tilde{x},2}$ will be an open subset of $O_2'$ and hence is an
      open subset of $\tilde{X}$. Chasing definitions, the pre-image of $O$
      under $\phi$ is the union of all such $O_{\tilde{X},2}$. More
      precisely, $\phi^{-1}(O)=\cup_{\tilde{x}\in O}O_{\tilde{x},2}$, which
      is open in $\tilde{X}$ since each $O_{\tilde{x},2}$ is open in
      $\tilde{X}$.  Hence $\phi$ is continuous, which completes our proof.
    \end{proof}

  \item Exercise 3.17: Let $f:\tilde{X}\rightarrow X$ be a covering space.
    \begin{enumerate}
      \item If $g:Y\rightarrow X$ is a continuous map, then define
        \begin{equation*}
          g^*(\tilde{X}) = \{(y,p)\,|\; g(y)=f(p)\}\subset
          Y\times\tilde{X}.
        \end{equation*}
        Also, let $g^*(f):g(\tilde{X})\rightarrow Y$ be the restriction of
        the projection $Y\times\tilde{X}\rightarrow Y$ onto the first
        factor. Prove that $g^*(f):g^*(\tilde{X})\rightarrow Y$ is a
        covering space. \label{qn:covering}
        \begin{proof}
          Done in homework 2.
        \end{proof}

      \item If $g:X\rightarrow X$ is the identity map, then prove that
        $g^*(\tilde{X})$ is isomorphic to $\tilde{X}$.
        \begin{proof}
          Consider the map $\phi:g^*(\tilde{X})\rightarrow\tilde{X}$ which
          sends $(x,p)$ to $p$. We show that $\phi$ is an isomorphism.
          It is routine to show that $\phi$ is bijective from chasing
          definitions. Also, $\phi$ is a projection map equipped with
          product topology, so it is continuous and open, which completes
          the proof.
        \end{proof}

      \item If $X'$ is a subspace of $X$ and $g:X'\rightarrow X$ is the
        inclusion of $X'$ into $X$, prove that
        $g^*(f):g^*(\tilde{X})\rightarrow X'$ is isomorphic to the
        restriction of $f$ to $X'$. \label{qn:subspace}

        \begin{proof}
          Let $\tilde{X}'=f^{-1}(X')$ and $f'=f|_{\tilde{X}'}$. We have
          shown in question~\ref{qn:covering} that $g^*(f)$ is a covering
          space. It remains to find a homeomorphism $\pi$ from
          $g^*(\tilde{X})$ to $\tilde{X}'$ such that $g^*(f)=\pi\circ f'$.
          Let $\pi$ be the projection map which sends $(x',\tilde{x}')\in
          g^*(f)$ to $\tilde{x}'\in\tilde{X}$. $\pi$ is open and continuous
          since it is a projection map equipped with product topology. The
          map is also bijective from chasing the definitions of $g^*(f)$
          and $\pi$. Hence $g^*(f)$ and $f'$ are isomorphic covers of
          $\tilde{X}'$.
        \end{proof}

      \item If $\tilde{X}$ is a trivial cover of $X$ and $g:Y\rightarrow X$
        is a continuous map, prove that $g^*(\tilde{X})$ is a trivial cover
        of $Y$. \label{qn:trivial}

        \begin{proof}
          Since $\tilde{X}$ is a trivial cover of $X$, we can assume
          $\tilde{X}=X\times D$ for some discrete $D$, and $f$ is the
          projection that sends $(x,d)\in X\times D$ to $x$. We have shown
          in question~\ref{qn:covering} that $g^*(f):g^*(\tilde{X}\times
          D)\rightarrow Y$ is a covering space. Chasing definitions,
          $g^*(X\times D)=\{(y,(g(y),d))\,|\; y\in Y, d\in D\}$, and
          $g^*(f)$ sends $(y,(g(y),d))$ to $y$. Now $Y\times D$ is a
          trivial cover of $Y$ with covering space map $f':Y\times D$
          sending $(y,d)\in Y\times D$ to $y$. Consider the map
          $\phi:g^*(Y\times D)\rightarrow g^*(X\times D)$ defined by
          mapping $(y,(g(y),d))$ to $(y,d)$. We show that $\phi$ is a
          homeomorphism, which would imply that $Y\times D$ and
          $g^*(\tilde{X})$ are isomorphic covers of $Y$, and hence
          $g^*(\tilde{X})$ is a trivial cover of $Y$. \\

          $\phi$ is bijective from the way it was defined. Also, it is a
          projection map under the product topology, which makes it open
          and continuous. Therefore $\phi$ is a homeomorphism, which
          completes the proof.
        \end{proof}

      \item If $g:Y\rightarrow X$ and $h:Z\rightarrow Y$ are continuous
        maps, prove that the cover $(g\circ h)^*(\tilde{X})$ of $Z$ is
        isomorphic to $h^*(g^*(\tilde{X}))$ of $Z$. \label{qn:compose}

        \begin{proof}
          $(g\circ h)^*(\tilde{X})$ is a cover of $Z$ by
          question~\ref{qn:covering}. $h^*(g^*(\tilde{X}))$ is also a cover
          of $Z$ by two applications of question~\ref{qn:covering}.  We
          need to show that there is a homeomorphism from $(g\circ
          h)^*(\tilde{X})$ to $h^*(g^*(\tilde{X}))$ such that $(g\circ
          h)^*(f)=h^*(g^*(f))\circ h^*(g^*(\tilde{X}))$. Consider the map
          $\phi$ which sends $(z,\tilde{x})$ to $(z,(h(z),\tilde{x}))$,
          where $\tilde{x}\in f^{-1}(g(h(z)))$. This map is well-defined
          because $g(h(z))=f(f^{-1}(g(h(z))))$. Chasing definitions, it is
          also bijective. Finally, the map is continuous and open because
          it is a projection map under the product topology. Hence $(g\circ
          h)^*(\tilde{X})$ and $h^*(g^*(\tilde{X}))$ are isomorphic covers
          of $Z$.
        \end{proof}

      \item Let $g:Y\rightarrow X$ be the constant map that takes every
        point of $Y$ to a fixed point $p_0\in X$, prove that
        $g^*(\tilde{X})$ is a trivial cover of $Y$. 

        \begin{proof}
          Consider $X'=\{p_0\}$ as a subspace of $X$.  From
          questions~\ref{qn:covering} and \ref{qn:subspace},
          $g^*(f'):g^*(\tilde{X}')\rightarrow X'$ is a covering space.
          Because any covering space looks locally like a trivial cover,
          the covering space $f':\tilde{X}'\rightarrow X'$ of $X'$ is a
          trivial cover. Then from question~\ref{qn:trivial}, since the map
          $h:Y\rightarrow X'$ that sends $y$ to $p_0$ is continuous,
          $h^*(\tilde{X'})$ will also be a trivial cover of $Y$. Finally,
          from question~\ref{qn:compose}, $h^*(\tilde{X'})$ is isomorphic
          to $g^*(\tilde{X})$, which makes it also a trivial cover of $Y$.
        \end{proof}
    \end{enumerate}

  \item Exercise 4.11: Let $Y$ and $X$ be spaces. Prove that the relation
    of being homotopic is an equivalence relation between continuous maps
    from $Y$ to $X$.
    \begin{proof}
      Reflexive: Given a continuous map $g:Y\rightarrow X$, consider the
      map $G:Y\times[0,1]\rightarrow X$ that sends $G(y,t)$ to $g(y)$ for
      all $t\in[0,1]$. This is a continuous map since the pre-image of
      any open subset $O$ of $X$ is $g^{-1}(O)\times[0,1]$, which is open
      in $Y\times[0,1]$. Also, $G(y,0)=g(y)=G(y,1)$. Hence $g$ is homotopic
      to $g$. \\

      Symmetric: Assume $g_0,g_1:Y\rightarrow X$ are homotopic via the
      continuous map $G:Y\times[0,1]\rightarrow X$ such that
      $g_0(y)=G(y,0)$ and $g_1(y)=G(y,1)$. Consider the map
      $H:Y\times[0,1]\rightarrow X$ that sends $(y,t)$ to $G(y,1-t)$. Then
      $H(y,0)=g_1(y)$ and $H(y,1)=g_0(y)$. Also, $H$ is the composition of
      two continuous functions $G\circ F$, where $F:Y\times[0,1]\rightarrow
      Y\times[0,1]$ sends $(y,t)$ to $(y,1-t)$. Hence, $H$ is a continuous
      map that sends $g_1$ to $g_0$, implying that homotopy is symmetric.
      \\

      Transitive: Assume $g_0$ is homotopic to $g_1$ via continuous map
      $G_0$, and $g_1$ is homotopic to $g_2$ via continuous map $G_1$.
      Consider the map $H:Y\times[0,1]\rightarrow X$ that maps $(y,t)$ to
      $G_0(y,2t)$ if $0\leq t\leq 0.5$ and maps $(y,t)$ to $G_1(y,2t-1)$ if
      $0.5<t\leq1$. $H$ is a concatenation of continuous functions $G_1$ to
      $G_0$, which agree at the common domain, and hence $H$ is also
      continuous. Therefore homotopy is also transitive.
    \end{proof}

  \item Exercise 4.12: Let $X$ be a space and let $X_1,X_2\subset X$ be
    subspaces such that $X=X_1\cup X_2$. Assume that both $X_1$ and $X_2$
    only have trivial covers and that $X_1\cap X_2$ is path-connected.
    Prove that $X$ only has trivial covers.

    \begin{proof}
      Let $\pi:\tilde{X}\rightarrow X$ be a covering space of $X$. We show
      that $\pi$ can only be trivial. In the earlier exercise we have shown
      that restriction of $\pi$ to any subset $X_i$ of $X$ is also a
      covering space of $X_i$. From assumption, these two restriction
      covers are trivial, so $\pi^{-1}(X_1)=\sqcup_{\alpha}X_{1,\alpha}$
      and $\pi^{-1}(X_2)=\sqcup_{\alpha}X_{2,\beta}$, which are disjoint
      unions of copies of $X_1$ and $X_2$ respectively. We form a bijection
      between the $X_{1,\alpha}$'s and $X_{2,\beta}$'s. \\

      Given $X_{1,\alpha}$, pick any $\tilde{x}\in X_{1,\alpha}$ that lies
      in the fiber of $X_1\cap X_2$.  This $\tilde{x}$ must also lie in
      some $X_{2,\beta}$ because $X_1\cap X_2\subseteq X_2$ implies the
      fibers of $X_1\cap X_2$ lies in the fibres of $X_2$. We map the
      $X_{1,\alpha}$ to the $X_{2,\beta}$ and show that this map is
      well-defined, or in other words, does not depend on the choice of
      $\tilde{x}$. Assume by contradiction that there is some $\tilde{x}'$
      in $X_{1,\alpha}\cap\pi^{-1}(X_1\cap X_2)$ that is contained in some
      $X_{2,\beta'}\neq X_{2,\beta}$. Consider the path $p\subset X_1\cap
      X_2$ from $\pi(\tilde{x})$ to $\pi(\tilde{x'})$. This path exists
      since $X_1\cap X_2$ is path connected. By the unique lifting of
      paths, if $p$ is lifted to a path that starts at $\tilde{x}\in
      X_{1,\alpha}$, the end of the lifted path must be unique, and cannot
      be in both $X_{2,\beta}$ and $X_{2,\beta'}$ at the same time. Hence
      the each $X_{1,\alpha}$ can be mapped to exactly one $X_{2,\beta}$.
      \\

      Next, we prove that if we tried to match each fiber of $X_2$ with a
      fiber of $X_1$ instead using the same method, we would pair the same
      fibers together. Assume not. Then there must be some $X_{1,\alpha}$
      and $X_{2,\beta}$ that share a common point $\tilde{x}$ in the fiber
      of $X_1\cap X_2$, yet there is some $\tilde{y_1}\in X_{1,\alpha}$ and
      $\tilde{y_2}\in X_{2,\beta}$ such that
      $\pi(\tilde{y_1})=\pi(\tilde{y_2})=y\in X_1\cap X_2$. Forming a path
      from $\pi(\tilde{x})$ to $y$ and lifting the path, we get from
      uniqueness of the lifted path that the end point of the path cannot
      be both $\tilde{y_1}$ and $\tilde{y_2}$. Hence there is a bijection
      between the fibers of $X_1$ and the fibers of $X_2$. \\

      We show that each matched $X_{1,\alpha}$ and $X_{2,\beta}$ forms a
      copy of $X$ under a restriction of $\pi$, which would imply
      that $\pi$ is a trivial cover. Let $Y=X_{1,\alpha}\cup X_{2,\beta}$.
      By definition, $\pi$ maps $Y$ to $X$ surjectively. The map is also
      injective: restricted to areas that are not in the fiber of $X_1\cap
      X_2$, $\pi$ must map $Y$ injectively because of the homeomorphism
      between $X_1$ with $X_{1,\alpha}$ and between $X_2$ and
      $X_{2,\beta}$. If $\tilde{x}\neq\tilde{x'}\in Y$ is in the fibre of
      $X_1\cap X_2$ and they are mapped to the same point $x\in X_1\cap
      X_2$ by $\pi$, then choosing any $\tilde{y}\in
      X_{1,\alpha}\cup\pi^{-1}(X_1\cap X_2)$ that mapped $X_{1,\alpha}$ to
      $X_{2,\beta}$, we construct a path from $\pi(y)$ to $x$ and lift that
      path up by $\pi$. Then by uniqueness of lifted paths, since the
      endpoint of the lifted path must be unique, they cannot be both
      $\tilde{x}$ and $\tilde{x'}$ at the same time. Hence $\pi$ maps $Y$
      injectively to $X$. So, $X$ is a copy of $Y$ under the restriction of
      $\pi$, which completes the proof. \\
    \end{proof}

  \item Exercise 4.13: Prove that for $n\geq2$ the $n$-sphere $S^n$ has
    only trivial covers.
    \begin{proof}
      We apply the claim from exercise 4.12. An $n$-sphere is a union of
      two hemispheres whose intersection is an $(n-1)$-sphere. Each
      hemisphere is contractible and hence has trivial covering space.
      Also, their intersection, the $(n-1)$-sphere, is path-connected.
      Hence from exercise 4.12, the $n$-sphere has only trivial covers.
    \end{proof}
\end{enumerate}
\end{document}
