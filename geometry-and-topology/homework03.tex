\documentclass{article}
\usepackage[left=3cm,right=3cm,top=3cm,bottom=3cm]{geometry}
\usepackage{amsmath,amssymb,amsthm,pgfplots,tikz}
\usetikzlibrary{patterns}
\usepackage{color}
\setlength{\parindent}{2mm}
\newcommand{\TODO}[1]{\textcolor{red}{TODO: #1}}

\begin{document}
\title{Geometry and Topology: Homework 3}
\author{Li Ling Ko\\ lko@nd.edu}
\date{\today}
\maketitle

\begin{enumerate}
  \item Exercise 2.16: Let $T$ be the infinite 4-valent tree and let
    $\tilde{X}$ be any connected graph which is oriented and labeled as in
    Example 2.10. Also, let $X$ be the graph from Figure 2.3, so we have
    covering maps $f:T\rightarrow X$ and $g:\tilde{X}\rightarrow X$. Let
    $v$ be any vertex of $T$ and let $w$ be any vertex of $\tilde{X}$.
    Prove that there exists a covering map $h:T\rightarrow\tilde{X}$ such
    that $f=g\circ h$.

    \begin{proof}
      Let $h$ map the middle vertex $t_m$ in $T$ to
      $x_m=g^{-1}(f(t_m))\in\tilde{X}$. Then for other vertices $t$ in $T$,
      we find the unique path $p$ from $t_m$ to $t$, which can be
      represented as a word from $\{a,-a,b,-b\}^{<\omega}$. Then let $h$
      map $t$ to the unique vertex $x$ in $\tilde{X}$ that is reached from
      starting at vertex $x_m$ and following the same path directions given
      by $p$. Also, given any edge from vertex $t_0$ to $t_1$ in $T$, let
      $h$ map the edge linearly in the direction from $h(t_0)$ to $h(t_1)$.
      \\

      We first show that $f=g\circ h$. At the middle node, we have
      $g(h(t_m))=f(t_m)$ by our choice of $h(t_m)=g^{-1}(f(t_m))$. For the
      remaining nodes, it is routine to show by induction on the path
      distance a node $t$ is from the middle node $t_m$ that
      $g(h(t))=f(t)$. \\
    \end{proof}
\end{enumerate}
\end{document}
