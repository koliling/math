\documentclass{article}
\usepackage[left=3cm,right=3cm,top=3cm,bottom=3cm]{geometry}
\usepackage{amsmath,amssymb,amsthm,pgfplots,tikz}
\usetikzlibrary{patterns}
\usepackage{color}
\setlength{\parindent}{2mm}
\newcommand{\TODO}[1]{\textcolor{red}{TODO: #1}}

\begin{document}
\title{Geometry and Topology: Homework 3}
\author{Li Ling Ko\\ lko@nd.edu}
\date{\today}
\maketitle

\begin{enumerate}
  \item Exercise 2.16: Let $T$ be the infinite 4-valent tree and let
    $\tilde{X}$ be any connected graph which is oriented and labeled as in
    Example 2.10. Also, let $X$ be the graph from Figure 2.3, so we have
    covering maps $f:T\rightarrow X$ and $g:\tilde{X}\rightarrow X$. Let
    $v$ be any vertex of $T$ and let $w$ be any vertex of $\tilde{X}$.
    Prove that there exists a covering map $h:T\rightarrow\tilde{X}$ such
    that $f=g\circ h$.

    \begin{proof}
      Let $h$ map the middle vertex $t_m$ in $T$ to any vertex $x_m$ in
      $\tilde{X}$. Then for other vertices $t$ in $T$, we find the unique
      path $p$ from $t_m$ to $t$, which can be represented as a word from
      $\{a,-a,b,-b\}^{<\omega}$. Then let $h$ map $t$ to the unique vertex
      $x$ in $\tilde{X}$ that is reached from starting at vertex $x_m$ and
      following the same path directions given by $p$. Then, given any edge
      from vertex $t_0$ to $t_1$ in $T$, let $h$ map the edge linearly in
      the direction from $h(t_0)$ to $h(t_1)$. This map is well-defined
      since there is only one path between any two vertices on $T$ and
      every vertex in $\tilde{X}$ has valence 4 with one $a$ (resp. $b$)
      edge entering and one $a$ (resp. $b$) edge exiting. \\

      Next, we show that $f=g\circ h$. The middle vertex is mapped to the
      only vertex in $X$, as required. For the remaining nodes, it is
      routine to show by induction on the path distance between a node $t$
      to the middle node $t_m$ that $g\circ h$ will map all vertices to the
      vertex in $X$, all $a$-edges to the $a$-edge in $X$, and $b$-edges to
      the $b$-edge in $X$, in the proper direction. \\

      Finally, we show that $h$ is a covering map. We note that the map
      is surjective from the connectivity of $\tilde{X}$ and the fact that
      every word from $\{a,-a,b,-b\}^{<\omega}$ can be reached on $T$
      starting from the middle vertex of $T$. Also, $h$ inherits continuity
      from $f$ and $g$. Furthermore, given any point on $\tilde{x}\in\tilde{X}$
      that is not a vertex, consider the open edge $e\subset\tilde{X}$ that
      $x$ lies on. The pre-image of this edge will be a union of open edges
      in $T$ that are homeomorphic to $e$.  If $\tilde{x}\in\tilde{X}$ is a
      vertex, let $O$ be the union of $\tilde{x}$ with half of the open
      $a$-edge that starts at $\tilde{x}$, and half of the open $-a$-edge
      that ends at $\tilde{x}$. Then $O$ is open in $\tilde{X}$, and its
      pre-image will be a union of disjoint copies of $O$.
    \end{proof}
\end{enumerate}
\end{document}
