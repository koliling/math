\documentclass{article}
\usepackage[left=3cm,right=3cm,top=3cm,bottom=3cm]{geometry}
\usepackage{amsmath,amssymb,amsthm,pgfplots,tikz}
\usetikzlibrary{patterns}
\usepackage{color}
\setlength{\parindent}{2mm}
\newcommand{\TODO}[1]{\textcolor{red}{TODO: #1}}

\begin{document}
\title{Geometry and Topology: Homework 3}
\author{Li Ling Ko\\ lko@nd.edu}
\date{\today}
\maketitle

\begin{enumerate}
  \item Exercise 2.16: Let $T$ be the infinite 4-valent tree and let
    $\tilde{X}$ be any connected graph which is oriented and labeled as in
    Example 2.10. Also, let $X$ be the graph from Figure 2.3, so we have
    covering maps $f:T\rightarrow X$ and $g:\tilde{X}\rightarrow X$. Let
    $v$ be any vertex of $T$ and let $w$ be any vertex of $\tilde{X}$.
    Prove that there exists a covering map $h:T\rightarrow\tilde{X}$ such
    that $f=g\circ h$.

    \begin{proof}
      Let $h$ map the middle vertex $t_m$ in $T$ to any vertex $x_m$ in
      $\tilde{X}$. Then for other vertices $t$ in $T$, we find the unique
      path $p$ from $t_m$ to $t$, which can be represented as a word from
      $\{a,-a,b,-b\}^{<\omega}$. Then let $h$ map $t$ to the unique vertex
      $x$ in $\tilde{X}$ that is reached from starting at vertex $x_m$ and
      following the same path directions given by $p$. Then, given any edge
      from vertex $t_0$ to $t_1$ in $T$, let $h$ map the edge linearly in
      the direction from $h(t_0)$ to $h(t_1)$. This map is well-defined
      since there is only one path between any two vertices on $T$ and
      every vertex in $\tilde{X}$ has valence 4 with one $a$ (resp. $b$)
      edge entering and one $a$ (resp. $b$) edge exiting. \\

      Next, we show that $f=g\circ h$. The middle vertex is mapped to the
      only vertex in $X$, as required. For the remaining nodes, it is
      routine to show by induction on the path distance between a node $t$
      to the middle node $t_m$ that $g\circ h$ will map all vertices to the
      vertex in $X$, all $a$-edges to the $a$-edge in $X$, and $b$-edges to
      the $b$-edge in $X$, in the proper direction. \\

      Finally, we show that $h$ is a covering map. We note that the map
      is surjective from the connectivity of $\tilde{X}$ and the fact that
      every word from $\{a,-a,b,-b\}^{<\omega}$ can be reached on $T$
      starting from the middle vertex of $T$. Also, $h$ inherits continuity
      from $f$ and $g$. Furthermore, given any point on $\tilde{x}\in\tilde{X}$
      that is not a vertex, consider the open edge $e\subset\tilde{X}$ that
      $x$ lies on. The pre-image of this edge will be a union of open edges
      in $T$ that are homeomorphic to $e$.  If $\tilde{x}\in\tilde{X}$ is a
      vertex, let $O$ be the union of $\tilde{x}$ with half of the open
      $a$-edge that starts at $\tilde{x}$, and half of the open $-a$-edge
      that ends at $\tilde{x}$. Then $O$ is open in $\tilde{X}$, and its
      pre-image will be a union of disjoint copies of $O$.
    \end{proof}

  \item Exercise 3.16: Let $f:\tilde{X}\rightarrow X$ be a degree 2 cover.
    Prove that $\tilde{X}$ is a regular cover.
    \begin{proof}
      Consider the map $\phi:\tilde{X}\rightarrow\tilde{X}$ which sends
      $\tilde{x}\in\tilde{X}$ to the only other pre-image of $f(\tilde{x})$
      that is not equal to $\tilde{x}$. It suffices to show that $\phi$ is
      a deck transformation of $f$. $\phi$ is bijective from chasing
      definitions. By a symmetrical argument, if $\phi$ is continuous, then
      applying the same argument to $\phi^{-1}$ which is also $\phi$, we
      know $\phi^{-1}$ is also continuous, which would imply that $\phi$ is
      a covering space isomorphism. So, we only need to show that $\phi$ is
      continuous. \\

      Let $O$ be open in $\tilde{X}$. We wish to show that $\phi^{-1}(O)$
      is open in $\tilde{X}$. For each $\tilde{x}$ in $O$, let $o\subset
      f(\tilde{x})$ be a trivialized neighborhood of $f(\tilde{x})$, and
      $O_1'$ and $O_2'$ be the disjoint open sheets lying above $o$. Assume
      without loss of generality that $\tilde{x}$ is contained in $O_1'$,
      and let $O_{\tilde{x},1}=O_1'\cap O$, and
      $O_{\tilde{x},2}=\phi^{-1}(O_1)$. Then since $O_1'$ is homeomorphic
      to $O_2'$ and $O_{\tilde{x},1}$ is an open subset of $O_1'$,
      $O_{\tilde{x},2}$ will be an open subset of $O_2'$ and hence is an
      open subset of $\tilde{X}$. Chasing definitions, the pre-image of $O$
      under $\phi$ is the union of all such $O_{\tilde{X},2}$. More
      precisely, $\phi^{-1}(O)=\cup_{\tilde{x}\in O}O_{\tilde{x},2}$, which
      is open in $\tilde{X}$ since each $O_{\tilde{x},2}$ is open in
      $\tilde{X}$.  Hence $\phi$ is continuous, which completes our proof.
    \end{proof}

  \item Exercise 3.17: Let $f:\tilde{X}\rightarrow X$ be a covering space.
    \begin{enumerate}
      \item If $g:Y\rightarrow X$ is a continuous map, then define
        \begin{equation*}
          g^*(\tilde{X}) = \{(y,p)\,|\; g(y)=f(p)\}\subset
          Y\times\tilde{X}.
        \end{equation*}
        Also, let $g^*(f):g(\tilde{X})\rightarrow Y$ be the restriction of
        the projection $Y\times\tilde{X}\rightarrow Y$ onto the first
        factor. Prove that $g^*(f):g^*(\tilde{X})\rightarrow Y$ is a
        covering space. \label{qn:covering}
        \begin{proof}
          Done in homework 2.
        \end{proof}

      \item If $g:X\rightarrow X$ is the identity map, then prove that
        $g^*(\tilde{X})$ is isomorphic to $\tilde{X}$.
        \begin{proof}
          Consider the map $\phi:g^*(\tilde{X})\rightarrow\tilde{X}$ which
          sends $(x,p)$ to $p$. We show that $\phi$ is an isomorphism.
          It is routine to show that $\phi$ is bijective from chasing
          definitions. Also, $\phi$ is a projection map equipped with
          product topology, so it is continuous and open, which completes
          the proof.
        \end{proof}

      \item If $X'$ is a subspace of $X$ and $g:X'\rightarrow X$ is the
        inclusion of $X'$ into $X$, prove that
        $g^*(f):g^*(\tilde{X})\rightarrow X'$ is isomorphic to the
        restriction of $f$ to $X'$.

        \begin{proof}
          Let $\tilde{X}'=f^{-1}(X')$ and $f'=f|_{\tilde{X}'}$. We have
          shown in question~\ref{qn:covering} that $g^*(f)$ is a covering
          space. It remains to find a homeomorphism $\pi$ from
          $g^*(\tilde{X})$ to $\tilde{X}'$ such that $g^*(f)=\pi\circ f'$.
          Let $\pi$ be the projection map which sends $(x',\tilde{x}')\in
          g^*(f)$ to $\tilde{x}'\in\tilde{X}$. $\pi$ is open and continuous
          since it is a projection map equipped with product topology. The
          map is also bijective from chasing the definitions of $g^*(f)$
          and $\pi$. Hence $g^*(f)$ and $f'$ are isomorphic covers of
          $\tilde{X}'$.
        \end{proof}

      \item If $\tilde{X}$ is a trivial cover of $X$ and $g:Y\rightarrow X$
        is a continuous map, prove that $g^*(\tilde{X})$ is a trivial cover
        of $Y$.

        \begin{proof}
          Since $\tilde{X}$ is a trivial cover of $X$, we can assume
          $\tilde{X}=X\times D$ for some discrete $D$, and $f$ is the
          projection that sends $(x,d)\in X\times D$ to $x$. We have shown
          in question~\ref{qn:covering} that $g^*(f):g^*(\tilde{X\times
          D})\rightarrow Y$ is a covering space. Chasing definitions,
          $g^*(X\times D)=\{(y,(f(y),d))\,|\; y\in Y, d\in D\}$, and
          $g^*(f)$ sends $(y,(f(y),d))$ to $y$. Now $Y\times D$ is a
          trivial cover of $Y$ with covering space map $f':Y\times D$
          sending $(y,d)\in Y\times D$ to $y$. Consider the map
          $\phi:g^*(Y\times D)\rightarrow g^*(X\times D)$ defined by
          mapping $(y,(f(y),d))$ to $(y,d)$. We show that $\phi$ is a
          homeomorphism, which would imply that $Y\times D$ and
          $g^*(\tilde{X})$ are isomorphic covers of $Y$, and hence
          $g^*(\tilde{X})$ is a trivial cover of $Y$. \\

          $\phi$ is bijective from the way it was defined. Also, it is a
          projection map under the product topology, which makes it open
          and continuous. Therefore $\phi$ is a homeomorphism, which
          completes the proof.
        \end{proof}

      \item If $g:Y\rightarrow X$ and $h:Z\rightarrow Y$ are continuous
        maps, prove that the cover $(g\circ h)^*(\tilde{X})$ of $Z$ is
        isomorphic to $h^*(g^*(\tilde{X}))$ of $Z$.

        \begin{proof}
          $(g\circ h)^*(\tilde{X})$ is a cover of $Z$ by
          question~\ref{qn:covering}. $h^*(g^*(\tilde{X}))$ is also a cover
          of $Z$ by two applications of question~\ref{qn:covering}.  We
          need to show that there is a homeomorphism from $(g\circ
          h)^*(\tilde{X})$ to $h^*(g^*(\tilde{X}))$ such that $(g\circ
          h)^*(f)=h^*(g^*(f))\circ h^*(g^*(\tilde{X}))$. Consider the map
          $\phi$ which sends $(z,\tilde{x})$ to $(z,(h(z),\tilde{x}))$,
          where $\tilde{x}\in f^{-1}(g(h(z)))$. This map is well-defined
          because $g(h(z))=f(f^{-1}(g(h(z))))$. Chasing definitions, it is
          also bijective. Finally, the map is continuous and open because
          it is a projection map under the product topology. Hence $(g\circ
          h)^*(\tilde{X})$ and $h^*(g^*(\tilde{X}))$ are isomorphic covers
          of $Z$.
        \end{proof}

      \item Let $g:Y\rightarrow X$ be the constant map that takes every
        point of $Y$ to a fixed point $p_0\in X$, prove that
        $g^*(\tilde{X})$ is a trivial cover of $Y$. 
        \begin{proof}
        \end{proof}
    \end{enumerate}
\end{enumerate}
\end{document}
