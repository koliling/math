\documentclass{article}
\usepackage[left=3cm,right=3cm,top=3cm,bottom=3cm]{geometry}
\usepackage{amsmath,amssymb,amsthm,pgfplots,tikz}
\usetikzlibrary{patterns}
\usepackage{color}
\setlength{\parindent}{2mm}
\newcommand{\TODO}[1]{\textcolor{red}{TODO: #1}}

\begin{document}
\title{Geometry and Topology: Homework 3}
\author{Li Ling Ko\\ lko@nd.edu}
\date{\today}
\maketitle

\begin{enumerate}
  \item Exercise 2.16: Let $T$ be the infinite 4-valent tree and let
    $\tilde{X}$ be any connected graph which is oriented and labeled as in
    Example 2.10. Also, let $X$ be the graph from Figure 2.3, so we have
    covering maps $f:T\rightarrow X$ and $g:\tilde{X}\rightarrow X$. Let
    $v$ be any vertex of $T$ and let $w$ be any vertex of $\tilde{X}$.
    Prove that there exists a covering map $h:T\rightarrow\tilde{X}$ such
    that $f=g\circ h$.

    \begin{proof}
      Let $m$ be the middle vertex in $T$, and let
      $v=g^{-1}(f(m))\in\tilde{X}$. Consider the map $h$ which sends $m$
      to $v$, and for any other vertex $u$ in $T$, we find the unique path
      $p$ from $m$ to $u$, which can be represented as a word from
      $\{a,-a,b,-b\}^{<\omega}$. Then $h$ maps $u$ to the unique vertex
      $x$ in $\tilde{X}$ that is reached from starting at vertex $v$ and
      following the same path directions given by $p$.
    \end{proof}
\end{enumerate}
\end{document}
