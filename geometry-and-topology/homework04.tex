\documentclass{article}
\usepackage[left=3cm,right=3cm,top=3cm,bottom=3cm]{geometry}
\usepackage{amsmath,amssymb,amsthm,pgfplots,tikz,mathtools}
\usetikzlibrary{patterns}
\usepackage{color}
\setlength{\parindent}{2mm}
\newcommand{\TODO}[1]{\textcolor{red}{TODO: #1}}

\begin{document}
\title{Geometry and Topology: Homework 4}
\author{Li Ling Ko\\ lko@nd.edu}
\date{\today}
\maketitle

\begin{enumerate}
  \item \begin{proof} We show by induction on $n$ that the fibers of
    $\cup_{k=1}^n\widetilde{X}^{i}$ can be endowed with the structure of a
    CW complex such that $f$ takes the interiors of the $k$-cells in the
    fibers homeomorphically to the interiors of $k$-cells in
    $\cup_{k=1}^n\widetilde{X}^{i}$. For $k=0$, the fiber of each
    $X^0_{\alpha}$ is a union of $0$-cells. These 0-cells are contained in
    disjoint sheets lying above a trivialized neighborhood of
    $X^0_{\alpha}$, and hence can be assigned the discrete topology in
    $\widetilde{X}$. \\

    For the inductive step $n+1$, consider the embedding
    $\varphi_\alpha:D^{n+1}\xhookrightarrow{}X$ which embeds an
    $(n+1)$-cell to $X^{n+1}$. This embedding glues the boundary
    $S^n\subset D^{n+1}$ onto $X^n$ and takes the interior of $D^{n+1}$ to
    the interior of an $(n+1)$-cell in $X^{n+1}$. Pick a point $x_0\in
    S^n\subset D^{n+1}$. The fiber of $\varphi_\alpha(x_0)$ is a union of
    points $\{\widetilde{x}_{0,\beta}\}_{\beta\in I}$ in $\widetilde{X}^n$,
    contained in disjoint sheets lying above the trivialized neighborhood
    of $\varphi_\alpha(x_0)$. Then for each $\widetilde{x}_{0,\beta}$,
    consider the lift of $\varphi_\alpha$ to $\widetilde{X}^{n+1}$ that
    sends $x_0$ to $\widetilde{x}_{0,\beta}$. By the lifting criterion,
    because $D^{n+1}$ is contractable, such a lift
    $\widetilde{\varphi_\alpha}$ exists. This lift will glue $D_{n+1}$ at
    the boundary to $\widetilde{X}^n$, forming $\widetilde{X}^{n+1}$, which
    completes the inductive step.
  \end{proof}

  \item
    \begin{enumerate}
      \item
      \begin{proof}
        The left to right implication is trivial by chasing definitions.
        For the converse, assume $Y_1^{(1)}\rightarrow X^{(1)}$ and
        $Y_2^{(1)}\rightarrow X^{(1)}$ are isomorphic. We prove by
        induction on $n$ that the restrictions $Y_1^{(n)}\rightarrow
        X^{(n)}$ and $Y_2^{(n)}\rightarrow X^{(n)}$ are isomorphic. The
        base case $n=1$ is true by assumption. \\

        For the inductive step $n+1$, consider the embedding
        $\varphi_\alpha:D^{n+1}\xhookrightarrow{}X$ which embeds an
        $(n+1)$-cell to $X^{(n+1)}$. This embedding glues the boundary
        $S^n\subset D^{n+1}$ onto $X^n$ and takes the interior of $D^{n+1}$
        to the interior of an $(n+1)$-cell in $X^{(n+1)}$. Pick a point
        $x\in S^n\subset D^{n+1}$. For each $Y_i$, $i\in\{1,2\}$, the fiber
        of $\varphi_\alpha(x)$ is a union of points
        $\{\widetilde{y}_{i,\beta}\}_{\beta\in J_i}$ in
        $\widetilde{Y_i}^{(n)}$, contained in disjoint sheets lying above
        the trivialized neighborhood of $\varphi_\alpha(x)$. By induction
        hypothesis, since $\widetilde{Y_1}^{(n)}$ and
        $\widetilde{Y_2}^{(n)}$ are isomorphic covers of $X^{(n)}$, the
        homeomorphism between $\widetilde{Y_1}^{(n)}$ and
        $\widetilde{Y_2}^{(n)}$ will induce a bijection between
        $\{\widetilde{y}_{1,\beta}\}_{\beta\in J_1}$ and
        $\{\widetilde{y}_{2,\beta}\}_{\beta\in J_2}$, so we can assume
        without loss of generality that $J_1=J_2=J$ and the homeomorphism
        maps $\widetilde{y}_{1,\beta}$ to $\widetilde{y}_{2,\beta}$. Then
        for each $\widetilde{y}_{i,\beta}$, consider the lift of
        $\varphi_\alpha$ to $\widetilde{Y_i}^{n+1}$ that sends $x$ to
        $\widetilde{y}_{i,\beta}$. By the lifting criterion, because
        $D^{n+1}$ is contractable, such a lift
        $\widetilde{\varphi_\alpha}$ exists. The image of each lift is a
        $(n+1)$-cell that is projected down to a $(n+1)$-cell in $X$. Also,
        the lift to $Y_1$ and the lift to $Y_2$ will be homeomorphic, which
        completes the inductive step.
      \end{proof}

      \item
      \begin{proof}
        The left to right implication is trivial by chasing definitions.
        For the converse, assume $Y^{(1)}\rightarrow X^{(1)}$ is a regular
        covering space. We prove by induction on $n$ that the restriction
        $Y^{(n)}\rightarrow X^{(n)}$ is a regular covering space. The base
        case $n=1$ is true by assumption. \\

        For the inductive step $n+1$, let $x\in X^{(n+1)}\setminus
        X^{(n)}$. We wish to show that given any two elements in the fibers
        of $x$, there is a covering map in $\text{Deck}(Y^{(n)})$ that
        sends one element to the other. Now for each element in the

        consider the embedding
        $\varphi_\alpha:D^{n+1}\xhookrightarrow{}X$ which embeds an
        $(n+1)$-cell to $X^{(n+1)}$. This embedding glues the boundary
        $S^n\subset D^{n+1}$ onto $X^n$ and takes the interior of $D^{n+1}$
        to the interior of an $(n+1)$-cell in $X^{(n+1)}$. Pick a point
        $x\in S^n\subset D^{n+1}$. For each $Y_i$, $i\in\{1,2\}$, the fiber
        of $\varphi_\alpha(x)$ is a union of points
        $\{\widetilde{y}_{i,\beta}\}_{\beta\in J_i}$ in
        $\widetilde{Y_i}^{(n)}$, contained in disjoint sheets lying above
        the trivialized neighborhood of $\varphi_\alpha(x)$. By induction
        hypothesis, since $\widetilde{Y_1}^{(n)}$ and
        $\widetilde{Y_2}^{(n)}$ are isomorphic covers of $X^{(n)}$, the
        homeomorphism between $\widetilde{Y_1}^{(n)}$ and
        $\widetilde{Y_2}^{(n)}$ will induce a bijection between
        $\{\widetilde{y}_{1,\beta}\}_{\beta\in J_1}$ and
        $\{\widetilde{y}_{2,\beta}\}_{\beta\in J_2}$, so we can assume
        without loss of generality that $J_1=J_2=J$ and the homeomorphism
        maps $\widetilde{y}_{1,\beta}$ to $\widetilde{y}_{2,\beta}$. Then
        for each $\widetilde{y}_{i,\beta}$, consider the lift of
        $\varphi_\alpha$ to $\widetilde{Y_i}^{n+1}$ that sends $x$ to
        $\widetilde{y}_{i,\beta}$. By the lifting criterion, because
        $D^{n+1}$ is contractable, such a lift
        $\widetilde{\varphi_\alpha}$ exists. The image of each lift is a
        $(n+1)$-cell that is projected down to a $(n+1)$-cell in $X$.
        Also, the lift to $Y_1$ and the lift to $Y_2$ will be
        homeomorphic, which completes the inductive step.
      \end{proof}
    \end{enumerate}

    \item
    \begin{proof}
    \end{proof}

    \item
    \begin{proof}
    \end{proof}

    \item
    \begin{proof}
    \end{proof}

    \item
    \begin{proof}
    \end{proof}

    \item
    \begin{proof}
      Let $[\gamma]\in\pi_1(X,p)$. We wish to show that $[\gamma]=1$. Now
      $\gamma$ is a continuous map $\gamma:[0,1]\rightarrow X$ such that
      $\gamma(0)=\gamma(1)=p$. The loop defined by $\gamma$ is covered by
      $\{U_\alpha\}$. For each $\alpha$, consider the pre-image of $\gamma$
      on $\{U_\alpha\}$. This pre-image $I_\alpha=\gamma^{-1}(U_\alpha\cap
      \gamma([1,0]))$ will be an open subset of $[0,1]$. The union of all
      these $I_\alpha$'s will be an open cover of $[0,1]$. Then since
      $[0,1]$ is a compact metric space and $\{I_\alpha\}$ is an open
      cover, the Lebesgue number of the covering exists. We can assume that
      this number is $2/n$ for some $n\in\mathbb{N}^+$. Partition the
      interval $[0,1]$ into $n$ intervals $I_1,\ldots,I_n$ of equal length,
      where $I_i=[(i-1)/n,i/n]$. Then by the definition of the Lebesgue
      number, each $I_i$ is contained in some $I_{\alpha_i}$, and therefore
      $\{U_{\alpha_i}\}_{i\leq n}$ covers path $\gamma([0,1])$. \\

      For a given $i\in\{1,\ldots,n\}$, we iteratively form a simple loop
      $\gamma_i\in\pi_1(X,p)$ that lies in $U_{\alpha_i}$, and such that
      $\gamma=\gamma_1\cdots\gamma_n$. For $i=1$, let
      $\gamma_1=\gamma\upharpoonright_{I_1}\sigma_1$, where $\sigma_1$ is a
      path from $\gamma(1/n)$ to $p$ contained in $U_{\alpha_1}$.
      $\sigma_1$ exists because $U_{\alpha_1}$ is path-connected, and
      $\gamma_1$ is simple since $U_{\alpha_1}$ is simply-connected. Then
      for $1<i<n$, let
      $\gamma_i=\theta_i\gamma\upharpoonright_{I_i}\sigma_i$, where
      $\theta_i$ is a path in $U_{\alpha_i}$ from $p$ to
      $\gamma((i-1)/n)$, and $\sigma_i$ is a path in $U_{\alpha_i}$ from
      $\gamma(i/n)$ to $p$. Repeating the argument, $\gamma_i$ exists
      and is a simple loop starting and ending at $p$. Finally, for
      $i=n$, let $\gamma_n=\gamma\upharpoonright_{I_n}\sigma_n$. \\

      Since each $\gamma_i$ is a simple loop at $p$, the concatenation of
      them will also be simple loop at $p$, as required.
    \end{proof}

    \item
    \begin{proof}
      We apply results from the previous question. Fix any point $p\in
      S^n$. Without loss of generality we can assume $p$ is the north pole.
      Let $\mathcal{U}=\{U_i\}_{i\in\mathbb{N}}$ be an open cover of $S_n$,
      where for $i>0$, $U_i$ is the open ``cap'' of $p$ that hangs at a
      height of $1/i$ from the south pole, and $U_0$ is the open strip that
      covers the north and south poles, as illustrated in the diagram
      below. Then $\mathcal{U}$ covers $S_n$, each $U_i$ contains $p$, each
      $U_i$ is simply-connected, and each $U_i\cap U_j$ is path-connected.
      So from the previous question, $S^n$ is simply connected.
    \end{proof}
\end{enumerate}
\end{document}
