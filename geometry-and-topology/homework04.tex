\documentclass{article}
\usepackage[left=3cm,right=3cm,top=3cm,bottom=3cm]{geometry}
\usepackage{amsmath,amssymb,amsthm,pgfplots,tikz,mathtools}
\usetikzlibrary{patterns}
\usepackage{color}
\setlength{\parindent}{2mm}
\newcommand{\TODO}[1]{\textcolor{red}{TODO: #1}}

\begin{document}
\title{Geometry and Topology: Homework 4}
\author{Li Ling Ko\\ lko@nd.edu}
\date{\today}
\maketitle

\begin{enumerate}
  \item \begin{proof} We show by induction on $n$ that the fibers of
    $\cup_{k=1}^n\widetilde{X}^{i}$ can be endowed with the structure of a
    CW complex such that $f$ takes the interiors of the $k$-cells in the
    fibers homeomorphically to the interiors of $k$-cells in
    $\cup_{k=1}^n\widetilde{X}^{i}$. For $k=0$, the fiber of each
    $X^0_{\alpha}$ is a union of $0$-cells. These 0-cells are contained in
    disjoint sheets lying above a trivialized neighborhood of
    $X^0_{\alpha}$, and hence can be assigned the discrete topology in
    $\widetilde{X}$. For the inductive step $n+1$, consider the embedding
    $\varphi_\alpha:D^{n+1}\xhookrightarrow{}X$ which embeds an
    $(n+1)$-cell to $X^{n+1}$. This embedding glues the boundary
    $S^n\subset D^{n+1}$ onto $X^n$ and takes the interior of $D^{n+1}$ to
    the interior of an $(n+1)$-cell in $X^{n+1}$. Pick a point $x_0\in
    S^n\subset D^{n+1}$. The fiber of $\varphi_\alpha(x_0)$ is a union of
    points $\{\widetilde{x}_{0,\beta}\}_{\beta\in I}$ in $\widetilde{X}^n$,
    contained in disjoint sheets lying above the trivialized neighborhood
    of $\varphi_\alpha(x_0)$. Then for each $\widetilde{x}_{0,\beta}$,
    consider the lift of $\varphi_\alpha$ to $\widetilde{X}^{n+1}$ that
    sends $x_0$ to $\widetilde{x}_{0,\beta}$. By the lifting criterion,
    because $D^{n+1}$ is contractable, such a lift
    $\widetilde{\varphi_\alpha}$ exists. This lift will glue $D_{n+1}$ at
    the boundary to $\widetilde{X}^n$, forming $\widetilde{X}^{n+1}$, which
    completes the inductive step.
  \end{proof}
\end{enumerate}
\end{document}
