\documentclass{article}
\usepackage[left=3cm,right=3cm,top=3cm,bottom=3cm]{geometry}
\usepackage{amsmath,amssymb,amsthm,pgfplots,tikz,mathtools}
\usepackage[inline]{enumitem}
\usetikzlibrary{patterns}
\usepackage{color}
\setlength{\parindent}{2mm}
\newcommand{\TODO}[1]{\textcolor{red}{TODO: #1}}

\newtheorem{prob}{Problem}

\begin{document}
\title{Geometry and Topology: Homework 6}
\author{Li Ling Ko\\ lko@nd.edu}
\date{\today}
\maketitle

\begin{enumerate}[label={\bf Q\arabic*:}]
  \item Let $\sum_g$ be an oriented genus $g$ surface with a basepoint
    $p\in\sum_g$. Assume that $g\geq2$. Prove that $\pi_1(\sum_g,p)$ is not
    abelian.
    \begin{proof}
    \end{proof}

  \item Let $X$ be a connected graph with vertex set $V(X)$ and edge set
    $E(X)$. Assume that both $V(X)$ and $E(X)$ are finite sets.
    \begin{enumerate}
      \item If $X$ is a tree, prove that $|V(X)|-|E(X)|=1$.
        \begin{proof}
          A tree is defined as an undirected graph where every two vertices
          is connected by exactly one path. First we prove that in a
          connected and finite tree with more than one node, there exist a
          vertex with exactly one edge connected to it: Since the graph is
          connected with more than one node, all vertices must have at
          least one edge connected to it, otherwise the graph would be
          disconnected. Assume by contradiction that all vertices have at
          least two edges connected to it. Then by starting at any vertex,
          we trace a path in the tree by following one of the two edges
          that has not been traversed yet. Because the graph is finite, we
          will eventually traverse the same vertex twice, resulting in a
          cycle. This contradicts the fact that any two vertices are
          connected by exactly one path. \\

          Now we prove by induction on $|V(X)|$ that the assertion is true.
          Base cases $|V(X)|\leq 1$ is trivially true. Given a finite
          connected tree with $|V(X)|=n+1$, from the previous paragraph,
          we can find a vertex with exactly one edge. Removing that vertex
          gives a connected tree with $n$ vertices, so by induction
          hypothesis, that tree will have $n-1$ edges. Adding the
          vertex back again will give us a tree with $n+1$ vertices and
          $n$ edges, so the assertion still holds.
        \end{proof}

      \item For $p\in V(X)$, prove that $\pi_1(X,p)$ is a free group of
        rank $r$ with $|V(X)|-|E(X)|=1-r$.
        \begin{proof}
          From Proposition 1A.1 of Hatcher, $X$ contains a maximal tree
          $T$ which would include every vertex in the graph. Choose any
          vertex $v$ in $T$ to be the base point, and consider the open
          cover $\mathcal{U}$ of $X$ given by
          \begin{equation*}
            \mathcal{U} := \{T\cup\{e\}:\; e\in E(X)-E(T)\}.
          \end{equation*}
          Each open set in the cover contains the base point and the
          intersection of every pair of sets in the cover is the tree $T$
          and is therefore path-connected. Hence the conditions for the van
          Kampen theorem are satisfied and we can apply the van Kampen
          theorem to get the fundamental group of $X$. \\

          Now the fundamental group of $T$ is the trivial group 1 because
          every pair of vertices in a connected tree is connected by only
          one path. Also, each $T\cup\{e\}$ in $\mathcal{U}$ has
          fundamental group $F_1=\mathbb{Z}$ because there is only one loop
          in $T\cup\{e\}$. Then applying van Kampen's theorem, the
          fundamental group of $X$ is the free product of rank
          $r=|E(X)-E(T)|$. From the previous part of this question,
          $|E(T)|=|V(T)|-1$, and since $|V(T)|=|V(X)|$, we get the relation
          $|V(X)|-|E(X)|=1-r$ as required.
        \end{proof}

      \item Let $F_n$ be a free group of rank $n$ and let $G\subset F_n$ be
        a subgroup. As we showed in class, $G$ is a free group; let $m$ be
        its rank. Assume that $r=[F_n:G]$ is finite. Find a formula for $m$
        in terms of $n$ and $r$.

        \begin{proof}
          A free group of rank $n$ is a cover of wedge of $n$ loops, i.e.
          $F_n=\pi_1(X,p)$, where
          \begin{equation*}
            X = \underbrace{S^1\vee\ldots\vee S^1}_{n\;\text{times}}.
          \end{equation*}
          Now since $G$ is a subgroup of $F_n$, by the one-to-one
          correspondence between subgroups of fundamental groups and
          connected covers, there is a connected cover $\tilde{X_G}$ of $X$
          such that $\pi_1(\tilde{X_G})=G$. Now from Hatcher, we know that
          a connected cover of a graph is also a graph, and the also the
          degree of the cover will equal to the index $r=[F_n:G]$ of the
          subgroup $G$ of $F_n$. Hence the connected cover will be a graph
          with $r$ vertices and $nr$ edges. Now from the previous part of
          this question, such a graph will have a rank of $m=1-r+nr$.
        \end{proof}
    \end{enumerate}
\end{enumerate}
\end{document}
