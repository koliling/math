\documentclass{article}
\usepackage[left=3cm,right=3cm,top=3cm,bottom=3cm]{geometry}
\usepackage{amsmath,amssymb,amsthm,pgfplots,tikz,mathtools}
\usepackage[inline]{enumitem}
\usetikzlibrary{patterns}
\usepackage{color}
\setlength{\parindent}{2mm}
\newcommand{\TODO}[1]{\textcolor{red}{TODO: #1}}

\newtheorem{prob}{Problem}

\begin{document}
\title{Geometry and Topology: Homework 6}
\author{Li Ling Ko\\ lko@nd.edu}
\date{\today}
\maketitle

\begin{enumerate}[label={\bf Q\arabic*:}]
  \item Let $\sum_g$ be an oriented genus $g$ surface with a basepoint
    $p\in\sum_g$. Assume that $g\geq2$. Prove that $\pi_1(\sum_g,p)$ is not
    abelian.
    \begin{proof}
      We find $\pi_1(\sum_g,p)$ by van Kampen's theorem. $\sum_g$ is
      constructed from gluing the edges of a 2-cell in the manner described
      in Figure 1a. Consider the open cover of $\sum_g=U_1\cap U_2$, where
      $U_1$ is the interior of the 2-cell, and $U_2$ is the 2-cell with an
      interior point removed, as illustrated in Figure 1b. Let the base
      point $p$ be any point in the intersection $U_1\cap U_2$. \\

      Now $U_1\cap U_2$ is an open donut, which is a strong deformation
      retract to $S^1$, and so $\pi_1(U_1\cap
      U_2,p)=\pi_1(S^1,p)=\mathbb{Z}$. As for $U_1$, it is a 2-cell, and
      so it has trivial fundamental group since it is contractible.
      Finally, for $U_2$, from Figure 1c, it is a strong deformation
      retract to the wedge of $2g$ copies of $S^1$. Hence $\pi_1(U_2,p)$ is
      the free group
      \begin{equation*}
        \pi_1(U_2,p) = F_{2g} = \langle a_1,b_1,\ldots,a_g,b_g|\;\rangle.
      \end{equation*}

      Referring to the commutative diagram in Figure 1d, the image
      $i_2(\pi_1(U_2))$ can be described by the word
      $a_1b_1a_1^{-1}b_1^{-1}\ldots a_gb_ga_g^{-1}b_g^{-1}$ in $F_{2g}$. On
      the other hand, since $\pi_1(U_1)$ is trivial, the image of $i_1$ is
      trivial. Then by van Kempen's theorem, the fundamental group of
      $\sum_g$ is
      \begin{equation*}
        \pi_1(\sum_g,p) = \langle a_1,b_1,\ldots,a_g,b_g|\;
        a_1b_1a_1^{-1}b_1^{-1}\ldots a_gb_ga_g^{-1}b_g^{-1}=1\rangle.
      \end{equation*}

      To show that this fundamental group is non-abelian, it suffices to
      find a surjective homomorphism from $\pi_1(\sum_g,p)$ to the dihedral
      group $D_8$, because $D_8$ is non-abelian and surjective
      homomorphisms should preserve commutativity. Consider the map
      $\phi:\pi_1(\sum_g,p)\rightarrow D_8$ defined by:
      \begin{equation*}
        \phi(g) :=
        \begin{cases}
          r & \text{if}\; g=a_1\; \text{or}\; a_2, \\
          s & \text{otherwise}. \\
        \end{cases}
      \end{equation*}

      $\phi$ can be extended to a homomorphism since it defines a map for
      all generators. This homomorphism will be subjective because all
      generators $r$ and $s$ of the range $D_8$ has a pre-image.
      Furthermore, this homomorphism is well-defined because it preserves
      the relations on $\pi_1(\sum_g,p)$; we can verify that
      \begin{equation*}
        \phi(a_1b_1a_1^{-1}b_1^{-1}\ldots a_gb_ga_g^{-1}b_g^{-1}) =
        1_{D_8}.
      \end{equation*}
      Hence $\pi_1(\sum_g,p)$ is not abelian.
    \end{proof}

  \item Let $X$ be a connected graph with vertex set $V(X)$ and edge set
    $E(X)$. Assume that both $V(X)$ and $E(X)$ are finite sets.
    \begin{enumerate}
      \item If $X$ is a tree, prove that $|V(X)|-|E(X)|=1$.
        \begin{proof}
          A tree is defined as an undirected graph where every two vertices
          is connected by exactly one path. First we prove that in a
          connected and finite tree with more than one node, there exist a
          vertex with exactly one edge connected to it: Since the graph is
          connected with more than one node, all vertices must have at
          least one edge connected to it, otherwise the graph would be
          disconnected. Assume by contradiction that all vertices have at
          least two edges connected to it. Then by starting at any vertex,
          we trace a path in the tree by following one of the two edges
          that has not been traversed yet. Because the graph is finite, we
          will eventually traverse the same vertex twice, resulting in a
          cycle. This contradicts the fact that any two vertices are
          connected by exactly one path. \\

          Now we prove by induction on $|V(X)|$ that the assertion is true.
          Base cases $|V(X)|\leq 1$ is trivially true. Given a finite
          connected tree with $|V(X)|=n+1$, from the previous paragraph,
          we can find a vertex with exactly one edge. Removing that vertex
          gives a connected tree with $n$ vertices, so by induction
          hypothesis, that tree will have $n-1$ edges. Adding the
          vertex back again will give us a tree with $n+1$ vertices and
          $n$ edges, so the assertion still holds.
        \end{proof}

      \item For $p\in V(X)$, prove that $\pi_1(X,p)$ is a free group of
        rank $r$ with $|V(X)|-|E(X)|=1-r$.
        \begin{proof}
          From Proposition 1A.1 of Hatcher, $X$ contains a maximal spanning
          tree $T$ which would include every vertex in the graph. We can
          add a small initial interval of edges that are not in $T$ into
          the spanning tree if necessary to ensure that $T$ is open. Choose
          any vertex $v$ in $T$ to be the base point, and consider the open
          cover $\mathcal{U}$ of $X$ given by
          \begin{equation*}
            \mathcal{U} := \{T\cup\{e\}:\; e\in E(X)-E(T)\}.
          \end{equation*}
          Each open set in the cover contains the base point and the
          intersection of every pair of sets in the cover is the tree $T$
          and is therefore path-connected. Hence the conditions for the van
          Kampen theorem are satisfied and we can apply the van Kampen
          theorem to get the fundamental group of $X$. \\

          Now the fundamental group of $T$ is the trivial group 1 because
          every pair of vertices in a connected tree is connected by only
          one path. Also, each $T\cup\{e\}$ in $\mathcal{U}$ has
          fundamental group $F_1=\mathbb{Z}$: consider the loop that
          contains edge $e=(u,v)$ and the edges and vertices in only path
          between $u$ and $v$ in the spanning tree $T$; then $T\cup\{e\}$
          can be retracted to this loop, which is homomorphic to $S^1$, and
          thus has fundamental group $F_1=\mathbb{Z}$. Then applying van
          Kampen's theorem, the fundamental group of $X$ is the free
          product of rank $r=|E(X)-E(T)|$. From the previous part of this
          question, $|E(T)|=|V(T)|-1$, and since $|V(T)|=|V(X)|$, we get
          the relation $|V(X)|-|E(X)|=1-r$ as required.
        \end{proof}

      \item Let $F_n$ be a free group of rank $n$ and let $G\subset F_n$ be
        a subgroup. As we showed in class, $G$ is a free group; let $m$ be
        its rank. Assume that $r=[F_n:G]$ is finite. Find a formula for $m$
        in terms of $n$ and $r$.

        \begin{proof}
          A free group of rank $n$ is a cover of wedge of $n$ loops, i.e.
          $F_n=\pi_1(X,p)$, where
          \begin{equation*}
            X = \underbrace{S^1\vee\ldots\vee S^1}_{n\;\text{times}}.
          \end{equation*}
          Now since $G$ is a subgroup of $F_n$, by the one-to-one
          correspondence between subgroups of fundamental groups and
          connected covers, there is a connected cover $\tilde{X_G}$ of $X$
          such that $\pi_1(\tilde{X_G})=G$. Now from Hatcher, we know that
          a connected cover of a graph is also a graph, and the also the
          degree of the cover will equal to the index $r=[F_n:G]$ of the
          subgroup $G$ of $F_n$. Hence the connected cover will be a graph
          with $r$ vertices and $nr$ edges. Now from the previous part of
          this question, such a graph will have a rank of $m=1-r+nr$.
        \end{proof}
    \end{enumerate}

  \item Let $X\subset\mathbb{R}^n$ be a finite set of $k$ points and let
    $p\in\mathbb{R}^n\setminus X$.
    \begin{enumerate}
      \item If $n=2$, then calculate $\pi_1(\mathbb{R}^n\setminus X,p)$.
        \begin{proof}
          $\mathbb{R}^2\setminus X$ is a strong deformation retract to the
          wedge of $k$ circles at $p$, which has fundamental group $F_k$.
          Since strong deformation retracts preserve fundamental group,
          $\pi_1(\mathbb{R}^n\setminus X,p)=F_k$.
        \end{proof}

      \item If $n\geq3$, then prove that $\pi_1(\mathbb{R}^n\setminus
        X,p)=1$.
        \begin{proof}
          Similar to above, $\mathbb{R}^2\setminus X$ is a strong
          deformation retract to the wedge of $k$ $S^{n-1}$'s at $p$. Now
          for $n\geq 3$, $S^{n-1}$ has the trivial fundamental group. Also,
          since each $S^{n-1}$ is path-connected and a neighbourhood $U$
          of $p$ is contractible, the fundamental group
          $\pi_1(\mathbb{R}^n\setminus X,p)=1$.
        \end{proof}
    \end{enumerate}

  \item Let $T^2=S^1\times S^1$ and let $X$ be the quotient of $T^2\sqcup
    T^2$ that identifies the circles $S^1\times1$ in both tori
    homeomorphically. Calculate the fundamental group of $X$.
    \begin{proof}
      By stacking the donuts on top of each other like shown in Figure 4,
      we can see that $X=(S^1\vee S^1)\times S^1$. Then since taking
      fundamental group preserves products, we get
      \begin{align*}
        \pi_1(X,p) &= \pi_1((S^1\vee S^1)\times S^1,p) \\
                   &= \pi_1((S^1\vee S^1),p)\times \pi_1(S^1,p) \\
                   &= (\pi_1(S^1,p)\ast\pi_1(S^1,p))\times \pi_1(S^1,p) \\
                   &= (\mathbb{Z}\ast\mathbb{Z})\times\mathbb{Z}.
      \end{align*}
    \end{proof}

  \item Let $W=S^1\vee S^1$ and let $p\in W$ be a wedge point. Identify
    $\pi_1(W,p)$ with the free group on $a$ and $b$, where $a$ goes around
    one $S^1$ and $b$ goes around the other one. Construct three connected
    4-fold covers of W that are distinct up to covering space equivalence,
    including at least 2 irregular cover. For each of these three covers,
    describe the covering map, say whether or not the cover is regular and
    give a free basis for the corresponding subgroup of $\pi_1(W,p)$.

    \begin{proof}
      The first covering map is a regular one, shown in Figure 5a. The
      basis is given by
      \begin{equation*}
        \langle a,b^4,bab^{-1},b^2ab^{-2},b^3ab^{-3}\rangle.
      \end{equation*}

      The second covering map is also regular, shown in Figure 5b. The
      basis is given by
      \begin{equation*}
        \langle b^4,ba,b^2ab^{-1},b^3ab^{-1},ab^{-3}\rangle.
      \end{equation*}

      The third covering is irregular, and is shown in Figure 5c. It is
      irregular because there is no automorphism that can swap points $p_1$
      and $p_2$. Its basis is given by
      \begin{equation*}
        \langle
        a,b^2,ba^2b^{-1},bab^2a^{-1}b^{-1},babab^{-1}a^{-1}b^{-1}\rangle.
      \end{equation*}
    \end{proof}

  \item Let $\{p_1,\ldots,p_n\}$ be a set of $n$ distinct points in $S^2$
    and let $X$ be the quotient space of $S^2$ that identifies all the
    $p_i$ to a single point. Let $q\in X$ be a basepoint. Calculate
    $\pi_1(X,q)$.

    \begin{proof}
      In the first homework assignment, we showed that $X$ can be
      constructed by gluing two disks $D^2$ according to Figure 6a. By
      extending the disks slightly, these two disks can form an open cover
      $X=U_1\cup U_2$ of $X$. Taking a common point in the two open sets as
      base point, the fundamental groups of $U_1$ and $U_2$ are given as:
      \begin{align*}
        \pi_1(U_1,p) &= \langle a_1,\ldots,a_n|\; a_1\cdots
          a_{n-1}=a_n\rangle, \\
        \pi_1(U_2,p) &= \langle b_1,\ldots,b_n|\; b_1\cdots
          b_{n-1}=b_n\rangle. \\
      \end{align*}
      Their intersection $U_1\cap U_2$ will look shown in Figure 6b, and
      will have fundamental group 
      \begin{align*}
        \pi_1(U_1\cap U_2,p) &= \langle c_1,\ldots,c_n|\; \rangle. \\
      \end{align*}

      Applying van Kampen's theorem to quotient out the edges $U_1\cap U_2$
      glued together, we get
      \begin{align*}
        \pi_1(U_1\cup U_2,p)  &= \langle
          a_1,\ldots,a_n,b_1,\ldots,b_n,c_1,\ldots,c_n|\; \\
          & a_1\cdots a_{n-1}=a_n, \\
          & b_1\cdots b_{n-1}=b_n, \\
          & a_1=b_1=c_1,\ldots a_n=b_n=c_n\rangle \\
          &= \langle a_1,\ldots,a_n|\; a_1\cdots a_{n-1}=a_n\rangle. \\
      \end{align*}
    \end{proof}
\end{enumerate}
\end{document}
