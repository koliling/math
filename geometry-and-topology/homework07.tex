\documentclass{article}
\usepackage[left=3cm,right=3cm,top=3cm,bottom=3cm]{geometry}
\usepackage{amsmath,amssymb,amsthm,pgfplots,tikz,mathtools}
\usepackage[inline]{enumitem}
\usetikzlibrary{patterns}
\usepackage{color}
\setlength{\parindent}{2mm}
\newcommand{\TODO}[1]{\textcolor{red}{TODO: #1}}
\newtheorem{prob}{Problem}

\begin{document}
\title{Geometry and Topology: Homework 7}
\author{Li Ling Ko\\ lko@nd.edu}
\date{\today}
\maketitle

\begin{enumerate}[label={\bf Q\arabic*:}]
  \item Prove that the two atlases for $S^n$ on pages 2 and 3 of the notes
    are equivalent.

    \begin{proof}
      To prove that the atlases are equivalent, we need to show that the
      transition functions $\tau_{2,1}$ from any $U_1$ in the first atlas
      to any $U_2$ in the second atlas is smooth, and the inverse function
      $\tau_{1,2}$ is also smooth. Without loss of generality, assume $U_1$
      is $S^n\setminus\{(0,0,\ldots,1)\}$, and $U_2$ is either the top
      hemisphere $U_{x_{n+1}>0}$, the bottom hemisphere $U_{x_{n+1}<0}$, or
      one of the side hemispheres $U_{x_1<0}$. \\

      When $U_2$ is the top hemisphere $U_{x_{n+1}>0}$, we have
      \begin{align*}
        \tau_{2,1}(x_1,\ldots,x_{n}) &= (x_1,\ldots,x_{n})t_1, &
        \text{where}\; \|(x_1,\ldots,x_{n+1})\|^2\geq1, \\
        \tau_{1,2}(x_1,\ldots,x_{n}) &= (x_1,\ldots,x_{n})t_2, &
        \text{where}\; 0<\|(x_1,\ldots,x_{n+1})\|^2\leq1, \\
      \end{align*}
      and $t_1=2/\sqrt{x_1^2+\ldots+x_{n}^2+1}$ and
      $t_2=1/(1-\sqrt{1-x_1^2-\ldots-x_{n}^2})$. So both $\tau_{2,1}$ and
      $\tau_{1,2}$ are rational functions with non-zero denominators at
      their respective domains, so they are both smooth. \\

      When $U_2$ is the top hemisphere $U_{x_{n+1}>0}$, the functions
      $\tau_{2,1}$ and $\tau_{1,2}$ are the same as above
      except that the domain of $\tau_{1,2}$ includes the origin. At these
      domains, the functions remain smooth. \\

      Finally, consider the case when $U_2$ is the side hemisphere
      $U_{x_1<0}$. Then
      \begin{align*}
        \tau_{2,1}(x_1,\ldots,x_{n}) &= (1-t_1,x_2t_1,\ldots,x_{n}t_1), &
        \text{where}\; \|(x_1,\ldots,x_{n+1})\|^2\geq1, \\
        \tau_{1,2}(x_1,\ldots,x_{n}) &= (1-t_2,x_2t_2,\ldots,x_{n}t_2), &
        \text{where}\; 0<\|(x_1,\ldots,x_{n+1})\|^2\leq1, \\
      \end{align*}
      and $t_1=2/\sqrt{x_1^2+\ldots+x_{n}^2+1}$ and
      $t_2=1/(1-\sqrt{1-x_1^2-\ldots-x_{n}^2})$. So both $\tau_{2,1}$ and
      $\tau_{1,2}$ are rational functions with non-zero denominators at
      their respective domains, so they are both smooth. \\
    \end{proof}

  \item Define a function $f:S^n\rightarrow\mathbb{R}$ via the formula
    \begin{equation*}
      f(x_1,\ldots,x_{n+1}) = \frac{x_{n+1}^2}{7+e^{x_1}}.
    \end{equation*}
    Prove that $f$ is smooth directly.

    \begin{proof}
      In the lectures it was mentioned that the choice of the atlas doesn't
      matter. We use the atlas defined on page 2 of the notes. Given
      $p=(x_1,\ldots,x_n)$ in $S^n$, depending on the chart $\phi_i$ in the
      atlas we use, we have $\phi_i(p)=(x_1,\ldots,\hat{x_i},\ldots,x_n)$.
      So if $i\not\in\{1,n+1\}$, we have
      \begin{equation*}
        f(\phi_i^{-1}(p)) = \frac{x_{n}^2}{7+e^{x_1}},
      \end{equation*}
      which is smooth because it is a composition of smooth functions and
      that the denominator is always non-zero. \\

      If $i=n+1$, we have
      \begin{equation*}
        f(\phi_{n+1}^{-1}(p)) = \frac{1-x_1^2-\ldots-x_n^2}{7+e^{x_1}},
      \end{equation*}
      which is still smooth because it is a composition of smooth functions
      and that the denominator is always non-zero. \\

      Finally, if $i=1$, we have
      \begin{equation*}
        f(\phi_i^{-1}(p)) =
        \frac{x_{n}^2}{7+e^{\sqrt{1-x_1^2-\ldots-x_n^2}}},\;
        \text{where}\;\; \|(x_1,\ldots,x_n)\|^2\leq1,
      \end{equation*}
      which is still smooth because it is a composition of smooth functions
      and that the denominator is always non-zero in the given domain. \\
    \end{proof}
\end{enumerate}
\end{document}
