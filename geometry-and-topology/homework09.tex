\documentclass{article}
\usepackage[left=3cm,right=3cm,top=3cm,bottom=3cm]{geometry}
\usepackage{amsmath,amssymb,amsthm,pgfplots,tikz,mathtools}
\usepackage[inline]{enumitem}
\usetikzlibrary{patterns}
\usepackage{color}
\setlength{\parindent}{2mm}
\newcommand{\TODO}[1]{\textcolor{red}{TODO: #1}}
\newtheorem{prob}{Problem}

\begin{document}
\title{Geometry and Topology: Homework 9}
\author{Li Ling Ko\\ lko@nd.edu}
\date{\today}
\maketitle

\begin{enumerate}[label={\bf Q\arabic*:}]
  \item
  \item
    \begin{enumerate}
      \item Let $\overline{v}=[v]\in T_pM$, and consider the map
        \begin{align*}
          \delta_{\bar{v}}:C^\infty(M^n)\rightarrow\mathbb{R},\;\; f\mapsto
          D_p[f\circ\phi^{-1}](v), \\
        \end{align*}
        where $(U,\phi)$ is a chart containing $p$ and $v\in
        T_{\phi(p)}\phi(U)$. We show that $\delta_{\bar{v}}$ is a
        derivation of $C^\infty(M^n)$ at $p$. \\

        We first show that $\delta_{\bar{v}}$ is $\mathbb{R}$-linear: We
        can assume wlog that $\phi(p)=0$. Let $f,g\in C^\infty(M^n)$, and
        $r\in\mathbb{R}$. Also, let $\sigma_r$ denote the map
        $\sigma_r:\mathbb{R}\rightarrow\mathbb{R}$ that sends
        $x\in\mathbb{R}$ to $rx\in\mathbb{R}$. Then
        \begin{align*}
          \delta_{\bar{v}}(rf+g)  &=\delta_{\bar{v}}(\sigma_r\circ f+g) \\
            &=D_p[(\sigma_r\circ f+g)\circ\phi^{-1}](v) \\
            &=D_p[(\sigma_r\circ f\circ\phi^{-1} +g\circ\phi^{-1}](v) \\
            &=rD_p[f\circ\phi^{-1}](v) +D[g\circ\phi^{-1}](v) \\
            &=r\delta_{\bar{v}}(f) +\delta_{\bar{v}}(g), \\
        \end{align*}
        and so $\delta_{\bar{v}}$ is $\mathbb{R}$-linear.
    \end{enumerate}
\end{enumerate}
\end{document}
