\documentclass{article}
\usepackage[left=3cm,right=3cm,top=3cm,bottom=3cm]{geometry}
\usepackage{amsmath,amssymb,amsthm,pgfplots,tikz,mathtools}
\usepackage[inline]{enumitem}
\usetikzlibrary{patterns}
\usepackage{color}
\setlength{\parindent}{2mm}
\newcommand{\TODO}[1]{\textcolor{red}{TODO: #1}}
\newtheorem{prob}{Problem}

\begin{document}
\title{Geometry and Topology: Homework 9}
\author{Li Ling Ko\\ lko@nd.edu}
\date{\today}
\maketitle

\begin{enumerate}[label={\bf Q\arabic*:}]
  \item $f(z)$ is a nonconstant polynomial and $g:S^2\to S^2$ is defined as
    follows: Let $\infty=(0,0,1)\in S^2$ and
    $\phi:S^2\setminus\{\infty\}\to\mathbb{C}$ be stereographic projection.
    Then $g$ is defined via the formula
    \[g(p)=\begin{cases} \infty &\mbox{if }p=\infty\\ \phi^{-1}(f(\phi(p)))
      &\mbox{if }p\not=\infty\end{cases}\]
    Prove the following two facts:

    \begin{enumerate}
      \item Prove that $g$ is a smooth function.
      \item Prove that the only critical points of $g$ are the images under
        $\phi$ of the zeroes of $f'(z)$ together with possibly the point at
        $\infty$.
    \end{enumerate}

  \item Let $M^n$ be a smooth manifold and let $p\in M^n$ be a point. Let
    $C^\infty(M^n)$ be the ring of smooth functions $f:M^n\to\mathbb{R}$. A
    derivation of $\mathbb{C}^\infty(M^n)$ at $p$ is an $\mathbb{R}$-linear
    map $\sigma:C^\infty(M^n)\to\mathbb{R}$ such that
    \[\sigma(fg)=g(p)\cdot \sigma(f)+f(p)\cdot\sigma(g)\] let
    $\text{Der}_pM^n$ be the vector space of derivations of $C^\infty(M^n)$
    at $p$.

    \begin{enumerate}
      \item If $\vec{v}\in T_pM^n$, then prove that the derivative at $p$
        in the direction of $\vec{v}$ is a derivation at $p$.
        \begin{proof}
          Let $\vec{v}=[v]\in T_pM^n$, and consider the map
          \begin{align*}
            \delta_{\vec{v}}:C^\infty(M^n)\rightarrow\mathbb{R},\;\; f\mapsto
            D_p[f\circ\phi^{-1}](v), \\
          \end{align*}
          where $(U,\phi)$ is a chart containing $p$ and $v\in
          T_{\phi(p)}\phi(U)$. We show that $\delta_{\vec{v}}$ is a
          derivation of $C^\infty(M^n)$ at $p$. \\

          We first show that $\delta_{\vec{v}}$ is $\mathbb{R}$-linear: We
          can assume wlog that $\phi(p)=\overline{0}$. Let $f,g\in
          C^\infty(M^n)$, and $r\in\mathbb{R}$. Also, let $\sigma_r$ denote
          the map $\sigma_r:\mathbb{R}\rightarrow\mathbb{R}$ that sends
          $x\in\mathbb{R}$ to $rx\in\mathbb{R}$. Then
          \begin{align*}
            \delta_{\vec{v}}(rf+g)  &=\delta_{\vec{v}}(\sigma_r\circ f+g) \\
              &=D_p[(\sigma_r\circ f+g)\circ\phi^{-1}](v) \\
              &=D_p[(\sigma_r\circ f\circ\phi^{-1} +g\circ\phi^{-1}](v) \\
              &=rD_p[f\circ\phi^{-1}](v) +D_p[g\circ\phi^{-1}](v) \\
              &=r\delta_{\vec{v}}(f) +\delta_{\vec{v}}(g), \\
          \end{align*}
          and so $\delta_{\vec{v}}$ is $\mathbb{R}$-linear. Next, we check
          that $\delta_{\vec{v}}$ satisfies the product rule. We have
          \begin{align*}
            \delta_{\vec{v}}(f\cdot g) &=D_p[(f\cdot g)\circ\phi^{-1}](v) \\
              &=D_p[(f\circ\phi^{-1})\cdot(g\circ\phi^{-1})](v) \\
              &=(g\circ\phi^{-1})(\overline{0})D_p(f\circ\phi^{-1})(v)+
                (f\circ\phi^{-1})(\overline{0})D_p(g\circ\phi^{-1})(v) \\
              &=g(p)\delta_{\vec{v}}(f) +f(p)\delta_{\vec{v}}(g). \\
          \end{align*}
          Therefore, $\delta_{\vec{v}}\in\text{Der}_pM^n$.
        \end{proof}

      \item If $f\in C^\infty(M^n)$ is a constant function and $\sigma\in
        \text{Der}_pM^n$ , then prove that $\sigma(f)=0$.
        \begin{proof}
          Given $r\in\mathbb{R}$, let $f_r:M^n\rightarrow\mathbb{R}$
          denote the constant function that sends $q\in M^n$ to $r$. Now
          \begin{align*}
            \delta(f_1) &=\delta(f_1f_1) \\
              &=f_1(p)\delta(f_1) +\delta(f_1)f_1(p) \\
              &=\delta(f_1) +\delta(f_1) \\
              &=2\delta(f_1), \\
          \end{align*}
          so rearranging gives $\delta(f_1)=0$. Hence by the
          $\mathbb{R}$-linear property of $\delta$, we have
          \[\delta(f_r) =\delta(rf_1) =r\delta(f_1)=0,\] which implies that
          $\delta$ vanishes on constant functions.
        \end{proof}

      \item If $f,g\in C^\infty(M^n)$ are such that there exists a
        neighborhood $U$ of $p$ with $f|_U=g|_U$ and
        $\sigma\in\text{Der}_pM^n$, then prove that $\sigma(f)=\sigma(g)$.
        \begin{proof}
          Denote $C=M^n\setminus U$, and $V=M^n\setminus\{p\}$. Then $V$
          is an open set containing $C$, because $\{p\}$ is closed from the
          Hausdorff property of $M^n$. Then from the existence of a strong
          bump function (Lemma 1.4), there is a smooth function
          $h:M^n\rightarrow\mathbb{R}$ such that $h(x)=1$ for all $x\in C$
          and $\text{supp}(h)\subseteq V$. So we have $h(p)=0$, $(f-g)(x)=0$
          for all $x\in U$, and $h(x)=1$ for all $x\in V$, which implies that
          $(f-g)(x)h(x)=(f-g)(x)$ for all $x\in M^n$, or equivalently,
          $(f-g)h=f-g$. Thus,
          \begin{align*}
            \delta(f-g) &=\delta((f-g)h) \\
              &=\delta(f-g)h(p)+\delta(h)(f(p)-h(p)) \\
              &=\delta(f-g)\cdot0+\delta(h)\cdot0 & (\because
                h(p)=f(p)-g(p)=0) \\
              &=0. \\
          \end{align*}
          Then rearranging we get $\delta(f)=\delta(g)$, as required.
        \end{proof}

      \item Define a map $\Psi:T_pM^n\to\text{Der}_pM^n$ by letting
        $\Psi(\vec{v})$ be the directional derivative at $p$ in the direction
        of $\vec{v}$. Prove that $\Psi$ is injective.
        \begin{proof}
          We define $\Psi:T_pM^n\rightarrow\text{Der}_pM^n$ by
          $\Psi(\vec{v})=\delta_{\vec{v}}$, and let $\vec{u},\vec{v}\in
          T_pM^n$ be such that $\delta_{\vec{u}}=\delta_{\vec{v}}$. Let
          $(U,\phi)$ be a chart around $p$. We can assume that
          $\phi(p)=\bar{0}$. Also, let $u,v\in T_{\bar{0}}\phi(U)$. Then let
          $B(\bar{0},r)\subseteq\phi(U)$ be an open ball of radius $r$ in
          $\phi(U)$ around $\bar{0}$, where $r>0$. Let
          $U'=\phi^{-1}(B(\bar{0},r/4))$, and let
          $C=\phi^{-1}(\overline{B(\bar{0},r/2)})$. Note that $U'$ is open
          and $C$ is closed from the continuity of $\phi$. Then for
          $i\in\{1,\ldots,n\}$, define $\phi_i:C\rightarrow\mathbb{R}$ by
          $\phi_i(x)=\pi_i(\phi(x))$, where
          $\pi_i:\mathbb{R}^n\rightarrow\mathbb{R}$ is the projection map. \\

          Then by extension of smooth functions (Lemma 1.5), there exists a
          smooth function $f_i:M^n\rightarrow\mathbb{R}$ extending $\phi_i$
          and whose support lies in $U$. Then $f_i\in C^\infty(M^n)$, and
          since $\delta_{\vec{u}}(f_i)=\delta_{\vec{v}}(f_i)$ by assumption,
          we get
          \begin{equation} \label{eqn:D}
            D_p(f_i\circ\phi^{-1})(u)= D_p(f_i\circ\phi^{-1})(v).
          \end{equation}

          Now since $f_i\restriction U'=\phi_i\restriction U'$, we have
          \[f_i\circ\phi^{-1}\restriction B(\bar{0},r/4)=
          \phi_i\circ\phi^{-1}\restriction B(\bar{0},r/4),\]
          and therefore
          \begin{align*}
            \left.\frac{\partial(f_i\circ\phi^{-1})}{\partial
              x_j}\right\rvert_{\bar{0}} &=
              \left.\frac{\partial(\phi_i\circ\phi^{-1})}{\partial
              x_j}\right\rvert_{\bar{0}} \\
            &=\left.\frac{\partial(0,\ldots,0,x_i,0,\ldots,0)}{\partial
              x_j}\right\rvert_{\bar{0}} \\
            &=\begin{cases}
              1,\; \text{if}\; i=j \\
              0,\; \text{if}\; i\neq j \\
            \end{cases} \\
          \end{align*}
          Thus, $D_p(f_i\circ\phi^{-1})$ has the matrix representation
          $[0,\ldots,0,1,0,\ldots,0]$, where the $1$ is at the $i$th
          position. And so for each $i\in\{1,\ldots,n\}$,
          \begin{align*}
            u_i &=[0,\ldots,0,1,0,\ldots,0](u) &(\text{where 1 is at the
              $i$th position}) \\
              &=D_p(f_i\circ\phi^{-1})(u) \\
              &=D_p(f_i\circ\phi^{-1})(v) &(\text{equation~\ref{eqn:D}}) \\
              &=[0,\ldots,0,1,0,\ldots,0](v) &(\text{where 1 is at the
              $i$th position}) \\
              &=v_i.
          \end{align*}
          Thus $\vec{u}=\vec{v}$, which means $\Psi$ is injective.
        \end{proof}

      \item Prove that the map $\Psi$ in the previous step is surjective.
        \begin{proof}
          Let $\delta$ be an arbitrary derivative in $\text{Der}_pM^n$.
          Let $(U,\phi)$ be a chart around $p\in M^n$. We can assume wlog
          that $\phi(p)=0$. Then since
          $f\circ\phi^{-1}:\phi(U)\rightarrow\mathbb{R}$ is smooth, by
          Taylor's theorem, there exists some $r\in\mathbb{R}$ and smooth
          functions $f_1,\ldots,f_n:\mathbb{R}^n\rightarrow\mathbb{R}$ such
          that
          \begin{align*}
            (f\circ\phi^{-1})(\bar{x}) &=
              r+\sum_{i=1}^nx_if_i(\bar{x}) \\
            &=r+\sum_{i=1}^n\pi_i(\bar{x})f_i(\bar{x}), \\
          \end{align*}
          where $\pi_i:\mathbb{R}^n\rightarrow\mathbb{R}$ is the $i$th
          projection map. \\

          Now since $\phi(U)\subseteq\mathbb{R}^n$ is open, let
          $B(\bar{0},d)\subseteq\phi(U)$ be an open ball of radius $d$
          around $\bar{0}$, and let
          $U'=\phi^{-1}(\overline{B(\bar{0},d/2)})\subseteq U$. Then by
          Lemma 1.5, we extend functions $\pi_i\circ\phi$ and
          $f_i\circ\phi$ to smooth functions
          $\Pi_i:M^n\rightarrow\mathbb{R}$ and
          $F_i:M^n\rightarrow\mathbb{R}$ such that $\Pi_i\restriction
          U'=\pi\circ\phi$ and $F_i\restriction U'=g_i\circ\phi$, and
          $\text{supp}(\Pi)\subseteq U$ and $\text{supp}(F_i)\subseteq U$.
          \\

          Then we have
          \begin{align*}
            \delta(f) &=\delta(f\restriction U) \\
              &=\delta(((f\circ\phi^{-1})\circ\phi)\restriction U) \\
              &=\delta(r+\sum_{i=1}^n(\pi_i\circ\phi)(f_i\circ\phi)) &
                (\because f\circ\phi^{-1} =r+\sum_{i=1}^n\pi_if_i) \\
              &=\delta(r+\sum_{i=1}^n\Pi_iF_i) & \\
              &=\delta(r)+\sum_{i=1}^n\delta(\Pi_iF_i) \\
              &=\sum_{i=1}^n\left[\delta(\Pi_i)F_i(p)
                +\Pi_i(p)\delta(F_i)\right] \\
              &=\sum_{i=1}^n\delta(\Pi_i)F_i(p) & (\because
                \Pi_i(p)=(\pi_i\circ\phi)(p)=\pi_i(\bar{0})=\bar{0}) \\
              &=\sum_{i=1}^n\delta(\Pi_i)(f_i\circ\phi)(p) \\
              &=\sum_{i=1}^n\delta(\Pi_i)f_i(\bar{0}) \\
              &=\sum_{i=1}^n\delta(\Pi_i)
                \left.\frac{\partial}{\partial
                x_i}\left[\sum_{j=1}^nx_jf_j(\bar{x})\right]\right
                \rvert_{\bar{0}} \\
              &=\sum_{i=1}^n\delta(\Pi_i)
                \left[\left.\frac{\partial(f\circ\phi^{-1})}{\partial
                x_i}\right\rvert_{\bar{0}}\right] \\
              &=D_p(f\circ\phi^{-1})(v) &(\text{for some}\;
                v=(\delta(\pi_1),\ldots,\delta(\pi_n))\in T_pM^n) \\
              &=\delta_{\vec{v}}(f). \\
          \end{align*}
          Thus, $\delta=\Psi(\vec{v})$, and since
          $\delta\in\text{Der}_pM^n$ was arbitrary, $\Psi$ is surjective.
        \end{proof}
    \end{enumerate}

  \item Let $G$ be a lie group, that is, a smooth manifold that is also a
    group such that the product map $G\times G\to G$ and the inversion map
    $G\to G$ are all smooth. Let $e\in G$ be the identity. A vector field
    $V$ on $G$ is left invariant if it satisfies the following property for
    all $g\in G$: Let $\tau_g:G\to G$ be defined via the formula
    $\tau_g(h)=gh$. Then we require that $(D_h\tau_g)(V(h))=V(gh)$ for all
    $h\in G$.

    \begin{enumerate}
      \item Let $V_1$ and $V_2$ be left invariant vector fields on $G$ such
        that $V_1(e)=V_2(e)$. Prove that $V_1=V_2$.

        \begin{proof}
          Let $g\in G$. Then we have
          \begin{align*}
            V_1(g) &=V_1(ge) \\
              &=(D_e\tau_g)(V_1(e)) \\
              &=(D_e\tau_g)(V_2(e)) \\
              &=V_2(ge) \\
              &=V_2(g). \\
          \end{align*}
          Thus $V_1=V_2$.
        \end{proof}

      \item For $\vec{v}\in T_eG$, prove that there exists a left invariant
        vector field $V$ on $G$ such that $V(e)=\vec{v}$.

        \begin{proof}
          Given $\vec{v}\in T_eG$, consider the map $V:G\rightarrow TG$
          defined by $V(g)=(D_e\tau_g)(\vec{v})$. We show that $V$ is a
          left invariant vector field. First, note that
          $D_e\tau_e=\text{id}$ because $\tau_e=\text{id}$, hence
          $V(e)=(D_e\tau_e)(\vec{v})=\vec{v}$. Also, given $g,h\in G$,
          \begin{align*}
            (D_h\tau_g)(V(h)) &= (D_h\tau_g\cdot D_e\tau_h)(\vec{v}) \\
              &=D_e(\tau_g\circ\tau_h)(\vec{v}) \\
              &=(D_e\tau_{gh})(\vec{v}) \\
              &=V(gh), \\
          \end{align*}
          and so $V$ is left-invariant.
        \end{proof}
    \end{enumerate}
\end{enumerate}
\end{document}
