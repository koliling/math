\documentclass{article}
\usepackage[left=3cm,right=3cm,top=3cm,bottom=3cm]{geometry}
\usepackage{amsmath,amssymb,amsthm,tikz,mathtools}
\usepackage[inline]{enumitem}
\usetikzlibrary{patterns}
\usepackage{color}
\setlength{\parindent}{2mm}
\newcommand{\TODO}[1]{\textcolor{red}{TODO: #1}}
\newtheorem{prob}{Problem}

\begin{document}
\title{Geometry and Topology: Homework 10}
\author{Li Ling Ko\\ lko@nd.edu}
\date{\today}
\maketitle

\begin{enumerate}[label={\bf Q\arabic*:}]
  \item Let $M^n$ be a smooth manifold, let $\gamma:[a,b]\rightarrow M^n$
    be a smooth path, and let $\omega\in\Omega^1(M^n)$. Finally, let
    $h:[a,b]\rightarrow[a,b]$ be a smooth map such that $h(a)=a$ and
    $h(b)=b$. Define $\gamma_2=\gamma\circ h$. Prove that
    $\int_{\gamma}\omega=\int_{\gamma_2}\omega$.

    \begin{proof}
      Chasing definitions, we have
      \[\int_\gamma\omega =\int_a^b\gamma^*(\omega)dt,\]
      where $\gamma^*(\omega):[a,b]\rightarrow\mathbb{R}$ is a smooth
      function defined by
      \[\gamma^*(\omega)(t) =(\omega(\gamma(t)))\circ(D_t\gamma).\]

      Similarly, we have
      \[\int_{\gamma_2}\omega =\int_a^b\gamma_2^*(\omega)dt,\]
      where $\gamma_2^*(\omega):[a,b]\rightarrow\mathbb{R}$ is a smooth
      function defined by
      \begin{align*}
        \gamma_2^*(\omega)(t) &=(\omega((\gamma\circ h)(t)))
          \circ(D_t(\gamma\circ h)) &(\because\gamma_2=\gamma\circ h) \\
          &=(\omega(\gamma(h(t)))) \circ(D_t(\gamma\circ h)) \\
          &=(\omega(\gamma(h(t)))) \circ(D_{h(t)}\gamma\circ(D_th)).
            &(\text{by chain rule}) \\
      \end{align*}

      Now observe that
      \begin{align*}
        \gamma_2^*(\omega)(t)dt &=(\omega(\gamma(h(t))))
          \circ(D_{h(t)}\gamma\circ(D_th)) dt \\
          &=\gamma^*(\omega)(h(t))\; dh(t), \\
      \end{align*}
      and therefore
      \begin{align*}
        \int_{\gamma_2}\omega &=\int_a^b\gamma_2^*(\omega)dt \\
          &=\int^{t=b}_{t=a}\gamma^*(\omega)(h(t))\; dh(t) \\
          &=\int^{t=b}_{t=a}\gamma^*(\omega)(t)\; dt \\
          &=\int_{\gamma}\omega. \\
      \end{align*}
    \end{proof}

  \item Let $M^n$ be a smooth manifold and let $\omega\in\Omega^1(M^n)$ be
    such that $\omega=df$ for some smooth function
    $f:M^n\rightarrow\mathbb{R}$.

    \begin{enumerate}
      \item If $\gamma:[a,b]\rightarrow M^n$ is a smooth path, prove that
        $\int_\gamma\omega=f(\gamma(a))-f(\gamma(b))$.
        \begin{proof}
          Chasing definitions, we have
          \begin{align*}
            \int_\gamma\omega &=\int_a^b\gamma^*(\omega)dt \\
              &=\int_a^b(\omega(\gamma(t))) \circ(D_t\gamma)\; dt \\
              &=\int_a^b(df(\gamma(t))) \circ(D_t\gamma)\; dt \\
              &=\int_a^b df_{\gamma(t)}\; (D_t\gamma)\; dt &(\text{by
                definition of}\; dt) \\
              &=\int_a^b D_{\gamma(t)}\; f(D_t\gamma)\; dt &(\text{by
                definition}) \\
              &=\int_a^b D_{\gamma(t)}\; f\circ D_t\gamma\; dt &(\text{by
                definition}) \\
              &=\int_a^b D_{t}(f\circ\gamma)(\frac{\partial t}{\partial
                t})\; dt &(\text{by chain rule}) \\
              &=\int_a^b (f\circ\gamma)'(t)\; dt &(\text{by definition}) \\
              &=f(\gamma(b))-f(\gamma(a)). &(\text{by fundamental theorem of
                calculus}) \\
          \end{align*}
        \end{proof}

      \item If $\gamma:[a,b]\rightarrow M^n$ is a closed path (i.e. a path
        such that $\gamma(a)=\gamma(b)$), prove that $\int_\gamma\omega=0$.
        \begin{proof}
          This follows directly from part (a) above:
          $\int_\gamma\omega=f(\gamma(b))-f(\gamma(a))=0$ because
          $\gamma(a)=\gamma(b)$.
        \end{proof}
    \end{enumerate}
\end{enumerate}
\end{document}
