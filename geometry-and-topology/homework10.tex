\documentclass{article}
\usepackage[left=3cm,right=3cm,top=3cm,bottom=3cm]{geometry}
\usepackage{amsmath,amssymb,amsthm,tikz,mathtools}
\usepackage[inline]{enumitem}
\usetikzlibrary{patterns}
\usepackage{color}
\setlength{\parindent}{2mm}
\newcommand{\TODO}[1]{\textcolor{red}{TODO: #1}}
\newtheorem{prob}{Problem}

\begin{document}
\title{Geometry and Topology: Homework 10}
\author{Li Ling Ko\\ lko@nd.edu}
\date{\today}
\maketitle

\begin{enumerate}[label={\bf Q\arabic*:}]
  \item Let $M^n$ be a smooth manifold, let $\gamma:[a,b]\rightarrow M^n$
    be a smooth path, and let $\omega\in\Omega^1(M^n)$. Finally, let
    $h:[a,b]\rightarrow[a,b]$ be a smooth map such that $h(a)=a$ and
    $h(b)=b$. Define $\gamma_2=\gamma\circ h$. Prove that
    $\int_{\gamma}\omega=\int_{\gamma_2}\omega$.

    \begin{proof}
      Chasing definitions, we have
      \[\int_\gamma\omega =\int_a^b\gamma^*(\omega)dt,\]
      where $\gamma^*(\omega):[a,b]\rightarrow\mathbb{R}$ is a smooth
      function defined by
      \[\gamma^*(\omega)(t) =(\omega(\gamma(t)))\circ(D_t\gamma).\]

      Similarly, we have
      \[\int_{\gamma_2}\omega =\int_a^b\gamma_2^*(\omega)dt,\]
      where $\gamma_2^*(\omega):[a,b]\rightarrow\mathbb{R}$ is a smooth
      function defined by
      \begin{align*}
        \gamma_2^*(\omega)(t) &=(\omega((\gamma\circ h)(t)))
          \circ(D_t(\gamma\circ h)) &(\because\gamma_2=\gamma\circ h) \\
          &=(\omega(\gamma(h(t)))) \circ(D_t(\gamma\circ h)) \\
          &=(\omega(\gamma(h(t)))) \circ(D_{h(t)}\gamma\circ(D_th)).
            &(\text{by chain rule}) \\
      \end{align*}

      Now observe that
      \begin{align*}
        \gamma_2^*(\omega)(t)dt &=(\omega(\gamma(h(t))))
          \circ(D_{h(t)}\gamma\circ(D_th)) dt \\
          &=\gamma^*(\omega)(h(t))\; dh(t), \\
      \end{align*}
      and therefore
      \begin{align*}
        \int_{\gamma_2}\omega &=\int_a^b\gamma_2^*(\omega)dt \\
          &=\int^{t=b}_{t=a}\gamma^*(\omega)(h(t))\; dh(t) \\
          &=\int^{t=b}_{t=a}\gamma^*(\omega)(t)\; dt \\
          &=\int_{\gamma}\omega. \\
      \end{align*}
    \end{proof}

  \item Let $M^n$ be a smooth manifold and let $\omega\in\Omega^1(M^n)$ be
    such that $\omega=df$ for some smooth function
    $f:M^n\rightarrow\mathbb{R}$.

    \begin{enumerate}
      \item If $\gamma:[a,b]\rightarrow M^n$ is a smooth path, prove that
        $\int_\gamma\omega=f(\gamma(a))-f(\gamma(b))$.
        \begin{proof}
          Chasing definitions, we have
          \begin{align*}
            \int_\gamma\omega &=\int_a^b\gamma^*(\omega)dt \\
              &=\int_a^b(\omega(\gamma(t))) \circ(D_t\gamma)\; dt \\
              &=\int_a^b(df(\gamma(t))) \circ(D_t\gamma)\; dt \\
              &=\int_a^b df_{\gamma(t)}\; (D_t\gamma)\; dt &(\text{by
                definition of}\; dt) \\
              &=\int_a^b D_{\gamma(t)}\; f(D_t\gamma)\; dt &(\text{by
                definition}) \\
              &=\int_a^b D_{\gamma(t)}\; f\circ D_t\gamma\; dt &(\text{by
                definition}) \\
              &=\int_a^b D_{t}(f\circ\gamma)(\frac{\partial t}{\partial
                t})\; dt &(\text{by chain rule}) \\
              &=\int_a^b (f\circ\gamma)'(t)\; dt &(\text{by definition}) \\
              &=f(\gamma(b))-f(\gamma(a)). &(\text{by fundamental theorem of
                calculus}) \\
          \end{align*}
        \end{proof}

      \item If $\gamma:[a,b]\rightarrow M^n$ is a closed path (i.e. a path
        such that $\gamma(a)=\gamma(b)$), prove that $\int_\gamma\omega=0$.
        \begin{proof}
          This follows directly from part (a) above:
          $\int_\gamma\omega=f(\gamma(b))-f(\gamma(a))=0$ because
          $\gamma(a)=\gamma(b)$.
        \end{proof}
    \end{enumerate}

  \item Define a 1-form on $M^2=\mathbb{R}^2\setminus\{0\}$ via the formula
    \[\omega =\left(\frac{-y}{x^2+y^2}\right)dx
    +\left(\frac{x}{x^2+y^2}\right)dy.\]

    \begin{enumerate}
      \item Let $\gamma:[0,1]\rightarrow M^2$ be a circle of radius $r>0$
        around $(0,0)$. Calculate $\int_\gamma\omega$.
        \begin{proof}
          Note that $\gamma(t)=(r\cos(t),r\sin(t))$. Chasing definitions,
          we have
          \begin{align*}
            \int_\gamma\omega &=\int_a^b\gamma^*(\omega)dt \\
              &=\int_0^1(\omega(\gamma(t))) \circ(D_t\gamma)\; dt \\
              &=\int_0^1(\omega(r\cos(t),r\sin(t))) \circ(D_t\gamma)\; dt \\
              &=\int_0^1 \left(\left(\frac{-r\sin(t)}{r^2}\right)dx,
                \left(\frac{r\cos(t)}{r^2}\right)dy\right) \circ(D_t\gamma)\;
                dt \\
              &=\int_0^1 \left(\left(\frac{-\sin(t)}{r}\right)dx,
                \left(\frac{\cos(t)}{r}\right)dy\right) \circ(D_t\gamma)\;
                dt \\
              &=\int_0^1 \left(\left(\frac{-\sin(t)}{r}\right)dx,
                \left(\frac{\cos(t)}{r}\right)dy\right)
                \binom{-r\sin(t)\frac{\partial}{\partial x}}
                {r\cos(t)\frac{\partial}{\partial y}}\; dt \\
              &=\int_0^1 \sin^2(t)dx\frac{\partial}{\partial x} +
                \cos^2(t)dy\frac{\partial}{\partial y}\;\; dt \\
              &=\int_0^1 \sin^2(t)+\cos^2(t)\; dt \\
              &=\int_0^1 1 dt \\
              &=1. \\
          \end{align*}
        \end{proof}

      \item Prove that there does not exist some smooth function
        $f:M^2\rightarrow\mathbb{R}$ such that $\omega=df$.
        \begin{proof}
          If such a function exists, then from Question 2b above, we would
          have $\int_\gamma\omega=0$, which would contradict Question 3a
          above.
        \end{proof}

      \item Define $M^2_2=\{(x,y)|x>0\}$. Construct an explicit function
        $f:M^2_2\rightarrow\mathbb{R}$ such that $\omega=df$.
        \begin{proof}
          Consider $f((x,y))=\arctan{\frac{y}{x}}$. Then
          \begin{align*}
            df &=\frac{\partial f}{\partial x}dx + \frac{\partial
              f}{\partial y}dy \\
              &=\left(\frac{-y}{x^2+y^2}\right)dx
                +\left(\frac{x}{x^2+y^2}\right)dy \\
              &=\omega. \\
          \end{align*}
        \end{proof}
    \end{enumerate}

  \item Let $M^n$ be a smooth connected manifold and let
    $\omega\in\Omega^1(M^n)$. Assume that for all closed paths
    $\gamma:[a,b]\rightarrow M^n$, we have $\int_\gamma\omega=0$. The goal
    of this problem is to prove the converse of Problem 2, i.e. that there
    exists some smooth function $f:M^n\rightarrow\mathbb{R}$ such that
    $\omega=df$.

    \begin{enumerate}
      \item Let $\gamma_1:[0,1]\rightarrow M^n$ and
        $\gamma_2:[0,1]\rightarrow M^n$ be two smooth paths such that
        $\gamma_1(0)=\gamma_2(0)$ and $\gamma_1(1)=\gamma_2(1)$. Prove that
        $\int_{\gamma_1}\omega=\int_{\gamma_2}\omega$.

        \begin{proof}
          Let $h:[0,1]\rightarrow[0,1]$ be a smooth function such that
          $h\restriction[0,t_0]=0$ and $h\restriction[t_1,1]=1$ for some
          $0<t_0<t_1<1$, and such that $h$ is a non-decreasing function
          from 0 to 1. Such a function exists from Lemma 1.5. \\

          Consider $\gamma:[0,2]\rightarrow M^n$ defined as:
          \begin{equation*}
            \gamma(t) =
            \begin{cases}
              \gamma_1(h(t)), &0\leq t\leq1 \\
              \gamma_2(h(2-t)), &1<t\leq2 \\
            \end{cases}.
          \end{equation*}

          Then $\gamma$ is a smooth path, even at $t=1$, because from
          our choice of $h$, there will be an open interval containing
          $t=1$ where $\gamma$ is constant. Furthurmore, $\gamma$ is a
          closed path, so from Question 2b we have $\int_\gamma\omega=0$.
          Now
          \begin{align*}
            0 &=\int_\gamma\omega \\
              &=\int_{\gamma_1\circ h}\omega - \int_{\gamma_2\circ h}\omega
                \\
              &=\int_{\gamma_1}\omega - \int_{\gamma_2}\omega &(\text{from
                Question 1}), \\
          \end{align*}
          and so $\int_{\gamma_1}\omega=\int_{\gamma_2}\omega$ as required.
        \end{proof}

      \item Fix some basepoint $x_0\in M^n$. Define a function
        $f:M^n\rightarrow\mathbb{R}$ by letting $f(p)=\int_\gamma\omega$,
        wehre $\gamma:[0,1]\rightarrow M^n$ is a path such that
        $\gamma(0)=x_0$ and $\gamma(1)=p$ (this is well-defined by part a).
        Prove that $df=\omega$.

        \begin{proof}
        \end{proof}
    \end{enumerate}
\end{enumerate}
\end{document}
