\documentclass{article}
\usepackage[left=3cm,right=3cm,top=3cm,bottom=3cm]{geometry}
\usepackage{amsmath,amssymb,amsthm,pgfplots,tikz,mathtools}
\usepackage[inline]{enumitem}
\usetikzlibrary{patterns}
\usepackage{color}
\setlength{\parindent}{2mm}
\newcommand{\TODO}[1]{\textcolor{red}{TODO: #1}}

\begin{document}
\title{Geometry and Topology: Midterm Exam}
\author{Li Ling Ko\\ lko@nd.edu}
\date{\today}
\maketitle

\begin{enumerate}[label={\bf Q\arabic*:}]
  \item Let $T^2=S^1\times S^1$ be the 2-dimensional torus. Show that any
    map $f:S^2\rightarrow T^2$ is homotopic to a constant map.
    \begin{proof}
      Note that since $T^2$ is path-connected, locally path-connected, and
      semi-locally simply connected, it has a universal cover.
      $\mathbb{R}^2$ is a universal cover of $T^2$. Now since $S^2$ is
      simply connected, and $\mathbb{R}^2$ is contractible, $f$ will lift
      to a null-homotopic map $f^*:S^2\rightarrow\mathbb{R}^2$. Then since
      $f$ is the projection of $f^*$ down to $T^2$, $f$ will also be
      null-homotopic because $f^*$ is.
    \end{proof}

  \item
    \begin{enumerate}
      \item Let $x_1,\ldots,x_k\in S^2$ be distinct points and let $p\in
        S^2\setminus\{x_1,\ldots,x_k\}$ be a base point. Calculate a
        presentation for the group
        $\pi_1(S^2\setminus\{x_1,\ldots,x_k\},p)$.

        \begin{proof}
          Note that the base point $p$ does not matter since the space is
          connected. When $k=1$, space $S^2\setminus\{x_1\}$ is a sphere
          with a punctured hole, which is homeomorphic to $\mathbb{R}^2$,
          which has trivial fundamental group $1=F_0$. When we remove
          additional $(k-1)$ points from $\mathbb{R}^2$, we get a space
          homeomorphic to $\mathbb{R}^2\setminus\{x_2,\ldots,x_k\}$, which
          is a strong deformation retract to the wedge of $k$ circles at a
          fixed point $p$. Such a wedge has fundamental group $F_{k-1}$.
          Summarizing,
          \begin{align*}
            \pi_1(S^2\setminus\{x_1,\ldots,x_k\},p) &=F_k \\
              &=\langle a_1,\ldots,a_k\; |\; \rangle. \\
          \end{align*}
        \end{proof}

      \item Let $L_1,\ldots,L_k$ be distinct non-intersecting lines in
        $\mathbb{R}^3$ and let $q\in\mathbb{R}^3\setminus(L_1\cup\ldots\cup
        L_k)$ be a base point. Calculate a presentation for the group
        $\pi_1(\mathbb{R}^3\setminus(L_1\cup\ldots\cup L_k),q)$.

        \begin{proof}
          Note that the base point $q$ does not matter since the space is
          connected. The space $X=\mathbb{R}^3\setminus\{L_1\cup\ldots\cup
          L_k\}$ for non-intersecting $L_i$'s is a deformation retract to
          space $X'=\mathbb{R}^3\setminus\{L_1'\cup\ldots\cup L_k'\}$,
          where the $L_i'$ are parallel lines. $X'$ is a deformation
          retract to the cylinder containing $k$ hollow and parallel tubes.
          Making the height of this cylinder and the radius of the tubes
          arbitrarily small, the cylinder is a deformation retract of $S^2$
          minus $k$ points, which we have shown in the previous part of
          this question to have a fundamental group of $F_k$. Hence
          \begin{align*}
            \pi_1(\mathbb{R}^3\setminus\{L_1\cup\ldots\cup L_k\},q) &=F_k \\
              &=\langle a_1,\ldots,a_k\; |\; \rangle. \\
          \end{align*}
        \end{proof}
    \end{enumerate}

  \item Let $X$ be a space. Define the suspension of $X$, denoted $\sum X$,
    to be the quotient of $X\times[0,1]$ that identifies $X\times\{0\}$ to
    a single point and $X\times\{1\}$ to a single point (these are two
    different points!). Fix a base point $p\in\sum X$. Prove that if $X$ is
    path-connected, then $\pi_1(\sum X,p)=1$. Give an example to show that
    path-connectedness is necessary. 

    \begin{proof}
      We apply van-Kampen's theorem. Fix base point associated with
      $p=(x_0,0.5)$ in $\sum X$ for any $x_0\in X$. Let
      $U_0=X\times[0,0.6)/\sim$ be the left side of $\sum X$ containing
      $X\times\{0\}$, and $U_1=X\times(0.4,1]/\sim$ be the right side of
      $\sum X$ containing $X\times\{1\}$, as illustrated in Figure 3a. Then
      both $U_0$ and $U_1$ are path-connected by the path-connectedness of
      $X$: given points $(a_0,r_0)$ and $(a_1,r_1)$ in $U_i$, there is a
      path which joins $(a_0,r_0)$ to $(a_1,r_0)$ due to connectedness of
      $X$; concatenate this path with the straight line which joins
      $(a_1,r_0)$ to $(a_1,r_1)$ to get a path that connects the two
      points. Also, $U_0$ and $U_1$ form an open cover of $\sum X$, they
      both contain the base point, and their intersection $U_0\cap U_1$ is
      also path connected because of path-connectedness of $X$. So all
      conditions for van-Kampen's theorem are satisfied, and we can apply
      the theorem to compute the fundamental group of $\sum X=U_0\cup U_1$.
      \\

      Now each $U_i$ is contractible to the point $X\times\{i\}/\sim$ in
      $\sum X$, hence each $U_i$ has trivial fundamental group
      $1=\langle|\rangle$. Applying van-Kampen's theorem illustrated in
      Figure 3b,
      \begin{align*}
        \pi_1(\sum X,p) &= \langle|\rangle / \pi_1(U_0\cap U_1,p) \\
                        &= \langle|\rangle \\
                        &= 1. \\
      \end{align*}
      Hence $\sum X$ has trivial fundamental group. \\

      Consider the example where $X$ comprises of two distinct points. Then
      $\sum X$ will be homeomorphic to $S^1$, which has fundamental group
      $\mathbb{Z}\neq 1$.
    \end{proof}

  \item Let $X$ be a path-connected Hausdorff space and let
    $f:\widetilde{X}\rightarrow X$ be a covering space with $\widetilde{X}$
    compact. Prove that $f:\widetilde{X}\rightarrow X$  is a
    finite-sheeted cover.

    \begin{proof}
      Assume by contradiction that $f$ is not finite-sheeted. Define an
      open cover $\mathcal{O}$ of $\widetilde{X}$ by the set of all sheets
      lying above each point $x$ in $X$, i.e.
      \begin{equation*}
        \mathcal{O} := \bigcup_{x\in X}
          \{\widetilde{U}_{\alpha,x}:\alpha\in A\}.
      \end{equation*}
      From compactness of $\widetilde{X}$, $\mathcal{O}$ contains a finite
      sub-cover $\mathcal{U}\subset\mathcal{O}$. Write $\mathcal{U}$ as
      \begin{equation*}
        \mathcal{U} := \bigcup_{i=1}^n \widetilde{U}_{\alpha_i,x_i}.
      \end{equation*}
      Consider the sheets of $x_1$. Each sheet $\widetilde{U}_{\alpha,x_1}$
      of $x_1$ must be contained in some sheet in $\mathcal{U}$, which
      implies that the degree of $x_1$ is at most $n$. Since the degree of
      a cover is independent on the choice of $x\in X$, $f$ must have
      degree no more than $n$, a contradiction.
    \end{proof}

  \item Using covering space theory, construct a free basis for the
    commutator subgroup $[F_2,F_2]$ of the free group $F_2$ on two
    generators $a$ and $b$.

    \begin{proof}
      Let $X=S^1\wedge S^1$. Let $p$ denote the point of intersection of
      the two circles in $X$, and let $a$ denote the left circle and $b$
      denote the right circle. Consider the covering space
      $f:\widetilde{X}\rightarrow X$ given in Figure 5a, composed of an
      infinite grid where each point in the grid is sent to $p$, each
      horizontal interval from left to right is sent to $a$, and each
      vertical interval from bottom to top is sent to $b$. Then $f$ is
      clearly a regular cover with deck group $\mathbb{Z}^2$. Also, $X$ and
      $\widetilde{X}$ are path-connected. Hence, covering space theory says
      that $f$ will induce an exact sequence
      \begin{align*}
        1\rightarrow \pi_1(\widetilde{X},\widetilde{p})\xrightarrow{f_*}
          \pi_1(X,p)\xrightarrow{\varphi} \mathbb{Z}^2\rightarrow 1,
      \end{align*}
      where $\widetilde{p}$ is any point in the grid $\widetilde{X}$, say
      $\widetilde{p}=(0,0)$. \\

      To determine $\varphi$, first note that $\pi_1(X,p)=F_2=\langle
      a,b\;|\rangle$, so $\varphi$ is fully determined by the images of $a$
      and $b$ in $\pi_1(X,p)$. Also, since $a$ in $X$ lifts to a path from
      $\widetilde{p}$ to the horizontal interval of one unit, and $b$ in
      $X$ lifts to a path from $\widetilde{p}$ to the vertical interval of
      one unit, we have
      \begin{align*}
        \varphi(a)  &=(1,0), \\
        \varphi(b)  &=(0,1).
      \end{align*}
      Hence $\ker\varphi$ comprises of exactly all the words in
      $F_2=\langle a,b\;|\rangle$ where the $a$'s occur as frequently as
      the $a^{-1}$'s and the $b$'s occur as frequently as the $b^{-1}$'s.
      This subgroup is the commutator subgroup $[F_2,F_2]$. \\

      So from the properties of exact sequence, we have
      \begin{align*}
        \pi_1(\widetilde{X},\widetilde{p})=\ker\varphi = [F_2,F_2]. \\
      \end{align*}

      Thus, we can obtain the free basis of $[F_2,F_2]$ from all loops in
      $\widetilde{X}$ that start and end at $\widetilde{p}$, and that do
      not pass through $\widetilde{p}$ except at the end points. Formally,
      a word $a^{r_1}b^{s_1}\ldots a^{r_n}b^{s_n}\in F_2$, where
      $r_i,s_i\in\mathbb{Z}$ is in the free basis of $[F_2,F_2]$ if
      and only $r_1+\ldots+r_n=s_1+\ldots+s_n=0$, and we do not have
      $r_1+\ldots+r_m=s_1+\ldots+s_m=0$ for any $m<n$.
    \end{proof}
\end{enumerate}
\end{document}
