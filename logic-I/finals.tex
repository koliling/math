\documentclass{article}
\usepackage[left=3cm,right=3cm,top=3cm,bottom=3cm]{geometry}
\usepackage{amsmath,amssymb,amsthm,tikz,mathtools}
\usepackage{color}
\usepackage[inline]{enumitem}
\usetikzlibrary{patterns}
\setlength{\parindent}{0mm}
\newcommand{\TODO}[1]{\textcolor{red}{TODO: #1}}

\begin{document}
\title{Basic Logic I: Final Exam}
\author{Li Ling Ko\\ lko@nd.edu}
\date{\today}
\maketitle

\begin{enumerate}[label={\bf Q\arabic*:}]
  \item \it Let $\mathcal{L}$ be a language. Assume $\mathcal{L}$ has a
    binary predicate symbol $P(x,y)$. Let $T$ be a universal
    $\mathcal{L}$-theory such that $T\models\forall x\exists y\; P(x,y)$.
    Prove that there exists $\mathcal{L}$-terms $t_1(x),\ldots,t_n(x)$ such
    that \[T\models\forall x\; \bigvee_{m=1}^n P(x,t_m(x)).\]

    \begin{proof}
      Assume by contradiction that for any finite set of terms
      $t_1(x),\ldots,t_m(x)$ in variable $x$,
      \begin{equation}
        \tag{$*$}
        T\models\exists x\; \bigwedge_{m=1}^n \neg P(x,t_m(x).
        \label{eqn:neg}
      \end{equation}
      Then there is a model $\mathcal{M}$ of $T$ satisfying all the
      existential sentences given at equation~\eqref{eqn:neg}. Note that
      because $T\models\forall x\exists y\; P(x,y)$, every element $a\in M$
      has a witness $b\in M$ such that $P(a,b)$. \\

      There are two possible cases for a model $\mathcal{M}$ satisfying all
      the equations in \eqref{eqn:neg}. In the first case, there is an
      element $a\in M$ whose witness $a'\in M$ of $P(a,a')$ is not any of the
      terms $t(a)$ of $a$. In this case, consider the substructure $\langle
      a\rangle<\mathcal{M}$, which is the smallest substructure of
      $\mathcal{M}$ containing $a$, defined rigorously as
      \begin{align*}
        \langle a\rangle &=\{t^\mathcal{M}(a):
          t^\mathcal{M}(x)\; \text{is an}\; \mathcal{L}\text{-term}\}. \\
      \end{align*}

      Note that $\langle a\rangle$ defined above will be closed under
      functions since the composition of terms in variable $x$ remains a
      term in variable $x$. Thus $\langle a\rangle$ is a substructure of
      $\mathcal{M}$. Then since $T$ is a universal theory, it is closed
      under substructures (Theorem 11.3), thus $\langle a\rangle$ is also a
      model of $T$. However, $a\in\langle a\rangle$ will have no witness to
      $P(a,y)$ because its witness is not $t(a)$ for any $\mathcal{L}$-term
      $t(x)$. Thus $\langle a\rangle\models\forall y\; \neg P(a,y)$,
      contradicting $T\models\forall x\exists y\; P(x,y)$. \\

      In the second case, every element $a\in M$ has a witness $a'\in M$
      for $P(a,a')$ such that $a'=t(a)$ for some $\mathcal{L}$-term $t(x)$.
      We construct an elementary extension $\mathcal{N}\succ\mathcal{M}$
      such that $\mathcal{N}$ falls under case 1 above. Add to language
      $\mathcal{L}(\mathcal{M})$ a new constant symbol $c$, i.e.
      $\mathcal{L}':=\mathcal{L}(\mathcal{M})\cup\{c\}$. In this extended
      language, consider the theory
      \[T':=T_\mathcal{M}\cup \{\neg P(c,t(c)):t^\mathcal{M}(x)\;
        \text{is an}\; \mathcal{L}\text{-term}\}.\]

      Then $T'$ is finitely satisfiable by $\mathcal{M}$: Clearly
      $\mathcal{M}$ satisfies $T_\mathcal{M}$. Given any finite set of
      $\mathcal{L}$-terms $t_1(x),\ldots,t_m(x)$ in variable $x$, because
      $\mathcal{M}$ satisfies equation~\eqref{eqn:neg}, there must be an
      element $a\in M$ that satisfies $\neg P(a,t_i(a))$ for all
      $i\in\{1,\ldots,m\}$. Assigning $c$ to this element $a$, we will have
      $\mathcal{M}\models T'$. Thus by Compactness theorem, $T'$
      has a model $\mathcal{N}$, which will also be a model of $T$ since
      $\mathcal{N}\models T_\mathcal{M}\supset T$. However in
      $\mathcal{N}$, the constant element $c$ does not have a witness $c'$
      of $P(c,c')$ which is a term $t(c)$ of $c$. Thus $\mathcal{N}$ is the
      same as case 1 above, from which we can define a sub-structure that
      gives a contradiction.
    \end{proof}
\end{enumerate}
\end{document}
