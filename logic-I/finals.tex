\documentclass{article}
\usepackage[left=3cm,right=3cm,top=3cm,bottom=3cm]{geometry}
\usepackage{amsmath,amssymb,amsthm,tikz,mathtools}
\usepackage{color}
\usepackage[inline]{enumitem}
\usetikzlibrary{patterns}
\setlength{\parindent}{0mm}
\newcommand{\TODO}[1]{\textcolor{red}{TODO: #1}}

\begin{document}
\title{Basic Logic I: Final Exam}
\author{Li Ling Ko\\ lko@nd.edu}
\date{\today}
\maketitle

Consider the language $\mathcal{L}=\{P\}$ and the inductive theory
$T=\{\forall x\exists y P(x,y)\}$. Consider the model $\mathcal{M}$ of $T$
with only two elements defined by
\begin{align*}
  M &=\{a,b\}, \\
  P^\mathcal{M} &=\{(a,b),(b,a)\}. \\
\end{align*}
Then $\mathcal{M}\models T$.  Then since this language does not have any
function symbols, the only terms involving a given variable $x$ is $x$
itself. However, \[T\not\vdash \forall x\; P(x,x)\vee\ldots\vee P(x,x).\]

%\begin{enumerate}[label={\bf Q\arabic*:}]
%  \item \it 
%\end{enumerate}
\end{document}
