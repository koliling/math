\documentclass{article}
\usepackage[left=3cm,right=3cm,top=3cm,bottom=3cm]{geometry}
\usepackage{amsmath,amssymb,amsthm}
\usepackage{color}
%\setlength{\parindent}{0mm}

\newcommand{\TODO}[1]{\textcolor{red}{TODO: #1}}

\begin{document}
\title{Basic Logic I: Homework Set 1}
\author{Li Ling Ko\\ lko@nd.edu}
\date{\today}
\maketitle

\begin{enumerate}
  \item Show that there are no sets $X$ and $Y$ such that $X\in Y$ and
    $Y\in X$. \label{qn:foundation}

    \begin{proof}
      We use the axiom of foundation and the axiom of pairing. Assume by
      contradiction that such sets $X$ and $Y$ exist. By Pairing, there
      exists a set $Z$ containing exactly $X$ and $Y$. This set would not
      have a smallest element since each of its elements contains the
      other. This contradicts the axiom of foundation.
    \end{proof}

  \item If we define the ordered pair $(x,y)$ to be the set $\{x,\{x,y\}\}$,
    show that $(x,y)=(u,v)$ if and only if $x=u$ and $y=v$.

    \begin{proof}
      We use the axiom of extensionality repeatedly. We first prove the
      forward implication. Assume that $\{x,\{x,y\}\}=\{u,\{u,v\}\}$. Then
      by extensionality, either $x=u$ and $\{x,y\}=\{u,v\}$, or $x=\{u,v\}$
      and $u=\{x,y\}$. In the former, we get $x=u$, then from applying
      extensionality on $\{x,y\}=\{u,v\}$, we get $y=v$ as required. The
      latter case is not possible from the axiom of foundation, as we have
      proven in question~\ref{qn:foundation} earlier. \\

      The converse is straightforward with two applications of the
      axiom of extensionality.
    \end{proof}

  \item For binary relations $B_1,B_2\subseteq S^2$ we define their
    composition $B_1\circ B_2$ as:
    \begin{equation*}
      B_1\circ B_2=\{(x,y)\in S^2: \exists z\, (x,z)\in B_1\;\text{and}\;
      (z,y)\in B_2\}.
    \end{equation*}
    Prove or disprove: The composition of two order relations on a set $S$
    is an order relation.

    \begin{proof}
      No, the composition is an order relation because it is symmetric.
      Consider the counter example $B_1=\{(x,y)\}$ and $B_2=\{(y,x)\}$.
      Then $B_1\circ B_2=\{(x,x)\}$, which is not an asymmetric relation.
    \end{proof}

  \item Prove that if $S$ is a set of ordinals then $\cup S$ is an ordinal.
    \label{qn:cup-ordinals}
    \begin{proof}
      This is equivalent to showing that $\cup S$ is transitive and is
      well-ordered by $\in$. For transitivity, given $\alpha\in\cup S$ and
      $\beta\in\alpha$, we need to show that $\beta\in\cup S$. Since
      $\alpha\in\cup S$, $\alpha$ must be contained in some ordinal
      $\gamma\in S$. By transitivity of $\gamma$, $\beta$ must also be
      contained in $\gamma$, and hence in $\cup S$, as we are required to
      show. \\

      Now we prove well-orderedness. By axiom of foundation, since $\cup S$
      is transitive, it suffices that show that it is linearly ordered by
      $\in$. This is equivalent to showing that with respect to $\in$,
      $\cup S$ is transitive, asymmetric, and that for all
      $\alpha\neq\beta\in\cup S$, either $\alpha\in\beta$ or
      $\beta\in\alpha$. We have already shown transitivity. Asymmetry is
      inherited from the asymmetries of each ordinal in $S$. Finally, given
      $\alpha\neq\beta\in\cup S$, by Theorem 1.14.1 in the notes, $\alpha$
      and $\beta$ are also ordinals, then by Theorem 1.14.3, either
      $\alpha\in\beta$ or $\beta\in\alpha$ as required.
    \end{proof}

  \item Let $X$ be a set of ordinals and $\alpha=\cup X$, i.e. $\alpha=\cup
    X$. Show that $x\leq\alpha$ for all $x\in X$ and $\alpha$ is the least
    ordinal satisfying this property.

    \begin{proof}
      First note that from question~\ref{qn:cup-ordinals}, we have shown
      that $\alpha=\cup X$ is an ordinal. Let $x\in X$. There are two
      possible cases. In the first case, there is an ordinal $y\in X$ such
      that $x<y$. Then by Theorem 1.14.3, $y$ contains $x$, so $\alpha=cup
      X$ must also contain $x$. Applying Theorem 1.14.3 again, this means
      $x<\alpha$. In the second case, all ordinals in $X$ are either
      smaller than or equal $x$. Then by Theorem 1.14.3, all ordinals in
      $X$ that are different from $x$ must be contained in $x$. Then from
      transitivity of $x$, $\alpha=\cup X$ must be $x$ itself. \\

      Now we show that $\alpha$ is the least ordinal satisfying this
      property. Let $\beta$ be an ordinal smaller than $\alpha=\cup X$. By
      Theorem 1.14.3, $\beta$ must be contained in $\alpha$, which means it
      is contained in some ordinal $x$ in $X$. By Theorem 1.14.3 again,
      $\beta$ cannot be larger or equal to $x$.
    \end{proof}

  \item Prove that for an ordinal $\alpha$, the following conditions are
    equivalent:
    \begin{enumerate}
      \item $\alpha$ is the least infinite ordinal. \label{cond:inf}
      \item $\alpha$ is the least non-empty limit ordinal.
        \label{cond:limit}
      \item $\alpha$ is the least non-empty ordinal satisfying
        $x\in\alpha\implies S(x)\in\alpha$. \label{cond:closed-succ}
    \end{enumerate}

    \begin{proof}
      (\ref{cond:inf})~$\rightarrow$~(\ref{cond:limit}): We first show that all
      non-empty limit ordinals must be infinite. This is true because
      finite ordinals are either empty or a successors. Hence if $\beta$
      is a non-empty limit ordinal smaller than the least infinite ordinal
      $\alpha$, it must also be an infinite ordinal smaller than $\alpha$,
      a contradiction. \\

      (\ref{cond:limit})~$\rightarrow$~(\ref{cond:closed-succ}): We first
      show that all non-empty ordinals satisfying $x\in\alpha\implies
      S(x)\in\alpha$ must be non-empty limit ordinals.
    \end{proof}

  \item Consider the set of positive natural numbers
    $\mathbb{N}^+=\{1,2,\ldots\}$. Define the relation $x\in y$ on
    $\mathbb{N}^+$ as
    \begin{equation*}
      x\in y\; \Leftrightarrow\; x<y\; \text{and}\; x|y.
    \end{equation*}
    Which of the following axioms of ZF hold in $(\mathbb{N}^+,\in)$:
    Extensionality, Foundation, Pairing, Union, Separation, Power set,
    Empty set, Replacement?
\end{enumerate}
\end{document}
