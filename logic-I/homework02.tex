\documentclass{article}
\usepackage[left=3cm,right=3cm,top=3cm,bottom=3cm]{geometry}
\usepackage{amsmath,amssymb,amsthm}
\usepackage{color}
%\setlength{\parindent}{0mm}

\newcommand{\TODO}[1]{\textcolor{red}{TODO: #1}}

\begin{document}
\title{Basic Logic I: Homework Set 2}
\author{Li Ling Ko\\ lko@nd.edu}
\date{\today}
\maketitle

\begin{enumerate}
  \item Prove Lemma 1.35: If $C$ is a non-empty set of cardinals then
    $\cup C$ is a cardinal.

    \begin{proof}
      Assume by contradiction that $\cup C$ is not a cardinal. 
      From Lemma 1.16.2, $\alpha=\cup C$ is an ordinal. By assumption
      hypothesis $|\alpha|=\lambda$ for some cardinal $\lambda\in\alpha$.
      Then $\lambda$ must be contained in some cardinal $\kappa\in C$.
      Since $\lambda$ and $\kappa$ are cardinals and $\kappa$ contains
      $\lambda$, $\kappa$ must be greater than $\lambda$ as cardinals. Now
      $\cup C$ is a superset of $\kappa$ since $\kappa$ is contained in
      $C$, implying that the cardinality of $\cup C$ is at least $\kappa$,
      a contradiction.
    \end{proof}

  \item Let $C=\{\kappa_i\,|\, i\in\omega\}$ be a set of cardinals such
    that $\kappa_i<\kappa_j$ for $i<j$. Let $\kappa=\cup C$. By the
    previous problem $\kappa$ is a cardinal. Show that $\kappa>\kappa_i$
    for all $i\in\omega$.

    \begin{proof}
      Given any $\kappa_i$, $\kappa$ is a superset of $\kappa_{i+1}$ and
      hence is at least as large as $\kappa_{i+1}$, which is larger than
      $\kappa_i$, as we are required to show.
    \end{proof}

  \item Show that the cardinality of the set
    \begin{equation*}
      \mathcal{S}=\{f\in\omega^\omega:\, \forall p\in\omega\; \forall
      n\in\omega\; (p<n\rightarrow f(p)<f(n))\}
    \end{equation*}
    equals $|2^\omega|$.

    \begin{proof}
      $\mathcal{S}$ is the set of monotonically increasing countable
      sequences of integers. Since $\mathcal{S}$ is a subset of
      $\omega^\omega$ which has cardinality $|2^\omega|$ from Corollary
      1.31, the cardinality of $\mathcal{S}$ must be smaller or equal to
      $|2^\omega|$. Hence it remains to show that $|\mathcal{S}|>\omega$.
      To prove that, it suffices to show that $\omega$ cannot enumerate
      $\mathcal{S}$. We prove this by using Cantor's diagonal argument.
      Assume that $\{f_i\}_{i\in\omega}$ enumerates $\mathcal{S}$. We
      inductively construct a new monotonically increasing function $f$
      that does not belong to any $f_i$'s, as follows: Set $f(0)=f_0(0)+1$.
      For each $n\in\omega$, set $f(n+1)=\max(f(n),f_n(n))+1$. By
      construction, $f$ cannot equal to any of the $f_i$'s in $\mathcal{S}$
      because $f(i)>f_i(i)$. Hence no countable enumeration of
      $\mathcal{S}$ exists.
    \end{proof}

  \item Show that the set
    \begin{equation*}
      \mathcal{S}=\{f\in\omega^\omega:\, \exists p\in\omega\; \forall
      n\in\omega\; (p<n\rightarrow f(p)=f(n))\}
    \end{equation*}
    is countable.

    \begin{proof}
      $\mathcal{S}$ is bijective to the set of all finite sequences of
      integers. Hence $\mathcal{S}$ has cardinality $\omega^{<\omega}$,
      which equals $\omega$ by Corollary 1.32.
    \end{proof}
\end{enumerate}
\end{document}
