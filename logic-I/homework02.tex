\documentclass{article}
\usepackage[left=3cm,right=3cm,top=3cm,bottom=3cm]{geometry}
\usepackage{amsmath,amssymb,amsthm}
\usepackage{color}
%\setlength{\parindent}{0mm}

\newcommand{\TODO}[1]{\textcolor{red}{TODO: #1}}

\begin{document}
\title{Basic Logic I: Homework Set 2}
\author{Li Ling Ko\\ lko@nd.edu}
\date{\today}
\maketitle

\begin{enumerate}
  \item Prove Lemma 1.35: If $C$ is a non-empty set of cardinals then
    $\cup C$ is a cardinal.

    \begin{proof}
      Assume by contradiction that $\cup C$ is not a cardinal. 
      From Lemma 1.16.2, $\alpha=\cup C$ is an ordinal. By assumption
      hypothesis $|\alpha|=\lambda$ for some cardinal $\lambda\in\alpha$.
      Then $\lambda$ must be contained in some cardinal $\kappa\in C$.
      Since $\lambda$ and $\kappa$ are cardinals and $\kappa$ contains
      $\lambda$, $\kappa$ must be greater than $\lambda$ as cardinals. Now
      $\cup C$ is a superset of $\kappa$ since $\kappa$ is contained in
      $C$, implying that the cardinality of $\cup C$ is at least $\kappa$,
      a contradiction.
    \end{proof}
\end{enumerate}
\end{document}
