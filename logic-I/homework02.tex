\documentclass{article}
\usepackage[left=3cm,right=3cm,top=3cm,bottom=3cm]{geometry}
\usepackage{amsmath,amssymb,amsthm}
\usepackage{color}
%\setlength{\parindent}{0mm}

\newcommand{\TODO}[1]{\textcolor{red}{TODO: #1}}

\begin{document}
\title{Basic Logic I: Homework Set 2}
\author{Li Ling Ko\\ lko@nd.edu}
\date{\today}
\maketitle

\begin{enumerate}
  \item Prove Lemma 1.35: If $C$ is a non-empty set of cardinals then
    $\cup C$ is a cardinal.

    \begin{proof}
      Assume by contradiction that $\cup C$ is not a cardinal. 
      From Lemma 1.16.2, $\alpha=\cup C$ is an ordinal. By assumption
      hypothesis $|\alpha|=\lambda$ for some cardinal $\lambda\in\alpha$.
      Then $\lambda$ must be contained in some cardinal $\kappa\in C$.
      Since $\lambda$ and $\kappa$ are cardinals and $\kappa$ contains
      $\lambda$, $\kappa$ must be greater than $\lambda$ as cardinals. Now
      $\cup C$ is a superset of $\kappa$ since $\kappa$ is contained in
      $C$, implying that the cardinality of $\cup C$ is at least $\kappa$,
      a contradiction.
    \end{proof}

  \item Let $C=\{\kappa_i\,|\, i\in\omega\}$ be a set of cardinals such
    that $\kappa_i<\kappa_j$ for $i<j$. Let $\kappa=\cup C$. By the
    previous problem $\kappa$ is a cardinal. Show that $\kappa>\kappa_i$
    for all $i\in\omega$.

    \begin{proof}
      Given any $\kappa_i$, $\kappa$ is a superset of $\kappa_{i+1}$ and
      hence is at least as large as $\kappa_{i+1}$, which is larger than
      $\kappa_i$, as we are required to show.
    \end{proof}

  \item Show that the cardinality of the set
    \begin{equation*}
      \mathcal{S}=\{f\in\omega^\omega:\, \forall p\in\omega\; \forall
      n\in\omega\; (p<n\rightarrow f(p)<f(n))\}
    \end{equation*}
    equals $|2^\omega|$.

    \begin{proof}
      First we note that $\mathcal{S}$ is the set of monotonically
      increasing infinitely countable sequences of integers. We encode each
      element of $\mathcal{S}$ with an element of $2^\omega$ by defining a
      bijection $\theta:\mathcal{S}\rightarrow2^\omega$. Given
      $f\in\mathcal{S}$, we construct $\theta(f)$ inductively on
      $n\in\omega$ such that $\theta(f)=\cup_{n\in\omega}g_n$, where
      $g_n\in2^{<\omega}$ and $g_{n+1}$ extends $g_n$ for each
      $n\in\omega$.

      \begin{enumerate}
        \item Let $g_0$ be the sequence of $0$'s of length $f(0)$.
        \item Given $g_n$, let $g_{n+1}$ be $g_n$ concatenated with a
          sequence of bits of length $f(n+1)-f(n)$, where we choose the bit
          to be 0 if $n+1$ is even and 1 if $n+1$ is odd.
      \end{enumerate}

      It is routine to verify that $\theta$ is a bijection, hence
      $\mathcal{S}$ has the same cardinality as $2^\omega$. 
    \end{proof}

  \item Show that the set
    \begin{equation*}
      \mathcal{S}=\{f\in\omega^\omega:\, \exists p\in\omega\; \forall
      n\in\omega\; (p<n\rightarrow f(p)=f(n))\}
    \end{equation*}
    is countable.

    \begin{proof}
      $\mathcal{S}$ is bijective to the set of all finite sequences of
      integers. Hence $\mathcal{S}$ has cardinality $\omega^{<\omega}$,
      which equals $\omega$ by Corollary 1.32.
    \end{proof}

  \item For given structures $\mathcal{M}$ and $\mathcal{N}$ decide if
    $\mathcal{M}$ can be embedded into $\mathcal{N}$. If it can, give an
    embedding; if not, explain why.

    \begin{enumerate}
      \item $\mathcal{M}=\langle\mathbb{N},\leq,+,0\rangle$ and
        $\mathcal{N}=\langle\mathbb{N},\leq,\cdot,0\rangle$
        \begin{proof}
          No such embedding exists. Assume that an embedding
          $\alpha:\mathbb{N}\hookrightarrow\mathbb{N}$ exists. Then
          $\alpha(0)=0$ since embeddings preserve constants. Consider the
          image of $1$ under $\alpha$. Embeddings should preserve
          functions, so
          $\alpha(1)=\alpha(0+1)=\alpha(0)\cdot\alpha(1)=0\cdot\alpha(1)=0$.
          So $\alpha(0)=\alpha(1)=0$, contradicting the injectiveness of
          $\alpha$.
        \end{proof}

      \item $\mathcal{M}=\langle\mathbb{N},\leq,\cdot,0\rangle$ and
        $\mathcal{N}=\langle\mathbb{N},\leq,+,0\rangle$
        \begin{proof}
          No such embedding exists. Assume that an embedding
          $\alpha:\mathbb{N}\hookrightarrow\mathbb{N}$ exists. Then
          $\alpha(0)=0$ since embeddings preserve constants. Embeddings
          should also preserve functions, so
          $\alpha(0)=\alpha(0\cdot1)=\alpha(0)+\alpha(1)=0+\alpha(1)=\alpha(1)$,
          which contradicts the injectiveness of $\alpha$.
        \end{proof}

      \item $\mathcal{M}=\langle\mathbb{N}\setminus\{0\},\cdot,1\rangle$ and
        $\mathcal{N}=\langle\mathbb{N},+,0\rangle$
        \begin{proof}
          No such embedding exists. Assume that an embedding
          $\alpha:\mathbb{N}\hookrightarrow\mathbb{N}$ exists. Since
          embeddings preserve function maps, we can prove by induction on
          $k$ that for any $n\in\mathbb{N}\setminus\{0\}$,
          $\alpha(n^k)=k\cdot \alpha(n)$.  Let $m=\alpha(2)$ and
          $n=\alpha(3)$. Note that $m\neq0$ and $n\neq0$ since
          $\alpha(1)=0$ and $\alpha$ is injective. Then $\alpha(2^n)=n\cdot
          m=m\cdot n=\alpha(3^m)$, yet $2^n\neq3^m$ since $m,n\neq0$,
          contradicting the injectiveness of $\alpha$.
        \end{proof}
    \end{enumerate}

  \item Give an example of two structures $\mathcal{M}$ and $\mathcal{N}$
    such that $\mathcal{M}$ is embeddable into $\mathcal{N}$, $\mathcal{N}$
    is embeddable into $\mathcal{M}$, but $\mathcal{M}$ is not isomorphic
    to $\mathcal{N}$.

    \begin{proof}
      Consider the structures $\mathcal{M}=\langle[0,1),<\rangle$ and
      $\mathcal{N}=\langle[0,1],<\rangle$. The identity map $id$ will embed
      $\mathcal{M}$ into $\mathcal{N}$, and scaling by 0.5 will embed
      $\mathcal{N}$ into $\mathcal{M}$. However, these two structures
      cannot be isomorphic because structure $\mathcal{N}$ has a maximal
      element 1 while structure $\mathcal{M}$ does not, so any bijection
      $\theta$ from $[0,1]$ to $[0,1)$ will not preserve order. More
      specifically, let $r\in[0,1)$ be any element larger than $\theta(1)$.
      Then $\theta^{-1}(r)<1$ in $[0,1]$, yet after mapping with $\theta$,
      order is no longer preserved since $r>\theta(1)$.
    \end{proof}

  \item Find an infinite structure $\mathcal{M}$ in a finite language whose
    only automorphism is the identity map.

    \begin{proof}
      $\mathcal{M}=\langle\mathbb{N},<\rangle$ is one such structure. Let
      $\alpha:\mathbb{N}\rightarrow\mathbb{N}$ be an automorphism. We show by
      induction on $n\in\mathbb{N}$ that $\alpha(n)=n$. For the base case,
      if $\alpha(0)=k>0$, then since automorphisms preserve relations, and
      $k-1<k$, we have $\alpha^{-1}(k-1)<\alpha^{-1}(k)=0$, which is not
      possible because nothing in $\mathbb{N}$ is smaller than $0$. Hence
      $\alpha(0)=0$. For the inductive step, assume $\alpha(k)=k$ for
      $k\leq n\in\mathbb{N}$. If $\alpha(n+1)=n+1+k$ for some $k>0$, then
      from $n+1<n+1+k$, we get, from the preservation of relations, that
      $\alpha^{-1}(n+1)<\alpha^{-1}(n+1+k)=n+1$. Yet $\alpha^{-1}(n+1)$
      must be greater than $n+1$ from inductive hypothesis, which is a
      contradiction. Hence $\alpha(n+1)=n+1$ as required, and $\alpha$ can
      only be the identity map.
    \end{proof}

  \item Find a structure $\mathcal{M}$ and a bijective homomorphism
    $\alpha:\mathcal{M}\rightarrow\mathcal{M}$ such that $\alpha$ is not an
    isomorphism.

    \begin{proof}
      Consider the structure
      $\mathcal{M}=\langle\mathbb{N}\cup\{i\},<\rangle$. Also,
      consider the map
      $\theta:\mathbb{N}\cup\{i\}\rightarrow\mathbb{N}\cup\{i\}$ which
      sends all natural numbers $n$ to $n+1$ and sends the imaginary number
      $i$ to 0. $\theta$ is a bijection, and also a homomorphism because
      it preserves the linear orders of the natural numbers. However, it is
      not an automorphism because its inverse does not preserve order. In
      particular, $0<1$ in this structure, but the inverse images
      $\theta^{-1}(0)=i$ and $\theta^{-1}(1)=0$ are not comparable under
      the $<$ relation.
    \end{proof}
\end{enumerate}
\end{document}
