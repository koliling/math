\documentclass{article}
\usepackage[left=3cm,right=3cm,top=3cm,bottom=3cm]{geometry}
\usepackage{amsmath,amssymb,amsthm}
\usepackage{color}
%\setlength{\parindent}{0mm}

\newcommand{\TODO}[1]{\textcolor{red}{TODO: #1}}

\begin{document}
\title{Basic Logic I: Homework Set 1}
\author{Li Ling Ko\\ lko@nd.edu}
\date{\today}
\maketitle

\begin{enumerate}
  \item Show that there are no sets $X$ and $Y$ such that $X\in Y$ and
    $Y\in X$. \label{qn:foundation}

    \begin{proof}
      We use the axiom of foundation and the axiom of pairing. Assume by
      contradiction that such sets $X$ and $Y$ exist. By Pairing, there
      exists a set $Z$ containing exactly $X$ and $Y$. This set would not
      have a smallest element since each of its elements contains the
      other. This contradicts the axiom of foundation.
    \end{proof}

  \item If we define the ordered pair $(x,y)$ to be the set $\{x,\{x,y\}\}$,
    show that $(x,y)=(u,v)$ if and only if $x=u$ and $y=v$.

    \begin{proof}
      We use the axiom of extensionality repeatedly. We first prove the
      forward implication. Assume that $\{x,\{x,y\}\}=\{u,\{u,v\}\}$. Then
      by extensionality, either $x=u$ and $\{x,y\}=\{u,v\}$, or $x=\{u,v\}$
      and $u=\{x,y\}$. In the former, we get $x=u$, then from applying
      extensionality on $\{x,y\}=\{u,v\}$, we get $y=v$ as required. The
      latter case is not possible from the axiom of foundation, as we have
      proven in question~\ref{qn:foundation} earlier. \\

      The converse is straightforward with two applications of the
      axiom of extensionality.
    \end{proof}

  \item For binary relations $B_1,B_2\subseteq S^2$ we define their
    composition $B_1\circ B_2$ as:
    \begin{equation*}
      B_1\circ B_2=\{(x,y)\in S^2: \exists z\, (x,z)\in B_1\;\text{and}\;
      (z,y)\in B_2\}.
    \end{equation*}
    Prove or disprove: The composition of two order relations on a set $S$
    is an order relation.

    \begin{proof}
      No, the composition is not an order relation because it is symmetric.
      Consider the counter example $B_1=\{(x,y)\}$ and $B_2=\{(y,x)\}$.
      Then $B_1\circ B_2=\{(x,x)\}$, which is not an asymmetric relation.
    \end{proof}

  \item Prove that if $S$ is a set of ordinals then $\cup S$ is an ordinal.
    \label{qn:cup-ordinals}
    \begin{proof}
      This is equivalent to showing that $\cup S$ is transitive and is
      well-ordered by $\in$. For transitivity, given $\alpha\in\cup S$ and
      $\beta\in\alpha$, we need to show that $\beta\in\cup S$. Since
      $\alpha\in\cup S$, $\alpha$ must be contained in some ordinal
      $\gamma\in S$. By transitivity of $\gamma$, $\beta$ must also be
      contained in $\gamma$, and hence in $\cup S$, as we are required to
      show. \\

      Now we prove well-orderedness. By axiom of foundation, since $\cup S$
      is transitive, it suffices that show that it is linearly ordered by
      $\in$. This is equivalent to showing that with respect to $\in$,
      $\cup S$ is transitive, asymmetric, and that for all
      $\alpha\neq\beta\in\cup S$, either $\alpha\in\beta$ or
      $\beta\in\alpha$. We have already shown transitivity. Asymmetry is
      inherited from the asymmetries of each ordinal in $S$. Finally, given
      $\alpha\neq\beta\in\cup S$, by Theorem 1.14.1 in the notes, $\alpha$
      and $\beta$ are also ordinals, then by Theorem 1.14.3, either
      $\alpha\in\beta$ or $\beta\in\alpha$ as required.
    \end{proof}

  \item Let $X$ be a set of ordinals and $\alpha=\sup X$, i.e. $\alpha=\cup
    X$. Show that $x\leq\alpha$ for all $x\in X$ and $\alpha$ is the least
    ordinal satisfying this property.

    \begin{proof}
      First note that from question~\ref{qn:cup-ordinals}, we have shown
      that $\alpha=\cup X$ is an ordinal. Let $x\in X$. There are two
      possible cases. In the first case, there is an ordinal $y\in X$ such
      that $x<y$. Then by Theorem 1.14.3, $y$ contains $x$, so $\alpha=\cup
      X$ must also contain $x$. Applying Theorem 1.14.3 again, this means
      $x<\alpha$. In the second case, all ordinals in $X$ are either
      smaller than or equal $x$. Then by Theorem 1.14.3, all ordinals in
      $X$ that are different from $x$ must be contained in $x$. Then from
      transitivity of $x$, $\alpha=\cup X$ must be $x$ itself. \\

      Now we show that $\alpha$ is the least ordinal satisfying this
      property. Let $\beta$ be an ordinal smaller than $\alpha=\cup X$. By
      Theorem 1.14.3, $\beta$ must be contained in $\alpha$, which means it
      is contained in some ordinal $x$ in $X$. By Theorem 1.14.3 again,
      $\beta$ cannot be larger or equal to $x$.
    \end{proof}

  \item Prove that for an ordinal $\alpha$, the following conditions are
    equivalent:
    \begin{enumerate}
      \item $\alpha$ is the least infinite ordinal. \label{cond:inf}
      \item $\alpha$ is the least non-empty limit ordinal.
        \label{cond:limit}
      \item $\alpha$ is the least non-empty ordinal satisfying
        $x\in\alpha\implies S(x)\in\alpha$. \label{cond:closed-succ}
    \end{enumerate}

    \begin{proof}
      We use the proof of Claim 1.19. By the infinity axiom, there is a
      set $A$ that contains $\emptyset$ and is closed under the successor
      operation. Let $\mathcal{S}$ be the set of all subsets of $A$ that
      contains $\emptyset$ and is closed under successor, let
      $A_0=\bigcap\mathcal{S}$, and let $\alpha$ be the set of elements in
      $A_0$ that are ordinals. The proof of Claim 1.19 shows that $\alpha$
      is an ordinal. We further show that $\alpha$ satisfies the three
      conditions~(\ref{cond:inf}), (\ref{cond:limit}), and
      (\ref{cond:closed-succ}). \\

      We first show that $\alpha$ is the least non-empty ordinal that is
      closed under successor. If there is a smaller ordinal $\beta$ that is
      also non-empty and closed under successor, then it would be a
      strict subset of $\alpha$, and by construction, would be a subset of
      $A$ that is non-empty and closed under successor. Then by
      construction of $\alpha$, $\alpha$ must be equal to or a subset of
      $\beta$, contradicting the choice of $\beta$. \\

      Next, we show that $\alpha$ is also the least non-empty limit ordinal.
      $\alpha$ is clearly non-empty since it contains the empty set. Assume
      by contradiction that $\beta$ is a non-empty limit ordinal that is
      smaller than $\alpha$. Being a limit ordinal, $\beta$ must be closed
      under successor: if there is a $\gamma\in\beta$ whose successor
      $S(\gamma)$ is not in $\beta$, then $\beta$ must either be equal
      $S(\gamma)$ or be smaller than $S(\gamma)$. The former cannot be true
      since $\beta$ is a limit ordinal. Yet the latter also cannot be true
      because ordinals containing $\gamma$ must either equal $S(\gamma)$ or
      contain $S(\gamma)$. So $\beta$ is non-empty, closed under
      successor, and smaller than $\alpha$, which contradicts our earlier
      proof that $\alpha$ is the least non-empty ordinal closed under
      successor. \\

      Finally, we show that $\alpha$ is also the least infinite ordinal.
      Assume by contradiction that the least infinite ordinal $\beta$ is
      contained in $\alpha$. Then since $\alpha$ is the least non-empty
      limit ordinal as we have shown in the previous paragraph, $\beta$
      must be a successor ordinal. Then the predecessor of $\beta$ would
      also be an infinite ordinal, which contradicts our choice of $\beta$
      as the least infinite ordinal. 
    \end{proof}

  \item Consider the set of positive natural numbers
    $\mathbb{N}^+=\{1,2,\ldots\}$. Define the relation $x\in y$ on
    $\mathbb{N}^+$ as
    \begin{equation*}
      x\in y\; \Leftrightarrow\; x<y\; \text{and}\; x|y.
    \end{equation*}
    Which of the following axioms of ZF hold in $(\mathbb{N}^+,\in)$:
    Extensionality, Foundation, Pairing, Union, Separation, Power set,
    Empty set, Replacement?

    \begin{proof}
      Observe that given $n\in\mathbb{N}^+$, the elements in $n$ are
      exactly its positive divisors smaller than itself. Also, all sets
      are finite, 1 is the only empty set, and all other sets contains 1.
      \\

      Extensionality does not hold. 2 and 3 both contain only 1, yet they
      are not equal. \\

      Foundation holds. Given $n\in\mathbb{N}^+$, if $n=1$, then $n$ is the
      empty set and foundation holds trivially. If $n\neq 1$, then $1\in
      n$, and $1$ is the minimal element of $n$ since 1 is the empty set.
      \\

      Pairing does not hold. There is no element that contains exactly 2
      and 3, because all elements either contain 1, or is 1 itself. \\

      Union does not hold. Consider positive integers of the form
      $n=p^2q^2$, where $p$ and $q$ are distinct primes. Then
      $n=\{1,p,p^2,q,q^2,pq,p^2q,qp^2\}$, and $\bigcup
      n=\{1,p,q,pq,p^2,q^2\}$. Suppose $\bigcup n=m$ for some positive
      integer $m$. Then $p^2$ and $q^2$ divide $m$, hence $p^2q$ also
      divide $m$, yet $p^2q$ is not contained in $m$, a contradiction. \\

      Separation does not hold. Consider the formula $\theta(n):=\exists
      m\,(m\in n)$, which says ``$n$ is not empty''. Then
      $k=\{n\in4:\theta(n)\}=\{2\}$, but every set containing 2 must also
      contain 1, so no such $k$ exists. \\

      Power set does not hold. The power set of $6$ does not exist.
      $6=\{1,2,3\}$, so the possible subsets are $\emptyset=1$,
      $\{1\}=p$ for any prime $p$, $\{1,2\}=4$, $\{1,3\}=9$, $\{1,2,3\}=6$,
      and $\{2,3\}$ which does not exist in $\mathbb{N}^+$. Hence the power
      set of 6 $\mathcal{P}(6)$ should be a set containing exactly 1, 4, 6,
      and 9, and also a non-zero finite number of primes $p_1,\ldots,p_k$.
      However, since $\mathcal{P}(6)$ contains 4 and 9, it must also contain
      $4\times3=12$, a contradiction. \\

      Empty set axiom holds. 1 is the empty set. \\

      Replacement does not hold. Let $\varphi(x,y)$ be the formula
      describing the function that sends 1 to 1 and any other positive
      number $n$ to $\{1,n\}$, if such a number exists and is unique. More
      precisely, we can write $\varphi(x,y)$ as:
      \begin{equation*}
        (\text{empty}(x)\wedge\text{empty}(y))\; \vee
        (\neg\text{empty}(x)\wedge \text{square}(x,y)),
      \end{equation*}
      where $\text{empty}(x)$ says ``set $x$ is the empty'', abbreviating
      $\forall n (n\not\in x)$, and $\text{square}(x,y)$ says ``$y=\{1,x\}$'',
      abbreviated by
      \begin{equation*}
        \forall w (w\in y \leftrightarrow (\text{empty}(w)\vee w=x))
      \end{equation*}

      Consider the set of the image of 4 under $\varphi$. $4=\{1,2\}$, so
      under the function described by $\varphi$, 1 should be sent to 1 and
      2 should be sent to $\{1,2\}=4$. Yet, the set of images $\{1,4\}$
      does not exist since any number that contains 4 must also contain 2.
    \end{proof}
\end{enumerate}
\end{document}
