\documentclass{article}
\usepackage[left=3cm,right=3cm,top=3cm,bottom=3cm]{geometry}
\usepackage{amsmath,amssymb,amsthm}
\usepackage{color}
%\setlength{\parindent}{0mm}

\newcommand{\TODO}[1]{\textcolor{red}{TODO: #1}}

\begin{document}
\title{Basic Logic I: Homework Set 3}
\author{Li Ling Ko\\ lko@nd.edu}
\date{\today}
\maketitle

\begin{enumerate}
  \item Show that if $\mathcal{M}$ is a finite $\mathcal{L}$-structure and
    $\mathcal{N}$ is elementarily equivalent to $\mathcal{M}$ then
    $\mathcal{N}$ is also finite.
    \begin{proof}
      Let $m\in\mathbb{N}$ be the number of elements in $M$. Then the
      sentence $\phi(m)$ which says ``I do not have more than $m$
      elements'' is true in $\mathcal{M}$. $\phi(m)$ can be written as a
      first order sentence as
      \begin{equation*}
        \phi(m) := \forall(v_1,\ldots,v_{m+1}) \mathop{\bigvee}_{1\leq
        i<j\leq m+1} v_i=v_j.
      \end{equation*}
      By elementary equivalence, $\phi(m)$ also be true in $\mathcal{N}$,
      and so $\mathcal{N}$ must also be finite.
    \end{proof}

  \item Show that there is a sentence $\varphi$ in the language of groups
    $\mathcal{L}_g$ such that $\mathcal{M}\vDash\varphi$ if and only if
    $\mathcal{M}\cong\mathbb{Z}/2\mathbb{Z}\times\mathbb{Z}/2\mathbb{Z}$.
    \begin{proof}
      There are only two groups of order 4 up to isomorphism: the cyclic
      group $\mathbb{Z}/4\mathbb{Z}$, and the Klein four group
      $\mathcal{M}\cong\mathbb{Z}/2\mathbb{Z}\times\mathbb{Z}/2\mathbb{Z}$.
      Each element of the Klein four group has order less than or equal to
      2, while some elements of the $\mathbb{Z}/4\mathbb{Z}$ has order
      equal to 4. Hence to distinguish between the two groups we can use
      the sentence $\varphi_0$ which says ``The square of each element
      equals the identity.'' Also, we need another two sentences
      $\varphi_1$ and $\varphi_2$ which say ``I have at least four
      elements.'' and ``I do not have more than four elements.'' Then
      $\varphi=\varphi_0\wedge\varphi_1\wedge\varphi_2$ is a sentence that
      will be satisfied by exactly all Klein four groups.
      \begin{align*}
        \varphi   &= \varphi_0\wedge\varphi_1\wedge\varphi_2 \\
        \varphi_0 &= \forall v v\cdot v=e \\
        \varphi_1 &= \exists v_1,\ldots,v_4 \bigvee_{1\leq i<j\leq 4} \neg
                      v_i=v_j \\
        \varphi_2 &= \forall(v_1,\ldots,v_5) \mathop{\bigvee}_{1\leq
                      i<j\leq 5} v_i=v_j. \\
      \end{align*}
    \end{proof}
\end{enumerate}
\end{document}
