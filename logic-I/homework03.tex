\documentclass{article}
\usepackage[left=3cm,right=3cm,top=3cm,bottom=3cm]{geometry}
\usepackage{amsmath,amssymb,amsthm}
\usepackage{color}
%\setlength{\parindent}{0mm}

\newcommand{\TODO}[1]{\textcolor{red}{TODO: #1}}

\begin{document}
\title{Basic Logic I: Homework Set 3}
\author{Li Ling Ko\\ lko@nd.edu}
\date{\today}
\maketitle

\begin{enumerate}
  \item Show that if $\mathcal{M}$ is a finite $\mathcal{L}$-structure and
    $\mathcal{N}$ is elementarily equivalent to $\mathcal{M}$ then
    $\mathcal{N}$ is also finite.
    \begin{proof}
      Let $m\in\mathbb{N}$ be the number of elements in $M$. Then the
      sentence $\phi(m)$ which says ``I do not have more than $m$
      elements'' is true in $\mathcal{M}$. $\phi(m)$ can be written as a
      first order sentence as
      \begin{equation*}
        \phi(m) := \forall(v_1,\ldots,v_{m+1}) \mathop{\bigvee}_{1\leq
        i<j\leq m+1} v_i=v_j.
      \end{equation*}
      By elementary equivalence, $\phi(m)$ also be true in $\mathcal{N}$,
      and so $\mathcal{N}$ must also be finite.
    \end{proof}

  \item Show that there is a sentence $\varphi$ in the language of groups
    $\mathcal{L}_g$ such that $\mathcal{M}\vDash\varphi$ if and only if
    $\mathcal{M}\cong\mathbb{Z}/2\mathbb{Z}\times\mathbb{Z}/2\mathbb{Z}$.
    \begin{proof}
      There are only two groups of order 4 up to isomorphism: the cyclic
      group $\mathbb{Z}/4\mathbb{Z}$, and the Klein four group
      $\mathcal{M}\cong\mathbb{Z}/2\mathbb{Z}\times\mathbb{Z}/2\mathbb{Z}$.
      Each element of the Klein four group has order less than or equal to
      2, while some elements of the $\mathbb{Z}/4\mathbb{Z}$ has order
      equal to 4. Hence to distinguish between the two groups we can use
      the sentence $\varphi_0$ which says ``The square of each element
      equals the identity.'' Also, we need another two sentences
      $\varphi_1$ and $\varphi_2$ which say ``I have at least four
      elements.'' and ``I do not have more than four elements.'' Then
      $\varphi=\varphi_0\wedge\varphi_1\wedge\varphi_2$ is a sentence that
      will be satisfied by exactly all Klein four groups.
      \begin{align*}
        \varphi   &= \varphi_0\wedge\varphi_1\wedge\varphi_2 \\
        \varphi_0 &= \forall v v\cdot v=e \\
        \varphi_1 &= \exists v_1,\ldots,v_4 \bigvee_{1\leq i<j\leq 4} \neg
                      v_i=v_j \\
        \varphi_2 &= \forall(v_1,\ldots,v_5) \mathop{\bigvee}_{1\leq
                      i<j\leq 5} v_i=v_j. \\
      \end{align*}
    \end{proof}

  \item Let $\mathcal{L}$ be the language containing only one binary
    relational symbol $\leq$, and $\mathcal{M}$ be the
    $\mathcal{L}$-structure whose universe is the set of positive
    integers, with $\leq^{\mathcal{M}}$ defined as $n\leq^{\mathcal{M}}m$
    iff $n$ divides $m$.

    \begin{enumerate}
      \item Show that the set $\{1\}$ and the set of all primes are both
        definable over the empty set.
        \begin{proof}
          The set $\{1\}$ can be defined by the formula
          \begin{equation*}
            \varphi_1(x) := \forall v\; (x\leq^{\mathcal{M}} v).
          \end{equation*}
          The set of all primes can be defined by the formula
          \begin{equation*}
            \varphi_p(x) := \forall v\; (v\leq^{\mathcal{M}}x \rightarrow
            (\varphi_1(v) \vee v=x))\; \wedge\; \neg\varphi_1(x).
          \end{equation*}
        \end{proof}

      \item Show that the set of square-free numbers is definable in
        $\mathcal{M}$.
        \begin{proof}
          The set of square-free numbers can be defined by the formula
          \begin{equation*}
            \varphi_{\text{sq-free}}(x) := \forall v\;
            (v\leq^{\mathcal{M}}x\; \rightarrow\;
            \neg\varphi_{p^2}(v)),
          \end{equation*}
          where formula $\varphi_{p^2}(x)$ says $x$ is a square of a prime.
          Since the squares of primes are exactly the numbers that have
          exactly three distinct divisors, we can define $\varphi_{p^2}(x)$
          as
          \begin{equation*}
            \varphi_{p^2}(x) := \varphi_{\geq 3\,\text{divisors}}(x) \wedge
            \varphi_{\leq 3\,\text{divisors}}(x),
          \end{equation*}
          where
          \begin{align*}
            \varphi_{\geq 3\,\text{divisors}}(x) :=\;
              & \exists v_1,v_2,v_3\; ( \\
              & v_1\leq^{\mathcal{M}}x\,\wedge\,
                v_2\leq^{\mathcal{M}}x\,\wedge\,
                v_3\leq^{\mathcal{M}}x\,\wedge\, \\
              & \neg v_1=v_2\, \wedge\,
                \neg v_1=v_3\, \wedge\,
                \neg v_2=v_3), \\
            \varphi_{\leq 3\,\text{divisors}}(x) :=\;
              & \forall v_1,v_2,v_3,v_4\; ( \\
              & v_1\leq^{\mathcal{M}}x\,\wedge\,
                v_2\leq^{\mathcal{M}}x\,\wedge\,
                v_3\leq^{\mathcal{M}}x\,\wedge\,
                v_4\leq^{\mathcal{M}}x\;\rightarrow\; \\
              & v_1=v_2\, \vee\,
                v_1=v_3\, \vee\,
                v_1=v_4\, \vee\,
                v_2=v_3\, \vee\,
                v_2=v_4\, \vee\,
                v_3=v_4). \\
          \end{align*}
        \end{proof}

      \item Let $\mathcal{M}$ be the graph whose universe is the set of all
        two element sets $\{m,n\}$ of natural numbers with the relation
        $R^{\mathcal{M}}$ defined as $a\,R^{\mathcal{M}}\,b$ iff $a\cap
        b\neq\emptyset$ and $a\neq b$.

        \begin{enumerate}
          \item Show that $\mathcal{M}$ is not minimal.
            \begin{proof}
              The formula $\varphi(x):=x\,R^{\mathcal{M}}\,\{0,1\}$
              defines the infinite set of elements
              \begin{equation*}
                \{\{0,2\},\{0,3\},\ldots,\{1,2\},\{1,3\},\ldots\},
              \end{equation*}
              which does not contain any of the elements
              \begin{equation*}
                \{\{3,4\},\{3,5\},\ldots\}.
              \end{equation*}
              Hence $\mathcal{M}$ is not minimal.
            \end{proof}

          \item Show that $\mathcal{M}$ has infinitely many minimal
            definable subsets.
            \begin{proof}
              Fix any $n\in\mathbb{N}\setminus\{0,1\}$. Consider the
              sentence
              \begin{equation*}
                \varphi_n(x) := x=\{0,n\}\;\vee\; x=\{1,n\}\;\vee\;
                (x\,R^{\mathcal{M}}\, \{0,n\}\,\wedge\,
                \neg\,x\,R^{\mathcal{M}}\,\{0,1\}).
              \end{equation*}
              Then $\varphi_n(x)$ defines the set
              \begin{equation*}
                S_n := \{\{i,n\}\,|\, i\in\mathbb{N},\,i\neq n\}\subset M.
              \end{equation*}
              We show that $S_n$ is a minimal definable subset by using the
              invariance of definable sets under automorphisms. Let
              $\phi(x,\overline{a})$ be a first order formula that defines
              an infinite subset $A_0$ of $S_n$, where $\overline{a}\subset
              M$ is a finite subset of $M$. Assume by contradiction that
              the set $A_1=S_n-A_0$ is also definable. We construct an
              automorphism $\alpha$ of $\mathcal{M}$ that preserves
              elements in $\overline{a}$, but that sends elements in $A_0$ to
              $A_1$, contradicting invariance of definable sets
              under automorphisms. Observe that any permutation of
              $\mathbb{N}$ would induce an automorphism of $\mathcal{M}$.
              Now since $\overline{a}$ contains a finite number of
              elements, there must be a natural number $m_0>n$ that does
              not appear in any of the sets in $\overline{a}$ and such that
              $\{m_0,n\}$ is contained in $A_0$. Similarly, since $A_1$ is
              infinite, there must be a natural number $m_1>m_0$ such that
              $\{m_1,n\}$ is contained in $A_1$. Let $\alpha$ be induced by
              the permutation of $\mathcal{N}$ which swaps $m_0$ with
              $m_1$. Then $\alpha$ would send $\{m_0,n\}\in A_0$ to
              $\{m_1,n\}\not\in A_0$, a contradiction. \\

              For each $n\in\mathbb{N}\setminus\{0,1\}$, $S_n$ is unique,
              hence we have infinitely many minimal definable and infinite
              subsets $S_n$.
            \end{proof}
        \end{enumerate}
    \end{enumerate}
\end{enumerate}
\end{document}
