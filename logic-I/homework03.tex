\documentclass{article}
\usepackage[left=3cm,right=3cm,top=3cm,bottom=3cm]{geometry}
\usepackage{amsmath,amssymb,amsthm}
\usepackage{color}
%\setlength{\parindent}{0mm}

\newcommand{\TODO}[1]{\textcolor{red}{TODO: #1}}

\begin{document}
\title{Basic Logic I: Homework Set 3}
\author{Li Ling Ko\\ lko@nd.edu}
\date{\today}
\maketitle

\begin{enumerate}
  \item Show that if $\mathcal{M}$ is a finite $\mathcal{L}$-structure and
    $\mathcal{N}$ is elementarily equivalent to $\mathcal{M}$ then
    $\mathcal{N}$ is also finite.
    \begin{proof}
      Let $m\in\mathbb{N}$ be the number of elements in $M$. Then the
      sentence $\phi(m)$ which says ``I do not have more than $m$
      elements'' is true in $\mathcal{M}$. $\phi(m)$ can be written as a
      first order sentence as
      \begin{equation*}
        \phi(m) := \forall(v_1,\ldots,v_{m+1}) \mathop{\bigvee}_{1\leq
        i<j\leq m+1} v_i=v_j.
      \end{equation*}
      By elementary equivalence, $\phi(m)$ also be true in $\mathcal{N}$,
      and so $\mathcal{N}$ must also be finite.
    \end{proof}

  \item Show that there is a sentence $\varphi$ in the language of groups
    $\mathcal{L}_g$ such that $\mathcal{M}\models\varphi$ if and only if
    $\mathcal{M}\cong\mathbb{Z}/2\mathbb{Z}\times\mathbb{Z}/2\mathbb{Z}$.
    \begin{proof}
      There are only two groups of order 4 up to isomorphism: the cyclic
      group $\mathbb{Z}/4\mathbb{Z}$, and the Klein four group
      $\mathcal{M}\cong\mathbb{Z}/2\mathbb{Z}\times\mathbb{Z}/2\mathbb{Z}$.
      Each element of the Klein four group has order less than or equal to
      2, while some elements of the $\mathbb{Z}/4\mathbb{Z}$ has order
      equal to 4. Hence to distinguish between the two groups we can use
      the sentence $\varphi_{v^2=e}$ which says ``The square of each
      element equals the identity.'' Also, we need another two sentences
      $\varphi_{\geq4}$ and $\varphi_{\leq4}$ which respectively say ``I
      have at least four elements.'' and ``I do not have more than four
      elements.'' Then
      $\varphi_{\text{K4}}:=\varphi_{v^2=e}\wedge\varphi_{\geq4}\wedge\varphi_{\leq4}$
      is a sentence that will be satisfied by exactly all Klein four
      groups.
      \begin{align*}
        \varphi_{\text{K4}}   &:=
          \varphi_{v^2=e}\wedge\varphi_{\geq4}\wedge\varphi_{\leq4} \\
        \varphi_{v^2=e} &:= \forall v\; v\cdot v=e \\
        \varphi_{\geq4} &:= \exists v_1,\ldots,v_4 \bigwedge_{1\leq i<j\leq 4}
          \neg v_i=v_j \\
        \varphi_{\leq4} &:= \forall v_1,\ldots,v_5 \mathop{\bigvee}_{1\leq
          i<j\leq 5} v_i=v_j. \\
      \end{align*}
    \end{proof}

  \item Let $\mathcal{L}$ be the language containing only one binary
    relational symbol $\leq$, and $\mathcal{M}$ be the
    $\mathcal{L}$-structure whose universe is the set of positive
    integers, with $\leq^{\mathcal{M}}$ defined as $n\leq^{\mathcal{M}}m$
    iff $n$ divides $m$.

    \begin{enumerate}
      \item Show that the set $\{1\}$ and the set of all primes are both
        definable over the empty set.
        \begin{proof}
          The set $\{1\}$ can be defined by the formula
          \begin{equation*}
            \varphi_1(x) := \forall v\; (x\leq^{\mathcal{M}} v).
          \end{equation*}
          The set of all primes can be defined by the formula
          \begin{equation*}
            \varphi_p(x) := \forall v\; (v\leq^{\mathcal{M}}x \rightarrow
            (\varphi_1(v) \vee v=x))\; \wedge\; \neg\varphi_1(x).
          \end{equation*}
        \end{proof}

      \item Show that the set of square-free numbers is definable in
        $\mathcal{M}$.
        \begin{proof}
          The set of square-free numbers can be defined by the formula
          \begin{equation*}
            \varphi_{\text{sq-free}}(x) := \forall v\;
            (v\leq^{\mathcal{M}}x\; \rightarrow\;
            \neg\varphi_{p^2}(v)),
          \end{equation*}
          where formula $\varphi_{p^2}(x)$ says $x$ is a square of a prime.
          Since the squares of primes are exactly the numbers that have
          exactly three distinct divisors, we can define $\varphi_{p^2}(x)$
          as
          \begin{equation*}
            \varphi_{p^2}(x) := \varphi_{\geq 3\,\text{divisors}}(x) \wedge
            \varphi_{\leq 3\,\text{divisors}}(x),
          \end{equation*}
          where
          \begin{align*}
            \varphi_{\geq 3\,\text{divisors}}(x) :=\;
              & \exists v_1,v_2,v_3\; ( \\
              & v_1\leq^{\mathcal{M}}x\,\wedge\,
                v_2\leq^{\mathcal{M}}x\,\wedge\,
                v_3\leq^{\mathcal{M}}x\,\wedge\, \\
              & \neg v_1=v_2\, \wedge\,
                \neg v_1=v_3\, \wedge\,
                \neg v_2=v_3), \\
            \varphi_{\leq 3\,\text{divisors}}(x) :=\;
              & \forall v_1,v_2,v_3,v_4\; ( \\
              & v_1\leq^{\mathcal{M}}x\,\wedge\,
                v_2\leq^{\mathcal{M}}x\,\wedge\,
                v_3\leq^{\mathcal{M}}x\,\wedge\,
                v_4\leq^{\mathcal{M}}x\;\rightarrow\; \\
              & v_1=v_2\, \vee\,
                v_1=v_3\, \vee\,
                v_1=v_4\, \vee\,
                v_2=v_3\, \vee\,
                v_2=v_4\, \vee\,
                v_3=v_4). \\
          \end{align*}
        \end{proof}
    \end{enumerate}

  \item Let $\mathcal{M}$ be the graph whose universe is the set of all
    two element sets $\{m,n\}$ of natural numbers with the relation
    $R^{\mathcal{M}}$ defined as $a\,R^{\mathcal{M}}\,b$ iff $a\cap
    b\neq\emptyset$ and $a\neq b$.

    \begin{enumerate}
      \item Show that $\mathcal{M}$ is not minimal.
        \begin{proof}
          The formula $\varphi(x):=x\,R^{\mathcal{M}}\,\{0,1\}$
          defines the infinite set of elements
          \begin{equation*}
            \{\{0,2\},\{0,3\},\ldots,\{1,2\},\{1,3\},\ldots\},
          \end{equation*}
          which does not contain any of the elements
          \begin{equation*}
            \{\{3,4\},\{3,5\},\ldots\}.
          \end{equation*}
          Hence $\mathcal{M}$ is not minimal.
        \end{proof}

      \item Show that $\mathcal{M}$ has infinitely many minimal
        definable subsets.
        \begin{proof}
          Fix any $n\in\mathbb{N}\setminus\{0,1\}$. Consider the
          sentence
          \begin{equation*}
            \varphi_n(x) := x=\{0,n\}\;\vee\; x=\{1,n\}\;\vee\;
            (x\,R^{\mathcal{M}}\, \{0,n\}\,\wedge\,
            \neg\,x\,R^{\mathcal{M}}\,\{0,1\}).
          \end{equation*}
          Then $\varphi_n(x)$ defines the set
          \begin{equation*}
            S_n = \{\{i,n\}\,|\, i\in\mathbb{N},\,i\neq n\}\subset M.
          \end{equation*}
          We show that $S_n$ is a minimal definable subset by using the
          invariance of definable sets under automorphisms. Let
          $\phi(x,\overline{a})$ be a first order formula that defines
          an infinite subset $A_0$ of $S_n$, where $\overline{a}\subset
          M$ is a finite subset of $M$. Assume by contradiction that
          the set $A_1=S_n-A_0$ is also definable. We construct an
          automorphism $\alpha$ of $\mathcal{M}$ that preserves
          elements in $\overline{a}$, but that sends elements in $A_0$ to
          $A_1$, contradicting invariance of definable sets
          under automorphisms. Observe that any permutation of
          $\mathbb{N}$ would induce an automorphism of $\mathcal{M}$.
          Now since $\overline{a}$ contains a finite number of
          elements, there must be a natural number $m_0>n$ that does
          not appear in any of the sets in $\overline{a}$ and such that
          $\{m_0,n\}$ is contained in $A_0$. Similarly, since $A_1$ is
          infinite, there must be a natural number $m_1>m_0$ such that
          $\{m_1,n\}$ is contained in $A_1$. Let $\alpha$ be induced by
          the permutation of $\mathcal{N}$ which swaps $m_0$ with
          $m_1$. Then $\alpha$ would send $\{m_0,n\}\in A_0$ to
          $\{m_1,n\}\not\in A_0$, a contradiction. \\

          For each $n\in\mathbb{N}\setminus\{0,1\}$, $S_n$ is unique,
          hence we have infinitely many minimal definable and infinite
          subsets $S_n$.
        \end{proof}
    \end{enumerate}

  \item Let $\mathcal{M}$ be the structure
    $\langle\mathbb{R},<\rangle$ and $\mathcal{N}$ the substructure of
    $\mathcal{M}$ whose universe is the open interval $(0,1)$. Show
    that $\mathcal{N}$ is an elementary substructure of $\mathcal{M}$.

    \begin{proof}
      We use Tarski-Vaught test by applying induction on complexity of
      formulas $\varphi$ that for any
      $\overline{a}=(a_1,\ldots,a_n)\subset(0,1)$, if
      $\mathcal{M}\models\varphi(\overline{a},\overline{b})$ for some
      $\overline{b}=(b_1,\ldots,b_m)\subset\mathbb{R}$, then
      $\mathcal{N}\models\varphi(\overline{a},\overline{c})$ for some
      $\overline{c}=(c_1,\ldots,c_m)\subset(0,1)$. \\

      Atomic: The atomic formulas we need to consider are of the forms
      $a<b$, $b<a$, or $b_1<b_2$ where $a\in(0,1)$ and
      $b,b_1,b_2\in\mathbb{R}$. For the first case, choose $c=(a+1)/2$. For
      the second case, choose $c=a/2$. For the last case, choose $c_1=0.1$
      and $c_2=0.2$. \\

      $\neg$: Let $\varphi(\overline{a},\overline{b})$ be a formula where
      the property holds via $\overline{c}\subset(0,1)$. Then
      $\neg\varphi(\overline{a},\overline{b})$ holds in $\mathcal{M}$ iff
      $\varphi(\overline{a},\overline{b})$ does not hold in $\mathcal{M}$
      iff $\varphi(\overline{a},\overline{c})$ does not hold in
      $\mathcal{N}$ by induction hypothesis iff
      $\neg\varphi(\overline{a},\overline{c})$ holds in $\mathcal{N}$. \\

      $\wedge$: Let $\varphi_1(\overline{a_1},\overline{b_1})$ be a formula
      where the property holds via $\overline{c_1}\subset(0,1)$, and let
      $\varphi_2(\overline{a_2},\overline{b_2})$ be a formula where the
      property holds via $\overline{c_2}\subset(0,1)$. Then
      $\varphi_1(\overline{a_1},\overline{b_1})\wedge\varphi_2(\overline{a_2},\overline{b_2})$
      holds in $\mathcal{M}$ iff $\varphi_1(\overline{a_1},\overline{b_1})$
      and $\varphi_2(\overline{a_2},\overline{b_2})$ hold in $\mathcal{M}$
      iff $\varphi_1(\overline{a_1},\overline{c_1})$ and
      $\varphi_2(\overline{a_2},\overline{c_2})$ hold in $\mathcal{N}$ via
      induction hypothesis iff
      $\varphi_1(\overline{a_1},\overline{c_1})\wedge\varphi_2(\overline{a_2},\overline{c_2})$
      holds in $\mathcal{N}$. \\

      $\exists$: Let $\varphi(\overline{a},b_0,\overline{b_1})$ be a
      formula where the property holds via $c_0\in(0,1)$ and
      $\overline{c_1}\subset(0,1)$ for any $b_0\in\mathbb{R}$ and
      $\overline{b_1}\subset\mathbb{R}$. Then $\exists
      x\,\varphi(\overline{a},x,\overline{b_1})$ holds in $\mathcal{M}$ iff
      there exists some $b_0\in\mathbb{R}$ such that
      $\varphi(\overline{a},b_0,\overline{b_1})$ holds in $\mathcal{M}$ iff
      $\varphi(\overline{a},c_0,\overline{c_1})$ holds in $\mathcal{M}$ iff
      $\exists x\,\varphi(\overline{a},x,\overline{c_1})$ holds in
      $\mathcal{N}$.
    \end{proof}

    %\begin{proof}
    %  Since isomorphic structures are elementarily equivalent, it
    %  suffices to prove that $\mathcal{M}$ is isomorphic to
    %  $\mathcal{N}=\langle(0,1),<\rangle$. Consider the map
    %  $\varphi:(0,1)\rightarrow\mathbb{R}$ which first
    %  shifts and scales $(0,1)$ to $(-\pi/2,\pi/2)$, then applies the
    %  $\tan$ function to send $(-\pi/2,\pi/2)$ to $(-\infty,\infty)$.
    %  More precisely, define $\varphi(x)=\tan((x-0.5)\pi/2)$. Then
    %  $\varphi$ is bijective and preserves order, and is therefore an
    %  isomorphism between $\mathcal{M}$ and $\mathcal{N}$. Hence
    %  $\mathcal{N}$ is elementarily equivalent to $\mathcal{M}$, and
    %  since $(0,1)\subset\mathbb{R}$, $\mathcal{N}$ is an elementary
    %  substructure of $\mathcal{M}$.
    %\end{proof}

  \item (Tarski's Chain Lemma) Prove that the union of an elemenatary
    directed family is an elementary extension of all its members.
    \begin{proof}
      Let $\mathcal{M}_i$, $i\in I$ be an elementary directed family, and
      let $\mathcal{M}=\cup_{i\in I}\mathcal{M}_i$ be its union. Fix any
      $i\in I$. We wish to show that $\mathcal{M}_i\preceq\mathcal{M}$. We
      apply the Tarski-Vaught test. Let $\varphi(x_1,\ldots,x_n,y)$ be a
      formula and $a_1,\ldots,a_n\in M_i$ such that
      $\mathcal{M}\models\varphi(a_1,\ldots,a_n,b)$ for some $b\in M$. Then
      $b\in M_j$ for some index $j>i$ where $j\in I$. Then since
      $\mathcal{M}_i\preceq\mathcal{M}_j$, there exists some $c\in M_i$
      such that $\mathcal{M}_i\models\varphi(a_1,\ldots,a_n,c)$. Hence
      $\mathcal{M}_i\preceq\mathcal{M}$ by the Tarski-Vaught test.
    \end{proof}

  \item Let $\mathcal{N}$ be an $\mathcal{L}$-structure and
    $\mathcal{M}_i$, $i\in I$ a directed set of elementary substructures of
    $\mathcal{N}$. Show that $\mathcal{M}=\cup_{i\in I}\mathcal{M}_i$ is an
    elementary substructure of $\mathcal{N}$.

    \begin{proof}
      We apply the Tarski-Vaught test. Let $\varphi(x_1,\ldots,x_n,y)$ be a
      formula and $a_1,\ldots,a_n\in M$ such that
      $\mathcal{N}\models\varphi(a_1,\ldots,a_n,b)$ for some $b\in N$. Now
      each $a_i$ is contained in $M_{i_j}$ for some $i_j\in I$. Let $k\in
      I$ be the maximal of all these $i_j$'s. Then $a_1,\ldots,a_n\in M_k$,
      and since $\mathcal{M}_k\preceq\mathcal{N}$, there exists some $c\in
      M_k$, such that $\mathcal{M}_k\models\varphi(a_1,\ldots,a_n,c)$. Then
      from Tarski's Chain Lemma which we have proven in the previous
      question, $\mathcal{M}_k\preceq\mathcal{M}$, and so
      $\mathcal{M}\models\varphi(a_1,\ldots,a_n,c)$ for some $c\in M$. Hence
      $\mathcal{M}\preceq\mathcal{N}$ by the Tarski-Vaught test.
    \end{proof}

  \item Let $\mathcal{L}$ be a language and for each $i\in I$ let $T_i$ be
    an $\mathcal{L}$ theory.
    \begin{enumerate}
      \item Show that $Mod(\cup_{i\in I}T_i)=\cap_{i\in I}Mod(T_i)$.
        \begin{proof}
          The equality can be proven by chasing definitions. \\

          $\subseteq$: Let $\mathcal{M}\in Mod(\cup_{i\in I}T_i)$. Then
          given any $T_i$, $\mathcal{M}$ satisfies each sentence in $T_i$,
          so $\mathcal{M}\in Mod(T_i)$. Hence $\mathcal{M}\in \cap_{i\in
          I}Mod(T_i)$. \\

          $\supseteq$: Let $\mathcal{M}\not\in Mod(\cup_{i\in I}T_i)$.
          Then there exists a sentence in $\cup_{i\in I}T_i$ that is
          not satisfied by $\mathcal{M}$. This sentence must be contained
          in some $T_i$, so $\mathcal{M}$ is not a model of $T_i$. Hence
          $\mathcal{M}$ also cannot be a model of $\cap_{i\in I}Mod(T_i)$.
        \end{proof}

      \item Show that $Mod(\cap_{i\in I}T_i)=\cup_{i\in I}Mod(T_i)$ holds
        when $I$ is finite and each $T_i$ is of the form $Th(K_i)$ for some
        class $K_i$.

        \begin{proof}
          $\supseteq$: This is the easier direction which can be proven
          from chasing definitions. Let $\mathcal{M}\in \cup_{i\in
          I}Mod(T_i)$. Then $\mathcal{M}$ is a model of some $T_j$, and is
          therefore also a model of $\cap_{i\in I}T_i$. Hence,
          $\mathcal{M}\in Mod(\cap_{i\in I}T_i)$. \\

          $\subseteq$: Assume $I=\{1,\ldots,n\}$ and each $T_i$ is of the
          form $Th(K_i)$ for some class $K_i$. Also assume that
          $\mathcal{M}\not\in\cup_{i\leq n}Mod(T_i)$. We wish to show that
          $\mathcal{M}\not\in Mod(\cap_{i\leq n}T_i)$. For each $1\leq
          i\leq n$, since $\mathcal{M}\not\in Mod(Th(K_i))$, there exists some
          sentence $\varphi_i$ satisfied by all models in $K_i$ but is not
          satisfied by $\mathcal{M}$. Let
          $\varphi=\varphi_1\vee\ldots\vee\varphi_n$ be the disjunctions
          of these sentences. Then $\mathcal{M}$ does not satisfy
          $\varphi$, but for each $1\leq i\leq n$, all models in $K_i$
          satisfy $\varphi$ since these models satisfy $\varphi_i$, and
          hence $\varphi\in\cap_{i\leq n}T_i$. This implies
          $\mathcal{M}\not\in Mod(\cap_{i\leq n}T_i)$ as we are required to
          show.
        \end{proof}
    \end{enumerate}
\end{enumerate}
\end{document}
