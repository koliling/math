\documentclass{article}
\usepackage[left=3cm,right=3cm,top=3cm,bottom=3cm]{geometry}
\usepackage{amsmath,amssymb,amsthm}
\usepackage{color}
%\setlength{\parindent}{0mm}

\newcommand{\TODO}[1]{\textcolor{red}{TODO: #1}}

\begin{document}
\title{Basic Logic I: Homework Set 3}
\author{Li Ling Ko\\ lko@nd.edu}
\date{\today}
\maketitle

\begin{enumerate}
  \item Show that if $\mathcal{M}$ is a finite $\mathcal{L}$-structure and
    $\mathcal{N}$ is elementarily equivalent to $\mathcal{M}$ then
    $\mathcal{N}$ is also finite.
    \begin{proof}
      Let $m\in\mathbb{N}$ be the number of elements in $M$. Then the
      sentence $\phi(m)$ which says ``I do not have more than $m$
      elements'' is true in $\mathcal{M}$. $\phi(m)$ can be written as a
      first order sentence as
      \begin{equation*}
        \phi(m) := \forall(v_1,\ldots,v_{m+1}) \mathop{\bigvee}_{1\leq
        i<j\leq m+1} v_i=v_j.
      \end{equation*}
      By elementary equivalence, $\phi(m)$ also be true in $\mathcal{N}$,
      so $\mathcal{N}$ must also be finite.
    \end{proof}
\end{enumerate}
\end{document}
