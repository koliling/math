\documentclass{article}
\usepackage[left=3cm,right=3cm,top=3cm,bottom=3cm]{geometry}
\usepackage{amsmath,amssymb,amsthm}
\usepackage{color}
%\setlength{\parindent}{0mm}

\newcommand{\TODO}[1]{\textcolor{red}{TODO: #1}}

\begin{document}
\title{Basic Logic I: Homework Set 4}
\author{Li Ling Ko\\ lko@nd.edu}
\date{\today}
\maketitle

\begin{enumerate}
  \item Let $\mathcal{L}=\{+,<,0\}$ be the original language, and let
    $\mathcal{L}_1=\{+,<,0,1,c\}$ be an extension of $\mathcal{L}$ by
    adding two new constants $1$ and $c$. Also, let
    $T=Th((\mathbb{Q},+,<,0,1))$ be the theory of $\mathbb{Q}$ in the
    language $\{+,<,0,1\}$. For each $n\in\mathbb{N}^+$, let $\varphi_n$ be
    the sentence 
    \begin{equation*}
      \varphi_n := \neg\,(1 < \underbrace{c+\ldots+c}_{n\,\text{times}}).
    \end{equation*}
    Then $T'=T\cup\{\varphi_n:n\in\mathbb{N}^+\}\cup\{\neg c=0\}$ is
    finitely satisfiable by $\mathbb{Q}$: Given any finite subset $S\subset
    T'$, let $N\in\mathbb{N}^+$ be the largest natural number such that
    $\varphi_N$ appears in $S$. Then $\mathbb{Q}$ would satisfy $S$ with
    $c$ assigned to $1/(N+1)$. Hence by compactness theorem,
    $T'$ is satisfiable and has a model $\mathcal{M}$. Now in the original
    language $\mathcal{L}$, $\mathcal{M}$ is elementarily equivalent to
    $\mathbb{Q}$ because $\mathcal{M}\models T$, and $T$ is a complete
    theory in language $\mathcal{L}$. Also, $\mathcal{M}$ is not
    archimedian because in $\mathcal{M}$, no finite sum of $c$ is greater
    than 1.

  \item In language $\mathcal{L}=\{<\}$, let
    $T=Th(\langle\mathbb{Z},<\rangle)$ be the complete theory of
    $\mathbb{Z}$. Consider the expanded language $\mathcal{L}(\mathbb{Q})$
    and the sentences $S=\{c_a<c_b|\; a,b\in\mathbb{Q}, a<b\}\cup\{c_a\neq
    c_b|\; a,b\in\mathbb{Q}, a<b\}$. We show that $T\cup S$ is finitely
    satisfiable in the new language $\mathcal{L}(\mathbb{Q})$: Given any
    finite subset $S'\subset T\cup S$, $\mathbb{Z}$ will satisfy $S'$ as
    long as we assign the constants in an order-preserving manner. Assuming
    that $c_{a_1}<\ldots<c_{a_n}$ are the constants that appear in $S'$, we
    can assign $c_{a_i}$ to $i\in\mathbb{Z}$. This assignment will ensure
    that $\mathbb{Z}$ satisfies $S'$. Hence by compactness
    theorem, $T\cup S$ is satisfiable, and has a model $\mathcal{M}$. We
    show that $\mathcal{M}$ satisfies the required criteria. \\

    First we show that in the original language $\mathcal{L}=\{<\}$,
    $\mathbb{Q}$ can be embedded into $\mathcal{M}$. Consider the map
    $i:\mathbb{Q}\rightarrow M,q\mapsto c_q$. This map is injective since
    in $\mathcal{L}(\mathbb{Q})$, $\mathcal{M}\models \{c_a\neq c_b|\;
    a,b\in\mathbb{Q}, a<b\}$. Also, this map is an embedding because
    $a<b\leftrightarrow c_a<c_b$. Hence $\mathbb{Q}$ can be embedded into
    $\mathcal{M}$ in the language $\mathcal{L}=\{<\}$. \\

    Next we show that in the original language $\mathcal{L}=\{<\}$,
    $\mathbb{Z}$ can be embedded into $\mathcal{M}$, using a proof similar
    to the previous paragraph: Consider the map $j:\mathbb{Z}\rightarrow
    M,a\mapsto c_a$. This map is injective since in
    $\mathcal{L}(\mathbb{Q})$, $\mathcal{M}\models \{c_a\neq c_b|\;
    a,b\in\mathbb{Z}, a<b\}$. Also, this map is an embedding because
    $a<b\leftrightarrow c_a<c_b$. Hence $\mathbb{Z}$ can be embedded into
    $\mathcal{M}$ in the language $\mathcal{L}=\{<\}$. \\

    So $\mathbb{Z}$ is a substructure of $\mathcal{M}$ in the language
    $\mathcal{L}=\{<\}$. It remains to show that in $\mathcal{L}=\{<\}$,
    $\mathbb{Z}$ is an elementary substructure of $\mathcal{M}$. We prove
    this by induction on formulas. We claim that given any formula
    $\varphi(x_1,\ldots,x_n)$ in the language $\mathcal{L}=\{<\}$ and any
    $\bar{a}\in\mathbb{Z}^n$,
    $\langle\mathbb{Z},<\rangle\models\varphi(\bar{a})$ if and only if
    $\mathcal{M}\models\varphi(j(\bar{a}))$. The most important step is the
    atomic formulas; induction by $\neg$, $\wedge$, and $\exists$ follows
    from chasing definitions: \\

    Atomic formulas: Since this language has no constants, atomic formulas
    can only be of the form $u<v$ or $u=v$ where $u$ and $v$ are variables.
    Then
    \begin{align*}
      \mathbb{Z}\models a<b &\leftrightarrow\mathcal{M}\models c_a<c_b. \\
    \end{align*}

    $\neg$: Let $\varphi(\bar{x})$ be a formula where the claim holds.
    Then
    \begin{align*}
      \mathbb{Z}\models\neg\varphi(\bar{a})
        &\leftrightarrow\mathbb{Z}\not\models\varphi(\bar{a}) & \\
        &\leftrightarrow\mathcal{M}\not\models\varphi(j(\bar{a}))  & (\text{by
          induction hypothesis}) \\
        &\leftrightarrow\mathcal{M}\models\neg\varphi(j(\bar{a})). & \\
    \end{align*}

    $\wedge$: Let $\varphi_1(\bar{x})$, $\varphi_2(\bar{y})$ be formulas
    where the claim holds. Then
    \begin{align*}
      \mathbb{Z}\models\varphi_1(\bar{a})\wedge\varphi_2(\bar{b})
        &\leftrightarrow\mathbb{Z}\models\varphi_1(\bar{a})\;\text{and}\;
        \mathbb{Z}\models\varphi_2(\bar{b})& \\
        &\leftrightarrow\mathcal{M}\models\varphi_1(j(\bar{a}))\;\text{and}\;
        \mathbb{Z}\models\varphi_2(j(\bar{b}))& (\text{by induction
        hypothesis}) \\
        &\leftrightarrow\mathcal{M}\models\varphi_1(j(\bar{a}))\wedge\varphi_2(j(\bar{b})). & \\
    \end{align*}

    $\exists$: Let $\varphi(\bar{x},y)$ be a formula where the claim holds.
    Then
    \begin{align*}
      \mathbb{Z}\models\exists y\, \varphi(\bar{a},y)
        &\leftrightarrow\mathbb{Z}\models\varphi(\bar{a},b)\; \text{for
        some}\; b\in\mathbb{Z} & \\
        &\leftrightarrow\mathcal{M}\models\varphi(j(\bar{a}),j(b))\;
        \text{for some}\; b\in\mathbb{Z}& (\text{by induction hypothesis})
        \\
        &\leftrightarrow\mathcal{M}\models\exists y\; \varphi(j(\bar{a}),y). & \\
    \end{align*}

  \item Assume that ultrafilter $\mathcal{F}$ does not contain the Frechet
    filter. Then there must exist some cofinite set
    $A=I\setminus\{a_1,\ldots,a_n\}$ that is not in $\mathcal{F}$. Now
    ultrafilters must contain any given subset or its complement (Theorem
    4.3), so $\mathcal{F}$ must contain $B=\{a_1,\ldots,a_n\}$. We show by
    induction on $n$ that $\mathcal{F}$ must contain some $\{a_i\}$:
    $\mathcal{F}$ must contain either $\{a_1,\ldots,a_{n-1}\}$ or $\{a_n\}$
    otherwise $\mathcal{F}$ contains $I\setminus\{a_1,\ldots,a_{n-1}\}$ and
    $I\setminus\{a_n\}$, and from closure under intersection, when these
    two sets intersect with $B$, we get $\mathcal{F}$ contains both
    $\{a_1,\ldots,a_{n-1}\}$ and $\{a_n\}$, which means $\mathcal{F}$
    contains $\emptyset$, a contradiction. Hence $\mathcal{F}$ contains
    some $\{a_i\}$ and therefore is a superset of $\mathcal{F}_{a_i}$.
    Since $\mathcal{F}_{a_i}$ is an ultrafilter and ultrafilters are
    maximal, $\mathcal{F}$ must be the principal ultrafilter
    $\mathcal{F}_{a_i}$. \\

    Conversely, if $\mathcal{F}=\mathcal{F}_a$ is principal, then it
    contains the finite set $\{a\}$ and from Theorem 4.3 cannot contain the
    cofinite set $I\setminus\{a\}$, therefore $\mathcal{F}$ cannot contain
    the Frechet filter.

  \item We first show that $\beta X$ is Hausdorff. Let $\mathcal{U}$ and
    $\mathcal{V}$ be distinct ultrafilters in $\beta X$. Then there must be
    some set $U\in\mathcal{U}$ that is not contained in $\mathcal{V}$. By
    Theorem 4.3, $X\setminus U$ must be contained in $\mathcal{V}$. Then
    $\mathcal{U}\in[U]$ and $\mathcal{V}\in[X\setminus U]$. Also, by
    Theorem 4.3 again, $[U]$ and $[X\setminus U]$ are disjoint. Hence
    $\mathcal{U}$ and $\mathcal{V}$ are separated by disjoint open sets
    $[U]$ and $[X\setminus U]$ respectively. \\

    Now we show that $\beta X$ is compact. First, we show that
    $\mathcal{S}=\{[Y]:Y\subseteq X\}$ is a basis for the closed sets of $\beta
    X$, i.e. any closed set of $\beta X$ is an intersection of sets in
    $\mathcal{S}$: The sets in $\mathcal{S}$ are both closed and open by
    Theorem 4.11.3. Since $\mathcal{S}$ form a basis of the topology, any
    closed set of $\beta X$ must be the complement of a union of sets in
    $\mathcal{S}$, which is the same as the intersection of the complements
    of sets in $\mathcal{S}$. Then since $\beta X-[Y]=[X-Y]$ by Theorem
    4.3, the complement of a set in $\mathcal{S}$ also lies in
    $\mathcal{S}$, hence $\mathcal{S}$ forms a closed basis of $\beta X$.
    \\

    To show that a space is compact is equivalent to proving that any set
    $\mathcal{C}=\{V_i\}_{i\in I}$ of closed sets whose intersection is
    empty must have a finite subset whose intersection is empty. Assume
    that any finite subset of $\mathcal{C}$ is non-empty. We wish to show
    that the intersection of $\mathcal{C}$ is also non-empty. Since
    $\mathcal{S}$ is a basis of the closed sets in $\beta X$, we can assume
    that each $V_i=[Y_i]$ for some $Y_i\subset X$. We construct a filter
    $\mathcal{F}$ using the finite intersection property of $\mathcal{C}$,
    i.e.
    \begin{equation*}
      \mathcal{F} := \{\bigcap_{k=1}^n Y_{i_k}:\; i_k\in I\} \cup \{X\}.
    \end{equation*}
    It is routine to verify that $\mathcal{F}$ is a filter. We extend
    $\mathcal{F}$ to an ultrafilter $\mathcal{U}$. Then since
    $Y_i\in\mathcal{U}$ for every $i\in I$, $\mathcal{U}$ is contained in
    $\cap_{i\in I}[Y_i]$, which completes the proof.

  \item We first claim that given a finite set of subsets
    $Y_1,\ldots,Y_n\subseteq X$, we have
    $\cup_{k=1}^n[Y_k]=[\cup_{k=1}^nY_k]$: By induction on $n$, it suffices
    to prove the case for $n=2$.  $[Y_1]\cup[Y_2]\subseteq[Y_1\cup Y_2]$
    from the upward closure property of ultrafilters. For the reverse
    inclusion, let $\mathcal{U}$ be an ultrafilter that contains $Y_1\cup
    Y_2$. Assume by contradiction that $\mathcal{U}$ but does not contain
    $Y_1$ or $Y_2$. Then $\mathcal{U}$ contains $X\setminus Y_1$, and
    $X\setminus Y_2$, and intersecting these two sets with $Y_1\cup Y_2$,
    we get $\emptyset\in\mathcal{U}$, a contradiction. \\
  
    Let $V\subseteq\beta X$ be open and closed. Then $V=\cup_{i\in
    I}[Y_i]$ for some subsets $Y_i\subseteq X$. Now $V$ is closed subset of
    a compact space, so it must also be compact. Then since the $[Y_i]$'s
    form an open cover of $V$, by compactness $V=\cup_{k=1}^n[Y_{i_k}]$ for
    some finite $n\in\mathbb{N}$. Then from the above claim,
    $V=[\cup_{k=1}^nY_{i_k}]$, which is the form we require.

  \item Consider the map $F:\beta X\rightarrow C$,
    $\mathcal{U}\mapsto\lim_{\mathcal{U}}f$. This map is well-defined by
    Fact 4.9. \\

    We first show that $F$ extends $f$. Given $x\in X$, we need to show
    that $F(\mathcal{F}_x)=\lim_{\mathcal{F}_x}f=f(x)$. By Fact 4.9, it
    suffices to show that for every open set $U\in C$ containing $f(x)$,
    $f^{-1}(U)$ is contained in $\mathcal{F}_x$. This is true since $x\in
    f^{-1}(U)$. \\

    Next we show that there can only be one possible continuous extension
    of $f$, if any. This follows from the density of $\{\mathcal{F}_x:x\in
    X\}$ in $\beta X$ and from the Hausdorff property of $C$: Assume by
    contradiction that $F_1$ and $F_2$ are two distinct continuous
    extensions of $f$. Then there exists some $\mathcal{U}\in\beta X$ such
    that $F_1(\mathcal{U})\neq F_2(\mathcal{U})$. Let $U_1$ and $U_2$ be
    disjoint open sets containing $F_1(\mathcal{U})$ and $F_2(\mathcal{U})$
    respectively. Then $F^{-1}(U_1)\cap F^{-1}(U_2)$ is an open subset of
    $\beta X$ that contains $\mathcal{U}$, and this open subset must
    contain some $\mathcal{F}_x$ from the density of $\{\mathcal{F}_x:x\in
    X\}$. However $F_1(\mathcal{F}_x)$ and $F_2(\mathcal{F}_x)$ are
    contained in disjoint neighborhoods and hence cannot be equal, a
    contradiction. \\

    Since there can only be one possible continuous extension, it suffices
    to show that the function $F$ we defined is continuous. Given
    $\mathcal{U}\in\beta X$, let $V\subset C$ be an open neighborhood of
    $F(\mathcal{U})$. We wish to show that there is an open neighborhood
    $\mathcal{O}\subset\beta X$ of $\mathcal{U}$ such that
    $F(\mathcal{O})\subset V$. By compact Hausdorff property of $C$, $V$
    must contain some open set $U$ such that $\bar{U}\subset V$ and
    $f(\mathcal{U})\in U$. Now since $F(\mathcal{U})=\lim_{\mathcal{U}}f$,
    $f^{-1}(U)\subset X$ must be contained in $\mathcal{U}$. Consider
    $\mathcal{O}=[f^{-1}(U)]$. Since $f^{-1}(U)\in\mathcal{U}$,
    $\mathcal{O}$ is an open neighborhood of $\mathcal{U}$. We show that
    $F(\mathcal{O})\subset\bar{U}\subset V$. Assume by contradiction that
    there is some $\mathcal{A}\in\mathcal{O}$ such that $F(\mathcal{A})$
    lies outside $\bar{U}$. Since $\mathcal{A}\in\mathcal{O}=[f^{-1}(U)]$,
    $\mathcal{A}$ must contain $f^{-1}(U)$. On the other hand,
    $F(\mathcal{A})=\lim_{\mathcal{A}}f$ lies in the open set
    $C\setminus\bar{U}$, which means that $\mathcal{A}$ also contains
    $f^{-1}(C\setminus\bar{U})$. Since $U$ and $C\setminus\bar{U}$ are
    disjoint, their pre-images under $f$ must also be disjoint, implying
    that $\mathcal{A}$ contains the empty set, a contradiction.

  \item We first show that $\mu$ is a finitely additive probability
    measure on $\mathbb{Z}$. We claim that for disjoint subsets
    $X,Y\subset\mathbb{Z}$, $\mu(X\sqcup Y)=\mu(X)+\mu(Y)$. Let $x$, $y$,
    $z$ denote $\mu(X)$, $\mu(Y)$, $\mu(X\sqcup Y)$ respectively. Let
    $U\subset[0,1]$ be and open neighborhood of $z$. Since ultrafilters are
    closed under upward inclusion, we can assume without loss of generality
    that $U=B_\epsilon(z)=(z-\epsilon,z+\epsilon)$, the open ball of radius
    $\epsilon$ around $z$, for some $\epsilon>0$. Now
    $U_x=\mu^{-1}(B_{\epsilon/2}(x))$ and $U_y=\mu^{-1}(B_{\epsilon/2}(y))$
    are both contained in $\mathcal{U}$. Also, for each $k\in U_x$,
    $f_X(k)$ lies within $B_{\epsilon/2}(x)$, and similarly, for each $k\in
    U_y$, $f_Y(k)$ lies within $B_{\epsilon/2}(y)$. Consider $U_z=U_x\cup
    U_y$. Then $U_z$ is contained in $\mathcal{U}$ since filters are closed
    under finite intersections. Hence, for each $k\in U_z$, $f_{X\sqcup
    Y}(k)=f_X(k)+f_Y(k)$ must lie within $B_\epsilon(z)$. Therefore
    $\mu^{-1}(B_\epsilon(z))$ is a superset of $U_z\in\mathcal{U}$, and must
    also be contained in $\mathcal{U}$ by closure of ultrafilters under
    upward inclusion. Since $U$ was arbitrary, we have shown that
    $\mu(Z)=\lim_\mathcal{U}f_{X\sqcup Y}=\mu(X)+\mu(Y)$, as required. This
    map is well-defined by Fact 4.9. \\

    We use the above claim to prove that $\mu$ is a finitely additive
    probability measure on $\mathbb{Z}$. By chasing definitions it is
    routine to verify that $\mu(\emptyset)=0$ and $\mu(\mathbb{Z})=1$.
    Since for disjoint $A,B\subset\mathbb{Z}$, $\mu(A\sqcup
    B)=\mu(A)+\mu(B)$, we also have for $Y\subset X\subset\mathbb{Z}$,
    $\mu(X-Y)=\mu(X)-\mu(Y)$. Hence, for arbitrary $X,Y\subset\mathbb{Z}$,
    since
    \begin{equation*}
      X\cup Y = (X-X\cup Y)\sqcup(X\cup Y)\sqcup(Y-X\cup Y),
    \end{equation*}
    we can apply the rules above to get
    \begin{align*}
      \mu(X\cup Y) &= \mu(X-X\cup Y)+\mu(X\cup Y)+\mu(Y-X\cup Y) \\
                   &= \mu(X)-\mu(X\cup Y)+\mu(X\cup Y)+\mu(Y)-\mu(X\cup Y) \\
                   &= \mu(X)-\mu(X\cup Y)+\mu(Y), \\
    \end{align*}
    which completes the proof that $\mu$ is a finitely additive probability
    measure. \\

    Next we show that $\mu$ is translation invariant. We need to show that
    for any $n\in\mathbb{Z}$ and any $X\subset\mathbb{Z}$, we have
    $\mu(X+n)=\mu(X)$. By induction on $n$, it suffices to show for the
    case $n=1$. Let $U\in[0,1]$ be an open neighborhood of $\mu(X+1)$.
    Since ultrafilters are closed under upward inclusion, we can assume
    without loss of generality that $O=B_{1/m}(\mu(X+1))$ for some
    $m\in\mathbb{N}^+$. Consider $O'=B_{1/2m}(\mu(X))$. Now
    $U'=\mu^{-1}(O')$ is contained in $\mathcal{U}$, and for each $k\in
    U'$, $f_X(k)$ lies within $O'$. Observe that for $k>2m$,
    $|f_X(k)-f_{X+1}(k)|<1/2m$. Hence for $k\in U'-\{1,\ldots,2m\}$,
    $f_{X+1}(k)$ will lie within $O$. Let
    $U=U'-\{1,\ldots,2m\}\subset\mathcal{N}$. Then $U\in\mathcal{U}$
    because $U'\in\mathcal{U}$ and
    $\mathcal{N}-\{1,\ldots,2m\}\in\mathcal{U}$ since $\mathcal{U}$
    contains the Frechet filter from question 3.
\end{enumerate}
\end{document}
