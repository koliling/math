\documentclass{article}
\usepackage[left=3cm,right=3cm,top=3cm,bottom=3cm]{geometry}
\usepackage{amsmath,amssymb,amsthm}
\usepackage{color}
%\setlength{\parindent}{0mm}

\newcommand{\TODO}[1]{\textcolor{red}{TODO: #1}}

\begin{document}
\title{Basic Logic I: Homework Set 4}
\author{Li Ling Ko\\ lko@nd.edu}
\date{\today}
\maketitle

\begin{enumerate}
  \item Let $\mathcal{L}=\{+,<,0\}$ be the original language, and let
    $\mathcal{L}_1=\{+,<,0,1,c\}$ be an extension of $\mathcal{L}$ by
    adding two new constants $1$ and $c$. Also, let
    $T=Th((\mathbb{Q},+,<,0,1))$ be the theory of $\mathbb{Q}$ in the
    language $\{+,<,0,1\}$. For each $n\in\mathbb{N}^+$, let $\varphi_n$ be
    the sentence 
    \begin{equation*}
      \varphi_n := 1 > \underbrace{c+\ldots+c}_{n\,\text{times}}.
    \end{equation*}
    Then $T\cup\{\varphi_n:n\in\mathbb{N}^+\}$ is finitely satisfiable by
    $\mathbb{Q}$: Given \TODO{Finish typing this proof.}

  \item In language $\mathcal{L}=\{<\}$, let
    $T=Th(\langle\mathbb{Z},<\rangle)$ be the complete theory of
    $\mathbb{Z}$. Consider the expanded language $\mathcal{L}(\mathbb{Q})$
    and the sentences $S=\{c_a<c_b|\; a,b\in\mathbb{Q}, a<b\}\cup\{c_a\neq
    c_b|\; a,b\in\mathbb{Q}, a<b\}$. We show that $T\cup S$ is finitely
    satisfiable in the new language $\mathcal{L}(\mathbb{Q})$: \TODO{Expand
    this routine proof.} Hence by compactness theorem, $T\cup S$ is
    satisfiable, and has a model $\mathcal{M}$. \\

    Next we show that in the original language $\mathcal{L}=\{<\}$,
    $\mathbb{Q}$ can be embedded into $\mathcal{M}$. Consider the map
    $i:\mathbb{Q}\rightarrow M,q\mapsto c_q$. This map is injective since
    in $\mathcal{L}(\mathbb{Q})$, $\mathcal{M}\models \{c_a\neq c_b|\;
    a,b\in\mathbb{Q}, a<b\}$. Also, this map is an embedding because
    $a<b\leftrightarrow c_a<c_b$. Hence $\mathbb{Q}$ can be embedded into
    $\mathcal{M}$ in the language $\mathcal{L}=\{<\}$. \\

    Next we show that in the original language $\mathcal{L}=\{<\}$,
    $\mathbb{Z}$ can be embedded into $\mathcal{M}$, using a proof similar
    to the previous paragraph: Consider the map $j:\mathbb{Z}\rightarrow
    M,a\mapsto c_a$. This map is injective since in
    $\mathcal{L}(\mathbb{Q})$, $\mathcal{M}\models \{c_a\neq c_b|\;
    a,b\in\mathbb{Z}, a<b\}$. Also, this map is an embedding because
    $a<b\leftrightarrow c_a<c_b$. Hence $\mathbb{Z}$ can be embedded into
    $\mathcal{M}$ in the language $\mathcal{L}=\{<\}$. \\

    So $\mathbb{Z}$ is a substructure of $\mathcal{M}$ in the language
    $\mathcal{L}=\{<\}$. It remains to show that in $\mathcal{L}=\{<\}$,
    $\mathbb{Z}$ is an elementary substructure of $\mathcal{M}$. We prove
    this by induction on formulas. We claim that given any formula
    $\varphi(x_1,\ldots,x_n)$ in the language $\mathcal{L}=\{<\}$ and any
    $\bar{a}\in\mathbb{Z}^n$,
    $\langle\mathbb{Z},<\rangle\models\varphi(\bar{a})$ if and only if
    $\mathcal{M}\models\varphi(j(\bar{a}))$. The most important step is the
    atomic formulas; induction by $\neg$, $\wedge$, and $\exists$ follows
    from chasing definitions: \\

    Atomic formulas: Since this language has no constants, atomic formulas
    can only be of the form $u<v$ or $u=v$ where $u$ and $v$ are variables.
    Then
    \begin{align*}
      \mathbb{Z}\models a<b &\leftrightarrow\mathcal{M}\models c_a<c_b. \\
    \end{align*}

    $\neg$: Let $\varphi(\bar{x})$ be a formula where the claim holds.
    Then
    \begin{align*}
      \mathbb{Z}\models\neg\varphi(\bar{a})
        &\leftrightarrow\mathbb{Z}\not\models\varphi(\bar{a}) & \\
        &\leftrightarrow\mathcal{M}\not\models\varphi(j(\bar{a}))  & (\text{by
          induction hypothesis}) \\
        &\leftrightarrow\mathcal{M}\models\neg\varphi(j(\bar{a})). & \\
    \end{align*}

    $\wedge$: Let $\varphi_1(\bar{x})$, $\varphi_2(\bar{y})$ be formulas
    where the claim holds. Then
    \begin{align*}
      \mathbb{Z}\models\varphi_1(\bar{a})\wedge\varphi_2(\bar{b})
        &\leftrightarrow\mathbb{Z}\models\varphi_1(\bar{a})\;\text{and}\;
        \mathbb{Z}\models\varphi_2(\bar{b})& \\
        &\leftrightarrow\mathcal{M}\models\varphi_1(j(\bar{a}))\;\text{and}\;
        \mathbb{Z}\models\varphi_2(j(\bar{b}))& (\text{by induction
        hypothesis}) \\
        &\leftrightarrow\mathcal{M}\models\varphi_1(j(\bar{a}))\wedge\varphi_2(j(\bar{b})). & \\
    \end{align*}

    $\exists$: Let $\varphi(\bar{x},y)$ be a formula where the claim holds.
    Then
    \begin{align*}
      \mathbb{Z}\models\exists y\, \varphi(\bar{a},y)
        &\leftrightarrow\mathbb{Z}\models\varphi(\bar{a},b)\; \text{for
        some}\; b\in\mathbb{Z} & \\
        &\leftrightarrow\mathcal{M}\models\varphi(j(\bar{a}),j(b))\;
        \text{for some}\; b\in\mathbb{Z}& (\text{by induction hypothesis})
        \\
        &\leftrightarrow\mathcal{M}\models\exists y\; \varphi(j(\bar{a}),y). & \\
    \end{align*}
\end{enumerate}
\end{document}
