\documentclass{article}
\usepackage[left=3cm,right=3cm,top=3cm,bottom=3cm]{geometry}
\usepackage{amsmath,amssymb,amsthm}
\usepackage{color}
%\setlength{\parindent}{0mm}
\newcommand{\TODO}[1]{\textcolor{red}{TODO: #1}}

\begin{document}
\title{Basic Logic I: Homework Set 5}
\author{Li Ling Ko\\ lko@nd.edu}
\date{\today}
\maketitle

\begin{enumerate}
  \item Prove that a class $\mathcal{K}$ is finitely axiomatizable if both
    $\mathcal{K}$ and its complement $\mathcal{K}^c$ are axiomatizable.
    \begin{proof}
      Let $T$ and $T^c$ be the set of sentences that axiomatize
      $\mathcal{K}$ and $\mathcal{K}^c$ respectively. Assume by
      contradiction that $\mathcal{K}$ is not finitely axiomatizable.
      Consider $S=T\cup T^c$. We show that $T$ is finitely satisfiable:
      Given any finite subset $A=A_0\cup A_1\subset S$, where $A_0\in T$ and
      $A_1\in T^c$, there exists models $\mathcal{M}\in\mathcal{K}^c$ that
      satisfy $A_0$ because $A_0$ does not axiomatize $\mathcal{K}$. Such
      models will also satisfy $A_1$ since they satisfy $T^c$, hence these
      models will satisfy $A$.

      Thus by compactness theorem $T$ is satisfiable and should have a
      model, which contradicts $\mathcal{K}\cap\mathcal{K}^c=\emptyset$.
    \end{proof}

  \item Show that the class of infinite fields is not finitely
    axiomatizable.

  \item Let $\mathcal{M}$, $\mathcal{N}$ be first order
  $\mathcal{L}$-structures with $\mathcal{M}\equiv\mathcal{N}$. Show that
  there is an $\mathcal{L}$-structure $\mathcal{K}$ such that both
  $\mathcal{M}$ and $\mathcal{N}$ can be elementarily embedded into
  $\mathcal{K}$. 

  \begin{proof}
    In the extended language $\mathcal{L}(M\sqcup N)$, where the constants
    for $\mathcal{M}$ are disjoint from the constants for $\mathcal{N}$,
    consider the theory $T=T_\mathcal{M}\cup T_\mathcal{N}$. We show that
    $T$ is finitely satisfiable by $\mathcal{M}$ after reassigning some
    constants of $\mathcal{N}$: Given a finite set of sentences
    $S=S_M\cup S_N\subset T$ where $S_M\subset T_\mathcal{M}$ and
    $S_N\subset T_\mathcal{N}$, we have $\mathcal{M}$ satisfies $S_M$
    because $S_M\subset T_\mathcal{M}$. Write the conjunction of all the
    sentences in $S_N$ as
    \begin{equation*}
      \varphi(c_{n_1},\ldots,c_{n_k})\in T_\mathcal{N},
    \end{equation*}
    where the $c_{n_i}$ are all the constants of $\mathcal{N}$ that appear in
    the conjunction. Then $\mathcal{N}$ satisfies
    \begin{equation*}
      \phi := \exists x_1,\ldots,x_k\; \varphi(x_1,\ldots,x_k),
    \end{equation*}

    with $c_{n_1},\ldots,c_{n_k}$ as witnesses. So from elementarily
    equivalence of $\mathcal{M}$ and $\mathcal{N}$, $\phi$ is also
    satisfied by $\mathcal{M}$. Assigning each $c_{n_i}$ to their respective
    element in $M$ that satisfies $\phi$ will give us a model of $S$. \\

    Therefore from compactness theorem, $T$ has a model $\mathcal{K}$. Then
    from part (iii) of Claim 3.2, since $\mathcal{K}$ satisfies
    $T_\mathcal{M}$ and $T_\mathcal{N}$, both $\mathcal{M}$ and
    $\mathcal{N}$ can be elementarily embedded into $\mathcal{K}$.
  \end{proof}
\end{enumerate}
\end{document}
