\documentclass{article}
\usepackage[left=3cm,right=3cm,top=3cm,bottom=3cm]{geometry}
\usepackage{amsmath,amssymb,amsthm}
\usepackage{color}
%\setlength{\parindent}{0mm}
\newcommand{\TODO}[1]{\textcolor{red}{TODO: #1}}

\begin{document}
\title{Basic Logic I: Homework Set 5}
\author{Li Ling Ko\\ lko@nd.edu}
\date{\today}
\maketitle

\begin{enumerate}
  \item Prove that a class $\mathcal{K}$ is finitely axiomatizable if both
    $\mathcal{K}$ and its complement $\mathcal{K}^c$ are axiomatizable.
    \begin{proof}
      Let $T$ and $T^c$ be the set of sentences that axiomatize
      $\mathcal{K}$ and $\mathcal{K}^c$ respectively. Assume by
      contradiction that $\mathcal{K}$ is not finitely axiomatizable.
      Consider $S=T\cup T^c$. We show that $T$ is finitely satisfiable:
      Given any finite subset $A=A_0\cup A_1\subset S$, where $A_0\in T$ and
      $A_1\in T^c$, there exists models $\mathcal{M}\in\mathcal{K}^c$ that
      satisfy $A_0$ because $A_0$ does not axiomatize $\mathcal{K}$. Such
      models will also satisfy $A_1$ since they satisfy $T^c$, hence these
      models will satisfy $A$.

      Thus by compactness theorem $T$ is satisfiable and should have a
      model, which contradicts $\mathcal{K}\cap\mathcal{K}^c=\emptyset$.
    \end{proof}
\end{enumerate}
\end{document}
