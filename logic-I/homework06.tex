\documentclass{article}
\usepackage[left=3cm,right=3cm,top=3cm,bottom=3cm]{geometry}
\usepackage{amsmath,amssymb,amsthm,pgfplots,tikz,mathtools}
\usepackage[inline]{enumitem}
\usetikzlibrary{patterns}
\usepackage{color}
\setlength{\parindent}{2mm}
\newcommand{\TODO}[1]{\textcolor{red}{TODO: #1}}

\begin{document}
\title{Basic Logic I: Homework 6}
\author{Li Ling Ko\\ lko@nd.edu}
\date{\today}
\maketitle

\begin{enumerate}[label={\bf Q\arabic*:}]
  \item Let $T$ be a Henkin theory in a language $\mathcal{L}$. Let $C$ be
    the set of all constant symbols of $\mathcal{L}$. We defined an
    equivalence relation $\sim$ on $C$ by $c_1\sim c_2$ if and only if
    $c_1=c_2\in T$. Let $M=C/\sim$ be the set of all $\sim$-classes, and
    for $c\in C$ we denote by $[c]$ its $\sim$-class. We defined
    $\mathcal{L}$-structure $\mathcal{M}$ with the universe $M$ in an
    obvious way. Show that for an atomic formula
    $\varphi(x_1,\ldots,x_n)$ and $c_1,\ldots,c_n\in C$ we have
    $\mathcal{M}\models\varphi([c_1],\ldots,[c_n])$ if and only if
    $\varphi(c_1,\ldots,c_n)\in T$.

    \begin{proof}
      We prove by induction on formulas. \\

      Terms: By definition of $\mathcal{M}$, we have $M\models
      [c]=f([c_1],\ldots,[c_n])$ implies $c=f(c_1,\ldots,c_n)\in T$.
      Conversely, if $c=f(c_1,\ldots,c_n)\in T$ and
      $c_i\sim a_i$ for $1\leq i\leq n$ and $a\sim c$, then
      $a=f(a_1,\ldots,a_n)\in T$ by Completeness and consistency of $T$,
      which then implies $\mathcal{M}\models [c]=f([c_1],\ldots,[c_n])$ by
      definition of $\mathcal{M}$. \\

      Relations: By definition of $\mathcal{M}$, we have $M\models
      R([c_1],\ldots,[c_n])$ implies $R(c_1,\ldots,c_n)\in T$. Conversely,
      if $R(c_1,\ldots,c_n)\in T$ and $c_i\sim a_i$
      for $1\leq i\leq n$, then $R(a_1,\ldots,a_n)\in T$ by Completeness
      and consistency of $T$, which then implies $\mathcal{M}\models
      R([c_1],\ldots,[c_n])$ by definition of $\mathcal{M}$.

      $\neg$: We have $\mathcal{M}\models\neg\varphi([c_1],\ldots,[c_n])$
      \begin{align*}
          &\leftrightarrow
            \mathcal{M}\not\models\varphi([c_1],\ldots,[c_n]) &\\
          &\leftrightarrow R(c_1,\ldots,c_n)\not\in T & (\text{by inductive
            hypothesis}) \\
          &\leftrightarrow \neg R(c_1,\ldots,c_n)\in T & (\text{by
            Completeness of}\; T). \\
      \end{align*}

      $\wedge$: We have $\mathcal{M}\models\varphi([c_1],\ldots,[c_n])
      \wedge \phi([d_1],\ldots,[d_m])$
      \begin{align*}
          &\leftrightarrow \mathcal{M}\models\varphi([c_1],\ldots,[c_n])\;
          \text{and}\; \mathcal{M}\models\phi([d_1],\ldots,[d_m]) & \\
          &\leftrightarrow R(c_1,\ldots,c_n)\in T\; \text{and}\;
            R(d_1,\ldots,d_m)\in T & (\text{by inductive hypothesis}) \\
          &\leftrightarrow R(c_1,\ldots,c_n)\wedge R(d_1,\ldots,d_m)\in T &
            (\text{by Completeness of}\; T). \\
      \end{align*}
    \end{proof}

  \item Let $\mathcal{L}=\{E\}$ , where $E$ is a binary relational symbol.
    Let $T$ be an $\mathcal{L}$-theory of an equivalence relation without
    finite classes and with infinitely many classes.

    \begin{enumerate}
      \item Write down the axioms for $T$.
        \begin{proof}
          First, we have the axioms of equivalence classes, given by the
          following three sentences:
          \begin{align*}
            \varphi_{\text{reflexive}}  &:= \forall x E(x,x) \\
            \varphi_{\text{symmetric}}  &:= \forall x,y E(x,y)\rightarrow
              E(y,x) \\
            \varphi_{\text{transitive}} &:= \forall x,y,z E(x,y)\wedge
              E(y,z)\rightarrow E(x,z). \\
          \end{align*}

          Then, we have a countable series of sentences
          $\{\varphi_{\text{class}\;\geq n}:n\in\mathbb{N}^+\}$, which says
          there are no finite classes:
          \begin{equation*}
            \varphi_{\text{class}\;\geq n} := \forall x\exists y_1,\ldots,y_n
            \bigwedge_{1\leq i<j\leq n} y_i\neq y_j \wedge
            \bigwedge_{1\leq i\leq n} E(x,y_i).
          \end{equation*}

          Finally, we have a countable series of sentences
          $\{\varphi_{\geq n\;\text{class}}:n\in\mathbb{N}^+\}$, which says
          there are infinitely many classes:
          \begin{equation*}
            \varphi_{\geq n\;\text{class}} := \exists x_1,\ldots,x_n
            \bigwedge_{1\leq i<j\leq n} x_i\neq x_j \wedge
            \bigwedge_{1\leq i<j\leq n} \neg E(x_i,x_j).
          \end{equation*}
        \end{proof}

      \item Is $T$ $\omega$-categorical?
        \begin{proof}
          Yes. Let $\mathcal{M}_1$ and $\mathcal{M}_2$ be two countable
          models of $T$. We construct an isomorphism $\varphi$ from $M_1$
          to $M_2$. \\

          Since there are infinitely many classes, these
          models must have countably infinite elements, and also countably
          infinite classes. Hence there is a bijection $f$ between the
          classes of $\mathcal{M}_1$ and the classes of $\mathcal{M}_2$.
          For each class $[c]$ of $\mathcal{M}_1$, the class size is
          must be countably infinite since there are at least infinite
          elements in the class and there cannot be more than $\omega$
          elements in the class. Similarly, $f([c])$, which is a class of
          $\mathcal{M}_2$, must also have countably infinite elements.
          Hence there must be a bijection $g_{[c]}:[c]\rightarrow f([c])$
          between elements in $[c]$ and elements in $f([c])$. \\

          Consider the function $\varphi:M_1\rightarrow M_2$ induced by the
          functions $f$ and $g_{[c]}$ for each equivalence class $[c]$ of
          $\mathcal{M}_1$. Chasing definitions, we get
          $\mathcal{M}_1\models E(a,b) \leftrightarrow \mathcal{M}_2\models
          E(\varphi(a),\varphi(b))$, which completes the proof that
          $\varphi$ is an isomorphism between $\mathcal{M}_1$ and
          $\mathcal{M}_2$.
        \end{proof}

      \item Is $T$ $\aleph_1$-categorical?
        \begin{proof}
          No. Consider model $\mathcal{M}_1:=\langle\mathbb{R},E_1\rangle$,
          where $E_1(a,b)\leftrightarrow\lfloor a\rfloor=\lfloor b\rfloor$.
          This model has a countable number of classes each with
          uncountable number of elements. Consider also model
          $\mathcal{M}_2:=\langle\mathbb{R},E_2\rangle$, where
          $E_2(a,b)\leftrightarrow|a-b|\in\mathbb{Z}$. This model has an
          uncountable number of classes each with countable number of
          elements. Assume by contradiction that there is an isomorphism
          $\varphi:M_1\rightarrow M_2$ between these models. Restricted the
          domain to $[0,1)$, $\varphi$ should send these uncountable number
          of elements injectively to a countable class in $\mathcal{M}_2$,
          a contradiction.
        \end{proof}

      \item Is $T$ complete?
        \begin{proof}
          Yes. $T$ has no finite models and is $\omega$-categorical, hence
          by Vaught's test it is complete.
        \end{proof}
    \end{enumerate}

  \item Let $\mathcal{L}$ be the language consisting of one binary
    relational symbol $<$ and countably many constant symbols $c_n$ for
    $n\in\mathbb{N}$. Let $T$ be the $\mathcal{L}$-theory whose axioms are:
    \begin{itemize}
      \item Axioms of linear dense orders without endpoints;
      \item For each $i<j\in\mathbb{N}$ the axiom $c_i<c_j$.
    \end{itemize}

    \begin{enumerate}
      \item Show that $T$ is complete.
        \begin{proof}
          Let $\mathcal{L}':=\{<\}$ denote the reduced language without the
          constant symbols, and let $T'$ denote the axioms of linear dense
          order in the reduced language. Note that from Corollary 4.38,
          theory $T'$ is complete in the reduced language. Also, notice
          that any model of $T$ is also a model of $T'$ in the reduced
          language. We apply these observations in our proof. \\

          Let $\mathcal{M}$ be a model of $T$. Let $\varphi(\bar{c})$ be a
          sentence in $\mathcal{L}$, where $\bar{c}$ are the only
          constants appearing in the sentence. Without loss of generality
          we can assume $\bar{c}=c_1,\ldots,c_n$ for some $n\in\mathbb{N}$.
          We first prove that if
          $\mathcal{M}\models\varphi(c_1,\ldots,c_n)$, then
          \begin{equation*}
            T\models\forall x_1,\ldots,x_n\;\; x_1<\ldots<
            x_n\rightarrow\varphi(x_1,\ldots,x_n).
          \end{equation*}
          Assume by contradiction that the claim is false. Then there must
          be a model $\mathcal{N}$ of $T$ such that there are elements
          $a_1<\ldots<a_n$ in $N$ but
          $\mathcal{N}\not\models\varphi(a_1,\ldots,a_n)$. Now since models
          of $T$ are also models of $T'$ in the reduced language
          $\mathcal{L}'$, by Remark 4.39, there is an isomporhism $\varphi$
          in the language of $\mathcal{L}'$ between $M$ and $N$ that sends
          $c_i$ to $a_i$ for $1\leq i\leq n$. Then by isomorphism property,
          $\mathcal{N}\not\models\varphi(a_1,\ldots,a_n)$ will imply
          $\mathcal{M}\not\models\varphi(c_1,\ldots,c_n)$, a contradiction.
          \\

          Now we prove that $T$ is complete. Assume by contradiction that
          it is not. Then there is a sentence $\varphi(c_1,\ldots,c_n)$ in
          language $\mathcal{L}$ and models $\mathcal{M}$ and $\mathcal{N}$
          of $T$ such that $\mathcal{M}\models\varphi(c_1,\ldots,c_n)$ but
          $\mathcal{N}\models\neg\varphi(c_1,\ldots,c_n)$. Note that the
          sentence $\varphi(c_1,\ldots,c_n)$ cannot be a sentence in the
          reduced language $\mathcal{L}'$ because $\mathcal{M}$ and
          $\mathcal{N}$ are also models of $T'$ and $T'$ is a complete
          theory. Now from the claim in the above paragraph,
          $\mathcal{M}\models\varphi(c_1,\ldots,c_n)$ implies
          $\mathcal{M}\models\phi$, where $\phi$ is defined as
          \begin{align*}
            \phi:= \forall x_1,\ldots,x_n\;\;
              x_1<\ldots<x_n\rightarrow\varphi(x_1,\ldots,x_n).
          \end{align*}
          Then $\phi$ is a sentence in the reduced language $\mathcal{L}'$,
          so by completeness of $T'$ and from the fact that $\mathcal{N}$
          is also a model of $T'$ in the reduced language, $\phi$ must also
          be true in $\mathcal{N}$. This would contradict
          $\mathcal{N}\models\neg\varphi(c_1,\ldots,c_n)$.
        \end{proof}

      \item Show that up-to isomorphism $T$ has exactly 3 countable models.
        \begin{proof}
          Using the notation used in previous part of this question, all
          models of $T$ are also models of $T'$ in the reduced language
          $\mathcal{L}'$. Then since $T'$ is $\omega$-categorical with
          respect to $\mathcal{L}'$ with countable model
          $\langle\mathbb{Q},<\rangle$, it suffices to work with
          $\langle\mathbb{Q},<\rangle$ when characterizing all countable
          models of the expanded language $\mathcal{L}$.
        \end{proof}
    \end{enumerate}
\end{enumerate}
\end{document}
