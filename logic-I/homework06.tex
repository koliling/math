\documentclass{article}
\usepackage[left=3cm,right=3cm,top=3cm,bottom=3cm]{geometry}
\usepackage{amsmath,amssymb,amsthm,pgfplots,tikz,mathtools}
\usepackage[inline]{enumitem}
\usetikzlibrary{patterns}
\usepackage{color}
\setlength{\parindent}{2mm}
\newcommand{\TODO}[1]{\textcolor{red}{TODO: #1}}

\begin{document}
\title{Basic Logic I: Homework 6}
\author{Li Ling Ko\\ lko@nd.edu}
\date{\today}
\maketitle

\begin{enumerate}[label={\bf Q\arabic*:}]
  \item Let $T$ be a Henkin theory in a language $\mathcal{L}$. Let $C$ be
    the set of all constant symbols of $\mathcal{L}$. We defined an
    equivalence relation $\sim$ on $C$ by $c_1\sim c_2$ if and only if
    $c_1=c_2\in T$. Let $M=C/\sim$ be the set of all $\sim$-classes, and
    for $c\in C$ we denote by $[c]$ its $\sim$-class. We defined
    $\mathcal{L}$-structure $\mathcal{M}$ with the universe $M$ in an
    obvious way. Show that for an atomic formula
    $\varphi(x_1,\ldots,x_n)$ and $c_1,\ldots,c_n\in C$ we have
    $\mathcal{M}\models\varphi([c_1],\ldots,[c_n])$ if and only if
    $\varphi(c_1,\ldots,c_n)\in T$.

    \begin{proof}
      We prove by induction on formulas. \\

      Terms: By definition of $\mathcal{M}$, we have $M\models
      [c]=f([c_1],\ldots,[c_n])$ implies $c=f(c_1,\ldots,c_n)\in T$.
      Conversely, if $c=f(c_1,\ldots,c_n)\in T$ and
      $c_i\sim a_i$ for $1\leq i\leq n$ and $a\sim c$, then
      $a=f(a_1,\ldots,a_n)\in T$ by Completeness and finite satisfiability
      of $T$, which then implies $\mathcal{M}\models
      [c]=f([c_1],\ldots,[c_n])$ by definition of $\mathcal{M}$. \\

      Relations: By definition of $\mathcal{M}$, we have $M\models
      R([c_1],\ldots,[c_n])$ implies $R(c_1,\ldots,c_n)\in T$. Conversely,
      if $R(c_1,\ldots,c_n)\in T$ and $c_i\sim a_i$
      for $1\leq i\leq n$, then $R(a_1,\ldots,a_n)\in T$ by Completeness
      and finite satisfiability of $T$, which then implies
      $\mathcal{M}\models R([c_1],\ldots,[c_n])$ by definition of
      $\mathcal{M}$.

      $\neg$: We have $\mathcal{M}\models\neg\varphi([c_1],\ldots,[c_n])$
      \begin{align*}
          &\leftrightarrow
            \mathcal{M}\not\models\varphi([c_1],\ldots,[c_n]) &\\
          &\leftrightarrow R(c_1,\ldots,c_n)\not\in T & (\text{by inductive
            hypothesis}) \\
          &\leftrightarrow \neg R(c_1,\ldots,c_n)\in T & (\text{by
            Completeness of}\; T). \\
      \end{align*}

      $\wedge$: We have $\mathcal{M}\models\varphi([c_1],\ldots,[c_n])
      \wedge \phi([d_1],\ldots,[d_m])$
      \begin{align*}
          &\leftrightarrow \mathcal{M}\models\varphi([c_1],\ldots,[c_n])\;
          \text{and}\; \mathcal{M}\models\phi([d_1],\ldots,[d_m]) & \\
          &\leftrightarrow R(c_1,\ldots,c_n)\in T\; \text{and}\;
            R(d_1,\ldots,d_m)\in T & (\text{by inductive hypothesis}) \\
          &\leftrightarrow R(c_1,\ldots,c_n)\wedge R(d_1,\ldots,d_m)\in T &
            (\text{by Completeness of}\; T). \\
      \end{align*}
    \end{proof}

  \item Let $\mathcal{L}=\{E\}$ , where $E$ is a binary relational symbol.
    Let $T$ be an $\mathcal{L}$-theory of an equivalence relation without
    finite classes and with infinitely many classes.

    \begin{enumerate}
      \item Write down the axioms for $T$.
        \begin{proof}
          First, we have the axioms of equivalence classes, given by the
          following three sentences:
          \begin{align*}
            \varphi_{\text{reflexive}}  &:= \forall x E(x,x) \\
            \varphi_{\text{symmetric}}  &:= \forall x,y E(x,y)\rightarrow
              E(y,x) \\
            \varphi_{\text{transitive}} &:= \forall x,y,z E(x,y)\wedge
              E(y,z)\rightarrow E(x,z). \\
          \end{align*}

          Then, we have a countable series of sentences
          $\{\varphi_{\text{class}\;\geq n}:n\in\mathbb{N}^+\}$, which says
          there are no finite classes:
          \begin{equation*}
            \varphi_{\text{class}\;\geq n} := \forall x\exists y_1,\ldots,y_n
            \bigwedge_{1\leq i<j\leq n} y_i\neq y_j \wedge
            \bigwedge_{1\leq i\leq n} E(x,y_i).
          \end{equation*}

          Finally, we have a countable series of sentences
          $\{\varphi_{\geq n\;\text{class}}:n\in\mathbb{N}^+\}$, which says
          there are infinitely many classes:
          \begin{equation*}
            \varphi_{\geq n\;\text{class}} := \exists x_1,\ldots,x_n
            \bigwedge_{1\leq i<j\leq n} x_i\neq x_j \wedge
            \bigwedge_{1\leq i<j\leq n} \neg E(x_i,x_j).
          \end{equation*}
        \end{proof}

      \item Is $T$ $\omega$-categorical?
        \begin{proof}
          Yes. Let $\mathcal{M}_1$ and $\mathcal{M}_2$ be two countable
          models of $T$. We construct an isomorphism $\varphi$ from $M_1$
          to $M_2$. \\

          Since there are infinitely many classes, these
          models must have countably infinite elements, and also countably
          infinite classes. Hence there is a bijection $f$ between the
          classes of $\mathcal{M}_1$ and the classes of $\mathcal{M}_2$.
          For each class $[c]$ of $\mathcal{M}_1$, the class size is
          must be countably infinite since there are at least infinite
          elements in the class and there cannot be more than $\omega$
          elements in the class. Similarly, $f([c])$, which is a class of
          $\mathcal{M}_2$, must also have countably infinite elements.
          Hence there must be a bijection $g_{[c]}:[c]\rightarrow f([c])$
          between elements in $[c]$ and elements in $f([c])$. \\

          Consider the function $\varphi:M_1\rightarrow M_2$ induced by the
          functions $f$ and $g_{[c]}$ for each equivalence class $[c]$ of
          $\mathcal{M}_1$. Chasing definitions, we get
          $\mathcal{M}_1\models E(a,b) \leftrightarrow \mathcal{M}_2\models
          E(\varphi(a),\varphi(b))$, which completes the proof that
          $\varphi$ is an isomorphism between $\mathcal{M}_1$ and
          $\mathcal{M}_2$.
        \end{proof}

      \item Is $T$ $\aleph_1$-categorical?
        \begin{proof}
          No. Consider model $\mathcal{M}_1:=\langle\mathbb{R},E_1\rangle$,
          where $E_1(a,b)\leftrightarrow\lfloor a\rfloor=\lfloor b\rfloor$.
          This model has a countable number of uncountable classes.
          Consider also model
          $\mathcal{M}_2:=\langle\mathbb{R},E_2\rangle$, where
          $E_2(a,b)\leftrightarrow|a-b|\in\mathbb{Z}$. This model has an
          uncountable number countable classes.  Assume by contradiction
          that there is an isomorphism $\varphi:M_1\rightarrow M_2$ between
          these models. Restricted the domain to $[0,1)$, $\varphi$ should
          send these uncountable number of elements injectively to a
          countable class in $\mathcal{M}_2$, a contradiction.
        \end{proof}

      \item Is $T$ complete?
        \begin{proof}
          Yes. $T$ has no finite models and is $\omega$-categorical, hence
          by Vaught's test it is complete.
        \end{proof}
    \end{enumerate}

  \item Let $\mathcal{L}$ be the language consisting of one binary
    relational symbol $<$ and countably many constant symbols $c_n$ for
    $n\in\mathbb{N}$. Let $T$ be the $\mathcal{L}$-theory whose axioms are:
    \begin{itemize}
      \item Axioms of linear dense orders without endpoints;
      \item For each $i<j\in\mathbb{N}$ the axiom $c_i<c_j$.
    \end{itemize}

    \begin{enumerate}
      \item Show that $T$ is complete.
        \begin{proof}
          Let $\mathcal{L}':=\{<\}$ denote the reduced language without the
          constant symbols, and let $T'$ denote the axioms of linear dense
          order in the reduced language. Note that from Corollary 4.38,
          theory $T'$ is complete in the reduced language. Also, notice
          that any model of $T$ is also a model of $T'$ in the reduced
          language. We apply these observations in our proof. \\

          Let $\mathcal{M}$ be a model of $T$. Let $\varphi(\bar{c})$ be a
          sentence in $\mathcal{L}$, where $\bar{c}$ are the only
          constants appearing in the sentence. Without loss of generality
          we can assume $\bar{c}=c_1,\ldots,c_n$ for some $n\in\mathbb{N}$.
          We first prove that if
          $\mathcal{M}\models\varphi(c_1,\ldots,c_n)$, then
          \begin{equation*}
            T\models\forall x_1,\ldots,x_n\;\; x_1<\ldots<
            x_n\rightarrow\varphi(x_1,\ldots,x_n).
          \end{equation*}
          Assume by contradiction that the claim is false. Then there must
          be a model $\mathcal{N}$ of $T$ such that there are elements
          $a_1<\ldots<a_n$ in $N$ but
          $\mathcal{N}\not\models\varphi(a_1,\ldots,a_n)$. Now since models
          of $T$ are also models of $T'$ in the reduced language
          $\mathcal{L}'$, by Remark 4.39, there is an isomorhism $\varphi$
          in the language of $\mathcal{L}'$ between $M$ and $N$ that sends
          $c_i$ to $a_i$ for $1\leq i\leq n$. Then by isomorphism property,
          $\mathcal{N}\not\models\varphi(a_1,\ldots,a_n)$ will imply
          $\mathcal{M}\not\models\varphi(c_1,\ldots,c_n)$, a contradiction.
          \\

          Now we prove that $T$ is complete. Assume by contradiction that
          it is not. Then there is a sentence $\varphi(c_1,\ldots,c_n)$ in
          language $\mathcal{L}$ and models $\mathcal{M}$ and $\mathcal{N}$
          of $T$ such that $\mathcal{M}\models\varphi(c_1,\ldots,c_n)$ but
          $\mathcal{N}\models\neg\varphi(c_1,\ldots,c_n)$. Note that the
          sentence $\varphi(c_1,\ldots,c_n)$ cannot be a sentence in the
          reduced language $\mathcal{L}'$ because $\mathcal{M}$ and
          $\mathcal{N}$ are also models of $T'$ and $T'$ is a complete
          theory. Now from the claim in the above paragraph,
          $\mathcal{M}\models\varphi(c_1,\ldots,c_n)$ implies
          $\mathcal{M}\models\phi$, where $\phi$ is defined as
          \begin{align*}
            \phi:= \forall x_1,\ldots,x_n\;\;
              x_1<\ldots<x_n\rightarrow\varphi(x_1,\ldots,x_n).
          \end{align*}
          Then $\phi$ is a sentence in the reduced language $\mathcal{L}'$,
          so by completeness of $T'$ and from the fact that $\mathcal{N}$
          is also a model of $T'$ in the reduced language, $\phi$ must also
          be true in $\mathcal{N}$. This would contradict
          $\mathcal{N}\models\neg\varphi(c_1,\ldots,c_n)$.
        \end{proof}

      \item Show that up-to isomorphism $T$ has exactly 3 countable models.
        \begin{proof}
          Consider these three countable models
          $\mathcal{M}_i=\langle\mathbb{Q},<,c_0,\ldots\rangle$ of $T$,
          defined by the following assignment of constants for
          $n\in\mathbb{N}$:
          \begin{align*}
            \mathcal{M}_1:\;  &c_n = n & \\
            \mathcal{M}_2:\;  &c_n = n/(n+1) & \\
            \mathcal{M}_3:\;  &c_n = \text{any rational between}\;
              n\sqrt{2}/(n+1)\; \text{and}\; (n+1)\sqrt{2}/(n+2), & \\
          \end{align*}

          We show that these are exactly the three distinct (up to
          isomorphism) countable models of $T$. By definition, these are
          models of $T$. Observe that the constants in $\mathcal{M}_1$
          diverge, the constants in $\mathcal{M}_2$ converge to
          $1\in\mathbb{Q}$, and the constants in $\mathcal{M}_3$ converge
          to $\sqrt{2}\not\in\mathbb{Q}$. \\

          $\mathcal{M}_1$ cannot be isomorphic to $\mathcal{M}_2$ or
          $\mathcal{M}_3$ because $\mathcal{M}_1$ is the only model for
          which no element in $\mathbb{Q}$ is larger than all the constants
          in the model. More specifically, assume by contradiction that
          there is an isomorphism $\varphi$ between $\mathcal{M}_i$ to
          $\mathcal{M}_1$, for $i=2$ or $3$. Then because isomorphisms
          preserve relations and $10>c_n$ for all constants $c_n$ in models
          $\mathcal{M}_i$, we must also have $\varphi(10)>c_n$ for all
          constants $c_n$ in model $\mathcal{M}_1$, which is false. \\

          Also, $\mathcal{M}_2$ is not isomorphic to $\mathcal{M}_3$
          because the constants in $\mathcal{M}_2$ converge to an element
          in $\mathbb{Q}$ but the constants in $\mathcal{M}_3$ do not.
          More specifically, assume by contradiction that there is an
          isomorphism $\varphi$ between $\mathcal{M}_2$ to $\mathcal{M}_3$.
          Then because isomorphisms preserve relations, $\varphi(1)$ must
          be larger than $\sqrt{2}$ in $\mathcal{M}_3$. Fix any rational
          $q$ between $\sqrt{2}$ and $\varphi(1)$. Now no rational in
          $M_2=\mathbb{Q}$ can be mapped to $q$ because rationals below $1$
          are mapped to rationals below $\sqrt{2}$. Hence $\varphi$ is not
          surjective, a contradiction. \\

          Finally, we prove that any model of $T$ is isomorphic to one of
          $\mathcal{M}_1$, $\mathcal{M}_2$, or $\mathcal{M}_3$. Given any
          model $\mathcal{M}$ of $T$, $\mathcal{M}$ will also be a model of
          the reduced language $\mathcal{L}'$ and satisfy $T'$. Hence by
          $\omega$-categoricity of $T'$ with respect to language
          $\mathcal{L}'$, $\mathcal{M}$ will be isomorphic to
          $\langle\mathbb{Q},<\rangle$, in the reduced language. Therefore
          we can assume $M=\mathbb{Q}$. Then the assignment of constants in
          $\mathcal{M}$ can only fall under exactly one of the following
          three cases: (1) $c_n$ diverges; (2) $c_n$ converges to an
          element in $\mathbb{Q}$; or (3) $c_n$ converges to an element in
          $\mathbb{R}\setminus\mathbb{Q}$. For each of the three cases, we
          construct an isomorphism between $\mathcal{M}$ and
          $\mathcal{M}_i$, where $i$ is the case that $\mathcal{M}$ belongs
          to. \\

          For all three cases, for each $n\in\mathbb{N}$, let $\varphi$ map
          the interval $[c_n,c_{n+1})$ in $\mathcal{M}$ linearly to
          $[c_n,c_{n+1})$ in $\mathcal{M}_i$. Also, let $\varphi$ map
          $(-\infty,c_0]$ in $\mathcal{M}$ to $(-\infty,c_0]$ in
          $\mathcal{M}_i$ by translation. Then for case (1), it is routine
          to check that $\varphi$ defines an isomorphism between
          $\mathcal{M}$ and $\mathcal{M}_1$. For case (2), we complete the
          map by sending $[\lim_{n\in\mathbb{N}}c_n,\infty)$ in
          $\mathcal{M}$ to $[\lim_{n\in\mathbb{N}}c_n,\infty)$ in
          $\mathcal{M}_2$ translation. Finally for case (3), we complete
          the map by sending $(\lim_{n\in\mathbb{N}}c_n,\infty)$ in
          $\mathcal{M}$ to $(\lim_{n\in\mathbb{N}}c_n,\infty)$ in
          $\mathcal{M}_2$ by translation. It is routine to show that
          $\varphi$ also defines an isomorphism between $\mathcal{M}$ and
          $\mathcal{M}_i$ for cases $i=2$ or $3$.
        \end{proof}
    \end{enumerate}

  \item We work in the language of groups. Show that for $n,m\geq1$,
    $\langle\mathbb{Z}^n,+,0\rangle\equiv\langle\mathbb{Z}^m,+,0\rangle$ if
    and only if $n=m$.

    \begin{proof}
      The converse of the claim is true from definition. Let
      $n>m\in\mathbb{N}^+$. We wish to find a sentence that is satisfied by
      $\mathbb{Z}^m$ but not by $\mathbb{Z}^n$. Given $k\in\mathbb{N}^+$,
      define an equivalence relation on $\mathbb{Z}^k$ by
      \begin{equation*}
        (a_1,\ldots,a_k)\sim(b_1,\ldots,b_k) \leftrightarrow a_i\equiv
        b_i\pmod{2}\; \text{for all}\; i\in\{1,\ldots,k\}.
      \end{equation*}

      It is routine to show that $\sim$ is an equivalence relation.
      Consider the size of $\mathbb{Z}^k/\sim$. We have
      $|\mathbb{Z}^k/\sim|=2^k$. Hence in $\mathbb{Z}^k$, we can find
      exactly $2^k$ elements from distinct classes, and no more than $2^k$
      such elements. Then given $n>m\in\mathbb{N}^+$, the sentence
      $\varphi_n$ which says ``I have $2^k$ elements from distinct
      classes'' will be satisfied by $\mathbb{Z}^n$ but not by
      $\mathbb{Z}^m$. We can formalize $\varphi_n$ in first order language
      as
      \begin{equation*}
        \varphi_n := \exists x_1,\ldots,x_{2^n}\; \bigwedge_{1\leq
        i<j\leq2^n} \varphi_{\text{diff class}}(x_i,x_j),
      \end{equation*}
      where $\varphi_{\text{diff class}}(x,y)$ says ``$x$ and $y$ are from
      different equivalence classes, formalized in first order language as
      \begin{equation*}
        \varphi_{\text{diff class}}(x,y) := \exists z\; y=x+z+z.
      \end{equation*}
      Then $\varphi_n$ is a sentence that witnesses
      $\langle\mathbb{Z}^n,+,0\rangle\not\equiv\langle\mathbb{Z}^m,+,0\rangle$.
    \end{proof}

  \item Two structures $\mathcal{M}$ and $\mathcal{N}$ are partially
    isomorphic if there is a non-empty set $\mathcal{F}$ of isomorphisms
    between substructures of $\mathcal{M}$ and $\mathcal{N}$ with the back
    and forth property:
    \begin{enumerate}
      \item For every $f\in\mathcal{F}$ and $a\in M$ there is an extension
        $g$ of $f$ with $a\in\text{dom}(g)$.
      \item For every $f\in\mathcal{F}$ and $b\in N$ there is an extension
        $g$ of $f$ with $b\in\text{range}(g)$.
    \end{enumerate}
    Show that partially isomorphic structures are elementarily equivalent.

    \begin{proof}
      First, we observe that given any finite $a_1,\ldots,a_n\in M$, there
      is an isomorphism $f\in\mathcal{F}$ between a substructure of
      $\mathcal{M}$ and a substructure of $\mathcal{N}$ such that
      $a_1,\ldots,a_n\in\text{dom}(f)$, because we can apply the forth
      direction in the back-and-forth property $n$ number of times.
      Similarly, given $b_1,\ldots,b_n\in M$, there is an isomorphism
      $f\in\mathcal{F}$ between a substructure of $\mathcal{M}$ and a
      substructure of $\mathcal{N}$ such that
      $b_1,\ldots,b_n\in\text{range}(f)$, when we apply the back direction in
      the back-and-forth property $n$ number of times. We will use these
      properties repeatedly in our proof. \\

      We prove elementary equivalence by induction on formulas. First, we
      show that for atomic formulas $\varphi(x_1,\ldots,x_n)$, we have
      $\mathcal{M}\models\varphi(a_1,\ldots,a_n)$ for some
      $a_1,\ldots,a_n\in M$ if and only if
      $\mathcal{N}\models\varphi(b_1,\ldots,b_n)$ for some
      $b_1,\ldots,b_n\in N$. Similarly,
      $\mathcal{M}\not\models\varphi(a_1,\ldots,a_n)$ for some
      $a_1,\ldots,a_n\in M$ if and only if
      $\mathcal{N}\not\models\varphi(b_1,\ldots,b_n)$ for some
      $b_1,\ldots,b_n\in N$. We prove this claim by induction on formulas.
      \\

      \underline{Terms}: Let $\pi(x_1,\ldots,x_n)$ be a function in the
      language. If $\mathcal{M}\models\pi(a_1,\ldots,a_n)=a$ for some
      $a,a_1,\ldots,a_n\in M$, let $f\in\mathcal{F}$ such that
      $\text{dom}(f)$ contains $a,a_1,\ldots,a_n$. Then
      \begin{align*}
                    && \mathcal{M}     &\models\pi(a_1,\ldots,a_n)=a  & \\
        \Rightarrow && \text{dom}(f)   &\models\pi(a_1,\ldots,a_n)=a
                    & (\because\text{substructures preserve functions}) \\
        \Rightarrow && \text{range}(f)
                    & \models\pi(f(a_1),\ldots,f(a_n))=f(a)
                    & (\because\text{isomorphisms preserve functions}) \\
        \Rightarrow && \mathcal{N}
                    & \models\pi(f(a_1),\ldots,f(a_n))=f(a)
                    & (\because\text{substructures preserve functions}). \\
      \end{align*}
      Similary, if $\mathcal{N}\models\pi(b_1,\ldots,b_n)=b$ for some
      $b,b_1,\ldots,b_n\in N$, let $f\in\mathcal{F}$ such that
      $\text{range}(f)$ contains $b,b_1,\ldots,b_n\in N$, and similar to
      the above argument we will get
      \begin{align*}
        \mathcal{M}\models\pi(f^{-1}(b_1),\ldots,f^{-1}(b_n))=f^{-1}(b).
      \end{align*}
      Also, if $\mathcal{M}\not\models\pi(a_1,\ldots,a_n)=a$, then
      \begin{align*}
                    && \mathcal{M}     &\not\models\pi(a_1,\ldots,a_n)=a & \\
        \Rightarrow && \mathcal{M}     &\models\pi(a_1,\ldots,a_n)\neq a & \\
        \Rightarrow && \text{dom}(f)   &\models\pi(a_1,\ldots,a_n)\neq a
                    & (\because\text{substructures preserve functions}) \\
        \Rightarrow && \text{range}(f)
                    & \models\pi(f(a_1),\ldots,f(a_n))\neq f(a)
                    & (\because\text{isomorphisms preserve functions}) \\
        \Rightarrow && \mathcal{N}
                    & \models\pi(f(a_1),\ldots,f(a_n))\neq f(a)
                    & (\because\text{substructures preserve functions}) \\
        \Rightarrow && \mathcal{N}
                    & \not\models\pi(f(a_1),\ldots,f(a_n))=f(a). & \\
      \end{align*}
      Similary, if $\mathcal{N}\not\models\pi(b_1,\ldots,b_n)=b$ for some
      $b,b_1,\ldots,b_n\in N$, let $f\in\mathcal{F}$ such that
      $\text{range}(f)$ contains $b,b_1,\ldots,b_n\in N$, and similar to
      the above argument we will get
      \begin{align*}
        \mathcal{M}\not\models\pi(f^{-1}(b_1),\ldots,f^{-1}(b_n))=f^{-1}(b).
      \end{align*}

      \underline{Relations}: If $\mathcal{M}\models R(a_1,\ldots,a_n)$ for
      some $a_1,\ldots,a_n\in M$, let $f\in\mathcal{F}$ such that
      $\text{dom}(f)$ contains $a_1,\ldots,a_n$. Then
      \begin{align*}
                    && \mathcal{M}     &\models R(a_1,\ldots,a_n)  & \\
        \Rightarrow && \text{dom}(f)   &\models R(a_1,\ldots,a_n)
                    & (\because\text{substructures preserve relations}) \\
        \Rightarrow && \text{range}(f)
                    & \models R(f(a_1),\ldots,f(a_n))
                    & (\because\text{isomorphisms preserve relations}) \\
        \Rightarrow && \mathcal{N}
                    & \models R(f(a_1),\ldots,f(a_n))
                    & (\because\text{substructures preserve relations}). \\
      \end{align*}
      Similary, if $\mathcal{N}\models R(b_1,\ldots,b_n)$ for some
      $b_1,\ldots,b_n\in N$, let $f\in\mathcal{F}$ such that
      $\text{range}(f)$ contains $b_1,\ldots,b_n\in N$, and similar to
      the above argument we will get
      \begin{align*}
        \mathcal{M}\models R(f^{-1}(b_1),\ldots,f^{-1}(b_n)).
      \end{align*}
      Also, if $\mathcal{M}\not\models R(a_1,\ldots,a_n)$ for
      some $a_1,\ldots,a_n\in M$, let $f\in\mathcal{F}$ such that
      $\text{dom}(f)$ contains $a_1,\ldots,a_n$. Then
      \begin{align*}
                    && \mathcal{M}     &\not\models R(a_1,\ldots,a_n)  & \\
        \Rightarrow && \mathcal{M}     &\models\neg R(a_1,\ldots,a_n)  & \\
        \Rightarrow && \text{dom}(f)   &\models\neg R(a_1,\ldots,a_n)
                    & (\because\text{substructures preserve relations}) \\
        \Rightarrow && \text{range}(f)
                    & \models\neg R(f(a_1),\ldots,f(a_n))
                    & (\because\text{isomorphisms preserve relations}) \\
        \Rightarrow && \mathcal{N}
                    & \models\neg R(f(a_1),\ldots,f(a_n))
                    & (\because\text{substructures preserve relations}) \\
        \Rightarrow && \mathcal{N}
                    & \not\models R(f(a_1),\ldots,f(a_n)). & \\
      \end{align*}
      Similary, if $\mathcal{N}\not\models R(b_1,\ldots,b_n)$ for some
      $b_1,\ldots,b_n\in N$, let $f\in\mathcal{F}$ such that
      $\text{range}(f)$ contains $b_1,\ldots,b_n\in N$, and similar to
      the above argument we will get
      \begin{align*}
        \mathcal{M}\not\models R(f^{-1}(b_1),\ldots,f^{-1}(b_n)).
      \end{align*}

      \underline{Negation}: We have
      \begin{align*}
                    && \mathcal{M}
                    & \models\neg\varphi(a_1,\ldots,a_n)
                    & \text{for some atomic}\; \varphi\; \text{and}\;
                      a_1,\ldots,a_n\in M \\
        \Leftrightarrow && \mathcal{M}
                    & \not\models\varphi(a_1,\ldots,a_n) & \\
        \Leftrightarrow && \mathcal{N}
                    & \not\models\varphi(f(a_1),\ldots,f(a_n))
                    & (\text{induction hypothesis}) \\
        \Leftrightarrow && \mathcal{N}
                    & \models\neg\varphi(f(a_1),\ldots,f(a_n)). & \\
      \end{align*}
      Similary, $\mathcal{N}\models\neg\varphi(b_1,\ldots,b_n)$ for some
      $b_1,\ldots,b_n\in N$ if and only if
      \begin{align*}
        \mathcal{M}\models\neg\varphi(f^{-1}(b_1),\ldots,f^{-1}(b_n)),
      \end{align*}
      where we choose $f\in\mathcal{F}$ to be the partial isomorphism whose
      range contains $b_1,\ldots,b_n\in N$.

      \underline{Conjunction}: We have
      \begin{align*}
                    && \mathcal{M}
                    & \models\varphi_1(\overline{a_1})\wedge\varphi_2(\overline{a_2})
                    & \text{for some atomic}\; \varphi_1,\varphi_2\;
                      \text{and}\;
                      \overline{a_1},\overline{a_2}\subset M \\
        \Leftrightarrow && \mathcal{M}
                    & \models\varphi_1(\overline{a_1})\;
                      \text{and}\;
                      \mathcal{M}\models\varphi_2(\overline{a_2}) & \\
        \Leftrightarrow && \mathcal{N}
                    & \models\varphi_1(f(\overline{a_1}))\;
                      \text{and}\;
                      \mathcal{N}\models\varphi_2(f(\overline{a_2}))
                    & (\text{induction hypothesis}) \\
        \Leftrightarrow && \mathcal{N}
                    &
                    \models\varphi_1(f(\overline{a_1}))\wedge\varphi_2(f(\overline{a_2})).
                    & \\
      \end{align*}
      Similary,
      $\mathcal{N}\models\varphi_1(\overline{b_1})\wedge\varphi_2(\overline{b_2})$
      for some
      $\overline{b_1},\overline{b_2}\subset N$ if and only if
      \begin{align*}
        \mathcal{M}\models\varphi_1(f^{-1}(\overline{b_1}))\wedge\varphi_2(f^{-1}(\overline{b_2})),
      \end{align*}
      where we choose $f\in\mathcal{F}$ to be the partial isomorphism whose
      range contains $\overline{b_1}$ and $\overline{b_2}$.

      Finally, we show that for any formula $\varphi(\overline{x})$, we
      have $\mathcal{M}\models\varphi(\overline{a})$ for some
      $\overline{a}\subset M$ if and only if
      $\mathcal{N}\models\varphi(\overline{b})$ for some
      $\overline{b}\subset N$. Similarly,
      $\mathcal{M}\not\models\varphi(\overline{a})$ for some
      $\overline{a}\subset M$ if and only if
      $\mathcal{N}\not\models\varphi(\overline{b})$ for some
      $\overline{b}\subset N$. We prove by induction on formulas. \\

      \underline{Atomic}: We have already proved this claim for atomic
      formulas. \\

      \underline{Exists}: We have
      \begin{align*}
                    && \mathcal{M}
                    & \models\exists\overline{x}\;
                      \varphi(\overline{x},\overline{a})
                    & \text{for some}\; \overline{a}\subset M \\
        \Leftrightarrow && \mathcal{M}
                    & \models\varphi(\overline{a_0},\overline{a})
                    & \text{for some}\; \overline{a_0}\subset M \\
        \Leftrightarrow && \mathcal{N}
                    & \models\varphi(\overline{b_0},\overline{b})
                    & \text{for some}\; \overline{b},\overline{b_0}\subset
                      N,\; \text{by induction hypothesis} \\
        \Leftrightarrow && \mathcal{N}
                    & \models\exists\overline{x}\;
                      \varphi(\overline{x},\overline{b}).
      \end{align*}

      \underline{Negation}: We have
      \begin{align*}
                    && \mathcal{M}
                    & \models\neg\varphi(\overline{a})
                    & \text{for some}\; \overline{a}\subset M \\
        \Leftrightarrow && \mathcal{M}
                    & \not\models\varphi(\overline{a}) & \\
        \Leftrightarrow && \mathcal{N}
                    & \not\models\varphi(\overline{b})
                    & \text{for some}\; \overline{b}\subset N,\; \text{by
                      induction hypothesis} \\
        \Leftrightarrow && \mathcal{N}
                    & \models\neg\varphi(\overline{b}). & \\
      \end{align*}

      \underline{Conjunction}: We have
      \begin{align*}
                    && \mathcal{M}
                    & \models\neg\varphi(\overline{a})
                    & \text{for some}\; \overline{a}\subset M \\
        \Leftrightarrow && \mathcal{M}
                    & \not\models\varphi(\overline{a}) & \\
        \Leftrightarrow && \mathcal{N}
                    & \not\models\varphi(\overline{b})
                    & \text{for some}\; \overline{b}\subset N,\; \text{by
                      induction hypothesis} \\
        \Leftrightarrow && \mathcal{N}
                    & \models\neg\varphi(\overline{b}). & \\
      \end{align*}
    \end{proof}
\end{enumerate}
\end{document}
