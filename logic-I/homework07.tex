\documentclass{article}
\usepackage[left=3cm,right=3cm,top=3cm,bottom=3cm]{geometry}
\usepackage{amsmath,amssymb,amsthm,pgfplots,tikz,mathtools}
\usepackage[inline]{enumitem}
\usetikzlibrary{patterns}
\usepackage{color}
\setlength{\parindent}{2mm}
\newcommand{\TODO}[1]{\textcolor{red}{TODO: #1}}

\begin{document}
\title{Basic Logic I: Homework 7}
\author{Li Ling Ko\\ lko@nd.edu}
\date{\today}
\maketitle

\begin{enumerate}[label={\bf Q\arabic*:}]
  \item Let $F$ be a field. The language $\mathcal{L}_F$ for vector spaces
    over $F$ consists of one binary function symbol $+$, one constant
    symbol $0$, and a unary function symbol $\lambda_a$ for every $a\in F$.
    Let $T$ be the theory of infinite vector spaces over $F$ in the
    language $\mathcal{L}_F$. Show that $T$ has quantifier elimination.

    \begin{proof}
    \end{proof}

  \item Let $\mathcal{R}=\langle\mathbb{R},+,-,\cdot,<,0,1\rangle$ be the
    ordered field of real numbers. Consider the type
    \begin{equation*}
      p(x) =
      \{x>\underbrace{1+\cdots+1}_{n-\text{times}}:n\in\mathbb{N}^{>0}\}.
    \end{equation*}
    Show that every proper elementary extension of $\mathcal{R}$ realizes
    $p$.

    \begin{proof}
      First, we note that if an elementary extension does not realize $p$,
      then the extension also cannot contain an element that is smaller
      than all elements in $\mathbb{R}$: If such an element $s$ exists,
      then $-s$ will be greater than all elements in $\mathbb{R}$, because
      $\mathcal{R}$ satisfies the sentence
      \begin{equation*}
        \forall x\; (x>0\leftrightarrow -x<0).
      \end{equation*}

      Let $\mathcal{S}=\langle\mathbb{S},+,-,\cdot,<,0,1\rangle$ be a
      proper elementary extension of $\mathcal{R}$. We need to show that
      $\mathbb{S}$ contains an element that is greater than all elements in
      $\mathbb{R}$. \\

      %Now $\mathcal{S}$ must satisfy the theory of
      %$\mathcal{R}$, which includes the axioms of fields. Hence
      %$\mathbb{S}$ is a field extension of $\mathbb{R}$. \\

      %There are two possible cases. In the first case, the degree of
      %the field extension $d=[\mathbb{S}:\mathbb{R}]$ is finite. We first show
      %that $\mathbb{S}$ must contain the field of complex numbers
      %$\mathbb{C}$: Pick any element $s\in\mathbb{S}\setminus\mathbb{R}$.
      %Then $s$ is a solution to an irreducible polynomial in
      %$\mathbb{R}[x]$, which implies that $s\in\mathbb{C}$, by the
      %algebraic closure of $\mathbb{C}$. Then we have
      %$\mathbb{R}\subset\mathbb{R}[s]\subseteq\mathbb{C}$. But
      %$[\mathbb{C}:\mathbb{R}]=2$ because $\mathbb{C}=\mathbb{R}[i]$, we
      %have $\mathbb{R}[s]=\mathbb{C}$, which implies that
      %$\mathbb{C}\subseteq\mathbb{S}$. \\

      %Consider the ordering of $\mathbb{S}$ restricted to $\mathbb{C}$.
      %Assume by contradiction that $\mathbb{S}$ does not realize $p$. If
      %$i\geq0$, then multiplying the equation by $i$, we get $-1=i\cdot
      %i<0$, which contradicts the sentence
      %\begin{equation*}
      %  \forall x,y\; (x>0\wedge y>0\rightarrow x\cdot y>0),
      %\end{equation*}
      %which is satisfied by $\mathcal{R}$ and hence by $\mathcal{S}$.
      %On the other hand, if $i<0$, then $-i>0$, so similarly multiplying
      %the equation by $-i$, we get $-1=-i\cdot(-i)<0$, still a
      %contradiction. \\

      %Hence the degree of the field extension $d=[\mathbb{S}:\mathbb{R}]$
      %must be infinite. In this case, $\mathbb{S}$ must contain the field
      %of fractions $\mathbb{R}(y)$. \\

      Fix any $y\in\mathbb{S}\setminus\mathbb{R}$. Assume by contradiction
      that $\mathbb{S}$ does not realize $p$, and consider the ordering of
      $\mathbb{S}$ restricted to $\mathbb{R}\cup\{y\}$. Partition
      $\mathbb{R}$ into two disjoint sets $R_1=\{r<y:r\in\mathbb{R}\}$ and
      $R_2=\{r>y:r\in\mathbb{R}\}$. Note that the two sets are disjoint,
      their union is $\mathbb{R}$, and elements in $R_1$ are smaller than
      all elements in $R_2$, because $\mathbb{R}$ and thus $\mathbb{S}$
      satisfy the axioms of linear ordering. Then from assumption,
      $\sup{R_1}$ and $\inf{R_2}$ are contained in $\mathbb{R}$, and from
      the axioms of linear ordering these values must be equal. Hence we
      have $a=\sup{R_1}=\inf{R_2}\in\mathbb{R}$. Then since $R_1\cup
      R_2=\mathbb{R}$, $a$ must belong to either $R_1$ or $R_2$. Assume
      without loss of generality that $a\in R_1$. \\

      Consider the element $y-a\in\mathbb{S}$ and its multiplicative
      inverse $(y-a)^{-1}\in\mathbb{S}$. Since $a\in R_1$, we have $y-a>0$.
      Then the multiplicative inverse $(y-a)^{-1}$ must also be greater
      than 0, because $\mathcal{R}$ and thus $\mathcal{S}$ satisfies the
      sentence which says that the multiplicative inverse of a positive
      element is also positive. Then from assumption, $0\leq(y-a)^{-1}\leq
      N$ for some $N\in\mathbb{N}^+$. At the same time, since $a\not\in
      R_2$, we have $0<y-a<1/N$. Multiplying these two inequalities give us
      $1=(y-a)^{-1}\cdot(y-a)<N\cdot 1/N=1$, a contradiction. Note that we
      can multiply these inequalities because $\mathcal{R}$ and thus
      $\mathcal{S}$ satisfies the sentence that says if $0\leq a_1\leq a_2$
      and $0<b_1<b_2$ then $a_1\cdot b_1\leq a_2\cdot b_2$.
    \end{proof}

  \item
    \begin{enumerate}
      \item Two functions $f,g:\mathbb{N}\rightarrow\mathbb{N}$ are called
        almost disjoint if $f(n)\neq g(n)$ for almost all of
        $n\in\mathbb{N}$. Show that there are $2^\omega$-many almost
        disjoint functions from $\mathbb{N}$ to $\mathbb{N}$.
    \end{enumerate}
\end{enumerate}
\end{document}
