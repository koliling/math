\documentclass{article}
\usepackage[left=3cm,right=3cm,top=3cm,bottom=3cm]{geometry}
\usepackage{amsmath,amssymb,amsthm,pgfplots,tikz,mathtools}
\usepackage[inline]{enumitem}
\usetikzlibrary{patterns}
\usepackage{color}
\setlength{\parindent}{2mm}
\newcommand{\TODO}[1]{\textcolor{red}{TODO: #1}}

\begin{document}
\title{Basic Logic I: Homework 8}
\author{Li Ling Ko\\ lko@nd.edu}
\date{\today}
\maketitle

\begin{enumerate}[label={\bf Q\arabic*:}]
  \item Let $B\subseteq A\subseteq M$. For $p(\overline{x})\in
    S^{\mathcal{M}}_n(A)$ let \[p\restriction B=\{\varphi(x)\in
    p(\overline{x}):\varphi(\overline{x})\; \text{is a formula over}\;
    B\}.\] Show that the map $p\mapsto p\restriction B$ is a continuous
    map from $S^{\mathcal{M}}_n(A)$ to $S^{\mathcal{M}}_n(B)$.

    \begin{proof}
      First note that the given restriction map is a well-defined because
      any subset of a consistent set of formulas remains consistent, and also
      after restriction the set of sentences remain Complete. Let
      $r:S^{\mathcal{M}}_n(A)\rightarrow S^{\mathcal{M}}_n(B)$ denote the
      restriction map function, let $\phi(\bar{x})$ be a formula over $B$,
      and let $\mathcal{O}_{\phi,B}=\{q\in S^{\mathcal{M}}_n(B):\phi\in
      q\}$. We need to show that $r^{-1}(\mathcal{O}_{\phi,B})$ is open in
      the topology over $S^{\mathcal{M}}_n(A)$. This is true since
      $r^{-1}(\mathcal{O}_{\phi,B})=\mathcal{O}_{\phi,A}$: $\subseteq$
      follows because every $p\in r^{-1}(\mathcal{O}_{\phi,B})$ will
      contain $\phi$; $\supseteq$ follows because every restriction of a
      type in $S^{\mathcal{M}}_n(A)$ that contains $\phi$ is a type in
      $S^{\mathcal{M}}_n(B)$ that contains $\phi$.
    \end{proof}

  \item Let $\mathcal{M}$ be a structure and $B\subseteq M$. For
    $p(x_1,\ldots,x_{m+n})\in S^{\mathcal{M}}_{m+n}(B)$ let $p\restriction
    m=\{\varphi(x_1,\ldots,x_m)\in p(x_1,\ldots,x_{m+n})\}$.

    \begin{enumerate}
      \item Show that $p\restriction m\in S^{\mathcal{M}}_m(B)$ for $p\in
        S^{\mathcal{M}}_{m+n}(B)$.
        \begin{proof}
          This is true from chasing definitions: After restriction the set
          of sentences remain consistent, and hence is still a
          type. If $p\restriction m$ is not Complete over $B$ in
          free variables $x_1,\ldots,x_m$, then there must exist a formula
          $\varphi(x_1,\ldots,x_m)$ over $B$ such that neither itself nor
          its negation is in $p\restriction m$; then since formulas with
          free variables $x_1,\ldots,x_m$ are also considered formulas with
          free variables $x_1,\ldots,x_{m+n}$, then neither
          $\varphi(x_1,\ldots,x_m)$ nor its negation is in $p$, which
          contradicts the Completeness of $p$.
        \end{proof}

      \item Show that the restriction map $p\mapsto p\restriction m$ is
        open, continuous, and surjective as a map from
        $S^{\mathcal{M}}_{m+n}(B)$ to $S^{\mathcal{M}}_{m}(B)$.

        \begin{proof}
          The restriction map is well-defined from the previous part of
          this question. Let $r:S^{\mathcal{M}}_{m+n}(B)\rightarrow
          S^{\mathcal{M}}_{m}(B)$ denote the restriction map. Given a
          formula $\phi$, use $\mathcal{O}_{\phi,n}$ to denote $\{p\in
          S^{\mathcal{M}}_n(B):\phi\in p\}$. \\

          To show that the map is continuous, let $\phi(x_1,\ldots,x_{m})$
          be a formula over $B$. We need to show that
          $r^{-1}(\mathcal{O}_{\phi,m})$ is open in
          $S^{\mathcal{M}}_{m+n}(B)$. Now
          $r^{-1}(\mathcal{O}_{\phi,m})=\mathcal{O}_{\phi,m+n}$:
          $\subseteq$ follows because every $p\in
          r^{-1}(\mathcal{O}_{\phi,m})$ will contain $\phi$; $\supseteq$
          follows because every restriction of a type in
          $S^{\mathcal{M}}_{m+n}(B)$ that contains $\phi$ is a type in
          $S^{\mathcal{M}}_m(A)$ that contains $\phi$. Since
          $\mathcal{O}_{\phi_B,m+n}$ is open in $S^{\mathcal{M}}_{m+n}(B)$,
          the map is continuous. \\

          To show that the map is surjective, let $q$ a type in
          $S^{\mathcal{M}}_{m}(B)$. We need to show that there exists a
          type $p$ in $S^{\mathcal{M}}_{m+n}(B)$ whose restriction is $q$.
          Since $q$ is a type, by Corollary 6.9, there is an elementary
          extension $\mathcal{N}\succeq\mathcal{M}$ and
          $\overline{a'}=(a_1,\ldots,a_m)\subseteq N$ such that
          $q=\text{tp}^\mathcal{N}(\overline{a'}/B)$. Fix an additional
          $a_{m+1},\ldots,a_{m+n}\subseteq N$, let
          $\overline{a}=(a_1,\ldots,a_{m+n})\subseteq N$, and let
          $p=\text{tp}^\mathcal{N}(\overline{a}/B)$. Then from Corollary
          6.9 again, $p\in S^{\mathcal{M}}_{m+n}(B)$. Moreover, the
          restriction of $p$ is $q$. \\

          Finally, to show that the map is open, let
          $\phi'(x_1,\ldots,x_{m+n})$ be a formula over $B$. We need to
          show that $\mathcal{O}_{\phi',m+n}\restriction m$ is open in
          $S^{\mathcal{M}}_{m}(B)$. Let $\phi(x_1,\ldots,x_m)$ be the
          formula over $B$ in $m$ variables, defined as
          \[\phi(x_1,\ldots,x_m) := \exists x_{m+1},\ldots,x_{m+n}\;
          \phi'(x_1,\ldots,x_{m+n}).\] We show that
          $\mathcal{O}_{\phi',m+n}\restriction m=\mathcal{O}_{\phi,m}$,
          which is open in $S^{\mathcal{M}}_{m}(B)$. $\subseteq$ follows
          because every $p$ in $\mathcal{O}_{\phi',m+n}$ contains $\phi'$
          and thus also contains $\phi$ from consistency of $p$. To show
          $\supseteq$, let $q\in\mathcal{O}_{\phi,m}$. We need to find some
          $p\in\mathcal{O}_{\phi',m+n}$ such that $q=p\restriction m$. By
          Corollary 6.9, there is an elementary extension
          $\mathcal{N}\succeq\mathcal{M}$ and
          $\overline{a'}=(a_1,\ldots,a_m)\subseteq N$ such that
          $q=\text{tp}^\mathcal{N}(\overline{a'}/B)$. Then
          $\mathcal{N}\models\exists x_{m+1},\ldots,x_{m+n}\;
          \phi'(a_1,\ldots,a_m,x_{m+1},\ldots,x_{m+n})$, so there exists
          $a_{m+1},\ldots,a_{m+n}\subseteq N$ such that
          $\mathcal{N}\models\phi'(a_1,\ldots,a_{m+n})$. Let
          $\overline{a}=(a_1,\ldots,a_{m+n})\subseteq N$, and let
          $p=\text{tp}^\mathcal{N}(\overline{a}/B)$. Then from Corollary
          6.9 again, $p\in S^{\mathcal{M}}_{m+n}(B)$ and $\phi'\in p$, so
          $p\in\mathcal{O}_{\phi',m+n}$. Moreover, the
          restriction of $p$ is $q$. \\
        \end{proof}
    \end{enumerate}

  \item Show that a subset $X\subseteq S^{\mathcal{M}}_n(A)$ is closed and
    open if and only if $X=\mathcal{O}_\varphi$  for some formula $\varphi$
    over $A$.

    \begin{proof}
      Let $X=\mathcal{O}_\varphi$. Then $X$ is open by definition of the
      basis of the topological space. Also, $S^{\mathcal{M}}_n(A)\setminus
      X=\mathcal{O}_{\neg\varphi}$, because every $p\in
      S^{\mathcal{M}}_n(A)$ must contain exactly one of $\varphi$ or
      $\neg\varphi$ by Completeness and consistency. Hence $X$ is also
      closed. \\

      Let $X\subseteq S^{\mathcal{M}}_n(A)$ be both open and closed. Since
      $X$ is open, $X=\cup_{i\in I}\mathcal{O}_{\varphi_i}$ for some set of
      formulas $\{\varphi_i\}_{i\in I}$. Similarly, since
      $S^{\mathcal{M}}_n(A)\setminus X$ is open, it is equals to
      $\cup_{j\in J}\mathcal{O}_{\phi_j}$ for some set of formulas
      $\{\phi_j\}_{j\in J}$. Then $\cup_{i\in
      I}\mathcal{O}_{\varphi_i}\cup\cup_{j\in J}\mathcal{O}_{\phi_j}$ is an
      open cover of $S^{\mathcal{M}}_n(A)$, so by compactness of
      $S^{\mathcal{M}}_n(A)$ as stated in Claim 6.12,
      $S^{\mathcal{M}}_n(A)$ is covered by some finite
      $\mathcal{O}_{\varphi_1},\ldots,\mathcal{O}_{\varphi_m}$ and
      $\mathcal{O}_{\phi_1},\ldots,\mathcal{O}_{\phi_n}$. Then
      $X=\cup_{i=1}^{m}\mathcal{O}_{\varphi_i}$, so it suffices to show
      that $\cup_{i=1}^{m}\mathcal{O}_{\varphi_i}=\mathcal{O}_{\varphi}$,
      where $\varphi$ is the disjunction of $\varphi_1,\ldots,\varphi_m$.
      $\subseteq$ holds because any type $p\in S^{\mathcal{M}}_n(A)$ that
      contains $\varphi_i$ must also contain $\varphi$ from Completeness
      and consistency of $p$. Similarly, $\supseteq$ holds because any type
      $p\in S^{\mathcal{M}}_n(A)$ that contains $\varphi$ must also
      contain $\varphi_i$ for some $i\in\{1,\ldots,m\}$ from Completeness
      and consistency of $p$. 
    \end{proof}

  \item For a type $p(v_1,\ldots,v_n)$ in $\mathcal{M}$ over $A$ (not
    necessarily complete) we denote by $[p]$ the subset of
    $S^{\mathcal{M}}_n(A)$ defined as \[[p]=\bigcap_{\varphi\in
    p}\mathcal{O}_{\varphi}.\] Show that for types $p$, $q$ in
    $\mathcal{M}$ over $A$ we have $p\sim q$ if and only if $[p]=[q]$.

    \begin{proof}
      First, note that $[p]$ is non-empty: By Proposition 6.3, there is an
      elementary extension $\mathcal{N}$ of $\mathcal{M}$ that realizes
      $p$; we can Complete $p$ by including all formulas over $A$ in
      $n$-variables that are true in the $\mathcal{N}$; the Completion will
      be in $[p]$. Similarly $[q]$ is non-empty by consistency of $q$. \\

      $\Rightarrow$: Assume $p\sim q$. By symmetry it suffices to show
      $[p]\subseteq[q]$. Let $r\in[p]$. Then $r$ is a Complete $n$-type
      that contains all formulas in $p$. Then given any $\phi\in q$, since
      $p\vdash\phi$, $r$ must also contain $\phi$ by consistency and
      Completeness of $r$. Hence $r$ contains all formulas in $q$, which
      implies $r\in[q]$. \\

      $\Leftarrow$: Assume $[p]=[q]$. By symmetry it suffices to show that
      $p\vdash q$. Let $\phi$ be a formula in $q$. We need to show that
      $p\vdash\phi$. Assume by contradiction that $p\not\vdash\phi$. Then
      $p\cup\{\neg\phi\}$ is consistent. Let $\mathcal{N}$ be an elementary
      extension of $\mathcal{M}$ that realizes $p\not\vdash\phi$; such
      $\mathcal{N}$ exists from Proposition 6.3. Let $p'\supseteq
      p\cup\{\neg\phi\}$ be the Completion of $p\cup\{\neg\phi\}$ by
      including all formulas over $A$ in $n$-variables that are true in
      $\mathcal{N}$. Then $p'\in[q]$ since $p'\in[p]$ and $[p]=[q]$, yet
      $p'$ contains both $\phi$ and $\neg\phi$, contradicting the
      consistency of $p'$.
    \end{proof}

  \item Let $\mathcal{M}$ be a first order structure and $a,b\in M$.
    \begin{enumerate}
      \item Assume $\varphi(x,y)$ isolates the type
        $\text{tp}^{\mathcal{M}}(a,b/\emptyset)$. Show that $\exists
        x\; \varphi(x,y)$ isolates the type
        $\text{tp}^{\mathcal{M}}(b/\emptyset)$ and $\varphi(x,b)$ isolates
        $\text{tp}^{\mathcal{M}}(a/b)$.

        \begin{proof}
          We first prove the first assertion. Assume by contradiction that
          $\exists x\varphi(x,y)$ does not isolate
          $\text{tp}^{\mathcal{M}}(b/\emptyset)$. Then there must exist a
          formula $\theta(y)\in\text{tp}^{\mathcal{M}}(b/\emptyset)$ such
          that $\mathcal{M}\not\models\exists x
          \varphi(x,y)\rightarrow\theta(y)$. This means there exists $b'\in
          M$ such that $\mathcal{M}\models\exists x\varphi(x,b')$ and
          $\mathcal{M}\models\neg\theta(b')$. The first formula implies
          that there exists $a'\in M$ such that
          $\mathcal{M}\models\varphi(a',b')$. So we have
          $\mathcal{M}\models\varphi(a',b')\wedge\neg\theta(b')$, which
          means $\mathcal{M}\not\models\varphi(x,y)\rightarrow\theta(y)$.
          However, since
          $\theta(y)\in\text{tp}^{\mathcal{M}}(b/\emptyset)\subseteq
          \text{tp}^{\mathcal{M}}(a,b/\emptyset)$, the statement
          contradicts $\varphi(x,y)$ isolating
          $\text{tp}^{\mathcal{M}}(a,b/\emptyset)$. \\

          The second assertion can be proven in a similar manner. Assume by
          contradiction that $\varphi(x,b)$ does not isolate
          $\text{tp}^{\mathcal{M}}(a/b)$. Then there must exist a formula
          $\theta(x,b)\in\text{tp}^{\mathcal{M}}(a/b)$ such that
          $\mathcal{M}\not\models\varphi(x,b)\rightarrow\theta(x,b)$. This
          means there exists $a'\in M$ such that
          $\mathcal{M}\models\varphi(a',b)\wedge\neg\theta(a',b)$, which
          means $\mathcal{M}\not\models\varphi(x,y)\rightarrow\theta(x,y)$.
          However, since
          $\theta(x,y)\in\text{tp}^{\mathcal{M}}(a,b/\emptyset)$, the
          statement contradicts $\varphi(x,y)$ isolating
          $\text{tp}^{\mathcal{M}}(a,b/\emptyset)$. \\
        \end{proof}

      \item Prove Claim 7.1: The type
        $\text{tp}^{\mathcal{M}}(\overline{a},\overline{b}/\emptyset)$ is
        isolated if and only if both types
        $\text{tp}^{\mathcal{M}}(\overline{a}/\emptyset)$ and
        $\text{tp}^{\mathcal{M}}(\overline{b}/\overline{a})$ are isolated.

        \begin{proof}
          If $\varphi(\overline{x},\overline{y})$ isolates type
          $\text{tp}^{\mathcal{M}}(\overline{a},\overline{b}/\emptyset)$,
          then from the previous part of this question, $\exists y\;
          \varphi(\overline{x},\overline{y})$ isolates the type
          $\text{tp}^{\mathcal{M}}(\overline{a}/\emptyset)$. Also from the
          previous part of this question,
          $\varphi(\overline{x},\overline{b})$ isolates
          $\text{tp}^{\mathcal{M}}(\overline{a}/\overline{b})$. \\

          For the converse, assume $\varphi_1(\overline{x})$ isolates
          $\text{tp}^{\mathcal{M}}(\overline{a}/\emptyset)$ and
          $\varphi_2(\overline{a},\overline{y})$ isolates
          $\text{tp}^{\mathcal{M}}(\overline{b}/\overline{a})$. We claim
          that
          $\varphi(\overline{x},\overline{y}):=
          \varphi_1(\overline{x})\wedge\varphi_2(\overline{x},\overline{y})$
          isolates
          $\text{tp}^{\mathcal{M}}(\overline{a},\overline{b}/\emptyset)$.
          Suppose not by contradiction. Then there exists a formula
          $\theta(\overline{x},\overline{y})\in
          \text{tp}^{\mathcal{M}}(\overline{a},\overline{b}/\emptyset)$
          such that $\mathcal{M}\models
          \varphi_1(\overline{x})\wedge\varphi_2(\overline{x},\overline{y})
          \wedge \neg\theta(\overline{x},\overline{y})$. Consider the
          formula $\exists y\; [\varphi_2(\overline{x},\overline{y})\wedge
          \theta(\overline{x},\overline{y})]\in
          \text{tp}^{\mathcal{M}}(\overline{a}/\emptyset)$.
        \end{proof}
    \end{enumerate}
\end{enumerate}
\end{document}
