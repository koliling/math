\documentclass{article}
\usepackage[left=3cm,right=3cm,top=3cm,bottom=3cm]{geometry}
\usepackage{amsmath,amssymb,amsthm,pgfplots,tikz,mathtools}
\usepackage[inline]{enumitem}
\usetikzlibrary{patterns}
\usepackage{color}
\setlength{\parindent}{2mm}
\newcommand{\TODO}[1]{\textcolor{red}{TODO: #1}}

\begin{document}
\title{Basic Logic I: Homework 8}
\author{Li Ling Ko\\ lko@nd.edu}
\date{\today}
\maketitle

\begin{enumerate}[label={\bf Q\arabic*:}]
  \item Let $B\subseteq A\subseteq M$. For $p(\overline{x})\in
    S^{\mathcal{M}}_n(A)$ let \[p\restriction B=\{\varphi(x)\in
    p(\overline{x}):\varphi(\overline{x})\; \text{is a formula over}\;
    B\}.\] Show that the map $p\mapsto p\restriction B$ is a continuous
    map from $S^{\mathcal{M}}_n(A)$ to $S^{\mathcal{M}}_n(B)$.

    \begin{proof}
      Let $r:S^{\mathcal{M}}_n(A)\rightarrow S^{\mathcal{M}}_n(B)$ denote
      the restriction map function, let $\phi(\bar{x})$ be a formula over
      $B$, and let $\mathcal{O}_{\phi,B}=\{q\in
      S^{\mathcal{M}}_n(B):\phi\in q\}$. We need to show that
      $r^{-1}(\mathcal{O}_{\phi,B})$ is open in the topology over
      $S^{\mathcal{M}}_n(A)$. This is true since
      $r^{-1}(\mathcal{O}_{\phi,B})=\mathcal{O}_{\phi,A}$; every
      $p\in S^{\mathcal{M}}_n(A)$ is the inverse map of some $q\in
      S^{\mathcal{M}}_n(B)$ because its restriction $p\restriction B$ is in
      $S^{\mathcal{M}}_n(B)$.
    \end{proof}

  \item Let $\mathcal{M}$ be a structure and $B\subseteq M$. For
    $p(x_1,\ldots,x_{m+n})\in S^{\mathcal{M}}_{m+n}(B)$ let $p\restriction
    m=\{\varphi(x_1,\ldots,x_m)\in p(x_1,\ldots,x_{m+n})\}$.
    \begin{enumerate}
      \item Show that $p\restriction m\in S^{\mathcal{M}}_m(B)$ for $p\in
        S^{\mathcal{M}}_{m+n}(B)$.
        \begin{proof}
          This is true from chasing definitions: If $p\restriction m$ is
          not a Complete type over $B$ in free variables $x_1,\ldots,x_m$,
          then there must exist a formula $\varphi(x_1,\ldots,x_m)$ over
          $B$ such that neither itself nor its negation is in
          $p\restriction m$; then since formulas with free variables
          $x_1,\ldots,x_m$ are also considered formulas with free variables
          $x_1,\ldots,x_{m+n}$, then neither $\varphi(x_1,\ldots,x_m)$ nor
          its negation is in $p$, which contradicts the Completeness of
          $p$.
        \end{proof}

      \item Show that the restriction map $p\mapsto p\restriction m$ is
        open, continuous, and surjective as a map from
        $S^{\mathcal{M}}_{m+n}(B)$ to $S^{\mathcal{M}}_{m}(B)$.
        \begin{proof}
        \end{proof}
    \end{enumerate}
\end{enumerate}
\end{document}
