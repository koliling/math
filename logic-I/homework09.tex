\documentclass{article}
\usepackage[left=3cm,right=3cm,top=3cm,bottom=3cm]{geometry}
\usepackage{amsmath,amssymb,amsthm,pgfplots,tikz,mathtools}
\usepackage[inline]{enumitem}
\usetikzlibrary{patterns}
\usepackage{color}
\setlength{\parindent}{2mm}
\newcommand{\TODO}[1]{\textcolor{red}{TODO: #1}}

\begin{document}
\title{Basic Logic I: Homework 9}
\author{Li Ling Ko\\ lko@nd.edu}
\date{\today}
\maketitle

\begin{enumerate}[label={\bf Q\arabic*:}]
  \item Let $A=\{a_1,\ldots,a_m\}$ be a finite subset of a structure
    $\mathcal{M}$. For $p(x_1,\ldots,x_n)\in S_n(A)$ let
    $\tilde{p}(x_1,\ldots,x_n)$ be the following set of formulas over
    $\emptyset$
    \[\tilde{p}=\{\varphi(x_1,\ldots,x_{n+m}):
    \varphi(x_1,\ldots,x_n,a_1,\ldots,a_m)\in p\}.\]
    Show that $\tilde{p}$ is a Complete type over $\emptyset$ and the
    map $p\mapsto\tilde{p}$ is an injective map from $S_n(A)$ to
    $S_{n+m}(T)$.

    \begin{proof}
      To show that $\tilde{p}$ is Complete, let
      $\varphi(x_1,\ldots,x_{n+m})$ be any formula in $(n+m)$ variables. We
      need to show that either $\varphi$ or $\neg\varphi$ is in
      $\tilde{p}$. Now since $p$ is Complete, either
      $\varphi(x_1,\ldots,x_n,a_1,\ldots,a_m)$ or
      $\neg\varphi(x_1,\ldots,x_n,a_1,\ldots,a_m)$ is in $p$, so by
      definition of $\tilde{p}$, either $\varphi$ or $\neg\varphi$ will be
      in $\tilde{p}$. \\

      To show that the map is injective, assume that $\tilde{p}=\tilde{q}$.
      Then given any formula $\varphi(x_1,\ldots,x_{n+m})$ in $(n+m)$
      variables, we have
      \begin{align*}
                        & \varphi(x_1,\ldots,x_n,a_1,\ldots,a_m) &\in p & \\
        \Leftrightarrow & \varphi(x_1,\ldots,x_{n+m}) &\in \tilde{p} & \\
        \Leftrightarrow & \varphi(x_1,\ldots,x_{n+m}) &\in \tilde{q} & \\
        \Leftrightarrow & \varphi(x_1,\ldots,x_n,a_1,\ldots,a_m) &\in q & \\
      \end{align*}
      Hence $p=q$, and the map is injective.
    \end{proof}

  \item Let $\mathcal{M}$, $\mathcal{N}$ be structures and
    $f:\mathcal{M}\rightarrow\mathcal{N}$ a partial elementary map with a
    finite domain $A=\text{dom}(f)$. If $\varphi(x)$ is a formula over $A$
    then it has form $\Psi(x,a_1,\ldots,a_m)$ for some formula
    $\Psi(x,x_1,\ldots,x_m)$ over $\emptyset$ and $a_1,\ldots,a_m \in A$,
    and we will denote by $f(\varphi)$ the formula
    $\Psi(x,f(a_1),\ldots,f(a_m))$. For a type $p(x)\in S_1(A)$ let
    $f(p)=\{f(\varphi):\varphi\in p\}$.
    \begin{enumerate}
      \item Show that $f(p)$ is a type in $\mathcal{N}$ over
        $\text{range}(f)$.

        \begin{proof}
          Let $\overline{a}$ denote $(a_1,\ldots,a_m)$. Given any finite
          set of formulas
          $\varphi_1(x,f(\overline{a})),\ldots,\varphi_n(x,f(\overline{a}))$
          in $f(p)$, we need to show that $\mathcal{N}$ realizes their
          conjunction \[\varphi(x,f(\overline{a})):=
          \varphi_1(x,f(\overline{a}))\wedge
          \ldots\wedge\varphi_n(x,f(\overline{a})),\] or equivalently,
          \[\mathcal{N}\models\exists x\; \varphi(x,f(\overline{a})).\]
          Now for each $i\in\{1,\ldots,n\}$, $\varphi_i(x,\overline{a})$ is
          a formula in $p$ by definition of $f(p)$, therefore by
          Completeness and consistency of $p$, $\varphi(x,\overline{a})$ is
          also in $p$. In other words, $\mathcal{M}$ realizes
          $\varphi(x,\overline{a})$, or equivalently,
          \[\mathcal{M}\models\exists x\; \varphi(x,\overline{a}).\]
          Therefore
          $\exists x\; \varphi(x,\overline{x})\in
          \text{tp}^\mathcal{M}(\overline{a}/\emptyset)$, which implies
          $\exists x\; \varphi(x,\overline{x})\in
          \text{tp}^\mathcal{N}(f(\overline{a})/\emptyset)$ since
          $\text{tp}^\mathcal{M}(\overline{a}/\emptyset)=
          \text{tp}^\mathcal{N}(f(\overline{a})/\emptyset)$. Hence
          $\mathcal{N}\models\exists x\; \varphi(x,f(\overline{a}))$ as
          required.
        \end{proof}

      \item Let $a$ realizes $p$ in $\mathcal{M}$ and $b$ realize $f(p)$ in
        $\mathcal{N}$. Show that $f\cup\{(a,b)\}$ is a partial elementary
        map.
        \begin{proof}
          We want to show that
          $\text{tp}^\mathcal{M}(a\overline{a}/\emptyset)
          =\text{tp}^\mathcal{N}(bf(\overline{a})/\emptyset)$, or
          equivalently, given any formula $\varphi(x,\overline{x})$ in
          $(m+1)$ variables, we have
          \[\mathcal{M}\models\varphi(a,\overline{a}))\Leftrightarrow
          \mathcal{N}\models\varphi(b,f(\overline{a})).\]
          Let $\varphi(x,\overline{x})$ be a formula in $(m+1)$ variables.
          Then
          \begin{align*}
            \mathcal{M}\models\varphi(a,\overline{a}))  & \Leftrightarrow
            \varphi(x,\overline{a})\in p(x) & (\text{by Completeness of}\;
              p(x)) \\
              & \Leftrightarrow \varphi(x,f(\overline{a}))\in f(p(x)) &
              (\text{by definition of}\; f(p)\; \text{and Completeness
              of}\; f(p)) \\
              & \Leftrightarrow \mathcal{N}\models\varphi(b,f(\overline{a}))) &
              (\text{by Completeness of}\; f(p)), \\
          \end{align*}
          as we are required to show.
        \end{proof}
    \end{enumerate}

  \item Let $T$ be the theory of vector spaces over $\mathbb{Q}$,
    $\mathcal{M}$ a model of $T$ and $A\subseteq M$. Show that
    $|S^{\mathcal{M}}_1(A)|\leq|A|+\omega$.

    \begin{proof}
      First note that if $\mathcal{M}$ is finite, then there can only be a
      maximum of $|M|$ types in $S^{\mathcal{M}}_1(A)$ (one type for each
      element in $M$), so the assertion will be trivially true. Hence we
      can assume that $\mathcal{M}$ is infinite. \\

      Assume by contradiction that $|S^{\mathcal{M}}_1(A)|>|A|+\omega$.
      Then $|S^{\mathcal{M}}_1(\emptyset)|>|A|+\omega\geq\omega$, since
      $S^{\mathcal{M}}_1(\emptyset)\subseteq S^{\mathcal{M}}_1(A)$. Then by
      Theorem 7.16, we have $S^{\mathcal{M}}_1(\emptyset)=2^\omega$. Thus
      from Theorem 7.13, we will have
      $I(\text{Th}(\mathcal{M}),\omega)=2^\omega$. \\

      Now from the argument up to the second paragraph of Question 4 below,
      if we add to $T$ the set of sentences to make $T$ the theory of
      infinite vector spaces over $\mathbb{Q}$ (by adding to $T$
      sentences which say ``I have at least $n$ distinct elements'' for
      each $n\in\mathbb{N}$), then $T$ will be a complete theory. Thus
      since $\mathcal{M}$ is infinite, we get
      $T'\vdash\text{Th}(\mathcal{M})$, where $T'$ is the theory of
      infinite vector spaces over $\mathbb{Q}$. Then
      $I(\text{Th}(\mathcal{M}),\omega)=2^\omega$ will imply
      $I(T',\omega)=2^\omega$, which means that there are $2^\omega$
      countable vector spaces over $\mathbb{Q}$, up to vector space
      isomorphism. However this statement is not true, because there are
      only $\omega$ distinct countable vector spaces over $\mathbb{Q}$: 
      vector spaces of equal and finite dimensions are isomorphic to each
      other, and vector spaces with countable basis are also isomorphic to
      each other, giving a total of $\omega$ distinct countable vector
      spaces over $\mathbb{Q}$.
    \end{proof}

  \item Let $T$ be the theory of infinite vector spaces over $\mathbb{Q}$.
    Show that $|S_1(T)|<\omega$ but $S_2(T)$ is infinite.

    \begin{proof}
      First, note that considering $\mathbb{Q}$ as an infinite vector space
      $V_\mathbb{Q}$ over $\mathbb{Q}$ (defining $+$ and $\cdot$ the same
      way these operations are defined in $\mathbb{Q}$ as a ring),
      $V_\mathbb{Q}$ can be embedded in every model of $T$: Given any other
      infinite vector space $V$ over $\mathbb{Q}$, we can embed
      $V_\mathbb{Q}$ into $V$ by sending $0$ to $0$, $1$ to a fixed element
      $e$ of a basis of $V$ (such $e$ exists since $V$ is infinite), and
      the remaining $q\in V_\mathbb{Q}$ to $q\cdot e\in V$. It is routine
      to check that this map will preserve the axioms of vector spaces,
      which are associativity and commutativity of addition, identity
      element of addition, inverse element of addition, compatibility of
      scalar multiplication with field multiplication, identity element of
      scalar multiplication, and the distributive laws. Also, this map will
      be injective because $q\cdot e=0$ if and only if $q=0$ because $e$ is
      an element of a basis of $V$. \\

      Next, observe that $T$ is model complete from Theorem 5.3, because
      $T$ admits quantifier elimination by Question 1 of Homework 7. Then
      since $T$ has a model that can be embedded into any model of $T$, $T$
      must be complete: Given any sentence $\varphi$ in the language and
      any model $\mathcal{M}$ of $T$, $\varphi$ is satisfied by
      $\mathcal{M}$ if and only if $\varphi$ is satisfied by
      $V_\mathbb{Q}$ embedded in $\mathcal{M}$, because the embedding is
      elementary from model completeness of $T$. Hence $T$ is complete, and
      furthermore $T\models\text{Th}(V_\mathcal{Q})$. \\

      Therefore, $V_\mathbb{Q}$ is a prime model of $T$, and $S_n(T)$ is
      well-defined from completeness of $T$. Thus, from Corollary 6.17, the
      number of types in $S_n(T)$ will be the number of types in
      $S_n^{V_{\mathbb{Q}}}(T)$, so we can work within the model
      $V_\mathbb{Q}$. \\

      We show that $S_1^{V_{\mathbb{Q}}}(T)$ only has 2 types -
      $\text{tp}^{V_{\mathbb{Q}}}(0/\emptyset)$ and
      $\text{tp}^{V_{\mathbb{Q}}}(1/\emptyset)$. These two types are
      clearly distinct since $\text{tp}^{V_{\mathbb{Q}}}(0/\emptyset)$
      contains the formula $x=0$ while type
      $\text{tp}^{V_{\mathbb{Q}}}(1/\emptyset)$ does not. It remains to
      show that given any non-zero element $q\in V_\mathbb{Q}$, we will
      have $\text{tp}^{V_{\mathbb{Q}}}(q/\emptyset)=
      \text{tp}^{V_{\mathbb{Q}}}(1/\emptyset)$. To prove this claim,
      consider the map $\alpha:V_{\mathbb{Q}}\rightarrow V_{\mathbb{Q}}$,
      that sends $a$ to $aq$. This map will send $1$ to $q$, and can be
      verified to be automorphism of $\mathbb{Q}$-vector spaces. Since
      isomorphic structures are elementary equivalent, we have
      $V_\mathbb{Q}\models\varphi(1)\Leftrightarrow
      V_\mathbb{Q}\models\varphi(\alpha(1))$. Thus
      $\text{tp}^{V_{\mathbb{Q}}}(q/\emptyset)=
      \text{tp}^{V_{\mathbb{Q}}}(1/\emptyset)$, which concludes the proof
      that $S_1(T)$ has exactly 2 types. \\

      Finally, we show that $S_2(T)$ is infinite. Like before, it suffices
      to work within the model $V_\mathbb{Q}$. Consider the set of types
      \[S=\{\text{tp}^{V_\mathbb{Q}}(1,n/\emptyset): n\in\mathbb{N}^+\}.\]
      Each type in $S$ is distinct because
      $\text{tp}^{V_\mathbb{Q}}(1,n/\emptyset)$ contains the formula
      \[x_2=\underbrace{x_1+\ldots+x_1}_{n-\text{times}},\] which is not
      contained in any other type in $S$. Thus $S_2(T)$ is infinite.
    \end{proof}
\end{enumerate}
\end{document}
