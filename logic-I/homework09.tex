\documentclass{article}
\usepackage[left=3cm,right=3cm,top=3cm,bottom=3cm]{geometry}
\usepackage{amsmath,amssymb,amsthm,pgfplots,tikz,mathtools}
\usepackage[inline]{enumitem}
\usetikzlibrary{patterns}
\usepackage{color}
\setlength{\parindent}{2mm}
\newcommand{\TODO}[1]{\textcolor{red}{TODO: #1}}

\begin{document}
\title{Basic Logic I: Homework 9}
\author{Li Ling Ko\\ lko@nd.edu}
\date{\today}
\maketitle

\begin{enumerate}[label={\bf Q\arabic*:}]
  \item Let $A=\{a_1,\ldots,a_m\}$ be a finite subset of a structure
    $\mathcal{M}$. For $p(x_1,\ldots,x_n)\in S_n(A)$ let
    $\tilde{p}(x_1,\ldots,x_n)$ be the following set of formulas over
    $\emptyset$
    \[\tilde{p}=\{\varphi(x_1,\ldots,x_{n+m}):
    \varphi(x_1,\ldots,x_n,a_1,\ldots,a_m)\in p\}.\]
    Show that $\tilde{p}$ is a Complete type over $\emptyset$ and the
    map $p\mapsto\tilde{p}$ is an injective map from $S_n(A)$ to
    $S_{n+m}(T)$.

    \begin{proof}
      To show that $\tilde{p}$ is Complete, let
      $\varphi(x_1,\ldots,x_{n+m})$ be any formula in $(n+m)$ variables. We
      need to show that either $\varphi$ or $\neg\varphi$ is in
      $\tilde{p}$. Now since $p$ is Complete, either
      $\varphi(x_1,\ldots,x_n,a_1,\ldots,a_m)$ or
      $\neg\varphi(x_1,\ldots,x_n,a_1,\ldots,a_m)$ is in $p$, so by
      definition of $\tilde{p}$, either $\varphi$ or $\neg\varphi$ will be
      in $\tilde{p}$. \\

      To show that the map is injective, assume that $\tilde{p}=\tilde{q}$.
      Then given any formula $\varphi(x_1,\ldots,x_{n+m})$ in $(n+m)$
      variables, we have
      \begin{align*}
                        & \varphi(x_1,\ldots,x_n,a_1,\ldots,a_m) &\in p & \\
        \Leftrightarrow & \varphi(x_1,\ldots,x_{n+m}) &\in \tilde{p} & \\
        \Leftrightarrow & \varphi(x_1,\ldots,x_{n+m}) &\in \tilde{q} & \\
        \Leftrightarrow & \varphi(x_1,\ldots,x_n,a_1,\ldots,a_m) &\in q & \\
      \end{align*}
      Hence $p=q$, and the map is injective.
    \end{proof}

  \item Let $\mathcal{M}$, $\mathcal{N}$ be structures and
    $f:\mathcal{M}\rightarrow\mathcal{N}$ a partial elementary map with a
    finite domain $A=\text{dom}(f)$. If $\varphi(x)$ is a formula over $A$
    then it has form $\Psi(x,a_1,\ldots,a_m)$ for some formula
    $\Psi(x,x_1,\ldots,x_m)$ over $\emptyset$ and $a_1,\ldots,a_m \in A$,
    and we will denote by $f(\varphi)$ the formula
    $\Psi(x,f(a_1),\ldots,f(a_m))$. For a type $p(x)\in S_1(A)$ let
    $f(p)=\{f(\varphi):\varphi\in p\}$.
    \begin{enumerate}
      \item Show that $f(p)$ is a type in $\mathcal{N}$ over
        $\text{range}(f)$.

        \begin{proof}
        \end{proof}
    \end{enumerate}
\end{enumerate}
\end{document}
