\documentclass{article}
\usepackage[left=3cm,right=3cm,top=3cm,bottom=3cm]{geometry}
\usepackage{amsmath,amssymb,amsthm,tikz,mathtools}
\usepackage{color}
\usepackage[inline]{enumitem}
\usetikzlibrary{patterns}
\setlength{\parindent}{0mm}
\newcommand{\TODO}[1]{\textcolor{red}{TODO: #1}}

\begin{document}
\title{Basic Logic I: Homework 10}
\author{Li Ling Ko\\ lko@nd.edu}
\date{\today}
\maketitle

\begin{enumerate}[label={\bf Q\arabic*:}]
  \item Let $\mathcal{M}$ be an $\mathcal{L}$-structure, and $\{p_i(x):i\in
    I\}$ a family of types over $M$. Show that there is
    $\mathcal{M}\preceq\mathcal{N}$ realizing all types $p_i,i\in I$, with
    $|N|\leq|M|+|I|+\mathcal{L}+\omega$.

    \begin{proof}
      First, we show that given an $\mathcal{L}$-structure $\mathcal{K}$
      and a type $q(x)$ over $K$, there is an elementary extension
      $\mathcal{K}(q)\succeq\mathcal{K}$ of realizing $q(x)$, and such that
      the cardinality of $\mathcal{K}(q)$ is less than or equal
      $|K|+|\mathcal{L}|+\omega$: Let $\mathcal{K}'$ be an elementary
      extension of $\mathcal{K}$ that contains an element $a\in K'$ which
      realizes $q(x)$. Such $\mathcal{K}'$ exists from Proposition 6.3.
      Then by Downward Lowenheim-Skolem, $\mathcal{K}'$ will contain an
      elementary substructure $\mathcal{K}(a)$ that contains $K\cup\{a\}$
      and with cardinality less than or equal $|K|+|\mathcal{L}|+\omega$.
      Note that $a\in\mathcal{K}(q)$ will realize $q(x)$ because
      \[\mathcal{K}(q)\models\varphi(a,\overline{m}) \Leftrightarrow
      \mathcal{K}'\models\varphi(a,\overline{m})\] for every
      $\varphi(x,\overline{m})\in q(x)$ by elementary extension property. \\

      Given a family of types over $M$, by axiom of choice we
      can assume $I=\kappa$ for some cardinal $\kappa$. Then we construct
      an elementary chain $(\mathcal{M}_\alpha:\alpha<\kappa)$ of models,
      where
      \begin{align*}
        \mathcal{M}_0 &=\mathcal{M}, \\
        \mathcal{M}_{\alpha+1} &=\mathcal{M}_{\alpha}(p_\alpha), \\
        \mathcal{M}_{\alpha} &=\bigcup_{\beta<\alpha}\mathcal{M}_{\beta},
          &\text{if}\; \beta\; \text{is a limit ordinal}. \\
      \end{align*}
      By Proposition 8.8 this is an elementary chain. Then from an earlier
      homework assignment, we know that
      $\mathcal{N}=\cup_{\alpha<\kappa}\mathcal{M}_\beta$ is an elementary
      extension of $\mathcal{M}_\beta$ for all $\beta<\kappa$. Then by
      earlier argument, $\mathbb{N}$ will realize every $p_\alpha(x)$.
      Also, since $|\mathcal{M}_\alpha|\leq|M|+|\mathcal{L}|+\omega$, we
      have $|\mathcal{N}|\leq|M|+|I|+|\mathcal{L}|+\omega$ as required.
    \end{proof}

  \item Prove Proposition 8.8: Let $\alpha$ be an ordinal and
    $(\mathcal{M}_\beta,\beta<\alpha)$ be a chain of models such that
    $\mathcal{M}_\beta\preceq\mathcal{M}_{\beta+1}$ for all $\beta<\alpha$
    and $\mathcal{M}_\gamma=\cup_{\beta<\gamma}\mathcal{M}_\beta$ for all
    limit $\gamma<\alpha$. Then $(\mathcal{M}_\beta,\beta<\alpha)$ is an
    elementary chain.

    \begin{proof}
      Assume the claim is not true. Then by well-ordering of ordinals there
      exists some minimal ordinal $\beta\leq\alpha$ such that
      $(\mathcal{M}_\gamma,\gamma<\beta)$ is not an elementary chain. So
      there exists some $\gamma_1<\gamma_2<\beta$ such that
      $\mathcal{M}_{\gamma_1}\not\preceq\mathcal{M}_{\gamma_2}$. If $\beta$
      is a limit ordinal, then $(\mathcal{M}_\gamma,\gamma<\gamma_2+1)$
      would not be an elementary chain, contradicting minimality of
      $\beta$. Thus $\beta$ is a successor ordinal. Then we must have
      $\gamma_2=\beta-1$ because $\gamma_2<\beta-1$ would imply that
      $(\mathcal{M}_\gamma,\gamma<\beta-1)$ is not an elementary chain,
      contradicting minimality of $\beta$. If $\beta-1$ is a limit ordinal,
      then $(\mathcal{M}_\gamma,\gamma<\gamma_2+1)$ would not be an
      elementary chain, contradicting minimality of $\beta$. Thus $\beta-1$
      must be a successor ordinal. Then we have $\mathcal{M}_{\gamma_1}
      \preceq\mathcal{M}_{\beta-2} \preceq\mathcal{M}_{\beta-1}$, where the
      first equality follows from minimality of $\beta$ and the second
      equality follows from definition. Then transitivity of $\preceq$
      gives $\mathcal{M}_{\gamma_1}\preceq\mathcal{M}_{\beta-1}$, a
      contradiction.
    \end{proof}

  \item Prove Theorem 9.16: Let $\mathcal{M}$ be a structure. The operator
    $A\mapsto\text{acl}_\mathcal{M}(A)$ is a finitery closure operator on
    $M$.

    \begin{proof}
      $A\subseteq\text{acl}_\mathcal{M}(A)$ for any $A\subseteq M$:
      Given $a\in A$, $\text{acl}_\mathcal{M}(A)$ contains $a$ because
      $\text{tp}^{\mathcal{M}}(a/A)$ contains the algebraic formula $x=a$.
      \\

      $A\subseteq B\subseteq M\Rightarrow \text{acl}_\mathcal{M}(A)
      \subseteq\text{acl}_\mathcal{M}(B)$:
      If $\text{tp}^{\mathcal{M}}(a/A)$ where $a\in M$ contains an
      algebraic formula, then the same formula will also make
      $\text{tp}^{\mathcal{M}}(a/B) \supseteq\text{tp}^{\mathcal{M}}(a/A)$
      algebraic. \\

      $\text{acl}_\mathcal{M}(\text{acl}_\mathcal{M}(A))
      =\text{acl}_\mathcal{M}(A)$: Suffices to prove $\subseteq$ since
      $\supseteq$ follows directly from the previous claim. Let
      $a\in\text{acl}_\mathcal{M}(\text{acl}_\mathcal{M}(A))$. Then by
      Claim 9.8, $\varphi(x,\bar{b})\vdash
      \text{tp}^{\mathcal{M}}(a/\text{acl}_\mathcal{M}(A))$ for some
      formula $\varphi(x,\bar{y})$ and $\bar{b}=(b_0,\ldots,b_k)\subseteq
      \text{acl}_\mathcal{M}(A)$. Consider the formula
      \[\varphi'(x,\bar{y}):= \varphi(x,\bar{y})\wedge \exists^{\leq
      n} x'\varphi(x',\bar{y}),\] where
      $n=|\varphi(M,\bar{b})|$. Note that $n<\omega$ from algebraicity of
      $a$. \\

      Similarly, for each $b_i$, by Claim
      9.8, $\varphi_i(y,\bar{c_i})\vdash \text{tp}^{\mathcal{M}}(b_i/A)$
      for some formulas $\varphi_i(y,\overline{z})$ and
      $\overline{c_i}\subseteq A$. Denote
      \[\varphi_i'(y,\bar{z}):= \varphi_i(y,\bar{z})\wedge \exists^{\leq
      n_i} y'\varphi_i(y',\bar{z}),\] where
      $n_i=|\varphi_i(M,\overline{c_i})|$. Note that $n_i<\omega$ from
      algebraicity of $b_i$. \\

      Consider the formula \[\phi(x,\overline{z_1},\ldots,\overline{z_k})
      :=\exists y_1,\ldots,y_k\; \left[\varphi'(x,y_1,\ldots,y_k)
      \wedge\varphi_1'(y_1,\overline{z_1})\wedge\ldots
      \wedge\varphi_k'(y_k,\overline{z_k})\right].\] Then
      $\phi(x,\overline{c}):= \phi(x,\overline{c_1},\ldots,\overline{c_k})$
      is contained in $\text{tp}^{\mathcal{M}}(a/A)$. Furthermore,
      \[|\phi(M,\bar{c})|\leq n\cdot\max(n_1,\ldots,n_k),\] because given
      any $\bar{z}=\overline{z_1},\ldots,\overline{z_k}$, there cannot be
      more than $n\cdot\max(n_1,\ldots,n_k)$ values of $x$ satisfying
      $\phi(x,\bar{z})$. Hence $\phi(x,\overline{c})$ is algebraic, and so
      $a\in\text{acl}_\mathcal{M}(A)$. \\

      $\text{acl}_\mathcal{M}(A) =\bigcup_{A_0\subseteq_{\text{fin}} A}
      \text{acl}_\mathcal{M}(A_0)$: If $m\in\text{acl}_\mathcal{M}(A)$,
      then $\text{tp}^\mathcal{M}(m/A)$ contains an algebraic formula
      $\varphi(x,a_1,\ldots,a_k)$, where $a_1,\ldots,a_k\in A$. Write
      $A_0=\{a_1,\ldots,a_k\}$. Now $\varphi(x,a_1,\ldots,a_k)
      \in\text{tp}^\mathcal{M}(m/A_0)$, hence
      $m\in\text{acl}_\mathcal{M}(A_0)$. For the converse, if
      $m\in\text{acl}_\mathcal{M}(A_0)$ for some $A_0\subseteq_{\text{fin}}
      A$, then $\text{tp}^\mathcal{M}(m/A_0)$ contains an algebraic formula
      which will also be contained in $\text{tp}^\mathcal{M}(m/A_0)$ since
      $\text{tp}^\mathcal{M}(m/A_0)\subseteq \text{tp}^\mathcal{M}(m/A)$,
      hence $m\in\text{acl}_\mathcal{M}(A)$. 
    \end{proof}

  \item Let $T$ be a complete theory in a countable language, and
    $\mathcal{M}$ and $\omega$-saturated model fo $T$. Assume $\mathcal{M}$
    is minimal (i.e. every definable subset of $M$ is either finite or
    co-finite). Prove that $T$ is strongly minimal.

    \begin{proof}
      First, note that from minimality of $\mathcal{M}$, the formula
      ``$x=x$'' is a minimal formula in $\mathcal{M}$: Otherwise there is a
      formula $\varphi(x,\bar{m})$ where $\bar{m}\subseteq M$ such that the
      formulas ``$x=x\wedge\varphi(x,\bar{m})$'' and
      ``$x=x\wedge\neg\varphi(x,\bar{m})$'' both have infinite solutions
      for $x$, which contradicts $\{a\in M:\varphi(a,\bar{m})\}$ being
      finite or co-finite. \\

      We wish to show that the formula ``$x=x$''  is minimal in every
      elementary extension $\mathcal{N}\succeq\mathcal{M}$. Since ``$x=x$''
      is minimal in $\mathcal{M}$, it suffices to show that minimal
      formulas in $\mathcal{M}$ remain minimal in elementary extensions.
    \end{proof}

  \item Let $\mathcal{M}=\langle\mathbb{N},<\rangle$. Describe
    $\text{acl}(\emptyset)$.

    \begin{proof}
      $\text{acl}(\emptyset)=\mathbb{N}$. Given $n\in\mathbb{N}$, the
      formula $\varphi_n(x)$ which says ``there are exactly $n$ elements
      smaller than me'' is a formula in
      $\text{tp}^\mathcal{M}(n/\emptyset)$, and this formula is algebraic
      because it has only one solution in $\mathbb{N}$. Hence from Claim
      9.6 $n$ is algebraic, and must be contained in
      $\text{acl}(\emptyset)$.
    \end{proof}

  \item Prove that a strongly minimal theory $T$ in a countable language is
    $\omega$-categorical if and only if $\text{acl}(A)$ is fininte for
    every finite $A\subseteq M$ in any $\mathcal{M}\models T$. 

    \begin{proof}
    \end{proof}
\end{enumerate}
\end{document}
