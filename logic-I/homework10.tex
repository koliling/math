\documentclass{article}
\usepackage[left=3cm,right=3cm,top=3cm,bottom=3cm]{geometry}
\usepackage{amsmath,amssymb,amsthm,pgfplots,tikz,mathtools}
\usepackage[inline]{enumitem}
\usetikzlibrary{patterns}
\usepackage{color}
\setlength{\parindent}{2mm}
\newcommand{\TODO}[1]{\textcolor{red}{TODO: #1}}

\begin{document}
\title{Basic Logic I: Homework 10}
\author{Li Ling Ko\\ lko@nd.edu}
\date{\today}
\maketitle

\begin{enumerate}[label={\bf Q\arabic*:}]
  \item Let $\mathcal{M}$ be an $\mathcal{L}$-structure, and $\{p_i(x):i\in
    I\}$ a family of types over $M$. Show that there is
    $\mathcal{M}\preceq\mathcal{N}$ realizing all types $p_i,i\in I$, with
    $|N|\leq|M|+|I|+\mathcal{L}+\omega$.

    \begin{proof}
      First, we show that given an $\mathcal{L}$-structure $\mathcal{K}$
      and a type $q(x)$ over $K$, there is an elementary extension
      $\mathcal{K}(q)\succeq\mathcal{K}$ of realizing $q(x)$, and such that
      the cardinality of $\mathcal{K}(q)$ is less than or equal
      $|K|+|\mathcal{L}|+\omega$: Let $\mathcal{K}'$ be an elementary
      extension of $\mathcal{K}$ that contains an element $a\in K'$ which
      realizes $q(x)$. Such $\mathcal{K}'$ exists from Proposition 6.3.
      Then by Downward Lowenheim-Skolem, $\mathcal{K}'$ will contain an
      elementary substructure $\mathcal{K}(a)$ that contains $K\cup\{a\}$
      and with cardinality less than or equal $|K|+|\mathcal{L}|+\omega$.
      Note that $a\in\mathcal{K}(q)$ will realize $q(x)$ because
      \[\mathcal{K}(q)\models\varphi(a,\overline{m}) \Leftrightarrow
      \mathcal{K}'\models\varphi(a,\overline{m})\] for every
      $\varphi(x,\overline{m})\in q(x)$ by elementary extension property. \\

      Given a family of types over $M$, by axiom of choice we
      can assume $I=\kappa$ for some cardinal $\kappa$. Then we construct
      an elementary chain $(\mathcal{M}_\alpha:\alpha<\kappa)$ of models,
      where
      \begin{align*}
        \mathcal{M}_0 &=\mathcal{M}, \\
        \mathcal{M}_{\alpha+1} &=\mathcal{M}_{\alpha}(p_\alpha), \\
        \mathcal{M}_{\alpha} &=\bigcup_{\beta<\alpha}\mathcal{M}_{\beta},
          &\text{if}\; \beta\; \text{is a limit ordinal}. \\
      \end{align*}
      By Proposition 8.8 this is an elementary chain. Then from an earlier
      homework assignment, we know that
      $\mathcal{N}=\cup_{\alpha<\kappa}\mathcal{M}_\beta$ is an elementary
      extension of $\mathcal{M}_\beta$ for all $\beta<\kappa$. Then by
      earlier argument, $\mathbb{N}$ will realize every $p_\alpha(x)$.
      Also, since $|\mathcal{M}_\alpha|\leq|M|+|\mathcal{L}|+\omega$, we
      have $|\mathcal{N}|\leq|M|+|I|+|\mathcal{L}|+\omega$ as required.
    \end{proof}

  \item Prove Proposition 8.8: Let $\alpha$ be an ordinal and
    $(\mathcal{M}_\beta,\beta<\alpha)$ be a chain of models such that
    $\mathcal{M}_\beta\preceq\mathcal{M}_{\beta+1}$ for all $\beta<\alpha$
    and $\mathcal{M}_\gamma=\cup_{\beta<\gamma}\mathcal{M}_\beta$ for all
    limit $\gamma<\alpha$. Then $(\mathcal{M}_\beta,\beta<\alpha)$ is an
    elementary chain.

    \begin{proof}
      Assume the claim is not true. Then by well-ordering of ordinals there
      exists some minimal ordinal $\beta\leq\alpha$ such that
      $(\mathcal{M}_\gamma,\gamma<\beta)$ is not an elementary chain. So
      there exists some $\gamma_1<\gamma_2<\beta$ such that
      $\mathcal{M}_{\gamma_1}\not\preceq\mathcal{M}_{\gamma_2}$. If $\beta$
      is a limit ordinal, then $(\mathcal{M}_\gamma,\gamma<\gamma_2+1)$
      would not be an elementary chain, contradicting minimality of
      $\beta$. Thus $\beta$ is a successor ordinal. Then we must have
      $\gamma_2=\beta-1$ because $\gamma_2<\beta-1$ would imply that
      $(\mathcal{M}_\gamma,\gamma<\beta-1)$ is not an elementary chain,
      contradicting minimality of $\beta$. If $\beta-1$ is a limit ordinal,
      then $(\mathcal{M}_\gamma,\gamma<\gamma_2+1)$ would not be an
      elementary chain, contradicting minimality of $\beta$. Thus $\beta-1$
      must be a successor ordinal. Then we have $\mathcal{M}_{\gamma_1}
      \preceq\mathcal{M}_{\beta-2} \preceq\mathcal{M}_{\beta-1}$, where the
      first equality follows from minimality of $\beta$ and the second
      equality follows from definition. Then transitivity of $\preceq$
      gives $\mathcal{M}_{\gamma_1}\preceq\mathcal{M}_{\beta-1}$, a
      contradiction.
    \end{proof}

  \item Prove Theorem 9.16: Let $\mathcal{M}$ be a structure. The operator
    $A\mapsto\text{acl}_\mathcal{M}(A)$ is a finitery closure operator on
    $M$.

    \begin{proof}
    \end{proof}
\end{enumerate}
\end{document}
