\documentclass{article}
\usepackage[left=3cm,right=3cm,top=3cm,bottom=3cm]{geometry}
\usepackage{amsmath,amssymb,amsthm,tikz,mathtools}
\usepackage{color}
\usepackage[inline]{enumitem}
\usetikzlibrary{patterns}
\setlength{\parindent}{0mm}
\newcommand{\TODO}[1]{\textcolor{red}{TODO: #1}}

\begin{document}
\title{Basic Logic I: Homework 11}
\author{Li Ling Ko\\ lko@nd.edu}
\date{\today}
\maketitle

\begin{enumerate}[label={\bf Q\arabic*:}]
  \item Let $T$ be an inductive theory in a countable language and $\kappa$
    an infinite cardinal. Prove that if $T$ is $\kappa$-categorical then
    $T$ is model complete.

    \begin{proof}
    \end{proof}

  \item Let $\mathcal{L}$ be the language containing a unary function $f$
    and a binary relation symbol $R$. Let $T$ be the $\mathcal{L}$-theory
    $T=\{\forall x\forall y\; [R(x,y)\rightarrow R(x,f(y))]\}$.

    \begin{enumerate}
      \item Let $\mathcal{M}$ be an existentially closed in $T$ structure,
        and $a,b\in M$ with $b\not\in\{a,f(a),f^2(a),\ldots\}$. Show that
        $\mathcal{M}\models \exists x\; (R(x,a)\wedge\neg R(x,b))$.

        \begin{proof}
          Assume by contradiction that $\mathcal{M}\models \forall x\;
          (R(x,a)\rightarrow R(x,b))$. We construct an extension
          $\mathcal{N}$ of $\mathcal{M}$ with $N=M\sqcup\{c\}$
          such that $\mathcal{N}\models \exists x\; (R(x,a)\wedge\neg
          R(x,b))$ with $c$ as the witness. Set $f^\mathcal{N}(c)=c$, and
          \[R^\mathcal{N}= R^\mathcal{M} \cup
          \{(c,f^n(a)):n\in\mathbb{N}\}.\]

          These relations will ensure that $\mathcal{N}$ is a model of $T$
          and that $c$ witnesses $\mathcal{N}\models \exists x\;
          (R(x,a)\wedge\neg R(x,b))$, which implies $\mathcal{M}\models
          \exists x\; (R(x,a)\wedge\neg R(x,b))$ by existential closure of
          $\mathcal{M}$, a contradiction.
        \end{proof}

      \item Let $\mathcal{M}$ be a model of $T$ and $a\in M$ with
        $f^n(a)\neq f^m(a)$ for all $n\neq m$. Show that there is an
        elementary extension $\mathcal{N}$ of $\mathcal{M}$ and
        $b\not\in\{a,f(a),f^2(a),\ldots\}$ with $\mathcal{N}\models\forall
        x\; [R(x,a)\rightarrow R(x,b)]$.

        \begin{proof}
          Note that if $R(M,a)=\emptyset$, then the assertion is trivially
          true with $\mathcal{N}=\mathcal{M}$. Hence we can assume there
          exists $a_0\in M$ with $\mathcal{M}\models R(a_0,a)$. \\

          Let $\mathcal{L}'=\mathcal{L}(\mathcal{M})\cup\{b\}$, where $b$
          is a constant that does not appear in $\mathcal{L}(\mathcal{M})$.
          Let
          \[\begin{array}{rrl}
            T' &:= &T_\mathcal{M} \\
              &&\cup \{c_m\neq c_n:m\neq n\in M\} \\
              &&\cup\{R(c_m,b):m\in R(M,a)\} \\
              &&\cup\{b\neq f^n(a):n\in\mathbb{N}\}
          \end{array}\]
          be a theory in the language $\mathcal{L}'$. We show that
          $T'$ is finitely satisfied by $\mathcal{M}$: Assign each $c_m$ to
          $m\in M$. Then clearly $\mathcal{M}$ satisfies all sentences in
          $T_\mathcal{M}$ and also formulas of the form $c_m\neq c_n\in
          T'$ by definition. Given a finite set of formulas
          $\{b\neq f^{0}(a),\ldots,b\neq f^{n}(a)\}\subset T'$, assign $b$
          to $f^{n+1}(a)\in M$. This assignment will satisfy all formulas
          $R(c_m,b)\in T'$ because $R(m,a)$ implies $R(m,f^{n+1}(a))$ by
          induction on $n$. \\

          Thus by Compactness theorem $T'$ is satisfied by some model
          $\mathcal{N}$. Then since $T\subset T'$ and $T_\mathcal{M}\subset
          T'$, $\mathcal{N}$ is an elementary extension of $\mathcal{M}$ in
          the original language $\mathcal{L}$, and also
          $\mathcal{N}\models\forall x\; [R(x,a)\rightarrow R(x,b)]$ by our
          choice of $T'$.
        \end{proof}

      \item Show that the class of existentially closed in $T$ structures
        is not axiomatizable.
        \begin{proof}
        \end{proof}
    \end{enumerate}
\end{enumerate}
\end{document}
