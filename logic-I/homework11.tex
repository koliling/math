\documentclass{article}
\usepackage[left=3cm,right=3cm,top=3cm,bottom=3cm]{geometry}
\usepackage{amsmath,amssymb,amsthm,tikz,mathtools}
\usepackage{color}
\usepackage[inline]{enumitem}
\usetikzlibrary{patterns}
\setlength{\parindent}{0mm}
\newcommand{\TODO}[1]{\textcolor{red}{TODO: #1}}

\begin{document}
\title{Basic Logic I: Homework 11}
\author{Li Ling Ko\\ lko@nd.edu}
\date{\today}
\maketitle

\begin{enumerate}[label={\bf Q\arabic*:}]
  \item Let $T$ be an inductive theory in a countable language and $\kappa$
    an infinite cardinal. Prove that if $T$ is $\kappa$-categorical then
    $T$ is model complete.

  \item Let $\mathcal{L}$ be the language containing a unary function $f$
    and a binary relation symbol $R$. Let $T$ be the $\mathcal{L}$-theory
    $T=\{\forall x\forall y\; [R(x,y)\rightarrow R(x,f(y))]\}$.

    \begin{enumerate}
      \item Let $\mathcal{M}$ be an existentially closed in $T$ structure,
        and $a,b\in M$ with $b\not\in\{a,f(a),f^2(a),\ldots\}$. Show that
        $\mathcal{M}\models \exists x\; (R(x,a)\wedge\neg R(x,b))$.

        \begin{proof}
          Assume by contradiction that $\mathcal{M}\models \forall x\;
          (\neg R(x,a)\vee R(x,b))$. We construct an extension
          $\mathcal{N}$ of $\mathcal{M}$ with $N=M\cup\{c\}$ such that
          $\mathcal{N}\models \exists x\; (R(x,a)\wedge\neg R(x,b))$ with
          $c$ as the witness. This would contradict existential-closure of
          $\mathcal{M}$. \\

          We set $N=M\cup\{c\}$, and $f^\mathcal{N}(c)=c$. Also, we set
          \[R^\mathcal{N}= R^\mathcal{M} \cup R_{x<c} \cup R_{c<x},\]
          where
          \[R_{x<c}:= \{R(b,c)\}\cup \{R(x,c):x\in M, R(x,b)\},\]
          and
          \[R_{c<x}:= \{R(c,a)\}\cup \{R(c,x):x\in M, R(a,x)\}.\]

          These relations will ensure that $\mathcal{N}$ is a model of $T$
          and that $\mathcal{N}\models \exists x\; (R(x,a)\wedge\neg
          R(x,b))$, which implies $\mathcal{M}\models \exists x\;
          (R(x,a)\wedge\neg R(x,b))$ by existential closure of
          $\mathcal{M}$, a contradiction.
        \end{proof}
    \end{enumerate}
\end{enumerate}
\end{document}
