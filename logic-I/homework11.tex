\documentclass{article}
\usepackage[left=3cm,right=3cm,top=3cm,bottom=3cm]{geometry}
\usepackage{amsmath,amssymb,amsthm,tikz,mathtools}
\usepackage{color}
\usepackage[inline]{enumitem}
\usetikzlibrary{patterns}
\setlength{\parindent}{0mm}
\newcommand{\TODO}[1]{\textcolor{red}{TODO: #1}}

\begin{document}
\title{Basic Logic I: Homework 11}
\author{Li Ling Ko\\ lko@nd.edu}
\date{\today}
\maketitle

\begin{enumerate}[label={\bf Q\arabic*:}]
  \item Let $T$ be an inductive theory in a countable language and $\kappa$
    an infinite cardinal. Prove that if $T$ is $\kappa$-categorical and
    does not have finite models then $T$ is model complete.

    \begin{proof}
      We first prove that every model $\mathcal{K}$ of $T$ of cardinality
      $\kappa$ is ec-closed in $T$: Let $\mathcal{M}$ be a model of $T$
      that embeds $\mathcal{K}$, let $\bar{k}\in\mathcal{K}$, and let
      $\varphi(\bar{x},\bar{y})$ be a quantifier-free formula such that
      \[\mathcal{M}\models \exists\bar{x}\; \varphi(\bar{x},\bar{k}).\] We
      wish to show that $\mathcal{K}\models \exists\bar{x}\;
      \varphi(\bar{x},\bar{k})$. Now $T$ has an ec-closed model
      $\mathcal{K}_0$ of cardinality $\kappa$ (Lemma 11.13). Then
      $\mathcal{K}$ is isomorphic to $\mathcal{K}_0$ by
      $\kappa$-categoricity of $T$. Let
      $\alpha:\mathcal{K}\rightarrow\mathcal{K}_0$ be an isomorphism
      between the models. Consider the set of formulas $T_{M_0}$ given by
      the $T_M$ but by renaming each constant $c_a$ for elements $a\in K$
      by $c_{\alpha(a)}$. Then $T_{M_0}$ is satisfied by model
      $\mathcal{M}_0\models T$, which is constructed from $\mathcal{M}$
      through renaming elements $a$ in $K$ by $\alpha(a)$. Furthermore,
      $\mathcal{M}_0$ embeds $\mathcal{K}_0$, and
      \[\mathcal{M}\models \exists\bar{x}\; \varphi(\bar{x},\bar{k})\;
        \Leftrightarrow \mathcal{M}_0\models
        \exists\bar{x}\; \varphi(\bar{x},\alpha(\bar{k})).\]

      from isomorphism $\alpha$ between $\mathcal{K}$ and $\mathcal{K}_0$.
      Then
      \begin{align*}
        \mathcal{M}_0\models \exists\bar{x}\;
          \varphi(\bar{x},\alpha(\bar{k})) &\Leftrightarrow
            \mathcal{K}_0\models \exists\bar{x}\;
            \varphi(\bar{x},\alpha(\bar{k})) &(\text{from ce-closure of}\;
            \mathcal{K}_0) \\
          &\Leftrightarrow \mathcal{K}\models
            \exists\bar{x}\; \varphi(\bar{x},\bar{k}),
            &(\text{from isomorphism}\;
            \alpha:\mathcal{K}\rightarrow\mathcal{K}_0) \\
      \end{align*}
      which completes our proof that $\mathcal{K}$ is ce-closed in $T$. \\

      Assume by contradiction that $T$ is not model-complete. Then $T$ must
      have a model $\mathcal{M}$ that is not ec-closed in $T_\forall$
      (Theorem 11.14.3). Since models of $T_\forall$ are exactly the
      substructures of models of $T$ (Lemma 11.2), $\mathcal{M}$ must be
      embedded in some substructure $\mathcal{N}_0$ of $\mathcal{N}\models
      T$, where
      \[\mathcal{M}\not\leq_{\text{ec}}\mathcal{N}_0\leq\mathcal{N}\models
        T.\]
      Then $\mathcal{M}\leq\mathcal{N}$ since embeddings are transitive,
      but $\mathcal{M}$ cannot be ec-closed in $\mathcal{N}$, because
      \begin{align*}
        \mathcal{N}_0\models \exists\bar{x}\;
          \varphi(\bar{x},\bar{m}) &\Rightarrow
          \mathcal{N}\models \exists\bar{x}\;
          \varphi(\bar{x},\bar{m})
          &(\because\mathcal{N}_0\leq\mathcal{N};\; \text{where}\;
          \bar{m}\in M\; \text{and}\; \varphi\; \text{is quantifier-free}) \\
        &\Rightarrow\mathcal{M}\models \exists\bar{x}\;
          \varphi(\bar{x},\bar{m}), &(\text{assuming}\;
          \mathcal{M}\leq_{\text{ec}}\mathcal{N}) \\
      \end{align*}
      which would contradict
      $\mathcal{M}\not\leq_{\text{ec}}\mathcal{N}_0$. Thus we have an
      infinite model $\mathcal{M}$ of $T$ that is not ec-closed in
      $\mathcal{N}\models T$, which implies that $T$ has a model of
      cardinality $\kappa$ that is not ec-closed in $T$ (Lemma 11.15). This
      would contradict our claim in the first paragraph.
    \end{proof}

  \item Let $\mathcal{L}$ be the language containing a unary function $f$
    and a binary relation symbol $R$. Let $T$ be the $\mathcal{L}$-theory
    $T=\{\forall x\forall y\; [R(x,y)\rightarrow R(x,f(y))]\}$.

    \begin{enumerate}
      \item Let $\mathcal{M}$ be an existentially closed in $T$ structure,
        and $a,b\in M$ with $b\not\in\{a,f(a),f^2(a),\ldots\}$. Show that
        $\mathcal{M}\models \exists x\; (R(x,a)\wedge\neg R(x,b))$.

        \begin{proof}
          Assume by contradiction that $\mathcal{M}\models \forall x\;
          (R(x,a)\rightarrow R(x,b))$. We construct an extension
          $\mathcal{N}\models T$ of $\mathcal{M}$ with $N=M\sqcup\{c\}$
          such that $\mathcal{N}\models \exists x\; (R(x,a)\wedge\neg
          R(x,b))$ with $c$ as the witness. Set $f^\mathcal{N}(c)=c$, and
          \[R^\mathcal{N}= R^\mathcal{M} \cup
          \{(c,f^n(a)):n\in\mathbb{N}\}.\]

          These relations will ensure that $\mathcal{N}$ is a model of $T$.
          Also, since $b\neq f^n(a)$ for all $n\in\mathbb{N}$, $c$ will
          witness $\mathcal{N}\models \exists x\; (R(x,a)\wedge\neg
          R(x,b))$. Also, since the new relations each involve the new
          element $c$, $\mathcal{N}$ is an extension of $\mathcal{M}$.
          Thus, from existential closure of $\mathcal{M}$, we have
          $\mathcal{M}\models \exists x\; (R(x,a)\wedge\neg R(x,b))$, a
          contradiction.
        \end{proof}

      \item Let $\mathcal{M}$ be a model of $T$ and $a\in M$ with
        $f^n(a)\neq f^m(a)$ for all $n\neq m$. Show that there is an
        elementary extension $\mathcal{N}$ of $\mathcal{M}$ and
        $b\not\in\{a,f(a),f^2(a),\ldots\}$ with $\mathcal{N}\models\forall
        x\; [R(x,a)\rightarrow R(x,b)]$.

        \begin{proof}
          Let $\mathcal{L}'=\mathcal{L}(\mathcal{M})\cup\{b\}$, where $b$
          is a constant that does not appear in $\mathcal{L}(\mathcal{M})$.
          Let
          \[\begin{array}{rrl}
            T' &:= &T_\mathcal{M} \\
              &&\cup \{c_m\neq c_n:m\neq n\in M\} \\
              &&\cup\{R(c_m,b):m\in R(M,a)\} \\
              &&\cup\{b\neq f^n(a):n\in\mathbb{N}\}
          \end{array}\]
          be a theory in the language $\mathcal{L}'$. We show that
          $T'$ is finitely satisfied by $\mathcal{M}$: Assign each $c_m$ to
          $m\in M$. Then clearly $\mathcal{M}$ satisfies all sentences in
          $T_\mathcal{M}$ and also formulas of the form $c_m\neq c_n\in
          T'$ by definition. Given a finite set of formulas
          $\{b\neq f^{0}(a),\ldots,b\neq f^{n}(a)\}\subset T'$, assign $b$
          to $f^{n+1}(a)\in M$. This assignment will satisfy formulas of
          the form $R(c_m,b)\in T'$ for each $m\in R(M,a)$ because $R(m,a)$
          implies $R(m,f^{n+1}(a))$ by induction on $n$. Also, since
          $f^{n+1}(a)\neq f^i(a)$ for all $i<n+1$, the assignment will also
          satisfy the chosen finite set of formulas. \\

          Thus by Compactness theorem $T'$ is satisfied by some model
          $\mathcal{N}$. Then since $T\subset T'$ and $T_\mathcal{M}\subset
          T'$, $\mathcal{N}$ is an elementary extension of $\mathcal{M}$ in
          the original language $\mathcal{L}$, and also
          $\mathcal{N}\models\forall x\; [R(x,a)\rightarrow R(x,b)]$ by our
          choice of $T'$.
        \end{proof}

      \item Show that the class of existentially-closed in $T$ structures
        is not axiomatizable.
        \begin{proof}
          The class of existentially-closed in $T$ substructures is
          axiomatized by $T_{\forall\exists}$ (Lemma 11.5). Since $T$ is
          trivially a $\forall\exists$-theory, this class is axiomatized by
          $T$. Then $T\cup T=T$ should axiomatize the class of
          existentially-closed in $T$ structures, if such an axiomatization
          exists. Thus it suffices to show that $T$ does not axiomatize the
          class of existentially-closed in $T$ structures, i.e. we find two
          models $\mathcal{M}$ and $\mathcal{N}$ of $T$ where
          $\mathcal{M}\leq\mathcal{N}$ but
          $\mathcal{M}\not\leq_{\text{ec}}\mathcal{N}$. \\

          Consider models
          \[\mathcal{M}= \langle\{0\},=,\text{id}\rangle,\;\;
            \text{and}\;\; \mathcal{N}=
            \langle\{0,1\},=,\text{id}\rangle.\]
          Then $\mathcal{M}<\mathcal{N}\models T$ in the language
          $\mathcal{L}=\{=,\text{id}\}$, but
          $\mathcal{M}\not<_{\text{ec}}\mathcal{N}$ because the
          existential-formula $\exists x\; x\neq 0$ is satisfied by
          $\mathcal{N}$ but not by $\mathcal{M}$.

          %Note that $T\subseteq T_\forall$,
          %that all substructures of models of $T$ must satisfy $T$ (Lemma
          %11.2). Assume by contradiction that there is a set of sentences
          %$S$ in the language $\mathcal{L}$ that axiomatizes the
          %existentially-closed in $T$ structures. Then since models of $S$
          %are substructures of models of $T$, the models of $S$ must
          %satisfy $T$ by our earlier claim, and thus $S\vdash T$. Also
          %since all models of $T$ are trivially substructures of models of
          %$T$, we have $T\vdash S$, thus $S\sim T$.

          %Then from Question 2.1, we have \[S\vdash \exists x\;
          %(R(x,a)\wedge\neg R(x,b)).\]
        \end{proof}
    \end{enumerate}
\end{enumerate}
\end{document}
