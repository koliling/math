\documentclass{article}
\usepackage[left=3cm,right=3cm,top=3cm,bottom=3cm]{geometry}
\usepackage{amsmath,amssymb,amsthm,tikz,mathtools}
\usepackage{color}
\usepackage[inline]{enumitem}
\usetikzlibrary{patterns}
\setlength{\parindent}{0mm}
\newcommand{\TODO}[1]{\textcolor{red}{TODO: #1}}

\begin{document}
\title{Basic Logic I: Homework 12}
\author{Li Ling Ko\\ lko@nd.edu}
\date{\today}
\maketitle

\begin{enumerate}[label={\bf Q\arabic*:}]
  \item \it Prove Claim 13.11 from the notes: Let $F$ be a field and
    $P\subseteq F$ a semi-positive cone satisfying $P\cup(-P)=F$. Then $F$
    is an ordered field with the ordering $a>b$ defined as $a-b\in P$ and
    $a-b\neq0$. 

    \begin{proof}
      We check that the given ordering satisfies the axioms of linear
      order and the other ordering relations of an ordered field. \\

      Total order: The order is clearly total because $a-b$ is either in
      $P$ or not in $P$. \\

      Anti-symmetry: If $a>b$, then $a-b\in P$ and $a\neq b$, which implies
      $-(b-a)\notin P$ ($\because P\cap -P=0$), and thus $a\not<b$. \\

      Transitivity: If $a>b$ and $b>c$, then $a-b,b-c\in P$ and $a\neq b$
      and $b\neq c$, which implies $a-c=(a-b)+(b-c)\in P$ (by closure of
      $P$ under addition), and so $a\geq c$. \\

      $a>b\Rightarrow a+c>b+c$: If $a>b$, then $a-b\in P$ and $a\neq b$,
      which means $(a+c)-(b+c)=a-b\in P$ and $a+c\neq b+c$, which means
      $a+c>b+c$. \\

      $a>b\wedge c>0\Rightarrow ac>bc\wedge ca>cb$: If $a>b$ and $c>0$,
      then $a-b,c\in P$ and $a\neq b$ and $c\neq0$, so $ac-bc=(a-b)c\in P$
      (by closure under multiplication) and $ac-bc=(a-b)c\neq0$ (since $F$
      is an integral domain), and thus $ac>bc$. A similar argument shows
      that $ca>cb$.
    \end{proof}

  \item \it Prove Lemma 13.17 from the notes: If $F$ is a real-closed field
    and $a\in F$ then $a$ is a square or $-a$ is a square.

    \begin{proof}
      Since $F$ is real-closed, $a$ must be a sum of squares or $-a$
      is a sum of squares (Corollary 13.16). Assume $a$ is a sum of squares.
      Then if $a\neq0$, $-a$ cannot be a sum of squares: Otherwise
      we have $a=\sum_{i=1}^n x_i^2$ and $-a=\sum_{j=1}^m y_j^2$, which
      gives $\sum_{i=1}^n x_i^2+ \sum_{j=1}^m y_j^2=0$, which can only be
      true if $x_i=y_j=0$ because if $x_i>0$ or $y_j>0$, then $x_i^2>0$ or
      $y_j^2>0$, and then $\sum_{i=1}^n x_i^2+ \sum_{j=1}^m y_j^2>0$. \\

      Then $F(\sqrt{a})$ will be formally real (Lemma 13.15). But since $F$
      is real-closed, $F(\sqrt{a})$ cannot properly extend $F$, which
      implies that $\sqrt{a}$ is contained in $F$, and thus $a$ is a square
      in $F$. On the other hand, if $-a$ is a sum of squares, then
      exchanging the roles of $a$ and $-a$ in the above argument, we will
      get $-a$ is a square in $F$.
    \end{proof}

  \item \it Prove that the field $\mathbb{Q}(t)$ can be ordered so that
    $t>q$ for any $q\in\mathbb{Q}$.

    \begin{proof}
      Let $\mathcal{L}=\{+,-,\cdot,0,1\}$ be the language of fields, and
      $\mathcal{L}'=\{+,-,\cdot,0,1,<\}$ be the extended language of ordered
      fields. Write $T_{\mathbb{Q}(t),\mathcal{L}}$ as the complete
      diagram of the field $\mathbb{Q}(t)$ in the language
      of fields $\mathcal{L}$. Let $T_{\text{lin ring}}$ be the sentences of
      linear-orderings in rings, i.e.
      \begin{align*}
        T_{\text{lin ring}}:=\{ &\forall x,y\; (x\neq y\rightarrow x<y\vee
          y<x) \;\; \wedge \\
        &\forall x,y,z\; (x<y\wedge y<z\rightarrow x<z) \;\; \wedge \\
        &\forall x,y,z\; (x<y\rightarrow x\not>y) \;\; \wedge \\
        &\forall x,y,z\; (x<y\rightarrow x+z<y+z) \;\; \wedge \\
        &\forall x,y,z\; (x<y\wedge z>0\rightarrow xz<yz\wedge zx<zy)\}. \\
      \end{align*}
      Finally, let $T_{t>\mathbb{Q}}:= \{t>q:q\in\mathbb{Q}\}$. \\

      We show that $T_{\mathbb{Q}(t),\mathcal{L}}\cup
      T_{\text{lin ring}}\cup T_{t>\mathbb{Q}}$ is finitely satisfiable: Given
      any finite subset of formulas $\{t>q_1,\ldots,t>q_n\}$ in
      $T_{t>\mathbb{Q}}$, choose any positive irrational
      $r\in\mathbb{R}^+\setminus\mathbb{Q}^+$ with absolute value larger
      than the absolute values of the $q_i$'s, i.e.
      $r>\max(|q_1|,\ldots,|q_n|)$. Since there are uncountably many
      irrationals, such $r$ must exist. In particular, we can choose
      $r=\sqrt{(\max(|q_1|,\ldots,|q_n|))^2+1}$. Then by assigning $r$ to
      $t$, $\mathbb{Q}(r)$ would satisfy $T_{\mathbb{Q}(t),\mathcal{L}}\cup
      T_{\text{lin ring}}\cup \{t>q_1,\ldots,t>q_n\}$. \\

      Thus from compactness theorem, $T:= T_{\mathbb{Q}(t),\mathcal{L}}\cup
      T_{\text{lin ring}}\cup T_{t>\mathbb{Q}}$ is satisfiable. Any model
      that satisfies $T$ must embed $\mathbb{Q}(t)$ elementarily as a field
      since this model satisfies the complete diagram of $\mathbb{Q}(t)$ in
      the language $\mathcal{L}$ of fields. Also, since a subset of a set
      that satisfies $T_{\text{lin ring}}$ will also satisfy $T_{\text{lin
      ring}}$, the embedded $\mathbb{Q}(t)$ will be an ordered field.
      Finally, this embedded $\mathbb{Q}(t)$ will contain $t$ that
      satisfies $T_{t>\mathbb{Q}}$, and thus have an ordering where $t$ is
      larger than all rationals in $\mathbb{Q}$.
    \end{proof}

  \item \it Let $F$ be a field such that $-1$ is not a sum of two squares
    and for any $a\in F$ either $a$ or $-a$ is a square. Show that $F$ is
    formally real.

    \begin{proof}
      Let $P:= \{a\in F:a\; \text{is a square}\}$. We show that $P$ is a
      semi-positive cone that satisfies $P\cup(-P)=F$:

      Thus defining the ordering $a>b\Leftrightarrow (a\neq b \wedge a-b\in
      P)$ would give a field ordering of $F$ (Claim 13.11), which is
      equivalent to $F$ being formally real (Theorem 13.12).
    \end{proof}
\end{enumerate}
\end{document}
