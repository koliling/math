\documentclass{article}
\usepackage[left=3cm,right=3cm,top=3cm,bottom=3cm]{geometry}
\usepackage{amsmath,amssymb,amsthm,tikz,mathtools}
\usepackage{color}
\usepackage[inline]{enumitem}
\usetikzlibrary{patterns}
\setlength{\parindent}{0mm}
\newcommand{\TODO}[1]{\textcolor{red}{TODO: #1}}

\begin{document}
\title{Basic Logic I: Homework 12}
\author{Li Ling Ko\\ lko@nd.edu}
\date{\today}
\maketitle

\begin{enumerate}[label={\bf Q\arabic*:}]
  \item \it Prove Claim 13.11 from the notes: Let $F$ be a field and
    $P\subseteq F$ a semi-positive cone satisfying $P\cup(-P)=F$. Then $F$
    is an ordered field with the ordering $a>b$ defined as $a-b\in P$ and
    $a-b\neq0$. 

    \begin{proof}
      We check that the given ordering satisfies the axioms of linear
      order and the other ordering relations of an ordered field. \\

      Total order: The order is clearly total because $a-b$ is either in
      $P$ or not in $P$. \\

      Anti-symmetry: If $ajb$, then $a-b\in P$ and $a\neq b$, which implies
      $-(b-a)\notin P$ ($\because P\cap -P=0$), and thus $a\not<b$. \\

      Transitivity: If $a>b$ and $b>c$, then $a-b,b-c\in P$ and $a\neq b$
      and $b\neq c$, which implies $a-c=(a-b)+(b-c)\in P$ (by closure of
      $P$ under addition), and so $a\geq c$. \\

      $a>b\Rightarrow a+c>b+c$: If $a>b$, then $a-b\in P$ and $a\neq b$,
      which means $(a+c)-(b+c)=a-b\in P$ and $a+c\neq b+c$, which means
      $a+c>b+c$. \\

      $a>b\wedge c>0\Rightarrow ac>bc\wedge ca>cb$: If $a>b$ and $c>0$,
      then $a-b,c\in P$ and $a\neq b$ and $c\neq0$, so $ac-bc=(a-b)c\in P$
      (by closure under multiplication) and $ac-bc=(a-b)c\neq0$ (since $F$
      is an integral domain), and thus $ac>bc$. A similar argument shows
      that $ca>cb$.
    \end{proof}
\end{enumerate}
\end{document}
