\documentclass{article}
\usepackage[left=3cm,right=3cm,top=3cm,bottom=3cm]{geometry}
\usepackage{amsmath,amssymb,amsthm,tikz,mathtools}
\usepackage{stmaryrd} % For double square bracket [[]]
\usepackage{bm} % For bold vectors
\usepackage{color}
\usepackage{cancel} % To use the cancel function
\usepackage[inline]{enumitem}
\usetikzlibrary{patterns}
\setlength{\parindent}{0mm}
\newcommand{\TODO}[1]{\textcolor{red}{TODO: #1}}

\begin{document}
\title{Model Theory of Valued Fields: Homework}
\author{Li Ling Ko\\ lko@nd.edu}
\date{\today}
\maketitle

\it \textbf{Q1.} Find $J_X(T)$ for the variety defined by $xy=0$. (I
worked with Justin Miller on this question)
\begin{proof}
  Fix $m\in\mathbb{N}$. We want to find $\mathcal{L}_m(X)$. Let $\langle x,
  y\rangle$ be in $X(\mathbb{C}^{2m})$. Write
  \begin{align*}
    x &=a_0+a_1t+\ldots+a_mt^m\\
    y &=b_0+b_1t+\ldots+b_mt^m.\\
  \end{align*}
  Then
  \begin{equation}
    \label{main}
    (a_0+a_1t+\ldots+a_mt^m) (b_0+b_1t+\ldots+b_mt^m)=0 \mod t^{m+1}.
  \end{equation}

  Observe that the solutions of the above formula can be split into $(m+2)$
  disjoint cases -- in the first case, we consider the solutions when
  $x=0$. Then $y$ can be any value in $\mathbb{C}^{m+1}$, so the space of
  solutions is definably bijective to 
  \[\mathbb{C}^{m+1} \equiv \mathbb{L}^{m+1}.\]

  For the remaining $(m+1)$ cases, in the $i$-th case, where
  $i\in\{0,\ldots,m\}$, we consider solutions of equation~\eqref{main} when
  $i$ is the smallest power of $t$ where the coefficient $a_i$ of $t^i$ is
  non-zero. The space of $x$'s of such form is
  \[(\mathbb{C}-\{0\}) \times\mathbb{C}^{m-i} \equiv
  \mathbb{L}^{m-i+1}-\mathbb{L}^{m-i}.\]

  For each $x$ in this space, $y$ can be any formula in
  $\mathbb{C}[t]/t^{m+1}$ as long as its lowest power of $t$ with non-zero
  coefficient is greater or equal to than $(m-i+1)$. Thus the space of
  possible $y$'s for each such $x$ is of the form
  \[\mathbb{C}^{i} \equiv \mathbb{L}^{i}.\]
  This gives a total space of
  \[(\mathbb{L}^{m-i+1}-\mathbb{L}^{m-i}) \times\mathbb{L}^{i} \equiv
  \mathbb{L}^{m+1}-\mathbb{L}^{m},\]
  for each $i\in\{0,\ldots,m\}$. \\

  Adding up the $(m+2)$ disjoint cases,
  \begin{align*}
    [\mathcal{L}_m(X)] &=\mathbb{L}^{m+1}
      +(m+1)(\mathbb{L}^{m+1}-\mathbb{L}^{m})\\
      &=(2\mathbb{L}-1)\mathbb{L}^m +(\mathbb{L}-1)m\mathbb{L}^m.\\
  \end{align*}

  Using the series $\displaystyle\sum_{m=0}^\infty S^m =\frac{1}{1-S}$
  and $\displaystyle\sum_{m=0}^\infty mS^m =\frac{S}{(1-S)^2}$, we simplify
  \begin{align*}
    J_X(T) &:=\sum_{m=0}^\infty [\mathcal{L}_m(X)]T^m\\
      &=\sum_{m=0}^\infty [(2\mathbb{L}-1)\mathbb{L}^m
        +(\mathbb{L}-1)m\mathbb{L}^m] T^m\\
      &=(2\mathbb{L}-1) \sum_{m=0}^\infty (\mathbb{L}T)^m +(\mathbb{L}-1)
        \sum_{m=0}^\infty m(\mathbb{L}T)^m\\
      &=\frac{2\mathbb{L}-1}{1-\mathbb{L}T}
        +\frac{(\mathbb{L}-1)\mathbb{L}T}{(1-\mathbb{L}T)^2}\\
      &=\frac{2\mathbb{L}-\mathbb{L}^2T-1}{(1-\mathbb{L}T)^2}.\\
  \end{align*}
\end{proof}

\it \textbf{Q2.} Find $J_X(T)$ for the variety defined by $x^3=y^2$. (I
received significant help from Greg, Kyle, Rachael, Nicolas, and Leo on
this question)
\begin{proof}
  We attempt to find a recursive relation between $[\mathcal{L}_{m+1}(X)]$
  and $[\mathcal{L}_{m}(X)]$. To get $[\mathcal{L}_{m+1}(X)]$ given
  $[\mathcal{L}_{m}(X)]$, we shall apply a theorem of algebraic geometry,
  which says that the space of $X(\mathbb{C}[t]/t^{m+1})$ is the space of
  tangent bundles of $X(\mathbb{C}[t]/t^{m})$. \\

  In the base case when $m=0$, $X(\mathbb{C}) \subseteq\mathbb{C}^2$
  is the solutions of the equation
  \[x_0^3=y_0^2.\]
  Observe that the map $\mathbb{C}\rightarrow X$ given by $t\mapsto\langle
  t^2,t^3\rangle$ is a definable bijection, therefore
  \[[\mathcal{L}_0(X)] =[\mathcal{L}_0(\mathcal{C})] =\mathbb{L}.\]

  To find $[\mathcal{L}_{1}(X)]$, the theorem says $X(\mathbb{C}[t]/t)
  \subseteq\mathbb{C}^4$ is the space of tangent bundles of
  $X(\mathbb{C})\subseteq\mathbb{C}^2$. Thus we differentiate the
  equation $x_0^3=y_0^2$ and write $x_0'$ as $x_1$ and $y_0'$ as $y_1$, and
  get $X(\mathbb{C}[t]/t) \subseteq\mathbb{C}^4$ as solutions of the
  equations
  \[x_0^3=y_0^2\;\;\; \text{and}\;\;\; 3x_0^2x_1=2y_0y_1.\]
  The solutions can be split into two disjoint cases - in the first case,
  consider when $x_0=0$; then $y_0=0$ and $y_1$ can be any value in
  $\mathbb{C}$. So for each of the $\mathbb{C}$ possibilities of $x$, we
  have $\mathbb{C}$ possibilities for $y$, giving a solution space
  bijective to $\mathbb{L}^2$. In the second case, when $x_0\neq0$, there
  are exactly two possibilities for $y_0$, and only one possibility for
  $y_1$. So for each of the $(\mathbb{C}-1)\mathbb{C}$ possibilities of
  $x$, we have $2$ possibilities for $y$, giving a solution space bijective
  to $2(\mathbb{L}-1)\mathbb{L}$. Adding the disjoint spaces, we get
  \[\mathcal{L}_1(X) =\mathbb{L}^2 +2(\mathbb{L}-1)\mathbb{L}
  =3\mathbb{L}^2-2\mathbb{L}.\]

  Don't know how to continue...
\end{proof}
\end{document}
