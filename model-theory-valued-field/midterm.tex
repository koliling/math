\documentclass{article}
\usepackage[left=3cm,right=3cm,top=3cm,bottom=3cm]{geometry}
\usepackage{amsmath,amssymb,amsthm,tikz,mathtools}
\usepackage{stmaryrd} % For double square bracket [[]]
\usepackage{bm} % For bold vectors
\usepackage{color}
\usepackage{cancel} % To use the cancel function
\usepackage[inline]{enumitem}
\usetikzlibrary{patterns}
\setlength{\parindent}{0mm}
\newcommand{\TODO}[1]{\textcolor{red}{TODO: #1}}

\begin{document}
\title{Model Theory of Valued Fields: Midterm}
\author{Li Ling Ko\\ lko@nd.edu}
\date{\today}
\maketitle

\textbf{Exercise 12:} \it Show that for local rings $R_0\sqsubseteq R$ we
  have $\bm{m}_0=\bm{m}\cap R_0$.

  \begin{proof}
    We are assuming that $R_0$ is a subring of $R$. \\

    $\bm{m}_0\subseteq\bm{m}\cap R_0$ follows by definition of
    $R_0\sqsubseteq R$. For the reverse containment, since $R_0$ is local,
    it suffices to show that $\bm{m}\cap R_0$ is an ideal of $R_0$, and
    that $\bm{m}\cap R_0$ is not equal to $R_0$. To prove the second
    claim, observe that $R_0$ cannot be contained in contain $\bm{m}$,
    because as a subring of $R_0$, $R_0$ must contain the identity of $R$,
    which does not lie in $\bm{m}$ since $\bm{m}$ is a non-trivial ideal of
    $R$. Thus $\bm{m}\cap R_0\neq R_0$. \\ 

    To prove the first claim, observe that as rings, both $\bm{m}$
    and $R_0$ are closed under subtraction. As an ideal of $R$, $\bm{m}$ is
    closed under multiplication with elements of $R_0$. $R_0$ is also
    clearly closed under multiplication with elements of $R_0$. Thus
    $\bm{m}\cap R_0$ is an ideal of $R_0$.
  \end{proof}
\end{document}
