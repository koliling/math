\documentclass{article}
\usepackage[left=3cm,right=3cm,top=3cm,bottom=3cm]{geometry}
\usepackage{amsmath,amssymb,amsthm,tikz,mathtools}
\usepackage{stmaryrd} % For double square bracket [[]]
\usepackage{bm} % For bold vectors
\usepackage{color}
\usepackage{cancel} % To use the cancel function
\usepackage[inline]{enumitem}
\usetikzlibrary{patterns}
\setlength{\parindent}{0mm}
\newcommand{\TODO}[1]{\textcolor{red}{TODO: #1}}

\begin{document}
\title{Model Theory of Valued Fields: Midterm}
\author{Li Ling Ko\\ lko@nd.edu}
\date{\today}
\maketitle

\textbf{Exercise 12:} \it Show that for local rings $R_0\sqsubseteq R$ we
  have $\bm{m}_0=\bm{m}\cap R_0$.

  \begin{proof}
    We are assuming that $R_0$ is a subring of $R$. \\

    $\bm{m}_0\subseteq\bm{m}\cap R_0$ follows by definition of
    $R_0\sqsubseteq R$. For the reverse containment, since $R_0$ is local,
    it suffices to show that $\bm{m}\cap R_0$ is an ideal of $R_0$, and
    that $\bm{m}\cap R_0$ is not equal to $R_0$. To prove the second
    claim, observe that $R_0$ cannot be contained in contain $\bm{m}$,
    because as a subring of $R_0$, $R_0$ must contain the identity of $R$,
    which does not lie in $\bm{m}$ since $\bm{m}$ is a non-trivial ideal of
    $R$. Thus $\bm{m}\cap R_0\neq R_0$. \\ 

    To prove the first claim, observe that as rings, both $\bm{m}$
    and $R_0$ are closed under subtraction. As an ideal of $R$, $\bm{m}$ is
    closed under multiplication with elements of $R_0$. $R_0$ is also
    clearly closed under multiplication with elements of $R_0$. Thus
    $\bm{m}\cap R_0$ is an ideal of $R_0$.
  \end{proof}

\textbf{Q5:} \it Let $R<K$ be domains and every $k\in K$ is integral over
  $R$. If $K$ is a field, show that $R$ is a field as well.
  \begin{proof}
    Let $r\in R^*$. We want to show that $r^{-1}\in F$ also lies in $R$.
    Since $r^{-1}$ is integral over $R$, there exists
    $r_0,\ldots,r_{n-1}\in R$ such that
    \[r^{-n}+r_{n-1}r^{-n+1} +\ldots +r_1r^{-1}+r_0=0.\]
    Rearranging, we get
    \[r(-r_0r^{n-1} -r_1r^{n-2} -\ldots -r_{n-2}r -r_{n-1}) = 1.\]
    So
    \[-r_0r^{n-1} -r_1r^{n-2} -\ldots -r_{n-2}r -r_{n-1} \in R\]
    is the multiplicative inverse of $r$, and it lies in $R$, as required.
  \end{proof}

\textbf{Q7:} \it (Corollary 9.6) For a fixed prime $p$ the theory of
  $p$-adically closed fields in the language $\mathcal{L}_v$ is complete
  and model complete.

  \begin{proof}
    Since $T_p^c$ has QE in the expanded language $\mathcal{L}_p^c$
    (Theorem 9.4), and therefore is model-complete with respect to
    $\mathcal{L}_p^c$. Since no new axioms were added in the expanded
    language, the models of $T_p^c$ in the original language are the same
    as the models in the expanded one. Thus $T_p^c$ is model-complete. \\

    To prove completeness, fix an arbitrary model $\mathcal{M}$ of $T_p^c$.
    Then $\mathcal{M}$ must embed $\mathbb{Q}$ as a valued field with
    $p$-adic valuation, and thus also embeds $\mathbb{Q}_p^h$, the
    hensalization of $\mathbb{Q}$. So $\mathbb{Q}_p^h$ is a substructure of
    $\mathcal{M}$, thus by model-completeness, is an elementary
    substructure of $\mathcal{M}$. In particular, $\mathbb{Q}_p^h$ and
    $\mathcal{M}$ are elementary equivalent.  Since $\mathcal{M}$ is an
    arbitrary model of $T_p^c$, $T_p^c$ is the complete theory of
    $\mathbb{Q}_p^h$.
  \end{proof}
\end{document}
