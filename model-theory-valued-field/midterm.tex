\documentclass{article}
\usepackage[left=3cm,right=3cm,top=3cm,bottom=3cm]{geometry}
\usepackage{amsmath,amssymb,amsthm,tikz,mathtools}
\usepackage{stmaryrd} % For double square bracket [[]]
\usepackage{bm} % For bold vectors
\usepackage{color}
\usepackage{cancel} % To use the cancel function
\usepackage[inline]{enumitem}
\usetikzlibrary{patterns}
\setlength{\parindent}{0mm}
\newcommand{\TODO}[1]{\textcolor{red}{TODO: #1}}

\begin{document}
\title{Model Theory of Valued Fields: Midterm}
\author{Li Ling Ko\\ lko@nd.edu}
\date{\today}
\maketitle

\textbf{Exercise 12:} \it Show that for local rings $R_0\sqsubseteq R$ we
  have $\bm{m}_0=\bm{m}\cap R_0$.

  \begin{proof}
    We are assuming that $R_0$ is a subring of $R$. \\

    $\bm{m}_0\subseteq\bm{m}\cap R_0$ follows by definition of
    $R_0\sqsubseteq R$. For the reverse containment, since $R_0$ is local,
    it suffices to show that $\bm{m}\cap R_0$ is an ideal of $R_0$, and
    that $\bm{m}\cap R_0$ is not equal to $R_0$. To prove the second
    claim, observe that $R_0$ cannot be contained in contain $\bm{m}$,
    because as a subring of $R_0$, $R_0$ must contain the identity of $R$,
    which does not lie in $\bm{m}$ since $\bm{m}$ is a non-trivial ideal of
    $R$. Thus $\bm{m}\cap R_0\neq R_0$. \\ 

    To prove the first claim, observe that as rings, both $\bm{m}$
    and $R_0$ are closed under subtraction. As an ideal of $R$, $\bm{m}$ is
    closed under multiplication with elements of $R_0$. $R_0$ is also
    clearly closed under multiplication with elements of $R_0$. Thus
    $\bm{m}\cap R_0$ is an ideal of $R_0$.
  \end{proof}

\textbf{Q3:} \it (Lemma 7.19) Let $(R,F)$ be a valued field of
  characteristic 0. Let $P[x]\in F[x]$ and $a\in F$ be in henselian
  configuration with $F(a)\neq0$. There is at most one $b\in F$ with
  $P(b)=0$ and $v(a-b)\geq v(P(a))-v(P'(a))$. If $F$ is henselian then
  there is $b\in F$ with $F(b)=0$ and $v(a-b)=v(P(a))-v(P'(a))$.

  \begin{proof}
  \end{proof}

\textbf{Q4:} \it (Exercise 18) If $(R,F)$ is finitely ramified then for any
  $k\in\mathbb{N}$ the set $\{\gamma\in\Gamma: 0\leq\gamma<v(k)\}$ is finite.

  \begin{proof}
    Since $R/\bm{m}$ is a field of characteristic $p$, the set
    \[\{\bm{m},1+\bm{m},\ldots,p-1+\bm{m}\}\]
    is an additive subgroup of $R/\bm{m}$ generated by $1+\bm{m}$.
    Since $v(k)=0$ for all $k\in i+\bm{m}$ where $i\in\{1,\ldots,p-1\}$,
    so if $p$ does not divide $k$, $v(k)$ must be 0, making the set
    $\{\gamma\in\Gamma: 0\leq\gamma<v(k)\}$ trivially finite. \\

    It remains to prove the claim for $k=pk'$. Now $v(pk')=v(p)+v(k')$. So
    if $k'$ is not a multiple of $p$, the claim would hold since $(R,F)$ is
    finitely ramified. Thus we only need consider the cases when $k=p^n$
    for some $n>1$. In these cases we have $v(p^n)=n\cdot v(p)$. Now if
    \[0\leq x\leq n\cdot v(p),\]
    then
    \[-v(p)\leq x-v(p)\leq (n-1)\cdot v(p).\]
    By induction on $n$, there are only finitely many elements between $0$
    and $(n-1)\cdot v(p)$, and by the base case there are also only
    finitely many elements between $-v(p)$ and $0$. Therefore there are
    only finitely many possibilities for $x-v(p)$, implying there are only
    finitely many $x$ that can lie between 0 and $n\cdot v(p)$, which
    completes the proof.
  \end{proof}

\textbf{Q5:} \it Let $R<K$ be domains and every $k\in K$ is integral over
  $R$. If $K$ is a field, show that $R$ is a field as well.
  \begin{proof}
    Let $r\in R^*$. We want to show that $r^{-1}\in F$ also lies in $R$.
    Since $r^{-1}$ is integral over $R$, there exists
    $r_0,\ldots,r_{n-1}\in R$ such that
    \[r^{-n}+r_{n-1}r^{-n+1} +\ldots +r_1r^{-1}+r_0=0.\]
    Rearranging, we get
    \[r(-r_0r^{n-1} -r_1r^{n-2} -\ldots -r_{n-2}r -r_{n-1}) = 1.\]
    So
    \[-r_0r^{n-1} -r_1r^{n-2} -\ldots -r_{n-2}r -r_{n-1} \in R\]
    is the multiplicative inverse of $r$, and it lies in $R$, as required.
  \end{proof}

\textbf{Q6:} \it Show that $\mathbb{Q}_p$ does not have proper immediate
  extensions.
  \begin{proof}
    By Theorem 7.15, it suffices to show that $\mathbb{Q}_p$ is spherically
    complete. Let $B_{\geq n}(a)$ be a ball in $\mathbb{Q}_p$, where
    $n\in\mathbb{Z}$ and $a=p^{-d}(b_i)$ for some $(b_i)\in\mathbb{Z}_p$
    and $d\in\mathbb{N}$. Observe that elements in $B_{\geq n}(a)$ are
    exactly those of the form $p^{-d}(c_i)$, where $c_i=b_i$ for the first
    $(n+d)$ indices. Thus $B_{\geq n}(a)=B_{\geq n}(a')$, where
    $a=p^{-d}(b_i')$, and $b_i'$ is defined as
    \begin{equation*}
      b_i':=
      \begin{cases}
        b_i, &\text{if}, i\in\{0,1,\ldots,n+d-1\},\\
        0, &\text{otherwise}.\\
      \end{cases}
    \end{equation*}

    Therefore for any ball $B_{\geq n}(p^{-d}(b_i))$, we can assume $b_i$
    is zero for all indices after the first $n+d$ ones. Since there are
    only countably many balls of such form, any chain of balls must be
    countable. Observe further that given two balls $B_{\geq
    n_0}(p^{d_0}(a_i))$ and $B_{\geq n_1}(p^{d_1}(b_i))$, the first ball
    contains the second if and only $d_0=d_1$, $n_0\leq n_1$, and $a_i=b_i$
    for the first $n_0+d_0$ indices. \\

    Thus, given a chain $\mathcal{B}=\{B_i:i\in\omega\}$ of balls, we can
    assume $B_i=B_{\geq n_i}(p^d(a_{i,j}))$, where $d\in\mathbb{Z}$ is
    fixed, $n_i\geq n_{i+1}$, $a_{i,j}=0$ for all $j\geq n_i+d$, and
    $a_{i,j}=a_{i+1,j}$ for the first $n_i+d$ indices. Then the element
    $p^d(a_j)\in\mathbb{Q}_p$ where
    \[a_j :=\lim_{i\in\omega} a_{i,j},\]
    will be contained in every ball in $\mathcal{B}$, which completes the
    proof.
  \end{proof}

\textbf{Q7:} \it (Corollary 9.6) For a fixed prime $p$ the theory of
  $p$-adically closed fields in the language $\mathcal{L}_v$ is complete
  and model complete.

  \begin{proof}
    Since $T_p^c$ has QE in the expanded language $\mathcal{L}_p^c$
    (Theorem 9.4), it is model-complete with respect to
    $\mathcal{L}_p^c$. Since no new axioms were added in the expanded
    language, the models of $T_p^c$ in the original language are the same
    as the models in the expanded one. Thus $T_p^c$ is also model-complete
    in the original language. \\

    To prove completeness, fix an arbitrary model $\mathcal{M}$ of $T_p^c$.
    Then $\mathcal{M}$ must embed $\mathbb{Q}$ as a valued field with
    $p$-adic valuation, and thus also embeds $\mathbb{Q}_p^h$, the
    hensalization of $\mathbb{Q}$ in $\mathbb{Q}_p\cap\bar{\mathbb{Q}}$. So
    $\mathbb{Q}_p^h$ is a substructure of $\mathcal{M}$, thus by
    model-completeness, is an elementary substructure of $\mathcal{M}$. In
    particular, $\mathbb{Q}_p^h$ and $\mathcal{M}$ are elementary
    equivalent. Since $\mathcal{M}$ is an arbitrary model of $T_p^c$,
    $T_p^c$ is the complete theory of $\mathbb{Q}_p^h$.
  \end{proof}
\end{document}
