\documentclass{article}
\usepackage[left=3cm,right=3cm,top=3cm,bottom=3cm]{geometry}
\usepackage{amsmath,amssymb,amsthm}
\usepackage{color}
%\setlength{\parindent}{0mm}
\newcommand{\TODO}[1]{\textcolor{red}{TODO: #1}}

\begin{document}
\title{Basic Logic I: Homework Set 5}
\author{Li Ling Ko\\ lko@nd.edu}
\date{\today}
\maketitle

\begin{enumerate}
  \item Prove that a class $\mathcal{K}$ is finitely axiomatizable if both
    $\mathcal{K}$ and its complement $\mathcal{K}^c$ are axiomatizable.
    \begin{proof}
      Let $T$ and $T^c$ be the set of sentences that axiomatize
      $\mathcal{K}$ and $\mathcal{K}^c$ respectively. Assume by
      contradiction that $\mathcal{K}$ is not finitely axiomatizable.
      Consider $S=T\cup T^c$. We show that $T$ is finitely satisfiable:
      Given any finite subset $A=A_0\cup A_1\subset S$, where $A_0\in T$ and
      $A_1\in T^c$, there exists models $\mathcal{M}\in\mathcal{K}^c$ that
      satisfy $A_0$ because $A_0$ does not axiomatize $\mathcal{K}$. Such
      models will also satisfy $A_1$ since they satisfy $T^c$, hence these
      models will satisfy $A$.

      Thus by compactness theorem $T$ is satisfiable and should have a
      model, which contradicts $\mathcal{K}\cap\mathcal{K}^c=\emptyset$.
    \end{proof}

  \item Let $\mathcal{L}_g$ be the language of groups, and let $G$ be the
    group $\mathbb{Z}_2\times\mathbb{Z}_2$. Let $\mathcal{U}$ be an
    ultrafilter on $\omega$. Describe the group $H=G^\omega/\mathcal{U}$.

    \begin{proof}
      From Claim 4.19, we know that ultrapowers $\mathcal{N}^I/\mathcal{U}$
      are elementarily equivalent to $\mathcal{N}$. Hence $H$ is
      elementarily equivalent to $G$ in the language of groups. From
      Homework 3 Question 2, we have shown that $G$ is axiomatizable. So
      since the $H$ satisfies the same axioms as $G$, $H$ must also be
      isomorphic to $G$, the group $\mathbb{Z}_2\times\mathbb{Z}_2$.
    \end{proof}

  \item Let $\mathcal{N}=\langle\mathbb{N},+,\cdot,0,1\rangle$ and let
    $\mathcal{N}^*=\mathcal{N}^\omega/\mathcal{U}$, where $\mathcal{U}$ is
    a non-principal ultrafilter on $\omega$. Using Claim 4.19 we may view
    $\mathcal{N}$ as an elementary substructure of $\mathcal{N}^*$. Show
    that there is an element $\alpha\in\mathbb{N}^*$ that is divisible by
    every prime $p\in\mathbb{N}$.

    \begin{proof}
      Let $\alpha=\overline{\sigma}\in\mathbb{N}^*$, where
      $\sigma\in\mathbb{N}^\omega$ is defined by $\sigma(i)=i!$ for
      $i\in\omega$ (taking 0!=1).

      We show that $\alpha$ is divisible by every prime in $\mathbb{N}$.
      Given a prime $p\in\mathbb{N}$, first notice that $p$ as an element
      in $\mathcal{N}^*$ can be represented by
      $\alpha_p=\overline{\sigma_p}\in\mathcal{N}^*$, where
      $\sigma_p\in\mathbb{N}^\omega$ is defined as:
      \begin{equation*}
        \sigma_p(i) :=
        \begin{cases}
          1 & \text{if}\; i<p \\
          p & \text{if}\; i\geq p
        \end{cases}.
      \end{equation*}
      Then $\overline{\sigma_p}$ is equivalent to $\alpha_p$ in
      $\mathcal{N}^*$ because $\mathcal{U}$ contains the Frechet filter
      (since $\mathcal{U}$ is not principal, and we have proven in earlier
      homework that non-principal filters must contain the Frechet filter),
      and $\sigma_p(i)$ is only not $p$ for a finite number of $i$'s in
      $\omega$. Also, $\alpha=\overline{\sigma}$ divides
      $\alpha_p=\overline{\sigma_p}$ because $\sigma(i)$ divides
      $\sigma_p(i)$ for each $i\in\omega$. Hence $\alpha$ is a multiple of
      $\alpha_p$ for each prime $p$ in $\mathbb{N}$.
    \end{proof}

  \item Let $\mathcal{L}$ be a language and $\mathcal{K}$ be a class of
    $\mathcal{L}$-structures. As usual let
    \begin{equation*}
      \text{Th}(\mathcal{K}) = \{\varphi:\; \varphi\; \text{is an}\;
        \mathcal{L}\text{-sentence},\; \mathcal{M}\models\varphi\; \text{for
        all}\; \mathcal{M}\in\mathcal{K}\}.
    \end{equation*}
    Show that $\mathcal{M}$ is a model of $\text{Th}(\mathcal{K})$ if and
    only if $\mathcal{M}$ is elementarily equivalent to an ultraproduct of
    elements of $\mathcal{K}$.

    \begin{proof}
      $\Rightarrow$: This is the easier direction. Assume that $\mathcal{M}$
      is elementarily equivalent to an ultraproduct of elements of
      $\mathcal{K}$. Let $\varphi$ be any sentence in
      $\text{Th}(\mathcal{K})$. We wish to show that $\mathcal{M}$
      satisfies $\varphi$. Now each model in $\mathcal{K}$ satisfies
      $\varphi$, so by Los's theorem, the ultraproduct of elements of
      $\mathcal{K}$ also satisfies $\varphi$. Then since $\mathcal{M}$ is
      elementarily equivalent to this ultraproduct, $\mathcal{M}$ also
      satisfies $\varphi$. \\

      $\Leftarrow$: Let $\mathcal{M}\models\text{Th}(\mathcal{K})$. We
      first show that given any finite theory $T_0$, if $\mathcal{M}\models
      T_0$, then $\mathcal{N}\models T_0$ for some
      $\mathcal{N}\in\mathcal{K}$: Let $\varphi_0$ be the conjunction of
      the sentences in $T_0$. If no model in $\mathcal{K}$ satisfies
      $\varphi_0$, then all models in $\mathcal{K}$ will satisfy
      $\neg\varphi_0$, and so $\neg\varphi_0\in\text{Th}(\mathcal{K})$,
      which implies that $\mathcal{M}\models\neg\varphi_0$, a
      contradiction. \\

      Now similar to the proof of the compactness theorem via ultrafilters,
      we build an ultraproduct $\mathcal{N}$ of structures from
      $\mathcal{K}$ such that $\mathcal{N}\models\text{Th}(\mathcal{M})$:
      Let $I$ be the set of all finite subsets of $\text{Th}(\mathcal{M})$.
      For $i\in I$, let $i^*$ be the set of all elements in $I$ that
      contain $i$. Then $B=\{i^*:i\in I\}$ is a filter base on $I$, hence
      there is an ultrafilter $\mathcal{U}$ on $I$ containing all elements
      of $B$. For each $i\in I$, let $\mathcal{N}_i$ be the model in
      $\mathcal{K}$ that satisfies the finite set of sentences in
      $\text{Th}(\mathcal{M})$.  $\mathcal{N}_i$ exists from the argument
      in the preceding paragraph.  \\

      Consider $\mathcal{N}:=\prod_i\mathcal{N}_i/\mathcal{U}$. We show
      that $\mathcal{N}\models\text{Th}(\mathcal{M})$. Given any sentence
      $\varphi\in\text{Th}(\mathcal{M})$, let $i$ be the index in $I$ that
      is associated with the set $\{\varphi\}$. Then for all indices $j\in
      i^*$, model $\mathcal{N}_j$ will satisfy $\varphi$. Hence the indices
      in $I$ whose model satisfies $\varphi$ is a supserset of $i^*$, which
      is contained in $\mathcal{U}$. So by Los's theorem, $\mathcal{N}$ also
      satisfies $\varphi$, as required.
    \end{proof}

  \item Let $\mathcal{L}$ be a language and $S$ be the set of all
    consistent complete $\mathcal{L}$-theories. For an
    $\mathcal{L}$-sentence $\varphi$ let $\mathcal{O}_\varphi=\{T\in
    S:\varphi\in T\}$. Consider $S$ as a topological space with
    $\mathcal{O}_\varphi$ as a basis for the topology. Show that $S$ with
    this topology is a compact Hausdorff space.

    \begin{proof}
      We first show the space is Hausdorff. Let $T_0$ and $T_1$ be distinct
      theories in $S$. Since they are distinct, there must be a sentence
      $\varphi$ in $T_0$ that is not in $T_1$. Then since $T_1$ is
      complete, if it does not contain $\varphi$ it must contain
      $\neg\varphi$. Then $T_0$ and $T_1$ are contained in open
      sets$\mathcal{O}_\varphi$ and $\mathcal{O}_{\neg\varphi}$
      respectively. These open sets do not intersect because no theory in
      $S$ can contain both $\varphi$ and $\neg\varphi$ from because
      theories in $S$ are consistent. Hence the space is Hausdorff. \\

      Next we show compactness. Let $\Phi=\{\mathcal{O}_\alpha\}_{\alpha\in
      A}$ be an open cover of $S$. We can assume that each open set
      $\mathcal{O}_\alpha$ is of the form $\mathcal{O}_{\varphi}$ for some
      sentence $\varphi_\alpha$ since the $\mathcal{O}_\varphi$ form a
      basis for the topology. Assume by contradiction that no finite
      subset of $\Phi$ covers $S$. This means that for every finite
      subset
      $\{\mathcal{O}_{\alpha_1},\ldots,\mathcal{O}_{\alpha_n}\}\subset\Phi$,
      there is consistent and complete $\mathcal{L}$-theory $T$ that does
      not contain any of $\mathcal{O}_{\alpha_i}$, or equivalently,
      $\neg\mathcal{O}_{\alpha_1}\wedge\ldots\wedge\neg\mathcal{O}_{\alpha_n}$
      is satisfiable. Hence,
      $\neg\Phi:=\{\neg\mathcal{O}_\alpha\}_{\alpha\in A}$ is finitely
      satisfiable, so by compactness theory, has a model $\mathcal{M}$.
      Now $\text{Th}(\mathcal{M})$ is contained in $S$, and this theory
      contains every sentence in $\neg\Phi$. Hence $\text{Th}(\mathcal{M})$
      does not contain any sentence in $\Phi$, therefore $\Phi$
      does not cover $\text{Th}(\mathcal{M})$, contradicting $\Phi$ being
      an open cover of $S$.
    \end{proof}

  \item Show that the class of infinite fields is not finitely
    axiomatizable.
    \begin{proof}
      Assume by contradiction that the class of infinite fields is finitely
      axiomatizable in the language $\mathcal{L}$. By taking conjunction of
      sentences, we can assume without loss of generality that the class is
      axiomatized by a single sentence $\varphi_{\text{inf}}$. Extend the
      language if necessary to include $\{+,-,\cdot,^{-1},0,1\}$, let
      $\varphi_F$ be the sentence which axiomatizes fields, which can be
      defined by the conjunction of the following sentences:
      \begin{align*}
        \forall a,b,c && (a+b)+c=a+(b+c)\; \wedge (a\cdot b)\cdot
          c=a\cdot(b\cdot c)  && (\text{associativity}) \\
        \forall a,b   && a+b=b+a\; \wedge a\cdot b=b\cdot a  &&
          (\text{commutativity}) \\
        \forall a,b,c && a\cdot(b+c)=a\cdot b+a\cdot c\; \wedge (a+b)\cdot
          c=a\cdot c+b\cdot c  && (\text{distributivity}) \\
        \forall a     && a+0=a=0+a\; \wedge a\cdot1=a=1\cdot a  &&
          (\text{identity}) \\
        \forall a     && a+(-a)=0=(-a)+a\; \wedge a\cdot
          a^{-1}=1=a^{-1}\cdot a  && (\text{inverse}) \\
      \end{align*}

      Then $\neg\varphi_{\text{inf}}\wedge\varphi_F$ will axiomatize the
      finite fields. For each $n\in\mathbb{N}^+$, consider the sentence
      $\varphi_n$ which says ``I have at least $n$ distinct elements'',
      defined by
      \begin{align*}
        \varphi_n := \exists x_1,\ldots,x_n\; \bigwedge_{1\leq i<j\leq n}
        \neg x_i=x_j.
      \end{align*}
      Then because arbitrarily large finite fields exist (take
      $\mathbb{F}_p$ for large enough primes $p$), the set of sentences
      $\{\neg\varphi_{\text{inf}}\wedge\varphi_F\}\cup_{n\in\mathbb{N}^+}\varphi_n$
      is finitely satisfiable, and by compactness should be satisfiable by
      a field. However such a field is both finite and infinite, a
      contradiction.
    \end{proof}

  \item Let $\mathcal{M}$, $\mathcal{N}$ be first order
    $\mathcal{L}$-structures with $\mathcal{M}\equiv\mathcal{N}$. Show that
    there is an $\mathcal{L}$-structure $\mathcal{K}$ such that both
    $\mathcal{M}$ and $\mathcal{N}$ can be elementarily embedded into
    $\mathcal{K}$. 

    \begin{proof}
      In the extended language $\mathcal{L}(M\sqcup N)$, where the constants
      for $\mathcal{M}$ are disjoint from the constants for $\mathcal{N}$,
      consider the theory $T=T_\mathcal{M}\cup T_\mathcal{N}$. We show that
      $T$ is finitely satisfiable by $\mathcal{M}$ after reassigning some
      constants of $\mathcal{N}$: Given a finite set of sentences
      $S=S_M\cup S_N\subset T$ where $S_M\subset T_\mathcal{M}$ and
      $S_N\subset T_\mathcal{N}$, we have $\mathcal{M}$ satisfies $S_M$
      because $S_M\subset T_\mathcal{M}$. Write the conjunction of all the
      sentences in $S_N$ as
      \begin{equation*}
        \varphi(c_{n_1},\ldots,c_{n_k})\in T_\mathcal{N},
      \end{equation*}
      where the $c_{n_i}$ are all the constants of $\mathcal{N}$ that appear in
      the conjunction. Then $\mathcal{N}$ satisfies
      \begin{equation*}
        \phi := \exists x_1,\ldots,x_k\; \varphi(x_1,\ldots,x_k),
      \end{equation*}

      with $c_{n_1},\ldots,c_{n_k}$ as witnesses. So from elementarily
      equivalence of $\mathcal{M}$ and $\mathcal{N}$, $\phi$ is also
      satisfied by $\mathcal{M}$. Assigning each $c_{n_i}$ to their respective
      element in $M$ that satisfies $\phi$ will give us a model of $S$. \\

      Therefore from compactness theorem, $T$ has a model $\mathcal{K}$. Then
      from part (iii) of Claim 3.2, since $\mathcal{K}$ satisfies
      $T_\mathcal{M}$ and $T_\mathcal{N}$, both $\mathcal{M}$ and
      $\mathcal{N}$ can be elementarily embedded into $\mathcal{K}$.
    \end{proof}
\end{enumerate}
\end{document}
