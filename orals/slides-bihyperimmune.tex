\begin{frame}{Bi-Hyperimmune sets}
  \begin{define}
    $A\subseteq\omega$ is \textbf{hyperimmune} if there is no recursive
    function $f$ such that $f(n)>A(n)$ for every $n\in\omega$.
  \end{define}

  \begin{define}
    $A$ is \textbf{bi-hyperimmune} if both $A$ and $\bar{A}$ are hyperimmune.
  \end{define}

  \begin{thm}
    Bi-hyperimmune sets exist.
  \end{thm}

  \vspace{0.5em}
  \textbf{Pf:} Enumerate recursive functions $f_0,f_1,\ldots$. At stage
  $s$, let $i$ be the number of elements in $A$ and $\bar{A}$ that has been
  defined so far. Put enough elements into $\bar{A}$ till the $(i+1)$-th
  element of $A$ exceeds $f_s(i+1)$. Repeat with roles of $A$ and $\bar{A}$
  reversed. $\blacksquare$
\end{frame}

\begin{frame}{Class of sets computing subsets of hyperimmune is null}
  \begin{thm}
    \label{thm:bihyper-null}
    Given a hyperimmune set $A$, the class of sets that can compute an
    infinite subset of $A$ is null.
  \end{thm}

  \vspace{1em}
  \textbf{Pf:} The idea is, if the measure is positive, there will be
  enough reals computing subsets of $A$ that one can effectively get them
  to ``vote'' for when new elements of $A$ have appeared, contradicting
  hyperimmunity of $A$.

  \vspace{1em}
  Since there are only countably many Turing functionals, and the union of
  countably many null sets is null, it suffices to show $\{X:
  \Gamma^X\in[A]^\omega\}$ is null, where
  $\Gamma:2^\omega\rightarrow2^\omega$ is an arbitrary Turing functional.
\end{frame}

\begin{frame}{Class of sets computing subsets of hyperimmune is null}
  Write $\mathcal{B} :=\{X: \Gamma^X\in[A]^\omega\}$.
  Assume by contradiction $\mu(\mathcal{B})=4m>0$. Tightly approximate
  $\mathcal{B}$ by open cover $\mathcal{U}\supseteq\mathcal{B}$ so that
  $\mu(\mathcal{U}-\mathcal{B})<m$, then again tightly approximate
  $\mathcal{U}$ by $\llbracket\sigma_0\rrbracket,
  \ldots,\llbracket\sigma_n\rrbracket \subseteq\mathcal{U}$ so
  that $\mu(\mathcal{U}-(\llbracket\sigma_0\rrbracket \cup\ldots
  \cup\llbracket\sigma_n\rrbracket)) <m$. This approximation of
  $\mathcal{B}$ is tight enough that
  \begin{align*}
    \mu(\{X\in\llbracket\sigma_0\rrbracket \cup\ldots\cup
    \llbracket\sigma_n\rrbracket: \Gamma^X\in[A]^\omega\}) &>2m,\\
    \mu(\{X\in\llbracket\sigma_0\rrbracket \cup\ldots\cup
    \llbracket\sigma_n\rrbracket: \Gamma^X\not\in[A]^\omega\}) &<2m.
  \end{align*}

  Assume the $(s-1)$-th element of $A$ has been found to be below $k$.
  Enumerate the branches in $\llbracket\sigma_0\rrbracket \cup\ldots\cup
  \llbracket\sigma_n\rrbracket$.  Wait
  till a measure of $2m$ of them claim to have found an element larger than
  $k$. Then one of these branches truly computes an infinite subset of
  $A$, so the $s$-th element of $A$ must be below the largest element
  found. $\blacksquare$
\end{frame}

\begin{frame}{$\text{RT}_2^1$ $\nleq_{\text{soc}}$ WWKL}
  \begin{thm}
    $\text{RT}_2^1$ $\nleq_{\text{soc}}$ WWKL.
  \end{thm}

  \vspace{1em}
  \textbf{Pf:} Let $c:\omega\rightarrow\{0,1\}$ be a 2-coloring of the
  graph of a fixed bi-hyperimmune $A$, that is,
  \[c(n)=0 \Leftrightarrow n\in A.\]
  
  From hyperimmunity of $A$ and $\bar{A}$, the class of sets computing an
  infinite subset of $A$ or of $\bar{A}$ is null. Equivalently, the class
  of $c$-homogeneous sets is null. Therefore given arbitrary tree
  $T\subseteq\omega^{<\omega}$ of positive measure, some real must fail to
  compute any $c$-homogeneous set. $\blacksquare$
\end{frame}

\begin{frame}{RT $\nleq_{\text{soc}}$ WKL, WKL $\nleq_{\text{soc}}$ WWKL}
  \begin{coro}
    \label{coro:rt-wwkl}
    RT $\nleq_{\text{soc}}$ WWKL.
  \end{coro}
  \textbf{Pf:} Follows from $\text{RT}_2^1$ $\nleq_{\text{soc}}$ WWKL.
  $\blacksquare$

  \vspace{2em}
  \begin{coro}
    WKL $\nleq_{\text{soc}}$ WWKL.
  \end{coro}
  \textbf{Pf:} Follows from transitivity of $\leq_\text{soc}$,
  RT $\leq_{soc}$ WKL, and RT $\nleq_{\text{soc}}$ WWKL. $\blacksquare$
\end{frame}
