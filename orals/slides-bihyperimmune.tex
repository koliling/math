\begin{frame}{Bi-Hyperimmune sets}
  \begin{define}
    $X\subseteq\omega$ is \textbf{hyperimmune} if there is no recursive
    function $f$ such that $f(n)>p_X(n)$ for every $n\in\omega$.
  \end{define}

  \begin{define}
    $X$ is \textbf{bi-hyperimmune} if both $X$ and $\bar{X}$ are hyperimmune.
  \end{define}

  \begin{thm}
    Bi-hyperimmune sets exist.
  \end{thm}

  \vspace{0.5em}
  \textbf{Pf:} Enumerate recursive functions $f_0,f_1,\ldots$. At stage
  $s$, let $i$ be the number of elements in $X$ and $\bar{X}$ that has been
  defined so far. Put enough elements into $\bar{X}$ till the $(i+1)$-th
  element of $X$ exceeds $f_s(i+1)$. Repeat with roles of $X$ and $\bar{X}$
  reversed. $\blacksquare$
\end{frame}

\begin{frame}{Class of sets computing subsets of hyperimmune is null}
  \begin{thm}
    \label{thm:bihyper-null}
    Given a hyperimmune set $X$, the class of sets that can compute an
    infinite subset of $X$ is null.
  \end{thm}

  \vspace{1em}
  \textbf{Pf:} The idea is, if the class is not null, there are enough
  reals computing subsets of $X$ that one can effectively get them to
  ``vote'' for when new elements of $X$ have appeared, contradicting
  hyperimmunity of $X$.

  \vspace{1em}
  Given arbitrary Turing functional $\Phi$, let
  \[\mathcal{B} :=\{A: \Phi^A \subseteq X\; \text{infinite}\}.\]
  It is enough to show $\mathcal{B}$ is null since the union of countably
  many null sets is null.
\end{frame}

\begin{frame}{Class of sets computing subsets of hyperimmune is null}
  Assume by contradiction $\mu(\mathcal{B})=4m>0$.\\
  Tightly approximate $\mathcal{B}$ by open cover
  $\mathcal{U}'\supseteq\mathcal{B}$ so
  $\mu(\mathcal{U}'-\mathcal{B})<m$.\\
  Tightly approximate $\mathcal{U}'$ by $\mathcal{U} :=[\sigma_0] \cup
  \ldots\cup[\sigma_n] \subseteq\mathcal{U}'$ so
  $\mu(\mathcal{U}'-\mathcal{U}) <m$.\\
  Then $\mathcal{U}$ approximates $\mathcal{B}$ tightly enough that
  \begin{align*}
    \mu(\{A\in\mathcal{U}: \Phi^A \subseteq X\; \text{infinite}\}) &>2m,\\
    \mu(\{A\in\mathcal{U}: \Phi^A \not\subseteq X\; \text{infinite}\}) &<2m.
  \end{align*}

  To effectively dominate the $k$-th element of $X$, enumerate the nodes in
  $\mathcal{U}$. Wait till a measure of $2m$ of them find $k$-elements. By
  tight approximation, such nodes exist, and one of them must be an initial
  segment of a real that computes an infinite subset of $X$. Thus the
  $k$-th element of $X$ must be below the largest element found by all
  these nodes. $\blacksquare$
\end{frame}

\begin{frame}{$\text{RT}_2^1$ $\nleq_{\text{soc}}$ WWKL}
  \begin{thm}
    $\text{RT}_2^1$ $\nleq_{\text{soc}}$ WWKL.
  \end{thm}

  \vspace{1em}
  \textbf{Pf:} Let $c:\omega\rightarrow\{0,1\}$ be a 2-coloring of the
  graph of a fixed bi-hyperimmune $X$, that is,
  \[c(n)=0 \Leftrightarrow n\in X.\]
  
  From hyperimmunity of $X$ and $\bar{X}$, the class of sets computing an
  infinite subset of $X$ or of $\bar{X}$ is null. Equivalently, the class
  of $c$-homogeneous sets is null. Therefore given arbitrary tree
  $T\subseteq\omega^{<\omega}$ of positive measure, some real must fail to
  compute any $c$-homogeneous set. $\blacksquare$
\end{frame}

\begin{frame}{Homogeneity is not weaker than Randomness}
  \begin{coro}[RT $\nleq_{\text{soc}}$ WWKL]
    $\text{RT}_2^1$ $\nleq_{\text{soc}}$ WWKL. $\blacksquare$
  \end{coro}
  \begin{coro}[WKL $\nleq_{\text{soc}}$ WWKL]
    RT $\leq_{soc}$ WKL, RT $\nleq_{\text{soc}}$ WWKL. $\blacksquare$
  \end{coro}

  \vspace{1em}
  \begin{center}
    \begin{tikzpicture}[node distance=3cm,auto,thick,>=latex']
      \node (KL) {KL$\leftrightarrow$WKL};
      \node (WWKL) [below right of=KL] {WWKL};
      \node (RT) [below left of=KL] {RT};
      \draw[->] (KL) -- (RT);
      \draw[->] (KL) -- (WWKL);
      \draw [<-,red] (RT) -- coordinate (m) (WWKL);
      \draw[shift={(m)},red](-0.1,-0.1)--(0.1,+0.1);
    \end{tikzpicture}
  \end{center}
\end{frame}
