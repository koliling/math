\begin{frame}{Homogeneity does not imply Randomness}
  \begin{center}
    \begin{tikzpicture}[node distance=3cm,auto,thick,>=latex']
      \node (KL) {KL$\leftrightarrow$WKL};
      \node (WWKL) [below right of=KL] {WWKL};
      \node (RT) [below left of=KL] {RT};
      \draw[->] (KL) -- (RT);
      \draw[->] (KL) -- (WWKL);
      \draw [->,red] (RT) -- coordinate (m) (WWKL);
      \draw[shift={(m)},red](-0.1,-0.1)--(0.1,+0.1);
    \end{tikzpicture}
  \end{center}

  \begin{define*}
    $\mathcal{A}\subseteq\omega^\omega$ is $\Sigma_1^1$ iff there is a
    computable predicate $R(f,g)$ such that
    \[\mathcal{A} =\{f\in\omega^\omega: (\exists g)\; [R(f,g)]\}.\]
  \end{define*}

  \begin{main-thm*}[Monin, Patey, 2016]
    $\mathcal{C}\subseteq\omega^\omega$ compact, contains no
    $\Sigma_1^1$-subset ($\neq\emptyset$). Then there exists infinite set
    $X$ whose subsets cannot compute any real in $\mathcal{C}$.
  \end{main-thm*}
\end{frame}

\begin{frame}{Every $\Sigma_1^1$-set of $2^\omega$ contains an $O$-recursive}
  \begin{fact}
    Every $\Sigma^1_1$-set is $O$-recursive.
  \end{fact}

  \begin{lemma}
    Every $\Sigma_1^1$-set of $2^\omega$ contains an $O$-recursive.
  \end{lemma}

  \textbf{Pf (sketch):} Given a $\Sigma_1^1$-set $\mathcal{A}
  =\{f\in2^\omega: (\exists g)\; R(f,g)\}$, observe that for a given
  $\sigma,\tau\in 2^{<\omega}$, the set
  \[\{f\succ \sigma: (\exists g\succ\tau)\; R(f,g)\}\]
  is also $\Sigma^1_1$. So from the above fact, $O$ can enumerate the
  $\langle \sigma,\tau \rangle$ pairs for which the above set is nonempty.
  Arbitrarily long pairs exist since $\mathcal{A}$ is nonempty. Thus, $O$
  can iteratively construct a real $f=\bigcup_{s\in\omega} \sigma_s$ in
  $\mathcal{A}$, and $g=\bigcup_{s\in\omega} \tau_s$ witnessing
  $f\in\mathcal{A}$. $\square$
\end{frame}

\begin{frame}{Tree of positive measure with no $\Sigma_1^1$-subset}
  \begin{thm}
    There is a tree $T\subset2^{<\omega}$ such that $[T]\subset
    2^\omega$ is compact, has positive measure, and contains no
    $\Sigma_1^1$-subset ($\neq\emptyset$).
  \end{thm}

  \vspace{2em}
  \textbf{Pf:} Let $T\subset2^{<\omega}$ be a set of 1-randoms relativized
  to Kleene's $O$. Then $[T]\subset 2^\omega$ has positive measure, and is
  closed and therefore compact. Also, $T$ has no nonempty
  $\Sigma_1^1$-subset from the previous lemma that every $\Sigma^1_1$-set
  contains an $O$-recursive. $\blacksquare$
\end{frame}

\begin{frame}{Homogeneity does not imply Randomness}
  \begin{theorem}
    WWKL $\nleq_{\text{soc}}$ RT.
  \end{theorem}

  \vspace{1em}
  \textbf{Pf:} Let $T\subset2^{<\omega}$ be the tree from the previous
  theorem, and let $c:[\omega]^n\rightarrow k$ be an arbitrary
  $k$-coloring. Recall:
  \begin{main-thm*}
    $\mathcal{C}\subseteq\omega^\omega$ compact, contains no
    $\Sigma_1^1$-subset ($\neq\emptyset$). Then there exists infinite set
    $X$ whose subsets cannot compute any real in $\mathcal{C}$.
  \end{main-thm*}

  \vspace{1em}
  The main theorem gives an infinite $X$ whose subsets cannot compute any
  real in $[T]$. Let $Y\subseteq X$ be an infinite subset of $X$ that is
  $c$-homogeneous. Then $Y$ is a $c$-homogeneous set that cannot compute
  any real in $[T]$. $\blacksquare$
\end{frame}

\begin{frame}{Homogeneity ``independent from'' Randomness}
  \begin{coro}[WKL $\nleq_{\text{soc}}$ RT]
    WWKL $\leq_{soc}$ WKL, WWKL $\nleq_{\text{soc}}$ RT. $\blacksquare$
  \end{coro}

  \vspace{2em}
  \begin{center}
    \begin{tikzpicture}[node distance=4cm,auto,thick,>=latex']
      \node (KL) {KL$\leftrightarrow$WKL};
      \node (WWKL) [below right of=KL] {WWKL};
      \node (RT) [below left of=KL] {RT};
      \draw[->] (KL) -- (RT);
      \draw[->] (KL) -- (WWKL);
      %\draw [<-,red] (RT) -- coordinate (m) (WWKL);
      %\draw[shift={(m)},red](-0.1,-0.1)--(0.1,+0.1);
    \end{tikzpicture}
  \end{center}
\end{frame}
