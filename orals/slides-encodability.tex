\begin{frame}{Recursive-Encodability, $\Pi_1^0$-Encodability}
  For the remaining directions
  \begin{itemize}
    \item WKL $\nleq_{\text{soc}}$ RT
    \item WWKL $\nleq_{\text{soc}}$ RT
  \end{itemize}
  we need the following definitions and characterizations:

  \begin{define}[Recursive-Encodable $A\subseteq\omega$]
    $A\subseteq\omega$ is recursively-encodable if given any
    $X\in[\omega]^\omega$, some subset of $X$ computes $A$.
  \end{define}

  \begin{define}[$\Pi_1^0$-encodable $\mathcal{A}\subseteq\omega^\omega$]
    $\mathcal{A}\subseteq \omega^{\omega}$ is $\Pi_1^0$-encodable if for
    every $X\in[\omega]^\omega$, there exists $Y\subseteq X$ and tree
    $T\leq_T Y$ such that
    \[\mathcal{A} \supseteq [T]\neq\emptyset,\]
    and some $f\leq_T Y$ bounds every branch in $T$.
  \end{define}
\end{frame}

\begin{frame}{Characterizing Encodability}
  \begin{main-thm*}[Characterizing $\Pi_1^0$-encodable
  $\mathcal{A}\subseteq\omega^\omega$]
    $\mathcal{A}\subseteq \omega^{\omega}$ compact. Then
    \[\mathcal{A}\; \text{is}\; \Pi_1^0\text{-encodable}\; \Leftrightarrow
    \mathcal{A}\; \text{contains}\; \Sigma_1^1\; \text{subset}\;
    (\neq\emptyset).\]
  \end{main-thm*}

  \begin{coro*}[Characterizing Recursively-Encodable $A\subseteq\omega$]
    $A\subseteq\omega$ recursively-encodable $\Leftrightarrow$ $A$ is
    hyperarithmetic.
  \end{coro*}

  \vspace{1em}
  From Main Theorem, suffices to show
  \begin{itemize}
    \item $A\subseteq\omega$ is hyperarithmetic $\Leftrightarrow$
      $\{A\}\subseteq\omega^\omega$ is $\Sigma_1^1$
    \item $A\subseteq\omega$ recursively-encodable $\Leftrightarrow$
      $\{A\}\subseteq\omega^\omega$ is $\Pi_1^0$-encodable
  \end{itemize}
\end{frame}

\begin{frame}{$A$ hyperarithmetic $\Leftrightarrow$ $\{A\}$ is $\Sigma_1^1$}
  \begin{align*}
    \;&A\subseteq\omega\; \text{hyperarithmetic}\\
    \Rightarrow\; & A\in\Sigma_1^1\\
    \Rightarrow\; & A=\{n:(\exists f)(\forall s)\; [R(f\restriction s,n)]\}\\
    \Rightarrow\; & \{A\}= \{g:(\exists f)(\forall s)(\forall n)\\
    &[R(f\restriction s,n) \rightarrow g(n)=1 \wedge \neg R(f\restriction s,n)
      \rightarrow g(n)=0]\}\\
    \Rightarrow\; &\{A\}\subseteq\omega^\omega\; \text{is}\; \Sigma_1^1,\\
    &\\
    \Rightarrow\; & \{A\}= \{g:(\exists f)(\forall s)\; [Q(f\restriction
      s,g\restriction s)]\}\\
    \Rightarrow\; &
      \begin{cases}
        A=\{n:(\exists g)\; [g(n)=1\; \wedge\; (\exists f)(\forall s)\;
          Q(f\restriction s,g\restriction s)]\}\\
        A=\{n:(\forall g)\; [g(n)=0\; \rightarrow\; (\forall f)(\exists s)\;
          \neg Q(f\restriction s,g\restriction s)]\}\\
      \end{cases}\\
    \Rightarrow\; &A\in\Delta_1^1\\
    \Leftrightarrow\; &A\subseteq\omega\; \text{hyperarithmetic}.\\
  \end{align*}
\end{frame}

\begin{frame}{$A$ recursively-encodable $\Leftrightarrow$
$\{A\}$ is $\Pi_1^0$-encodable}
  \begin{align*}
    \;&A\subseteq\omega\; \text{recursively-encodable}\\
    \Leftrightarrow\; & (\forall X\in[\omega]^\omega)(\exists Y\subseteq
      X)\; [Y\geq_T A]\\
    \Rightarrow\; & (\forall X\in[\omega]^\omega)(\exists Y\subseteq X)\;
      [\{A\}=\{\Phi^Y\}]\\
    \Rightarrow\; &\{A\}\subseteq\omega^\omega\; \text{is}\;
      \Pi_1^0\text{-encodable},\\
    &\\
    \Rightarrow\; & (\forall X\in[\omega]^\omega)(\exists Y\subseteq
      X) [\{A\}=[R^Y]]
  \end{align*}
  Given $A\restriction n$, to decide if $n\in A$, enumerate all nodes
  $\sigma\in2^{<\omega}$ extending $A\restriction n$ and check if
  $R^Y(\sigma)$. Since $A$ is the unique path in $[R^Y]$, by weak Konig's
  lemma, eventually exactly one of $[(A\restriction n)^\frown 0]$ or
  $[(A\restriction n)^\frown 1]$ will be covered by a finite set of basic
  open covers $[\sigma_0]\cup\ldots\cup[\sigma_m]$ with $\neg R(\sigma_i)$.
  Then the other $[(A\restriction n)^\frown j]$ gives the correct value
  of $A(n)$.
  \vspace{0.5em}

  Thus $Y\geq_TA$, making $A$ recursively-encodable.
\end{frame}

\begin{frame}{WKL $\nleq_{\text{soc}}$ RT}
  \begin{itemize}
    \item Fix instance of WKL with only one path $f$ that is
      not hyperarithmetic.
    \item Fix instance \textbf{Q} of RT, assume by contradiction all
      its solutions compute $f$.
    \item Given arbitrary $X\in[\omega]^\omega$, relativize \textbf{Q} to
      $X$, there exists some homogeneous $Y\in[X]^\omega$.
    \item By assumption, $Y$ computes $f$. Since $X$ is arbitrary, $f$ is
      recursively-encodable.
    \item But recursive-encodability is the same as being
      hyperarithmetic, $\Rightarrow\Leftarrow$.
  \end{itemize}
\end{frame}
