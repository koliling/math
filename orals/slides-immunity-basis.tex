\begin{frame}{$\Sigma_1^{1}$-immunity (fixed predicate)}
  \begin{lemma*}[$\Sigma_1^{1}$-immunity, fixed predicate]
    \begin{align*}
      \text{Given } &\begin{cases}
        \mathcal{C}\subseteq\omega^\omega \text{ compact, contains
        no nonempty } \Sigma_1^{1}\text{-set},\\
        \Sigma_1^{1}\text{-set } \mathcal{D}\neq\emptyset,\\
        \Sigma_1^{1}\text{-predicate } P(X,Y),
      \end{cases}
    \end{align*}
    $\mathcal{D}$ has a nonempty $\Sigma_1^{1}$-subset
    $\mathcal{X}\subseteq\mathcal{D}$ such that for every
    $X\in\mathcal{X}$, $\mathcal{C}$ does not contain the
    $\Sigma_1^{1,X}$-set $\{Y:P(X,Y)\}$.
  \end{lemma*}
\end{frame}

\begin{frame}{$\Sigma_1^{1}$-immunity (fixed predicate, cont.)}
  \textbf{Pf:} \textbf{Case 1}: For every $X\in\mathcal{D}$, $\mathcal{C}$
  contains the $\Sigma_1^{1,X}$-set $\{Y:P(X,Y)\}$. Then
  \[\bigcup_{X\in\mathcal{D}} \{Y:P(X,Y)\}\]
  will be a $\Sigma_1^{1}$-subset of $\mathcal{C}$, $\Rightarrow\Leftarrow$.

  \vspace{2em}
  \textbf{Case 2}: For some $X_0\in\mathcal{D}$, $\mathcal{C}$ does not
  contain the $\Sigma_1^{1,X_0}$-set $\{Y:P(X,Y)\}$. Let $Y_0 \in
  \{Y:P(X_0,Y)\} \setminus\mathcal{C}$. By compactness of $\mathcal{C}$,
  $Y_0$ has an initial segment $\sigma$ outside $\mathcal{C}$. This
  $\Sigma_1^{1}$-subset of $\mathcal{D}$ works: \[\mathcal{X}:=
  \{X\in\mathcal{D}: (\exists Y\succ\sigma)\; P(X,Y)\}.\; \blacksquare\]
\end{frame}

\begin{frame}{$\Sigma_1^{1}$-immunity}
  \begin{thm*}[$\Sigma_1^{1}$-immunity]
    \begin{align*}
      \text{Given } &\begin{cases}
        \mathcal{C}\subseteq\omega^\omega \text{ compact, contains
        no nonempty } \Sigma_1^{1}\text{-set},\\
        \Sigma_1^{1}\text{-set } \mathcal{D}\neq\emptyset,
      \end{cases}
    \end{align*}
    $\mathcal{D}$ contains some $X$ such that $\mathcal{C}$ does not
    contain any $\Sigma_1^{1,X}$-set.
  \end{thm*}

  \vspace{1em}
  \textbf{Pf:} The idea is to iterate the previous lemma across all
  $\Sigma^1_1$-predicates $P_0,P_1,\ldots$ to construct a descending
  sequence of $\Sigma^1_1$-sets
  \[\mathcal{D} =\mathcal{X}_0 \supseteq\mathcal{X}_1 \supseteq\ldots,\]
  and pick any $X$ in the intersection.
\end{frame}

\begin{frame}{$\Sigma_1^{1}$-immunity (cont.)}
  To ensure nonempty intersection, simultaneously construct
  $\sigma_0\prec\sigma_1\prec\ldots$ and
  $\tau_0\prec\tau_1\prec\ldots$ such that
  \[\tau_s \text{ witnesses } \llbracket\sigma_s\rrbracket
  \cap\mathcal{X}_s\neq\emptyset.\]

  \vspace{1em}
  \textbf{Stage 0}: Initialize $\mathcal{X}_0=\mathcal{D}$,
  $\sigma_0=\tau_0=\emptyset$.

  \vspace{1em}
  \textbf{Stage $s+1$}:
  At stage $n$, $[f\restriction n]\cap\mathcal{U} \neq\emptyset$, witnessed
  by an extension of $g\restriction n$. Now the class of reals in
  $[f\restriction n]$ that lie in $\mathcal{U}$ by a witness extending
  $g\restriction n$ is a $\Sigma^1_1$-set, so this class intersects
  $\mathcal{D}_0\cap\ldots\cap\mathcal{D}_{n+1}$. Thus one can choose
  $f\restriction (n+1) \succ f\restriction n$ in this intersection, and
  $g\restriction (n+1) \succ g\restriction n$ witnessing $[f\restriction
  (n+1)]\cap\mathcal{U} \neq\emptyset$. $\blacksquare$
\end{frame}

\begin{frame}{$\Sigma_1^{1}$-immunity, relativized}
  \begin{coro*}[$\Sigma_1^{1}$-immunity, \textcolor{red}{relativized}]
    \begin{align*}
      \text{Given } &\begin{cases}
        \mathcal{C}\subseteq\omega^\omega \text{ compact, contains
        no nonempty } \Sigma^{1,\textcolor{red}{Z}}_{1}\text{-set},\\
        \Sigma^{1,\textcolor{red}{Z}}_{1}\text{-set }
        \mathcal{D}\neq\emptyset,
      \end{cases}
    \end{align*}
    $\mathcal{D}$ contains some $X$ such that $\mathcal{C}$ does not
    contain any $\Sigma_1^{1,X}$-set.
  \end{coro*}

  \vspace{2em}
  \textbf{Pf:} Every step in the previous two proofs can be directly
  relativized.
  $\blacksquare$
\end{frame}
