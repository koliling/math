\begin{frame}{$\Sigma_1^{1}$-immunity (fixed predicate)}
  \begin{lemma*}[$\Sigma_1^{1}$-immunity, fixed predicate]
    \begin{align*}
      \text{Given } &\begin{cases}
        \mathcal{C}\subseteq\omega^\omega \text{ compact, contains
        no nonempty } \Sigma_1^{1}\text{-set},\\
        \Sigma_1^{1}\text{-set } \mathcal{D}\neq\emptyset,\\
        \Sigma_1^{1}\text{-predicate } P(X,Y),
      \end{cases}
    \end{align*}

    \pause
    $\mathcal{D}$ has a nonempty $\Sigma_1^{1}$-subset
    $\mathcal{X}\subseteq\mathcal{D}$ such that for every
    $X\in\mathcal{X}$, $\mathcal{C}$ does not contain the
    $\Sigma_1^{1,X}$-set $\{Y:P(X,Y)\}$.
  \end{lemma*}
\end{frame}

\begin{frame}{$\Sigma_1^{1}$-immunity (fixed predicate, cont.)}
  \textbf{Pf:} \textbf{Case 1}: For every $X\in\mathcal{D}$, $\mathcal{C}$
  contains the $\Sigma_1^{1,X}$-set $\{Y:P(X,Y)\}$. Then
  \[\bigcup_{X\in\mathcal{D}} \{Y:P(X,Y)\}\]
  will be a $\Sigma_1^{1}$-subset of $\mathcal{C}$, $\Rightarrow\Leftarrow$.

  \pause
  \vspace{2em}
  \textbf{Case 2}: For some $X_0\in\mathcal{D}$, $\mathcal{C}$ does not
  contain the $\Sigma_1^{1,X_0}$-set $\{Y:P(X,Y)\}$. Let $Y_0 \in
  \{Y:P(X_0,Y)\} \setminus\mathcal{C}$. By compactness of $\mathcal{C}$,
  $Y_0$ has an initial segment $\sigma$ outside $\mathcal{C}$. This
  $\Sigma_1^{1}$-subset of $\mathcal{D}$ works: \[\mathcal{X}:=
  \{X\in\mathcal{D}: (\exists Y\succ\sigma)\; P(X,Y)\}.\; \blacksquare\]
\end{frame}

\begin{frame}{$\Sigma_1^{1}$-immunity}
  \begin{thm*}[$\Sigma_1^{1}$-immunity]
    \begin{align*}
      \text{Given } &\begin{cases}
        \mathcal{C}\subseteq\omega^\omega \text{ compact, contains
        no nonempty } \Sigma_1^{1}\text{-set},\\
        \Sigma_1^{1}\text{-set } \mathcal{D}\neq\emptyset,
      \end{cases}
    \end{align*}
    $\mathcal{D}$ contains some $X$ such that $\mathcal{C}$ does not
    contain any $\Sigma_1^{1,X}$-set.
  \end{thm*}

  \pause
  \vspace{1em}
  \textbf{Pf:} The idea is to iterate the previous lemma across all
  $\Sigma^1_1$-predicates $P_0,P_1,\ldots$ to construct a descending
  sequence of nonempty $\Sigma^1_1$-sets
  \[\mathcal{D} =\mathcal{X}_0 \supseteq\mathcal{X}_1 \supseteq\ldots,\]
  then pick any $X$ in the intersection.
\end{frame}

\begin{frame}{$\Sigma_1^{1}$-immunity (cont.)}
  To ensure nonempty intersection, simultaneously construct
  $\emptyset= \sigma_0\prec\sigma_1\prec\ldots$ and $\emptyset=
  \tau_0\prec\tau_1\prec\ldots$ such that $\tau_s$ witnesses $\mathcal{X}_s
  \subseteq \llbracket\sigma_s\rrbracket$, in the sense that
  $\mathcal{X}_s$ is a $\Sigma^1_1$-set of the form
  \[\mathcal{X}_s =\{X\succ\sigma_s: (\exists Z\succ\tau_s)\;
  [R_s(X,Z)]\}\]
  for some recursive $R_s(X,Z)$. Then $X=\bigcup_{s\in\omega} \sigma_s \in
  \bigcap_{s\in\omega} \mathcal{X}_s$ works, and
  $\bigcup_{s\in\omega} \tau_s$ witnesses the inclusion.

  \pause
  \vspace{1.5em}
  \textbf{Stage $s+1$}: Write $P_s(X,Y) =(\exists Z)\; R_s'(X,Y,Z)$ where
  $R_s'$ is recursive. Apply previous lemma with $\Sigma^1_1$-set
  $\mathcal{X}_s$ and $\Sigma^1_1$-predicate 
  \[P_s'(X,Y) :=(\exists Z\succ\tau_s)\; [R_s'(X,Y,Z) \wedge
  X\succ\sigma_s],\]
  to get nonempty $\Sigma^1_1$-subset $\mathcal{X}_{s+1} \subseteq
  \mathcal{X}_s \subseteq \llbracket\sigma_s\rrbracket$. Pick any
  $\sigma_{s+1}\succ\sigma_s$ such that $\mathcal{X}_{s+1}\cap
  \llbracket\sigma_{s+1}\rrbracket \neq\emptyset$, then pick any
  $\tau_{s+1}\succ\tau_s$ witnessing this nonempty intersection.
  $\blacksquare$
\end{frame}

\begin{frame}{$\Sigma_1^{1}$-immunity, relativized}
  \begin{coro*}[$\Sigma_1^{1}$-immunity, \textcolor{red}{relativized}]
    \begin{align*}
      \text{Given } &\begin{cases}
        \mathcal{C}\subseteq\omega^\omega \text{ compact, contains
        no nonempty } \Sigma^{1,\textcolor{red}{Z}}_{1}\text{-set},\\
        \Sigma^{1,\textcolor{red}{Z}}_{1}\text{-set }
        \mathcal{D}\neq\emptyset,
      \end{cases}
    \end{align*}
    $\mathcal{D}$ contains some $X$ such that $\mathcal{C}$ does not
    contain any $\Sigma_1^{1,X}$-set.
  \end{coro*}

  \pause
  \vspace{2em}
  \textbf{Pf:} Every step in the previous two proofs can be directly
  relativized.
  $\blacksquare$
\end{frame}
