\begin{frame}{Bi-Hyperimmune sets}
  \begin{define}[Hyperimmune, Bi-hyperimmune]
    $A\subseteq\omega$ is \textbf{hyperimmune} if there is no recursive set
    $B$ such that for every $n\in\omega$, the $n$-th element of $B$ is
    larger than the $n$-element of $A$. $A$ is \textbf{bi-hyperimmune} if
    both $A$ and $\bar{A}$ are hyperimmune.
  \end{define}

  \begin{thm}
    Bi-hyperimmune sets exist.
  \end{thm}

  \vspace{0.5em}
  \textbf{Pf:} Enumerate recursive sets $B_0,B_1,\ldots$. At stage $s$, let
  $i$ be the number of elements in $A$ and $\bar{A}$ that has been defined
  so far. Put enough elements into $\bar{A}$ till the $(i+1)$-th element of
  $A$ exceeds that of $B_s$, then put the next element into $A$. Repeat
  with roles of $A$ and $\bar{A}$ reversed. $\blacksquare$
\end{frame}

\begin{frame}{Class of sets computing subsets of hyperimmune is null}
  \begin{thm}
    \label{thm:bihyper-null}
    Given a hyperimmune set $A$, the class of sets that can compute an
    infinite subset of $A$ is null.
  \end{thm}

  \vspace{1em}
  \textbf{Pf:} The idea is, if the measure is positive, there will be
  enough paths computing subsets of $A$ that one can effectively get them
  to ``vote'' for when new elements of $A$ have appeared, contradicting
  hyperimmunity of $A$.

  \vspace{1em}
  Since there are only countably many Turing functionals, and the union of
  countably many null sets is null, it suffices to show $\{X:
  \Gamma^X\in[A]^\omega\}$ is null, where
  $\Gamma:2^\omega\rightarrow2^\omega$ is a fixed Turing functional.
  Write $\mathcal{B} :=\{X: \Gamma^X\in[A]^\omega\}$.
\end{frame}

\begin{frame}{Class of sets computing subsets of hyperimmune is null}
  Assume by contradiction $\mu(\mathcal{B})=4m>0$. Tightly approximate
  $\mathcal{B}$ by open cover $\mathcal{O}\supseteq\mathcal{B}$ so that
  $\mu(\mathcal{O}-\mathcal{B})<m$, then again tightly approximate
  $\mathcal{O}$ by $[\sigma_0],\ldots,[\sigma_n] \subseteq\mathcal{O}$ so
  that $\mu(\mathcal{O}-([\sigma_0]\cup\ldots\cup[\sigma_n])) <m$.
  Then $[\sigma_0]\cup\ldots\cup[\sigma_n]$ approximates of the class
  of paths computing infinite subsets of $A$ so tightly that
  \begin{align*}
    &\mu(\text{``true'' infinite subset-computers in }
    [\sigma_0]\cup\ldots\cup[\sigma_n])>2m,\\
    &\mu(\text{``false'' infinite subset-computers in }
    [\sigma_0]\cup\ldots\cup[\sigma_n])<2m.
  \end{align*}

  Assume the $(s-1)$-th element of $A$ has been found to be below $k$.
  Enumerate the branches in $[\sigma_0]\cup\ldots\cup[\sigma_n]$.  Wait
  till a measure of $2m$ of them claim to have found (via $\Gamma$) an
  element larger than $k$. Then one of these branches must be a true
  infinite-subset computer, so the $s$-th element of $A$ must be below the
  largest element found.
\end{frame}

\begin{frame}{$\text{RT}_2^1$ $\nleq_{\text{soc}}$ WWKL}
  \begin{thm}
    $\text{RT}_2^1$ $\nleq_{\text{soc}}$ WWKL.
  \end{thm}

  \vspace{1em}
  \textbf{Pf:} Let $c:\omega\rightarrow\{0,1\}$ be a 2-coloring of the
  graph of a fixed bi-hyperimmune $A$, that is,
  \[c(n)=0 \Leftrightarrow n\in A.\]
  
  From bi-hyperimmunity of $A$, the class of sets computing an infinite
  subset of $A$ or of $\bar{A}$ is null. Equivalently, the class of
  $c$-homogeneous sets is null. Therefore given arbitrary tree
  $T\subseteq\omega^{<\omega}$ of positive measure, some path must fail to
  compute any $c$-homogeneous set. $\blacksquare$
\end{frame}

\begin{frame}{RT $\nleq_{\text{soc}}$ WKL, WKL $\nleq_{\text{soc}}$ WWKL}
  \begin{coro}
    \label{coro:rt-wwkl}
    RT $\nleq_{\text{soc}}$ WWKL.
  \end{coro}
  \textbf{Pf:} Follows from $\text{RT}_2^1$ $\nleq_{\text{soc}}$ WWKL.
  $\blacksquare$

  \vspace{2em}
  \begin{coro}
    WKL $\nleq_{\text{soc}}$ WWKL.
  \end{coro}
  \textbf{Pf:} Follows from transitivity of $\leq_\text{soc}$,
  RT $\leq_{soc}$ WKL, and RT $\nleq_{\text{soc}}$ WWKL. $\blacksquare$
\end{frame}
