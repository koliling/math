\begin{frame}{Constructing bi-hyperimmunes}
  \begin{itemize}
    \item Want $A$ where $p_A,p_{\bar{A}}\not<f$ for all recursive $f$.
    \item Fix $\emptyset''$-enumeration $f_0,f_1,\ldots$ of all recursive
      functions.
    \item At stage $n$, let $i$ be the number of elements in $A$ and
      $\bar{A}$ that has been defined so far.
    \item Put enough elements into $\bar{A}$ till the $(i+1)$th element
      of $A$ exceeds that of $f_n$, then put the next element into $A$.
    \item Repeat this step but with the rolse of $A$ and $\bar{A}$
      reversed: Letting $j$ be the number of elements that has been defined
      after the previous step, put enough elements into $A$ till the
      $(j+1)$th element of $\bar{A}$ exceeds that of $f_n$, then put the
      next element into $\bar{A}$.
    \item Go to stage $n+1$. $\blacksquare$
  \end{itemize}
\end{frame}

\begin{frame}{Class of sets computing subsets of hyperimmune is null}
  Hyperimmune sets $A$ are characterized as those whose principal
  function $p_A$ is not dominated by any recursive $f$. \\
  \vspace{1em}

  Bi-Hyperimmune sets $A$ are those where $A$, $\bar{A}$ are hyperimmune.\\
  \vspace{1em}

  \textbf{Proof idea:} Assume not. Then there are enough paths computing
  infinite subsets of $A$ such that we can effectively ask them to ``vote''
  for when new elements of $A$ have appeared. Thus effectively dominating
  $p_A$, $\Rightarrow\Leftarrow$.\\
  \vspace{1em}

  Fix Turing functional $\Phi$. Write $\mathcal{B} :=\{X:
  \Phi^X\in[A]^\omega\}$. Suffices to show $\mu(\mathcal{B})=0$. Assume
  $\mu(\mathcal{B})=4m>0$.
\end{frame}

\begin{frame}{Class of sets computing subsets of hyperimmune is null}
  \begin{itemize}
    \item Approximate $\mathcal{B}$ by open cover
      $\mathcal{O}\supseteq\mathcal{B}$ with
      $\mu(\mathcal{O}-\mathcal{B})<m$.
    \item Approximate $\mathcal{O}$ by basic open sets
      $[\sigma_0],\ldots,[\sigma_n] \subseteq\mathcal{O}$ with
      \[\mu(\mathcal{O}-([\sigma_0]\cup\ldots\cup[\sigma_n])) <m.\]
    \item Fix effective enumeration of branches $\sigma$ in
      $[\sigma_0]\cup\ldots\cup[\sigma_n]$.
    \item At stage $s=0$, wait till a measure of $2m$ of such $\sigma$'s
      finds an element via $\Phi$; let $f(0)$ be the largest element found.
    \item Since $[\sigma_0]\cup\ldots\cup[\sigma_n]$ approximate
      $\mathcal{B}$ tightly, such $\sigma$'s exist.
      \begin{align*}
        \mu(\text{``true'' voters})>2m,\\
        \mu(\text{``false'' voters})<2m.
      \end{align*}
      Thus $f(0)\geq p_A(0)$.
    \item At stage $s+1$, wait till a measure of $2m$ of $\sigma$'s
      finds an element $>f(s)$; let $f(s+1)$ be the largest
      element found. $\blacksquare$
  \end{itemize}
\end{frame}

\begin{frame}{RT $\nleq_{\text{soc}}$ WWKL}
  \begin{itemize}
    \item $\text{RT}_2^1$ $\nleq_{\text{soc}}$ WWKL
    \item Fix $A\in[\omega]^\omega$ where
      \begin{align*}
        &\mu(\{X: X\; \text{computes element in}\; [A]^\omega\}) =0,\\
        \text{and}\;\;\; &\mu(\{X: X\; \text{computes element in}\;
        [\bar{A}]^\omega\}) =0.\\
      \end{align*}
    \item Choose 2-coloring $c:\omega\rightarrow\{0,1\}$, $c(n)=0
      \Leftrightarrow n\in A$.
    \item Given any instance of WWKL, since the set of its paths has
      measure $>0$, some path must fail to compute any infinite
      homogeneous set of $c$. 
    \item Any bi-hyperimmune set $A$ works, as we show next.
    $\blacksquare$
  \end{itemize}
\end{frame}
