\begin{frame}{Bi-Hyperimmune sets}
  \begin{define}[Hyperimmune, Bi-hyperimmune]
    $A\subseteq\omega$ is \textbf{hyperimmune} if there is no effective way
    of knowing when new elements of $A$ have appeared.  Formally, no
    recursive function dominates $p_A$, the principal function of $A$. $A$
    is \textbf{bi-hyperimmune} if both $A$ and $\bar{A}$ are hyperimmune.
  \end{define}

  \begin{thm}
    Bi-hyperimmune sets exist.
  \end{thm}

  \textbf{Pf:} Enumerate recursive functions $f_0,f_1,\ldots$. At stage
  $s$, let $i$ be the number of elements in $A$ and $\bar{A}$ that has been
  defined so far. Put enough elements into $\bar{A}$ till the $(i+1)$-th
  element of $A$ exceeds that of $f_s$, then put the next element into $A$.
  This ensures that $f_s$ cannot tell when the $(i+1)$-th element of $A$
  has appeared. Repeat with roles of $A$ and $\bar{A}$ reversed.
  $\blacksquare$
\end{frame}

\begin{frame}{Class of sets computing subsets of hyperimmune is null}
  \begin{thm}
    \label{thm:bihyper-null}
    Given a hyperimmune set $A$, the class of sets that can compute an
    infinite subset of $A$ is null.
  \end{thm}

  \vspace{1em}
  \textbf{Proof idea:} Assume not. Then there are enough paths computing
  infinite subsets of $A$ such that we can effectively ask them to ``vote''
  for when new elements of $A$ have appeared,
  $\Rightarrow\Leftarrow$. $\square$\\

  \vspace{1em}
  \textbf{Pf:} Fix Turing functional $\Phi$. Write $\mathcal{B} :=\{X:
  \Phi^X\in[A]^\omega\}$. Suffices to show $\mu(\mathcal{B})=0$. Assume
  $\mu(\mathcal{B})=4m>0$.
\end{frame}

\begin{frame}{Class of sets computing subsets of hyperimmune is null}
  \begin{itemize}
    \item Approximate $\mathcal{B}$ by open cover
      $\mathcal{O}\supseteq\mathcal{B}$ with
      $\mu(\mathcal{O}-\mathcal{B})<m$.
    \item Approximate $\mathcal{O}$ by basic open sets
      $[\sigma_0],\ldots,[\sigma_n] \subseteq\mathcal{O}$ with
      \[\mu(\mathcal{O}-([\sigma_0]\cup\ldots\cup[\sigma_n])) <m.\]
    \item Fix effective enumeration of branches $\sigma$ in
      $[\sigma_0]\cup\ldots\cup[\sigma_n]$.
    \item At stage $s=0$, wait till a measure of $2m$ of such $\sigma$'s
      finds an element via $\Phi$; let $f(0)$ be the largest element found.
    \item Since $[\sigma_0]\cup\ldots\cup[\sigma_n]$ approximate
      $\mathcal{B}$ tightly, such $\sigma$'s exist.
      \begin{align*}
        \mu(\text{``true'' voters})>2m,\\
        \mu(\text{``false'' voters})<2m.
      \end{align*}
      Thus $f(0)\geq p_A(0)$.
    \item At stage $s+1$, wait till a measure of $2m$ of $\sigma$'s
      finds an element $>f(s)$; let $f(s+1)$ be the largest
      element found. $\blacksquare$
  \end{itemize}
\end{frame}

\begin{frame}{$\text{RT}_2^1$ $\nleq_{\text{soc}}$ WWKL}
  \begin{thm}
    $\text{RT}_2^1$ $\nleq_{\text{soc}}$ WWKL.
  \end{thm}

  \vspace{1em}
  \textbf{Pf:} Let $c:\omega\rightarrow\{0,1\}$ be a 2-coloring of the
  graph of a fixed bi-hyperimmune $A$, that is,
  \[c(n)=0 \Leftrightarrow n\in A.\]
  
  From Theorem~\ref{thm:bihyper-null}, the class of sets computing an
  infinite subset of $A$ or of $\bar{A}$ is null, thus the class of
  $c$-homogeneous sets is null. Therefore given arbitrary tree
  $T\subseteq\omega^{<\omega}$ of positive measure, some path must fail to
  compute any $c$-homogeneous set. $\blacksquare$
\end{frame}

\begin{frame}{RT $\nleq_{\text{soc}}$ WKL, WKL $\nleq_{\text{soc}}$ WWKL}
  \begin{coro}
    \label{coro:rt-wwkl}
    RT $\nleq_{\text{soc}}$ WWKL.
  \end{coro}
  \textbf{Pf:} Follows from previous theorem. $\blacksquare$

  \vspace{2em}
  \begin{coro}
    WKL $\nleq_{\text{soc}}$ WWKL.
  \end{coro}
  \textbf{Pf:} Follows from transitivity of $\leq_\text{soc}$,
  RT $\leq_{soc}$ WKL, and RT $\nleq_{\text{soc}}$ WWKL. $\blacksquare$
\end{frame}
