\begin{frame}{Ramsey's Theorem ($\text{RT}$)}
  \begin{itemize}
    \item State $\text{RT}_k^n$
    \item $\text{RT}$ is $(\forall n)\; \text{RT}_{<\infty}^n$
  \end{itemize}

  \begin{itemize}
    \item Asserts existence of homogeneity
  \end{itemize}
\end{frame}

\begin{frame}{Randomness}
  \begin{itemize}
    \item What is 1-random
    \item Lebesgue Measure
  \end{itemize}
\end{frame}

\begin{frame}{Strongly Omnisciently Computably Reducible
  ($\leq_{\text{soc}}$)}
  \begin{itemize}
    \item Define problem, solution
    \item Define $\leq_{\text{soc}}$ on WWKL, RT
  \end{itemize}
\end{frame}

\note{
  \begin{itemize}
    \item Mention other reductions
    \item soc doesn't require the second problem to be computable from the
      first one
  \end{itemize}
}

\begin{frame}{Homogeneity ``independent from'' Randomness}
  Dependencies under $\leq_{\text{soc}}$
  \vspace{2em}

  \begin{center}
    \begin{tikzpicture}[node distance=4cm,auto,thick,>=latex']
      \node (KL) {KL$\leftrightarrow$WKL};
      \node (WWKL) [below right of=KL] {WWKL};
      \node (RT) [below left of=KL] {RT};
      \draw[->] (KL) -- (RT);
      \draw[->] (KL) -- (WWKL);
      %\draw [->,red] (RT) -- coordinate (m) (WWKL);
      %\draw[shift={(m)},red](-0.1,-0.1)--(0.1,+0.1);
    \end{tikzpicture}
  \end{center}
\end{frame}

\begin{frame}{KL $\equiv_{\text{soc}}$ WKL}
  \begin{itemize}
    \item WKL $\leq_{\text{soc}}$ KL: From definition.
    \item KL $\leq_{\text{soc}}$ WKL: Given KL instance \textbf{P}, at
      each level, code branches into WKL instance \textbf{Q} using binary
      representation of branch number
  \end{itemize}
\end{frame}

\begin{frame}{RT $\leq_{\text{soc}}$ WKL}
  \begin{itemize}
    \item Fix $k$-coloring $c:[\omega]^n\rightarrow k$
    \item Define binary tree $T\subseteq 2^{<\omega}$ by $\sigma\in T$
      iff $\{n:\sigma(n)=1\}$ is homogeneous
  \end{itemize}
\end{frame}

\begin{frame}{WWKL $\lneq_{\text{soc}}$ WKL}
  \begin{itemize}
    \item WWKL $\leq_{\text{soc}}$ WKL: From definition.
    \item WKL $\nleq_{\text{soc}}$ WWKL: Fix instance of WKL with only one
      solution $f$, which is non-computable
    \item \TODO{Set of oracles computing $f$ has 0 measure}
    \item Given arbitrary instance of WWKL, since set of solutions has
      positive measure, there must be solutions that do not compute $f$ 
  \end{itemize}
\end{frame}

\begin{frame}{RT $\nleq_{\text{soc}}$ WWKL}
  \begin{itemize}
    \item $\text{RT}_2^1$ $\nleq_{\text{soc}}$ WWKL
    \item Fix $A\in[\omega]^\omega$ where
      \begin{align*}
        &\mu(\{X: X\; \text{computes element in}\; [A]^\omega\}) =0,\\
        \text{and}\;\;\; &\mu(\{X: X\; \text{computes element in}\;
        [\bar{A}]^\omega\}) =0.\\
      \end{align*}
    \item Choose 2-coloring $c:\omega\rightarrow\{0,1\}$, $c(n)=0
      \Leftrightarrow n\in A$.
    \item Given any instance of WWKL, since the set of its paths has
      measure $>0$, some path must fail to compute any infinite
      homogeneous set of $c$. 
    \item Bi-hyperimmune sets $A$ work.
  \end{itemize}
\end{frame}

\begin{frame}{$A$ Hyperimmune $\implies \mu(\{X: X\; \text{computes
element in}\; [A]^\omega\}) =0$}
  Hyperimmune sets $A$ are characterized as those whose principal
  function $p_A$ is not dominated by any recursive $f$. \\
  \vspace{1em}

  Bi-Hyperimmune sets $A$ are those where $A$, $\bar{A}$ are hyperimmune.\\
  \vspace{1em}

  \textbf{Main idea:} Assume not. Then there are enough paths computing
  infinite subsets of $A$ such that we can effectively ask them to ``vote''
  for when new elements of $A$ have appeared. Thus giving recursive
  $f$ that dominates $p_A$.\\
  \vspace{1em}

  Fix Turing functional $\Phi$. Write $\mathcal{C} :=\{X:
  \Phi^X\in[A]^\omega\}$. Note $\mathcal{C}$ is Borel and thus measurable.
  Since the measure of a countable union of null sets is 0, and
  there are only countably many Turing functionals, it suffices to show
  $\mu(\mathcal{C})=0$. Assume $\mu(\mathcal{C})=4m>0$.
\end{frame}

\begin{frame}{$A$ Hyperimmune $\implies \mu(\{X: X\; \text{computes
element in}\; [A]^\omega\}) =0$ (cont.)}
  \begin{itemize}
    \item Approximate $\mathcal{C}$ by open cover
      $\mathcal{O}\supseteq\mathcal{C}$ with
      $\mu(\mathcal{O}-\mathcal{C})<m$.
    \item Approximate $\mathcal{O}$ by basic open sets
      $[\sigma_0],\ldots,[\sigma_n] \subseteq\mathcal{O}$ with
      \[\mu(\mathcal{O}-([\sigma_0]\cup\ldots\cup[\sigma_n])) <m.\]
    \item Fix effective enumeration of branches $\sigma$ in
      $[\sigma_0]\cup\ldots\cup[\sigma_n]$.
    \item At stage $s=0$, wait till a measure of $2m$ of such $\sigma$'s
      finds an element via $\Phi$; let $f(0)$ be the largest element found.
      Since $[\sigma_0]\cup\ldots\cup[\sigma_n]$ approximate $\mathcal{C}$
      tightly, such $\sigma$'s exist, and more than a measure of $m$ of
      them indeed lie in $\mathcal{C}$.
    \item In particular, at least one of these $\sigma$'s really computes
      an infinite subset of $A$, so $f(0)\geq p_A(0)$.
    \item At stage $s+1$, wait till a measure of $2m$ of $\sigma$'s
      finds an element exceeding $f(s)$; let $f(s+1)$ be the largest
      element found.
  \end{itemize}
\end{frame}

\begin{frame}{Bi-Hyperimmune sets $A$ exist}
  \begin{itemize}
    \item Want $A$ where $p_A,p_{\bar{A}}\not<f$ for all recursive $f$.
    \item Fix $\emptyset''$-enumeration $f_0,f_1,\ldots$ of all recursive
      functions.
    \item At stage $n$, we have defined the first $i$th elements of $A$.
    \item Put enough elements into $\bar{A}$ so that the $(i+1)$th element
      of $A$ exceeds that of $f_n$.
    \item Then we would have defined the first $j$th elements of $\bar{A}$.
    \item Similarly, put enough elements into $A$ so that the $(j+1)$th
      element of $\bar{A}$ exceeds that of $f_n$.
  \end{itemize}
\end{frame}
