\begin{frame}{Subsets cannot compute trees containing path in $\mathcal{A}$}
  \begin{main-thm*}
    $\mathcal{A}\subseteq\omega^\omega$ compact, contains no
    $\Sigma_1^1$-subset ($\neq\emptyset$). Then there exists infinite set
    $X$ where none of its subsets can compute a tree that contains a path
    in $\mathcal{A}$.
  \end{main-thm*}

  The proof hinges on Galvin-Prikry. Fix Turing machine $\Gamma$, node
  $\tau$.
  \begin{fact}[Galvin-Prikry]
    $Z\in[\omega]^\omega$. There exists solution $X\in[Z]^\omega$
    where one of these holds:\\
    \textbf{$\Sigma$-solution:} All subsets of $X$ compute (via $\Gamma$)
    $\tau$ \\
    \textbf{$\Pi$-solution:} All subsets of $X$ do not compute (via
    $\Gamma$) $\tau$
  \end{fact}
  \textbf{Proof idea of GP:} The class of subsets of $Z$ that
  compute $\tau$ can be coded to form an open set in $2^\omega$. Being
  open provides enough ``structure'' for regions of homogeneity to exist.
\end{frame}

\begin{frame}{Subsets cannot compute trees containing path in $\mathcal{A}$}
  \textbf{Proof idea of main theorem:} Want $X$ whose subsets cannot
  compute path in $\mathcal{A}$. First fix machine. Given node
  $\tau$, GP will find $\Sigma$-solutions $X$ (all its subsets compute
  $\tau$), and/or $\Pi$-solutions $X$ (all its subsets do not compute
  $\tau$). Exhaust cases:

  \vspace{1em}
  \textbf{Case 0:} GP has $\Sigma$-solutions for arbitrarily long nodes in
  $\mathcal{A}$. Then by compactness of $\mathcal{A}$, these nodes contain
  a path in $\mathcal{A}$, from which we can define a non-empty
  $\Sigma_1^1$-subset of $\mathcal{A}$, $\Rightarrow\Leftarrow$.

  \vspace{0.5em}
  \textbf{Case 1:} GP only has $\Pi$-solutions for each node in
  $\mathcal{A}$ of a particular length. Iterate GP across each such node to
  get a solution that is negative for all of them.

  \vspace{1em}
  Thus for fixed machine, GP gives $X$ whose subsets compute no paths in
  $\mathcal{A}$. To work across all machines, iterate across them,
  constructing decreasing subsets of $X$, then take intersection.
\end{frame}

\begin{frame}{Subsets cannot compute (via $\Gamma$) trees containing path
in $\mathcal{A}$}
  \begin{lemma}[Fixed machine]
    \label{lemma:fixed-machine}
    $\mathcal{A}$ compact, contains no $\Sigma_1^{1,Z}$-subset
    ($\neq\emptyset$). Fix Turing functional $\Gamma$. Then there exists
    infinite set $X$ where none of its subsets can compute (via $\Gamma$) a
    tree that contains a path in $\mathcal{A}$.
  \end{lemma}

  \vspace{0.5em}
  \textbf{Case 0:} GP has $\Sigma$-solutions for arbitrarily long
  $\sigma\in\mathcal{A}$. Observe that by finite use principle, given
  $\sigma$, the class $\mathcal{P}_{\sigma}$ of $\Sigma$-solutions is
  $\Sigma_1^{1,Z}$:
  \begin{align*}
    \mathcal{P}_{\sigma}:= &\{X\in[Z]^\omega: \text{All subsets of}\; X\;
      \text{compute}\; \sigma\; (\text{via}\; \Gamma)\}\\
    =&\{X\in[Z]^\omega: \text{All finite subsets of}\; X\;
      \text{compute}\; \sigma\; (\text{via}\; \Gamma)\}
      \in\Sigma_1^{1,Z}.
  \end{align*}
  By compactness of $\mathcal{A}$, the set of arbitrarily long $\sigma$'s
  must contain a path in $\mathcal{A}$. Then $\{f:(\forall \sigma\prec f)\;
  [\mathcal{P}_\sigma \neq \emptyset]\}$ is a non-empty
  $\Sigma_1^{1,Z}$-set contained in $\mathcal{A}$, $\Rightarrow\Leftarrow$.
\end{frame}

\begin{frame}{Subsets cannot compute (via $\Gamma$) trees containing path
in $\mathcal{A}$}
  \textbf{Case 1:} For some $n$, GP has only $\Pi$-solutions for every
  $\sigma\in\omega^n\cap\mathcal{A}$. Since $\mathcal{A}$ is compact, it is
  finitely branching, and we can write $\sigma_0,\ldots,\sigma_m$ for the
  nodes in $\omega^n\cap\mathcal{A}$. Iterate GP across $\sigma_i$ to get
  decreasing subsets of $\Pi$-solutions
  \[Z=X_0 \supseteq X_1 \supseteq \ldots\supseteq X_m=X,\]

  where $X_{i+1}\in[X_i]^\omega$ is a $\Pi$-solution by GP under
  inputs $X_i$ and $\sigma_i$. $X_{i+1}$ exists since GP has no
  $\Sigma$-solutions for each $\sigma_i$ under $Z$, giving also no
  $\Sigma$-solutions under $X_i\subseteq Z$. $X=X_m$ works. $\blacksquare$
\end{frame}

\begin{frame}{Subsets cannot compute trees containing path in $\mathcal{A}$}
  Lemma~\ref{lemma:fixed-machine} finds $X$ that works for a fixed
  machine. We iterate the lemma across all machines to find $X$ that
  works across them. For the lemma to work at each iteration, it needs
  to be relativized to choose an $X$ where $\mathcal{A}$ contains no
  $\Sigma_1^{1,X}$-subset.

  \vspace{0.5em}
  Recall in the proof of Lemma~\ref{lemma:fixed-machine} that $X$ was
  chosen as the intersection of finitely descending subsets
  \[Z=X_0 \supseteq X_1 \supseteq \ldots\supseteq X_m=X,\]
  where $X_{i+1}$ lies in this $\Sigma_1^{1,X_i}$-set of $\Pi$-solutions
  from GP:
  \[\{X\in[X_i]^\omega: \text{No subset of}\; X\; \text{computes}\;
  \sigma\; (\text{via}\; \Gamma)\} \in\Sigma_1^{1,X_i}.\]
  We show that since $\mathcal{A}$ contains no $\Sigma_1^{1,X_i}$-subset,
  given arbitrary $\Sigma_1^{1,X_i}$-set, we can always find some $X_{i+1}$
  from it that is weak enough for $\mathcal{A}$ to also contain no
  $\Sigma_1^{1,X_{i+1}}$-subset.
\end{frame}

\begin{frame}{$\Sigma_1^1$-immunity basis theorem (fixed machine)}
  In fact, if we fix the machine for computations, we even find a
  $\Sigma_1^{1,X_i}$-subset of such $X_{i+1}$'s:
  \newtheorem*{immunity*}{$\Sigma_1^{1}$-immunity basis theorem}
  \begin{immunity*}[Fixed machine]
    $\mathcal{A}$ compact, contains no $\Sigma_1^{1,X_i}$-set
    ($\neq\emptyset$). Fix $\Sigma_1^{1,X_i}$-predicate $P(X,Y)$,
    $\Sigma_1^{1,X_i}$-set $\mathcal{B}_0\neq\emptyset$. Then
    $\mathcal{B}_0$ has a $\Sigma_1^{1,X_i}$-subset
    $\mathcal{B}\neq\emptyset$ such that for every $X_{i+1}\in\mathcal{B}$,
    $\mathcal{A}$ does not contain the $\Sigma_1^{1,X_{i+1}}$-set
    $\{Y:P(X_{i+1},Y)\}$, if this set is non-empty.
  \end{immunity*}

  \textbf{Case 1}: For every $X\in\mathcal{B}_0$,
  $\{Y:P(X,Y)\}\subseteq\mathcal{A}$. Then $\bigcup_{X\in\mathcal{B}_0}
  \{Y:P(X,Y)\}$ will be a $\Sigma_1^{1,X_i}$-subset of $\mathcal{A}$,
  $\Rightarrow\Leftarrow$.

  \vspace{0.5em}
  \textbf{Case 2}: For some $X_0\in\mathcal{B}_0$, $\{Y:P(X_0,Y)\}$
  contains a path $Y_0$ outside $\mathcal{A}$. By compactness of
  $\mathcal{A}$, $Y_0$ has an initial segment $\sigma$ outside
  $\mathcal{A}$. This $\Sigma_1^{1,X_i}$-subset of $\mathcal{B}_0$ works:
  \[\mathcal{B}:= \{X\in\mathcal{B}_0: (\exists Y\succ\sigma)\; P(X,Y)\}.\;
  \blacksquare\]
\end{frame}

\begin{frame}{$\Sigma_1^1$-immunity basis theorem}
  The previous theorem finds a $\Sigma_1^{1,X_i}$-subset of $X_{i+1}$'s
  that are weak enough for $\mathcal{A}$ to avoid them, when a fixed
  machine is used in the computations. To generalize across all machines,
  we iterate the theorem across them and take intersection. The previous
  theorem gives us enough $X_{i+1}$'s to ensure non-empty intersection.

  \vspace{1em}
  \begin{immunity*}
    $\mathcal{A}$ compact, contains no $\Sigma_1^{1,X_i}$-set
    ($\neq\emptyset$). Fix $\Sigma_1^{1,X_i}$-set
    $\mathcal{N}\neq\emptyset$. Then $\mathcal{N}$ contains some $X$ such
    that $\mathcal{A}$ contains no $\Sigma_1^{1,X}$-set ($\neq\emptyset$).
  \end{immunity*}

  \vspace{1em}
  For each $\Sigma_1^{1,X_i}$-predicate $P$, let $\mathcal{U}_P$ be the
  union of all the $\Sigma_1^{1,X_i}$-sets where every $X$ in the set gives
  a path outside $\mathcal{A}$ via $P$. From previous theorem,
  $\mathcal{U}_P$ is dense (under Gandy-Harrington topology where basic
  open sets are $\Sigma_1^{1,X_i}$-sets). Since this topology is Baire,
  $\bigcap_P\mathcal{U}_P$ is is non-empty. Any
  $X\in\bigcap_P\mathcal{U}_P$ works. $\blacksquare$
\end{frame}

\begin{frame}{Subsets cannot compute trees containing path in $\mathcal{A}$}
  The problem is, if we iterate with the current version of
  Lemma~\ref{lemma:fixed-machine}, we may not end up with an infinite $X$.
  To ensure infiniteness, at stage $s$ of the iteration, we want to choose
  the $X_{s+1}\in[X_s]^\omega$ where $X_{s+1}$ preserves the first
  $s$-elements of $X_s$. We strengthen Lemma~\ref{lemma:fixed-machine} by
  requiring that the $X\in[Z]^\omega$ found preserves arbitrary initial
  segments of $Z$.
\end{frame}

\begin{frame}{Strengthening Lemma~\ref{lemma:fixed-machine}}
  To iterate successfully, we strengthen Lemma~\ref{lemma:fixed-machine}
  twice:
  \newtheorem*{lemma-strengthened1*}{Lemma \ref{lemma:fixed-machine}':}
  \begin{lemma-strengthened1*}
    $\mathcal{A}$ compact, contains no $\Sigma_1^{1,Z}$-set
    ($\neq\emptyset$). Fix Turing functional $\Gamma$. Then
    there exists $X\in[Z]^\omega$ such that every subset of $X$ computes
    (via $\Gamma$) no path in $\mathcal{A}$.\\
    \vspace{0.5em}
    Furthermore, $\mathcal{A}$ contains no $\Sigma_1^{1,X}$-set
    ($\neq\emptyset$).
  \end{lemma-strengthened1*}

  \vspace{0.5em}
  \newtheorem*{lemma-strengthened2*}{Lemma \ref{lemma:fixed-machine}'':}
  \begin{lemma-strengthened2*}
    Given any $s\in\omega$, require the $X$ in
    Lemma~\ref{lemma:fixed-machine}' to preserve $Z\restriction s$.
  \end{lemma-strengthened2*}

  \vspace{0.5em}
  Lemma~\ref{lemma:fixed-machine}'' can be directly proven from
  Lemma~\ref{lemma:fixed-machine}' by iterating the latter across all
  subsets of $Z\restriction s$.
\end{frame}

\begin{frame}{Lemma~\ref{lemma:fixed-machine}'': $X\subseteq Z$ that
preserves $Z\restriction s$}
  List the subsets of $Z\restriction s$ as $d_0,\ldots,d_m$. Iterate the
  partially-strengthened lemma across $d_i$'s to get decreasing subsets
  \[Z=X_0 \supseteq X_1 \supseteq X_2 \supseteq\ldots \supseteq X_m.\]

  At stage $i$, Lemma~\ref{lemma:fixed-machine}' is applied with inputs
  $X_i$ and $\Gamma_i$ defined by
  \[\Gamma_i(Y) =\Gamma(Y\cup d_i),\]

  giving $X_{i+1}\in[X_i]^\omega$ whose subsets compute (via
  $\Gamma_i$) no path in $\mathcal{A}$, and where $\mathcal{A}$ contains no
  $\Sigma_1^{1,X_i}$-set ($\neq\emptyset$).

  \vspace{0.5em}
  Choose $X=X_m\cup Z\restriction s$. To see that this $X$ works, first
  observe that since $X$ almost equals $X_m$, $\mathcal{A}$ will
  contain no $\Sigma_1^{1,X}$-set since it contains no
  $\Sigma_1^{1,X_m}$-set. Fix arbitrary $Y\in[X]^\omega$. Then
  $Y\restriction i=d_i$ for some $i$. Stage $i$ ensured that the subsets of
  $Y$ compute no path in $\mathcal{A}$ via $\Gamma_i$. Since $Y\supset
  d_i$, the same is ensured via $\Gamma$. $\blacksquare$
\end{frame}

\begin{frame}{Subsets compute no path in $\mathcal{A}$ (all machines)}
  Iterating Lemma~\ref{lemma:fixed-machine}'' across all machines
  $\Gamma_0,\Gamma_1,\ldots$, we prove:
  \begin{main-thm*}
    $\mathcal{A}$ compact, contains no $\Sigma_1^1$-subset
    ($\neq\emptyset$). Then there exists infinite set $X$ whose subsets
    compute no path in $\mathcal{A}$.
  \end{main-thm*}

  Construct decreasing subsets of $X$'s from
  Lemma~\ref{lemma:fixed-machine}''
  \[\omega= X_0\supseteq X_1\supseteq\ldots,\]
  where at stage $s$, the lemma is applied with inputs $X_s$ and machine
  $\Gamma_s$, and we require $X_{s+1}$ to preserve the first $s$-elements
  of $X_s$. Take $X=\bigcap_{s\in\omega}X_s$; this is infinite by
  construction. Given $Y\in[X]^\omega$ and arbitary $\Gamma_i$, stage $i$
  of the construction ensures that since $Y\in[X_i]^\omega$, $Y$ does
  computes no path in $\mathcal{A}$ via $\Gamma_i$. $\blacksquare$
\end{frame}

\begin{frame}{$X\subseteq Z$ where $\mathcal{A}$ contains no
$\Sigma_1^{1,X}$-set}
  \begin{lemma-strengthened1*}
    $\mathcal{A}$ compact, contains no $\Sigma_1^{1,Z}$-set
    ($\neq\emptyset$). Fix Turing functional $\Gamma$. Then
    there exists $X\in[Z]^\omega$ such that every subset of $X$ computes
    (via $\Gamma$) no path in $\mathcal{A}$. Furthermore, $\mathcal{A}$
    contains no $\Sigma_1^{1,X}$-set ($\neq\emptyset$).
  \end{lemma-strengthened1*}

  \vspace{0.5em}
  For $\mathcal{A}$ to contain no $\Sigma_1^{1,X}$-set, we choose $X$ from
  Lemma~\ref{lemma:fixed-machine} with this property. Recall
  in the proof of the lemma, the $X$ found lies in
  $\mathcal{N}_{\sigma_0}\cap \ldots\cap\mathcal{N}_{\sigma_m}$,
  where $\mathcal{N}_{\sigma_i}$ is a $\Sigma_1^{1,X_i}$-set and $X_0=Z$.
  By iterating the following theorem $m$-times, it suffices to show

  \vspace{0.5em}
  \begin{immunity*}[Relativized]
    $\mathcal{A}$ compact, contains no $\Sigma_1^{1,Z}$-set
    ($\neq\emptyset$). Fix $\Sigma_1^{1,Z}$-set
    $\mathcal{N}\neq\emptyset$. Then $\mathcal{N}$ contains some $X$ such
    that $\mathcal{A}$ contains no $\Sigma_1^{1,X}$-set ($\neq\emptyset$).
  \end{immunity*}
\end{frame}

\begin{frame}{$\Sigma_1^{1,Z}$-immunity basis theorem (relativized)}
  Every step in the proof can be directly relativized to arbitary $Z$ since
  the relativized Gandy-Harrington topology is Baire, and the proof of the
  claim used in the theorem is also relativizable.

  \vspace{0.5em}
  \begin{immunity*}[Relativized]
    $\mathcal{A}$ compact, contains no $\Sigma_1^{1,Z}$-set
    ($\neq\emptyset$). Fix $\Sigma_1^{1,Z}$-set
    $\mathcal{N}\neq\emptyset$. Then $\mathcal{N}$ contains some $X$ such
    that $\mathcal{A}$ contains no $\Sigma_1^{1,X}$-set ($\neq\emptyset$).
  \end{immunity*}
\end{frame}

\begin{frame}{WWKL $\nleq_{\text{soc}}$ RT}
  \begin{itemize}
    \item Fix tree $T\subset 2^{<\omega}$ such that
      $[T]\subset 2^\omega$ is compact, has positive measure, and
      contains no $\Sigma_1^1$-subset ($\neq\emptyset$).
    \item Assuming such $T$ exists, fix RT instance
      $c:[\omega]^n\rightarrow k$ and assume every $c$-homogeneous set
      computes a path in $T$.
    \item Now given arbitrary infinite set $Z$, relativizing $c$ with
      respect to $Z$ gives us a $c$-homogeneous set $X\in[Z]^\omega$.
      By assumption, every element in $[X]^\omega$ computes a path in $T$.
    \item Thus every infinite set $Z$ has an infinite subset $X$ whose
      infinite subsets compute a path in $T$.
    \item By the main theorem, $[T]$ must contain a
      non-empty $\Sigma_1^1$-subset, $\Rightarrow\Leftarrow$.
      $\blacksquare$
    \item It remains to prove such $T$ exists.
  \end{itemize}
\end{frame}

\begin{frame}{$[T]$ with positive measure, containing no
$\Sigma_1^1$-subset}
  \begin{itemize}
    \item Let $T$ be set of Martin-Lof randoms relativized to Kleene's $O$.
    \item Then $\mu([T])>0$.
    \item $T$ has no non-empty $\Sigma_1^1$-subset because every
      $\Sigma_1^1$-set of $2^\omega$ contains an $O$-recursive.
      $\blacksquare$
  \end{itemize}
\end{frame}
