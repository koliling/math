\begin{frame}{$\Sigma_1^{1}$-immunity basis theorem (fixed machine)}
  \begin{lemma*}[$\Sigma_1^{1}$-immunity basis, fixed machine]
    $\mathcal{C}$ compact, contains no $\Sigma_1^{1}$-set
    ($\neq\emptyset$). Fix $\Sigma_1^{1}$-predicate $P(X,Y)$,
    $\Sigma_1^{1}$-set $\mathcal{D}_0\neq\emptyset$. Then
    $\mathcal{D}_0$ has a $\Sigma_1^{1}$-subset
    $\mathcal{D}\neq\emptyset$ such that for every $X\in\mathcal{D}$,
    $\mathcal{C}$ does not contain the $\Sigma_1^{1,X}$-set
    $\{Y:P(X,Y)\}$ (if $\neq\emptyset$).
  \end{lemma*}

  \vspace{1em}
  \textbf{Pf:} \underline{Case 1}: For every $X\in\mathcal{D}_0$,
  $\{Y:P(X,Y)\}\subseteq\mathcal{C}$. Then $\bigcup_{X\in\mathcal{D}_0}
  \{Y:P(X,Y)\}$ will be a $\Sigma_1^{1}$-subset of $\mathcal{C}$,
  $\Rightarrow\Leftarrow$.

  \vspace{1em}
  \underline{Case 2}: For some $X_0\in\mathcal{D}_0$, $\{Y:P(X_0,Y)\}$
  contains a path $Y_0$ outside $\mathcal{C}$. By compactness of
  $\mathcal{C}$, $Y_0$ has an initial segment $\sigma$ outside
  $\mathcal{C}$. This $\Sigma_1^{1}$-subset of $\mathcal{D}_0$ works:
  \[\mathcal{D}:= \{X\in\mathcal{D}_0: (\exists Y\succ\sigma)\; P(X,Y)\}.\;
  \blacksquare\]
\end{frame}

\begin{frame}{Gandy-Harrington topology is Baire}
  \begin{thm*}[Gandy-Harrington topology is Baire]
    The topology on $\omega^\omega$ where open sets are generated by the
    $\Sigma_1^{1}$-sets is known as the Gandy-Harrington topology. This
    topology is Baire, i.e. the countable union of dense open sets is
    dense.
  \end{thm*}

  \vspace{1em}
  \textbf{Pf:} Given dense open sets $\mathcal{D}_0,\mathcal{D}_1,\ldots$
  and open set $\mathcal{U}=\{f:(\exists g)\; R(f,g)\}$, iteratively
  construct $f\in\bigcap_{n\in\omega}\mathcal{D}_n$ and $g$ witnessing
  $f\in\mathcal{U}$:
  
  \vspace{1em}
  At stage $n$, $[f\restriction n]\cap\mathcal{U} \neq\emptyset$, witnessed
  by an extension of $g\restriction n$. Now the class of paths in
  $[f\restriction n]$ that lie in $\mathcal{U}$ by a witness extending
  $g\restriction n$ is an open set, so it intersects
  $\mathcal{D}_0\cap\ldots\cap\mathcal{D}_{n+1}$. Thus one can choose
  $f\restriction (n+1) \succ f\restriction n$ in this intersection, and
  $g\restriction (n+1) \succ g\restriction n$ witnessing $[f\restriction
  (n+1)]\cap\mathcal{U} \neq\emptyset$. $\blacksquare$
\end{frame}

\begin{frame}{$\Sigma_1^{1,Z}$-immunity basis theorem}
  \begin{thm*}[$\Sigma_1^{1}$-immunity basis]
    $\mathcal{C}$ compact, contains no $\Sigma_1^{1}$-set
    ($\neq\emptyset$). Fix $\Sigma_1^{1}$-set
    $\mathcal{D}\neq\emptyset$. Then $\mathcal{D}$ contains some $X$ such
    that $\mathcal{C}$ contains no $\Sigma_1^{1,X}$-set ($\neq\emptyset$).
  \end{thm*}
  \textbf{Pf:} For each $\Sigma_1^{1}$-predicate $P$, let $\mathcal{U}_P$
  be the union of all the $\Sigma_1^{1}$-sets where every $X$ in the set
  gives a path outside $\mathcal{C}$ via $P$. From previous lemma,
  $\mathcal{U}_P$ is dense under Gandy-Harrington topology. This topology
  is Baire, so $\bigcap_P\mathcal{U}_P$ is non-empty. Any
  $X\in\bigcap_P\mathcal{U}_P$ works. $\blacksquare$

  \vspace{0.5em}
  \begin{coro*}[$\Sigma_1^{1,Z}$-immunity basis]
    $\mathcal{C}$ compact, contains no $\Sigma_1^{1,Z}$-set
    ($\neq\emptyset$). Fix $\Sigma_1^{1,Z}$-set
    $\mathcal{D}\neq\emptyset$. Then $\mathcal{D}$ contains some $X$ such
    that $\mathcal{C}$ contains no $\Sigma_1^{1,X}$-set ($\neq\emptyset$).
  \end{coro*}
  \textbf{Pf:} Directly relativize every step in the proof.
  $\blacksquare$
\end{frame}

\begin{frame}{Galvin-Prikry}
  \begin{fact*}[Galvin-Prikry]
    Given infinite set $Z\subseteq\omega$, node $\tau\in\omega^{<\omega}$,
    and Turing functional $\Gamma:2^\omega\rightarrow\omega$, there exists
    an infinite subset $X\subseteq Z$ where one of these holds:\\

    \vspace{1em}
    \textbf{Positive-solution:} All subsets of $X$ compute (via $\Gamma$)
    $\tau$ \\
    \textbf{Negative-solution:} All subsets of $X$ do not compute (via
    $\Gamma$) $\tau$
  \end{fact*}

  \vspace{1em}
  \textbf{Proof idea:} The class of subsets of $Z$ that
  compute $\tau$ can be coded to form an open set in $2^\omega$. Being
  open provides enough ``structure'' for regions of homogeneity to exist.
  $\square$
\end{frame}

\begin{frame}{Set whose subsets compute elements outside $\mathcal{C}$}
  \newtheorem*{main-lemma*}{Main Lemma}
  \begin{main-lemma*}
    $\mathcal{C}\subseteq\omega^\omega$ compact, contains no
    $\Sigma_1^{1,Z}$-subset ($\neq\emptyset$), and
    $\Gamma:2^{\omega}\rightarrow \omega^{<\omega}$ a Turing functional on
    trees. Then one of the following holds:\\
    \vspace{0.5em}
    \textbf{Case 1:} For every $n\in\omega$, there exists infinite set
    $X\subseteq Z$ and $\sigma\in\omega^n$ outside $\mathcal{C}$ such that
    every tree computed (via $\Gamma$) by a subset of $X$ contains
    $\sigma$.\\
    \vspace{0.5em}
    \textbf{Case 2:} There exists infinite set $X\subseteq Z$ such that
    none of its subsets can compute (via $\Gamma$) a tree containing paths
    in $\mathcal{C}$.
  \end{main-lemma*}

  \vspace{1em}
  \textbf{Pf:} \textbf{Case 1:} There are arbitrarily long branches outside
  $\mathcal{C}$ for which GP has positive-solutions. Any such solution
  works.
\end{frame}

\begin{frame}{Set whose subsets compute elements outside $\mathcal{C}$
(cont.)}
  \textbf{Case 2:} For some $n$, every branch in $\mathcal{C}$ of length
  $n$ has only negative GP-solutions. Note that there are only fintely many
  such branches $\sigma_0,\ldots,\sigma_m$ since $\mathcal{C}$ is compact.
  Iterate GP across them to get decreasing subsets of negative solutions
  \[Z=X_0 \supseteq X_1 \supseteq \ldots\supseteq X_m=X,\]
  where $X_{i+1}\in[X_i]^\omega$ is a negative GP-solution under inputs
  $X_i$ and $\sigma_i$. $X_{i+1}$ exists since GP gives no positive
  solutions for each $\sigma_i$ under $Z$, giving also no positive
  solutions under $X_i\subseteq Z$.

  \vspace{0.5em}
  \textbf{Case 0:} Every sufficiently long branch with positive
  GP-solutions lies in $\mathcal{C}$, and $\mathcal{C}$ has arbitrarily
  long branches with positive GP-solutions. Now by compactness of
  $\mathcal{A}$, the set of arbitrarily long branches contains a path
  in $\mathcal{A}$. Then \[\{f:(\forall \sigma\prec f)\; [\sigma\; \text{has
  positive GP-solutions}]\}\] is a non-empty $\Sigma_1^{1,Z}$-set contained
  in $\mathcal{A}$, $\Rightarrow\Leftarrow$. $\blacksquare$
\end{frame}

\begin{frame}{Set whose subsets compute elements outside $\mathcal{C}$
(relativized)}
  \begin{main-lemma*}[Relativized]
    $\mathcal{C}$ compact, contains no $\Sigma_1^{1,Z}$-subset
    ($\neq\emptyset$), and $\Gamma:2^{\omega}\rightarrow \omega^{<\omega}$
    a Turing functional on trees. Then one of these holds:\\
    \vspace{0.5em}
    \textbf{Case 1:} For every $n\in\omega$, there exists infinite set
    $X\subseteq Z$ and $\sigma\in\omega^n$ outside $\mathcal{C}$ such that
    every tree computed (via $\Gamma$) by a subset of $X$ contains
    $\sigma$.\\
    \vspace{0.5em}
    \textbf{Case 2:} There exists infinite set $X\subseteq Z$ such that
    none of its subsets can compute (via $\Gamma$) a tree containing paths
    in $\mathcal{C}$.\\
    \vspace{0.5em}
    Furthermore, we can choose $X$ such that $\mathcal{C}$ also contains no
    $\Sigma_1^{1,X}$-subset ($\neq\emptyset$).
  \end{main-lemma*}

  \textbf{Pf:} In the proof, $X$ was chosen from a $\Sigma_1^{1,Z}$-set.
  The relativized immunity-basis theorem asserts that every such set
  contains an element $X$ where $\mathcal{C}$ contains no
  $\Sigma_1^{1,X}$-set. $\blacksquare$
\end{frame}

\begin{frame}{Preserving initial segment}
  \begin{main-lemma*}[Relativized, Segment-preserving]
    $\mathcal{C}$ compact, contains no $\Sigma_1^{1,Z}$-subset
    ($\neq\emptyset$), and $\Gamma:2^{\omega}\rightarrow \omega^{<\omega}$
    a Turing functional on trees. Then one of these holds:\\
    \vspace{0.5em}
    \textbf{Case 1:} For every $n\in\omega$, there exists infinite set
    $X\subseteq Z$ and $\sigma\in\omega^n$ outside $\mathcal{C}$ such that
    every tree computed (via $\Gamma$) by a subset of $X$ contains
    $\sigma$.\\
    \vspace{0.5em}
    \textbf{Case 2:} There exists infinite set $X\subseteq Z$ such that
    none of its subsets can compute (via $\Gamma$) a tree containing paths
    in $\mathcal{C}$.\\
    \vspace{0.5em}
    Furthermore, we can choose $X$ such that $\mathcal{C}$ also contains no
    $\Sigma_1^{1,X}$-subset ($\neq\emptyset$).\\
    Finally, given $s\in\omega$, $X$ can be chosen to preserve
    $Z\restriction s$.
  \end{main-lemma*}

  \vspace{0.5em}
  \textbf{Pf:} The idea is to iterate the relativized lemma
  across all subsets of the segment $Z\restriction s$.
\end{frame}

\begin{frame}{Preserving initial segment (cont.)}
  List the subsets of $Z\restriction s$ as $d_0,\ldots,d_m$. Iterate the
  relativized lemma across them to get decreasing subsets
  \[Z=X_0 \supseteq X_1 \supseteq X_2 \supseteq\ldots \supseteq X_m.\]
  At stage $i$, apply relativized lemma with inputs $X_i$, and
  $\Gamma_i$ defined by
  \[\Gamma_i(Y) =\Gamma(Y\cup d_i),\]

  giving $X_{i+1}\in[X_i]^\omega$ whose subsets compute (via
  $\Gamma_i$) no path in $\mathcal{C}$, and where $\mathcal{C}$ contains no
  $\Sigma_1^{1,X_i}$-set ($\neq\emptyset$).

  \vspace{0.5em}
  Choose $X=X_m\cup Z\restriction s$. To see that this $X$ works, first
  observe that since $X$ almost equals $X_m$, $\mathcal{C}$ will
  contain no $\Sigma_1^{1,X}$-set since it contains no
  $\Sigma_1^{1,X_m}$-set. Fix arbitrary $Y\in[X]^\omega$. Then
  $Y\restriction i=d_i$ for some $i$. Stage $i$ ensured that the subsets of
  $Y$ compute no path in $\mathcal{C}$ via $\Gamma_i$. Since $Y\supset
  d_i$, the same is ensured via $\Gamma$. $\blacksquare$
\end{frame}

\begin{frame}{Set whose subsets cannot compute any path in $\mathcal{C}$}
  \begin{main-thm*}
    $\mathcal{C}$ compact, contains no $\Sigma_1^1$-subset
    ($\neq\emptyset$). Then there exists infinite set $X$ whose subsets
    cannot compute any element in $\mathcal{C}$.
  \end{main-thm*}

  \textbf{Pf:} The main lemma worked for fixed machine $\Gamma$. Iterate
  the lemma across all machines $\Gamma_0,\Gamma_1,\ldots$ to construct
  decreasing subsets of $X$'s
  \[\omega= X_0\supseteq X_1\supseteq\ldots,\]
  where at stage $s$, the lemma is applied with inputs $s$, $X_s$ and machine
  $\Gamma_s$, and $X_{s+1}$ preserves the first $s$-elements of $X_s$. Take
  $X=\bigcap_{s\in\omega}X_s$; this is infinite from
  segment-preservation. Then given $i\in\omega$ and $Y\subseteq X$, either
  $[\Gamma_i^Y] \cap \mathcal{C} =\emptyset$, or $\Gamma_i^T$ contains
  arbitrarily long branches outside $\mathcal{C}$. In particular, $Y$
  cannot compute any element in $\mathcal{C}$ via $\Gamma_i$.
  $\blacksquare$
\end{frame}

\begin{frame}{There exists a non-null tree with no $\Sigma_1^1$-subset}
  \begin{fact}
    \label{fact:sigma-contains-O-recursive}
    Every $\Sigma_1^1$-set of $2^\omega$ contains an $O$-recursive.
  \end{fact}
  \textbf{Proof idea:} Kleene's $O$ is $\Sigma_1^1$-complete, thus given a 
  $\Sigma_1^1$-set of $2^\omega$, it can iteratively pick the first branch
  that has a path, thereby computing the ``left-most'' path of the set.
  $\square$

  \vspace{1em}
  \begin{thm}
    There is a tree $T\subset2^{<\omega}$ such that $[T]\subset
    2^\omega$ is compact, has positive measure, and contains no
    $\Sigma_1^1$-subset ($\neq\emptyset$).
  \end{thm}
  \textbf{Pf:} Let $T\subset2^{<\omega}$ be a set of Martin-Lof randoms
  relativized to Kleene's $O$. Then $[T]\subset 2^\omega$ is compact and
  has positive measure. Also, $T$ has no non-empty $\Sigma_1^1$-subset from
  Fact~\ref{fact:sigma-contains-O-recursive}. $\blacksquare$
\end{frame}

\begin{frame}{WWKL $\nleq_{\text{soc}}$ RT}
  \begin{theorem}
    WWKL $\nleq_{\text{soc}}$ RT.
  \end{theorem}

  Consider the tree $T\subset2^{<\omega}$ which is a set of Martin-Lof
  randoms relativized to Kleene's $O$. Then $[T]\subset 2^\omega$ is
  compact, has positive measure, and contains no $\Sigma_1^1$-subset
  ($\neq\emptyset$). Let $c:[\omega]^n\rightarrow k$ be an arbitrary
  $k$-coloring.
  
  \vspace{1em}
  From the main theorem, there is an infinte set $X$ such that none of its
  subsets can compute any path in $[T]$. Relativize the colouring $c$ with
  respect to $X$ to get an infinite subset $Y\subseteq X$ that is
  $c$-homogeneous. So $Y$ is a $c$-homogeneous set that cannot compute
  any path in $[T]$, witnessing WWKL $\nleq_{\text{soc}}$ RT.
  $\blacksquare$
\end{frame}

\begin{frame}{Homogeneity ``independent from'' Randomness}
  \begin{coro}
    WKL $\nleq_{\text{soc}}$ RT.
  \end{coro}
  \textbf{Pf:} Follows from transitivity of $\leq_\text{soc}$,
  WWKL $\leq_{soc}$ WKL, and WWKL $\nleq_{\text{soc}}$ RT. $\blacksquare$

  \begin{center}
    \begin{tikzpicture}[node distance=4cm,auto,thick,>=latex']
      \node (KL) {KL$\leftrightarrow$WKL};
      \node (WWKL) [below right of=KL] {WWKL};
      \node (RT) [below left of=KL] {RT};
      \draw[->] (KL) -- (RT);
      \draw[->] (KL) -- (WWKL);
    \end{tikzpicture}
  \end{center}
\end{frame}
