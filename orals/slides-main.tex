\begin{frame}{$\mathcal{A}$ contains $\Sigma_1^1$ subset ($\neq\emptyset$)
$\Rightarrow$ $\mathcal{A}$ $\Pi_1^0$-encodable}
  \begin{itemize}
    \item Fix $\mathcal{A}$ containing $\mathcal{B} =\{X: (\exists
      Y)(\forall s)\; R(X,Y,s)\} \neq\emptyset$.
    \item Fix $X\in\mathcal{B}$, $Y$ such that $(\forall s)\;R(X,Y,s)$.
    \item Define $f(n)=\max(X(n),Y(n))$.
    \item Observe that $f$ acts as $\Pi_1^0$-modulus of $\mathcal{A}$, in
      the sense that if $g\geq f$, then $\mathcal{A}\supseteq[T]
      \neq\emptyset$ for some $T\leq_T g$ whose nodes are bounded by $g$:
    \item Given $g\geq f$, the set
      \[\{X\leq g:(\forall s)(\exists \sigma\in 2^s, \sigma\leq g)(\forall
      s'<s)\; R(X,\sigma,s')\}\]
      contains $X$ and is a $\Pi_1^{0,g}$-subset contained in
      $\mathcal{A}$.
    \item Given arbitrary $X\in[\omega]^\omega$, remove enough elements to
      get a sparse $Y\in[X]^\omega$ where $p_Y>f$.
    \item Then $p_Y$ witnesses $\Pi_1^0$-encodability of $\mathcal{A}$ for
      $X$.
  \end{itemize}
\end{frame}

\begin{frame}{Hard: $\Pi_1^0$-encodable $\rightarrow$ Contains
non-empty $\Sigma_1^1$ subset}
  \begin{itemize}
    \item Fix compact $\mathcal{C}$ with no non-empty $\Sigma_1^1$ subset.
      We want $G\in[\omega]^\omega$ such that no $X\in[G]^\omega$
      can compute (in the $\Pi_1^{0,X}$ sense) a non-empty subset of
      $\mathcal{C}$.

    \item Fix enumeration of Turing functionals $\Gamma_s:\omega^{<\omega}
      \rightarrow \omega^{<\omega}$.

    \item Construct $G=\bigcap_{s\in\omega}G_s$, where
      $\omega=G_0\supset G_1\supset\ldots$.
      
    \item At stage $s$, fix the first $s$ elements of $G$.
    
    \item Then remove enough elements beyond the first $s$ ones so that no
      $X\in[G_s]^\omega$ can compute a non-empty subset of $\mathcal{C}$
      via $\Gamma_s$.

    \item Then $|G|=\omega$, and $G$ satisfies the desired property.

    \item The hard part is proving such $G_{s+1}\in[G_s]^\omega$ exists.
  \end{itemize}
\end{frame}

\begin{frame}{Desired $G_{s+1}\in[G_s]^\omega$ exists}
  \begin{itemize}
    \item Fix $G_0\in[\omega]^\omega$, $n\in\omega$, compact
      $\mathcal{C}\subseteq\omega^\omega$, Turing functional
      $\Gamma:\omega^{<\omega} \rightarrow \omega^{<\omega}$.

    \item Want $G_1\in[G_0]^\omega$ such that $G_1\restriction
      n=G_0\restriction n$, and no $X\in[G_1]^\omega$ computes a non-empty
      subset of $\mathcal{C}$ via $\Gamma$.

    \item If we do not require $G_1\restriction n=G_0\restriction n$, then
      we can find $G_1$ with stronger property:
  \end{itemize}

  \newtheorem{L1}{Lemma 1}
  \begin{L1}
    Fix $G_0\in[\omega]^\omega$, compact
    $\mathcal{C}\subseteq\omega^\omega$ with no non-empty
    $\Sigma_1^{1,G_0}$ subset, Turing functional $\Gamma:\omega^{<\omega}
    \rightarrow \omega^{<\omega}$, $t\in\omega$. Then there exists
    $G_1\in[G_0]^\omega$ such that every $X\in[G_1]^\omega$ does not
    compute a non-empty subset of $\mathcal{C}$ via $\Gamma$ in the
    sense:

    \begin{itemize}
      \item Either $\mathcal{C}\cap[\Gamma^X]=\emptyset$,
      \item Or...
    \end{itemize}
  \end{L1}
\end{frame}

\begin{frame}{Desired $G_{s+1}\in[G_s]^\omega$ exists (cont.)}
  \begin{itemize}
      \item Fix enumeration $\{D_0,\ldots,D_m\}$ of subsets of
        $G_0\restriction n$

      \item Apply Lemma 1 $m$ times, once for each $D_i$.
  \end{itemize}
\end{frame}

\begin{frame}{Lemma 1}
  \begin{itemize}
    \item Proving Lemma 1 uses $\Sigma_1^1$-immunity basis theorem and
      Galvin-Prikry
  \end{itemize}
\end{frame}

\begin{frame}{$\Sigma_1^1$-immunity basis theorem}
  \begin{itemize}
    \item Proving this theorem uses fact that relativized Gandy-Harrington
      topoloty is Baire space
  \end{itemize}
\end{frame}

\begin{frame}{WWKL $\nleq_{\text{soc}}$ RT}
  \begin{itemize}
    \item Fix $T\subseteq 2^{<\omega}$ with $\mu(T)>0$, $[T]$
      has no non-empty $\Sigma_1^1$ subset
    \item We prove such $T$ exists
    \item Fix RT instance $c:[\omega]^n\rightarrow k$ and assume every
      solution $H$ computes a path through $T$, i.e. $[T]$ has non-empty
      $\Pi_1^{0,H}$ subset
    \item Thus every $X\in[\omega]^\omega$ has a subset
      $H\in[X]^\omega$ such that $[T]$ has non-empty $\Pi_1^{0,H}$ subset
    \item We say that $T$ is $\Pi_1^0$-encodable
    \item We will prove that being $\Pi_1^0$-encodable is equivalent to
      containing non-empty $\Sigma_1^1$ subset, $\rightarrow\leftarrow$
  \end{itemize}
\end{frame}

\begin{frame}{There exists $[T]$ with no $\Sigma_1^1$ subset, $\mu(T)>0$}
  \begin{itemize}
    \item Let $T$ be set of Martin-Lof randoms relativized to Kleene's $O$
    \item Then $\mu(T)>0$
    \item $T$ has no non-empty $\Sigma_1^1$ subset because every
      $\Sigma_1^1$ subset of $2^\omega$ contains an element recursive in
      $O$
  \end{itemize}
\end{frame}

\begin{frame}{Extra slides}
  \begin{itemize}
    \item Set of oracles computing a fixed non-recursive has 0 measure
    \item Gandy-Harrington topology (relativized) is Baire
    \item Every $\Sigma_1^1$ subset of $2^\omega$ contains an element
      recursive in Kleene's $O$
    \item Galvin-Prikry
  \end{itemize}
\end{frame}
