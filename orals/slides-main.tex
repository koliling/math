\begin{frame}{$\mathcal{A}$ contains $\Sigma_1^1$-subset ($\neq\emptyset$)
$\Rightarrow$ $\mathcal{A}$ $\Pi_1^0$-encodable}
  \begin{itemize}
    \item Fix $\mathcal{A}$ containing $\mathcal{B} =\{X: (\exists
      Y)(\forall s)\; R(X,Y,s)\} \neq\emptyset$.
    \item Fix $X\in\mathcal{B}$, $Y$ such that $(\forall s)\;R(X,Y,s)$.
    \item Define $f(n)=\max(X(n),Y(n))$.
    \item Then $f$ is a $\Pi_1^0$-modulus of $\mathcal{A}$, i.e. $f$ is
      dominating enough that every $g\geq f$ computes a $g$-bounded tree
      $T$ whose paths all lie in $\mathcal{A}$:
    \item Given $g\geq f$, the $\Pi_1^{0,g}$-set of paths
      \[\{X\leq g:(\forall s)(\exists \sigma\in 2^s, \sigma\leq g)(\forall
      s'<s)\; R(X,\sigma,s')\}\]
      is bounded by $g$, contains $X$, and is contained in $\mathcal{A}$.
    \item Given arbitrary $X\in[\omega]^\omega$, remove enough elements to
      make a sparse $Y\in[X]^\omega$ with $p_Y>f$.
    \item Then $p_Y$ witnesses $\Pi_1^0$-encodability of $\mathcal{A}$ for
      $X$. $\blacksquare$
  \end{itemize}
\end{frame}

\begin{frame}{Subsets compute no path in $\mathcal{A}$}
  \begin{main-thm*}[Harder direction]
    $\mathcal{A}$ compact, contains no $\Sigma_1^1$-subset
    ($\neq\emptyset$). Then there exists infinite set $X$ whose subsets
    compute no path in $\mathcal{A}$.
  \end{main-thm*}

  The proof hinges on Galvin-Prikry. Fix Turing machine $\Gamma$, node
  $\sigma$.
  \begin{fact}[Galvin-Prikry]
    $Z\in[\omega]^\omega$. There exists solution $X\in[Z]^\omega$
    where one of these holds:\\
    \textbf{Positive solution:} All subsets of $X$ compute $\sigma$
    (via $\Gamma$)\\
    \textbf{Negative solution:} All subsets of $X$ do not compute $\sigma$
    (via $\Gamma$)
  \end{fact}
  \textbf{Proof sketch of GP:} The class of subsets of $Z$ that
  compute $\sigma$ can be coded to form an open set in $2^\omega$. Being
  open provides enough ``structure'' for regions of homogeneity to exist.
\end{frame}

\begin{frame}{Main theorem: Subsets compute no path in $\mathcal{A}$}
  \textbf{Proof idea:} Want $X$ whose subsets compute no path in 
  $\mathcal{A}$. Fix machine. Then given any node $\sigma$, GP finds
  solution $X$ whose subsets either all compute $\sigma$ ($X$ is positive
  solution), or all do not compute $\sigma$ ($X$ is negative solution).
  Exhaust cases:

  \vspace{1em}
  \textbf{Case 0:} GP has positive solutions for arbitrarily long nodes in
  $\mathcal{A}$. Then by compactness of $\mathcal{A}$, these nodes contain
  a path in $\mathcal{A}$, giving us a non-empty $\Sigma_1^1$-set contained
  in $\mathcal{A}$, $\Rightarrow\Leftarrow$.

  \vspace{0.5em}
  \textbf{Case 1:} GP only has negative solutions for all nodes in
  $\mathcal{A}$ of a particular length. Iterate GP across each such node to
  get $X$ where all these nodes are avoided. Note there are only finitely
  many nodes to avoid since $\mathcal{A}$ is compact.

  %\vspace{0.5em}
  %\textbf{Case 1:} GP answers positively for some node outside $X$.

  \vspace{1em}
  Thus for fixed machine, GP gives us $X$ whose subsets compute no paths in
  $\mathcal{A}$. To find $X$ that works across all machines, iterate across
  the machines, constructing decreasing subsets of such $X$, then take
  intersection.
\end{frame}

\begin{frame}{Lemma~\ref{lemma:fixed-machine}: Subsets compute no path in
$\mathcal{A}$ (fixed machine)}
  \begin{lemma}
    \label{lemma:fixed-machine}
    $\mathcal{A}$ compact, contains no $\Sigma_1^{1,Z}$-set
    ($\neq\emptyset$). Fix Turing functional $\Gamma$. Then there exists
    $X\in[Z]^\omega$ such that every subset of $X$ computes (via $\Gamma$)
    no path in $\mathcal{A}$.

    %\vspace{0.5em}
    %\textbf{Case 2:} For each $n\in\omega$, there exists $X\in[Z]^\omega$
    %such that some $\sigma\in\omega^n$ outside $\mathcal{A}$ is computed by
    %every subset of $X$ (via $\Gamma$).

    %\vspace{0.5em}
    %Furthermore, $X$ can be chosen such that every $\Sigma_1^{1,X}$-set
    %contains a path (if any) outside $\mathcal{A}$.
  \end{lemma}

  \vspace{0.5em}
  \textbf{Case 0:} GP has positive solutions for arbitrarily long
  $\sigma\in\mathcal{A}$. Observe that by finite use principle, given
  $\sigma$, the class $\mathcal{P}_{\sigma}$ of positive solutions is
  $\Sigma_1^{1,Z}$:
  \begin{align*}
    \mathcal{P}_{\sigma}:= &\{X\in[Z]^\omega: \text{All subsets of}\; X\;
      \text{compute}\; \sigma\; (\text{via}\; \Gamma)\}\\
    =&\{X\in[Z]^\omega: \text{All finite subsets of}\; X\;
      \text{compute}\; \sigma\; (\text{via}\; \Gamma)\}
      \in\Sigma_1^{1,Z}.
  \end{align*}
  By compactness of $\mathcal{A}$, the set of arbitrarily long $\sigma$'s
  must contain a path in $\mathcal{A}$. Then $\{f:(\forall \sigma\prec f)\;
  [\mathcal{P}_\sigma \neq \emptyset]\}$ is a non-empty
  $\Sigma_1^{1,Z}$-set contained in $\mathcal{A}$, $\Rightarrow\Leftarrow$.
\end{frame}

\begin{frame}{Lemma~\ref{lemma:fixed-machine}: Subsets compute no path in
$\mathcal{A}$ (fixed machine)}
  \textbf{Case 1:} For some $n$, GP has only negative solutions for every
  $\sigma\in\omega^n\cap\mathcal{A}$. Since $\mathcal{A}$ is compact, it is
  finitely branching, and we can write $\sigma_0,\ldots,\sigma_m$ for the
  nodes in $\omega^n\cap\mathcal{A}$. Iterate GP across $\sigma_i$ to get
  decreasing subsets of negative solutions
  \[Z=X_0 \supseteq X_1 \supseteq \ldots\supseteq X_m=X,\]

  where $X_{i+1}\in[X_i]^\omega$ is a negative solution by GP under
  inputs $X_i$ and $\sigma_i$. $X_{i+1}$ exists since GP has no positive
  solutions for each $\sigma_i$ under $Z$, giving also no
  positive solutions under $X_i\subseteq Z$. $X=X_m$ works. $\blacksquare$

  \vspace{0.5em}
  Observe that the $X_{i+1}$ constructed at stage $i$ lies in the
  $\Sigma_1^{1,X_i}$-set of negative GP solutions of $X_i$:
  \[\mathcal{N}_{\sigma_i}:= \{X\in[X_i]^\omega: \text{No subsets of}\; X\;
  \text{computes}\; \sigma\; (\text{via}\; \Gamma)\}
  \in\Sigma_1^{1,X_i}.\]
  So $X \in\mathcal{N}_{\sigma_0}\cap\ldots\cap\mathcal{N}_{\sigma_m}$,
  where $\mathcal{N}_{\sigma_i}\in\Sigma_1^{1,X_i}$. We need this
  information later when we strengthen Lemma~\ref{lemma:fixed-machine}
  later.
\end{frame}

\begin{frame}{Subsets compute no path in $\mathcal{A}$ (all
machines)}
  For a fixed machine, Lemma~\ref{lemma:fixed-machine} gives $X$ whose
  subsets compute no paths in $\mathcal{A}$. To find $X$ that works across
  all machines, first fix an enumeration of all machines
  $\Gamma_0,\Gamma_1,\ldots$. Then, iterate Lemma~\ref{lemma:fixed-machine}
  across them to construct decreasing subsets of such $X$
  \[Z= X_0\supseteq X_1\supseteq \ldots,\]
  where $X_s$ works for the first $s$ machines. Then take
  $X=\bigcap_{s\in\omega}X_s$.

  \vspace{1em}
  The problem is if we iterate with the current version of
  Lemma~\ref{lemma:fixed-machine}, we may not end up with an infinite $X$.
  To ensure infiniteness, at stage $s$ of the iteration, we want to choose
  the $X_{s+1}\in[X_s]^\omega$ where $X_{s+1}$ preserves the first
  $s$-elements of $X_s$. We strengthen Lemma~\ref{lemma:fixed-machine} to
  ensure that such segment-preserving $X_{s+1}$ exists.
\end{frame}

\begin{frame}{Strengthening Lemma~\ref{lemma:fixed-machine}}
  To iterate successfully, we strengthen Lemma~\ref{lemma:fixed-machine}
  twice:
  \newtheorem*{lemma-strengthened1*}{Lemma \ref{lemma:fixed-machine}':}
  \begin{lemma-strengthened1*}
    $\mathcal{A}$ compact, contains no $\Sigma_1^{1,Z}$-set
    ($\neq\emptyset$). Fix Turing functional $\Gamma$. Then
    there exists $X\in[Z]^\omega$ such that every subset of $X$ computes
    (via $\Gamma$) no path in $\mathcal{A}$.\\
    \vspace{0.5em}
    Furthermore, $\mathcal{A}$ contains no $\Sigma_1^{1,X}$-set
    ($\neq\emptyset$).
  \end{lemma-strengthened1*}

  \vspace{0.5em}
  \newtheorem*{lemma-strengthened2*}{Lemma \ref{lemma:fixed-machine}'':}
  \begin{lemma-strengthened2*}
    Given any $s\in\omega$, require the $X$ in
    Lemma~\ref{lemma:fixed-machine}' to preserve $Z\restriction s$.
  \end{lemma-strengthened2*}

  \vspace{0.5em}
  Lemma~\ref{lemma:fixed-machine}'' can be directly proven from
  Lemma~\ref{lemma:fixed-machine}' by iterating the latter across all
  subsets of $Z\restriction s$.
\end{frame}

\begin{frame}{Lemma~\ref{lemma:fixed-machine}'': $X\subseteq Z$ that
preserves $Z\restriction s$}
  List the subsets of $Z\restriction s$ as $d_0,\ldots,d_m$. Iterate the
  partially-strengthened lemma across $d_i$'s to get decreasing subsets
  \[Z=X_0 \supseteq X_1 \supseteq X_2 \supseteq\ldots \supseteq X_m.\]

  At stage $i$, Lemma~\ref{lemma:fixed-machine}' is applied with inputs
  $X_i$ and $\Gamma_i$ defined by
  \[\Gamma_i(Y) =\Gamma(Y\cup d_i),\]

  giving $X_{i+1}\in[X_i]^\omega$ whose subsets compute (via
  $\Gamma_i$) no path in $\mathcal{A}$, and where $\mathcal{A}$ contains no
  $\Sigma_1^{1,X_i}$-set ($\neq\emptyset$).

  \vspace{0.5em}
  Choose $X=X_m\cup Z\restriction s$. To see that this $X$ works, first
  observe that since $X$ almost equals $X_m$, $\mathcal{A}$ will
  contain no $\Sigma_1^{1,X}$-set since it contains no
  $\Sigma_1^{1,X_m}$-set. Fix arbitrary $Y\in[X]^\omega$. Then
  $Y\restriction i=d_i$ for some $i$. Stage $i$ ensured that the subsets of
  $Y$ compute no path in $\mathcal{A}$ via $\Gamma_i$. Since $Y\supset
  d_i$, the same is ensured via $\Gamma$. $\blacksquare$

  %Apply the partially-strengthened lemma with input $Z$ and $\Gamma'$,
  %where the latter is defined by
  %\[\Gamma'(Y) =\Gamma(Y\cup Z\restriction s).\]
  %The lemma gives us $X'\in[Z]^\omega$ whose subsets compute (via
  %$\Gamma'$) no path in $\mathcal{A}$, and where $\mathcal{A}$ contains no
  %$\Sigma_1^{1,X}$-set ($\neq\emptyset$). Choose $X=X'\cup Z\restriction
  %s$.

  %\vspace{1em}
  %To see that this $X$ works, fix arbitrary $Y\in[X]^\omega$. Then
  %$Y\restriction i=d_i$ for some $i<m$.
\end{frame}

\begin{frame}{Subsets compute no path in $\mathcal{A}$ (all machines)}
  Iterating Lemma~\ref{lemma:fixed-machine}'' across all machines
  $\Gamma_0,\Gamma_1,\ldots$, we prove:
  \begin{main-thm*}[Harder direction]
    $\mathcal{A}$ compact, contains no $\Sigma_1^1$-subset
    ($\neq\emptyset$). Then there exists infinite set $X$ whose subsets
    compute no path in $\mathcal{A}$.
  \end{main-thm*}

  Construct decreasing subsets of $X$'s from
  Lemma~\ref{lemma:fixed-machine}''
  \[\omega= X_0\supseteq X_1\supseteq\ldots,\]
  where at stage $s$, the lemma is applied with inputs $X_s$ and machine
  $\Gamma_s$, and we require $X_{s+1}$ to preserve the first $s$-elements
  of $X_s$. Take $X=\bigcap_{s\in\omega}X_s$; this is infinite by
  construction. Given $Y\in[X]^\omega$ and arbitary $\Gamma_i$, stage $i$
  of the construction ensures that since $Y\in[X_i]^\omega$, $Y$ does
  computes no path in $\mathcal{A}$ via $\Gamma_i$. $\blacksquare$
\end{frame}

\begin{frame}{$X\subseteq Z$ where $\mathcal{A}$ contains no
$\Sigma_1^{1,X}$-set}
  \begin{lemma-strengthened1*}
    $\mathcal{A}$ compact, contains no $\Sigma_1^{1,Z}$-set
    ($\neq\emptyset$). Fix Turing functional $\Gamma$. Then
    there exists $X\in[Z]^\omega$ such that every subset of $X$ computes
    (via $\Gamma$) no path in $\mathcal{A}$. Furthermore, $\mathcal{A}$
    contains no $\Sigma_1^{1,X}$-set ($\neq\emptyset$).
  \end{lemma-strengthened1*}

  \vspace{0.5em}
  For $\mathcal{A}$ to contain no $\Sigma_1^{1,X}$-set, we choose $X$ from
  Lemma~\ref{lemma:fixed-machine} with this property. Recall
  in the proof of the lemma, the $X$ found lies in
  $\mathcal{N}_{\sigma_0}\cap \ldots\cap\mathcal{N}_{\sigma_m}$,
  where $\mathcal{N}_{\sigma_i}$ is a $\Sigma_1^{1,X_i}$-set and $X_0=Z$.
  By iterating the following theorem $m$-times, it suffices to show

  \vspace{0.5em}
  \newtheorem*{immunity*}{$\Sigma_1^{1}$-immunity basis theorem}
  \begin{immunity*}[Relativized]
    $\mathcal{A}$ compact, contains no $\Sigma_1^{1,Z}$-set
    ($\neq\emptyset$). Fix $\Sigma_1^{1,Z}$-set
    $\mathcal{N}\neq\emptyset$. Then $\mathcal{N}$ contains some $X$ such
    that $\mathcal{A}$ contains no $\Sigma_1^{1,X}$-set ($\neq\emptyset$).
  \end{immunity*}
\end{frame}

\begin{frame}{$\Sigma_1^1$-immunity basis theorem (fixed machine)}
  By restricting computations to a fixed machine, we can even find a
  $\Sigma_1^1$-subset of such $X$'s from any given $\Sigma_1^1$-set.
  \begin{claim}
    $\mathcal{A}$ compact, contains no $\Sigma_1^{1}$-set
    ($\neq\emptyset$). Fix $\Sigma_1^{1}$-predicate $P(X,Y)$,
    $\Sigma_1^{1}$-set $\mathcal{N}_0\neq\emptyset$. Then $\mathcal{N}_0$
    has a $\Sigma_1^{1}$-subset $\mathcal{N}\neq\emptyset$ such that for
    every $X\in\mathcal{N}$, $\mathcal{A}$ contains no $\Sigma_1^{1,X}$-set
    ($\neq\emptyset$) via $P$.
  \end{claim}

  \textbf{Case 1}: For every $X\in\mathcal{N}_0$,
  $\{Y:P(X,Y)\}\subseteq\mathcal{A}$. Then $\bigcup_{X\in\mathcal{N}_0}
  \{Y:P(X,Y)\}$ will be a $\Sigma_1^1$-subset of $\mathcal{A}$,
  $\Rightarrow\Leftarrow$.

  \vspace{0.5em}
  \textbf{Case 2}: For some $X_0\in\mathcal{N}_0$, $\{Y:P(X_0,Y)\}$
  contains a path $Y_0$ outside $\mathcal{A}$. By compactness of
  $\mathcal{A}$, $Y_0$ has an initial segment $\sigma$ outside
  $\mathcal{A}$. This $\Sigma_1^1$-subset of $\mathcal{N}_0$ works:
  \[\mathcal{N}:= \{X\in\mathcal{N}_0: (\exists Y\succ\sigma)\; P(X,Y)\}.\;
  \blacksquare\]

  Note that every step of this proof can be relativized to arbitrary $Z$.
\end{frame}

\begin{frame}{$\Sigma_1^1$-immunity basis theorem}
  The immunity basis theorem asks for an $X$ whose $\Sigma_1^1$-trees do
  not lie in $\mathcal{A}$ regardless of the machine used in the
  computation. The previous claim gives us enough $X$'s that work for any
  given $\Sigma_1^1$-machine to guarantee that some $X$ works for all
  machines.

  \vspace{1em}
  \begin{immunity*}
    $\mathcal{A}$ compact, contains no $\Sigma_1^{1}$-set
    ($\neq\emptyset$). Fix $\Sigma_1^{1}$-set
    $\mathcal{N}\neq\emptyset$. Then $\mathcal{N}$ contains some $X$ such
    that $\mathcal{A}$ contains no $\Sigma_1^{1,X}$-set ($\neq\emptyset$).
  \end{immunity*}

  \vspace{1em}
  For each $\Sigma_1^1$-predicate $P$, let $\mathcal{U}_P$ be the union of
  all the $\Sigma_1^1$-sets where every $X$ in the set gives a path outside
  $\mathcal{A}$ via $P$. From previous claim, $\mathcal{U}_P$ is dense
  (under Gandy-Harrington topology where basic open sets are
  $\Sigma_1^1$-sets). Since this topology is Baire,
  $\bigcap_P\mathcal{U}_P$ is is non-empty. Any
  $X\in\bigcap_P\mathcal{U}_P$ works. $\blacksquare$
\end{frame}

\begin{frame}{$\Sigma_1^{1,Z}$-immunity basis theorem (relativized)}
  Every step in the proof can be directly relativized to arbitary $Z$ since
  the relativized Gandy-Harrington topology is Baire, and the proof of the
  claim used can also be directly relativized.

  \vspace{0.5em}
  \begin{immunity*}[Relativized]
    $\mathcal{A}$ compact, contains no $\Sigma_1^{1,Z}$-set
    ($\neq\emptyset$). Fix $\Sigma_1^{1,Z}$-set
    $\mathcal{N}\neq\emptyset$. Then $\mathcal{N}$ contains some $X$ such
    that $\mathcal{A}$ contains no $\Sigma_1^{1,X}$-set ($\neq\emptyset$).
  \end{immunity*}
\end{frame}

\begin{frame}{WWKL $\nleq_{\text{soc}}$ RT}
  \begin{itemize}
    \item Fix tree $T\subset 2^{<\omega}$ such that
      $[T]\subset 2^\omega$ is compact, has positive measure, and
      contains no $\Sigma_1^1$-subset ($\neq\emptyset$).
    \item Assuming such $T$ exists, fix RT instance
      $c:[\omega]^n\rightarrow k$ and assume every $c$-homogeneous set
      computes a path in $T$.
    \item Now given arbitrary infinite set $Z$, relativizing $c$ with
      respect to $Z$ gives us a $c$-homogeneous set $X\in[Z]^\omega$.
      By assumption, every element in $[X]^\omega$ computes a path in $T$.
    \item Thus every infinite set $Z$ has an infinite subset $X$ whose
      infinite subsets compute a path in $T$.
    \item By the harder direction of the main theorem, $[T]$ must contain a
      non-empty $\Sigma_1^1$-subset, $\Rightarrow\Leftarrow$.
      $\blacksquare$
    \item It remains to prove such $T$ exists.
  \end{itemize}
\end{frame}

\begin{frame}{$[T]$ with positive measure, containing no
$\Sigma_1^1$-subset}
  \begin{itemize}
    \item Let $T$ be set of Martin-Lof randoms relativized to Kleene's $O$.
    \item Then $\mu([T])>0$.
    \item $T$ has no non-empty $\Sigma_1^1$-subset because every
      $\Sigma_1^1$-set of $2^\omega$ contains an $O$-recursive.
      $\blacksquare$
  \end{itemize}
\end{frame}

\begin{frame}{Possible additional slides}
  \begin{itemize}
    \item Prove Gandy-Harrington topology (relativized) is Baire
    \item Prove every $\Sigma_1^1$-subset of $2^\omega$ contains an
      $O$-recursive
    \item Prove Galvin-Prikry for open sets
  \end{itemize}
\end{frame}
