\begin{frame}{$\mathcal{A}$ contains $\Sigma_1^1$ subset ($\neq\emptyset$)
$\Rightarrow$ $\mathcal{A}$ $\Pi_1^0$-encodable}
  \begin{itemize}
    \item Fix $\mathcal{A}$ containing $\mathcal{B} =\{X: (\exists
      Y)(\forall s)\; R(X,Y,s)\} \neq\emptyset$.
    \item Fix $X\in\mathcal{B}$, $Y$ such that $(\forall s)\;R(X,Y,s)$.
    \item Define $f(n)=\max(X(n),Y(n))$.
    \item Observe that $f$ acts as $\Pi_1^0$-modulus of $\mathcal{A}$, in
      the sense that if $g\geq f$, then $\mathcal{A}\supseteq[T]
      \neq\emptyset$ for some $T\leq_T g$ whose nodes are bounded by $g$:
    \item Given $g\geq f$, the set
      \[\{X\leq g:(\forall s)(\exists \sigma\in 2^s, \sigma\leq g)(\forall
      s'<s)\; R(X,\sigma,s')\}\]
      contains $X$ and is a $\Pi_1^{0,g}$-subset contained in
      $\mathcal{A}$.
    \item Given arbitrary $X\in[\omega]^\omega$, remove enough elements to
      get a sparse $Y\in[X]^\omega$ where $p_Y>f$.
    \item Then $p_Y$ witnesses $\Pi_1^0$-encodability of $\mathcal{A}$ for
      $X$.
  \end{itemize}
\end{frame}

\begin{frame}{$\mathcal{A}$ $\Pi_1^0$-encodable $\Rightarrow \mathcal{A}$
contains $\Sigma_1^1$ subset ($\neq\emptyset$)}
  \begin{itemize}
    \item Fix compact $\mathcal{A}$ with no $\Sigma_1^1$-subset
      ($\neq\emptyset$). Construct $X\in[\omega]^\omega$ such that for each
      $Y\in[X]^\omega$, every $Y$-recursive tree ($\neq\emptyset$) contains
      a path outside $\mathcal{A}$.

    \item Fix enumeration of Turing functionals on trees
      $\Gamma_s:\omega^{<\omega} \rightarrow \omega^{<\omega}$.

    \item Construct $X=\bigcap_{s\in\omega}X_s$, where
      $\omega=X_0\supset X_1\supset\ldots$, $X_{s+1}\restriction
      s=X_s\restriction s$.

    \item At stage $s$, remove enough elements beyond the first $s$
      elements of $X_s$ to get $X_{s+1}\in[X_s]^\omega$ such that for each
      $Y\in[X_{s+1}]^\omega$, if $[\Gamma_s^Y]\neq\emptyset$ then it
      contains a path outside $\mathcal{A}$.

    \item Set $|X|=\omega$; this $X$ satisfies the desired property.

    \item The hard part is proving such $X_{s+1}\in[X_s]^\omega$ exists.
  \end{itemize}
\end{frame}

\begin{frame}{Compute no path in $\mathcal{A}$ or same branch
outside $\mathcal{A}$}
  If we do not require $X_{s+1}\restriction s=X_s\restriction s$, we can
  find $X_{s+1}\in[X_s]^\omega$ with stronger property:

  \begin{lemma}[Compute no path in $\mathcal{A}$ or same branch
  outside $\mathcal{A}$]
  \label{lemma:all-outside-A}
    Fix $X_0\in[\omega]^\omega$, compact
    $\mathcal{A}\subseteq\omega^\omega$ with no $\Sigma_1^{1,X_0}$ subset
    ($\neq\emptyset$), $\Gamma:\omega^{<\omega} \rightarrow
    \omega^{<\omega}$. There exists $X\in[X_0]^\omega$ such that:

    \begin{itemize}
      \item Either every $Y\in[X]^\omega$ computes only paths outside
        $\mathcal{A}$ via $\Gamma$,
      \item Or for every $n\in\omega$, some $\sigma\in\omega^n$ with
        $[\sigma]\cap\mathcal{A}=\emptyset$ is computed by every
        $Y\in[X]^\omega$ via $\Gamma$.
    \end{itemize}
  \end{lemma}
\end{frame}

\begin{frame}{$\mathcal{A}$ $\Pi_1^0$-encodable $\Rightarrow \mathcal{A}$
contains $\Sigma_1^1$ subset ($\neq\emptyset$)}
  \begin{itemize}
    \item Want $X\in[Z]^\omega$ with $X\restriction s=Z\restriction s$
      and such that for each $Y\in[X]^\omega$, if $[\Gamma^Y]\neq\emptyset$
      then it contains a path outside $\mathcal{A}$.
    \item List subsets of $Z\restriction s$: $\{D_0,\ldots,D_m\}$.
    \item Apply Lemma~\ref{lemma:all-outside-A} once for each $D_i$:
  \end{itemize}
\end{frame}

\begin{frame}{Compute no path in $\mathcal{A}$ or same branch
  outside $\mathcal{A}$}
  \begin{itemize}
    \item Proving Lemma 1 uses $\Sigma_1^1$-immunity basis theorem and
      Galvin-Prikry
  \end{itemize}
\end{frame}

\begin{frame}{$\Sigma_1^1$-immunity basis theorem}
  \begin{itemize}
    \item Proving this theorem uses fact that relativized Gandy-Harrington
      topoloty is Baire space
  \end{itemize}
\end{frame}

\begin{frame}{WWKL $\nleq_{\text{soc}}$ RT}
  \begin{itemize}
    \item Fix $T\subseteq 2^{<\omega}$ with $\mu(T)>0$, $[T]$
      has no non-empty $\Sigma_1^1$ subset
    \item We prove such $T$ exists
    \item Fix RT instance $c:[\omega]^n\rightarrow k$ and assume every
      solution $H$ computes a path through $T$, i.e. $[T]$ has non-empty
      $\Pi_1^{0,H}$ subset
    \item Thus every $X\in[\omega]^\omega$ has a subset
      $H\in[X]^\omega$ such that $[T]$ has non-empty $\Pi_1^{0,H}$ subset
    \item We say that $T$ is $\Pi_1^0$-encodable
    \item We will prove that being $\Pi_1^0$-encodable is equivalent to
      containing non-empty $\Sigma_1^1$ subset, $\rightarrow\leftarrow$
  \end{itemize}
\end{frame}

\begin{frame}{There exists $[T]$ with no $\Sigma_1^1$ subset, $\mu(T)>0$}
  \begin{itemize}
    \item Let $T$ be set of Martin-Lof randoms relativized to Kleene's $O$
    \item Then $\mu(T)>0$
    \item $T$ has no non-empty $\Sigma_1^1$ subset because every
      $\Sigma_1^1$ subset of $2^\omega$ contains an element recursive in
      $O$
  \end{itemize}
\end{frame}

\begin{frame}{Extra slides}
  \begin{itemize}
    \item Set of oracles computing a fixed non-recursive has 0 measure
    \item Gandy-Harrington topology (relativized) is Baire
    \item Every $\Sigma_1^1$ subset of $2^\omega$ contains an element
      recursive in Kleene's $O$
    \item Galvin-Prikry
  \end{itemize}
\end{frame}
