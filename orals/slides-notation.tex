\begin{frame}{Notations}
  \begin{itemize}
    \item $\omega=\mathbb{N}$, $a,b,\ldots\in\omega$,
      $A,B,\ldots\subseteq\omega$, $\sigma,\tau\in\omega^{<\omega}$,
      $\mathcal{A},\mathcal{B},\ldots\subseteq\omega^\omega$.
    \item Trees $T$ are subsets of $\omega^{<\omega}$ closed under initial
      segments; i.e. $\sigma^{\frown}n\in T \rightarrow \sigma\in T$.
    \item Paths are functions $f:\omega\rightarrow\omega$. A tree contains
      a path $f$ iff $T$ contains every initial segment $f\restriction n$
      of $f$.
    \item $[T] :=\{f:f\; \text{is a path in}\; T\}$, $[\sigma]:=\{f:f\;
      \text{is a path containing}\; \sigma\}$.
    \item Turing functionals on trees are partial-recursive functionals
      $\Gamma:2^{<\omega}\rightarrow\omega^{<\omega}$ where
      $\tau\in\Gamma^{\sigma^\frown n} \rightarrow
      \tau\in\Gamma^{\sigma}$. $\Gamma$ can be extended naturally to
      $\Gamma:2^\omega\rightarrow\omega^{\omega}$.
    \item Given set $A$ and cardinal $\kappa$, $[A]^\kappa
      :=\{B\subseteq A: |B|=\kappa\}$.
    \item Given $A\in[\omega]^\omega$, can think of $A$ as its characteristic
      function $c_A:\omega\rightarrow\{0,1\}$, or as their principal
      function $p_A:\omega\rightarrow\omega$ which is a strictly increasing
      function enumerating $A$.
  \end{itemize}
\end{frame}

\begin{frame}{Ramsey's Theorem ($\text{RT}$)}
  \begin{define*}[$c$-homogeneous]
    Given $k$-coloring $c:[\omega]^n\rightarrow k$, a subset
    $A\subseteq\omega$ is $c$-homogeneous if every $n$-tuple over $A$ is
    given the same color by $c$.
  \end{define*}

  \vspace{1em}
  \begin{thm*}[Ramsey's Theorem]
    Fix $n,k\leq1$. $\text{RT}_k^n$ is the statement\\
    ``Every $k$-coloring $c:[\omega]^n\rightarrow k$ has an infinite
    $c$-homogeneous set.''\\
    RT is the statement $(\forall n)(\forall k)\; \text{RT}_k^n$.
  \end{thm*}

  \vspace{1em}
  RT asserts homogeneity exists.
\end{frame}

\begin{frame}{Randomness}
  Intuitively, a path $f\in2^\omega$ is random if it cannot be ``caught''
  by a descending series of sub-trees $T\subseteq2^{<\omega}$, because
  these trees reveal information on the sequences caught by all of them. A
  series of trees is eligible to judge for randomness only if it shrinks
  fast enough for the set of paths caught in all of them to be ``sparse'';
  if too many paths are caught the information on them is not useful.

  \vspace{1em}
  One way of formalizing the notion of a tree that is not sparse is:
  \begin{define*}
    A (measurable) binary tree $T\subseteq2^{<\omega}$ has positive measure
    if
    \[\lim_s \frac{|\{\sigma\in T: |\sigma|=s\}|}{2^s} >0.\]
  \end{define*}
\end{frame}

\begin{frame}{WWKL asserts Randomness}
  A series of sub-trees $T_0\subseteq T_1\subseteq\ldots$ is eligible to
  test for randomness if the limit of their measures is 0. Since there are
  only countably many Turing machines that can give randomness tests,
  and the class of non-randoms from each test has measure 0, and the
  measure of a countable union of (measurable) classes of measure 0 is
  still 0, the class of non-randoms has measure 0.

  \vspace{2em}
  Thus if a (measurable) tree has positive measure, it must contain a
  random. Weak Weak Konig's Lemma (WWKL) asserts that at least one of its
  paths is random by stating
  \begin{thm*}[Weak Weak Konig's Lemma]
    Every binary tree with positive measure contains a path.
  \end{thm*}
\end{frame}

\begin{frame}{WWKL, WKL, KL}
  The paths of an instance of Weak Konig's Lemma (WKL), however, may
  all be non-randoms, because WKL instances allow for binary trees with
  measure 0, by stating
  \begin{thm*}[Weak Konig's Lemma]
    Every binary tree with infinite nodes contains a path.
  \end{thm*}

  \vspace{2em}
  Konig's Lemma claims more, asserting that paths for non-binary
  trees also exist as long as they are finitely branching: 
  \begin{thm*}[Konig's Lemma]
    Every finitely-branching tree contains a path.
  \end{thm*}
\end{frame}

\begin{frame}{Homogeneity versus Randomness}
  A natural question to ask is which of the statements RT, KL, WKL, WWKL is
  stronger. Intuitively, WKL is stronger than WWKL, because a binary tree
  of positive measure has infinite nodes, so if all binary trees with
  infinite nodes have a path, then all binary trees with positive measure
  have one too.

  \vspace{0.5em}
  A more interesting question is
  \newtheorem*{question*}{Question}
  \begin{question*}
    Which notion is stronger, homogeneity or randomness?
  \end{question*}
  If a given a path is homogeneous, does that say anything about how random
  it is? Likewise, if a given path is random, can we use it to find
  homogeneous paths?

  \vspace{0.5em}
  Since the notions of homogeneity and randomness are captured by RT and
  WWKL respectively, we are essentially asking which of RT and WWKL is
  stronger.
\end{frame}

\begin{frame}{Strongly Omnisciently Computably Reducible
($\leq_{\text{soc}}$)}
  \begin{itemize}
    \item Define problem, solution
    \item Define $\leq_{\text{soc}}$ on WWKL, RT
    \item WWKL $\leq_{\text{soc}}$ WKL: From definition.
    \item WKL $\leq_{\text{soc}}$ KL: From definition.
  \end{itemize}
\end{frame}

\begin{frame}{KL, RT $\leq_{\text{soc}}$ WKL}
  WKL is powerful under $\leq_{\text{soc}}$. Given any KL or RT
  instance, fix one of its solutions, and choose the WKL instance whose only
  path codes that solution as an path in $2^\omega$. By definition of
  $\leq_{\text{soc}}$, the given instance reduces to this WKL instance.

  \vspace{1em}
  For example, given a KL instance, fix one of its solutions
  $f:\omega\rightarrow\omega$, code this solution as a path $g\in2^\omega$
  in a recoverable manner.

  \vspace{1em}
  For instance, if $f$ is $2,9,3,\ldots$, we can code $f$ as the binary
  sequence $g$ which starts with $2$ ones followed by a zero, then $9$ ones
  followed by a zero, then $3$ ones followed by a zero, and so on.

  \vspace{1em}
  Similarly, any solution of an RT instance can be considered as an
  infinite subset of $\omega$. We can use its characteristic function as a
  path in $2^\omega$.
\end{frame}

\begin{frame}{Homogeneity ``independent from'' Randomness}
  Dependencies under $\leq_{\text{soc}}$
  \vspace{2em}

  \begin{center}
    \begin{tikzpicture}[node distance=4cm,auto,thick,>=latex']
      \node (KL) {KL$\leftrightarrow$WKL};
      \node (WWKL) [below right of=KL] {WWKL};
      \node (RT) [below left of=KL] {RT};
      \draw[->] (KL) -- (RT);
      \draw[->] (KL) -- (WWKL);
      %\draw [->,red] (RT) -- coordinate (m) (WWKL);
      %\draw[shift={(m)},red](-0.1,-0.1)--(0.1,+0.1);
    \end{tikzpicture}
  \end{center}
\end{frame}
