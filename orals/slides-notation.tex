\begin{frame}{Notations}
  \begin{itemize}
    \item $\omega=\mathbb{N}$, $a,b,\ldots\in\omega$,
      $A,B,\ldots\subseteq\omega$, $\sigma,\tau\in\omega^{<\omega}$,
      $\mathcal{A},\mathcal{B},\ldots\subseteq\omega^\omega$.
    \item Trees $T$ are subsets of $\omega^{<\omega}$ closed under initial
      segments; i.e. $\sigma^{\frown}n\in T \rightarrow \sigma\in T$.
    \item Paths are functions $f:\omega\rightarrow\omega$. A tree contains
      a path $f$ iff $T$ contains every initial segment $f\restriction n$
      of $f$.
    \item $[T] :=\{f:f\; \text{is a path on}\; T\}$.
    \item Turing functionals on trees are partial-recursive functionals
      $\Gamma:2^{<\omega}\rightarrow\omega^{<\omega}$ where
      $\tau\in\Gamma^{\sigma^\frown n} \rightarrow
      \tau\in\Gamma^{\sigma}$. $\Gamma$ can be extended naturally to
      $\Gamma:2^\omega\rightarrow\omega^{\omega}$.
    \item Given set $A$ and cardinal $\kappa$, $[A]^\kappa
      :=\{B\subseteq A: |B|=\kappa\}$.
    \item Given $A\in[\omega]^\omega$, can think of $A$ as its characteristic
      function $c_A:\omega\rightarrow\{0,1\}$, or as their principal
      function $p_A:\omega\rightarrow\omega$ which is a strictly increasing
      function enumerating $A$.
  \end{itemize}
\end{frame}

\begin{frame}{Ramsey's Theorem ($\text{RT}$)}
  \begin{define*}[$c$-homogeneous]
    Given $k$-coloring $c:[\omega]^n\rightarrow k$, a subset
    $A\subseteq\omega$ is $c$-homogeneous if every $n$-tuple over $A$ is
    given the same color by $c$.
  \end{define*}

  \vspace{1em}
  \begin{define*}[Ramsey's Theorem]
    Fix $n,k\leq1$. $\text{RT}_k^n$ is the statement\\
    ``Every $k$-coloring $c:[\omega]^n\rightarrow k$ has an infinite
    $c$-homogeneous set.''\\
    RT is the statement $(\forall n)(\forall k)\; \text{RT}_k^n$.
  \end{define*}

  \vspace{1em}
  RT asserts homogeneity exists.
\end{frame}

\begin{frame}{Randomness}
  \begin{itemize}
    \item What is 1-random
    \item Lebesgue Measure
  \end{itemize}
\end{frame}

\begin{frame}{Strongly Omnisciently Computably Reducible
  ($\leq_{\text{soc}}$)}
  \begin{itemize}
    \item Define problem, solution
    \item Define $\leq_{\text{soc}}$ on WWKL, RT
    \item WWKL $\leq_{\text{soc}}$ WKL: From definition.
    \item WKL $\leq_{\text{soc}}$ KL: From definition.
  \end{itemize}
\end{frame}

\begin{frame}{KL $\leq_{\text{soc}}$ WKL}
  Given a KL instance with at least one path $f:\omega\rightarrow\omega$,
  code this path in $\omega^\omega$ as a path $g$ in $2^\omega$ in a
  recoverable manner.

  \vspace{2em}
  For example, if $f$ is $2,9,3,\ldots$, we can code $f$ as the binary
  sequence $g$ which starts with $2$ ones followed by a zero, then $9$ ones
  followed by a zero, then $3$ ones followed by a zero, and so on.

  \vspace{2em}
  Consider the WKL instance that has $g$ as the only path. Then every
  path of this instance computes a path in the KL instance.
\end{frame}

\begin{frame}{RT $\leq_{\text{soc}}$ WKL}
  \begin{itemize}
    \item Fix an RT instance $c:[\omega]^n\rightarrow k$.
    \item Define the WKL instance $T\subseteq 2^{<\omega}$ by
      \[\sigma\in T \Leftrightarrow \{n:\sigma(n)=1\}\; \text{is
      $c$-homogeneous}.\]
    \item Then every path in 
  \end{itemize}
\end{frame}

\begin{frame}{Homogeneity ``independent from'' Randomness}
  Dependencies under $\leq_{\text{soc}}$
  \vspace{2em}

  \begin{center}
    \begin{tikzpicture}[node distance=4cm,auto,thick,>=latex']
      \node (KL) {KL$\leftrightarrow$WKL};
      \node (WWKL) [below right of=KL] {WWKL};
      \node (RT) [below left of=KL] {RT};
      \draw[->] (KL) -- (RT);
      \draw[->] (KL) -- (WWKL);
      %\draw [->,red] (RT) -- coordinate (m) (WWKL);
      %\draw[shift={(m)},red](-0.1,-0.1)--(0.1,+0.1);
    \end{tikzpicture}
  \end{center}
\end{frame}
