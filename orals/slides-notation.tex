\begin{frame}{Ramsey's Theorem ($\text{RT}$)}
  \begin{define*}[$c$-homogeneous]
    Given $k$-coloring $c:[\omega]^n\rightarrow k$, a subset
    $A\subseteq\omega$ is $c$-homogeneous if every $n$-tuple over $A$ is
    given the same color by $c$.
  \end{define*}

  \vspace{1em}
  \begin{thm*}[Ramsey's Theorem]
    Fix $n,k\leq1$. $\text{RT}_k^n$ is the statement that every $k$-coloring
    $c:[\omega]^n\rightarrow k$ has an infinite $c$-homogeneous set.\\
    RT is the statement $(\forall n)(\forall k)\; \text{RT}_k^n$.
  \end{thm*}

  \vspace{1em}
  RT asserts homogeneity exists.
\end{frame}

\begin{frame}{Randomness}
  Intuitively, a path $f\in2^\omega$ is random if it cannot be ``caught''
  by any descending enumeration of trees $2^{<\omega}\supseteq T_0\supseteq
  T_1\supseteq \ldots$, because such trees would reveal information on the
  sequences caught by all of them. A descending sequence is eligible to judge
  for randomness only if the trees shrink fast enough for the class of paths
  caught by all of them to be ``sparse'', because if too many paths are
  caught, the information on the paths is not useful.

  \vspace{1em}
  One way of formalizing the notion of a sparseness of a tree is:
  \begin{define*}
    A (measurable) binary tree $T\subseteq2^{<\omega}$ has positive measure
    if
    \[\lim_s \frac{|\{\sigma\in T: |\sigma|=s\}|}{2^s} >0.\]
  \end{define*}
\end{frame}

\begin{frame}{WWKL asserts Randomness}
  A series of trees $T_0\supseteq T_1\supseteq\ldots$ is eligible to
  test for randomness if their limit is null. Since there are
  only countably many Turing machines that can give randomness tests,
  and the class of non-randoms from each test is null, and
  a countable union of null sets is still null, the class of non-randoms
  is null.

  \vspace{2em}
  Thus if a (measurable) tree has positive measure, it must contain a
  random. Weak Weak Konig's Lemma (WWKL) asserts that at least one of its
  paths is random by stating
  \begin{thm*}[Weak Weak Konig's Lemma]
    Every binary tree with positive measure contains a path.
  \end{thm*}
\end{frame}

\begin{frame}{WWKL, WKL, KL}
  The reals of a WKL problem however, may all be non-randoms, because WKL
  includes also binary trees which are null:
  \begin{thm*}[Weak Konig's Lemma]
    Every binary tree with infinite nodes contains a path.
  \end{thm*}

  \vspace{2em}
  Konig's Lemma includes even trees that are not binary:
  \begin{thm*}[Konig's Lemma]
    Every finitely-branching tree with infinite nodes contains a path.
  \end{thm*}
\end{frame}

\begin{frame}{Homogeneity versus Randomness}
  A natural question to ask is which of the statements RT, KL, WKL, WWKL is
  stronger. Intuitively, WKL is stronger than WWKL, because a binary tree
  of positive measure has infinite nodes, so if all binary trees with
  infinite nodes have a path, then those with positive measure must
  have one too. A question that is harder to answer is:
  \newtheorem*{question*}{Question}
  \begin{question*}
    Which notion is stronger, homogeneity or randomness?
  \end{question*}
  If a given a path is homogeneous, does that say anything about how random
  it is? Likewise, if a given path is random, will it necessarily provide
  information on finding homogeneous paths?

  \vspace{0.5em}
  Since the notions of homogeneity and randomness are captured by RT and
  WWKL respectively, we are essentially asking which of RT and WWKL is
  stronger.
\end{frame}

\begin{frame}{Problem, Instance, Solution}
  We formalize the notion of problems and the strength between them.

  \vspace{1em}
  \begin{define*}[Problem, instance, solution]
    A mathematical \textbf{problem} is a collection of \textbf{instances},
    with a collection of \textbf{solutions} for each instance.
  \end{define*}

  \vspace{1em}
  For example, RT is a problem; its instances are the collections of
  colorings $c:[\omega]^n\rightarrow k$ for every $n,k\in\omega$, and the
  solutions of each $c$ are the class of $c$-homogeneous paths.

  \vspace{1em}
  Similarly, WWKL is a problem; its instances are the collections of binary
  trees with positive measure, and the solutions of each instance is the
  class of paths in the tree.
\end{frame}

\begin{frame}{Strongly Omnisciently Computably Reducible
($\leq_{\text{soc}}$)}
  \begin{define*}[Strongly Omnisciently Computably Reducible]
    Problem \textbf{P} is \textit{soc}-reducible to problem \textbf{Q}
    (written \textbf{P} $\leq_{\text{soc}}$ \textbf{Q}) if for every
    \textbf{P}-instance \textit{I}, there is a \textbf{Q}-instance
    \textit{J} such that every solution to \textit{J} computes a solution
    to \textit{I}.
  \end{define*}

  $\leq_{\text{soc}}$ successfully captures this expected relation:
  \begin{center}
    WWKL $\leq_{\text{soc}}$ WKL $\leq_{\text{soc}}$ KL.
  \end{center}

  \vspace{0.5em}
  \textbf{WWKL $\leq_{\text{soc}}$ WKL:} Given a binary tree with
  positive measure, this tree has infinite branches and is therefore an
  instance of WKL. The desired relation follows from definition of
  $\leq_{\text{soc}}$.

  \vspace{0.5em}
  \textbf{WKL $\leq_{\text{soc}}$ KL:} Given a binary tree with infinite
  branches, this tree is finitely-branching and therefore an instance of
  WKL. The desired relation follows from definition of $\leq_{\text{soc}}$.
\end{frame}

\begin{frame}{KL, RT $\leq_{\text{soc}}$ WKL}
  WKL is powerful under $\leq_{\text{soc}}$. Given any KL or RT
  instance, fix one of its solutions, and choose the WKL instance whose only
  path codes that solution as an path in $2^\omega$. By definition of
  $\leq_{\text{soc}}$, the given instance reduces to this WKL instance.

  \vspace{1em}
  For example, given a KL instance, fix one of its solutions
  $f:\omega\rightarrow\omega$, code this solution as a path in $2^\omega$
  in a recoverable manner.

  \vspace{1em}
  For instance, if $f$ is $2,9,3,\ldots$, we can code $f$ as the path
  $110$ $1111111110$ $1110\ldots$.

  \vspace{1em}
  Similarly, any solution of an RT instance can be seen as an infinite
  subset of $\omega$. Its characteristic function will be a path in
  $2^\omega$.
\end{frame}

\begin{frame}{Homogeneity ``independent from'' Randomness}
  Dependencies under $\leq_{\text{soc}}$
  \vspace{2em}

  \begin{center}
    \begin{tikzpicture}[node distance=4cm,auto,thick,>=latex']
      \node (KL) {KL$\leftrightarrow$WKL};
      \node (WWKL) [below right of=KL] {WWKL};
      \node (RT) [below left of=KL] {RT};
      \draw[->] (KL) -- (RT);
      \draw[->] (KL) -- (WWKL);
      %\draw [->,red] (RT) -- coordinate (m) (WWKL);
      %\draw[shift={(m)},red](-0.1,-0.1)--(0.1,+0.1);
    \end{tikzpicture}
  \end{center}
\end{frame}
