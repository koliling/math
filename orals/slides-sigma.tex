\begin{frame}{Galvin-Prikry}
  \begin{fact*}[Galvin-Prikry, relativized to $Z$]
    Given infinite set $Z\subseteq\omega$, node $\sigma\in\omega^{<\omega}$,
    and Turing functional $\Phi:2^\omega\rightarrow \omega^{<\omega}$,
    there exists an infinite subset $X\subseteq Z$ where one of these
    holds:\\

    \vspace{1em}
    \textbf{Positive-solution:} All inf. subsets of $X$ compute (via
    $\Phi$) $\sigma$ \\
    \textbf{Negative-solution:} All inf. subsets of $X$ don't compute (via
    $\Phi$) $\sigma$
  \end{fact*}

  \vspace{1em}
  \textbf{Pf (sketch):} The class of subsets of $Z$ that compute (via
  $\Phi$) $\sigma$ can be coded to form an open set in $2^\omega$, by the
  finite use principle. Being open provides enough ``structure'' for
  regions of homogeneity to exist. $\square$
\end{frame}

\begin{frame}{Set whose subsets cannot compute reals in $\mathcal{C}$}
  \newtheorem*{main-lemma*}{Main Lemma}
  \begin{main-lemma*}[Monin, Patey, 2016]
    $\mathcal{C}\subseteq\omega^\omega$ compact, contains no non-empty
    $\Sigma_1^{1,Z}$-subset, $\Phi:2^{\omega}\rightarrow \omega^{<\omega}$
    Turing functional that converges on initial segments. Then there is an
    infinite $X\subseteq Z$ whose subsets cannot compute (via $\Phi$) any
    real in $\mathcal{C}$.
  \end{main-lemma*}

  \vspace{0.5em}
  \textbf{Pf:} \textbf{Case 0:} There is a node outside
  $\mathcal{C}$ with a positive GP-solution $X$. This $X$ works since
  all its subsets compute (via $\Phi$) a node outside $\mathcal{C}$.
\end{frame}

\begin{frame}{Set whose subsets cannot compute reals in $\mathcal{C}$
(cont.)}
  \textbf{Case 1:} Every node outside $\mathcal{C}$ has only negative
  GP-solutions, and $\mathcal{C}$ has arbitrarily long nodes with
  positive GP-solutions. By compactness of $\mathcal{C}$, this set of
  nodes contains a real in $\mathcal{C}$. Then
  \begin{align*}
    &\{f:(\forall \sigma\prec f)\; [\sigma\; \text{has
    positive GP-solutions}]\}\\
    =&\{f:(\forall \sigma\prec f)\; (\exists \text{ inf } X\subseteq
    Z)(\forall \text{ finite } X'\subseteq X)\;
    [\Phi^{X'}(\sigma)\downarrow]\}
  \end{align*}
  is a non-empty $\Sigma_1^{1,Z}$-set contained
  in $\mathcal{C}$, $\Rightarrow\Leftarrow$.
\end{frame}

\begin{frame}{Set whose subsets cannot compute reals in $\mathcal{C}$
(cont.)}
  \textbf{Case 2:} For some $n$, every node in $\mathcal{C}$ of length
  $n$ has only negative GP-solutions. Note that there are only fintely many
  such nodes $\sigma_0,\ldots,\sigma_m$ since $\mathcal{C}$ is compact.
  Iterate GP across them to get decreasing subsets of negative solutions
  \[Z=X_0 \supseteq X_1 \supseteq \ldots\supseteq X_m=X,\]
  where $X_{i+1}\subseteq X_i$ is a negative GP-solution under inputs
  $X_i$ and $\sigma_i$. $X_{i+1}$ exists since GP gives no positive
  solutions for each $\sigma_i$ under $Z$, giving also no positive
  solutions under $X_i\subseteq Z$.
  
  \vspace{1em}
  Take $X=X_m$. Then every subset of $X$ cannot compute (via $\Phi$) any
  node in $\mathcal{C}$ of length exceeding $n$, and in particular,
  cannot compute any real in $\mathcal{C}$. $\blacksquare$
\end{frame}

\begin{frame}{Set whose subsets cannot compute reals in $\mathcal{C}$
(relativized)}
  \begin{main-lemma*}[\textcolor{red}{Relativized}]
    $\mathcal{C}\subseteq\omega^\omega$ compact, contains no non-empty
    $\Sigma_1^{1,Z}$-subset, $\Phi:2^{\omega}\rightarrow \omega^{<\omega}$
    Turing functional that converges on initial segments. Then there is an
    infinite $X\subseteq Z$ whose subsets cannot compute (via $\Phi$) any
    real in $\mathcal{C}$.\\
    \vspace{0.5em}
    \textcolor{red}{Furthermore, we can choose $X$ such that $\mathcal{C}$
    also contains no nonempty $\Sigma_1^{1,X}$-subset.}
  \end{main-lemma*}

  \vspace{1em}
  \textbf{Pf:} In the proof of the non-relativized version, $X$ was chosen
  from a $\Sigma_1^{1,Z}$-set. The relativized immunity-basis theorem
  asserts that every such set contains an element $X$ where $\mathcal{C}$
  contains no $\Sigma_1^{1,X}$-set. $\blacksquare$
\end{frame}

\begin{frame}{Preserving initial segment}
  \begin{main-lemma*}[Relativized, \textcolor{red}{Segment-preserving}]
    $\mathcal{C}\subseteq\omega^\omega$ compact, contains no non-empty
    $\Sigma_1^{1,Z}$-subset, $\Phi:2^{\omega}\rightarrow \omega^{<\omega}$
    Turing functional that converges on initial segments. Then there is an
    infinite $X\subseteq Z$ whose subsets cannot compute (via $\Phi$) any
    real in $\mathcal{C}$.\\
    \vspace{0.5em}
    Furthermore, we can choose $X$ such that $\mathcal{C}$ also contains no
    nonempty $\Sigma_1^{1,X}$-subset.\\
    \vspace{0.5em}
    \textcolor{red}{Finally, given $s\in\omega$, $X$ can be chosen to preserve
    $Z\restriction s$.}
  \end{main-lemma*}

  \vspace{1em}
  \textbf{Pf:} The idea is to iterate the relativized lemma across all
  subsets of $Z\restriction s$. Iteration is possible after relativization.
\end{frame}

\begin{frame}{Preserving initial segment (cont.)}
  List the subsets of $Z\restriction s$ as $d_0,\ldots,d_n$. Iterate the
  relativized lemma across them to get decreasing subsets
  \[Z=X_0 \supseteq X_1 \supseteq X_2 \supseteq\ldots \supseteq X_n.\]
  At stage $i$, apply lemma with inputs $X_i$, and $\Phi_i$ defined by
  \[\Phi_i(Y) =\Phi(Y\cup d_i),\]

  giving $X_{i+1}\subseteq X_i$ whose subsets cannot compute (via
  $\Phi_i$) reals of $\mathcal{C}$, and where $\mathcal{C}$ contains no
  nonempty $\Sigma_1^{1,X_i}$-set.

  \vspace{1em}
  Choose $X=X_n\cup Z\restriction s$. Since $X=^*X_n$ and $\mathcal{C}$
  contains no $\Sigma_1^{1,X_n}$-set, $\mathcal{C}$ contains no
  $\Sigma_1^{1,X}$-set. Let $Y\subseteq X$ with $Y\restriction i=d_i$.
  Stage $i$ ensured that the subsets of $Y$ compute no real in
  $\mathcal{C}$ via $\Phi_i$. Since $Y\supset d_i$, the same is ensured via
  $\Phi$. $\blacksquare$
\end{frame}

\begin{frame}{Set whose subsets cannot compute reals in $\mathcal{C}$}
  \begin{main-thm*}[Monin, Patey, 2016]
    $\mathcal{C}\subseteq\omega^\omega$ compact, contains no
    $\Sigma_1^1$-subset ($\neq\emptyset$). Then there exists infinite set
    $X$ whose subsets cannot compute any real in $\mathcal{C}$.
  \end{main-thm*}

  \vspace{1em}
  \textbf{Pf:} The main lemma worked for fixed functional $\Phi$. Iterate
  the lemma across all functionals $\Phi_0,\Phi_1,\ldots$ to construct
  decreasing subsets of $X$'s
  \[\omega= X_0\supseteq X_1\supseteq\ldots,\]
  where at stage $s$, the lemma is applied with inputs $X_s$ and
  $\Phi_s$, and $X_{s+1}$ preserves the first $s$-elements of $X_s$. Take
  $X=\bigcap_{s\in\omega}X_s$; this is infinite from segment-preservation.
  $\blacksquare$
\end{frame}
