\begin{frame}{Ramsey's Theorem asserts Homogeneity}
  \begin{define*}[$c$-homogeneous]
    Given $k$-coloring $c:[\omega]^n\rightarrow k$, a subset
    $A\subseteq\omega$ is \textbf{$c$-homogeneous} if every $n$-tuple over
    $A$ is given the same color by $c$.
  \end{define*}

  \vspace{0.5em}
  \begin{thm*}[Ramsey's]
    Fix $n,k\leq1$. \textbf{$\text{RT}_k^n$} is the statement that every
    $k$-coloring $c:[\omega]^n\rightarrow k$ has an infinite
    $c$-homogeneous set.\\
    \vspace{0.5em}
    \textbf{RT} is the statement $(\forall n)(\forall k)\; \text{RT}_k^n$.
  \end{thm*}
  \textbf{Proof sketch:} $\text{RT}_2^1$ holds since in any 2-coloring of
  $\omega$, one of the colors gives an infinite set. Induct on $n$ to get
  $\text{RT}_2^n$. Then inductively group colors together to get
  $\text{RT}_k^n$ from $\text{RT}_2^n$. $\square$

  \vspace{0.5em}
  \textbf{RT asserts homogeneity.}
\end{frame}

\begin{frame}{Lebesgue measure}
  \begin{define*}
    $\mu(\llbracket\sigma\rrbracket) =2^{-|\sigma|}$. The \textbf{outer
    measure} of $\mathcal{A}\subseteq2^\omega$ is
    \[\mu^*(\mathcal{A}):= \inf\left\{\sum_{n\in\omega}
    \mu(\llbracket\sigma_n\rrbracket): \mathcal{A}\subseteq
    \bigcup_{n\in\omega} \llbracket\sigma_n\rrbracket\right\}.\]
    Its \textbf{inner measure} is $\mu_*(\mathcal{A}):=
    1-\mu^*(\bar{\mathcal{A}})$.
  \end{define*}

  \begin{define*}
    $\mathcal{A}\subseteq2^\omega$ is \textbf{measurable} iff
    $\mu^*(\mathcal{A}) =\mu_*(\mathcal{A})$.  Then we say its
    \textbf{measure} is $\mu(\mathcal{A}) =\mu^*(\mathcal{A})
    =\mu_*(\mathcal{A})$. $\mathcal{A}$ is \textbf{null} iff
    $\mu(\mathcal{A})=0$.
  \end{define*}
  \begin{fact*}[Measurable]
    $\mathcal{A}\subseteq2^\omega$ is measurable iff it is
    $G_\delta$ (countable intersection of open sets) modulo a null. E.g.
    open sets, closed sets are measurable.
  \end{fact*}
\end{frame}

\begin{frame}{1-random}
  \begin{notation*}
    The set of \textbf{reals} $A\in\omega^\omega$ contained in a
    \textbf{tree} $T\subseteq\omega^{<\omega}$ is denoted by $\llbracket
    T\rrbracket:= \{A\in\omega^\omega: (\exists n)\; [A\restriction n \in
    T]\}$.
  \end{notation*}

  \begin{define*}
    An effective enumeration of recursive trees $T_0,T_1,\ldots \subseteq
    2^{<\omega}$ is a \textbf{Martin-Lof test} of randomness if
    $\mu(\llbracket T_n\rrbracket) \leq 2^{-n}$ for all $n\in\omega$.\\
    \vspace{0.5em}
    A real $A\in2^\omega$ passes this randomness test iff $A\not\in
    \bigcup_{n\in\omega} \llbracket T_n\rrbracket$.
  \end{define*}

  \begin{define*}
    $A\in2^\omega$ is \textbf{1-random} iff it passes every Martin-Lof
    test.
  \end{define*}
\end{frame}

\begin{frame}{Set of non 1-randoms is null}
  \begin{fact*}
    The set of reals that is not 1-random is null.
  \end{fact*}

  \vspace{2em}
  \textbf{Proof sketch:} There are only countably many Turing machines, so
  there can only be countably many Martin-Lof tests. The set of reals that
  fail a test is null since the trees in a test descend quickly. The union
  of a countable set of null sets is still null, thus the set of reals that
  fail some Martin-Lof test is null. $\square$
\end{frame}

\begin{frame}{WWKL asserts Randomness}
  \begin{thm*}[Weak Weak Konig's Lemma]
    Every binary tree $T\subseteq 2^{<\omega}$ with positive measure i.e.
    \[\lim_n \frac{|\{\sigma\in T: |\sigma|=n\}|}{2^n} >0,\]
    contains a real.
  \end{thm*}
  \begin{fact}
    $T\subseteq2^{<\omega}$ is null if and only if the set of reals with
    infinitely many initial segments contained in $T$ is null.
  \end{fact}
  \begin{coro}
    If a binary tree has positive measure, it contains a 1-random.
  \end{coro}

  \textbf{WWKL asserts randomness.}
\end{frame}

\begin{frame}{WWKL, WKL, KL}
  \begin{thm*}[Weak Konig's Lemma]
    Every binary tree $T\subseteq2^{<\omega}$ with infinite nodes contains
    a real.
  \end{thm*}

  \begin{thm*}[Konig's Lemma]
    Every finitely-branching tree $T\subseteq\omega^{<\omega}$ with
    infinite nodes contains a real.
  \end{thm*}
  \textbf{Proof:} Construct real by iteratively choosing the branch with
  infinite nodes extending it. $\blacksquare$

  \vspace{2em}
  WKL follows from KL since binary trees are finitely branching.\\
  \vspace{1em}
  WWKL follows from WKL since trees with positive measure have infinite
  branches.
\end{frame}

\begin{frame}{Homogeneity versus Randomness}
  \newtheorem*{question*}{Question}
  \begin{question*}
    Which notion is stronger, homogeneity or randomness?
  \end{question*}

  \vspace{2em}
  Does knowing the homogeneity of a real tell us anything about its
  randomness, and vice versa?

  \vspace{2em}
  Since RT asserts homogeneity and WWKL asserts randomness, we are asking
  which of RT and WWKL is stronger.
\end{frame}

\begin{frame}{Problem, Instance, Solution}
  \begin{define*}
    A mathematical \textbf{problem} is a collection of \textbf{instances},
    with a collection of \textbf{solutions} for each instance.
  \end{define*}

  \vspace{2em}
  \textbf{E.g:} RT is a problem; its instances are the collections of
  colorings $c:[\omega]^n\rightarrow k$ for every $n,k\in\omega$, and the
  solutions of each $c$ are the class of $c$-homogeneous reals.

  \vspace{2em}
  \textbf{E.g:} WWKL is a problem; its instances are the collections of
  binary trees with positive measure, and the solutions of each such tree
  is the set of reals in the tree.
\end{frame}

\begin{frame}{Strongly Omnisciently Computably Reducible
($\leq_{\text{soc}}$)}
  \begin{define*}
    Problem \textbf{P} is \textbf{soc}-reducible to problem \textbf{Q}
    (written \textbf{P} $\leq_{\text{soc}}$ \textbf{Q}) if for every
    \textbf{P}-instance \textit{I}, there is a \textbf{Q}-instance
    \textit{J} such that every solution to \textit{J} computes a solution
    to \textit{I}.
  \end{define*}

  \vspace{2em}
  \textbf{E.g:} WWKL $\leq_{\text{soc}}$ WKL $\leq_{\text{soc}}$ KL follows
  from definition.

  \vspace{2em}
  \textbf{E.g:} KL, RT $\leq_{\text{soc}}$ WKL.
\end{frame}

\begin{frame}{Goal: Homogeneity ``independent from'' Randomness}
  Dependencies under $\leq_{\text{soc}}$

  \vspace{2em}
  \begin{center}
    \begin{tikzpicture}[node distance=4cm,auto,thick,>=latex']
      \node (KL) {KL$\leftrightarrow$WKL};
      \node (WWKL) [below right of=KL] {WWKL};
      \node (RT) [below left of=KL] {RT};
      \draw[->] (KL) -- (RT);
      \draw[->] (KL) -- (WWKL);
      %\draw [->,red] (RT) -- coordinate (m) (WWKL);
      %\draw[shift={(m)},red](-0.1,-0.1)--(0.1,+0.1);
    \end{tikzpicture}
  \end{center}
\end{frame}
