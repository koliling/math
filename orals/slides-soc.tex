\begin{frame}{Ramsey's Theorem asserts Homogeneity}
  \begin{notation*}
    Given $A\subseteq\omega$ and cardinal $\kappa$,
    $[A]^\kappa :=\{B\subseteq A:|B|=\kappa\}$.
  \end{notation*}
  \pause

  \begin{define*}[$c$-homogeneous]
    Given $k$-coloring $c:[\omega]^n\rightarrow k$, $A\subseteq\omega$ is
    \textbf{$c$-homogeneous} if every $n$-tuple over $A$ is given the same
    color by $c$.
  \end{define*}
  \pause

  \begin{thm*}[Ramsey's]
    \textbf{$\text{RT}_k^n$} states that every $k$-coloring
    $c:[\omega]^n\rightarrow k$ has an infinite $c$-homogeneous set.
    \textbf{RT} states $(\forall n)(\forall k)\; \text{RT}_k^n$.
  \end{thm*}
  \pause

  \vspace{0.5em}
  \textbf{RT asserts homogeneity.}
\end{frame}

\begin{frame}{Lebesgue measure}
  \begin{notation*}
    $\llbracket\sigma\rrbracket :=\{X\in2^\omega: \sigma\prec X\}$.
  \end{notation*}
  \pause

  \begin{define*}
    The \textbf{measure} of $\llbracket\sigma\rrbracket$ is
    $\mu(\llbracket\sigma\rrbracket) =2^{-|\sigma|}$.
  \end{define*}
  \pause

  \begin{define*}
    The \textbf{measure} of an open or closed
    $\mathcal{A}\subseteq2^\omega$ is
    \[\mu(\mathcal{A}):= \inf\left\{\sum_{n\in\omega}
    \mu(\llbracket\sigma_n\rrbracket): \mathcal{A}\subseteq
    \bigcup_{n\in\omega} \llbracket\sigma_n\rrbracket\right\}.\]
  \end{define*}
  \pause

  \begin{define*}
    $\mathcal{A}\subseteq2^\omega$ is \textbf{null} iff
    $\mu(\mathcal{A})=0$.
  \end{define*}
\end{frame}

\begin{frame}{1-random}
  \begin{define*}
    An effective enumeration of strings
    $A_0,A_1,\ldots \subseteq 2^{<\omega}$ is \textbf{effectively-null} (or a
    Martin-Lof test) if for all $n\in\omega$,
    \[\mu\left(\bigcup_{\sigma \in A_n} \llbracket\sigma\rrbracket \right)
    \leq 2^{-n}.\]

    \pause
    A \textbf{real} $X\in2^\omega$ avoids this effectively-null set iff
    $X\not\in \bigcup_{\sigma \in A_n} \llbracket\sigma\rrbracket$ for some
    $n\in\omega$.
  \end{define*}
  \pause

  \begin{define*}
    $X\in2^\omega$ is \textbf{1-random} iff it avoids every
    effectively-null set.
  \end{define*}
  \pause

  \begin{thm*}
    The class 1-randoms has measure 1.
  \end{thm*}
\end{frame}

\begin{frame}{Class of 1-randoms has measure 1}
  \textbf{Pf:} Enumerate all effectively-null sets
  $\mathcal{A}_0,\mathcal{A}_1,\ldots$.
  \pause
  \[\begin{array}{c|ccccc}
    \mu &1 &\frac{1}{2} &\frac{1}{4} &\frac{1}{8} &\ldots\\
    \hline
    \mathcal{A}_0 &A_{0,0} &\textcolor{blue}{A_{0,1}}
    &\textcolor{red}{A_{0,2}} &\textcolor{green}{A_{0,3}} &\ldots\\
    \mathcal{A}_1 &A_{1,0} &A_{1,1} &\textcolor{blue}{A_{1,2}}
    &\textcolor{red}{A_{1,3}} &\ldots\\
    \vdots &\ddots &\ddots &\ddots
    &\textcolor{blue}{\ddots} &\textcolor{red}{\ddots}
  \end{array}\]
  \pause

  Consider the effectively-null set $\mathcal{B}$ defined by the diagonals
  above:
  \begin{align*}
    B_0&= \textcolor{blue}{A_{0,1}} \cup \textcolor{blue}{A_{1,2}} \cup
    \textcolor{blue}{A_{2,3}} \cup \textcolor{blue}{\ldots}\\
    B_1&= \textcolor{red}{A_{0,2}} \cup \textcolor{red}{A_{1,3}} \cup
    \textcolor{red}{A_{2,4}} \cup \textcolor{red}{\ldots}\\
    B_2&= \textcolor{green}{A_{0,3}} \cup \textcolor{green}{A_{1,4}} \cup
    \textcolor{green}{A_{2,5}} \cup \textcolor{green}{\ldots}\\
    \vdots&
  \end{align*}
  \pause
  Now $X\in2^\omega$ is 1-random iff it avoids every $\mathcal{A}_i$ iff
  it avoids $\mathcal{B}$. Since $\mathcal{B}$ is effectively-null, the
  class of reals that avoid $\mathcal{B}$ has measure 1. $\square$
\end{frame}

\begin{frame}{WWKL asserts Randomness}
  \begin{notation*}
    $T\subseteq\omega^{<\omega}$ is a \textbf{tree} if it is closed under
    initial segments.\\
    Denote $[T]:= \{X\in2^\omega: (\forall n)\; [X\restriction n \in T]\}$.
  \end{notation*}
  \pause

  \begin{thm*}[Weak Weak Konig's Lemma]
    Every binary tree $T\subseteq 2^{<\omega}$ with positive measure i.e.
    \[\lim_n \frac{|\{\sigma\in T: |\sigma|=n\}|}{2^n}
    >0\;\;\;\;\;\;\;\;\;\;
    \text{contains a real.}\]
  \end{thm*}
  \pause

  \begin{fact}
    Tree $T\subseteq2^{<\omega}$ has positive measure iff
    $[T]\subseteq2^\omega$ is not null.
  \end{fact}
  \pause

  \begin{coro}
    If a binary tree has positive measure, it contains a 1-random.
  \end{coro}

  \pause
  \textbf{WWKL asserts randomness.}
\end{frame}

\begin{frame}{WKL, KL}
  \begin{thm*}[Weak Konig's Lemma]
    Every binary tree $T\subseteq2^{<\omega}$ with infinite nodes contains
    a real.
  \end{thm*}

  \pause
  \begin{thm*}[Konig's Lemma]
    Every finitely-branching tree $T\subseteq\omega^{<\omega}$ with
    infinite nodes contains a real.
  \end{thm*}

  \pause
  \begin{observe*}[KL $\Rightarrow$ WKL]
    Binary trees are finitely branching. $\blacksquare$
  \end{observe*}

  \pause
  \begin{observe*}[WKL $\Rightarrow$ WWKL]
    Trees with positive measure have infinite nodes. $\blacksquare$
  \end{observe*}
\end{frame}

\begin{frame}{Homogeneity versus Randomness}
  \begin{goal*}
    Which notion is stronger, homogeneity or randomness?
  \end{goal*}

  \pause
  \vspace{2em}
  Given a class of 1-randoms, is there a $c$-coloring where every
  $c$-homogeneous set computes a 1-random in the class?

  \pause
  \vspace{2em}
  Similarly, given a $c$-coloring, is there a class of 1-randoms where
  every element in the class computes a $c$-homogeneous set?

  \pause
  \vspace{2em}
  Since RT asserts homogeneity and WWKL asserts randomness, we are
  comparing the strengths of RT and WWKL.
\end{frame}

\begin{frame}{Problem, Instance, Solution}
  \begin{define*}
    A mathematical \textbf{problem} is a collection of \textbf{instances},
    with a collection of \textbf{solutions} for each instance.
  \end{define*}

  \pause
  \vspace{2em}
  \begin{example*}[RT is a problem]
    Instances are $k$-colorings $c:[\omega]^n\rightarrow k$ for every
    $n,k\in\omega$, and the solutions of each $c$ are the set of
    $c$-homogeneous reals.
  \end{example*}

  \pause
  \vspace{2em}
  \begin{example*}[WWKL is a problem]
    Instances are binary trees with positive measure, and the solutions of
    each such tree is the set of reals in the tree.
  \end{example*}
\end{frame}

\begin{frame}{Strongly Omnisciently Computably Reducible
($\leq_{\text{soc}}$)}
  \visible<1->{
  \begin{define*}
    Problem \textbf{P} is \textbf{soc}-reducible to problem \textbf{Q}
    (written \textbf{P} $\leq_{\text{soc}}$ \textbf{Q}) if for every
    \textbf{P}-instance $I_P$, there is a \textbf{Q}-instance
    $I_Q$ such that every solution $S_{Q}$ to $I_Q$ computes a
    solution $S_P$ to $I_P$.
  \end{define*}}

  \visible<2->{
  \begin{observe*}[$\leq_{\text{soc}}$ is reflexive and transitive]
    From definition. $\blacksquare$
  \end{observe*}}

  \visible<1->{
  \begin{center}
    \begin{tikzpicture}[node distance=3cm,auto,thick,>=latex']
      \node (I) {$\forall I_P$};
      %\pause
      \node (J) [right of=I] {$\exists I_Q$};
      \node (sol-J) [below of=J] {$\forall S_Q$};
      \node (sol-I) [below of=I] {$\exists S_P$};
      \draw[->] (I) -- (J);
      \draw[->] (J) -- (sol-J);
      \draw[->] (sol-J) -- node[above]{computes} (sol-I);
    \end{tikzpicture}
  \end{center}}
\end{frame}

\begin{frame}{Goal: Homogeneity ``independent from'' Randomness}
  \begin{observe*}[WWKL $\leq_{\text{soc}}$ WKL $\leq_{\text{soc}}$ KL]
    From definition. $\blacksquare$
  \end{observe*}

  \pause
  \begin{observe*}[KL, RT $\leq_{\text{soc}}$ WKL]
    Pick a solution of the KL/RT problem. Choose WKL instance with only
    one path, and that path encodes the chosen solution. $\blacksquare$
  \end{observe*}

  \pause
  \begin{center}
    \begin{tikzpicture}[node distance=4cm,auto,thick,>=latex']
      \node (KL) {KL$\leftrightarrow$WKL};
      \node (WWKL) [below right of=KL] {WWKL};
      \node (RT) [below left of=KL] {RT};
      \draw[->] (KL) -- (RT);
      \draw[->] (KL) -- (WWKL);
      %\draw [->,red] (RT) -- coordinate (m) (WWKL);
      %\draw[shift={(m)},red](-0.1,-0.1)--(0.1,+0.1);
    \end{tikzpicture}
  \end{center}
\end{frame}
