\begin{frame}{Ramsey's Theorem asserts Homogeneity}
  \begin{notation*}
    Given $A\subseteq\omega$ and cardinal $\kappa$,
    $[A]^\kappa :=\{B\subseteq A:|B|=\kappa\}$.
  \end{notation*}

  \begin{define*}[$c$-homogeneous]
    Given $k$-coloring $c:[\omega]^n\rightarrow k$, a subset
    $A\subseteq\omega$ is \textbf{$c$-homogeneous} if every $n$-tuple over
    $A$ is given the same color by $c$.
  \end{define*}

  \begin{thm*}[Ramsey's]
    \textbf{$\text{RT}_k^n$} states that every $k$-coloring
    $c:[\omega]^n\rightarrow k$ has an infinite $c$-homogeneous set.
    \textbf{RT} states $(\forall n)(\forall k)\; \text{RT}_k^n$.
  \end{thm*}

  \vspace{0.5em}
  \textbf{RT asserts homogeneity.}
\end{frame}

\begin{frame}{Lebesgue measure}
  \begin{define*}
    $\mu(\llbracket\sigma\rrbracket) =2^{-|\sigma|}$. The \textbf{outer
    measure} of $\mathcal{A}\subseteq2^\omega$ is
    \[\mu^*(\mathcal{A}):= \inf\left\{\sum_{n\in\omega}
    \mu(\llbracket\sigma_n\rrbracket): \mathcal{A}\subseteq
    \bigcup_{n\in\omega} \llbracket\sigma_n\rrbracket\right\}.\]
    Its \textbf{inner measure} is $\mu_*(\mathcal{A}):=
    1-\mu^*(\bar{\mathcal{A}})$.
  \end{define*}

  \begin{define*}
    $\mathcal{A}\subseteq2^\omega$ is \textbf{measurable} iff
    $\mu^*(\mathcal{A}) =\mu_*(\mathcal{A})$.  Then we say its
    \textbf{measure} is $\mu(\mathcal{A}) =\mu^*(\mathcal{A})
    =\mu_*(\mathcal{A})$. $\mathcal{A}$ is \textbf{null} iff
    $\mu(\mathcal{A})=0$.
  \end{define*}
  \begin{fact*}[Measurable]
    $\mathcal{A}\subseteq2^\omega$ is measurable iff it is
    $G_\delta$ (countable intersection of open sets) modulo a null. E.g.
    open sets, closed sets are measurable.
  \end{fact*}
\end{frame}

\begin{frame}{1-random}
  \begin{notation*}
    Given a set of initial segments $A\subseteq\omega^{<\omega}$, denote
    $\llbracket A\rrbracket:= \{X\in2^\omega: (\exists n)\;
    [X\restriction n \in A]\}$.\\
    \vspace{0.5em}
    $T\subseteq\omega^{<\omega}$ is a \textbf{tree} if it is closed under
    initial segments. Denote $[T]:= \{X\in2^\omega: (\forall n)\;
    [X\restriction n \in T]\}$.
  \end{notation*}

  \begin{define*}
    An effective enumeration of recursive sets
    $A_0,A_1,\ldots \subseteq 2^{<\omega}$ is a \textbf{Martin-Lof test} of
    randomness if $\mu(\llbracket A_n\rrbracket) \leq 2^{-n}$ for all
    $n\in\omega$.\\
    \vspace{0.5em}
    A \textbf{real} $X\in2^\omega$ passes this test iff $X\not\in
    \bigcap_{n\in\omega} \llbracket A_n\rrbracket$.
  \end{define*}

  \begin{define*}
    $X\in2^\omega$ is \textbf{1-random} iff it passes every Martin-Lof
    test.
  \end{define*}
\end{frame}

\begin{frame}{Set of non 1-randoms is null}
  \begin{fact*}
    The set of reals that is not 1-random is null.
  \end{fact*}

  \vspace{2em}
  \textbf{Proof sketch:} There are only countably many Turing machines, so
  there can only be countably many Martin-Lof tests. The set of reals that
  fail a test is null since the trees in a test get sparse quickly. The
  union of a countable set of null sets is still null, thus the set of
  reals that fail some Martin-Lof test is null. $\square$
\end{frame}

\begin{frame}{WWKL asserts Randomness}
  \begin{thm*}[Weak Weak Konig's Lemma]
    Every binary tree $T\subseteq 2^{<\omega}$ with positive measure i.e.
    \[\lim_n \frac{|\{\sigma\in T: |\sigma|=n\}|}{2^n} >0,\]
    contains a real.
  \end{thm*}
  \begin{fact}
    A binary tree $T\subseteq2^{<\omega}$ has positive measure iff
    $[T]\subseteq2^\omega$ is not null.
  \end{fact}
  \begin{coro}
    If a binary tree has positive measure, it contains a 1-random.
  \end{coro}

  \textbf{WWKL asserts randomness.}
\end{frame}

\begin{frame}{WWKL, WKL, KL}
  \begin{thm*}[Weak Konig's Lemma]
    Every binary tree $T\subseteq2^{<\omega}$ with infinite nodes contains
    a real.
  \end{thm*}

  \begin{thm*}[Konig's Lemma]
    Every finitely-branching tree $T\subseteq\omega^{<\omega}$ with
    infinite nodes contains a real.
  \end{thm*}

  \begin{observe*}[KL $\Rightarrow$ WKL]
    Binary trees are finitely branching. $\blacksquare$
  \end{observe*}

  \begin{observe*}[WKL $\Rightarrow$ WWKL]
    Trees with positive measure have infinite branches. $\blacksquare$
  \end{observe*}
\end{frame}

\begin{frame}{Homogeneity versus Randomness}
  \newtheorem*{question*}{Question}
  \begin{question*}
    Which notion is stronger, homogeneity or randomness?
  \end{question*}

  \vspace{2em}
  Given a class of 1-randoms, is there a $c$-coloring where every
  $c$-homogeneous set computes some 1-randoms in the class?

  \vspace{2em}
  Similarly, given a $c$-coloring, is there a class of 1-randoms where
  every element computes some $c$-homogeneous set?

  \vspace{2em}
  Since RT asserts homogeneity and WWKL asserts randomness, we are asking
  which of RT and WWKL is stronger.
\end{frame}

\begin{frame}{Problem, Instance, Solution}
  \begin{define*}
    A mathematical \textbf{problem} is a collection of \textbf{instances},
    with a collection of \textbf{solutions} for each instance.
  \end{define*}

  \begin{example*}[RT is a problem]
    Instances are collections of colorings $c:[\omega]^n\rightarrow k$ for
    every $n,k\in\omega$, and the solutions of each $c$ are the set of
    $c$-homogeneous reals.
  \end{example*}

  \begin{example*}[WWKL is a problem]
    Instances are collections of binary trees with positive measure,
    and the solutions of each such tree is the set of reals in the tree.
  \end{example*}
\end{frame}

\begin{frame}{Strongly Omnisciently Computably Reducible
($\leq_{\text{soc}}$)}
  \begin{define*}
    Problem \textbf{P} is \textbf{soc}-reducible to problem \textbf{Q}
    (written \textbf{P} $\leq_{\text{soc}}$ \textbf{Q}) if for every
    \textbf{P}-instance $I_P$, there is a \textbf{Q}-instance
    $I_Q$ such that every solution $S_{Q}$ to $I_Q$ computes a
    solution $S_P$ to $I_P$.
  \end{define*}

  \begin{observe*}[$\leq_{\text{soc}}$ is reflexive and transitive]
    From definition. $\blacksquare$
  \end{observe*}

  \begin{center}
    \begin{tikzpicture}[node distance=3cm,auto,thick,>=latex']
      \node (I) {$\forall I_P$};
      %\pause
      \node (J) [right of=I] {$\exists I_Q$};
      \node (sol-J) [below of=J] {$\forall S_Q$};
      \node (sol-I) [below of=I] {$\exists S_P$};
      \draw[->] (I) -- (J);
      \draw[->] (J) -- (sol-J);
      \draw[->] (sol-J) -- node[above]{computes} (sol-I);
    \end{tikzpicture}
  \end{center}
\end{frame}

\begin{frame}{Goal: Homogeneity ``independent from'' Randomness}
  \begin{observe*}[WWKL $\leq_{\text{soc}}$ WKL $\leq_{\text{soc}}$ KL]
    From definition. $\blacksquare$
  \end{observe*}

  \begin{observe*}[KL, RT $\leq_{\text{soc}}$ WKL]
    Pick a solution of the KL/RT problem. Choose WKL instance with only
    one path, which is an encoding of that solution. $\blacksquare$
  \end{observe*}

  \begin{center}
    \begin{tikzpicture}[node distance=4cm,auto,thick,>=latex']
      \node (KL) {KL$\leftrightarrow$WKL};
      \node (WWKL) [below right of=KL] {WWKL};
      \node (RT) [below left of=KL] {RT};
      \draw[->] (KL) -- (RT);
      \draw[->] (KL) -- (WWKL);
      %\draw [->,red] (RT) -- coordinate (m) (WWKL);
      %\draw[shift={(m)},red](-0.1,-0.1)--(0.1,+0.1);
    \end{tikzpicture}
  \end{center}
\end{frame}
