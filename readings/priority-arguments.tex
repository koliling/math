\subsection{$\Pi_2$-constructions}

\begin{goal}
  Construct c.e. $A$ satisfying $\Pi_2$ requirements $R_0,R_1,\ldots$.
\end{goal}

For example, construct c.e. high $H\not\equiv_T\emptyset'$.
\begin{lemma}[Thickness]
  There is a c.e. set $W$ such that any thick subset $H$ of $W$ is high.
\end{lemma}
\begin{proof}
  Define $W$ by
  \[W^{[e]} :=\{n: |W_e|>n\}.\]

  Observe that
  \begin{align*}
    W^{[e]}&=
    \begin{cases}
      \{0,\ldots,|W_e|-1\} &\text{if } e\in\text{Fin},\\
      \omega &\text{otherwise}.
    \end{cases}
  \end{align*}

  Let $H$ be a thick subset of $W$. Define the $H$-recursive function
  \begin{align*}
    \Gamma^H(e,n):=
    \begin{cases}
      0 &\text{if } n\not\in H^{[e]},\\
      1 &\text{otherwise}.
    \end{cases}
  \end{align*}

  Then
  \begin{align*}
    \lim_n \Gamma^H(e,n)=
    \begin{cases}
      0 &\text{if } |H^{[e]}| \text{ is finite},\\
      1 &\text{otherwise}.
    \end{cases}
  \end{align*}

  Thus
  \begin{align*}
    H' &\geq_T \lim_n \Gamma^H(e,n) &(\text{by limit lemma})\\
    &\geq_T \text{Fin} &(\text{since } e\in\text{Fin} \Leftrightarrow
      |H^{[e]}|<\omega)\\
    &\geq_T \emptyset''. &(\text{since Fin is } \Sigma_2\text{-complete})
  \end{align*}
\end{proof}

\begin{theorem}[High incomplete set]
  There exists an incomplete c.e. set $H$ that is high.
\end{theorem}

To make $H$ high, suffices for $H$ to be a thick subset of $W$ from the
thickness lemma. Thus have positive requirements that tries to enumerate
elements into $H$:
\[P_e: H^{[e]} =^* W^{[e]}.\]

For incompleteness, fix complete c.e. set $C\equiv_T \emptyset'$ that $H$
is not able to compute. Thus $H$ and $C$ satisfy negative requirements
that try to stop elements from entering $H$:
\[N_e: \Phi_e^H \neq C.\]

Observe that this is a $\Pi_2$-construction because the $P_e$'s are
$\Pi_2$-requirements, in the sense that the outcomes are either $\Pi_2$
or $\Sigma_2$:
\begin{align*}
  \Pi_2\text{-outcome } &|W^{[e]}|=\omega: &(\forall n')(\exists
    n>n')(\exists s)\; [n'\in W^{[e]}_s],\\
  \Sigma_2\text{-outcome } &|W^{[e]}|<\omega: &(\exists n')(\forall
    n>n')(\forall s)\; [n'\not\in W^{[e]}_s].
\end{align*}

Similarly, the $N_e$'s are $\Pi_2$-requirements, in the sense that the
outcomes are either $\Pi_2$ or $\Sigma_2$:
\begin{align*}
  \Pi_2\text{-outcome } &\Phi_e^H(x)\uparrow: &(\forall s')(\exists
    s>s')\; [\Phi^{H_s}_{e,s}(x) \neq \Phi^{H_{s'}}_{e,s'}(x)],\\
  \Sigma_2\text{-outcome } &\Phi_e^H(x)\downarrow \neq C(x): &(\exists
    s')(\forall s>s')\; [\Phi^{H_s}_{e,s}(x) \downarrow=C(x)].
\end{align*}

Effectively prioritize requirements on tree, say
$P_0,N_0,P_1,N_1,\ldots$. A node for $P_e$ has possible outcomes
\[\{\infty, \ldots, 2, 1, 0\},\]
where $\infty$ means $P_e$ has $\Pi_2$-outcome $|W^{[e]}|=\omega$, and
where outcome $n<\omega$ means $P_e$ has $\Sigma_2$-outcome
$|W^{[e]}|=n$.\\

Likewise, a node for $N_e$ has possible outcomes
\[\{\uparrow, \downarrow\},\]
where $\uparrow$ refers to the $\Pi_2$-outcome where $\Phi_e^H(x)$
diverges, while $\downarrow$ refers to the $\Sigma_2$-outcome where
$\Phi_e^H(x)$ converges.\\

Together, the outcomes are
\[\Lambda :=\{\infty <\ldots <2<1<0 <\uparrow <\downarrow\},\]
ordered such that outcomes on the left (the smaller outcomes) have higher
priority than those on the right. The construction will be performed on
the tree
\[T =\Lambda^{<\omega},\]
and each node $\alpha\in T$ has a strategy $\sigma_\alpha$. At stage $s$,
let $\alpha\in \Lambda^s$ be our guess of the outcomes of the first
$s$ requirements, and play strategy $\sigma_\alpha$. When playing
$\sigma_\alpha$, we usually respect higher priority requirements and
higher priority outcomes by respecting the restraints of all nodes
$\beta$ shorter and to the `left' of $\alpha$.\\

The outcomes in $\Lambda$ are prioritized and the strategies
$\sigma_\alpha$ are designed such that the ``true'' outcome
$f\in\Lambda^\omega$ of the requirements will be exactly the
``left-most'' path visited infinitely often, i.e.
\[f =\liminf_s \alpha_s.\]

\newpage
\underline{\textbf{Strategy for node $\alpha$ of $P_e$}}
\begin{enumerate}
  \item Have $x\in W^{[e]}$ not yet in $H^{[e]}$.
  \item Put $x$ into $H^{[e]}$ if the restraints for nodes shorter than and
    to the left of $\alpha$ are respected.
\end{enumerate}

\underline{\textbf{Outcome for $\alpha$ of $P_e$}}
\begin{enumerate}
  \item If the strategy enumerated new elements into $H^{[e]}$, outcome is
    $\infty$.
  \item Otherwise outcome stays at $n$, the number of elements in $H^{[e]}$
    so far.
\end{enumerate}

\underline{\textbf{Strategy for node $\alpha$ of $N_e$}}
\begin{enumerate}
  \item Let $x$ be the smallest witness at this stage for
    $\Phi_e^H \neq C$.
  \item Set restraint to protect the computation up to and including
    that for $x$.
\end{enumerate}

\underline{\textbf{Outcome for $\alpha$ of $N_e$}}
\begin{enumerate}
  \item If $x$ is a divergent witness, set outcome to $\uparrow$.
  \item If the use for $\Phi_e^H(x)$ was injured at this stage, set outcome
    to $\uparrow$.
  \item Otherwise $\Phi_e^H(x)$ is a convergent witness, so set outcome to
    $\downarrow$.
\end{enumerate}

\begin{lemma}
  The true outcome $f\in\Lambda^\omega$ is $\liminf_s \alpha_s$.
\end{lemma}

\begin{proof}
  Induct on initial segment of $f$.
\end{proof}

\begin{lemma}
  $P_e$ is satisfied.
\end{lemma}

\begin{proof}
  Obviously.
\end{proof}

\begin{lemma}
  $N_e$ is satisfied.
\end{lemma}

\begin{proof}
  Obviously.
\end{proof}
