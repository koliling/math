\documentclass{article}
\usepackage[left=3cm,right=3cm,top=3cm,bottom=3cm]{geometry}
\usepackage{amsmath,amssymb,amsthm,tikz,mathtools}
\usepackage{stmaryrd} % For double square bracket [[]]
\usepackage{bm} % For bold vectors
\usepackage{color}
\usepackage{tcolorbox}
\usepackage{cancel} % To use the cancel function
\usepackage[inline]{enumitem}
\usetikzlibrary{patterns}
\setlength{\parindent}{0mm}
\newcommand{\COMMENT}[1]{\textcolor{red}{#1}}

\newtheorem{notation}{Notation}[subsection]
\newtheorem{definition}{Definition}[subsection]
\newtheorem{lemma}{Lemma}[subsection]
\newtheorem{theorem}{Theorem}[subsection]
\newtheorem{recall}{Recall}[subsection]
\newtheorem{question}{Question}[subsection]
\newtheorem{goal}{Goal}[subsection]
\newtheorem{remark}{Remark}[subsection]
\newtheorem{corollary}{Corollary}[subsection]

\begin{document}
\title{Readings}
\author{Li Ling Ko\\ lko@nd.edu}
\date{\today}
\maketitle

\section{Maximal Sets}
  \begin{notation}
    Given $A,B\subseteq\omega$, write
    \begin{itemize}
      \item $A=^*B$ iff $|A\Delta B|<\omega$, i.e. $A$ and $B$ are equal
        except for a finite number of elements.
      \item $A\subseteq^*B$ iff $|A-B|<\omega$, i.e., $B$ contains almost
        every element of $A$.
    \end{itemize}
  \end{notation}

  \begin{definition}
    A r.e. set $M\subset\omega$ is \textit{maximal} iff
    $|M|=|\overline{M}|=\omega$, and given any r.e. set $N$, if
    $M\subseteq^* N$, then either $N=^*M$ or $N=^*\omega$; i.e. anything
    that lies between $M$ and $\omega$ must be almost equal to one of them.
  \end{definition}

  \begin{definition}
    A set $C$ is \textit{cohesive} if there is no r.e. set $W$ such that
    both $W\cap C$ and $\overline{W}\cap C$ are infinite.
  \end{definition}

  \begin{lemma}
    A r.e. set is maximal if and only if its complement is cohesive.
    \label{lemma:maximal-cohesive}
  \end{lemma}
  \begin{proof}
    Chase definitions. If a r.e. set $M$ does not have a cohesive
    complement, let $W$ be the witness of the non-cohesiveness of
    $\overline{M}$. Then $M\cup W$ will witness the non-maximality of $M$.
    Conversely, if $M$ is not maximal with witness $W$, then $W$ will also
    witness the non-cohesiveness of $\overline{M}$.
  \end{proof}

\subsection{Maximal Sets}
  \begin{theorem}
    There exists a maximal set.
  \end{theorem}

  From Lemma~\ref{lemma:maximal-cohesive}, it suffices to construct a r.e.
  set $A$ that satisfies requirements
  \begin{equation}
    |\overline{A}|=\omega,
    \label{eqn:R}
  \end{equation}
  and
  \begin{equation}
    P_e: \overline{A} \subseteq^*\overline{W}_e\;\; \text{or}\;\;
    \overline{A}\subseteq^* W_e.
    \label{eqn:P_e}
  \end{equation}

  At stage $s$, write $\overline{A}_s =\{a_0^s<a_1^s<a_2^s\ldots\}$. Notice
  that $a_i^0\leq a_i^1\leq a_i^2\leq\ldots$ while we enumerate elements
  from $\overline{A}_s$ into $A_s$ at stage $s+1$. Hence to ensure that
  $|\overline{A}|=\omega$, it suffices to show that $a_n:=\lim_s a_n^s$
  exists for each $n\in\omega$. Assuming all $a_n$ exist, we can write
  $\overline{A}=\{a_0<a_1<\ldots\}$.

  \begin{notation}
    Given $x\in\omega$, we say the $e$-state of $x$ at stage $s$ is
    \[\sigma(e,x,s) =\{i\leq e: x\in W_{i,s}\} \in 2^{e+1}.\]
    For example, $e$-state 100 means $x\in W_{0,s}$ but $x\not\in
    W_{1,s}$ and $x\not\in W_{2,s}$. Notice that there are only $2^{e+1}$
    possible $e$-states.
  \end{notation}

  \begin{proof}
    At stage 0, let $A_0=\emptyset$, so that $\overline{A}_0=\omega$. At
    stage $s+1$, we consider which of $a_0^s<a_1^s<\ldots<a_s^s$ should be
    enumerated into $A_{s+1}$. Our strategy is to maximize the $e$-state
    corresponding to the highest priority requirement. Specifically, given
    $e\leq s$, see if there is some $e<i\leq s$ such that the $e$-state of
    $a_i^s$ is greater than the $e$-state of $a_e^s$. Let $e\leq s$ be the
    smallest requirement where such $i$ exists. Enumerate
    $a_e^s,\ldots,a_{i-1}^s$ into $A_s$.
  \end{proof}

  \begin{lemma}
    $a_i:=\lim_s a_i^s$ exists.
  \end{lemma}
  \begin{proof}
    Notice that $a_0^s$ increases at stage $s$ if and only if $0$-state of
    $a_0^s$ increases in the next stage. But there are only $2^{0+1}=2$
    possible $0$-states. So $a_0^s$ can only increase at most once. By
    strong induction, after $a_0,\ldots,a_{e-1}$ have stabilized, $a_e$
    can only increase at most $2^{e+1}$ times before stabilizing too.
  \end{proof}

  \begin{lemma}
    $P_e$ is satisfied.
  \end{lemma}
  \begin{proof}
    Assume $P_0$ is not satisfied. Then there must be some
    $a_i\in\overline{A}$ such that $a_i\in W_0$ but $a_{i+1}\not\in W_0$.
    Wait till $a_0,\ldots,a_{i+1}$ have stabilized. Then
    the $i$-state of $a_i$ will be smaller than the $i$-state of $a_{i+1}$
    since $a_i\in W_0$ but $a_{i+1}\not\in W_0$. By our strategy, we will
    enumerate $a_i$ into $A$, a contradiction. \\

    Assume $P_0,\ldots,P_e$ are satisfied, but not $P_{e+1}$. Let
    $a_n\in\overline{A}$ be the smallest element such that for all
    $a_m>a_n$ and all $i\leq e$, the $i$-state of $a_m$ equals the
    $i$-state of $a_n$. Wait till all $a_0,\ldots,a_n$ have stabilized.
    Choose $a_p<a_q$ bigger than $a_n$ such that $a_p\in W_{e+1}$ but
    $a_q\not\in W_{e+1}$. But then the $(e+1)$-state of $a_p$ will be
    smaller than that of $a_q$, and we will enumerate $a_p$ into $A$, a
    contradiction.
  \end{proof}

\subsection{Degrees of Maximal Sets}
  \begin{goal}
    $\bm{d}$ is high $\Leftrightarrow$ $\bm{d}$ contains maximal set
  \end{goal}

  \begin{remark}
    The easy direction is to show that maximal sets have high degree. 
  \end{remark}

  \begin{lemma}
    If $M$ is a maximal set then $p_{\overline{M}}$ is dominant.
    \label{lemma:high-maximal}
  \end{lemma}
  \begin{proof}
    Assume recursive function $p_{\overline{M}}$ is dominated by recursive
    function $f$. We can assume $f$ is strictly increasing and majorizes
    $p_{\overline{M}}$. Observe that
    $p_{\overline{M}}(n)\in[n,f(n)-1]$. In particular, $[0,f(0)-1]$
    contains an element of $\overline{M}$. Then $[f(0),f(f(0))-1]$ contains
    an element of $\overline{M}$, and $[f(f(0)),f(f(f(0)))-1]$ contains an
    element of $\overline{M}$, and so on. But these are disjoint intervals,
    so by picking every other interval, we can choose enumerate an infinite
    number of elements of $\overline{M}$, leaving an infinite number of
    elements of $\overline{M}$. \\

    Assume that $f$ does not dominate $p_{\overline{M}}$, but instead
    crosses over it infinitely many times. Then there are infinite disjoint
    intervals $(f(x),f(x+1)]$ that contain two or more elements of
    $\overline{M}$. So by enumerating $M$ and searching for the first
    element in each interval that is not in $M_s$, we can extend $M$ by
    infinite elements, leaving infinite elements unenumerated.
  \end{proof}

  \begin{theorem}
    \textit{(High Domination Theorem)} A set $A$ satisfies
    $\emptyset''\leq_T A'$ if and only if it computes a dominant function.
    \label{thm:high-domination}
  \end{theorem}
  \begin{proof}
    Recall that $\text{Tot}\equiv_T \emptyset''$, so by the limit lemma
    relativized to $A$, we have $\emptyset''\leq_T A'$ iff there is a
    $A$-recursive binary function $g(e,s)$ such that $\lim_s g(e,s)
    =\chi_{\text{Tot}}(e)$. \\

    Assume $\emptyset''\leq_T A'$. We use $g$ to construct a dominant
    function $f$. Given $e$ and stage $s$, let $t_s(e)$ be the first stage
    $t>s$ such that either $g(e,t)=0$ or $\varphi_{e,t}\restriction
    t\downarrow$. In other words, $t_s(e)$ be the first stage greater than
    $s$ where either $\varphi_e$ is shown to be not total from $g$, or
    shown to be total up to $t$. Then setting $f(s)=\max\{t_s(e):e\leq s\}$
    works. \\

    For the converse, let $f(s)$ be a dominant function. Construct $g(e,s)$
    as
    \begin{align*}
      g(e,s) :=
      \begin{cases}
        1 &\text{if}\; \varphi_{e,f(s)}\restriction s\downarrow,\\
        0 &\text{otherwise}.
      \end{cases}
    \end{align*}
  \end{proof}

  \begin{corollary}
    Every maximal set has a high degree.
  \end{corollary}
  \begin{proof}
    Follows directly from Theorem~\ref{thm:high-domination} and
    Lemma~\ref{lemma:high-maximal}.
  \end{proof}

  \begin{remark}
    Coding -- Given $B$, construct r.e. $A\; \prescript{}{T}{\geq}\; B$.
  \end{remark}
  Encode $B$ into $A$ in a recoverable way. E.g. if $B=\{b_0,b_1,\ldots\}$,
  then at stage $s$ enumerate $a_{b_s}^s$ into $A$. Given $A$, to decide if
  $b\in B$, simulate to stage $s$ where the first $b+1$ elements of $A$
  have stabilized; then $b\in B \Leftrightarrow a_{b}^s\in A_s$.

  \begin{remark}
    Permitting -- Given r.e. $B \prescript{}{T}{\geq}\; \emptyset'$,
    construct r.e. $A\leq_T B$.
  \end{remark}
  $x$ enters $A$ only if $B$ permits it. At stage $s$, to satisfy
  requirement $R$, want to enumerate $n$ into $A$. $B$ permits $n$ to
  enter only if the first $n$ elements of $B$ have not stabilized, i.e.
  when we find a stage $t\geq s$ such that $B_{t+1}\restriction n \neq
  B_t\restriction n$. \\

  Then given $B$, to decide if $n\in A$, wait for stage $s$ where the first
  $n$ elements of $B$ have stabilized; then $n\in A \Leftrightarrow n\in
  A_s$. \\

  We show that as long as requirement $R$ has infinitely many potential
  witnesses $n$, eventually one of them will be permitted to enter $A$.
  Assume otherwise. Then we will be able to compute $B$, as follows. To
  decide if $n\in B$, simulate till stage $s$ where $R$ offers potential
  witness $m>n$. Then since $m$ is never permitted to enter $A$,
  $B\restriction m=B_s\restriction m$. Thus $n\in B \Leftrightarrow n\in
  B_s$.

  \begin{theorem}
    Every high r.e. degree contains an r.e. set $D$ whose computation
    function $c_{\overline{D}}$ is dominant, where $\overline{D}_s
    =\{d_0^s<d_1^s<\ldots\}$, and $c_{\overline{D}}(x) =(\mu
    s)[d_x^s=d_x]$.
    \label{thm:high-dominant}
  \end{theorem}
  \begin{proof}
    By High Domination Theorem~\ref{thm:high-domination}, if
    $\text{deg}(A)\in\bm{H}_1$, then $A$ computes a dominant function
    $f=\Phi^A$. Let $c_f$ be modulus of $A$:
    \[c_f(x) =(\mu t) (\forall s\geq t)\; [\Phi_s^{A_s}(x)
    =\Phi_t^{A_t}(x)].\]

    Then $c_f\leq_T A$ by Modulus Lemma. Also, $c_f$ is dominant, because
    if recursive $g\prescript{*}{}{>}\; c_f$, then recursive
    $\Phi_{g(x)}^{A_{g(x)}}+1 \prescript{*}{}{>}\; f$. \\

    Construct $\{D_s\}_{s\in\omega}$, with $D_0=\emptyset$. At stage $s+1$,
    \\

    \textit{Step 1.} (Permitting) If $\Phi_{s+1}^{A_{s+1}}(x)
    \neq\Phi_{s}^{A_{s}}(x)$, enumerate $d_x^s$ into $D$. \\

    \textit{Step 2.} (Coding) Enumerate $d_{a_s}^s$ into $D$. \\

    \underline{Verification.} From permitting and coding, we have
    $A\equiv_T D$. From step 1, $c_f(x) \leq c_{\overline{D}}(x)$. In
    particular, $c_{\overline{D}}$ is dominant.
  \end{proof}

  \begin{theorem}
    Every high degree $\bm{d}$ contains a maximal set $A$.
  \end{theorem}
  \begin{proof}
    Satisfy requirements
    \[R_e: \overline{A} \subseteq^* \overline{W}_e\; \text{or}\;
    \overline{A} \subseteq^* W_e.\]

    Let $D\in\bm{d}$ be r.e. as in Theorem~\ref{thm:high-dominant}. Let
    $D_0=\emptyset$. At stage $s$, if $a_i^s<a_j^s$ and $\sigma(i,a_i^s,s)
    <\sigma(i,a_j^s,s)$, we say $a_i^s$ is attracted to $a_j^s$ for the
    sake of requirement $R_i$. At stage $s+1$: \\

    \textit{Step 1.} Choose least $i\leq s$ such that
    $a_i^s$ is attracted to some larger $a_j^s$ for $R_i$, and
    \[(\exists y\leq a_i^s)\; [d_y^{s+1} \neq d_y^s].\]
    In the first condition, we maximise the $i$-state of the element of
    highest priority to satisfy $R_i$, and in the second condition, $D$
    permits elements of lower $i$-state to be enumerated into $A$, to
    satisfy $A\leq_T D$. Choose the minimum $j>i$ such that $a_j^s$ gives
    the highest $i$-state. Enumerate
    \[a_i^s, a_{i+1}^s, \ldots, a_{j-1}^s, a_{j+1}^s, \ldots, a_s^s\]
    into $A$. If there is no such $a_j^s$ do nothing. \\

    \textit{Step 2.} (Code $D$ into $A$) Let
    \[z =(\mu y) [d_y^{s+1} \neq d_y^s].\]
    If some $x\leq a_{z+1}^s$ was enumerated into $A$ in Step 1 do nothing.
    Otherwise enumerate whichever of $a_z^s$ and $a_{z+1}^s$ has the lower
    $z$-state into $A$. By retaining the element of higher $z$-state, we
    ensure that if some $x<a_z^s$ was attracted to one of $a_z^s$ or
    $a_{z+1}^s$, then $x$ remains attracted to the element that remains in
    $\overline{A}$ after step 2.
  \end{proof}

  \begin{lemma}
    $a_i :=\lim_s a_i^s$ exists.
  \end{lemma}
  \begin{proof}
    From finiteness of the number of $i$-states.
  \end{proof}

  \begin{lemma}
    $A\equiv_T D$.
  \end{lemma}
  \begin{proof}
    Step 2 ensures $D\leq_T A$, because if $a_{x+1}^s=a_{x+1}$, then
    $d_x^s=d_x$. Second, $A\leq_T D$ from permitting.
  \end{proof}

  \begin{lemma}
    $R_e$ is satisfied, and $\forall^{\text{a.e.}} i,j\;
    [\sigma(e,a_i)=\sigma(e,a_j)]$.
  \end{lemma}
  \begin{proof}
    Assume lemma true for $i<e$ but false for $e$. We define a recursive
    $h(x)$ not dominated by $c_{\overline{D}}(x)$. Choose $n\in\omega$ and
    stage $s_0\in\omega$ large enough such that
    \begin{itemize}
      \item $\forall i\geq n\; [\sigma(e-1,a_i) =\sigma(e-1,a_n)]$
      \item $a_0,\ldots,a_n$ have stabilized by stage $s_0$
    \end{itemize}

    Then $\forall i\geq n$, $\sigma(e,a_m) =\sigma(e-1,a_n)^\frown0
    =\sigma_0$ or $\sigma(e-1,a_n)^\frown1 =\sigma_1$. Then since
    $|\overline{A}\cap W_e|=|\overline{A}\cap\overline{W}_e| =\omega$,
    there are infinite $a_i$ with $e$-state $\sigma_0$ and infinite $a_i$
    with $e$-state $\sigma_1$. \\

    Thus each $a_i\not\in W_e$ will be attracted to some $x>a_i$ during
    construction; the attractor may be some $a_j>a_i$ with $e$-state
    $\sigma_1$, or some $x$ that is eventually enumerated into $A$. \\

    So given arbitrary $x>a_n$, after stage $s_0$, one of the following
    cases must happen:
    \begin{enumerate}
      \item $x$ is enumerated into $A$
      \item $e$-state of $x$ is $\sigma_1$
      \item $e$-state of $x$ is $\sigma_0$ and $x$ is attracted to some
        larger element $y$ with $e$-state $\sigma_1$
    \end{enumerate}

    Define $h(x)$ to be the first stage where one of the above happens.
    Since $|\overline{A}\cap\overline{W}_e|=\omega$, there are infinite
    elements $a_i\in\overline{A}$ that belong to case 3 above. Assume that
    once such $a_i$ is attracted to a larger element, it will forever be
    attracted to a larger element, even if the larger element is enumerated
    into $A$. \\

    Now $D$ permits $a_i$ to be enumerated into $A$ if $a_0,\ldots,a_i$
    stabilizes before $d_0,\ldots,d_{a_i}$ does. So since these $a_i$'s
    were never permitted to enter $A$, and each $a_i$ is always attracted
    to something larger, it must be that $d_0,\ldots,d_{a_i}$
    stabilized before stage $h(a_i)$. Yet $c_{\overline{D}}$ dominates
    $h(x)$, so for large enough $a_i$, $d_0,\ldots,d_{a_i}$ should
    stabilize after $a_0,\ldots,a_i$, a contradiction. \\

    Thus it suffices to prove that if $a_i\in\overline{A}$ falls under case
    3 above for the first time, it will forever be attracted to a larger
    element, even if the larger element is enumerated into $A$. There are
    two possible cases. In the first case, the larger element is enumerated
    into $A$ under step 1. Then $a_i$ must be attracted to the first
    element that is brought forward after the enumeration. In the second
    case, the larger element is enumerated into $A$ under step 2. Then
    $a_i$ must be attracted to either the element before or after its
    initial attractor. \\
  \end{proof}

\section{$\bm{PS}=\bm{NC}=\bm{ENC}$}
  In this section we show that class of promptly simple degrees are exactly
  the class of non-capable degrees.

\subsection{Promptly Simple Sets}
  \begin{theorem}
    \textbf{(Promptly Simple Degree Theorem)} Let $A$ be an r.e. set and
    $\{A_s\}_{s\in\omega}$ a recursive enumeration of $A$. Then
    $\text{deg}(A) \in\bm{PS}$ iff there is a recursive function $p(s)$ such
    that for all $s$, $p(s)\geq s$, and for all $e$,
    \begin{equation}
      W_e\; \text{infinite}\; \implies (\exists x,s)\; [x\in
      W_{e,\text{at}\; s}\; \wedge\; A_s\restriction x \neq
      A_{p(s)}\restriction x]
      \label{eqn:promptly-permit}
    \end{equation}
  \end{theorem}
  \begin{proof}
    ($\Leftarrow$) Let $B=\Phi^A$ be promptly simple with enumeration
    $\{B_s\}_{s\in\omega}$ and recursive function $p(s)$.
  \end{proof}

  \begin{recall}
    An r.e. set $A$ has a PS degree if it is promptly permitting, i.e.
    there exists a recursive $p(s)\geq s$ such that $\forall e\in\omega$,
    \[|W_e|=\infty \implies (\exists x)(\exists s) [x\in W_{e, \text{at}\;
    s}\; \text{and}\; A_s\restriction x \neq A_{p(s)}\restriction x].\]
  \end{recall}

  \begin{recall}
    An r.e. set $A$ is cappable iff there exists an r.e. set
    $B\prescript{}{T}{>}0$ such that the only r.e. set computable by $A$
    and $B$ are the recursive sets.
  \end{recall}

  \begin{recall}
    An r.e. set $A$ is non-cappable (NC) if it is not cappable.
    Equivalently, an r.e. set $A$ is NC iff given any r.e. set
    $B\prescript{}{T}{>}0$, there exists an r.e. set $C\prescript{}{T}{>}0$
    computable by $A$ and $B$.
  \end{recall}

  \begin{recall}
    An r.e. set $A$ is effectively non-cappable (ENC) if the $C$ above can
    be computed effectively, i.e. there exist recursive functions
    $f(e)$, $g(e)$, $h(e)$ such that for all $e\in\omega$,
    \[\Phi_{f(e)} =\Phi_{g(e)}^A =\Phi_{h(e)}^{W_e},\]
    and if $W_e\prescript{}{T}{>}0$, then $\Phi_{f(e)}\prescript{}{T}{>}0$.
  \end{recall}

  \begin{recall}
    An r.e. degree is NC (ENC) iff it contains a non-cappable (effectively
    non-cappable) set.
  \end{recall}

  \begin{recall}
    $\text{ENC}$ forms a strong filter in $R$.
  \end{recall}

  \begin{recall}
    $\text{PS} \subseteq \text{ENC} \subseteq \text{NC}$.
  \end{recall}

  \begin{goal}
    $\text{NC} \subseteq \text{PS}$.
  \end{goal}

  \begin{corollary}
    $\text{PS}=\text{ENC}=\text{NC}$, and they all form strong filters in
    $R$.
  \end{corollary}

  \begin{proof}
    Fix an arbitrary r.e. set $B$. We show $B$ either has a PS degree or is
    cappable with witness $A$. We construct r.e. $A$ such that the either
    $B$ is found to be PS, or $A$ satisfies for all $i,j\in\omega$ the
    requirements
    \begin{equation}
      N_{i,j}: \Phi_i^A=\Phi_j^B =\text{total function} \implies
      \Phi_i^A\; \text{is recursive}
      \label{eqn:cappable1}
    \end{equation}
  \end{proof}

  \begin{lemma}
    \label{lemma:N}
    For $A$ above to satisfy $N_{i,j}$ for all
    $i,j\in\omega$, it suffices to show $A$ satisfies for all
    $e\in\omega$,
    \begin{equation}
      N_e: \Phi_e^A=\Phi_e^B =\text{total function} \implies \Phi_e^A\;
      \text{is recursive}.
      \label{eqn:cappable2}
    \end{equation}
  \end{lemma}
  \begin{proof}
    Assume that $N_{i,j}$ is not satisfied. If $A=B$, then
    clearly $N_i$ will not be satisfied. So assume $A\neq B$, and fix any
    $n$ such that $A(n)\neq B(n)$. Consider the function
    \begin{align*}
      \Phi_e^X(x) :=
      \begin{cases}
        \Phi_i^X(x) &\text{if}\; X(n)=A(n),\\
        \Phi_j^X(x) &\text{otherwise}.
      \end{cases}
    \end{align*}
    Then $N_e$ will not be satisfied.
  \end{proof}

  \begin{notation}
    From Lemma~\ref{lemma:N}, we can replace requirements $N_{i,j}$ by
    $N_e$. At a given stage $s$, define the length function as
    \[l(e,s) :=\max\{x: \Phi_{e,s}^{A_s}\restriction x
    \downarrow=\Phi_{e,s}^{B_s}\restriction x\},\]
    and
    \[m(e,s) :=\max_{t<s}\{l(e,t)\}.\]
    
    We say $s$ is an \textit{$e$-expansionary} stage if $l(e,s)>m(e,s)$.
    Observe that if $\Phi^A_e=\Phi^B_e$, then there are infinitely many
    $e$-expansionary stages.\\
  \end{notation}

  \begin{proof}
    We construct r.e. $A$ such that for each $e\in\omega$, $A$
    either satisfies $N_e$, or $B$ lies in PS via a recursive function
    $p_e(s)$ and the set of requirements $\forall i\in\omega$
    \[P_{e,i}: |W_i|=\infty \implies (\exists x)(\exists s) [x\in
    W_{i,\; \text{at}\; s}\; \text{and}\; B_s\restriction x \neq B_{p_e(s)}
    \restriction x].\]

    Write
    \[R_e: N_e\; \text{or}\; (\forall i)P_{e,i}.\]
    We construct $A$ and $p_e$ that satisfy $R_e$ for all $e\in\omega$. At
    the same time, we construct partial recursive function $\Psi_{e,i}$
    such that should $R_e$ not fail the second clause,
    $\Psi_{e,i}=\Phi_e^A=\Phi_e^B$ will be total recursive, where $i$ is
    the first index that fails $P_{e,i}$.\\

    \COMMENT{Observe that if there is some $e\in\omega$ such that $R_e$ is
    satisfied by the second clause, then $B\in\text{PS}$, and the remaining
    requirements need not be satisfied.}\\

    At the same time, to ensure $A\prescript{}{T}{>}\;0$, we make $A$
    simple:
    \[P_e: |W_e|=\infty \implies W_e\cap A\neq\emptyset.\]

    \textbf{Outline}
    \begin{tcolorbox}
      \underline{Stage $s$}: Have $x\in W_{i,s}$ and $y\in\omega$ s.t.
      \begin{equation}
        \Phi_{e,s}^{A_s}\restriction y \downarrow=
        \Phi_{e,s}^{B_s}\restriction y,
        \label{eqn:restraint}
      \end{equation}
      where use of above computation $<x$. Set
      \begin{equation}
        \Psi_{e,i}(y)=\Phi_{e,s}^{B_s}(y).
        \label{eqn:Psi}
      \end{equation}
      Set restraint $r(e,s,i)=0$, and cancel lower priority restraints
      $r(e,s,j)$ for all $j>i$. \COMMENT{We say we open the $R_{e,i}$-gap,
      because there are no restraints to protect the
      computation~\eqref{eqn:restraint}.}\\

      If $|W_e\cap A|=\emptyset$ at this stage, enumerate element from
      $W_e$ into $A$ if higher priority restraints
      $r(e,s,0),\ldots,r(e,s,i-1)$, and $r(0,s),\ldots,r(e-1,s)$ are
      respected, where
      \[r(e',s):=\max_{j\in\omega}\{r(e',s,j)\}.\]

      \underline{Stage $s'>s$}: Have $y'>y$ such that
      \begin{equation}
        \Phi_{e,s'}^{A_{s'}}\restriction y' \downarrow=
        \Phi_{e,s'}^{B_{s'}}\restriction y'.
      \end{equation}
      \COMMENT{Such a stage $s'$ the initial segment witnessing
      $\Phi_e^A=\Phi_e^B$ is the longest so far is called an
      $e$-expansionary stage, in the sense that the number of witnesses for
      $\Phi_e^A=\Phi_e^B$ increases.} Set $p_e(s)=s'$, and set
      \begin{equation}
        r(e,s,i)\; \text{to protect computation~\eqref{eqn:restraint}}.
        \label{eqn:r}
      \end{equation}
      \COMMENT{We say we close the $R_{e,i}$-gap because we no longer allow
      elements to be enumerated into $A$ if they may injure
      computation~\eqref{eqn:restraint}.}
    \end{tcolorbox}

    Use ``simultaneous strategies'' technique for infinite-injury
    construction (section~\ref{section:infinite-injury}) to ensure
    \[\liminf_{s\in\omega} \tilde{r}(e,s) <\infty.\]

    \textbf{Verification}
    \begin{tcolorbox}
      If there is some $e$ such that $R_e$ is satisfied by the second
      clause $(\forall i)P_{e,i}$, then $B\in\text{PS}$ and we are done.
      \COMMENT{Don't need to bother if other $R_{e'}$ are satisfied.} So
      assume for all $e$ there exists $i$ such that $\neg P_{e,i}$. Let $i$
      be the least such $i$. We show $N_e$ is satisfied. Specifically,
      \[\Phi^A_e=\Phi^B_e=\Psi_{e,i} =\text{recursive}.\]

      By assumption of $N_e$,
      \[\Phi^A_e=\Phi^B_e.\]
      Thus since $|W_i|=\infty$, the $R_{e,i}$-gap will be open and closed
      infinitely often, and $p_e(s)$ will be a total recursive function.\\

      To compute $\Phi^B_e(y)$, wait till higher priority requirements
      stabilize. In particular, since
      \[\liminf_s \max\{r(0,s),\ldots,r(e-1,s)\} <\infty,\]
      consider only stages $s\in S$ where the infinum occurs. Simulate to
      stage $s\in S$ where
      \begin{equation}
        \Phi_{e,s}^{A_s}\restriction y \downarrow=
        \Phi_{e,s}^{B_s}\restriction y.
      \end{equation}
      Then the above computation cannot be injured when the $R_{e,i}$-gap
      is open, otherwise we would have
      \[B_s\restriction x \neq B_{p_e(s)}\restriction x,\]
      contradicting $\neg P_{e,i}$. The computation also cannot be injured
      after the gap is closed because the restraint $r(e,s,i)$ will protect
      it forever since it is the highest priority restraint after higher
      priority requirements have stabilized.
    \end{tcolorbox}
  \end{proof}

\section{High Minimal Pair}
  \begin{goal}
    There exists a minimal pair of high r.e. degrees.
  \end{goal}

  We construct high r.e. sets $A$, $B$ whose degrees form a minimal pair.
  For minimality, satisfy the usual negative requirements
  \[N_e: \Phi_e^A=\Phi_e^B=\text{total} \implies \Phi_e^A\;
  \text{recursive}.\]

  For highness, we need the following lemma.
  \begin{lemma}
    There exists an r.e. set $C$ such that if r.e. set $A$ is a thick
    subset of $C$, then $A$ is high.
  \end{lemma}
  \begin{proof}
    Recall $\emptyset''\equiv_T\text{Tot}$. In
    particular, $\emptyset''\leq_T\text{Tot}$, so exists recursive $f(e)$
    s.t.
    \[e\in\emptyset'' \Leftrightarrow f(e)\not\in\text{Tot}.\]
    Consider recursive $g(e)$ defined as
    \begin{align*}
      \varphi_{g(e)}(x) :=
      \begin{cases}
        0, &\text{if}\; \varphi_e(0),\ldots,\varphi_e(x)\downarrow,\\
        \uparrow, &\text{otherwise}.\\
      \end{cases}
    \end{align*}

    Then
    \begin{align*}
      e\not\in\text{Tot} &\implies g(e)\in\text{Fin},\\
      e\in\text{Tot} &\implies g(e)\in\text{Tot}.
    \end{align*}

    Let $h=g\circ f$. Then
    \begin{align*}
      e\in\emptyset'' &\implies h(e)\in\text{Fin},\\
      e\not\in\emptyset'' &\implies h(e)\in\text{Tot}.
    \end{align*}

    Choose
    \[C :=\{\langle x,e\rangle: x\in W_{h(e)}\}.\]
    Then if $A$ is a thick subset of $C$,
    \[e\in\emptyset'' \implies C^{[e]}\; \text{finite} \implies A^{[e]}\;
    \text{finite} \implies \lim_x A(\langle x,e\rangle)=0,\]
    and
    \[e\not\in\emptyset'' \implies C^{[e]}=\omega \implies A^{[e]}\;
    \text{cofinite} \implies \lim_x A(\langle x,e\rangle)=1.\]

    Thus
    \[\emptyset''\leq_T \lim_x A(\langle x,e\rangle) \leq_T A',\]
    so $A$ is high.
  \end{proof}

\subsection{Tree Method for Infinite Injury Construction}
  \label{section:infinite-injury}
  Construct r.e. $A$ satisfying
  \[R_e: N_e\; \text{and}\; P_e.\]
  \begin{itemize}
    \item $N_e$: Sets restraint $r(e,s)$ to prevent elements from being
      enumerated into $A$ to protect computation.
      \begin{align*}
        N_e:& \;\Phi_e^A=\Phi_e^B=\text{total}\; \implies
          \Phi_e^A\;\text{recursive}\\
      \end{align*}

    \item $P_e$: Enumerate elements into $A$. Often there are
      infinite elements that can be enumerated to satisfy $P_e$.
      \[P_e: A^{[e]} =^* C^{[e]}\; \text{for some fixed r.e.}\; C\]
  \end{itemize}

  Some $P_e$'s enumerate infinite elements into $A$. Need
  higher priority restraints
  \[R(e,s) := \max\{r(0,s),\ldots,r(e,s)\}\]
  to fall simultaneously infinitely often to satisfy such $P_e$'s. I.e.
  need
  \begin{equation}
    \liminf_{s\in\omega} R(e,s) <\infty.
    \label{eqn:liminf}
  \end{equation}
  Use tree construction.

  \begin{definition}
    $\Lambda$ countable set with linear ordering $<_\Lambda$. Define
    $T=\Lambda^{<\omega}$, and let $[T]$ be the set of infinite paths
    through $T$. Lower cases Greek letters $\alpha,\beta,\gamma,\ldots$
    range over $A^{<\omega}$, and $f$ and $g$ range over $\Lambda^\omega$.
  \end{definition}

  \begin{definition}
    Let $\alpha$, $\beta\in T$. Define linear ordering $\alpha<\beta$
    if either $\alpha \prec \beta$, or
    \[(\exists \gamma \prec \alpha, \beta) (\exists a,b \in \Lambda)\;
    [\gamma^\frown a \preceq\alpha\; \wedge\; \gamma^\frown b
    \preceq\beta].\]
  \end{definition}

  In tree construction, each level in $T$ denotes a requirement,
  and $\Lambda$ is the set of outcomes.

  \underline{Construction Technique of Simultaneous Strategies
  for~\eqref{eqn:liminf}}\\
  Write
  \[\tilde{r}(e,s) :=\max_{e'\leq e}\{r(e',s)\}.\]
  We have
  \[\liminf_{s\in\omega} \tilde{r}(e-1,s) =p<\infty.\]
  We want
  \[\liminf_{s\in\omega} \tilde{r}(e,s)<\infty.\]

  When it is time for $R_e$ to act, say at stage $s$, the strategy
  $\sigma_e$ of $R_e$ differs depending on what $p$ is. ``Guess'' $p$ as
  \[k=\tilde{r}(e-1,s),\]
  and play strategy $\sigma_e^k$. Thus, $R_e$ simultaneously plays
  infinite strategies $\sigma_e^0,\sigma_e^1,\ldots$, playing strategy
  $\sigma_e^{k'}$ at stages in
  \[S_{k'} :=\{s\in\omega: \tilde{r}(e-1,s)=k'\}.\]

  Strategies $\sigma_e^{k'}$ with lower $k'$ values
  have higher priority - for each $k'>k$, destroy restraint
  $r^{k'}(e,s)$ associated with strategy $\sigma_e^{k'}$.
  \COMMENT{Destruction ensures that if $p$ is indeed $k$, then
  $\tilde{r}(e,s)$ will fall whenever $\tilde{r}(e-1,s)=p$.}\\

  \underline{Verification}\\
  Have
  \[\liminf_{s\in\omega} \tilde{r}(e-1,s) =p<\infty.\]
  Want to show
  \[\liminf_{s\in\omega} \tilde{r}(e,s)<\infty.\]
  Since 
  \[\liminf_{s\in\omega} \tilde{r}(e-1,s) =p,\]
  eventually strategies $\sigma_e^0,\ldots,\sigma_e^{p-1}$ will
  stop playing. In particular, restraints
  $r^0(e,s),\ldots,r^{p-1}(e,s)$ associated with these strategies
  stop rising. After such a stage $s_0$, strategy $\sigma_e^{p}$ will
  play infinitely often, and when it is played, restraints for lower
  priority strategies are destroyed. Yet when lower priority strategies are
  played at stages $s$ where $\tilde{r}(e-1,s)>p$,
  restraints established by $\sigma_e^{p}$ are not injured. Thus,
  strategy $\sigma_e^{p}$ is played infinitely often and never
  injured after a certain stage, so that
  \[\liminf_{s\in\omega} \tilde{r}(e,s) =\max\{p,
  \liminf_{s'\in\omega} r^{p}(e,s')\} <\infty,\]
  where
  \[\liminf_{s'\in\omega} r^{p}(e,s') <\infty\]
  by assumption.
\end{document}
