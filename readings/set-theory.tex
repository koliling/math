\subsection{Stable}
  \begin{theorem}
    There exists a minimal pair of r.e. sets with high degree.
  \end{theorem}

  Recall that we fix some r.e. $C$ where
  \begin{align*}
    e\in\emptyset'' &\implies |C^{[e]}|<\omega,\\
    e\not\in\emptyset'' &\implies C^{[e]}=\omega,\\
  \end{align*}
  and we construct r.e. $A$, $B$ satisfying
  \begin{align*}
    N_e: &\;\Phi_e^A=\Phi_e^B=\text{total} \implies \Phi_e^A\;
      \text{recursive},\\
    P_e: &\;A^{[e]} \subseteq^* C^{[e]}\; \text{and}\; B^{[e]} \subseteq^*
      C^{[e]}.\\
  \end{align*}

  Prioritize
  \[N_0, P_0, N_1, P_1, \ldots.\]

  At stage $s$, have $\delta_s\in T$, $|\delta_s|=2s$, where $\delta_s$
  `guesses' the true path $f\in[T]$, and where
  \begin{enumerate}[label=(\roman*)]
    \item $f(n) :=\liminf_s \delta_s(n) <\omega$ exists, being the
      `left-most' path visited infinitely often
    \item $f(2e)=r(e) :=\liminf_s r(e,s)$
    \item $f(2e+1)=\begin{cases} 0, &\text{if}\; |C^{[e]}|=\omega,\\ 0,
      &\text{otherwise}. \end{cases}$
  \end{enumerate}

  \textbf{Construction}
  \begin{tcolorbox}
    At stage $s+1$, do for $e$ from 0 to $s$:
    \begin{enumerate}
      \item For each $\beta\leq \delta_s\restriction e$, define
        \begin{align*}
          r(\beta,s) =
          \begin{cases}
            0, &\text{if}\; \beta\preceq\alpha\; \text{and}\; s\;
              \text{is}\; \beta\text{-expansionary},\\
            \text{the greatest}\; \beta\text{-expansionary stage}\; t<s,
              &\text{otherwise}.\\
          \end{cases}
        \end{align*}
      \item Then define
        \[\delta_s(e) =r(e,s) :=\max_{\beta\leq\alpha}\{r(\beta,s)\}.\]
    \end{enumerate}

    Choose smallest $x=\langle y,e\rangle<s$, $x\in C_s^{[e]}$ such
    that $x>r(e,s)=\delta_s(e)$ but $x$ has not been enumerated into
    $A$ or $B$ yet. Enumerate $x$ into $A$ if $x\not\in A_s$, otherwise
    enumerate $x$ into $B$. Note that at most one element enters $A$ or
    $B$, which is necessary to satisfy $N_e$ since at most one of $A$ or
    $B$ computation can be injured at one stage.
  \end{tcolorbox}

  Construct r.e. sets $A$, $B$ satisfying $N_e$ and $P_e$. $N_e$ sets
  restraints $r(e,s)$ to protect computation. Some $P_e$'s enumerate
  infinite elements into $A$ and $B$. Need higher priority restraints
  \[r(0,s),\ldots,r(e,s)\]
  to fall simultaneously infinitely often to satisfy such $P_e$'s. I.e.
  need
  \begin{equation}
    \liminf_{s\in\omega} \max\{r(0,s),\ldots,r(e,s)\} <\infty.
    \label{eqn:liminf}
  \end{equation}
  Use tree construction.

  \begin{definition}
    $\Lambda$ countable set with linear ordering $<_\Lambda$. Define
    $T=\Lambda^{<\omega}$, and let $[T]$ be the set of infinite paths
    through $T$. Lower cases Greek letters $\alpha,\beta,\gamma,\ldots$
    range over $A^{<\omega}$, and $f$ and $g$ range over $\Lambda^\omega$.
  \end{definition}

  \begin{definition}
    Let $\alpha$, $\beta\in T$. Define linear ordering $\alpha<\beta$
    if either $\alpha \prec \beta$, or
    \[(\exists \gamma \prec \alpha, \beta) (\exists a,b \in \Lambda)\;
    [\gamma^\frown a \preceq\alpha\; \wedge\; \gamma^\frown b
    \preceq\beta].\]
  \end{definition}

  In tree constructions, each level in $T$ represents a requirement, so
  $\Lambda$ represents the set of outcomes of the requirements. Prioritize
  requirements
  \[N_0, P_0, N_1, P_1, N_2, P_2, \ldots.\]
  Level $e$ represents $N_e$. Choose $\Lambda=\omega$ and
  $T=\omega^{<\omega}$. \\

  At stage $s$, observe outcome $N_0,\ldots,N_s$, and represent the
  outcomes by $\delta_s\in T$ where $|\delta_s|=s$:
  \[\delta_s(e) :=\text{guess for}\; \liminf_s r(e,s) =r(e,s).\]
  
  Thus have a sequence of strings $\delta_s$ of increasing lengths. At the
  end of construction, want ``true path'' $f\in[T]$ to be the $\liminf$ of
  $\delta_s$, in the sense that $f$ is the left-most path that is visited
  infinitely often: for all $n\in\omega$,
  \[(\exists^\infty s)\; [f\restriction n \prec
  \delta_s],\]
  and
  \[(\exists^{<\infty} s)\; [\delta_s < f\restriction n].\]

  Equivalently, for all $n\in\omega$,
  \[f(n) :=\liminf_s \delta_s(n).\]

  To construct the r.e. sets, at stage $s$, observe the outcome
  $\delta_s\in T$, and play the $\delta_s$-strategy. At stage $s$,
  associate with each string $\alpha\in T$ a restraint $r(\alpha,s)$.
  When playing $\alpha$-strategy, respect all strategies $\beta$ to the
  left of $\alpha$:
  \begin{itemize}
    \item Ensure $r(\alpha,s) \geq \max\{r(\beta,s): \beta<\alpha\}$
    \item $\forall\beta<\alpha$, reset $r(\beta,s+1)=0$.
  \end{itemize}

  The idea is that the strings to the right of the ``true path'' should not
  have a lasting effect on the construction because every time the ``true
  path'' is visited, we will destroy all restraints imposed by the paths on
  the right. On the other hand, paths to the left of the ``true path'' can
  only impose a finite restraint on the construction, since they are only
  visited finitely often. \\

  \begin{definition}
    At stage $s$, we say $s$ is called an $\alpha$-stage if
    $\alpha\preceq\delta_s$.
  \end{definition}

  \begin{definition}
    Given $\alpha\in T$, denote
    \[S^\alpha := \{s\in\omega: s\; \text{is an}\; \alpha\text{-stage}\}.\]
  \end{definition}

  \begin{definition}
    Given $\alpha\in T$, define length function
    \[l(\alpha,s) := \max\{x: (\forall y<x) [\Phi_{e,s}^{A_s}(y)
      \downarrow =\Phi_{e,s}^{B_s}(y)]\}.\]
    Define also
    \[m(\alpha,s) := \max\{l(\alpha,t): t\leq s\; \text{and}\; t\in
    S^{\alpha}\}.\]
  \end{definition}

  \begin{definition}
    A stage $s$ is $\alpha$-expansionary if $s=0$, or if $s>0$, $s$ is an
    $\alpha$-stage, and $l(\alpha,s)>m(\alpha,s-1)$.
  \end{definition}

  \textbf{Construction}
  \begin{tcolorbox}
    Stage 0: Set $A_0=B_0=\emptyset$, $r(\alpha,0)=0$ for all $\alpha\in
    T$. \\

    Stage $s+1$: Do for $e$ from 0 to $s$:
    \begin{enumerate}
      \item For each $\beta\leq \delta_s\restriction e$, define
        \begin{align*}
          r(\beta,s) =
          \begin{cases}
            0, &\text{if}\; \beta\preceq\alpha\; \text{and}\; s\;
              \text{is}\; \beta\text{-expansionary},\\
            \text{the greatest}\; \beta\text{-expansionary stage}\; t<s,
              &\text{otherwise}.\\
          \end{cases}
        \end{align*}
      \item Then define
        \[\delta_s(e) =r(e,s) :=\max_{\beta\leq\alpha}\{r(\beta,s)\}.\]
      \item Choose smallest $x=\langle y,e\rangle<s$, $x\in C_s^{[e]}$ such
        that $x>r(e,s)=\delta_s(e)$ but $x$ has not been enumerated into
        $A$ or $B$ yet. Enumerate $x$ into $A$ if $x\not\in A_s$, otherwise
        enumerate $x$ into $B$. Note that at most one element enters $A$ or
        $B$, which is necessary to satisfy $N_e$ in since we need to hold
        one side of the computation.
    \end{enumerate}
  \end{tcolorbox}

  \begin{lemma}
    \textbf{True Path Lemma} There exists $f\in T$, called the \textit{true
    path}, such that
    \begin{enumerate}[label=(\roman*)]
      \item $(\forall n) [f\restriction n =\liminf_s \delta_s\restriction
        n]$
      \item $(\forall e) [f(e) =r(e) :=\liminf_s r(e,s) <\infty]$
    \end{enumerate}
  \end{lemma}
  \begin{proof}
    We prove (i) and (ii) by simultaneously by induction. Assume statements
    are true for $n=0,\ldots e$. Choose $t_0\in\omega$ such that $\delta_s$
    stops visiting the left of $f\restriction e$. Let $t_1\geq t_0$ be such
    that $\delta_{t_1}$ is a $f\restriction e$-stage. Let
    \[k :=\max\{r(\beta, t_1): \beta<f\restriction e\}.\]

    If there are infinitely many $f\restriction e$-expansionary stages $s$,
    then $r(f\restriction e,s)=0$ at each such $s$ such, so $f(e)=r(e)=k$.
    Otherwise, there is a largest such stage $v$, so that $r(f\restriction
    e,s)=v$ for almost all $s$, then $r(e)=\max(k,v)$. In either cases, we
    have both (i) and (ii).
  \end{proof}

  \begin{lemma}
    $(\forall e) [A^{[e]}=^*C^{[e]}=^*B^{[e]}]$.
  \end{lemma}
  \begin{proof}
    We prove the stronger statement that
    \[A^{[e]}\supseteq C^{[e]}-\{0,1,\ldots,r(e)\},\]
    and similarly for $B^{[e]}$. Choose any $x\in C^{[e]}$ where $x>r(e)$.
    Choose $s_0$ such that $x\in C_{s_0}$ and all $y<x$ contained in $A$ or
    $B$ have already been enumerated into them. At stage $s>s_0$ where
    $r(e,s)=r(e)$, $x$ will be enumerated into $A$ if not already done so.
    Then at the stage after where $r(e,s)=r(e)$ again, $x$ will be
    enumerated into $B$ if not already done so.
  \end{proof}

  \begin{lemma}
    $(\forall e)[\Phi_e^A=\Phi_e^B=\text{total} \implies\text{recursive}]$.
  \end{lemma}

  \begin{notation}
    Given an $A$-computation $\Phi_{e,s}^{A_s}(x)=y$ and $\alpha\in T$, we
    say the computation is $\alpha$-correct at stage $s$ if in the use $u$ of
    the above computation,
    \[(\forall i<e)(\forall z\leq u) [z>\alpha(i) \wedge z\in \omega^{[i]}
    \implies z\in A_s^{[i]}].\]
    Similarly for $B$-computation but with $A$ replaced by $B$. \\

    Observe that for a fixed $\alpha\geq f\restriction e$, if
    $\Phi_e^A(x)=y$, then the $A$-computation will be $\alpha$-correct at
    cofinite stages.
  \end{notation}

  \begin{proof}
    Assume $\Phi_e^A=\Phi_e^B=\text{total}$. Write $\alpha:=f\restriction
    e$. To compute $\Phi_e^A(x)$, wait for stage $s_0$ where higher
    priority requirements stabilize, i.e.
    \[(\forall s\geq s_0)\; \alpha\preceq\delta_s.\]

    Now for each $y\leq x$, since $\Phi_e^A(y) \downarrow=\Phi_e^B(y)$,
    these computations must be $\alpha$-correct at cofinite stages. So wait
    for stage $s_1>s_0$, $s_1\in S^\alpha$ such that for all $y\leq x$,
    \begin{itemize}
      \item $\Phi_{e,s}^{A_s}(y) \downarrow=\Phi_{e,s}^{B_s}(y)$
      \item The above computations are $\alpha$-correct.
    \end{itemize}

    Then since $r(e,s)\geq \alpha(e)$ for all $s\geq s_1$, no elements
    below $\alpha(e)$ will be enumerated into $A$ or $B$ after stage $s_1$.
    Also, since our construction will enumerate elements into only one of
    $A$ or $B$ in the above computation at any stage and never both at the
    same time, and we always protect one side of the computation, the
    computation of $\Phi_e^A(x)$ at stage $s_1$ will be protected forever.
  \end{proof}
