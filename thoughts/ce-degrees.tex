\begin{theorem}
  What is the history behind simple sets?
\end{theorem}
\begin{proof}
  Perhaps simple sets came about when people tried to construct
  r.e. sets that are non-computable. 
\end{proof}

\begin{question}
  Must the cap of every pair of r.e. degrees, if it exist, also be an
  r.e. degree?
\end{question}
\begin{proof}
\end{proof}

\begin{theorem}
  There are non c.e. degrees below $\emptyset'$.
\end{theorem}
\begin{proof}
  Cooper Epstein-Lachlan theorem. Also, the 1-generics, from new Soare
  E6.3.5iii, and the fact that there exists 1-generics below
  $\emptyset'$. Finally, the 1-randoms too. By the Low Basis theorem,
  applied to the universal Martin-Lof tree, there must be a 1-random
  that is low. And 1-randoms cannot be c.e.
\end{proof}

\begin{theorem}
  There are no minimal c.e. degrees.
\end{theorem}
\begin{proof}
  Sack's Density theorem.
\end{proof}

\begin{question}
  Do all non-minimal degrees have infinite incomparable degrees below
  them? How about just finite?
\end{question}
\begin{proof}
\end{proof}
