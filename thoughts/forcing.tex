\begin{question}
  What exactly is forcing?
\end{question}
\begin{proof}
  Forcing is a technique of creating ZFC models from an existing countable
  transitive model by appending to the model a new element and other elements
  that are constructible from the new element. Forcing says that as long as
  the new element satisfies certain loose conditions, the extended model will
  satisfy ZFC.\\

  Model extension is analogous to extending an existing field $R$ with an
  object $x$ to get an extended field $R(x)$. To ensure $R(x)$ is still a
  field, $R(x)$ must also include constructible elements such $x+1$,
  $(x+1)/x^3$, and so on. Extending a model of ZFC however is not as easy
  as extending a model of fields. The difficulty is due to the infinite
  number of axioms to satisfy - we may un-satisfy a previously satisfied
  axiom after adding a new element. Hence, forcing was developed to ensure
  that the extended model preserves ZFC.\\

  There are two ways to understand forcing. The first way is by following
  Cohen's thoughts when he developed forcing: To create an extension model
  from an existing countable transitive model $M$, Cohen adds to $M$ a
  subset $G$ of a set $P\in M$ to get another countable transitive model
  $M[G]$. Forcing says that $P$ and $G$ only need satisfy very loose
  conditions to ensure that $M[G]$ is also a model of ZFC. \\

  So how do we define $M[G]$ given $M$ and $G$?
  $M[G]$ should extend $M$, include $G$, and also include all sets that are
  constructible from $G$. Is there a natural definition of $M[G]$ that
  would satisfy all these conditions? From Bell's argument mentioned
  earlier, we cannot define $M[G]$ as the class of elements constructible
  from $M$ and $G$, because this class may not satisfy ZFC. Hence Cohen
  explored a new way of defining $M[G]$, through exploiting $M$'s existing
  ZFC ``structure'', where any set $A\in M$ is associated with a class of
  ZFC extensions of $A$ which are $A$'s union sets, pairing sets,
  comprehension sets, replacement sets, and powerset. Similarly, $M[G]$
  should include $G$ and all its ZFC extensions. Hence, to define $M[G]$,
  Cohen first designed a function from $M$ to $M[G]$ that preserves the ZFC
  functions of union, pairing, comprehension, and replacement. More
  precisely, this function $f$ maps the union of $A$ to the union of
  $f(A)$, the pairing of $A$ and $B$ to the pairing of $f(A)$ and $f(B)$,
  and so on. Since $f$ preserves ZFC functions, if it also includes $G$ and
  all elements of $M$ in its range, then its range is likely to be a model
  of ZFC.  Hence, Cohen set $M[G]$ to be the range of $f$, and proved that
  $M[G]$ would be a ZFC extension of $M$.\\

  To preserve ZFC functions, it is natural to design the function $f$
  recursively, as follows:
  \begin{itemize}
    \item Map $\emptyset$ to $\emptyset$
    \item Map $\{s_0,s_1,\ldots\}$ to $\{f(g(s_0)),f(g(s_1)),\ldots)\}$
  \end{itemize}

  The question left unanswered is how should we define function $g$ such
  that $f$ contains $G$ and all of $M$ in its range and preserves all ZFC
  functions. To satisfy these two conditions, Cohen set $g$ to be defined
  only on ordered pairs $\langle a,b\rangle$ where $b\in G$, and mapped
  $\langle a,b\rangle$ to $a$. Cohen easily verified that this definition
  satisfies the first condition but only partially satisfies the second.
  Specifically, $f$ may not preserve more complex ZFC functions such as the
  power function. Hence, Cohen searched for additional constraints on $G$
  that will resolve this issue. The constraints turned out to very loose,
  yet tedious to verify how they helped $f$ preserve the power function.
  Very roughly speaking, in order for $f$ to preserve the power function,
  we need to find for any set $s\in M$
\end{proof}
