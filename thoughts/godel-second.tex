Godel's second incompleteness theorem appears to be a by-product of
formalizing derivability. Given sentence $\sigma$, let Dr($\sigma$)
abbreviate the sentence ``$\sigma$ is derivable''.  To formalize
derivability, some natural questions to ask are:
\begin{itemize}
  \item If PA derives $\sigma$, does it derive Dr($\sigma$)?
  \item If PA derives Dr($\sigma$), does it derive Dr(Dr($\sigma$))?
  \item If PA derives Dr($\sigma_0\rightarrow\sigma_1$) and
    Dr($\sigma_0$), does it derive Dr($\sigma_1$)?
\end{itemize}

The answer to these three questions turn out to be yes. The three rules
are called derivability conditions, and are used to formalize
derivability.\\

Godel's second incompleteness theorem is proved by applying the
derivability conditions on the sentence $\theta$ which says ``I am not
derivable''. In technical terms, PA derives the sentence
``$\theta\leftrightarrow\neg\text{Dr}(\theta)$''.\\

Let Con(PA) abbreviate the sentence that says PA is consistent. More
precisely, Con(PA) is defined as $\neg$Dr(0=1).\\

Our final goal is to show that PA cannot derive Con(PA). We prove this
by applying the derivability conditions on $\theta$.\\

First, by applying the first derivability condition, we get $\theta$ is
inconsistent: $\theta$ implies $\neg\text{Dr}(\theta)$ from choice of
$\theta$, yet also implies $\text{Dr}(\theta)$ from the first
condition.\\

Hence PA derives ``$\theta\rightarrow0=1$'', and by applying the third
derivability condition on this sentence, we get PA derives
``$\text{Dr}(\theta)\rightarrow\text{Dr}(0=1)$'',
which is equivalent to ``Con(PA)$\rightarrow\neg\text{Dr}(\theta)$''.
Since $\neg\text{Dr}(\theta)$ is the same as $\theta$ from our choice
of $\theta$, and $\theta$ is inconsistent from earlier argument, we
finally get ``Con(PA)$\rightarrow0=1$'', as desired.\\

We can interpret the above proof of the second incompleteness theorem as a
formalization of the following argument: Recall that the first
incompleteness theorem says that if PA is consistent then it cannot prove
$\theta$. But not proving $\theta$ is the same as proving $\theta$, from
the definition of $\theta$. Hence if we show that PA is consistent then we
get a contradiction. 
