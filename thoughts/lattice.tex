\begin{question}
  Where are the PA reals? Do the class of PA reals have measure 1?
\end{question}
\begin{proof}
  They are probably everywhere. Intuitively you can $\emptyset'$-encode
  any infinite set into some PA real, so it seems likely that the class
  of PA reals have measure 1. Given arbitrary real $A\in2^\omega$, the
  $\emptyset'$-encoding of $A$ into a PA real $P\in2^\omega$ can be
  performed as follows. Since PA reals are incomplete, every such real
  has infinite initial segments for which can branch both ways and
  still get PA reals. At stage $s$, let $\emptyset'$ decide when the
  first extension which branches both ways occur. Take the left branch
  iff $A(s)=0$. \\

  By construction, $A$ is $(\emptyset'\oplus P)$-recursive. In
  particular, the PA reals cannot be covered by any real $X$, for this
  would imply that all reals are covered by $X\oplus\emptyset'$.\\

  Furthermore, from new Soare Theorem 10.3.3.iii, the PA degrees are
  exactly those that compute a 2-valued d.n.c. function. Since
  $\emptyset'$ computes a d.n.c. function, every degree above
  $\emptyset'$ contains a PA real. So PA reals are everywhere. Though
  it is not clear the class of them must have measure 1.
\end{proof}

\begin{question}
  If $\emptyset' \leq_T A\leq_T \emptyset'\oplus B$, does this imply
  that $B\geq_T\emptyset'$?
\end{question}
\begin{proof}
\end{proof}

\begin{question}
  What is the motivation for defining simple sets? How about
  1-generics, or maximal sets?
\end{question}
\begin{proof}
  Simple sets were defined to answer Post's question, which asks if
  there exists a non-computable set below the jump. Perhaps 1-generics
  were defined to construct sets through forcing the jump.
\end{proof}

\begin{theorem}
  Make sense of fixed-point, and Arslanov's completeness criteria.
\end{theorem}
\begin{proof}
  How do we interpret the proof of the fixed-point (recursion) theorem?
  The theorem says given a recursive $f(e)$, there exists a function
  indexed $n\in\omega$ such that $\varphi_n=\varphi_{f(n)}$.\\

  The fixed-point theorem was proved because people failed to
  diagonalize out of the partial-recursives. So assume by contradiction
  we have a recursive $f$ with no fixed-point. We use shall $f$ to
  construct a partial-recursive $g$ which diagonalizes out of all the
  partial-recursives, i.e. $g(e)\neq\varphi_e(e)$ for all
  $e\in\omega$.\\

  Now $f$ has no fixed-point, so $\varphi_x\neq\varphi_{f(x)}$ for all
  $x\in\omega$. In particular, $\varphi_{\varphi_e(e)} \neq
  \varphi_{f(\varphi_e(e))}$ for all $e\in\omega$. Thus set $g(e)$ to
  be an index of $\varphi_{f(\varphi_e(e))}$; this index must be
  different from $\varphi_e(e)$. Also SMN-theorem gives us an effective
  way of computing $g(e)$ since $f$ is partial-recursive. Thus $g$
  diagonalizes out of the partial-recursives, yet it is
  partial-recursive, a contradiction.\\

  The second way of interpreting the proof of the fixed-point theorem
  is as follows. The fixed-point $n$ that we find shall say
  \begin{center}
    \textit{``I am the $f$ of myself.''}
  \end{center}

  This makes sense since we want $\varphi_n$ to equal in some sense to
  ``the $f$ of itself'' $\varphi_{f(n)}$. A function indexed by $e$ may
  refer to itself using the function indexed $\varphi_e(e)$. So we are
  looking for a function indexed $n=\varphi_v(v)$ for some $v$ such
  that
  $\varphi_v(v)=f(\varphi_v(v))$, i.e.
  \[\varphi_n :=\varphi_{\varphi_v(v)} =\varphi_{f(\varphi_v(v))}.\]

  So we consider the function indexed $v$ given by
  \[\varphi_v(e) :=f(\varphi_e(e)),\]
  and let $n=\varphi_v(v)$. Then we have
  \[\varphi_{f(n)} =\varphi_{f(\varphi_v(v))} =\varphi_{\varphi_v(v)}
  =\varphi_n\]
  as desired. Thus completes the general idea of the proof of the
  recursion theorem. \\

  Using the recursion theorem, we can define functions `recursively' in
  the sense that we can define a desired $g(x)$ in terms of itself. Say
  we want $g(x)$ defined recursively as
  \[g(x)=h(g(x),x)\]
  for some recursive $h(x,y)$. To prove that such $g(x)$ exists,
  define recursive $g'(e)$ such that
  \[\varphi_{g'(e)}(x) :=h(\varphi_e(x),x).\]

  Apply recursion theorem to get $n\in\omega$ such that
  \[\varphi_n(x) =\varphi_{g'(n)}(x) =h(\varphi_n(x),x).\]
  Such $\varphi_n(x)$ is the $g(x)$ we are looking for.
\end{proof}

\begin{theorem}
  There exists a degree incomparable with $\emptyset'$.
\end{theorem}
\begin{proof}
  From the paragraph after Soare's D5.6.1, the hyperimmune-free degrees
  are exactly those that are incomparable with $\emptyset'$. Also
  recall that a set is hyperimmune if and only if it is not computably
  dominated. The construction of Soare T9.5.1 gives a noncomputable set
  with a hyperimmune-free degree. \\

  Alternatively, from new Soare E6.3.5iii, 1-generics cannot compute
  $\emptyset'$. Then from E6.3.4, a nonlow 1-generic cannot be computed
  by $\emptyset'$. Therefore nonlow 1-generics are imcomparable with
  $\emptyset'$.
\end{proof}

\begin{question}
  Given $A\leq_T B$, must there be a degree between them? I.e. Does
  Sack's density theorem be generalized to degrees that are not
  necessarily c.e.? How about infinite pairwise incomparable degrees
  between $A$ and $B$?
\end{question}
\begin{proof}
  Consider local and global degrees?
\end{proof}

\begin{question}
  Given any non-recursive non-minimal degree, are there always infinite
  pairwise incomparable degrees beneath it?
\end{question}
\begin{proof}
\end{proof}

\begin{question}
  Is $\mathcal{D}$ a lattice? In particular, do any two Turing degrees
  have a greatest lower bound?
\end{question}
\begin{proof}
  No. Clearly the join of two given degrees give their lub. But from
  Klenne-Post-Spector (old Soare VI.C4.4), there are degrees with no
  greatest lower bound.
\end{proof}

\begin{question}
  What do we know about the minimal degree below $\emptyset'$
  constructed in Soare 13.1.1? In particular, is it low? Does it have a
  c.e. degree?
\end{question}
\begin{proof}
  From Soare 13.7.1 every non-computable c.e. degree computes a minimal
  degree. In particular, minimal degrees can be low.
\end{proof}

\begin{question}
  Does $\mathcal{D}_{\leq\emptyset'}$ have a maximal degree?
\end{question}
\begin{proof}
  \OPEN{This is asking if $\emptyset'$ is a strong minimal cover of any
  degree below it. We know there are degrees with strong minimal
  covers, but we do not know if $\emptyset'$ is such a cover. Look up
  Andrew Lewis \url{http://www.aemlewis.co.uk/pdf/SMCtssf.pdf} or
  \url{https://projecteuclid.org/euclid.jsl/1183746187}.}
\end{proof}
