\begin{question}
  Does pseudorandomness affect if P = NP?
\end{question}

\begin{proof}
  Before we can answer this question, we need to define what we mean by
  randomness in this context. By randomness, I do not mean true
  randomness, but rather a pseudorandom generator, where a deterministic
  program generates a series that is sufficiently random in the sense
  that no efficient computation can distinguish the series from a truly
  random one by a non-negligible advantage. Note that deterministic
  machines can never be truly random by Kolmogorov's definition of
  randomness, which says that a sequence is random if it cannot be
  generated by a Turing machine shorter than the series (wiki
  randomness, mathematics).\\

  The pseudorandom generator theorem says randomness implies P != NP. Yet,
  at first glance, randomness also seems to imply P = NP, because we may
  be able to use randomness to solve exponential problems in polynomial
  time. So which is it? The short answer is, the first implication is
  correct.\\

  More precisely, the existence of pseudorandom functions implies that
  one-way functions exists, which further implies that P != NP.\\

  To show the first implication, we construct a one-way function from a
  pseudorandom generator by simply taking the first half of the
  generator's output. Formally, let $f:\{0,1\}^n \rightarrow
  \{0,1\}^{2n}$ be the pseudorandom generator. Construct the one-way
  function $F:\{0,1\}^n\rightarrow\{0,1\}^n$ as $F(x)=$ first half of
  $f(x)$. Intuitively, we can see that $F$ will be easy to compute but
  difficult to invert.\\

  The existence of one-way functions almost directly implies P != NP by
  definition: the inverse of one-way functions are easy to verify but
  difficult to compute, which is the defining characteristic of
  NP-complete functions.\\

  For more information on the pseudorandom generator theorem, wiki one-way
  functions, pseudorandom generator theorem, PRG$\rightarrow$OWF.\\

  So why doesn't randomness imply P = NP? Most computer scientists believe
  randomness does not offer enough to solve exponential problems
  efficiently. For this statement, refer to the second last paragraph of
  section 7.2 of the paper titled ``The status of the P versus NP
  problem''.\\

  The paper also offers a tidbit of information that is somewhat relevant
  to our discussion: the factorization problem is likely to be easier than
  NP-complete.
\end{proof}
