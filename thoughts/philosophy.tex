\begin{question}
  What kind of ``computers'' are human brains? Can we ever figure out the
  axioms of the universe? Or can we prove that we can never figure out the
  axioms? Can we formulate these questions rigorously?  Can our brain be
  modeled as a Turing machine?  Can Turing machines represent the
  ``growth'' of our brain as having infinite memory?
\end{question}
\begin{proof}
\end{proof}

\begin{question}
  Does Godel's second incompleteness theorem imply that human brains are
  not Turing machines?
\end{question}
\begin{proof}
  Refer to Lucas-Penrose argument.
\end{proof}

\begin{question}
  Intuitionalist versus classical mathematics: can we find an example of a
  mathematical object that exists in the classical sense but not in the
  intuitive sense? In other words, an object that we can prove to exist but
  yet cannot construct? 
\end{question}
\begin{proof}
\end{proof}
