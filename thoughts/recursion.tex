\begin{theorem}
  Make sense of fixed-point, and Arslanov's completeness criteria.
\end{theorem}
\begin{proof}
  How do we interpret the proof of the fixed-point (recursion) theorem?
  The theorem says given a recursive $f(e)$, there exists a function
  indexed $n\in\omega$ such that $\varphi_n=\varphi_{f(n)}$.\\

  The fixed-point theorem was proved because people failed to
  diagonalize out of the partial-recursives. So assume by contradiction
  we have a recursive $f$ with no fixed-point. We use shall $f$ to
  construct a partial-recursive $g$ which diagonalizes out of all the
  partial-recursives, i.e. $g(e)\neq\varphi_e(e)$ for all
  $e\in\omega$.\\

  Now $f$ has no fixed-point, so $\varphi_x\neq\varphi_{f(x)}$ for all
  $x\in\omega$. In particular, $\varphi_{\varphi_e(e)} \neq
  \varphi_{f(\varphi_e(e))}$ for all $e\in\omega$. Thus set $g(e)$ to
  be an index of $\varphi_{f(\varphi_e(e))}$; this index must be
  different from $\varphi_e(e)$. Also SMN-theorem gives us an effective
  way of computing $g(e)$ since $f$ is partial-recursive. Thus $g$
  diagonalizes out of the partial-recursives, yet it is
  partial-recursive, a contradiction.\\

  The second way of interpreting the proof of the fixed-point theorem
  is as follows. The fixed-point $n$ that we find shall say
  \begin{center}
    \textit{``I am the $f$ of myself.''}
  \end{center}

  This makes sense since we want $\varphi_n$ to equal in some sense to
  ``the $f$ of itself'' $\varphi_{f(n)}$. A function indexed by $e$ may
  refer to itself using the function indexed $\varphi_e(e)$. So we are
  looking for a function indexed $n=\varphi_v(v)$ for some $v$ such
  that
  $\varphi_v(v)=f(\varphi_v(v))$, i.e.
  \[\varphi_n :=\varphi_{\varphi_v(v)} =\varphi_{f(\varphi_v(v))}.\]

  So we consider the function indexed $v$ given by
  \[\varphi_v(e) :=f(\varphi_e(e)),\]
  and let $n=\varphi_v(v)$. Then we have
  \[\varphi_{f(n)} =\varphi_{f(\varphi_v(v))} =\varphi_{\varphi_v(v)}
  =\varphi_n\]
  as desired. Thus completes the general idea of the proof of the
  recursion theorem. \\

  Using the recursion theorem, we can define functions `recursively' in
  the sense that we can define a desired $g(x)$ in terms of itself. Say
  we want $g(x)$ defined recursively as
  \[g(x)=h(g(x),x)\]
  for some recursive $h(x,y)$. To prove that such $g(x)$ exists,
  define recursive $g'(e)$ such that
  \[\varphi_{g'(e)}(x) :=h(\varphi_e(x),x).\]

  Apply recursion theorem to get $n\in\omega$ such that
  \[\varphi_n(x) =\varphi_{g'(n)}(x) =h(\varphi_n(x),x).\]
  Such $\varphi_n(x)$ is the $g(x)$ we are looking for.
\end{proof}

\begin{theorem}
  (Lachlan) If the halting set $H$ is many-one-reducible to $A\times B$,
  where $B$ is a c.e. set, then it is also many-one-reducible to either $A$
  or to $B$.
\end{theorem}
\begin{proof}
  Let $\alpha_e:\omega \rightarrow\omega$ and $\beta_e:\omega
  \rightarrow\omega$ be recursive functions witnessing $W_e\leq_m H\leq_m
  A\times B$, i.e.
  \[x\in W_e \Leftrightarrow \alpha_e(x)\in A \wedge \beta_e(x)\in B.\]

  Define $G_e:=\beta_e^{-1}(B)$. Now $G_e$ is c.e. since $B$ is, so let
  $x_0,x_1,\ldots$ be an enumeration of its elements. Then let
  $D_e\subseteq G_e$ be the ``encoding'' of $H$ in $G_e$, in the sense that
  $D_e$ contains exactly the $H$-th elements of $G_e$, i.e.
  \[D_e :=\{x_h\in G_e: h\in H\}.\]

  Now the indices for $\alpha_e$, $\beta_e$, $G_e$, and $D_e$ can be
  obtained uniformly in $e$ since $H\leq_m A\times B$. Thus by the
  fixed-point lemma, $D_e=W_e$ for some $e\in\omega$. Then the recursive
  function sending $h$ to $\alpha_e(x_h)$ witnesses $H\leq_m A$. But this
  strategy of constructing $D_e$ as the set of $H$-th elements of $G_e$
  will not work if $G_e$ is finite, because we will not be able to pick out
  its $H$-th elements.\\

  To prepare for such a situation, we simultaneously play a second strategy
  that will gives us $H\leq_m B$: At stages when $H$ enumerates a new
  element $h$ but $G_e$ doesn't yet have a $h$-th element to put into
  $D_e$, we put instead $h$ into $D_e$. This way, should $G_e$ turn out to
  be finite, then $W_e=D_e=^*H$, where the first equality follows from the
  fixed-point lemma. Then since $G_e:=\beta_e^{-1}(B)$ is finite,
  $G_e\cap \alpha_e^{-1}(\bar{A})$ is also finite. So modulo a finite
  number of elements, $\beta_e$ will witness $H=^*W_e\leq_m B$.\\

  To ensure that the second strategy of putting $h$ into $D_e$ will not
  mess up the first strategy of allowing the $h$-th element of $G_e$ into
  $D_e$ only if $h\in H$, we make sure that when we are constructing $G_e$,
  the position of its $h$-th element is reserved for $h$ only, i.e. want
  $x_h=h$. So eventually when $G_e$ has grown up to containing $x_{h-1}$,
  we do not put new elements into $G_e$ unless the new element is $h$.\\

  We construct $G_e$ and $D_e$ with the above described properties. Start
  with $G_{e,0}=D_{e,0}=\emptyset$, and fix an enumeration $h_0,h_1,\ldots$
  of $H$. At stage $s+1$, we want to put the $h_s$-th element of $G_e$ into
  $D_e$. So writing $G_{e,s}=x_0,\ldots,x_n$, if $|G_{e,s}|>h_s$, we play
  the first stragety of setting $D_{e,s+1} =D_{e,s} \cup \{x_{h_s}\}$.
  However if $|G_{e,s}|\leq h_s$, we play the second strategy and set
  $D_{e,s+1} =D_{e,s} \cup \{h_s\}$.\\

  Now we consider how to 
\end{proof}
