%\begin{theorem}
%  Make sense of fixed-point, and Arslanov's completeness criteria.
%\end{theorem}
%\begin{proof}
%  How do we interpret the proof of the fixed-point (recursion) theorem?
%  The theorem says given a recursive $f(e)$, there exists a function
%  indexed $n\in\omega$ such that $\varphi_n=\varphi_{f(n)}$.\\
%
%  The fixed-point theorem was proved because people failed to
%  diagonalize out of the partial-recursives. So assume by contradiction
%  we have a recursive $f$ with no fixed-point. We use shall $f$ to
%  construct a partial-recursive $g$ which diagonalizes out of all the
%  partial-recursives, i.e. $g(e)\neq\varphi_e(e)$ for all
%  $e\in\omega$.\\
%
%  Now $f$ has no fixed-point, so $\varphi_x\neq\varphi_{f(x)}$ for all
%  $x\in\omega$. In particular, $\varphi_{\varphi_e(e)} \neq
%  \varphi_{f(\varphi_e(e))}$ for all $e\in\omega$. Thus set $g(e)$ to
%  be an index of $\varphi_{f(\varphi_e(e))}$; this index must be
%  different from $\varphi_e(e)$. Also SMN-theorem gives us an effective
%  way of computing $g(e)$ since $f$ is partial-recursive. Thus $g$
%  diagonalizes out of the partial-recursives, yet it is
%  partial-recursive, a contradiction.\\
%
%  The second way of interpreting the proof of the fixed-point theorem
%  is as follows. The fixed-point $n$ that we find shall say
%  \begin{center}
%    \textit{``I am the $f$ of myself.''}
%  \end{center}
%
%  This makes sense since we want $\varphi_n$ to equal in some sense to
%  ``the $f$ of itself'' $\varphi_{f(n)}$. A function indexed by $e$ may
%  refer to itself using the function indexed $\varphi_e(e)$. So we are
%  looking for a function indexed $n=\varphi_v(v)$ for some $v$ such
%  that
%  $\varphi_v(v)=f(\varphi_v(v))$, i.e.
%  \[\varphi_n :=\varphi_{\varphi_v(v)} =\varphi_{f(\varphi_v(v))}.\]
%
%  So we consider the function indexed $v$ given by
%  \[\varphi_v(e) :=f(\varphi_e(e)),\]
%  and let $n=\varphi_v(v)$. Then we have
%  \[\varphi_{f(n)} =\varphi_{f(\varphi_v(v))} =\varphi_{\varphi_v(v)}
%  =\varphi_n\]
%  as desired. Thus completes the general idea of the proof of the
%  recursion theorem. \\
%
%  Using the recursion theorem, we can define functions `recursively' in
%  the sense that we can define a desired $g(x)$ in terms of itself. Say
%  we want $g(x)$ defined recursively as
%  \[g(x)=h(g(x),x)\]
%  for some recursive $h(x,y)$. To prove that such $g(x)$ exists,
%  define recursive $g'(e)$ such that
%  \[\varphi_{g'(e)}(x) :=h(\varphi_e(x),x).\]
%
%  Apply recursion theorem to get $n\in\omega$ such that
%  \[\varphi_n(x) =\varphi_{g'(n)}(x) =h(\varphi_n(x),x).\]
%  Such $\varphi_n(x)$ is the $g(x)$ we are looking for.
%\end{proof}

\begin{theorem}
  (Lachlan) If the halting set $H:=\{\langle e,x\rangle: x\in W_e\}$ is
  many-one-reducible to $A\times B$, where $B$ is a c.e. set, then it is
  also many-one-reducible to either $A$ or to $B$.
\end{theorem}
\begin{proof}
  For each $e\in\omega$, let $\alpha_e:\omega \rightarrow\omega$ and
  $\beta_e:\omega \rightarrow\omega$ be recursive functions witnessing
  $W_e\leq_m H\leq_m A\times B$, i.e.
  \[x\in W_e \Leftrightarrow \alpha_e(x)\in A \wedge \beta_e(x)\in B.\]

  Define $G_e:=\beta_e^{-1}(B)$. Now $G_e$ is c.e. since $B$ is, so let
  $x_0,x_1,\ldots$ be an enumeration of its elements. Then let
  $D_e\subseteq G_e$ be the ``encoding'' of $H$ in $G_e$, in the sense that
  $D_e$ contains exactly the $H$-th elements of $G_e$, i.e.
  \[D_e :=\{x_h\in G_e: h\in H\}.\]

  Now the indices for $\alpha_e$, $\beta_e$, $G_e$, and $D_e$ can be
  obtained uniformly in $e$ since $H\leq_m A\times B$. Thus by the
  fixed-point lemma, $D_e=W_e$ for some $e\in\omega$. Assuming $e$ is one
  such fixed-point, the recursive function sending $h$ to $\alpha_e(x_h)$
  witnesses $H\leq_m A$. But this strategy of constructing $D_e$ as the set
  of $H$-th elements of $G_e$ will not work if $G_e$ is finite, because we
  will not be able to pick out its $H$-th elements.\\

  To prepare for such a situation, we simultaneously play a second strategy
  that will gives us $H\leq_m B$: At stages when $H$ enumerates a new
  element $h$ but $G_e$ doesn't have a $h$-th element to be put into
  $D_e$, we put instead $h$ into $D_e$. This way, should $G_e$ turn out to
  be finite, then $W_e=D_e=^*H$, where the first equality follows from the
  fixed-point lemma. Then since $G_e:=\beta_e^{-1}(B)$ is finite,
  $G_e\cap \alpha_e^{-1}(\bar{A})$ is also finite. So modulo a finite
  number of elements, $\beta_e$ will witness $H=^*W_e\leq_m B$.\\

  To ensure that the second strategy of putting $h$ into $D_e$ will not
  destroy the first strategy which requires $x_h\in D_e \Leftrightarrow
  h\in H$, when we construct $G_e$ and some $h\in H$ entered $D_e$ from
  the second strategy, we set $x_h=h$: When $G_e$ has grown up to
  containing $x_{h-1}$, we do not put new elements into $G_e$ unless it is
  $h$. In addition, we also need to check that for each $h\in H$, if $x_h$
  already exists in $G_e$, we put it into $D_e$. Note that we will since
  $D_e=W_e\subseteq\beta_e^{-1}(B)$ and $h\in D_e$, eventually $h$ will be
  enumerated into $G_e=\beta_e^{-1}(B)$, so we will not need to wait
  forever for $x_h=h$. \\

  Thus in the construction, apart from keeping track of $G_e$ and $D_e$, we
  also maintain a list $I$ containing the $h$-values that entered $D_e$ by
  the second strategy, and for which we must ensure $h=x_h$. At stage $s$,
  let $h\in H$ be the new element enumerated into $H$. If $G_e$ already
  contains a $h$-th element, enumerate it into $D_e$ via the first
  strategy. On the other hand, if $G_e$ has less than $h$-elements,
  enumerate $h$ into $D_e$ via the second strategy, if it is not already in
  $D_e$. If the second strategy was played, add $h$ to $I$.\\

  Now we define the new element of $G_e$ in a way that is consistent with
  the first strategy. Assume there are $(n-1)$ elements in $G_e$ so far,
  and we want to define the $n$-th element if possible. Let $C$ be the set
  of elements that has been enumerated into $\beta_e^{-1}(B)$ so far and
  that have not yet entered $G_e$. If $n\in I$ and $n\not\in C$, do nothing
  and go to the next stage. If $n\in I$ and $n\in C$, enumerate $n$ into
  $G_e$, remove $n$ from $I$, then go to next stage. If $n\not\in I$ and
  $C-I=\emptyset$, do nothing and go to next stage. Finally, if $n\not\in
  I$ and $C-I$ is non-empty, enumerate its earlist element into $G_e$, then
  also into $D_e$ if $n\in H_s$. This completes the construction.
\end{proof}
