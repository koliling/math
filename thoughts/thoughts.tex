\documentclass{article}
\usepackage[left=3cm,right=3cm,top=3cm,bottom=3cm]{geometry}
\usepackage{amsmath,amssymb,amsthm,tikz,mathtools}
\usepackage{stmaryrd} % For double square bracket [[]]
\usepackage{bm} % For bold vectors
\usepackage{color}
\usepackage{cancel} % To use the cancel function
\usepackage[inline]{enumitem}
\usepackage{hyperref} % To use hyperlinks in URLs
\usetikzlibrary{patterns}
\setlength{\parindent}{0mm}
\newcommand{\OPEN}[1]{\textcolor{red}{OPEN: #1}}

\newtheorem{theorem}{Theorem}[subsection]
\newtheorem{question}{Question}[subsection]

\begin{document}
\title{Logic Thoughts}
\author{Li Ling Ko\\ lko@nd.edu}
\date{\today}
\maketitle

\section{Arithmetical Hierarchy}
  \subsection{Computably Enumerable Degrees}
    \begin{theorem}
  Is there a characterization of $m$-reducibility? $1$-reducibility has
  been characterized by
  \[A=W^B \Leftrightarrow A\leq_1 B',\]
  and by
  \[A\leq_T B \Leftrightarrow A'\leq_1 B'.\]
  Is there a similar characterization for $m$-reducibility?
\end{theorem}
\begin{proof}
\end{proof}

\begin{theorem}
  What is the history behind simple sets?
\end{theorem}
\begin{proof}
  Perhaps simple sets came about when people tried to construct
  r.e. sets that are non-computable. 
\end{proof}

\begin{question}
  Must the cap of every pair of r.e. degrees, if it exist, also be an
  r.e. degree?
\end{question}
\begin{proof}
\end{proof}

\begin{theorem}
  There are non c.e. degrees below $\emptyset'$.
\end{theorem}
\begin{proof}
  Cooper Epstein-Lachlan theorem. Also, the 1-generics, from new Soare
  E6.3.5iii, and the fact that there exists 1-generics below
  $\emptyset'$. Finally, the 1-randoms too. By the Low Basis theorem,
  applied to the universal Martin-Lof tree, there must be a 1-random
  that is low. And 1-randoms cannot be c.e.
\end{proof}

\begin{theorem}
  There are no minimal c.e. degrees.
\end{theorem}
\begin{proof}
  Sack's Density theorem.
\end{proof}

\begin{question}
  Do all non-minimal degrees have infinite incomparable degrees below
  them? How about just finite?
\end{question}
\begin{proof}
\end{proof}


  \subsection{Lattice $\mathcal{D}$}
    \begin{question}
  Where are the PA reals? Do the class of PA reals have measure 1?
\end{question}
\begin{proof}
  They are probably everywhere. Intuitively you can $\emptyset'$-encode
  any infinite set into some PA real, so it seems likely that the class
  of PA reals have measure 1. Given arbitrary real $A\in2^\omega$, the
  $\emptyset'$-encoding of $A$ into a PA real $P\in2^\omega$ can be
  performed as follows. Since PA reals are incomplete, every such real
  has infinite initial segments for which can branch both ways and
  still get PA reals. At stage $s$, let $\emptyset'$ decide when the
  first extension which branches both ways occur. Take the left branch
  iff $A(s)=0$. \\

  By construction, $A$ is $(\emptyset'\oplus P)$-recursive. In
  particular, the PA reals cannot be covered by any real $X$, for this
  would imply that all reals are covered by $X\oplus\emptyset'$.\\

  Furthermore, from new Soare Theorem 10.3.3.iii, the PA degrees are
  exactly those that compute a 2-valued d.n.c. function. Since
  $\emptyset'$ computes a d.n.c. function, every degree above
  $\emptyset'$ contains a PA real. So PA reals are everywhere. Though
  it is not clear the class of them must have measure 1.
\end{proof}

\begin{question}
  If $\emptyset' \leq_T A\leq_T \emptyset'\oplus B$, does this imply
  that $B\geq_T\emptyset'$?
\end{question}
\begin{proof}
\end{proof}

\begin{question}
  What is the motivation for defining simple sets? How about
  1-generics, or maximal sets?
\end{question}
\begin{proof}
  Simple sets were defined to answer Post's question, which asks if
  there exists a non-computable set below the jump. Perhaps 1-generics
  were defined to construct sets through forcing the jump.
\end{proof}

\begin{theorem}
  There exists a degree incomparable with $\emptyset'$.
\end{theorem}
\begin{proof}
  From the paragraph after Soare's D5.6.1, the hyperimmune-free degrees
  are exactly those that are incomparable with $\emptyset'$. Also
  recall that a set is hyperimmune if and only if it is not computably
  dominated. The construction of Soare T9.5.1 gives a noncomputable set
  with a hyperimmune-free degree. \\

  Alternatively, from new Soare E6.3.5iii, 1-generics cannot compute
  $\emptyset'$. Then from E6.3.4, a nonlow 1-generic cannot be computed
  by $\emptyset'$. Therefore nonlow 1-generics are imcomparable with
  $\emptyset'$.
\end{proof}

\begin{question}
  Given $A\leq_T B$, must there be a degree between them? I.e. Does
  Sack's density theorem be generalized to degrees that are not
  necessarily c.e.? How about infinite pairwise incomparable degrees
  between $A$ and $B$?
\end{question}
\begin{proof}
  Consider local and global degrees?
\end{proof}

\begin{question}
  Given any non-recursive non-minimal degree, are there always infinite
  pairwise incomparable degrees beneath it?
\end{question}
\begin{proof}
\end{proof}

\begin{question}
  Is $\mathcal{D}$ a lattice? In particular, do any two Turing degrees
  have a greatest lower bound?
\end{question}
\begin{proof}
  No. Clearly the join of two given degrees give their lub. But from
  Klenne-Post-Spector (old Soare VI.C4.4), there are degrees with no
  greatest lower bound.
\end{proof}

\begin{question}
  What do we know about the minimal degree below $\emptyset'$
  constructed in Soare 13.1.1? In particular, is it low? Does it have a
  c.e. degree?
\end{question}
\begin{proof}
  From Soare 13.7.1 every non-computable c.e. degree computes a minimal
  degree. In particular, minimal degrees can be low.
\end{proof}

\begin{question}
  Does $\mathcal{D}_{\leq\emptyset'}$ have a maximal degree?
\end{question}
\begin{proof}
  \OPEN{This is asking if $\emptyset'$ is a strong minimal cover of any
  degree below it. We know there are degrees with strong minimal
  covers, but we do not know if $\emptyset'$ is such a cover. Look up
  Andrew Lewis \url{http://www.aemlewis.co.uk/pdf/SMCtssf.pdf} or
  \url{https://projecteuclid.org/euclid.jsl/1183746187}.}
\end{proof}


  \subsection{Randoms}
    \begin{question}
  Is there a 1-random in every non-recursive degree? Is there a result
  similar to one for hyperimmune degrees, which says that the
  hyperimmune degrees are exactly those that are comparable with
  $\emptyset'$?
\end{question}
\begin{proof}
\end{proof}

\begin{question}
  Is randomness independent from ZFC?
\end{question}
\begin{proof}
  Perhaps this depends on our definition of randomness. Kolmogorov defined
  randomness as a series that cannot be generated by any Turing machine
  shorter than the series (Wiki Randomness, mathematics). By this
  definition, it appears randomness exists if certain reals exist, which
  may be relevant to the continuum hypothesis.
\end{proof}

\begin{question}
  Can Turing machines prove their lack of randomness?
\end{question}
\begin{proof}
  In the same vein, if the universe were deterministic, would we ever know
  it? Similarly, if the universe were random, would we eventually find out?
\end{proof}


  \subsection{Generics}
    \begin{question}
  Is there any relation between 1-random sets and 1-generic sets? I
  suspect all 1-generics are 1-random, but the converse isn't true? Is
  there a result similar to one for hyperimmune degrees, which says
  that the hyperimmune degrees are exactly those that are comparable
  with $\emptyset'$?
\end{question}
\begin{proof}
\end{proof}


\section{Reverse Math}
  \subsection{Rich Sets}
    \begin{question}
  By Simpson all hyperarithmetical sets are rich, and by Soare a
  poor set exists. Are there non-hyperarithmetical sets that are also
  rich?
\end{question}
\begin{proof}
  Yes. Consider the join $H\oplus P$ of a hyperarithmetical set $H$
  with a poor set $P$. Given arbitrary $X$ that computes $H\oplus P$,
  since $H$ is rich, it contains some subset $H_0\subset H$ that is
  Turing equivalent to $X$. Then $H_0\oplus\emptyset$ witnesses the
  richness of $H\oplus P$.
\end{proof}


  \subsection{Galvin-Prikry}
    \begin{question}
  Is Borel necessary for Galvin-Prikry to hold? In other words, is
  there a weaker condition that will ensure we observe the structure of
  Galvin-Prikry?
\end{question}
\begin{proof}
  Probably not. Silver's result improved Galvin-Prikry but only in a
  special case. There is a book with these results.
\end{proof}


\section{Set Theory}
  \subsection{Godel's Second Incompleteness}
    Godel's second incompleteness theorem appears to be a by-product of
formalizing derivability. Given sentence $\sigma$, let Dr($\sigma$)
abbreviate the sentence ``$\sigma$ is derivable''.  To formalize
derivability, some natural questions to ask are:
\begin{itemize}
  \item If PA derives $\sigma$, does it derive Dr($\sigma$)?
  \item If PA derives Dr($\sigma$), does it derive Dr(Dr($\sigma$))?
  \item If PA derives Dr($\sigma_0\rightarrow\sigma_1$) and
    Dr($\sigma_0$), does it derive Dr($\sigma_1$)?
\end{itemize}

The answer to these three questions turn out to be yes. The three rules
are called derivability conditions, and are used to formalize
derivability.\\

Godel's second incompleteness theorem is proved by applying the
derivability conditions on the sentence $\theta$ which says ``I am not
derivable''. In technical terms, PA derives the sentence
``$\theta\leftrightarrow\neg\text{Dr}(\theta)$''.\\

Let Con(PA) abbreviate the sentence that says PA is consistent. More
precisely, Con(PA) is defined as $\neg$Dr(0=1).\\

Our final goal is to show that PA cannot derive Con(PA). We prove this
by applying the derivability conditions on $\theta$.\\

First, by applying the first derivability condition, we get $\theta$ is
inconsistent: $\theta$ implies $\neg\text{Dr}(\theta)$ from choice of
$\theta$, yet also implies $\text{Dr}(\theta)$ from the first
condition.\\

Hence PA derives ``$\theta\rightarrow0=1$'', and by applying the third
derivability condition on this sentence, we get PA derives
``$\text{Dr}(\theta)\rightarrow\text{Dr}(0=1)$'',
which is equivalent to ``Con(PA)$\rightarrow\neg\text{Dr}(\theta)$''.
Since $\neg\text{Dr}(\theta)$ is the same as $\theta$ from our choice
of $\theta$, and $\theta$ is inconsistent from earlier argument, we
finally get ``Con(PA)$\rightarrow0=1$'', as desired.\\

We can interpret the above proof of the second incompleteness theorem as a
formalization of the following argument: Recall that the first
incompleteness theorem says that if PA is consistent then it cannot prove
$\theta$. But not proving $\theta$ is the same as proving $\theta$, from
the definition of $\theta$. Hence if we show that PA is consistent then we
get a contradiction. 


\section{Philosophy}
  \subsection{Computability of Human Mind}
    \begin{question}
  What kind of ``computers'' are human brains? Can we ever figure out
  the axioms of the universe? Or can we prove that we can never figure
  out the axioms? Can we formulate these questions rigorously?
\end{question}
\begin{proof}
\end{proof}

\end{document}
