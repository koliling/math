\documentclass{article}
\usepackage[left=3cm,right=3cm,top=3cm,bottom=3cm]{geometry}
\usepackage{amsmath,amssymb,amsthm,tikz,mathtools}
\usepackage{stmaryrd} % For double square bracket [[]]
\usepackage{bm} % For bold vectors
\usepackage{color}
\usepackage{cancel} % To use the cancel function
\usepackage[inline]{enumitem}
\usepackage{hyperref} % To use hyperlinks in URLs
\usetikzlibrary{patterns}
\setlength{\parindent}{0mm}
\newcommand{\OPEN}[1]{\textcolor{red}{OPEN: #1}}

\newtheorem{theorem}{Theorem}[subsection]
\newtheorem{question}{Question}[subsection]

\begin{document}
\title{Logic Thoughts}
\author{Li Ling Ko\\ lko@nd.edu}
\date{\today}
\maketitle

\section{Arithmetical Hierarchy}
%  \subsection{Computably Enumerable Degrees}
%    \begin{theorem}
  Is there a characterization of $m$-reducibility? $1$-reducibility has
  been characterized by
  \[A=W^B \Leftrightarrow A\leq_1 B',\]
  and by
  \[A\leq_T B \Leftrightarrow A'\leq_1 B'.\]
  Is there a similar characterization for $m$-reducibility?
\end{theorem}
\begin{proof}
\end{proof}

\begin{theorem}
  What is the history behind simple sets?
\end{theorem}
\begin{proof}
  Perhaps simple sets came about when people tried to construct
  r.e. sets that are non-computable. 
\end{proof}

\begin{question}
  Must the cap of every pair of r.e. degrees, if it exist, also be an
  r.e. degree?
\end{question}
\begin{proof}
\end{proof}

\begin{theorem}
  There are non c.e. degrees below $\emptyset'$.
\end{theorem}
\begin{proof}
  Cooper Epstein-Lachlan theorem. Also, the 1-generics, from new Soare
  E6.3.5iii, and the fact that there exists 1-generics below
  $\emptyset'$. Finally, the 1-randoms too. By the Low Basis theorem,
  applied to the universal Martin-Lof tree, there must be a 1-random
  that is low. And 1-randoms cannot be c.e.
\end{proof}

\begin{theorem}
  There are no minimal c.e. degrees.
\end{theorem}
\begin{proof}
  Sack's Density theorem.
\end{proof}

\begin{question}
  Do all non-minimal degrees have infinite incomparable degrees below
  them? How about just finite?
\end{question}
\begin{proof}
\end{proof}

  \subsection{Recursion Theory}
    %\begin{theorem}
%  Make sense of fixed-point, and Arslanov's completeness criteria.
%\end{theorem}
%\begin{proof}
%  How do we interpret the proof of the fixed-point (recursion) theorem?
%  The theorem says given a recursive $f(e)$, there exists a function
%  indexed $n\in\omega$ such that $\varphi_n=\varphi_{f(n)}$.\\
%
%  The fixed-point theorem was proved because people failed to
%  diagonalize out of the partial-recursives. So assume by contradiction
%  we have a recursive $f$ with no fixed-point. We use shall $f$ to
%  construct a partial-recursive $g$ which diagonalizes out of all the
%  partial-recursives, i.e. $g(e)\neq\varphi_e(e)$ for all
%  $e\in\omega$.\\
%
%  Now $f$ has no fixed-point, so $\varphi_x\neq\varphi_{f(x)}$ for all
%  $x\in\omega$. In particular, $\varphi_{\varphi_e(e)} \neq
%  \varphi_{f(\varphi_e(e))}$ for all $e\in\omega$. Thus set $g(e)$ to
%  be an index of $\varphi_{f(\varphi_e(e))}$; this index must be
%  different from $\varphi_e(e)$. Also SMN-theorem gives us an effective
%  way of computing $g(e)$ since $f$ is partial-recursive. Thus $g$
%  diagonalizes out of the partial-recursives, yet it is
%  partial-recursive, a contradiction.\\
%
%  The second way of interpreting the proof of the fixed-point theorem
%  is as follows. The fixed-point $n$ that we find shall say
%  \begin{center}
%    \textit{``I am the $f$ of myself.''}
%  \end{center}
%
%  This makes sense since we want $\varphi_n$ to equal in some sense to
%  ``the $f$ of itself'' $\varphi_{f(n)}$. A function indexed by $e$ may
%  refer to itself using the function indexed $\varphi_e(e)$. So we are
%  looking for a function indexed $n=\varphi_v(v)$ for some $v$ such
%  that
%  $\varphi_v(v)=f(\varphi_v(v))$, i.e.
%  \[\varphi_n :=\varphi_{\varphi_v(v)} =\varphi_{f(\varphi_v(v))}.\]
%
%  So we consider the function indexed $v$ given by
%  \[\varphi_v(e) :=f(\varphi_e(e)),\]
%  and let $n=\varphi_v(v)$. Then we have
%  \[\varphi_{f(n)} =\varphi_{f(\varphi_v(v))} =\varphi_{\varphi_v(v)}
%  =\varphi_n\]
%  as desired. Thus completes the general idea of the proof of the
%  recursion theorem. \\
%
%  Using the recursion theorem, we can define functions `recursively' in
%  the sense that we can define a desired $g(x)$ in terms of itself. Say
%  we want $g(x)$ defined recursively as
%  \[g(x)=h(g(x),x)\]
%  for some recursive $h(x,y)$. To prove that such $g(x)$ exists,
%  define recursive $g'(e)$ such that
%  \[\varphi_{g'(e)}(x) :=h(\varphi_e(x),x).\]
%
%  Apply recursion theorem to get $n\in\omega$ such that
%  \[\varphi_n(x) =\varphi_{g'(n)}(x) =h(\varphi_n(x),x).\]
%  Such $\varphi_n(x)$ is the $g(x)$ we are looking for.
%\end{proof}

\begin{theorem}
  (Lachlan) If the halting set $H:=\{\langle e,x\rangle: x\in W_e\}$ is
  many-one-reducible to $A\times B$, where $B$ is a c.e. set, then it is
  also many-one-reducible to either $A$ or to $B$.
\end{theorem}
\begin{proof}
  For each $e\in\omega$, let $\alpha_e:\omega \rightarrow\omega$ and
  $\beta_e:\omega \rightarrow\omega$ be recursive functions witnessing
  $W_e\leq_m H\leq_m A\times B$, i.e.
  \[x\in W_e \Leftrightarrow \alpha_e(x)\in A \wedge \beta_e(x)\in B.\]

  Define $G_e:=\beta_e^{-1}(B)$. Now $G_e$ is c.e. since $B$ is, so let
  $x_0,x_1,\ldots$ be an enumeration of its elements. Then let
  $D_e\subseteq G_e$ be the ``encoding'' of $H$ in $G_e$, in the sense that
  $D_e$ contains exactly the $H$-th elements of $G_e$, i.e.
  \[D_e :=\{x_h\in G_e: h\in H\}.\]

  Now the indices for $\alpha_e$, $\beta_e$, $G_e$, and $D_e$ can be
  obtained uniformly in $e$ since $H\leq_m A\times B$. Thus by the
  fixed-point lemma, $D_e=W_e$ for some $e\in\omega$. Then the recursive
  function sending $h$ to $\alpha_e(x_h)$ witnesses $H\leq_m A$. But this
  strategy of constructing $D_e$ as the set of $H$-th elements of $G_e$
  will not work if $G_e$ is finite, because we will not be able to pick out
  its $H$-th elements.\\

  To prepare for such a situation, we simultaneously play a second strategy
  that will gives us $H\leq_m B$: At stages when $H$ enumerates a new
  element $h$ but $G_e$ doesn't have a $h$-th element to be put into
  $D_e$, we put instead $h$ into $D_e$. This way, should $G_e$ turn out to
  be finite, then $W_e=D_e=^*H$, where the first equality follows from the
  fixed-point lemma. Then since $G_e:=\beta_e^{-1}(B)$ is finite,
  $G_e\cap \alpha_e^{-1}(\bar{A})$ is also finite. So modulo a finite
  number of elements, $\beta_e$ will witness $H=^*W_e\leq_m B$.\\

  To ensure that the second strategy of putting $h$ into $D_e$ will not
  destroy the first strategy which requires $x_h\in D_e \Leftrightarrow
  h\in H$, when we construct $G_e$ and some $h\in H$ entered $D_e$ from
  the second strategy, we set $x_h=h$: When $G_e$ has grown up to
  containing $x_{h-1}$, we do not put new elements into $G_e$ unless it is
  $h$. In addition, we also need to check that for each $h\in H$, if $x_h$
  already exists in $G_e$, we put it into $D_e$.\\

  Thus in the construction, apart from keeping track of $G_e$ and $D_e$, we
  also maintain a list $I$ containing the $h$-values that entered $D_e$ by
  the second strategy, and for which we must ensure $h=x_h$. At stage $s$,
  let $h\in H$ be the new element enumerated into $H$. If $G_e$ already
  contains a $h$-th element, enumerate it into $D_e$ via the first
  strategy. On the other hand, if $G_e$ has less than $h$-elements,
  enumerate $h$ into $D_e$ via the second strategy, if it is not already in
  $D_e$. If the second strategy was played, add $h$ to $I$.\\

  Now we define the new element of $G_e$ in a way that is consistent with
  the first strategy. Assume there are $(n-1)$ elements in $G_e$ so far,
  and we want to define the $n$-th element if possible. Let $C$ be the set
  of elements that has been enumerated into $\beta_e^{-1}(B)$ so far and
  that have not yet entered $G_e$. If $n\in I$ and $n\not\in C$, do nothing
  and go to the next stage. If $n\in I$ and $n\in C$, enumerate $n$ into
  $G_e$, remove $n$ from $I$, then go to next stage. If $n\not\in I$ and
  $C-I=\emptyset$, do nothing and go to next stage. Finally, if $n\not\in
  I$ and $C-I$ is non-empty, enumerate its earlist element into $G_e$, then
  also into $D_e$ if $n\in H_s$. This completes the construction.
\end{proof}

%  \subsection{Lattice $\mathcal{D}$}
%    \begin{question}
  Where are the PA reals? Do the class of PA reals have measure 1?
\end{question}
\begin{proof}
  They are probably everywhere. Intuitively you can $\emptyset'$-encode
  any infinite set into some PA real, so it seems likely that the class
  of PA reals have measure 1. Given arbitrary real $A\in2^\omega$, the
  $\emptyset'$-encoding of $A$ into a PA real $P\in2^\omega$ can be
  performed as follows. Since PA reals are incomplete, every such real
  has infinite initial segments for which can branch both ways and
  still get PA reals. At stage $s$, let $\emptyset'$ decide when the
  first extension which branches both ways occur. Take the left branch
  iff $A(s)=0$. \\

  By construction, $A$ is $(\emptyset'\oplus P)$-recursive. In
  particular, the PA reals cannot be covered by any real $X$, for this
  would imply that all reals are covered by $X\oplus\emptyset'$.\\

  Furthermore, from new Soare Theorem 10.3.3.iii, the PA degrees are
  exactly those that compute a 2-valued d.n.c. function. Since
  $\emptyset'$ computes a d.n.c. function, every degree above
  $\emptyset'$ contains a PA real. So PA reals are everywhere. Though
  it is not clear the class of them must have measure 1.
\end{proof}

\begin{question}
  If $\emptyset' \leq_T A\leq_T \emptyset'\oplus B$, does this imply
  that $B\geq_T\emptyset'$?
\end{question}
\begin{proof}
\end{proof}

\begin{question}
  What is the motivation for defining simple sets? How about
  1-generics, or maximal sets?
\end{question}
\begin{proof}
  Simple sets were defined to answer Post's question, which asks if
  there exists a non-computable set below the jump. Perhaps 1-generics
  were defined to construct sets through forcing the jump.
\end{proof}

\begin{theorem}
  There exists a degree incomparable with $\emptyset'$.
\end{theorem}
\begin{proof}
  From the paragraph after Soare's D5.6.1, the hyperimmune-free degrees
  are exactly those that are incomparable with $\emptyset'$. Also
  recall that a set is hyperimmune if and only if it is not computably
  dominated. The construction of Soare T9.5.1 gives a noncomputable set
  with a hyperimmune-free degree. \\

  Alternatively, from new Soare E6.3.5iii, 1-generics cannot compute
  $\emptyset'$. Then from E6.3.4, a nonlow 1-generic cannot be computed
  by $\emptyset'$. Therefore nonlow 1-generics are imcomparable with
  $\emptyset'$.
\end{proof}

\begin{question}
  Given $A\leq_T B$, must there be a degree between them? I.e. Does
  Sack's density theorem be generalized to degrees that are not
  necessarily c.e.? How about infinite pairwise incomparable degrees
  between $A$ and $B$?
\end{question}
\begin{proof}
  Consider local and global degrees?
\end{proof}

\begin{question}
  Given any non-recursive non-minimal degree, are there always infinite
  pairwise incomparable degrees beneath it?
\end{question}
\begin{proof}
\end{proof}

\begin{question}
  Is $\mathcal{D}$ a lattice? In particular, do any two Turing degrees
  have a greatest lower bound?
\end{question}
\begin{proof}
  No. Clearly the join of two given degrees give their lub. But from
  Klenne-Post-Spector (old Soare VI.C4.4), there are degrees with no
  greatest lower bound.
\end{proof}

\begin{question}
  What do we know about the minimal degree below $\emptyset'$
  constructed in Soare 13.1.1? In particular, is it low? Does it have a
  c.e. degree?
\end{question}
\begin{proof}
  From Soare 13.7.1 every non-computable c.e. degree computes a minimal
  degree. In particular, minimal degrees can be low.
\end{proof}

\begin{question}
  Does $\mathcal{D}_{\leq\emptyset'}$ have a maximal degree?
\end{question}
\begin{proof}
  \OPEN{This is asking if $\emptyset'$ is a strong minimal cover of any
  degree below it. We know there are degrees with strong minimal
  covers, but we do not know if $\emptyset'$ is such a cover. Look up
  Andrew Lewis \url{http://www.aemlewis.co.uk/pdf/SMCtssf.pdf} or
  \url{https://projecteuclid.org/euclid.jsl/1183746187}.}
\end{proof}

%  \subsection{Randoms}
%    \begin{question}
  Is there a 1-random in every non-recursive degree? Is there a result
  similar to one for hyperimmune degrees, which says that the
  hyperimmune degrees are exactly those that are comparable with
  $\emptyset'$?
\end{question}
\begin{proof}
\end{proof}

\begin{question}
  Is randomness independent from ZFC?
\end{question}
\begin{proof}
  Perhaps this depends on our definition of randomness. Kolmogorov defined
  randomness as a series that cannot be generated by any Turing machine
  shorter than the series (Wiki Randomness, mathematics). By this
  definition, it appears randomness exists if certain reals exist, which
  may be relevant to the continuum hypothesis.
\end{proof}

\begin{question}
  Can Turing machines prove their lack of randomness?
\end{question}
\begin{proof}
  In the same vein, if the universe were deterministic, would we ever know
  it? Similarly, if the universe were random, would we eventually find out?
\end{proof}

%  \subsection{Generics}
%    \begin{question}
  Is there any relation between 1-random sets and 1-generic sets? I
  suspect all 1-generics are 1-random, but the converse isn't true? Is
  there a result similar to one for hyperimmune degrees, which says
  that the hyperimmune degrees are exactly those that are comparable
  with $\emptyset'$?
\end{question}
\begin{proof}
\end{proof}


%\section{Reverse Math}
%  \subsection{Rich Sets}
%    \begin{question}
  By Simpson all hyperarithmetical sets are rich, and by Soare a
  poor set exists. Are there non-hyperarithmetical sets that are also
  rich?
\end{question}
\begin{proof}
  Yes. Consider the join $H\oplus P$ of a hyperarithmetical set $H$
  with a poor set $P$. Given arbitrary $X$ that computes $H\oplus P$,
  since $H$ is rich, it contains some subset $H_0\subset H$ that is
  Turing equivalent to $X$. Then $H_0\oplus\emptyset$ witnesses the
  richness of $H\oplus P$.
\end{proof}

%  \subsection{Galvin-Prikry}
%    \begin{question}
  Is Borel necessary for Galvin-Prikry to hold? In other words, is
  there a weaker condition that will ensure we observe the structure of
  Galvin-Prikry?
\end{question}
\begin{proof}
  Probably not. Silver's result improved Galvin-Prikry but only in a
  special case. There is a book with these results.
\end{proof}

%
%\section{Set Theory}
%  \subsection{Godel's Second Incompleteness}
%    Godel's second incompleteness theorem appears to be a by-product of
formalizing derivability. Given sentence $\sigma$, let Dr($\sigma$)
abbreviate the sentence ``$\sigma$ is derivable''.  To formalize
derivability, some natural questions to ask are:
\begin{itemize}
  \item If PA derives $\sigma$, does it derive Dr($\sigma$)?
  \item If PA derives Dr($\sigma$), does it derive Dr(Dr($\sigma$))?
  \item If PA derives Dr($\sigma_0\rightarrow\sigma_1$) and
    Dr($\sigma_0$), does it derive Dr($\sigma_1$)?
\end{itemize}

The answer to these three questions turn out to be yes. The three rules
are called derivability conditions, and are used to formalize
derivability.\\

Godel's second incompleteness theorem is proved by applying the
derivability conditions on the sentence $\theta$ which says ``I am not
derivable''. In technical terms, PA derives the sentence
``$\theta\leftrightarrow\neg\text{Dr}(\theta)$''.\\

Let Con(PA) abbreviate the sentence that says PA is consistent. More
precisely, Con(PA) is defined as $\neg$Dr(0=1).\\

Our final goal is to show that PA cannot derive Con(PA). We prove this
by applying the derivability conditions on $\theta$.\\

First, by applying the first derivability condition, we get $\theta$ is
inconsistent: $\theta$ implies $\neg\text{Dr}(\theta)$ from choice of
$\theta$, yet also implies $\text{Dr}(\theta)$ from the first
condition.\\

Hence PA derives ``$\theta\rightarrow0=1$'', and by applying the third
derivability condition on this sentence, we get PA derives
``$\text{Dr}(\theta)\rightarrow\text{Dr}(0=1)$'',
which is equivalent to ``Con(PA)$\rightarrow\neg\text{Dr}(\theta)$''.
Since $\neg\text{Dr}(\theta)$ is the same as $\theta$ from our choice
of $\theta$, and $\theta$ is inconsistent from earlier argument, we
finally get ``Con(PA)$\rightarrow0=1$'', as desired.\\

We can interpret the above proof of the second incompleteness theorem as a
formalization of the following argument: Recall that the first
incompleteness theorem says that if PA is consistent then it cannot prove
$\theta$. But not proving $\theta$ is the same as proving $\theta$, from
the definition of $\theta$. Hence if we show that PA is consistent then we
get a contradiction. 

%  \subsection{Ordinals}
%    \begin{question}
  Are inaccessible ordinals analogous to non-standard numbers?
\end{question}
\begin{proof}
  Inaccessible ordinals seem to appear out of the blue, breaking the
  intuitive way of visualizing ordinals as a single linear chain.
  Perhaps we can think of them the same way we think of non-standard
  numbers; the inaccessible ordinals (non-standard numbers) form linear
  chain of ordinals (numbers) parallel to the standard line of ordinals
  (numbers).
\end{proof}

%  \subsection{ZFC}
%    \begin{question}
  Why can't we extend a model of ZFC easily?
\end{question}
\begin{proof}
  To extend a model of fields, we can simply append an element and all other
  constructible elements to the original model and still get a model of
  Fields. For example, we can extend to $R$ an object $x$ and other
  elements such as $x+1$, $(x+6)/x^4$, etc to get a larger field $R(x)$.\\

  Can we do the same for ZFC? Intuitively, if we started out with a countable
  model, it seems like we can. We can enumerate all the axioms of ZFC and
  elements in the model and add elements for as long as they are missing,
  repeating the entire process omega number of times. Since the axioms of ZFC
  are always asking for the existence of sets instead of the non-existence of
  them, at first glance we will not be negating any previously satisfied
  axioms on a given set. For example, ZFC asks for the existence of pairings,
  unions, powersets, comprehension, and replacement sets, and never speaks of
  the non-existence of sets. Hence, by repeatedly adding sets, it seems we
  should be getting closer and closer to satisfying the whole of ZFC. Upon
  closer inspection however, by adding new sets, we might unintentionally
  negate previously satisfied axioms. For example, after ensuring a set has a
  powerset, we might add new subset of the set from comprehension, which
  would negate the previously satisfied powerset.\\

  Hence, extending a model of ZFC is not as straightforward as it seems. In
  fact, Cohen confirms that appending any set and all sets constructible from
  it does not necessarily give a model of ZFC. He explains that we could
  intentionally choose a set that contains information on the size of the
  original model, such that by appending this set, we get a self-referential
  paradox (Set Theory and the Continuum Hypothesis, Paul J. Cohen, Page 111,
  Addison-Wesley, 1966).
\end{proof}

%  \subsection{Forcing}
%    \begin{question}
  What exactly is forcing?
\end{question}
\begin{proof}
  Forcing is a technique of creating ZFC models from an existing countable
  transitive model by appending to the model a new element and other elements
  that are constructible from the new element. Forcing says that as long as
  the new element satisfies certain loose conditions, the extended model will
  satisfy ZFC.\\

  Model extension is analogous to extending an existing field $R$ with an
  object $x$ to get an extended field $R(x)$. To ensure $R(x)$ is still a
  field, $R(x)$ must also include constructible elements such $x+1$,
  $(x+1)/x^3$, and so on. Extending a model of ZFC however is not as easy
  as extending a model of fields. The difficulty is due to the infinite
  number of axioms to satisfy - we may un-satisfy a previously satisfied
  axiom after adding a new element. Hence, forcing was developed to ensure
  that the extended model preserves ZFC.\\

  There are two ways to understand forcing. The first way is by following
  Cohen's thoughts when he developed forcing: To create an extension model
  from an existing countable transitive model $M$, Cohen adds to $M$ a
  subset $G$ of a set $P\in M$ to get another countable transitive model
  $M[G]$. Forcing says that $P$ and $G$ only need satisfy very loose
  conditions to ensure that $M[G]$ is also a model of ZFC. \\

  So how do we define $M[G]$ given $M$ and $G$?
  $M[G]$ should extend $M$, include $G$, and also include all sets that are
  constructible from $G$. Is there a natural definition of $M[G]$ that
  would satisfy all these conditions? From Bell's argument mentioned
  earlier, we cannot define $M[G]$ as the class of elements constructible
  from $M$ and $G$, because this class may not satisfy ZFC. Hence Cohen
  explored a new way of defining $M[G]$, through exploiting $M$'s existing
  ZFC ``structure'', where any set $A\in M$ is associated with a class of
  ZFC extensions of $A$ which are $A$'s union sets, pairing sets,
  comprehension sets, replacement sets, and powerset. Similarly, $M[G]$
  should include $G$ and all its ZFC extensions. Hence, to define $M[G]$,
  Cohen first designed a function from $M$ to $M[G]$ that preserves the ZFC
  functions of union, pairing, comprehension, and replacement. More
  precisely, this function $f$ maps the union of $A$ to the union of
  $f(A)$, the pairing of $A$ and $B$ to the pairing of $f(A)$ and $f(B)$,
  and so on. Since $f$ preserves ZFC functions, if it also includes $G$ and
  all elements of $M$ in its range, then its range is likely to be a model
  of ZFC.  Hence, Cohen set $M[G]$ to be the range of $f$, and proved that
  $M[G]$ would be a ZFC extension of $M$.\\

  To preserve ZFC functions, it is natural to design the function $f$
  recursively, as follows:
  \begin{itemize}
    \item Map $\emptyset$ to $\emptyset$
    \item Map $\{s_0,s_1,\ldots\}$ to $\{f(g(s_0)),f(g(s_1)),\ldots)\}$
  \end{itemize}

  The question left unanswered is how should we define function $g$ such
  that $f$ contains $G$ and all of $M$ in its range and preserves all ZFC
  functions. To satisfy these two conditions, Cohen set $g$ to be defined
  only on ordered pairs $\langle a,b\rangle$ where $b\in G$, and mapped
  $\langle a,b\rangle$ to $a$. Cohen easily verified that this definition
  satisfies the first condition but only partially satisfies the second.
  Specifically, $f$ may not preserve more complex ZFC functions such as the
  power function. Hence, Cohen searched for additional constraints on $G$
  that will resolve this issue. The constraints turned out to very loose,
  yet tedious to verify how they helped $f$ preserve the power function.
  Very roughly speaking, in order for $f$ to preserve the power function,
  we need to find for any set $s\in M$
\end{proof}

%
%\section{Complexity}
%  \subsection{P, NP}
%    \begin{question}
  Does pseudorandomness affect if P = NP?
\end{question}

\begin{proof}
  Before we can answer this question, we need to define what we mean by
  randomness in this context. By randomness, I do not mean true
  randomness, but rather a pseudorandom generator, where a deterministic
  program generates a series that is sufficiently random in the sense
  that no efficient computation can distinguish the series from a truly
  random one by a non-negligible advantage. Note that deterministic
  machines can never be truly random by Kolmogorov's definition of
  randomness, which says that a sequence is random if it cannot be
  generated by a Turing machine shorter than the series (wiki
  randomness, mathematics).\\

  The pseudorandom generator theorem says randomness implies P != NP. Yet,
  at first glance, randomness also seems to imply P = NP, because we may
  be able to use randomness to solve exponential problems in polynomial
  time. So which is it? The short answer is, the first implication is
  correct.\\

  More precisely, the existence of pseudorandom functions implies that
  one-way functions exists, which further implies that P != NP.\\

  To show the first implication, we construct a one-way function from a
  pseudorandom generator by simply taking the first half of the
  generator's output. Formally, let $f:\{0,1\}^n \rightarrow
  \{0,1\}^{2n}$ be the pseudorandom generator. Construct the one-way
  function $F:\{0,1\}^n\rightarrow\{0,1\}^n$ as $F(x)=$ first half of
  $f(x)$. Intuitively, we can see that $F$ will be easy to compute but
  difficult to invert.\\

  The existence of one-way functions almost directly implies P != NP by
  definition: the inverse of one-way functions are easy to verify but
  difficult to compute, which is the defining characteristic of
  NP-complete functions.\\

  For more information on the pseudorandom generator theorem, wiki one-way
  functions, pseudorandom generator theorem, PRG$\rightarrow$OWF.\\

  So why doesn't randomness imply P = NP? Most computer scientists believe
  randomness does not offer enough to solve exponential problems
  efficiently. For this statement, refer to the second last paragraph of
  section 7.2 of the paper titled ``The status of the P versus NP
  problem''.\\

  The paper also offers a tidbit of information that is somewhat relevant
  to our discussion: the factorization problem is likely to be easier than
  NP-complete.
\end{proof}

%
%\section{Philosophy}
%  \subsection{Computability of Human Mind}
%    \begin{question}
  What kind of ``computers'' are human brains? Can we ever figure out
  the axioms of the universe? Or can we prove that we can never figure
  out the axioms? Can we formulate these questions rigorously?
\end{question}
\begin{proof}
\end{proof}

\end{document}
