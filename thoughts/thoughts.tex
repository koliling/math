\documentclass{article}
\usepackage[left=3cm,right=3cm,top=3cm,bottom=3cm]{geometry}
\usepackage{amsmath,amssymb,amsthm,tikz,mathtools}
\usepackage{stmaryrd} % For double square bracket [[]]
\usepackage{bm} % For bold vectors
\usepackage{color}
\usepackage{cancel} % To use the cancel function
\usepackage[inline]{enumitem}
\usepackage{hyperref} % To use hyperlinks in URLs
\usetikzlibrary{patterns}
\setlength{\parindent}{0mm}
\newcommand{\OPEN}[1]{\textcolor{red}{OPEN: #1}}

\newtheorem{theorem}{Theorem}[subsection]
\newtheorem{question}{Question}[subsection]

\begin{document}
\title{Logic Thoughts}
\author{Li Ling Ko\\ lko@nd.edu}
\date{\today}
\maketitle

\section{Arithmetical Hierarchy}
  \subsection{Computably Enumerable Degrees}
    \begin{theorem}
      There are non c.e. degrees below $\emptyset'$.
    \end{theorem}
    \begin{proof}
      Cooper Epstein-Lachlan theorem.
    \end{proof}

    \begin{theorem}
      There are no minimal c.e. degrees.
    \end{theorem}
    \begin{proof}
      Sack's Density theorem.
    \end{proof}

    \begin{question}
      Do all non-minimal degrees have infinite incomparable degrees below
      them? How about just finite?
    \end{question}
    \begin{proof}
    \end{proof}

  \subsection{Lattice $\mathcal{D}$}
    \begin{question}
      Given $A\leq_T B$, must there be a degree between them? I.e. Does
      Sack's density theorem be generalized to degrees that are not
      necessarily c.e.? How about infinite pairwise incomparable degrees
      between $A$ and $B$?
    \end{question}
    \begin{proof}
      Consider local and global degrees?
    \end{proof}

    \begin{question}
      Given any non-recursive non-minimal degree, are there always infinite
      pairwise incomparable degrees beneath it?
    \end{question}
    \begin{proof}
    \end{proof}

    \begin{question}
      Is $\mathcal{D}$ a lattice? In particular, do any two Turing degrees
      have a greatest lower bound?
    \end{question}
    \begin{proof}
      No. Clearly the join of two given degrees give their lub. But from
      Klenne-Post-Spector (old Soare VI.C4.4), there are degrees with no
      greatest lower bound.
    \end{proof}

    \begin{question}
      What do we know about the minimal degree below $\emptyset'$
      constructed in Soare 13.1.1? In particular, is it low? Does it have a
      c.e. degree?
    \end{question}
    \begin{proof}
      From Soare 13.7.1 every non-computable c.e. degree computes a minimal
      degree. In particular, minimal degrees can be low.
    \end{proof}

    \begin{question}
      Does $\mathcal{D}_{\leq\emptyset'}$ have a maximal degree?
    \end{question}
    \begin{proof}
      This is asking if $\emptyset'$ is a strong minimal cover of any
      degree below it. We know there are degrees with strong minimal
      covers, but we do not know if $\emptyset'$ is such a cover. Look up
      Andrew Lewis \url{http://www.aemlewis.co.uk/pdf/SMCtssf.pdf} or
      \url{https://projecteuclid.org/euclid.jsl/1183746187}.
    \end{proof}

  \subsection{Randoms and Generics}
    \begin{question}
      Is there any relation between 1-random sets and 1-generic sets? I
      suspect all 1-generics are 1-random, but the converse isn't true? Is
      there a result similar to one for hyperimmune degrees, which says
      that the hyperimmune degrees are exactly those that are comparable
      with $\emptyset'$?
    \end{question}
    \begin{proof}
    \end{proof}

    \begin{question}
      Is there a 1-random in every non-recursive degree? Is there a result
      similar to one for hyperimmune degrees, which says that the
      hyperimmune degrees are exactly those that are comparable with
      $\emptyset'$?
    \end{question}
    \begin{proof}
    \end{proof}

\section{Philosophy}
  \subsection{Computability of Human Mind}
    \begin{question}
      What kind of ``computers'' are human brains? Can we ever figure out
      the axioms of the universe? Or can we prove that we can never figure
      out the axioms? Can we formulate these questions rigorously?
    \end{question}
    \begin{proof}
    \end{proof}
\end{document}
