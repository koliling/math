\documentclass{article}
\usepackage[left=3cm,right=3cm,top=3cm,bottom=3cm]{geometry}
\usepackage{amsmath,amssymb,amsthm,tikz,mathtools}
\usepackage{stmaryrd} % For double square bracket [[]]
\usepackage{bm} % For bold vectors
\usepackage{color}
\usepackage{cancel} % To use the cancel function
\usepackage[inline]{enumitem}
\usetikzlibrary{patterns}
\setlength{\parindent}{0mm}
\newcommand{\TODO}[1]{\textcolor{red}{TODO: #1}}

\begin{document}
\title{Thoughts on Logic}
\author{Li Ling Ko\\ lko@nd.edu}
\date{\today}
\maketitle

\section{Arithmetical Hierarchy}
  \begin{enumerate}
    \item Are there non c.e. degrees below $\emptyset'$?

    \item Is there a minimal c.e. degree?

    \item Do all non-minimal degrees have infinite incomparable degrees
      below them? How about just finite?

    \item Given $A\leq_T B$, must there be a degree between them? I.e. Does
      Sack's density theorem be generalized to degrees that are not
      necessarily c.e.? How about infinite pairwise incomparable degrees
      between $A$ and $B$?

      \begin{enumerate}
        \item Consider local and global degrees?
      \end{enumerate}

    \item Given any non-recursive non-minimal degree, are there always
      infinite pairwise incomparable degrees beneath it?

    \item Is there any relation between 1-random sets and 1-generic sets? I
      suspect all 1-generics are 1-random, but the converse isn't true? Is
      there a result similar to one for hyperimmune degrees, which says
      that the hyperimmune degrees are exactly those that are comparable
      with $\emptyset'$?

    \item Is there a 1-random in every non-recursive degree? Is there a
      result similar to one for hyperimmune degrees, which says that the
      hyperimmune degrees are exactly those that are comparable with
      $\emptyset'$?
  \end{enumerate}

\section{Philosophy}
  \subsection{Computability of Human Mind}
    What kind of ``computers'' are human brains? Can we ever figure out the
    axioms of the universe? Or can we prove that we can never figure out
    the axioms? Can we formulate these questions rigorously?

\end{document}
