\documentclass{article}
\usepackage[left=3cm,right=3cm,top=3cm,bottom=3cm]{geometry}
\usepackage{amsmath,amssymb,amsthm,tikz,mathtools}
\usepackage{stmaryrd} % For double square bracket [[]]
\usepackage{bm} % For bold vectors
\usepackage{color}
\usepackage{cancel} % To use the cancel function
\usepackage[inline]{enumitem}
\usepackage{hyperref} % To use hyperlinks in URLs
\usetikzlibrary{patterns}
\setlength{\parindent}{0mm}
\newcommand{\OPEN}[1]{\textcolor{red}{OPEN: #1}}

\newtheorem{theorem}{Theorem}[subsection]
\newtheorem{question}{Question}[subsection]

\begin{document}
\title{Logic Thoughts}
\author{Li Ling Ko\\ lko@nd.edu}
\date{\today}
\maketitle

\section{Arithmetical Hierarchy}
  \subsection{Computably Enumerable Degrees}
    \begin{theorem}
      What is the history behind simple sets?
    \end{theorem}
    \begin{proof}
      Perhaps simple sets came about when people tried to construct
      r.e. sets that are non-computable. 
    \end{proof}

    \begin{question}
      Must the cap of every pair of r.e. degrees, if it exist, also be an
      r.e. degree?
    \end{question}
    \begin{proof}
    \end{proof}

    \begin{theorem}
      There are non c.e. degrees below $\emptyset'$.
    \end{theorem}
    \begin{proof}
      Cooper Epstein-Lachlan theorem. Also, the 1-generics, from new Soare
      E6.3.5iii, and the fact that there exists 1-generics below
      $\emptyset'$. Finally, the 1-randoms too. By the Low Basis theorem,
      applied to the universal Martin-Lof tree, there must be a 1-random
      that is low. And 1-randoms cannot be c.e.
    \end{proof}

    \begin{theorem}
      There are no minimal c.e. degrees.
    \end{theorem}
    \begin{proof}
      Sack's Density theorem.
    \end{proof}

    \begin{question}
      Do all non-minimal degrees have infinite incomparable degrees below
      them? How about just finite?
    \end{question}
    \begin{proof}
    \end{proof}

  \subsection{Lattice $\mathcal{D}$}
    \begin{question}
      What is the motivation for defining simple sets? How about
      1-generics, or maximal sets?
    \end{question}
    \begin{proof}
      Simple sets were defined to answer Post's question, which asks if
      there exists a non-computable set below the jump. Perhaps 1-generics
      were defined to construct sets through forcing the jump.
    \end{proof}

    \begin{theorem}
      Make sense of fixed-point, and Arslanov's completeness criteria.
    \end{theorem}
    \begin{proof}
      How do we interpret the proof of the fixed-point (recursion) theorem?
      The theorem says given a recursive $f(e)$, there exists a function
      indexed $n\in\omega$ such that $\varphi_n=\varphi_{f(n)}$. We interpret
      the proof of of this theorem follows. The $n$ that we find shall say
      \begin{center}
        \textit{``I am the $f$ of myself.''}
      \end{center}

      This makes sense since we want $\varphi_n$ to equal in some sense to
      ``the $f$ of itself'' $\varphi_{f(n)}$. A function indexed by $e$ may
      refer to itself using the function indexed $\varphi_e(e)$. So we are
      looking for a function indexed $n=\varphi_v(v)$ for some $v$ such
      that
      $\varphi_v(v)=f(\varphi_v(v))$, i.e.
      \[\varphi_n :=\varphi_{\varphi_v(v)} =\varphi_{f(\varphi_v(v))}.\]

      So we consider the function indexed $v$ given by
      \[\varphi_v(e) :=f(\varphi_e(e)),\]
      and let $n=\varphi_v(v)$. Then we have
      \[\varphi_{f(n)} =\varphi_{f(\varphi_v(v))} =\varphi_{\varphi_v(v)}
      =\varphi_n\]
      as desired. Thus completes the general idea of the proof of the
      recursion theorem. \\

      Using the recursion theorem, we can define functions `recursively' in
      the sense that we can define a desired $g(x)$ in terms of itself. Say
      we want $g(x)$ defined recursively as
      \[g(x)=h(g(x),x)\]
      for some recursive $h(x,y)$. To prove that such $g(x)$ exists,
      define recursive $g'(e)$ such that
      \[\varphi_{g'(e)}(x) :=h(\varphi_e(x),x).\]

      Apply recursion theorem to get $n\in\omega$ such that
      \[\varphi_n(x) =\varphi_{g'(n)}(x) =h(\varphi_n(x),x).\]
      Such $\varphi_n(x)$ is the $g(x)$ we are looking for.
    \end{proof}

    \begin{theorem}
      There exists a degree incomparable with $\emptyset'$.
    \end{theorem}
    \begin{proof}
      From the paragraph after Soare's D5.6.1, the hyperimmune-free degrees
      are exactly those that are incomparable with $\emptyset'$. Also
      recall that a set is hyperimmune if and only if it is not computably
      dominated. The construction of Soare T9.5.1 gives a noncomputable set
      with a hyperimmune-free degree. \\

      Alternatively, from new Soare E6.3.5iii, 1-generics cannot compute
      $\emptyset'$. Then from E6.3.4, a nonlow 1-generic cannot be computed
      by $\emptyset'$. Therefore nonlow 1-generics are imcomparable with
      $\emptyset'$.
    \end{proof}

    \begin{question}
      Given $A\leq_T B$, must there be a degree between them? I.e. Does
      Sack's density theorem be generalized to degrees that are not
      necessarily c.e.? How about infinite pairwise incomparable degrees
      between $A$ and $B$?
    \end{question}
    \begin{proof}
      Consider local and global degrees?
    \end{proof}

    \begin{question}
      Given any non-recursive non-minimal degree, are there always infinite
      pairwise incomparable degrees beneath it?
    \end{question}
    \begin{proof}
    \end{proof}

    \begin{question}
      Is $\mathcal{D}$ a lattice? In particular, do any two Turing degrees
      have a greatest lower bound?
    \end{question}
    \begin{proof}
      No. Clearly the join of two given degrees give their lub. But from
      Klenne-Post-Spector (old Soare VI.C4.4), there are degrees with no
      greatest lower bound.
    \end{proof}

    \begin{question}
      What do we know about the minimal degree below $\emptyset'$
      constructed in Soare 13.1.1? In particular, is it low? Does it have a
      c.e. degree?
    \end{question}
    \begin{proof}
      From Soare 13.7.1 every non-computable c.e. degree computes a minimal
      degree. In particular, minimal degrees can be low.
    \end{proof}

    \begin{question}
      Does $\mathcal{D}_{\leq\emptyset'}$ have a maximal degree?
    \end{question}
    \begin{proof}
      \OPEN{This is asking if $\emptyset'$ is a strong minimal cover of any
      degree below it. We know there are degrees with strong minimal
      covers, but we do not know if $\emptyset'$ is such a cover. Look up
      Andrew Lewis \url{http://www.aemlewis.co.uk/pdf/SMCtssf.pdf} or
      \url{https://projecteuclid.org/euclid.jsl/1183746187}.}
    \end{proof}

  \subsection{Randoms and Generics}
    \begin{question}
      Is there any relation between 1-random sets and 1-generic sets? I
      suspect all 1-generics are 1-random, but the converse isn't true? Is
      there a result similar to one for hyperimmune degrees, which says
      that the hyperimmune degrees are exactly those that are comparable
      with $\emptyset'$?
    \end{question}
    \begin{proof}
    \end{proof}

    \begin{question}
      Is there a 1-random in every non-recursive degree? Is there a result
      similar to one for hyperimmune degrees, which says that the
      hyperimmune degrees are exactly those that are comparable with
      $\emptyset'$?
    \end{question}
    \begin{proof}
    \end{proof}

\section{Reverse Math}
  \subsection{Rich Sets}
    \begin{question}
      By Simpson all hyperarithmetical sets are rich, and by Soare a
      poor set exists. Are there non-hyperarithmetical sets that are also
      rich?
    \end{question}
    \begin{proof}
      Yes. Consider the join $H\oplus P$ of a hyperarithmetical set $H$
      with a poor set $P$. Given arbitrary $X$ that computes $H\oplus P$,
      since $H$ is rich, it contains some subset $H_0\subset H$ that is
      Turing equivalent to $X$. Then $H_0\oplus\emptyset$ witnesses the
      richness of $H\oplus P$.
    \end{proof}

  \subsection{Galvin-Prikry}
    \begin{question}
      Is Borel necessary for Galvin-Prikry to hold? In other words, is
      there a weaker condition that will ensure we observe the structure of
      Galvin-Prikry?
    \end{question}
    \begin{proof}
      Probably not. Silver's result improved Galvin-Prikry but only in a
      special case. There is a book with these results.
    \end{proof}

\section{Philosophy}
  \subsection{Computability of Human Mind}
    \begin{question}
      What kind of ``computers'' are human brains? Can we ever figure out
      the axioms of the universe? Or can we prove that we can never figure
      out the axioms? Can we formulate these questions rigorously?
    \end{question}
    \begin{proof}
    \end{proof}
\end{document}
