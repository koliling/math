\begin{question}
  Why can't we extend a model of ZFC easily?
\end{question}
\begin{proof}
  To extend a model of fields, we can simply append an element and all other
  constructible elements to the original model and still get a model of
  Fields. For example, we can extend to $R$ an object $x$ and other
  elements such as $x+1$, $(x+6)/x^4$, etc to get a larger field $R(x)$.\\

  Can we do the same for ZFC? Intuitively, if we started out with a countable
  model, it seems like we can. We can enumerate all the axioms of ZFC and
  elements in the model and add elements for as long as they are missing,
  repeating the entire process omega number of times. Since the axioms of ZFC
  are always asking for the existence of sets instead of the non-existence of
  them, at first glance we will not be negating any previously satisfied
  axioms on a given set. For example, ZFC asks for the existence of pairings,
  unions, powersets, comprehension, and replacement sets, and never speaks of
  the non-existence of sets. Hence, by repeatedly adding sets, it seems we
  should be getting closer and closer to satisfying the whole of ZFC. Upon
  closer inspection however, by adding new sets, we might unintentionally
  negate previously satisfied axioms. For example, after ensuring a set has a
  powerset, we might add new subset of the set from comprehension, which
  would negate the previously satisfied powerset.\\

  Hence, extending a model of ZFC is not as straightforward as it seems. In
  fact, Cohen confirms that appending any set and all sets constructible from
  it does not necessarily give a model of ZFC. He explains that we could
  intentionally choose a set that contains information on the size of the
  original model, such that by appending this set, we get a self-referential
  paradox (Set Theory and the Continuum Hypothesis, Paul J. Cohen, Page 111,
  Addison-Wesley, 1966).
\end{proof}
